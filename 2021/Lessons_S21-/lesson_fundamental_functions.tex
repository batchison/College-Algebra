\documentclass[12pt]{article}
\usepackage[top=1in,left=1in,bottom=1in,right=1in,headsep=2pt]{geometry}	
\usepackage{amssymb,amsmath,amsthm,amsfonts}
\usepackage{chapterfolder,docmute,setspace}
\usepackage{cancel,multicol,tikz,verbatim,framed,polynom,enumitem}
\usepackage[colorlinks, hyperindex, plainpages=false, linkcolor=blue, urlcolor=blue, pdfpagelabels]{hyperref}
% Use the cc-by-nc-sa license for any content linked with Stitz and Zeager's text.  Otherwise, use the cc-by-sa license.
\usepackage[type={CC},modifier={by-sa},version={4.0},]{doclicense}
%\usepackage[type={CC},modifier={by-nc-sa},version={4.0},]{doclicense}

\theoremstyle{definition}
\newtheorem{example}{Example}
\newcommand{\Desmos}{\href{https://www.desmos.com/}{Desmos}}
\setlength{\parindent}{0em}
\setlist{itemsep=0em}
\setlength{\parskip}{0.1em}
% This document is used for ordering of lessons.  If an instructor wishes to change the ordering of assessments, the following steps must be taken:

% 1) Reassign the appropriate numbers for each lesson in the \setcounter commands included in this file.
% 2) Rearrange the \include commands in the master file (the file with 'Course Pack' in the name) to accurately reflect the changes.  
% 3) Rearrange the \items in the measureable_outcomes file to accurately reflect the changes.  Be mindful of page breaks when moving items.
% 4) Re-build all affected files (master file, measureable_outcomes file, and any lesson whose numbering has changed).

%Note: The placement of each \newcounter and \setcounter command reflects the original/default ordering of topics (linears, systems, quadratics, functions, polynomials, rationals).

\newcounter{lesson_solving_linear_equations}
\newcounter{lesson_equations_containing_absolute_values}
\newcounter{lesson_graphing_lines}
\newcounter{lesson_two_forms_of_a_linear_equation}
\newcounter{lesson_parallel_and_perpendicular_lines}
\newcounter{lesson_linear_inequalities}
\newcounter{lesson_compound_inequalities}
\newcounter{lesson_inequalities_containing_absolute_values}
\newcounter{lesson_graphing_systems}
\newcounter{lesson_substitution}
\newcounter{lesson_elimination}
\newcounter{lesson_quadratics_introduction}
\newcounter{lesson_factoring_GCF}
\newcounter{lesson_factoring_grouping}
\newcounter{lesson_factoring_trinomials_a_is_1}
\newcounter{lesson_factoring_trinomials_a_neq_1}
\newcounter{lesson_solving_by_factoring}
\newcounter{lesson_square_roots}
\newcounter{lesson_i_and_complex_numbers}
\newcounter{lesson_vertex_form_and_graphing}
\newcounter{lesson_solve_by_square_roots}
\newcounter{lesson_extracting_square_roots}
\newcounter{lesson_the_discriminant}
\newcounter{lesson_the_quadratic_formula}
\newcounter{lesson_quadratic_inequalities}
\newcounter{lesson_functions_and_relations}
\newcounter{lesson_evaluating_functions}
\newcounter{lesson_finding_domain_and_range_graphically}
\newcounter{lesson_fundamental_functions}
\newcounter{lesson_finding_domain_algebraically}
\newcounter{lesson_solving_functions}
\newcounter{lesson_function_arithmetic}
\newcounter{lesson_composite_functions}
\newcounter{lesson_inverse_functions_definition_and_HLT}
\newcounter{lesson_finding_an_inverse_function}
\newcounter{lesson_transformations_translations}
\newcounter{lesson_transformations_reflections}
\newcounter{lesson_transformations_scalings}
\newcounter{lesson_transformations_summary}
\newcounter{lesson_piecewise_functions}
\newcounter{lesson_functions_containing_absolute_values}
\newcounter{lesson_absolute_as_piecewise}
\newcounter{lesson_polynomials_introduction}
\newcounter{lesson_sign_diagrams_polynomials}
\newcounter{lesson_factoring_quadratic_type}
\newcounter{lesson_factoring_summary}
\newcounter{lesson_polynomial_division}
\newcounter{lesson_synthetic_division}
\newcounter{lesson_end_behavior_polynomials}
\newcounter{lesson_local_behavior_polynomials}
\newcounter{lesson_rational_root_theorem}
\newcounter{lesson_polynomials_graphing_summary}
\newcounter{lesson_polynomial_inequalities}
\newcounter{lesson_rationals_introduction_and_terminology}
\newcounter{lesson_sign_diagrams_rationals}
\newcounter{lesson_horizontal_asymptotes}
\newcounter{lesson_slant_and_curvilinear_asymptotes}
\newcounter{lesson_vertical_asymptotes}
\newcounter{lesson_holes}
\newcounter{lesson_rationals_graphing_summary}

\setcounter{lesson_solving_linear_equations}{1}
\setcounter{lesson_equations_containing_absolute_values}{2}
\setcounter{lesson_graphing_lines}{3}
\setcounter{lesson_two_forms_of_a_linear_equation}{4}
\setcounter{lesson_parallel_and_perpendicular_lines}{5}
\setcounter{lesson_linear_inequalities}{6}
\setcounter{lesson_compound_inequalities}{7}
\setcounter{lesson_inequalities_containing_absolute_values}{8}
\setcounter{lesson_graphing_systems}{9}
\setcounter{lesson_substitution}{10}
\setcounter{lesson_elimination}{11}
\setcounter{lesson_quadratics_introduction}{16}
\setcounter{lesson_factoring_GCF}{17}
\setcounter{lesson_factoring_grouping}{18}
\setcounter{lesson_factoring_trinomials_a_is_1}{19}
\setcounter{lesson_factoring_trinomials_a_neq_1}{20}
\setcounter{lesson_solving_by_factoring}{21}
\setcounter{lesson_square_roots}{22}
\setcounter{lesson_i_and_complex_numbers}{23}
\setcounter{lesson_vertex_form_and_graphing}{24}
\setcounter{lesson_solve_by_square_roots}{25}
\setcounter{lesson_extracting_square_roots}{26}
\setcounter{lesson_the_discriminant}{27}
\setcounter{lesson_the_quadratic_formula}{28}
\setcounter{lesson_quadratic_inequalities}{29}
\setcounter{lesson_functions_and_relations}{12}
\setcounter{lesson_evaluating_functions}{13}
\setcounter{lesson_finding_domain_and_range_graphically}{14}
\setcounter{lesson_fundamental_functions}{15}
\setcounter{lesson_finding_domain_algebraically}{30}
\setcounter{lesson_solving_functions}{31}
\setcounter{lesson_function_arithmetic}{32}
\setcounter{lesson_composite_functions}{33}
\setcounter{lesson_inverse_functions_definition_and_HLT}{34}
\setcounter{lesson_finding_an_inverse_function}{35}
\setcounter{lesson_transformations_translations}{36}
\setcounter{lesson_transformations_reflections}{37}
\setcounter{lesson_transformations_scalings}{38}
\setcounter{lesson_transformations_summary}{39}
\setcounter{lesson_piecewise_functions}{40}
\setcounter{lesson_functions_containing_absolute_values}{41}
\setcounter{lesson_absolute_as_piecewise}{42}
\setcounter{lesson_polynomials_introduction}{43}
\setcounter{lesson_sign_diagrams_polynomials}{44}
\setcounter{lesson_factoring_quadratic_type}{46}
\setcounter{lesson_factoring_summary}{45}
\setcounter{lesson_polynomial_division}{47}
\setcounter{lesson_synthetic_division}{48}
\setcounter{lesson_end_behavior_polynomials}{49}
\setcounter{lesson_local_behavior_polynomials}{50}
\setcounter{lesson_rational_root_theorem}{51}
\setcounter{lesson_polynomials_graphing_summary}{52}
\setcounter{lesson_polynomial_inequalities}{53}
\setcounter{lesson_rationals_introduction_and_terminology}{54}
\setcounter{lesson_sign_diagrams_rationals}{55}
\setcounter{lesson_horizontal_asymptotes}{56}
\setcounter{lesson_slant_and_curvilinear_asymptotes}{57}
\setcounter{lesson_vertical_asymptotes}{58}
\setcounter{lesson_holes}{59}
\setcounter{lesson_rationals_graphing_summary}{60}

\begin{document}
{\bf \large Lesson \arabic{lesson_fundamental_functions}: Fundamental Functions}\phantomsection\label{les:fundamental_functions}
%\\ CC attribute: \href{http://www.wallace.ccfaculty.org/book/book.html}{\it{Beginning and Intermediate Algebra}} by T. Wallace. 
%\\ CC attribute: \href{http://www.stitz-zeager.com}{\it{College Algebra}} by C. Stitz and J. Zeager. 
\hfill \doclicenseImage[imagewidth=5em]\\
\par
{\bf Objective:} Graph and identify the domain, range, and intercepts of any of the ten fundamental functions.\\
\par
{\bf Students will be able to:}
\begin{itemize}
	\item Identify (as well as produce) the graph of a variety of fundamental functions.
	\item Identify the domain and range of a variety of fundamental functions using a graph.
\end{itemize}
{\bf Prerequisite Knowledge:}
\begin{itemize}
	\item Definitions of domain and range of a function.
	\item Graph a function by plotting points.
\end{itemize}
\hrulefill

{\bf Lesson:}\\
\ \par
In this lesson, we focus on ten fundamental function types which will be referenced throughout the rest of the course, as well as one example of each.  Each type of function represents a ``building block'' for understanding the concepts of a traditional algebra course.\\
\ \par
Students should be able to both identify and sketch a graph of each function, as well as identify its intercepts, domain (both graphically and algebraically), and range (graphically).  Each representative form in the table below includes some element of generalization to reinforce understanding.
\begin{center}
\begin{tabular}{|c|c|l|}
\hline
Function Type & Representative Form & ~~~~~~Example\\
\hline
&&\\
Linear & $mx+b$ & $f(x)=3x-4$\\
&&\\
Quadratic & $ax^2+bx+c$ & $g(x)=x^2$\\
&&\\
Square Root & $\sqrt{x-h}$ & $k(x)=\sqrt{x}$\\
&&\\
Absolute Value & $|x-h|$ & $\ell(x)=|x|$\\
&&\\
Cubic & $(x-h)^3$ & $m(x)=x^3$\\
&&\\
Cube Root & $\sqrt[3]{x-h}$ & $n(x)=\sqrt[3]{x}$\\
&&\\
Reciprocal (Rational) & $\dfrac{1}{x-h}$ & $p(x)=\dfrac{1}{x}$\\
&&\\
\hline
\end{tabular}
\newpage
\begin{tabular}{|c|c|l|}
\hline
Function Type & Representative Form & ~~~~~~Example\\
\hline
&&\\
Semicircular & $\sqrt{r^2-x^2},~r>0$ & $q(x)=\sqrt{9-x^2}$\\
&&\\
Exponential* & $a^x,~a>0, a\neq 1$ & $r(x)=2^x$\\
&&\\
Logarithmic* & $\log_a(x),~a>0, a\neq1$ & $s(x)=\log_2(x)$\\
&&\\
\hline
\end{tabular}
\end{center}
*We have included Exponential and Logarithmic functions for a more complete list.  These functions are more formally treated in a Precalculus setting.\\ 
\ \par
{\bf I - Motivating Example(s):}\\
\ \par
{\bf Example:} 
\begin{multicols}{2}
\begin{tabular}{ll}
Function Type: & Linear $(m\neq 0)$\\
Representative: & $f(x)=3x-4$
\end{tabular}
\begin{center}
%\vspace{0.25in}
\begin{tabular}{c|c}
	$x$ & $f(x)$\\
	\hline
 & \\
 $-3$ & $-13$\\
 & \\
 $-2$ & $-10$\\
 & \\
 $-1$ & $-7$\\
 & \\
 0 & $-4$\\
 & \\
 1 & $-1$\\
 & \\
 $\dfrac{4}{3}$ & 0\\
 & \\
 2 & 2\\
 & \\
 3 & 5\\
 & \\
\end{tabular}
\end{center}
~\\
\vspace{0.25in}
~\\
\begin{tikzpicture}[xscale=0.75,yscale=0.75]
\draw [<->](-4,0) -- coordinate (x axis mid) (4,0) node[below right] {$x$};
\draw [<->](0,-5.25) -- coordinate (y axis mid) (0,5.25) node[above right] {$y$};
\foreach \x in {-3,...,-1}
\draw (\x,1pt) -- (\x,-3pt)
node[anchor=north] {\scriptsize \x};
\foreach \x in {1,...,3}
\draw (\x,1pt) -- (\x,-3pt)
node[anchor=north] {\scriptsize \x};
\foreach \y in {-5,...,-1}
\draw (1pt,\y) -- (-3pt,\y) 
node[anchor=east] {\scriptsize \y}; 
\foreach \y in {1,...,5}
\draw (1pt,\y) -- (-3pt,\y) 
node[anchor=east] {\scriptsize \y}; 
\draw [<->, domain=-0.33:3] plot (\x, {3*(\x)-4});
\draw[fill] (1.33,0) circle (0.075) node[above right] {}; 
\draw (1.4,0.6) node[right] {\scriptsize $\left(\dfrac{4}{3},0\right)$}; 
\draw[fill] (0,-4) circle (0.075);
\end{tikzpicture}
\begin{center}
Graph of $f(x)=3x-4$
\end{center}
\end{multicols}
\begin{multicols}{2}
\begin{tabular}{ll}
$y-$intercept: & $(0,-4)$\\
$x-$intercept(s): & $\left(\frac{4}{3},0\right)$
\end{tabular}

\columnbreak
\begin{tabular}{ll}
Domain: & $(-\infty,\infty)$\\
Range: & $(-\infty,\infty)$
\end{tabular}
\end{multicols}
Notes: If $m=0$, then the corresponding graph of $f(x)=b$ is a horizontal line.  The domain of $f$ is still $(-\infty,\infty)$, but the range consists of a single value, $\{b\}$.
\newpage
{\bf Example:} 
\begin{multicols}{2}
\begin{tabular}{ll}
Function Type: & Quadratic\\
Representative: &$g(x)=x^2$
\end{tabular}
\begin{center}
%\vspace{0.25in}
\begin{tabular}{c|c}
	$x$ & $g(x)$\\
	\hline
 & \\
 $-3$ & 9\\
 & \\
 $-2$ & 4\\
 & \\
 $-1$ & 1\\
 & \\
 0 & 0\\
 & \\
 1 & 1\\
 & \\
 2 & 4\\
 & \\
 3 & 9\\
 & \\
\end{tabular}
\end{center}
~\\
~\\
\begin{tikzpicture}[xscale=0.75,yscale=0.75]
\draw [<->](-4,0) -- coordinate (x axis mid) (4,0) node[below right] {$x$};
\draw [<->](0,-1) -- coordinate (y axis mid) (0,10) node[above right] {$y$};
\foreach \x in {-3,...,-1}
\draw (\x,1pt) -- (\x,-3pt)
node[anchor=north] {\scriptsize \x};
\foreach \x in {1,...,3}
\draw (\x,1pt) -- (\x,-3pt)
node[anchor=north] {\scriptsize \x};
%\foreach \y in {-3,...,-1}
%\draw (1pt,\y) -- (-3pt,\y); 
%node[anchor=east] {\scriptsize \y}; 
\foreach \y in {1,...,9}
\draw (1pt,\y) -- (-3pt,\y) 
node[anchor=east] {\scriptsize \y}; 
\draw [<->, domain=-3:3] plot (\x, {(\x)^2});
\draw[fill] (0,0) circle (0.075);
%\draw[fill] (-1,3) circle (0.075);
\end{tikzpicture}
\begin{center}
Graph of $g(x)=x^2$
\end{center}
\end{multicols}

\begin{multicols}{2}
\begin{tabular}{ll}
$y-$intercept: & $(0,0)$\\
$x-$intercept(s): & $(0,0)$
\end{tabular}

\columnbreak
\begin{tabular}{ll}
Domain: & $(-\infty,\infty)$\\
Range: & $[0,\infty),$ or $y\geq 0$
\end{tabular}
\end{multicols}

Notes: The domain of any quadratic function is $(-\infty,\infty)$.  If $g(x)=a(x-h)^2+k$, is a quadratic function in vertex form, then if $a>0$, the corresponding parabola will be concave \textit{up}, and the range of $g$ will be $[k,\infty)$.  If $a<0$, then the corresponding parabola will be concave \textit{down}, and the range of $g$ will be $(-\infty,k]$.\\
\ \par
{\bf II - Demo/Discussion Problems:}\\
\ \par
Complete each of the following for the functions listed below.
\begin{itemize}
	\item Use \Desmos \ to sketch a complete graph of the function.
	\item Construct a table of points from your graph.  Check your table by evaluating the function at the given $x-$coordinates.
	\item Identify the domain and range of your graph.
	\item Identify any $x-$ and $y-$intercepts of your graph.
\end{itemize}
\begin{enumerate}
	\begin{multicols}{2}
		\item $m(x)=x^3$
		\item $n(x)=\sqrt[3]{x}$
		\item $q(x)=\sqrt{9-x^2}$
		\item $r(x)=2^x$
	\end{multicols}
\end{enumerate}
{\bf III - Practice Problems:}\\
\ \par
Complete each of the following for the functions listed below.
\begin{itemize}
	\item Use \Desmos \ to sketch a complete graph of the function.
	\item Construct a table of points from your graph.  Check your table by evaluating the function at the given $x-$coordinates.
	\item Identify the domain and range of your graph.
	\item Identify any $x-$ and $y-$intercepts of your graph.
\end{itemize}
\begin{enumerate}
	\begin{multicols}{2}
		\item $k(x)=\sqrt{x}$
		\item $\ell(x)=|x|$
		\item $p(x)=\dfrac{1}{x}$
		\item $s(x)=\log_2(x)$
	\end{multicols}
\end{enumerate}
\newpage
\end{document}