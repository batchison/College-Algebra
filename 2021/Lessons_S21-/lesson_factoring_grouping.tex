\documentclass[12pt]{article}
\usepackage[top=1in,left=1in,bottom=1in,right=1in,headsep=2pt]{geometry}	
\usepackage{amssymb,amsmath,amsthm,amsfonts}
\usepackage{chapterfolder,docmute,setspace}
\usepackage{cancel,multicol,tikz,verbatim,framed,polynom,enumitem}
\usepackage[colorlinks, hyperindex, plainpages=false, linkcolor=blue, urlcolor=blue, pdfpagelabels]{hyperref}
% Use the cc-by-nc-sa license for any content linked with Stitz and Zeager's text.  Otherwise, use the cc-by-sa license.
\usepackage[type={CC},modifier={by-sa},version={4.0},]{doclicense}
%\usepackage[type={CC},modifier={by-nc-sa},version={4.0},]{doclicense}

\theoremstyle{definition}
\newtheorem{example}{Example}
\newcommand{\Desmos}{\href{https://www.desmos.com/}{Desmos}}
\setlength{\parindent}{0em}
\setlist{itemsep=0em}
\setlength{\parskip}{0.1em}
% This document is used for ordering of lessons.  If an instructor wishes to change the ordering of assessments, the following steps must be taken:

% 1) Reassign the appropriate numbers for each lesson in the \setcounter commands included in this file.
% 2) Rearrange the \include commands in the master file (the file with 'Course Pack' in the name) to accurately reflect the changes.  
% 3) Rearrange the \items in the measureable_outcomes file to accurately reflect the changes.  Be mindful of page breaks when moving items.
% 4) Re-build all affected files (master file, measureable_outcomes file, and any lesson whose numbering has changed).

%Note: The placement of each \newcounter and \setcounter command reflects the original/default ordering of topics (linears, systems, quadratics, functions, polynomials, rationals).

\newcounter{lesson_solving_linear_equations}
\newcounter{lesson_equations_containing_absolute_values}
\newcounter{lesson_graphing_lines}
\newcounter{lesson_two_forms_of_a_linear_equation}
\newcounter{lesson_parallel_and_perpendicular_lines}
\newcounter{lesson_linear_inequalities}
\newcounter{lesson_compound_inequalities}
\newcounter{lesson_inequalities_containing_absolute_values}
\newcounter{lesson_graphing_systems}
\newcounter{lesson_substitution}
\newcounter{lesson_elimination}
\newcounter{lesson_quadratics_introduction}
\newcounter{lesson_factoring_GCF}
\newcounter{lesson_factoring_grouping}
\newcounter{lesson_factoring_trinomials_a_is_1}
\newcounter{lesson_factoring_trinomials_a_neq_1}
\newcounter{lesson_solving_by_factoring}
\newcounter{lesson_square_roots}
\newcounter{lesson_i_and_complex_numbers}
\newcounter{lesson_vertex_form_and_graphing}
\newcounter{lesson_solve_by_square_roots}
\newcounter{lesson_extracting_square_roots}
\newcounter{lesson_the_discriminant}
\newcounter{lesson_the_quadratic_formula}
\newcounter{lesson_quadratic_inequalities}
\newcounter{lesson_functions_and_relations}
\newcounter{lesson_evaluating_functions}
\newcounter{lesson_finding_domain_and_range_graphically}
\newcounter{lesson_fundamental_functions}
\newcounter{lesson_finding_domain_algebraically}
\newcounter{lesson_solving_functions}
\newcounter{lesson_function_arithmetic}
\newcounter{lesson_composite_functions}
\newcounter{lesson_inverse_functions_definition_and_HLT}
\newcounter{lesson_finding_an_inverse_function}
\newcounter{lesson_transformations_translations}
\newcounter{lesson_transformations_reflections}
\newcounter{lesson_transformations_scalings}
\newcounter{lesson_transformations_summary}
\newcounter{lesson_piecewise_functions}
\newcounter{lesson_functions_containing_absolute_values}
\newcounter{lesson_absolute_as_piecewise}
\newcounter{lesson_polynomials_introduction}
\newcounter{lesson_sign_diagrams_polynomials}
\newcounter{lesson_factoring_quadratic_type}
\newcounter{lesson_factoring_summary}
\newcounter{lesson_polynomial_division}
\newcounter{lesson_synthetic_division}
\newcounter{lesson_end_behavior_polynomials}
\newcounter{lesson_local_behavior_polynomials}
\newcounter{lesson_rational_root_theorem}
\newcounter{lesson_polynomials_graphing_summary}
\newcounter{lesson_polynomial_inequalities}
\newcounter{lesson_rationals_introduction_and_terminology}
\newcounter{lesson_sign_diagrams_rationals}
\newcounter{lesson_horizontal_asymptotes}
\newcounter{lesson_slant_and_curvilinear_asymptotes}
\newcounter{lesson_vertical_asymptotes}
\newcounter{lesson_holes}
\newcounter{lesson_rationals_graphing_summary}

\setcounter{lesson_solving_linear_equations}{1}
\setcounter{lesson_equations_containing_absolute_values}{2}
\setcounter{lesson_graphing_lines}{3}
\setcounter{lesson_two_forms_of_a_linear_equation}{4}
\setcounter{lesson_parallel_and_perpendicular_lines}{5}
\setcounter{lesson_linear_inequalities}{6}
\setcounter{lesson_compound_inequalities}{7}
\setcounter{lesson_inequalities_containing_absolute_values}{8}
\setcounter{lesson_graphing_systems}{9}
\setcounter{lesson_substitution}{10}
\setcounter{lesson_elimination}{11}
\setcounter{lesson_quadratics_introduction}{16}
\setcounter{lesson_factoring_GCF}{17}
\setcounter{lesson_factoring_grouping}{18}
\setcounter{lesson_factoring_trinomials_a_is_1}{19}
\setcounter{lesson_factoring_trinomials_a_neq_1}{20}
\setcounter{lesson_solving_by_factoring}{21}
\setcounter{lesson_square_roots}{22}
\setcounter{lesson_i_and_complex_numbers}{23}
\setcounter{lesson_vertex_form_and_graphing}{24}
\setcounter{lesson_solve_by_square_roots}{25}
\setcounter{lesson_extracting_square_roots}{26}
\setcounter{lesson_the_discriminant}{27}
\setcounter{lesson_the_quadratic_formula}{28}
\setcounter{lesson_quadratic_inequalities}{29}
\setcounter{lesson_functions_and_relations}{12}
\setcounter{lesson_evaluating_functions}{13}
\setcounter{lesson_finding_domain_and_range_graphically}{14}
\setcounter{lesson_fundamental_functions}{15}
\setcounter{lesson_finding_domain_algebraically}{30}
\setcounter{lesson_solving_functions}{31}
\setcounter{lesson_function_arithmetic}{32}
\setcounter{lesson_composite_functions}{33}
\setcounter{lesson_inverse_functions_definition_and_HLT}{34}
\setcounter{lesson_finding_an_inverse_function}{35}
\setcounter{lesson_transformations_translations}{36}
\setcounter{lesson_transformations_reflections}{37}
\setcounter{lesson_transformations_scalings}{38}
\setcounter{lesson_transformations_summary}{39}
\setcounter{lesson_piecewise_functions}{40}
\setcounter{lesson_functions_containing_absolute_values}{41}
\setcounter{lesson_absolute_as_piecewise}{42}
\setcounter{lesson_polynomials_introduction}{43}
\setcounter{lesson_sign_diagrams_polynomials}{44}
\setcounter{lesson_factoring_quadratic_type}{46}
\setcounter{lesson_factoring_summary}{45}
\setcounter{lesson_polynomial_division}{47}
\setcounter{lesson_synthetic_division}{48}
\setcounter{lesson_end_behavior_polynomials}{49}
\setcounter{lesson_local_behavior_polynomials}{50}
\setcounter{lesson_rational_root_theorem}{51}
\setcounter{lesson_polynomials_graphing_summary}{52}
\setcounter{lesson_polynomial_inequalities}{53}
\setcounter{lesson_rationals_introduction_and_terminology}{54}
\setcounter{lesson_sign_diagrams_rationals}{55}
\setcounter{lesson_horizontal_asymptotes}{56}
\setcounter{lesson_slant_and_curvilinear_asymptotes}{57}
\setcounter{lesson_vertical_asymptotes}{58}
\setcounter{lesson_holes}{59}
\setcounter{lesson_rationals_graphing_summary}{60}

\begin{document}
{\bf \large Lesson \arabic{lesson_factoring_grouping}: Factor by Grouping}\phantomsection\label{les:factoring_grouping}
\\ CC attribute: \href{http://www.wallace.ccfaculty.org/book/book.html}{\it{Beginning and Intermediate Algebra}} by T. Wallace. 
%\\ CC attribute: \href{http://www.stitz-zeager.com}{\it{College Algebra}} by C. Stitz and J. Zeager. 
\hfill \doclicenseImage[imagewidth=5em]\\
\par
{\bf Objective:} Factor a tetranomial (four-term) expression by grouping.\\
\par
{\bf Students will be able to:}
\begin{itemize}
	\item Split a tetranomial expression into a pair of binomial  expressions that share a common factor.
	\item Factor out a GCF from each binomial expression, and group them using the distributive property.
\end{itemize}
{\bf Prerequisite Knowledge:}
\begin{itemize}
	\item Multiplication properties of exponents.
	\item Application of the distributive property.
	\item Multiplication and division of algebraic expressions.
	\item Identifying a greatest common factor.
\end{itemize}
\hrulefill

{\bf Lesson:}\\
\ \par
In this lesson, we will introduce another useful factorization strategy, known as {\it grouping}.  Grouping is typically employed when faced with an expression containing four terms.  Here, it is important to reinforce the fact that factoring is essentially expansion (multiplication) done in reverse.\\
\ \par
The first thing we will always do when factoring is try to factor out a GCF. This GCF is often a monomial (a single term) like in the expression $5 x y + 10 x z$.  Here, the GCF is the monomial $5 x$, so we would have $5 x (y + 2 z)$ as our answer. However, a GCF does not have to be a monomial.  It could, in fact, be a binomial and contain two terms, as we will see with grouping.\\
\ \par
When attempting to factor by grouping, we will always divide an expression into two parts, or groups: group one will usually contain the first two terms of our expression and group two will contain the last two terms. Then we can identify and factor the GCF out of each group.  In doing this, our hope is that what is left over in each group will be the same binomial expression. If the resulting expressions match, we can then factor out this matching expression from both of our designated groups, writing what remains in a new set of parentheses.\\
\ \par
The key for grouping to be successful is for the two binomials to match exactly, once the GCF has been factored out of both groups. If there is any difference between the two binomials, we either have to choose two different groups or we cannot factor by grouping.\\
\ \par

{\bf I - Motivating Example(s):}\\

{\bf Example:} Find and factor out the GCF of the given expression.
  \begin{eqnarray*}
    3 a x - 7 b x &  & \text{Both terms have an} \ x \ \text{in common, factor it out}\\
    x (3 a - 7 b) &  & \text{Our solution}
  \end{eqnarray*}

Now we will work with the same expression, replacing $x$ with $(2 a + 5 b)$.  In the same way that we factored out a GCF of $x$ we can factor out a GCF which is a binomial, such as $(2 a + 5 b)$.\\
\ \par
{\bf Example:} Find and factor out the GCF of the given expression.
  \begin{eqnarray*}
    3 a (2 a + 5 b) - 7 b (2 a + 5 b) &  & \text{Both terms have} \ (2 a + 5 b) \
    \text{in common, factor it out}\\
    (2 a + 5 b) (3 a - 7 b) &  & \text{Our solution}
  \end{eqnarray*}

{\bf Example:} Write the expanded form for the given expression.
  \begin{eqnarray*}
    (2 a + 3) (5 b + 2) &  & \text{Distribute} \ (2 a + 3) \ \text{into the second set of parentheses}\\
    5 b (2 a + 3) + 2 (2 a + 3) &  & \text{Distribute each monomial}\\
    10ab + 15 b + 4 a + 6 &  & \text{Our solution}
  \end{eqnarray*}

Our solution above has four terms in it.  We arrived at this solution by focusing on the two parts, $5 b (2 a + 3)$ and $2 (2 a + 3)$.  Reversing the process above is the central idea behind grouping.\\
\ \par
{\bf Example:} Factor the given expression by grouping.
  \begin{eqnarray*}
    10 ab + 15 b + 4 a + 6 &  & \text{Split the expression into two groups}\\
(10 ab + 15 b) + (4 a + 6)&  & \text{GCF on the left is} \ 5 b, \ \text{on the right is 2}\\
5 b (2 a + 3) + 2 (2 a + 3) &  & (2 a + 3) \ \text{appears twice! Factor out this GCF.}\\
    (2 a + 3) (5 b + 2) &  & \text{Our solution}
  \end{eqnarray*}
	
{\bf II - Demo/Discussion Problems:}\\
\ \par
Factor each of the given expressions by grouping.
\begin{multicols}{2}
\begin{enumerate}
	\item $5xy-8x-10y+16$
	\item $12ab-14a-6b+7$
	\item $6 x^2 + 9 x y - 14 x - 21 y$
	\item $6 x^3 - 15 x^2 + 2 x - 5$
	\item $4 a^2 - 21 b^3 + 6 a b - 14 a b^2$
	\item $7 + y - 3 x y - 21 x$
	\item $8 x y - 12 y + 15 - 10$
\end{enumerate}
\end{multicols}
\newpage
{\bf III - Practice Problems:}\\
\ \par
Factor each of the given expressions by grouping.
\begin{multicols}{2}
  \begin{enumerate}
	\item $40 r^3 - 8 r^2 - 25 r + 5$
  \item $35 x^3 - 10 x^2 - 56 x + 16$
  \item $3 n^3 - 2 n^2 - 9 n + 6$
  \item $14 v^3 + 10 v^2 - 7 v - 5$
  \item $15 b^3 + 21 b^2 - 35 b - 49$
  \item $6 x^3 - 48 x^2 + 5 x - 40$
  \item $3 x^3 + 15 x^2 + 2 x + 10$
  \item $28 p^3 + 21 p^2 + 20 p + 15$
  \item $35 x^3 - 28 x^2 - 20 x + 16$
  \item $7 n^3 + 21 n^2 - 5 n - 15$
  \item $7 x y - 49 x + 5 y - 35$
  \item $42 r^3 - 49 r^2 + 18 r - 21$
  \item $32 x y + 40 x^2 + 12 y + 15 x$
  \item $15 a b - 6 a + 5 b^3 - 2 b^2$
  \item $16 x y - 56 x + 2 y - 7$
  \item $3 m n - 8 m + 15 n - 40$
  \item $2 x y - 8 x^2 + 7 y^3 - 28 y^2 x$
  \item $5 m n + 2 m - 25 n - 10$
  \item $40 x y + 35 x - 8 y^2 - 7 y$
  \item $8 x y + 56 x - y - 7$
  \item $32 u v - 20 u + 24 v - 15$
  \item $4 u v + 14 u^2 + 12 v + 42 u$
  \item $10 x y + 30 + 25 x + 12 y$
  \item $24 x y + 25 y^2 - 20 x - 30 y^3$
  \item $3 u v + 14 u - 6 u^2 - 7 v$
  \item $56 a b + 14 - 49 a - 16 b$
  \item $16 x y - 3 x - 6 x^2 + 8 y$
	\end{enumerate}
\end{multicols}
\newpage
\ \newpage
\end{document}