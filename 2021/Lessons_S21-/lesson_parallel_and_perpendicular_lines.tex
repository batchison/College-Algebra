\documentclass[12pt]{article}
\usepackage[top=1in,left=1in,bottom=1in,right=1in,headsep=2pt]{geometry}	
\usepackage{amssymb,amsmath,amsthm,amsfonts}
\usepackage{chapterfolder,docmute,setspace}
\usepackage{cancel,multicol,tikz,verbatim,framed,polynom,enumitem}
\usepackage[colorlinks, hyperindex, plainpages=false, linkcolor=blue, urlcolor=blue, pdfpagelabels]{hyperref}
\usepackage[type={CC},modifier={by-sa},version={4.0},]{doclicense}

\theoremstyle{definition}
\newtheorem{example}{Example}
\newcommand{\Desmos}{\href{https://www.desmos.com/}{Desmos}}
\setlength{\parindent}{0em}
\setlist{itemsep=0em}
\setlength{\parskip}{0.1em}
% This document is used for ordering of lessons.  If an instructor wishes to change the ordering of assessments, the following steps must be taken:

% 1) Reassign the appropriate numbers for each lesson in the \setcounter commands included in this file.
% 2) Rearrange the \include commands in the master file (the file with 'Course Pack' in the name) to accurately reflect the changes.  
% 3) Rearrange the \items in the measureable_outcomes file to accurately reflect the changes.  Be mindful of page breaks when moving items.
% 4) Re-build all affected files (master file, measureable_outcomes file, and any lesson whose numbering has changed).

%Note: The placement of each \newcounter and \setcounter command reflects the original/default ordering of topics (linears, systems, quadratics, functions, polynomials, rationals).

\newcounter{lesson_solving_linear_equations}
\newcounter{lesson_equations_containing_absolute_values}
\newcounter{lesson_graphing_lines}
\newcounter{lesson_two_forms_of_a_linear_equation}
\newcounter{lesson_parallel_and_perpendicular_lines}
\newcounter{lesson_linear_inequalities}
\newcounter{lesson_compound_inequalities}
\newcounter{lesson_inequalities_containing_absolute_values}
\newcounter{lesson_graphing_systems}
\newcounter{lesson_substitution}
\newcounter{lesson_elimination}
\newcounter{lesson_quadratics_introduction}
\newcounter{lesson_factoring_GCF}
\newcounter{lesson_factoring_grouping}
\newcounter{lesson_factoring_trinomials_a_is_1}
\newcounter{lesson_factoring_trinomials_a_neq_1}
\newcounter{lesson_solving_by_factoring}
\newcounter{lesson_square_roots}
\newcounter{lesson_i_and_complex_numbers}
\newcounter{lesson_vertex_form_and_graphing}
\newcounter{lesson_solve_by_square_roots}
\newcounter{lesson_extracting_square_roots}
\newcounter{lesson_the_discriminant}
\newcounter{lesson_the_quadratic_formula}
\newcounter{lesson_quadratic_inequalities}
\newcounter{lesson_functions_and_relations}
\newcounter{lesson_evaluating_functions}
\newcounter{lesson_finding_domain_and_range_graphically}
\newcounter{lesson_fundamental_functions}
\newcounter{lesson_finding_domain_algebraically}
\newcounter{lesson_solving_functions}
\newcounter{lesson_function_arithmetic}
\newcounter{lesson_composite_functions}
\newcounter{lesson_inverse_functions_definition_and_HLT}
\newcounter{lesson_finding_an_inverse_function}
\newcounter{lesson_transformations_translations}
\newcounter{lesson_transformations_reflections}
\newcounter{lesson_transformations_scalings}
\newcounter{lesson_transformations_summary}
\newcounter{lesson_piecewise_functions}
\newcounter{lesson_functions_containing_absolute_values}
\newcounter{lesson_absolute_as_piecewise}
\newcounter{lesson_polynomials_introduction}
\newcounter{lesson_sign_diagrams_polynomials}
\newcounter{lesson_factoring_quadratic_type}
\newcounter{lesson_factoring_summary}
\newcounter{lesson_polynomial_division}
\newcounter{lesson_synthetic_division}
\newcounter{lesson_end_behavior_polynomials}
\newcounter{lesson_local_behavior_polynomials}
\newcounter{lesson_rational_root_theorem}
\newcounter{lesson_polynomials_graphing_summary}
\newcounter{lesson_polynomial_inequalities}
\newcounter{lesson_rationals_introduction_and_terminology}
\newcounter{lesson_sign_diagrams_rationals}
\newcounter{lesson_horizontal_asymptotes}
\newcounter{lesson_slant_and_curvilinear_asymptotes}
\newcounter{lesson_vertical_asymptotes}
\newcounter{lesson_holes}
\newcounter{lesson_rationals_graphing_summary}

\setcounter{lesson_solving_linear_equations}{1}
\setcounter{lesson_equations_containing_absolute_values}{2}
\setcounter{lesson_graphing_lines}{3}
\setcounter{lesson_two_forms_of_a_linear_equation}{4}
\setcounter{lesson_parallel_and_perpendicular_lines}{5}
\setcounter{lesson_linear_inequalities}{6}
\setcounter{lesson_compound_inequalities}{7}
\setcounter{lesson_inequalities_containing_absolute_values}{8}
\setcounter{lesson_graphing_systems}{9}
\setcounter{lesson_substitution}{10}
\setcounter{lesson_elimination}{11}
\setcounter{lesson_quadratics_introduction}{16}
\setcounter{lesson_factoring_GCF}{17}
\setcounter{lesson_factoring_grouping}{18}
\setcounter{lesson_factoring_trinomials_a_is_1}{19}
\setcounter{lesson_factoring_trinomials_a_neq_1}{20}
\setcounter{lesson_solving_by_factoring}{21}
\setcounter{lesson_square_roots}{22}
\setcounter{lesson_i_and_complex_numbers}{23}
\setcounter{lesson_vertex_form_and_graphing}{24}
\setcounter{lesson_solve_by_square_roots}{25}
\setcounter{lesson_extracting_square_roots}{26}
\setcounter{lesson_the_discriminant}{27}
\setcounter{lesson_the_quadratic_formula}{28}
\setcounter{lesson_quadratic_inequalities}{29}
\setcounter{lesson_functions_and_relations}{12}
\setcounter{lesson_evaluating_functions}{13}
\setcounter{lesson_finding_domain_and_range_graphically}{14}
\setcounter{lesson_fundamental_functions}{15}
\setcounter{lesson_finding_domain_algebraically}{30}
\setcounter{lesson_solving_functions}{31}
\setcounter{lesson_function_arithmetic}{32}
\setcounter{lesson_composite_functions}{33}
\setcounter{lesson_inverse_functions_definition_and_HLT}{34}
\setcounter{lesson_finding_an_inverse_function}{35}
\setcounter{lesson_transformations_translations}{36}
\setcounter{lesson_transformations_reflections}{37}
\setcounter{lesson_transformations_scalings}{38}
\setcounter{lesson_transformations_summary}{39}
\setcounter{lesson_piecewise_functions}{40}
\setcounter{lesson_functions_containing_absolute_values}{41}
\setcounter{lesson_absolute_as_piecewise}{42}
\setcounter{lesson_polynomials_introduction}{43}
\setcounter{lesson_sign_diagrams_polynomials}{44}
\setcounter{lesson_factoring_quadratic_type}{46}
\setcounter{lesson_factoring_summary}{45}
\setcounter{lesson_polynomial_division}{47}
\setcounter{lesson_synthetic_division}{48}
\setcounter{lesson_end_behavior_polynomials}{49}
\setcounter{lesson_local_behavior_polynomials}{50}
\setcounter{lesson_rational_root_theorem}{51}
\setcounter{lesson_polynomials_graphing_summary}{52}
\setcounter{lesson_polynomial_inequalities}{53}
\setcounter{lesson_rationals_introduction_and_terminology}{54}
\setcounter{lesson_sign_diagrams_rationals}{55}
\setcounter{lesson_horizontal_asymptotes}{56}
\setcounter{lesson_slant_and_curvilinear_asymptotes}{57}
\setcounter{lesson_vertical_asymptotes}{58}
\setcounter{lesson_holes}{59}
\setcounter{lesson_rationals_graphing_summary}{60}

\begin{document}
{\bf \large Lesson \arabic{lesson_parallel_and_perpendicular_lines}: Parallel and Perpendicular Lines}\\
CC attribute: \href{http://www.wallace.ccfaculty.org/book/book.html}{\it{Beginning and Intermediate Algebra}} by T. Wallace. \hfill \doclicenseImage[imagewidth=5em]\\
\par
{\bf Objective:} Write the equation of a line given a line parallel or perpendicular.\\
\par
{\bf Students will be able to:}
\begin{itemize}
	\item Find a perpendicular slope, given a slope or linear equation.
	\item Find a parallel slope, given a slope or linear equation.
	\item Find the equation of a line parallel or perpendicular to a given linear equation.
\end{itemize}
{\bf Prerequisite Knowledge:}
\begin{itemize}
	\item Identify the slope of a line.
	\item Work with the slope-intercept form of a line.
	\item Work with the point-slope form of a line.
\end{itemize}
\hrulefill

{\bf Lesson:}
\par
{\bf I - Motivating Example(s):}
\begin{multicols}{2}
	\begin{tikzpicture}[xscale=0.4,yscale=0.4]
		\draw[step=1.0,gray,very thin,dotted] (-8.5,-5.5) grid (8.5,5.5);
		\draw [<->](-8.5,0) -- coordinate (x axis mid) (8.5,0) node[below right] {$x$};
		\draw [<->](0,-5.5) -- coordinate (y axis mid) (0,5.5) node[above right] {$y$};
		\draw [<->,line width=0.4mm] plot [domain=-4:7, samples=100] (\x,{-0.667*\x+2});
		\draw [<->,line width=0.4mm] plot [domain=-7:4, samples=100] (\x,{-0.667*\x-2.667});
		\draw[fill] (0,2) circle (0.2);
		\draw[fill] (-3,4) circle (0.2);
		\draw[fill] (-4,0) circle (0.2);
		\draw[fill] (-1,-2) circle (0.2);
		\draw (-2,4.5) node {$\ell_1$};
		\draw (-6,2.5) node {$\ell_2$};
	\end{tikzpicture}
\columnbreak
\ \par
\ \par
This graph shows two parallel lines.\\
\par
The slope (rise over run) of each line is ``down 2, right 3,'' or $m_1=m_2=-\frac{2}{3}$.
\end{multicols}

\begin{multicols}{2}
	\begin{tikzpicture}[xscale=0.4,yscale=0.4]
		\draw[step=1.0,gray,very thin,dotted] (-8.5,-5.5) grid (8.5,5.5);
		\draw [<->](-8.5,0) -- coordinate (x axis mid) (8.5,0) node[below right] {$x$};
		\draw [<->](0,-5.5) -- coordinate (y axis mid) (0,5.5) node[above right] {$y$};
		\draw [<->,line width=0.4mm] plot [domain=-8:7, samples=100] (\x,{0.667*\x+0.333});
		\draw [<->,line width=0.4mm] plot [domain=-7:-0.5, samples=100] (\x,{-1.5*\x-6});
		\draw[fill] (1,1) circle (0.2);
		\draw[fill] (-2,-1) circle (0.2);
		\draw[fill] (-4,0) circle (0.2);
		\draw[fill] (-6,3) circle (0.2);
		\draw (6,3) node {$\ell_1$};
		\draw (-4.5,2.5) node {$\ell_2$};
	\end{tikzpicture}
\columnbreak
\ \par
\ \par
This graph shows two perpendicular lines.\\
\par
The slope (rise over run) of the more gradual line is ``up 2, right 3,'' or $m_1=\frac{2}{3}$.\\
\par
The slope of the steeper line is ``down 3, right 2,'' or $m_2=-\frac{3}{2}$.
\end{multicols}

\newpage

As the first graph illustrates, {\bf parallel lines} have the {\bf same slope}, $m_1=m_2$.\\
\par
On the other hand, the second graph shows us that {\bf perpendicular lines} have {\bf negative reciprocal slopes}, $m_2=-\frac{1}{m_1}$ (and so, $m_1\cdot m_2=-1$).\\
\par
We can use these properties to make conclusions about whether two lines are parallel, perpendicular, or neither.\\
\par
{\bf II - Demo/Discussion Problems:}
\begin{enumerate}
	\item Find the equation of the line through (6,-9) perpendicular to the line $y=-\frac{3}{5}x+4$.
	\item Find the equation of a line through (4,-5) and parallel to the line $2x-3y=6$.
	\item Find the equation of the line through (3,4) and perpendicular to the line $x=-2$.
\end{enumerate}
{\bf III - Practice Problems:}\\
\par
Find the slope of a line parallel to each given line.
\begin{multicols}{4}
  1) $y = 2 x + 4$\\
  2) $y = - \frac{2}{3} x + 5$\\
  3) $y = 4 x - 5$\\
  4) $y = - \frac{10}{3} x - 5$\\
  5) $x - y = 4$\\
  6) $6 x - 5 y = 20$\\
  7) $7 x + y = - 2$\\
  8) $3 x + 4 y = - 8$
\end{multicols}
Find the slope of a line perpendicular to each given line.
\begin{multicols}{4}
  9) $x = 3$\\
  10) $y = - \frac{1}{2} x - 1$\\
  11) $y = - \frac{1}{3} x$\\
  12) $y = \frac{4}{5} x$\\
  13) $x - 3 y = - 6$\\
  14) $3 x - y = - 3$\\
  15) $x + 2 y = 8$\\
  16) $8 x - 3 y = - 9$
\end{multicols}
Write the point-slope form of the equation of the line
described.\\
\par
17) through $(2, 5)$, parallel to $x = 0$\\
18) through $(5, 2)$, parallel to $y = \frac{7}{5} x + 4$\\
19) through $(3, 4)$, parallel to $y = \frac{9}{2} x - 5$\\
20) through $(1, - 1)$, parallel to $y = - \frac{3}{4} x + 3$\\
21) through $(2, 3)$, parallel to $y = \frac{7}{5} x + 4$\\
22) through $(- 1, 3)$, parallel to $y = - 3 x - 1$\\
23) through $(4, 2)$, parallel to $x = 0$\\
24) through $(1, 4)$, parallel to y = $\frac{7}{5} x + 2$\\
25) through (1, $- 5)$, perpendicular to $- x + y = 1$\\
26) through $(1, - 2)$, perpendicular to $- x + 2 y = 2$\\
27) through $(5, 2)$, perpendicular to $5 x + y = - 3$\\
28) through $(1, 3)$, perpendicular to $- x + y = 1$\\
29) through $(4, 2)$, perpendicular to $- 4 x + y = 0$\\
30) through $(- 3, - 5)$, perpendicular to $3 x + 7 y = 0$\\
31) through $(2, - 2)$, perpendicular to $3 y - x = 0$\\
32) through $(- 2, 5)$, perpendicular to $y - 2 x = 0$\\
\par
Write the slope-intercept form of the equation of the line
described.\\
\par
33) through $(4, - 3)$, parallel to $y = - 2 x$\\
34) through $(- 5, 2)$, parallel to $y = \frac{3}{5} x$\\
35) through $(- 3, 1)$, parallel to $y = - \frac{4}{3} x- 1$\\
36) through $(- 4, 0)$, parallel to $y = - \frac{5}{4} x+ 4$\\
37) through $(- 4, - 1)$, parallel to $y = - \frac{1}{2}x + 1$\\
38) through $(2, 3)$, parallel to $y = \frac{5}{2} x - 1$\\
39) through $(- 2, - 1)$, parallel to $y = - \frac{1}{2}x - 2$\\
40) through $(- 5, - 4)$, parallel to $y = \frac{3}{5} x- 2$\\
41) through $(4, 3)$, perpendicular to $x + y = - 1$\\
42) through $(- 3, - 5)$, perpendicular to $x + 2 y = -4$\\
43) through $(5, 2)$, perpendicular to $x = 0$\\
44) through $(5, - 1)$, perpendicular to $- 5 x + 2 y =10$\\
45) through $(- 2, 5)$, perpendicular to $- x + y = - 2$\\
46) through $(2, - 3)$, perpendicular to $- 2 x + 5 y = -10$\\
47) through $(4, - 3)$, perpendicular to $- x + 2 y = -6$\\
48) through $(- 4, 1)$, perpendicular to $4 x + 3 y = -9$
\newpage
\ \newpage
\end{document}