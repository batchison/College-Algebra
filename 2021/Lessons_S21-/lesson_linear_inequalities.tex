\documentclass[12pt]{article}
\usepackage[top=1in,left=1in,bottom=1in,right=1in,headsep=2pt]{geometry}	
\usepackage{amssymb,amsmath,amsthm,amsfonts}
\usepackage{chapterfolder,docmute,setspace}
\usepackage{cancel,multicol,tikz,verbatim,framed,polynom,enumitem}
\usepackage[colorlinks, hyperindex, plainpages=false, linkcolor=blue, urlcolor=blue, pdfpagelabels]{hyperref}
\usepackage[type={CC},modifier={by-sa},version={4.0},]{doclicense}

\theoremstyle{definition}
\newtheorem{example}{Example}
\newcommand{\Desmos}{\href{https://www.desmos.com/}{Desmos}}
\setlength{\parindent}{0em}
\setlist{itemsep=0em}
\setlength{\parskip}{0.1em}
% This document is used for ordering of lessons.  If an instructor wishes to change the ordering of assessments, the following steps must be taken:

% 1) Reassign the appropriate numbers for each lesson in the \setcounter commands included in this file.
% 2) Rearrange the \include commands in the master file (the file with 'Course Pack' in the name) to accurately reflect the changes.  
% 3) Rearrange the \items in the measureable_outcomes file to accurately reflect the changes.  Be mindful of page breaks when moving items.
% 4) Re-build all affected files (master file, measureable_outcomes file, and any lesson whose numbering has changed).

%Note: The placement of each \newcounter and \setcounter command reflects the original/default ordering of topics (linears, systems, quadratics, functions, polynomials, rationals).

\newcounter{lesson_solving_linear_equations}
\newcounter{lesson_equations_containing_absolute_values}
\newcounter{lesson_graphing_lines}
\newcounter{lesson_two_forms_of_a_linear_equation}
\newcounter{lesson_parallel_and_perpendicular_lines}
\newcounter{lesson_linear_inequalities}
\newcounter{lesson_compound_inequalities}
\newcounter{lesson_inequalities_containing_absolute_values}
\newcounter{lesson_graphing_systems}
\newcounter{lesson_substitution}
\newcounter{lesson_elimination}
\newcounter{lesson_quadratics_introduction}
\newcounter{lesson_factoring_GCF}
\newcounter{lesson_factoring_grouping}
\newcounter{lesson_factoring_trinomials_a_is_1}
\newcounter{lesson_factoring_trinomials_a_neq_1}
\newcounter{lesson_solving_by_factoring}
\newcounter{lesson_square_roots}
\newcounter{lesson_i_and_complex_numbers}
\newcounter{lesson_vertex_form_and_graphing}
\newcounter{lesson_solve_by_square_roots}
\newcounter{lesson_extracting_square_roots}
\newcounter{lesson_the_discriminant}
\newcounter{lesson_the_quadratic_formula}
\newcounter{lesson_quadratic_inequalities}
\newcounter{lesson_functions_and_relations}
\newcounter{lesson_evaluating_functions}
\newcounter{lesson_finding_domain_and_range_graphically}
\newcounter{lesson_fundamental_functions}
\newcounter{lesson_finding_domain_algebraically}
\newcounter{lesson_solving_functions}
\newcounter{lesson_function_arithmetic}
\newcounter{lesson_composite_functions}
\newcounter{lesson_inverse_functions_definition_and_HLT}
\newcounter{lesson_finding_an_inverse_function}
\newcounter{lesson_transformations_translations}
\newcounter{lesson_transformations_reflections}
\newcounter{lesson_transformations_scalings}
\newcounter{lesson_transformations_summary}
\newcounter{lesson_piecewise_functions}
\newcounter{lesson_functions_containing_absolute_values}
\newcounter{lesson_absolute_as_piecewise}
\newcounter{lesson_polynomials_introduction}
\newcounter{lesson_sign_diagrams_polynomials}
\newcounter{lesson_factoring_quadratic_type}
\newcounter{lesson_factoring_summary}
\newcounter{lesson_polynomial_division}
\newcounter{lesson_synthetic_division}
\newcounter{lesson_end_behavior_polynomials}
\newcounter{lesson_local_behavior_polynomials}
\newcounter{lesson_rational_root_theorem}
\newcounter{lesson_polynomials_graphing_summary}
\newcounter{lesson_polynomial_inequalities}
\newcounter{lesson_rationals_introduction_and_terminology}
\newcounter{lesson_sign_diagrams_rationals}
\newcounter{lesson_horizontal_asymptotes}
\newcounter{lesson_slant_and_curvilinear_asymptotes}
\newcounter{lesson_vertical_asymptotes}
\newcounter{lesson_holes}
\newcounter{lesson_rationals_graphing_summary}

\setcounter{lesson_solving_linear_equations}{1}
\setcounter{lesson_equations_containing_absolute_values}{2}
\setcounter{lesson_graphing_lines}{3}
\setcounter{lesson_two_forms_of_a_linear_equation}{4}
\setcounter{lesson_parallel_and_perpendicular_lines}{5}
\setcounter{lesson_linear_inequalities}{6}
\setcounter{lesson_compound_inequalities}{7}
\setcounter{lesson_inequalities_containing_absolute_values}{8}
\setcounter{lesson_graphing_systems}{9}
\setcounter{lesson_substitution}{10}
\setcounter{lesson_elimination}{11}
\setcounter{lesson_quadratics_introduction}{16}
\setcounter{lesson_factoring_GCF}{17}
\setcounter{lesson_factoring_grouping}{18}
\setcounter{lesson_factoring_trinomials_a_is_1}{19}
\setcounter{lesson_factoring_trinomials_a_neq_1}{20}
\setcounter{lesson_solving_by_factoring}{21}
\setcounter{lesson_square_roots}{22}
\setcounter{lesson_i_and_complex_numbers}{23}
\setcounter{lesson_vertex_form_and_graphing}{24}
\setcounter{lesson_solve_by_square_roots}{25}
\setcounter{lesson_extracting_square_roots}{26}
\setcounter{lesson_the_discriminant}{27}
\setcounter{lesson_the_quadratic_formula}{28}
\setcounter{lesson_quadratic_inequalities}{29}
\setcounter{lesson_functions_and_relations}{12}
\setcounter{lesson_evaluating_functions}{13}
\setcounter{lesson_finding_domain_and_range_graphically}{14}
\setcounter{lesson_fundamental_functions}{15}
\setcounter{lesson_finding_domain_algebraically}{30}
\setcounter{lesson_solving_functions}{31}
\setcounter{lesson_function_arithmetic}{32}
\setcounter{lesson_composite_functions}{33}
\setcounter{lesson_inverse_functions_definition_and_HLT}{34}
\setcounter{lesson_finding_an_inverse_function}{35}
\setcounter{lesson_transformations_translations}{36}
\setcounter{lesson_transformations_reflections}{37}
\setcounter{lesson_transformations_scalings}{38}
\setcounter{lesson_transformations_summary}{39}
\setcounter{lesson_piecewise_functions}{40}
\setcounter{lesson_functions_containing_absolute_values}{41}
\setcounter{lesson_absolute_as_piecewise}{42}
\setcounter{lesson_polynomials_introduction}{43}
\setcounter{lesson_sign_diagrams_polynomials}{44}
\setcounter{lesson_factoring_quadratic_type}{46}
\setcounter{lesson_factoring_summary}{45}
\setcounter{lesson_polynomial_division}{47}
\setcounter{lesson_synthetic_division}{48}
\setcounter{lesson_end_behavior_polynomials}{49}
\setcounter{lesson_local_behavior_polynomials}{50}
\setcounter{lesson_rational_root_theorem}{51}
\setcounter{lesson_polynomials_graphing_summary}{52}
\setcounter{lesson_polynomial_inequalities}{53}
\setcounter{lesson_rationals_introduction_and_terminology}{54}
\setcounter{lesson_sign_diagrams_rationals}{55}
\setcounter{lesson_horizontal_asymptotes}{56}
\setcounter{lesson_slant_and_curvilinear_asymptotes}{57}
\setcounter{lesson_vertical_asymptotes}{58}
\setcounter{lesson_holes}{59}
\setcounter{lesson_rationals_graphing_summary}{60}

\begin{document}
{\bf \large Lesson \arabic{lesson_linear_inequalities}: Linear Inequalities}\phantomsection\label{les:linear_inequalities}\\
CC attribute: \href{http://www.wallace.ccfaculty.org/book/book.html}{\it{Beginning and Intermediate Algebra}} by T. Wallace. \hfill \doclicenseImage[imagewidth=5em]\\
\par
{\bf Objective:} Solve, graph, and give interval notation for the solution to a linear inequality.  Create a sign diagram to identify those intervals where a linear expression is positive or negative.\\
\par
{\bf Students will be able to:}
\begin{itemize}
	\item Solve a linear inequality by isolating the variable.
	\item Recognize the need to change the direction of an inequality when multiplying or dividing by a negative value.
	\item Graph a linear inequality on a one-dimensional axis.
	\item Express solutions using interval notation.
\end{itemize}
{\bf Prerequisite Knowledge:}
\begin{itemize}
	\item Apply the distributive property.
	\item Verify the accuracy of a solution to an inequality by checking.
\end{itemize}
\hrulefill

{\bf Lesson:}
\par
{\bf I - Motivating Example(s):}\\
\par
Solve the linear inequality  $4x-3\geq 5$. 
\begin{eqnarray*}
4x-3\geq 5~~~&&\\
{\bf\underline{+3}~~~~\underline{+3}} &&  \text{Add~} 3 \text{~to~both~sides} \\
4x\geq 8~~~ && \\
{\bf\overline{4}~~~~\overline{4}}~~~&& \text{Divide~both~sides~by~} 4\\
x\geq 2~~~ && \text{Our~solution}
\end{eqnarray*}

Our solution can be expressed as follows.

\begin{enumerate}
	\item Verbally: ``The set of all values of $x$ that are greater than or equal to (at least) $2$''.
	\item Inequality: $\{x|x\geq 2\}$
	\item Interval: $[2,\infty)$
	\item Real-number Line (Graphically): 
\end{enumerate}

\begin{center}
\begin{tikzpicture}[xscale=1,yscale=1]
	\draw [<->](-5,0) -- coordinate (x axis mid) (5,0) node[below right] {$x$};
	\draw [->,line width=1.5mm](2,0) -- coordinate (x axis mid) (5,0);
	\draw (2,-1) node {$2$};
	\draw (2,0) node {\huge $[$};
\end{tikzpicture}
\end{center}
Note: A closed (shaded) circle at $x=2$ is also acceptable in place of a bracket.
\newpage
\underline{Check}:\par

\begin{center}
\begin{tabular}{ccccc}
\underline{Test Location} & \underline{Test Value} & \underline{Unsimplified} & \underline{Simplified} & \underline{Result}\\
Shaded region & $x=3$ & $4(3)-3\geq 5$ & $~9\geq 5$ & True\\
Boundary value & $x=2$ & $4(2)-3\geq 5$ & $~5\geq 5$ & True\\
Unshaded region & $x=0$ & $4(0)-3\geq 5$ & $-3\geq 5$ &False
\end{tabular}
\end{center}

{\bf II - Demo/Discussion Problems:}\\
\ \par
Solve the linear inequality  $-1-2(x-3)\leq 5x-9$.\\
\ \par
{\bf III - Practice Problems:}\\
\ \par
Draw a graph for each inequality below and provide interval notation.

\begin{multicols}{3}
  1) $n > - 5$\\
  2) $n > 4$\\
  3) $- 2 \geq k$\\
  4) $1 \geq k$\\
  5) $5 \geq x$\\
  6) $- 5 < x$
\end{multicols}

Solve each inequality, graph each solution, and provide interval
notation.

\begin{multicols}{3}
  7) $\dfrac{x}{11} \geq 10$\\
\ \\
  8) $- 2 \leq \dfrac{n}{13}$\\
\ \\
  9) $2 + r < 3$\\
\ \\
  10) $\dfrac{m}{5} \leq - \dfrac{6}{5}$\\
\ \\
  11) $8 + \dfrac{n}{3} \geq 6$\\
\ \\
  12) $11 > 8 + \dfrac{x}{2}$\\
\ \\
  13) $2 > \dfrac{a - 2}{5}$\\
\ \\
  14) $\dfrac{v - 9}{- 4} \leq 2$\\
\ \\
  15) $\dfrac{6 + x}{12} \leq - 1$\\
\end{multicols}
\begin{multicols}{2}
  16) $- 47 \geq 8 - 5 x$\\ \ \\
  17) $- 2 (3 + k) < - 44$\\ \ \\
  18) $- 7 n - 10 \geq 60$\\ \ \\
  19) $18 < - 2 (- 8 + p)$\\ \ \\
  20) $5 \geq \dfrac{x}{5} + 1$\\ \ \\
  21) $24 \geq - 6 (m - 6)$\\ \ \\
	22) $- 8 (n - 5) \geq 0$\\ \ \\
  23) $- r - 5 (r - 6) < - 18$\\ \ \\
  24) $- 60 \geq - 4 (- 6 x - 3)$\\ \ \\
	25) $24 + 4 b < 4 (1 + 6 b)$\\ \ \\
  26) $- 8 (2 - 2 n) \geq - 16 + n$\\ \ \\
  27) $- 5 v - 5 < - 5 (4 v + 1)$\\ \ \\
  28) $- 36 + 6 x > - 8 (x + 2) + 4 x$\\ \ \\
  29) $4 + 2 (a + 5) < - 2 (- a - 4)$\\ \ \\
  30) $3 (n + 3) + 7 (8 - 8 n) < 5 n + 5 + 2$\\ \ \\
  31) $- (k - 2) > - k - 20$\\ \ \\
  32) $- (4 - 5 p) + 3 \geq - 2 (8 - 5 p)$\\
\ \\
\end{multicols}
\newpage
\end{document}