\documentclass[12pt]{article}
\usepackage[top=1in,left=1in,bottom=1in,right=1in,headsep=2pt]{geometry}	
\usepackage{amssymb,amsmath,amsthm,amsfonts}
\usepackage{chapterfolder,docmute,setspace}
\usepackage{cancel,multicol,tikz,verbatim,framed,polynom,enumitem}
\usepackage[colorlinks, hyperindex, plainpages=false, linkcolor=blue, urlcolor=blue, pdfpagelabels]{hyperref}
\usepackage[metapost,truebbox]{mfpic}
% Use the cc-by-nc-sa license for any content linked with Stitz and Zeager's text.  Otherwise, use the cc-by-sa license.
%\usepackage[type={CC},modifier={by-sa},version={4.0},]{doclicense}
\usepackage[type={CC},modifier={by-nc-sa},version={4.0},]{doclicense}

\theoremstyle{definition}
\newtheorem{example}{Example}
\newcommand{\Desmos}{\href{https://www.desmos.com/}{Desmos}}
\setlength{\parindent}{0em}
\setlist{itemsep=0em}
\setlength{\parskip}{0.1em}
% This document is used for ordering of lessons.  If an instructor wishes to change the ordering of assessments, the following steps must be taken:

% 1) Reassign the appropriate numbers for each lesson in the \setcounter commands included in this file.
% 2) Rearrange the \include commands in the master file (the file with 'Course Pack' in the name) to accurately reflect the changes.  
% 3) Rearrange the \items in the measureable_outcomes file to accurately reflect the changes.  Be mindful of page breaks when moving items.
% 4) Re-build all affected files (master file, measureable_outcomes file, and any lesson whose numbering has changed).

%Note: The placement of each \newcounter and \setcounter command reflects the original/default ordering of topics (linears, systems, quadratics, functions, polynomials, rationals).

\newcounter{lesson_solving_linear_equations}
\newcounter{lesson_equations_containing_absolute_values}
\newcounter{lesson_graphing_lines}
\newcounter{lesson_two_forms_of_a_linear_equation}
\newcounter{lesson_parallel_and_perpendicular_lines}
\newcounter{lesson_linear_inequalities}
\newcounter{lesson_compound_inequalities}
\newcounter{lesson_inequalities_containing_absolute_values}
\newcounter{lesson_graphing_systems}
\newcounter{lesson_substitution}
\newcounter{lesson_elimination}
\newcounter{lesson_quadratics_introduction}
\newcounter{lesson_factoring_GCF}
\newcounter{lesson_factoring_grouping}
\newcounter{lesson_factoring_trinomials_a_is_1}
\newcounter{lesson_factoring_trinomials_a_neq_1}
\newcounter{lesson_solving_by_factoring}
\newcounter{lesson_square_roots}
\newcounter{lesson_i_and_complex_numbers}
\newcounter{lesson_vertex_form_and_graphing}
\newcounter{lesson_solve_by_square_roots}
\newcounter{lesson_extracting_square_roots}
\newcounter{lesson_the_discriminant}
\newcounter{lesson_the_quadratic_formula}
\newcounter{lesson_quadratic_inequalities}
\newcounter{lesson_functions_and_relations}
\newcounter{lesson_evaluating_functions}
\newcounter{lesson_finding_domain_and_range_graphically}
\newcounter{lesson_fundamental_functions}
\newcounter{lesson_finding_domain_algebraically}
\newcounter{lesson_solving_functions}
\newcounter{lesson_function_arithmetic}
\newcounter{lesson_composite_functions}
\newcounter{lesson_inverse_functions_definition_and_HLT}
\newcounter{lesson_finding_an_inverse_function}
\newcounter{lesson_transformations_translations}
\newcounter{lesson_transformations_reflections}
\newcounter{lesson_transformations_scalings}
\newcounter{lesson_transformations_summary}
\newcounter{lesson_piecewise_functions}
\newcounter{lesson_functions_containing_absolute_values}
\newcounter{lesson_absolute_as_piecewise}
\newcounter{lesson_polynomials_introduction}
\newcounter{lesson_sign_diagrams_polynomials}
\newcounter{lesson_factoring_quadratic_type}
\newcounter{lesson_factoring_summary}
\newcounter{lesson_polynomial_division}
\newcounter{lesson_synthetic_division}
\newcounter{lesson_end_behavior_polynomials}
\newcounter{lesson_local_behavior_polynomials}
\newcounter{lesson_rational_root_theorem}
\newcounter{lesson_polynomials_graphing_summary}
\newcounter{lesson_polynomial_inequalities}
\newcounter{lesson_rationals_introduction_and_terminology}
\newcounter{lesson_sign_diagrams_rationals}
\newcounter{lesson_horizontal_asymptotes}
\newcounter{lesson_slant_and_curvilinear_asymptotes}
\newcounter{lesson_vertical_asymptotes}
\newcounter{lesson_holes}
\newcounter{lesson_rationals_graphing_summary}

\setcounter{lesson_solving_linear_equations}{1}
\setcounter{lesson_equations_containing_absolute_values}{2}
\setcounter{lesson_graphing_lines}{3}
\setcounter{lesson_two_forms_of_a_linear_equation}{4}
\setcounter{lesson_parallel_and_perpendicular_lines}{5}
\setcounter{lesson_linear_inequalities}{6}
\setcounter{lesson_compound_inequalities}{7}
\setcounter{lesson_inequalities_containing_absolute_values}{8}
\setcounter{lesson_graphing_systems}{9}
\setcounter{lesson_substitution}{10}
\setcounter{lesson_elimination}{11}
\setcounter{lesson_quadratics_introduction}{16}
\setcounter{lesson_factoring_GCF}{17}
\setcounter{lesson_factoring_grouping}{18}
\setcounter{lesson_factoring_trinomials_a_is_1}{19}
\setcounter{lesson_factoring_trinomials_a_neq_1}{20}
\setcounter{lesson_solving_by_factoring}{21}
\setcounter{lesson_square_roots}{22}
\setcounter{lesson_i_and_complex_numbers}{23}
\setcounter{lesson_vertex_form_and_graphing}{24}
\setcounter{lesson_solve_by_square_roots}{25}
\setcounter{lesson_extracting_square_roots}{26}
\setcounter{lesson_the_discriminant}{27}
\setcounter{lesson_the_quadratic_formula}{28}
\setcounter{lesson_quadratic_inequalities}{29}
\setcounter{lesson_functions_and_relations}{12}
\setcounter{lesson_evaluating_functions}{13}
\setcounter{lesson_finding_domain_and_range_graphically}{14}
\setcounter{lesson_fundamental_functions}{15}
\setcounter{lesson_finding_domain_algebraically}{30}
\setcounter{lesson_solving_functions}{31}
\setcounter{lesson_function_arithmetic}{32}
\setcounter{lesson_composite_functions}{33}
\setcounter{lesson_inverse_functions_definition_and_HLT}{34}
\setcounter{lesson_finding_an_inverse_function}{35}
\setcounter{lesson_transformations_translations}{36}
\setcounter{lesson_transformations_reflections}{37}
\setcounter{lesson_transformations_scalings}{38}
\setcounter{lesson_transformations_summary}{39}
\setcounter{lesson_piecewise_functions}{40}
\setcounter{lesson_functions_containing_absolute_values}{41}
\setcounter{lesson_absolute_as_piecewise}{42}
\setcounter{lesson_polynomials_introduction}{43}
\setcounter{lesson_sign_diagrams_polynomials}{44}
\setcounter{lesson_factoring_quadratic_type}{46}
\setcounter{lesson_factoring_summary}{45}
\setcounter{lesson_polynomial_division}{47}
\setcounter{lesson_synthetic_division}{48}
\setcounter{lesson_end_behavior_polynomials}{49}
\setcounter{lesson_local_behavior_polynomials}{50}
\setcounter{lesson_rational_root_theorem}{51}
\setcounter{lesson_polynomials_graphing_summary}{52}
\setcounter{lesson_polynomial_inequalities}{53}
\setcounter{lesson_rationals_introduction_and_terminology}{54}
\setcounter{lesson_sign_diagrams_rationals}{55}
\setcounter{lesson_horizontal_asymptotes}{56}
\setcounter{lesson_slant_and_curvilinear_asymptotes}{57}
\setcounter{lesson_vertical_asymptotes}{58}
\setcounter{lesson_holes}{59}
\setcounter{lesson_rationals_graphing_summary}{60}

\begin{document}
{\bf \large Lesson \arabic{lesson_composite_functions}: Composite Functions}
%\\ CC attribute: \href{http://www.wallace.ccfaculty.org/book/book.html}{\it{Beginning and Intermediate Algebra}} by T. Wallace. 
\\ CC attribute: \href{http://www.stitz-zeager.com}{\it{College Algebra}} by C. Stitz and J. Zeager. 
\hfill \doclicenseImage[imagewidth=5em]\\
\par
{\bf Objective:} 	Construct, evaluate, and interpret composite functions.\\
\par
{\bf Students will be able to:}
\begin{itemize}
	\item Evaluate composite functions for at a specified value, $x=c$.
	\item Find and simplify composite functions in terms of a variable.
\end{itemize}
{\bf Prerequisite Knowledge:}
\begin{itemize}
	\item Order of operations.
	\item Evaluating Functions.
	\item Function Arithmetic.
\end{itemize}
\hrulefill

{\bf Lesson:}\\
\ \par
In addition to the four basic arithmetic operations ($+,-,~\cdot~,\div$), we will now discuss a fifth operation, known as a {\it composition} and denoted by $\circ$ (not to be confused with a product, $\cdot$). The result of a composition is called a {\it composite function} and is defined as follows.
\[(f \circ g) (x) = f (g (x))\]
The notation $(f\circ g)(x)$ above should always be interpreted as ``$f$ of $g$ of $x$''.  In this situation, we consider $g$ to be the {\it inner} function, since it is being substituted into $f$ for $x$.  Consequently, we refer to $f$ as the {\it outer} function.\\
\ \par
Similarly, if we reversed the order of the two functions $f$ and $g$, then the resulting composite function $(g\circ f)(x)=g(f(x))$ will have inner function $f$ and outer function $g$, and should be interpreted as ``$g$ of $f$ of $x$''.  As we will see, one should never assume that the two composite functions $f\circ g$ and $g\circ f$ will be equal.\\
\ \par
We will begin by evaluating a composite function at a single value.  This is accomplished by first evaluating the inner function at the specified value, and
then substituting (``plugging in'') the corresponding {\it output} into the outer function.\\
\ \par
We can also identify a composite function in terms of the variable. In our second example, we will substitute the inner function into the outer function for every instance of the variable and then simplify.  This approach is sometimes referred to as the ``inside-out'' approach.\\
\ \par
It is important to note that $(f \circ g) (x)$ usually will {\it not} equal $(g\circ f) (x)$, as our first Demo/Discussion problem will show.
\newpage
{\bf I - Motivating Example(s):}\\
\ \par
{\bf Example:} Find $(f\circ g)(3)$, where $f(x)=x^2-2x+1$ and $g(x)=x-5$.
  \begin{eqnarray*}
   (f \circ g) (3)=f (g (3)) &  & \text{Rewrite~} f\circ g \text{~as~inner~and~outer~functions}\\
	    &  & \\
 	g (3) = (3) - 5 = - 2~~~~~~~ &  & \text{Evaluate~inner~function~at~} x=3\\
		& & \text{Use~output~of~} -2 \text{~as~input~for~} f\\
    f (- 2) = (- 2)^2 - 2 (- 2) + 1 &  & \text{Evaluate~outer~function~at~} x=-2\\
    = 4 + 4 + 1~~~~~~~~~~~~ &  & \text{Simplify}\\
		(f \circ g) (3) = 9 &  & \text{Our solution}
  \end{eqnarray*}

{\bf Example:} Find $(f \circ g) (x)$, where $f (x) = x^2 - x$ and $g (x) = x + 3$.
  \begin{eqnarray*}
    (f \circ g) (x)=f (g (x)) &  & \text{Rewrite~} f\circ g \text{~as~inner~and~outer~functions}\\
	    &  & \text{Our~inner~function~is~} g(x) = x + 3\\
    f (x + 3) &  & \text{Replace~each~} x \text{~in~} f \text{~with~} (x + 3)\\
		  &  & \text{Make~sure~to~include~parentheses!}\\
    (x + 3)^2 - (x + 3) &  & \text{Simplify;~expand~binomial}\\
    (x^2 + 6 x + 9) - (x + 3) &  & \text{Distribute~negative}\\
    x^2 + 6 x + 9 - x - 3~~ &  & \text{Combine~like~terms}\\
	& & \\
		(f \circ g) (x)=x^2 + 5 x + 6~~~ &  & \text{Our solution}\\
		=(x+3)(x+2) & & \text{in~factored~form}
  \end{eqnarray*}
{\bf II - Demo/Discussion Problems:}\\
\ \par
Find each of the following.
\begin{enumerate}
	\item $(g\circ f)(x),$ where $f(x)=x^2-x$ and $g(x)=x+3$
	\item $(g\circ g)(x),$ where $g(x)=x^2-2x$
\end{enumerate}

In each problem below, use the given pair of functions to find the following six composite values if they exist.

\begin{multicols}{3}
\begin{itemize}
\item  $(g\circ f)(0)$
\item  $(f\circ g)(-1)$
\item  $(f \circ f)(2)$
\end{itemize}
\end{multicols}

\begin{multicols}{3}
\begin{itemize}
\item  $(g\circ f)(-3)$
\item  $(f\circ g)\left(\frac{1}{2}\right)$
\item  $(f \circ f)(-2)$
\end{itemize}
\end{multicols}

\begin{multicols}{2}
\begin{enumerate}
\item[3.]  $f(x) = 4-x$, \ \ $g(x) = 1-x^2$\\
\item[4.]  $f(x) = |x-1|$, \ \ $g(x) = x^2-5$\\
\item[5.]  $f(x) = 4x+5$, \ \ $g(x) = \sqrt{x}$\\
\item[6.]  $f(x) = \dfrac{3}{1-x}$, \ \ $g(x) = \dfrac{4x}{x^2+1}$\\
\end{enumerate}
\end{multicols}

\newpage

Use the given pair of functions to find and simplify expressions for the following three composite functions.  Then state the domain of each using interval notation.

\begin{multicols}{3}
\begin{itemize}
\item  $(g \circ f)(x)$
\item  $(f \circ g)(x)$
\item  $(f \circ f)(x)$
\end{itemize}
\end{multicols}


\begin{multicols}{2}
\begin{enumerate}
\item[7.]  $f(x) = 3x-5$, \ \ $g(x) = \sqrt{x}$\\ 
\item[8.]  $f(x) = 3-x^2$, \ \ $g(x) = \sqrt{x+1}$\\ 
\item[9.]  $f(x) = 3x-1$, \ \ $g(x) = \dfrac{1}{x+3}$\\
\item[10.]  $f(x) =  \dfrac{2x}{x^2-4}$, \ \ $g(x) =\sqrt{1-x}$\\
\end{enumerate}
\end{multicols}

Write each of the given functions as a composition of two or more non-identity functions.  (There are several correct answers, so check your answer using function composition.)

\begin{multicols}{2}
\begin{enumerate}
\item[11.]  $p(x) = (2x+3)^3$\\
\item[12.]  $H(x) = |7-3x|$\\
\item[13.]  $Q(x) = \dfrac{2x^3+1}{x^3-1}$\\
\item[14.]  $w(x) = \dfrac{x^2}{x^4+1}$\\
\end{enumerate}
\end{multicols}

{\bf III - Practice Problems:}\\
\ \par
In each problem below, use the given pair of functions to find the following six composite values if they exist.

\begin{multicols}{3}
\begin{itemize}
\item  $(g\circ f)(0)$
\item  $(f\circ g)(-1)$
\item  $(f \circ f)(2)$
\end{itemize}
\end{multicols}

\begin{multicols}{3}
\begin{itemize}
\item  $(g\circ f)(-3)$
\item  $(f\circ g)\left(\frac{1}{2}\right)$
\item  $(f \circ f)(-2)$
\end{itemize}
\end{multicols}

\begin{multicols}{2}
\begin{enumerate}
\item[1.]  $f(x) = x^2$, \ \ $g(x) = 2x+1$\\
\item[2.]  $f(x) = 4-3x$, \ \ $g(x) = |x|$\\
\item[3.]  $f(x) = \sqrt{3-x}$, \ \ $g(x) = x^2+1$\\
\item[4.]  $f(x) = \dfrac{x}{x+5}$, \ \ $g(x) = \dfrac{2}{7-x^2}$\\
\end{enumerate}
\end{multicols}

Use the given pair of functions to find and simplify expressions for the following three composite functions.  Then state the domain of each using interval notation.

\begin{multicols}{3}
\begin{itemize}
\item  $(g \circ f)(x)$
\item  $(f \circ g)(x)$
\item  $(f \circ f)(x)$
\end{itemize}
\end{multicols}


\begin{multicols}{2}
\begin{enumerate}
\item[5.]  $f(x) = 2x+3$, \ \ $g(x) = x^2-9$\\
\item[6.]  $f(x) = x^2 -x+1$, \ \ $g(x) = 3x-5$\\ 
\item[7.]  $f(x) = x^2-4$, \ \ $g(x) = |x|$\\
\item[8.]  $f(x) = |x+1|$, \ \ $g(x) = \sqrt{x}$\\
\item[9.]  $f(x) = |x|$, \ \ $g(x) = \sqrt{4-x}$\\
\item[10.]  \mbox{$f(x) = x^2-x-1$, \ \ $g(x) = \sqrt{x-5}$}\\ 
\item[11.]  $f(x) = \dfrac{3x}{x-1}$, \ \ $g(x) =\dfrac{x}{x-3}$\\
\item[12.]  $f(x) = \dfrac{x}{2x+1}$, \ \ $g(x) = \dfrac{2x+1}{x}$\\
\end{enumerate}
\end{multicols}

Write each of the given functions as a composition of two or more non-identity functions.  (There are several correct answers, so check your answer using function composition.)

\begin{multicols}{3}
\begin{enumerate}
\item[13.]  $P(x) = \left(x^2-x+1\right)^5$\\
\item[14.]  $h(x) = \sqrt{2x-1}$\\
\item[15.]  $r(x) = \dfrac{2}{5x+1}$\\
\item[16.]  $R(x) = \dfrac{7}{x^2-1}$\\
\item[17.]  $q(x) = \dfrac{|x|+1}{|x|-1}$\\
\item[18.]  $v(x) = \dfrac{2x+1}{3-4x}$\\
\end{enumerate}
\end{multicols}

\begin{enumerate}
\item[19.] Let $g(x) = -x, \, h(x) = x + 2, \, j(x) = 3x$ and $k(x) = x - 4$.  In what order must these functions be composed with $f(x) = \sqrt{x}$ to create $F(x) = 3\sqrt{-x + 2} - 4$?

\item[20.] What linear functions could be used to transform $f(x) = x^{3}$ into \mbox{$F(x) = -\frac{1}{2}(2x - 7)^{3} + 1$}?  What is the proper order of composition?
\end{enumerate}

Let $f$ be the function defined by \[f = \{(-3, 4), (-2, 2), (-1, 0), (0, 1), (1, 3), (2, 4), (3, -1)\}\] and let $g$ be the function defined \[g = \{(-3, -2), (-2, 0), (-1, -4), (0, 0), (1, -3), (2, 1), (3, 2)\}\].  Find each composite value if it exists.

\begin{multicols}{3}
\begin{enumerate}
\item[21.] $(f \circ g)(3)$\\
\item[22.] $f(g(-1))$\\
\item[23.] $(f \circ f)(0)$\\
\item[24.] $(f \circ g)(-3)$\\
\item[25.] $(g \circ f)(3)$\\
\item[26.] $g(f(-3))$\\
\item[27.] $(g \circ g)(-2)$\\
\item[28.] $(g \circ f)(-2)$\\
\item[29.] $g(f(g(0)))$\\
\end{enumerate}
\end{multicols}
\newpage
\end{document}