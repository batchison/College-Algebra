\documentclass[12pt]{article}
\usepackage[top=1in,left=1in,bottom=1in,right=1in,headsep=2pt]{geometry}	
\usepackage{amssymb,amsmath,amsthm,amsfonts}
\usepackage{chapterfolder,docmute,setspace}
\usepackage{cancel,multicol,tikz,verbatim,framed,polynom,enumitem}
\usepackage[colorlinks, hyperindex, plainpages=false, linkcolor=blue, urlcolor=blue, pdfpagelabels]{hyperref}
\usepackage[type={CC},modifier={by-sa},version={4.0},]{doclicense}

\theoremstyle{definition}
\newtheorem{example}{Example}
\newcommand{\Desmos}{\href{https://www.desmos.com/}{Desmos}}
\setlength{\parindent}{0em}
\setlist{itemsep=0em}
\setlength{\parskip}{0.1em}
% This document is used for ordering of lessons.  If an instructor wishes to change the ordering of assessments, the following steps must be taken:

% 1) Reassign the appropriate numbers for each lesson in the \setcounter commands included in this file.
% 2) Rearrange the \include commands in the master file (the file with 'Course Pack' in the name) to accurately reflect the changes.  
% 3) Rearrange the \items in the measureable_outcomes file to accurately reflect the changes.  Be mindful of page breaks when moving items.
% 4) Re-build all affected files (master file, measureable_outcomes file, and any lesson whose numbering has changed).

%Note: The placement of each \newcounter and \setcounter command reflects the original/default ordering of topics (linears, systems, quadratics, functions, polynomials, rationals).

\newcounter{lesson_solving_linear_equations}
\newcounter{lesson_equations_containing_absolute_values}
\newcounter{lesson_graphing_lines}
\newcounter{lesson_two_forms_of_a_linear_equation}
\newcounter{lesson_parallel_and_perpendicular_lines}
\newcounter{lesson_linear_inequalities}
\newcounter{lesson_compound_inequalities}
\newcounter{lesson_inequalities_containing_absolute_values}
\newcounter{lesson_graphing_systems}
\newcounter{lesson_substitution}
\newcounter{lesson_elimination}
\newcounter{lesson_quadratics_introduction}
\newcounter{lesson_factoring_GCF}
\newcounter{lesson_factoring_grouping}
\newcounter{lesson_factoring_trinomials_a_is_1}
\newcounter{lesson_factoring_trinomials_a_neq_1}
\newcounter{lesson_solving_by_factoring}
\newcounter{lesson_square_roots}
\newcounter{lesson_i_and_complex_numbers}
\newcounter{lesson_vertex_form_and_graphing}
\newcounter{lesson_solve_by_square_roots}
\newcounter{lesson_extracting_square_roots}
\newcounter{lesson_the_discriminant}
\newcounter{lesson_the_quadratic_formula}
\newcounter{lesson_quadratic_inequalities}
\newcounter{lesson_functions_and_relations}
\newcounter{lesson_evaluating_functions}
\newcounter{lesson_finding_domain_and_range_graphically}
\newcounter{lesson_fundamental_functions}
\newcounter{lesson_finding_domain_algebraically}
\newcounter{lesson_solving_functions}
\newcounter{lesson_function_arithmetic}
\newcounter{lesson_composite_functions}
\newcounter{lesson_inverse_functions_definition_and_HLT}
\newcounter{lesson_finding_an_inverse_function}
\newcounter{lesson_transformations_translations}
\newcounter{lesson_transformations_reflections}
\newcounter{lesson_transformations_scalings}
\newcounter{lesson_transformations_summary}
\newcounter{lesson_piecewise_functions}
\newcounter{lesson_functions_containing_absolute_values}
\newcounter{lesson_absolute_as_piecewise}
\newcounter{lesson_polynomials_introduction}
\newcounter{lesson_sign_diagrams_polynomials}
\newcounter{lesson_factoring_quadratic_type}
\newcounter{lesson_factoring_summary}
\newcounter{lesson_polynomial_division}
\newcounter{lesson_synthetic_division}
\newcounter{lesson_end_behavior_polynomials}
\newcounter{lesson_local_behavior_polynomials}
\newcounter{lesson_rational_root_theorem}
\newcounter{lesson_polynomials_graphing_summary}
\newcounter{lesson_polynomial_inequalities}
\newcounter{lesson_rationals_introduction_and_terminology}
\newcounter{lesson_sign_diagrams_rationals}
\newcounter{lesson_horizontal_asymptotes}
\newcounter{lesson_slant_and_curvilinear_asymptotes}
\newcounter{lesson_vertical_asymptotes}
\newcounter{lesson_holes}
\newcounter{lesson_rationals_graphing_summary}

\setcounter{lesson_solving_linear_equations}{1}
\setcounter{lesson_equations_containing_absolute_values}{2}
\setcounter{lesson_graphing_lines}{3}
\setcounter{lesson_two_forms_of_a_linear_equation}{4}
\setcounter{lesson_parallel_and_perpendicular_lines}{5}
\setcounter{lesson_linear_inequalities}{6}
\setcounter{lesson_compound_inequalities}{7}
\setcounter{lesson_inequalities_containing_absolute_values}{8}
\setcounter{lesson_graphing_systems}{9}
\setcounter{lesson_substitution}{10}
\setcounter{lesson_elimination}{11}
\setcounter{lesson_quadratics_introduction}{16}
\setcounter{lesson_factoring_GCF}{17}
\setcounter{lesson_factoring_grouping}{18}
\setcounter{lesson_factoring_trinomials_a_is_1}{19}
\setcounter{lesson_factoring_trinomials_a_neq_1}{20}
\setcounter{lesson_solving_by_factoring}{21}
\setcounter{lesson_square_roots}{22}
\setcounter{lesson_i_and_complex_numbers}{23}
\setcounter{lesson_vertex_form_and_graphing}{24}
\setcounter{lesson_solve_by_square_roots}{25}
\setcounter{lesson_extracting_square_roots}{26}
\setcounter{lesson_the_discriminant}{27}
\setcounter{lesson_the_quadratic_formula}{28}
\setcounter{lesson_quadratic_inequalities}{29}
\setcounter{lesson_functions_and_relations}{12}
\setcounter{lesson_evaluating_functions}{13}
\setcounter{lesson_finding_domain_and_range_graphically}{14}
\setcounter{lesson_fundamental_functions}{15}
\setcounter{lesson_finding_domain_algebraically}{30}
\setcounter{lesson_solving_functions}{31}
\setcounter{lesson_function_arithmetic}{32}
\setcounter{lesson_composite_functions}{33}
\setcounter{lesson_inverse_functions_definition_and_HLT}{34}
\setcounter{lesson_finding_an_inverse_function}{35}
\setcounter{lesson_transformations_translations}{36}
\setcounter{lesson_transformations_reflections}{37}
\setcounter{lesson_transformations_scalings}{38}
\setcounter{lesson_transformations_summary}{39}
\setcounter{lesson_piecewise_functions}{40}
\setcounter{lesson_functions_containing_absolute_values}{41}
\setcounter{lesson_absolute_as_piecewise}{42}
\setcounter{lesson_polynomials_introduction}{43}
\setcounter{lesson_sign_diagrams_polynomials}{44}
\setcounter{lesson_factoring_quadratic_type}{46}
\setcounter{lesson_factoring_summary}{45}
\setcounter{lesson_polynomial_division}{47}
\setcounter{lesson_synthetic_division}{48}
\setcounter{lesson_end_behavior_polynomials}{49}
\setcounter{lesson_local_behavior_polynomials}{50}
\setcounter{lesson_rational_root_theorem}{51}
\setcounter{lesson_polynomials_graphing_summary}{52}
\setcounter{lesson_polynomial_inequalities}{53}
\setcounter{lesson_rationals_introduction_and_terminology}{54}
\setcounter{lesson_sign_diagrams_rationals}{55}
\setcounter{lesson_horizontal_asymptotes}{56}
\setcounter{lesson_slant_and_curvilinear_asymptotes}{57}
\setcounter{lesson_vertical_asymptotes}{58}
\setcounter{lesson_holes}{59}
\setcounter{lesson_rationals_graphing_summary}{60}

\begin{document}
{\bf \large Lesson \arabic{lesson_elimination}: Solving Systems of Equations by Addition/Elimination}\\
CC attribute: \href{http://www.wallace.ccfaculty.org/book/book.html}{\it{Beginning and Intermediate Algebra}} by T. Wallace. \hfill \doclicenseImage[imagewidth=5em]\\
\par
{\bf Objective:} Solve linear systems by addition and elimination.\\
\par
{\bf Students will be able to:}
\begin{itemize}
	\item Solve linear systems by addition and elimination of one variable.
	\item Write system solutions as ordered pairs in the form $(x,y)$.
	\item Verify the accuracy of a solution by plugging it into each equation in the system.
\end{itemize}
{\bf Prerequisite Knowledge:}
\begin{itemize}
	\item Solving a linear equation.
	\item Applying the distributive property.
	\item Combining like terms.
\end{itemize}
\hrulefill

{\bf Lesson:}
\par
{\bf I - Motivating Example(s):}\\
\ \par
We present the steps for solving a system of linear equations by addition/elimination alongside an example.
\begin{center}
\begin{tabular}{|l|l|}
  \hline
	& \\
  \ \ \ \ Steps for Addition/Elmination & \begin{tabular}{l}
    System: $\begin{cases}
		2 x - 5 y = -13\\
    -3y+4 = - 5x\end{cases}$
	\end{tabular}\\
  & \\
  \hline
  1. Line up the variables and constants. & \begin{tabular}{l}
    Rearrange the second equation\\
    $2 x - 5 y = - 13$\\
    $5 x - 3 y = - 4$\\
	\end{tabular}\\
  \hline
  2. Multiply to get opposites (use LCM). & $\begin{array}{l}
 
    \text{First Equation: multiply by} \ - 5\\
    - 5\cdot(2 x - 5 y) = (- 13)\cdot(- 5)\\
    ~- 10 x + 25 y = 65\\
 \\
   \text{Second Equation: multiply by} \ 2\\
    2\cdot (5 x - 3 y) = (- 4)\cdot 2\\
    ~~10 x - 6 y = - 8\\
  \end{array}$\\
  \hline
  3. Add equations to eliminate a variable.&  
	
	  $\begin{array}{l}
 ~~~~~- 10 x + 25 y = 65~~~~~~\\
 ~~~~~~\underline{~10 x~ -~ 6 y = - 8~~}\\
~~~~~~~~~~~~~~~~19 y =57\\
\end{array}$\\
\hline
\end{tabular}
\begin{tabular}{|l|l|}
  \hline
	4. Solve for the remaining variable. & $\begin{array}{l}
     ~~~~~~~~~~~~~~19y = 57\\
    ~~~~~~~~~~~~~~~\overline{19} ~~~~ \overline{19}\\
    ~~~~~~~~~~~~~~~~~y = 3
  \end{array}$\\
  \hline
  5.~$\begin{array}{l}
	\text{Plug back into either of the given equations}\\
	\text{and solve.}
	\end{array}$
	
& \begin{tabular}{l}
    ~~~~~~$2 x - 5 (3) = - 13$\\
    ~~~~~~~~$2 x - 15 = - 13$\\
    ~~~~~~~~~~~~~$\underline{+ 15 ~+ 15}$\\
		~~~~~~~~~~~~~~~$2 x = 2$\\
    ~~~~~~~~~~~~~~~~$\overline{2}$ ~~ $\overline{2}$\\
    ~~~~~~~~~~~~~~~~$x = 1$
  \end{tabular}\\
  \hline
  \ \ \ Our solution, as a coordinate pair. & ~~~~~~~~~~~~$(x,y)=(1, 3)$ \ \ \ \ \ \ \ \\
  \hline
\end{tabular}
\end{center}

{\bf II - Demo/Discussion Problems:}\\
\ \par
Solve each of the following systems of linear equations by addition/elimination.
\begin{multicols}{3}
	\begin{enumerate}
		\item $\begin{cases}
					3x+6y=-9\\
			    2x+9y=-26
					\end{cases}$
		\item $\begin{cases}
					2x-5y=3\\
					-6x+15y=-9
					\end{cases}$
		\item $\begin{cases}
					4 x - 6 y = 8\\
					6 x - 9 y = 15
					\end{cases}$
	\end{enumerate}
\end{multicols}
\ \par
{\bf III - Practice Problems:}\\
\ \par
Solve each of the following systems of linear equations by addition/elimination.
\begin{multicols}{3}
	\begin{enumerate}
  \item $\begin{cases}
	4 x + 2 y = 0\\
  - 4 x - 9 y = - 28
  \end{cases}$
  \item $\begin{cases}
	- 6 x + 9 y = 3\\
  6 x - 9 y = - 9
  \end{cases}$ 
  \item $\begin{cases}
  - x - 5 y = 28\\
  - x + 4 y = - 17
  \end{cases}$
  \item $\begin{cases}
  10 x + 6 y = 24\\
  - 6 x + y = 4
  \end{cases}$
  \item $\begin{cases}
  - 7 x + 4 y = - 4\\
  10 x - 8 y = - 8
  \end{cases}$
  \item $\begin{cases}
  - 7 x - 3 y = 12\\
  - 6 x - 5 y = 20
  \end{cases}$
  \item $\begin{cases}
  9 x + 6 y = - 21\\
  - 10 x - 9 y = 28
  \end{cases}$
  \item $\begin{cases}
  - 8 x - 8 y = - 8\\
  10 x + 9 y = 1
  \end{cases}$
  \item $\begin{cases}
  0 = 9 x + 5 y\\
  y = \frac{2}{7} x
	\end{cases}$
  \item $\begin{cases}
  - 7 x + y = - 10\\
  - 9 x - y = - 22
  \end{cases}$
  \item $\begin{cases}
  5 x - 5 y = - 15\\
  5 x - 5 y = - 15
  \end{cases}$
  \item $\begin{cases}
  - 10 x - 5 y = 0\\
  10 x +10 y=30
  \end{cases}$
  \item $\begin{cases}
  x + 3 y = - 1\\
  10 x + 6 y = - 10
  \end{cases}$
  \item $\begin{cases}
  - 6 x + 4 y = 4\\
  - 3 x - y = 26
  \end{cases}$
  \item $\begin{cases}
  - 5 x + 4 y = 4\\
  - 7 x - 10 y = - 10
  \end{cases}$
  \item $\begin{cases}
  - 4 x - 5 y = 12\\
  - 10 x + 6 y = 30
  \end{cases}$
  \item $\begin{cases}
  - 7 x + 10 y = 13\\
  4 x + 9 y = 22
  \end{cases}$
  \item $\begin{cases}
  - 6 - 42 y = - 12 x\\
	x - \frac{7}{2} y = \frac{1}{2} 
  \end{cases}$
  \item $\begin{cases}
  - 9 x + 5 y = - 22\\
  9 x - 5 y = 13
  \end{cases}$
  \item $\begin{cases}
  4 x - 6 y = - 10\\
  4 x - 6 y = - 14
  \end{cases}$
  \item $\begin{cases}
  2 x - y = 5\\
  5 x + 2 y = - 28
  \end{cases}$
  \item $\begin{cases}
  2 x + 4 y = 24\\
  4 x - 12 y = 8
  \end{cases}$
  \item $\begin{cases}
  5 x + 10 y = 20\\
  - 6 x - 5 y = - 3
  \end{cases}$
  \item $\begin{cases}
  9 x - 2 y = - 18\\
  5 x - 7 y = - 10
  \end{cases}$
  \item $\begin{cases}
  - 7 x + 5 y = - 8\\
  - 3 x - 3 y = 12
  \end{cases}$
  \item $\begin{cases}
  9 y = 7 - x\\
  - 18 y + 4 x = - 26
	\end{cases}$
  \item $\begin{cases}
  - x - 2 y = - 7\\
  x + 2 y = 7
  \end{cases}$
  \item $\begin{cases}
  - 3 x + 3 y = - 12\\
  - 3 x + 9 y = - 24
  \end{cases}$
  \item $\begin{cases}
  - 5 x + 6 y = - 17\\
  x - 2 y = 5
  \end{cases}$
  \item $\begin{cases}
  - 6 x + 4 y = 12\\
  12 x + 6 y = 18
  \end{cases}$
  \item $\begin{cases}
  - 9 x - 5 y = - 19\\
  3 x - 7 y = - 11
  \end{cases}$
  \item $\begin{cases}
  3 x + 7 y = - 8\\
  4 x + 6 y = - 4
  \end{cases}$
  \item $\begin{cases}
  8 x + 7 y = - 24\\
  6 x + 3 y = - 18
  \end{cases}$
  \item $\begin{cases}
  21 = - 9 x + 12 y\\
	\frac{4}{3} y + \frac{7}{3} x=-1
  \end{cases}$
	\end{enumerate}
\end{multicols}
\newpage
\ \newpage
\end{document}