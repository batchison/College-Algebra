\documentclass[12pt]{article}
\usepackage[top=1in,left=1in,bottom=1in,right=1in,headsep=2pt]{geometry}	
\usepackage{amssymb,amsmath,amsthm,amsfonts}
\usepackage{chapterfolder,docmute,setspace}
\usepackage{cancel,multicol,tikz,verbatim,framed,polynom,enumitem}
\usepackage[colorlinks, hyperindex, plainpages=false, linkcolor=blue, urlcolor=blue, pdfpagelabels]{hyperref}
\usepackage[type={CC},modifier={by-sa},version={4.0},]{doclicense}

\theoremstyle{definition}
\newtheorem{example}{Example}
\newcommand{\Desmos}{\href{https://www.desmos.com/}{Desmos}}
\setlength{\parindent}{0em}
\setlist{itemsep=0em}
\setlength{\parskip}{0.1em}
% This document is used for ordering of lessons.  If an instructor wishes to change the ordering of assessments, the following steps must be taken:

% 1) Reassign the appropriate numbers for each lesson in the \setcounter commands included in this file.
% 2) Rearrange the \include commands in the master file (the file with 'Course Pack' in the name) to accurately reflect the changes.  
% 3) Rearrange the \items in the measureable_outcomes file to accurately reflect the changes.  Be mindful of page breaks when moving items.
% 4) Re-build all affected files (master file, measureable_outcomes file, and any lesson whose numbering has changed).

%Note: The placement of each \newcounter and \setcounter command reflects the original/default ordering of topics (linears, systems, quadratics, functions, polynomials, rationals).

\newcounter{lesson_solving_linear_equations}
\newcounter{lesson_equations_containing_absolute_values}
\newcounter{lesson_graphing_lines}
\newcounter{lesson_two_forms_of_a_linear_equation}
\newcounter{lesson_parallel_and_perpendicular_lines}
\newcounter{lesson_linear_inequalities}
\newcounter{lesson_compound_inequalities}
\newcounter{lesson_inequalities_containing_absolute_values}
\newcounter{lesson_graphing_systems}
\newcounter{lesson_substitution}
\newcounter{lesson_elimination}
\newcounter{lesson_quadratics_introduction}
\newcounter{lesson_factoring_GCF}
\newcounter{lesson_factoring_grouping}
\newcounter{lesson_factoring_trinomials_a_is_1}
\newcounter{lesson_factoring_trinomials_a_neq_1}
\newcounter{lesson_solving_by_factoring}
\newcounter{lesson_square_roots}
\newcounter{lesson_i_and_complex_numbers}
\newcounter{lesson_vertex_form_and_graphing}
\newcounter{lesson_solve_by_square_roots}
\newcounter{lesson_extracting_square_roots}
\newcounter{lesson_the_discriminant}
\newcounter{lesson_the_quadratic_formula}
\newcounter{lesson_quadratic_inequalities}
\newcounter{lesson_functions_and_relations}
\newcounter{lesson_evaluating_functions}
\newcounter{lesson_finding_domain_and_range_graphically}
\newcounter{lesson_fundamental_functions}
\newcounter{lesson_finding_domain_algebraically}
\newcounter{lesson_solving_functions}
\newcounter{lesson_function_arithmetic}
\newcounter{lesson_composite_functions}
\newcounter{lesson_inverse_functions_definition_and_HLT}
\newcounter{lesson_finding_an_inverse_function}
\newcounter{lesson_transformations_translations}
\newcounter{lesson_transformations_reflections}
\newcounter{lesson_transformations_scalings}
\newcounter{lesson_transformations_summary}
\newcounter{lesson_piecewise_functions}
\newcounter{lesson_functions_containing_absolute_values}
\newcounter{lesson_absolute_as_piecewise}
\newcounter{lesson_polynomials_introduction}
\newcounter{lesson_sign_diagrams_polynomials}
\newcounter{lesson_factoring_quadratic_type}
\newcounter{lesson_factoring_summary}
\newcounter{lesson_polynomial_division}
\newcounter{lesson_synthetic_division}
\newcounter{lesson_end_behavior_polynomials}
\newcounter{lesson_local_behavior_polynomials}
\newcounter{lesson_rational_root_theorem}
\newcounter{lesson_polynomials_graphing_summary}
\newcounter{lesson_polynomial_inequalities}
\newcounter{lesson_rationals_introduction_and_terminology}
\newcounter{lesson_sign_diagrams_rationals}
\newcounter{lesson_horizontal_asymptotes}
\newcounter{lesson_slant_and_curvilinear_asymptotes}
\newcounter{lesson_vertical_asymptotes}
\newcounter{lesson_holes}
\newcounter{lesson_rationals_graphing_summary}

\setcounter{lesson_solving_linear_equations}{1}
\setcounter{lesson_equations_containing_absolute_values}{2}
\setcounter{lesson_graphing_lines}{3}
\setcounter{lesson_two_forms_of_a_linear_equation}{4}
\setcounter{lesson_parallel_and_perpendicular_lines}{5}
\setcounter{lesson_linear_inequalities}{6}
\setcounter{lesson_compound_inequalities}{7}
\setcounter{lesson_inequalities_containing_absolute_values}{8}
\setcounter{lesson_graphing_systems}{9}
\setcounter{lesson_substitution}{10}
\setcounter{lesson_elimination}{11}
\setcounter{lesson_quadratics_introduction}{16}
\setcounter{lesson_factoring_GCF}{17}
\setcounter{lesson_factoring_grouping}{18}
\setcounter{lesson_factoring_trinomials_a_is_1}{19}
\setcounter{lesson_factoring_trinomials_a_neq_1}{20}
\setcounter{lesson_solving_by_factoring}{21}
\setcounter{lesson_square_roots}{22}
\setcounter{lesson_i_and_complex_numbers}{23}
\setcounter{lesson_vertex_form_and_graphing}{24}
\setcounter{lesson_solve_by_square_roots}{25}
\setcounter{lesson_extracting_square_roots}{26}
\setcounter{lesson_the_discriminant}{27}
\setcounter{lesson_the_quadratic_formula}{28}
\setcounter{lesson_quadratic_inequalities}{29}
\setcounter{lesson_functions_and_relations}{12}
\setcounter{lesson_evaluating_functions}{13}
\setcounter{lesson_finding_domain_and_range_graphically}{14}
\setcounter{lesson_fundamental_functions}{15}
\setcounter{lesson_finding_domain_algebraically}{30}
\setcounter{lesson_solving_functions}{31}
\setcounter{lesson_function_arithmetic}{32}
\setcounter{lesson_composite_functions}{33}
\setcounter{lesson_inverse_functions_definition_and_HLT}{34}
\setcounter{lesson_finding_an_inverse_function}{35}
\setcounter{lesson_transformations_translations}{36}
\setcounter{lesson_transformations_reflections}{37}
\setcounter{lesson_transformations_scalings}{38}
\setcounter{lesson_transformations_summary}{39}
\setcounter{lesson_piecewise_functions}{40}
\setcounter{lesson_functions_containing_absolute_values}{41}
\setcounter{lesson_absolute_as_piecewise}{42}
\setcounter{lesson_polynomials_introduction}{43}
\setcounter{lesson_sign_diagrams_polynomials}{44}
\setcounter{lesson_factoring_quadratic_type}{46}
\setcounter{lesson_factoring_summary}{45}
\setcounter{lesson_polynomial_division}{47}
\setcounter{lesson_synthetic_division}{48}
\setcounter{lesson_end_behavior_polynomials}{49}
\setcounter{lesson_local_behavior_polynomials}{50}
\setcounter{lesson_rational_root_theorem}{51}
\setcounter{lesson_polynomials_graphing_summary}{52}
\setcounter{lesson_polynomial_inequalities}{53}
\setcounter{lesson_rationals_introduction_and_terminology}{54}
\setcounter{lesson_sign_diagrams_rationals}{55}
\setcounter{lesson_horizontal_asymptotes}{56}
\setcounter{lesson_slant_and_curvilinear_asymptotes}{57}
\setcounter{lesson_vertical_asymptotes}{58}
\setcounter{lesson_holes}{59}
\setcounter{lesson_rationals_graphing_summary}{60}

\begin{document}
{\bf \large Lesson \arabic{lesson_two_forms_of_a_linear_equation}: Two Forms of a Linear Equation}\\
CC attribute: \href{http://www.wallace.ccfaculty.org/book/book.html}{\it{Beginning and Intermediate Algebra}} by T. Wallace. \hfill \doclicenseImage[imagewidth=5em]\\
\par
{\bf Objective:} Write the equation of a line in slope-intercept and point-slope form.\\
\par
{\bf Students will be able to:}
\begin{itemize}
	\item Find the slope of a line having certain characteristics.
	\item Identify a $y-$intercept.
	\item Convert between point-slope and slope-intercept forms.
\end{itemize}
{\bf Prerequisite Knowledge:}
\begin{itemize}
	\item Definitions of slope of a line and $y-$intercept.
	\item Graphing points on a coordinate plane .
	\item Point-testing.
	\item Multiplying and dividing fractions.
\end{itemize}
\hrulefill

{\bf Lesson:}
\par
The two forms for a linear equation are:
\begin{eqnarray*}
\text{slope-intercept form:} & & y=mx+b\\
\text{point-slope form:} & & y-y_1=m(x-x_1)
\end{eqnarray*}
{\bf I - Motivating Example(s):}\\
\par
Find the equation of the line through the points $(- 3, 4)$ and $(- 1,
  - 2)$.  Express your answer in slope-intercept form.
  \begin{eqnarray*}
    m = \frac{y_2 - y_1}{x_2 - x_1} &  & \text{Use the given points to find the slope.}\\
    m = \frac{- 2 - 4}{- 1 - (- 3)} = \frac{- 6}{2} = - 3 &  & \text{Substitute the} \ x- \ \text{and} \ y-\text{coordinates and simplify.}
	\end{eqnarray*}
	\begin{eqnarray*}
    y - y_1 = m (x - x_1) &  & \text{Use the point-slope form.  Substitute} \ m \ \text{and either point.}\\
    y - 4 = - 3 (x - (- 3)) &  & \text{Simplify to obtain slope-intercept form.  Distribute} \ m.\\
    y - 4 = - 3 x - 9 &  & \text{Solve for} \ y. \\
    y = - 3 x - 5 &  & \text{Our solution, in slope-intercept form.}
  \end{eqnarray*}
\newpage
{\bf II - Demo/Discussion Problems:}\\
\par
For each problem, find the equation of a line having the given characteristics.  In each case, find both the point-slope and slope-intercept forms.
\begin{enumerate}
	\item A line through the point $(-6,2)$ and having a slope of $-\frac{2}{3}$.
	\item A line through the points $(-2,5)$ and $(4,-3)$.
\end{enumerate}

{\bf III - Practice Problems:}\\
\par
Find the point-slope form of the line through the given point with the given slope.

\begin{multicols}{2}
  1) through $(2, 3)$, slope is undefined\\
  2) through $(1, 2)$, slope is undefined\\
  3) through $(2, 2)$, slope is $\frac{1}{2}$\\
  4) through $(2, 1)$, slope is $- \frac{1}{2}$\\
  5) through $(- 1, - 5)$, slope is $9$\\
  6) through $(2, - 2)$, slope is $- 2$\\
  7) through $(- 4, 1)$, slope is $\frac{3}{4}$\\
  8) through $(4, - 3)$, slope is $- 2$\\
  9) through $(0, - 2)$, slope is $- 3$\\
  10) through $(- 1, 1)$, slope is $4$\\
  11) through $(0, - 5)$, slope is $- \frac{1}{4}$\\
  12) through $(0, 2)$, slope is $- \frac{5}{4}$\\
  13) through $(- 5, - 3)$, slope is $\frac{1}{5}$\\
  14) through $(- 1, - 4)$, slope is $- \frac{2}{3}$\\
  15) through $(- 1, 4)$, slope is $- \frac{5}{4}$\\
  16) through $(1, - 4)$, slope is $- \frac{3}{2}$
\end{multicols}

Find the slope-intercept form of the line through the given point with the given slope.

\begin{multicols}{2}
  17) through $(- 1, - 5)$, slope is $2$\\
  18) through $(2, - 2)$, slope is $- 2$\\
  19) through $(5, - 1)$, slope is $- \frac{3}{5}$\\
  20) through $(- 2, - 2)$, slope is $- \frac{2}{3}$\\
  21) through $(- 4, 1)$, slope is $\frac{1}{2}$\\
  22) through $(4, - 3)$, slope is $- \frac{7}{4}$\\
  23) through $(4, - 2)$, slope is $- \frac{3}{2}$\\
  24) through $(- 2, 0)$, slope is $- \frac{5}{2}$\\
  25) through $(- 5, - 3)$, slope is $- \frac{2}{5}$\\
  26) through $(3, 3)$, slope is $\frac{7}{3}$\\
  27) through $(2, - 2)$, slope is $1$\\
  28) through $(- 4, - 3)$, slope is $0$\\
  29) through$(- 3, 4)$, slope is undefined\\
  30) through $(- 2, - 5)$, slope is $2$\\
  31) through $(- 4, 2)$, slope is $- \frac{1}{2}$\\
  32) through $(5, 3)$, slope is $\frac{6}{5}$
\end{multicols}

Find the point-slope form of the line through the given points.

\begin{multicols}{2}
  33) through $(- 4, 3)$ and $(- 3, 1)$\\
  34) through $(1, 3)$ and $(- 3, 3)$\\
  35) through $(5, 1)$ and $(- 3, 0)$\\
  36) through $(- 4, 5)$ and $(4, 4)$\\
  37) through $(- 4, - 2)$ and $(0, 4)$\\
  38) through $(- 4, 1)$ and $(4, 4)$\\
  39) through $(3, 5)$ and $(- 5, 3)$\\
  40) through $(- 1, - 4)$ and $(- 5, 0)$\\
  41) through $(3, - 3)$ and $(- 4, 5)$\\
  42) through $(- 1, - 5)$ and $(- 5, - 4)$
\end{multicols}

Find the slope-intercept form of the line through the given points.

\begin{multicols}{2}
  43) through $(- 5, 1)$ and $(- 1, - 2)$\\
  44) through $(- 5, - 1)$ and $(5, - 2)$\\
  45) through $(- 5, 5)$ and $(2, - 3)$\\
  46) through $(1, - 1)$ and $(- 5, - 4)$\\
  47) through $(4, 1)$ and $(1, 4)$\\
  48) through $(0, 1)$ and $(- 3, 0)$\\
  49) through $(0, 2)$ and $(5, - 3)$\\
  50) through $(0, 2)$ and $(2, 4)$\\
  51) through $(0, 3)$ and $(- 1, - 1)$\\
  52) through $(- 2, 0)$ and $(5, 3)$
\end{multicols}
\newpage
\ \newpage
\end{document}