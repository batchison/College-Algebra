\documentclass[12pt]{article}
\usepackage[top=1in,left=1in,bottom=1in,right=1in,headsep=2pt]{geometry}	
\usepackage{amssymb,amsmath,amsthm,amsfonts}
\usepackage{chapterfolder,docmute,setspace}
\usepackage{cancel,multicol,tikz,verbatim,framed,polynom,enumitem}
\usepackage[colorlinks, hyperindex, plainpages=false, linkcolor=blue, urlcolor=blue, pdfpagelabels]{hyperref}
% Use the cc-by-nc-sa license for any content linked with Stitz and Zeager's text.  Otherwise, use the cc-by-sa license.
\usepackage[type={CC},modifier={by-sa},version={4.0},]{doclicense}
%\usepackage[type={CC},modifier={by-nc-sa},version={4.0},]{doclicense}

\theoremstyle{definition}
\newtheorem{example}{Example}
\newcommand{\Desmos}{\href{https://www.desmos.com/}{Desmos}}
\setlength{\parindent}{0em}
\setlist{itemsep=0em}
\setlength{\parskip}{0.1em}
% This document is used for ordering of lessons.  If an instructor wishes to change the ordering of assessments, the following steps must be taken:

% 1) Reassign the appropriate numbers for each lesson in the \setcounter commands included in this file.
% 2) Rearrange the \include commands in the master file (the file with 'Course Pack' in the name) to accurately reflect the changes.  
% 3) Rearrange the \items in the measureable_outcomes file to accurately reflect the changes.  Be mindful of page breaks when moving items.
% 4) Re-build all affected files (master file, measureable_outcomes file, and any lesson whose numbering has changed).

%Note: The placement of each \newcounter and \setcounter command reflects the original/default ordering of topics (linears, systems, quadratics, functions, polynomials, rationals).

\newcounter{lesson_solving_linear_equations}
\newcounter{lesson_equations_containing_absolute_values}
\newcounter{lesson_graphing_lines}
\newcounter{lesson_two_forms_of_a_linear_equation}
\newcounter{lesson_parallel_and_perpendicular_lines}
\newcounter{lesson_linear_inequalities}
\newcounter{lesson_compound_inequalities}
\newcounter{lesson_inequalities_containing_absolute_values}
\newcounter{lesson_graphing_systems}
\newcounter{lesson_substitution}
\newcounter{lesson_elimination}
\newcounter{lesson_quadratics_introduction}
\newcounter{lesson_factoring_GCF}
\newcounter{lesson_factoring_grouping}
\newcounter{lesson_factoring_trinomials_a_is_1}
\newcounter{lesson_factoring_trinomials_a_neq_1}
\newcounter{lesson_solving_by_factoring}
\newcounter{lesson_square_roots}
\newcounter{lesson_i_and_complex_numbers}
\newcounter{lesson_vertex_form_and_graphing}
\newcounter{lesson_solve_by_square_roots}
\newcounter{lesson_extracting_square_roots}
\newcounter{lesson_the_discriminant}
\newcounter{lesson_the_quadratic_formula}
\newcounter{lesson_quadratic_inequalities}
\newcounter{lesson_functions_and_relations}
\newcounter{lesson_evaluating_functions}
\newcounter{lesson_finding_domain_and_range_graphically}
\newcounter{lesson_fundamental_functions}
\newcounter{lesson_finding_domain_algebraically}
\newcounter{lesson_solving_functions}
\newcounter{lesson_function_arithmetic}
\newcounter{lesson_composite_functions}
\newcounter{lesson_inverse_functions_definition_and_HLT}
\newcounter{lesson_finding_an_inverse_function}
\newcounter{lesson_transformations_translations}
\newcounter{lesson_transformations_reflections}
\newcounter{lesson_transformations_scalings}
\newcounter{lesson_transformations_summary}
\newcounter{lesson_piecewise_functions}
\newcounter{lesson_functions_containing_absolute_values}
\newcounter{lesson_absolute_as_piecewise}
\newcounter{lesson_polynomials_introduction}
\newcounter{lesson_sign_diagrams_polynomials}
\newcounter{lesson_factoring_quadratic_type}
\newcounter{lesson_factoring_summary}
\newcounter{lesson_polynomial_division}
\newcounter{lesson_synthetic_division}
\newcounter{lesson_end_behavior_polynomials}
\newcounter{lesson_local_behavior_polynomials}
\newcounter{lesson_rational_root_theorem}
\newcounter{lesson_polynomials_graphing_summary}
\newcounter{lesson_polynomial_inequalities}
\newcounter{lesson_rationals_introduction_and_terminology}
\newcounter{lesson_sign_diagrams_rationals}
\newcounter{lesson_horizontal_asymptotes}
\newcounter{lesson_slant_and_curvilinear_asymptotes}
\newcounter{lesson_vertical_asymptotes}
\newcounter{lesson_holes}
\newcounter{lesson_rationals_graphing_summary}

\setcounter{lesson_solving_linear_equations}{1}
\setcounter{lesson_equations_containing_absolute_values}{2}
\setcounter{lesson_graphing_lines}{3}
\setcounter{lesson_two_forms_of_a_linear_equation}{4}
\setcounter{lesson_parallel_and_perpendicular_lines}{5}
\setcounter{lesson_linear_inequalities}{6}
\setcounter{lesson_compound_inequalities}{7}
\setcounter{lesson_inequalities_containing_absolute_values}{8}
\setcounter{lesson_graphing_systems}{9}
\setcounter{lesson_substitution}{10}
\setcounter{lesson_elimination}{11}
\setcounter{lesson_quadratics_introduction}{16}
\setcounter{lesson_factoring_GCF}{17}
\setcounter{lesson_factoring_grouping}{18}
\setcounter{lesson_factoring_trinomials_a_is_1}{19}
\setcounter{lesson_factoring_trinomials_a_neq_1}{20}
\setcounter{lesson_solving_by_factoring}{21}
\setcounter{lesson_square_roots}{22}
\setcounter{lesson_i_and_complex_numbers}{23}
\setcounter{lesson_vertex_form_and_graphing}{24}
\setcounter{lesson_solve_by_square_roots}{25}
\setcounter{lesson_extracting_square_roots}{26}
\setcounter{lesson_the_discriminant}{27}
\setcounter{lesson_the_quadratic_formula}{28}
\setcounter{lesson_quadratic_inequalities}{29}
\setcounter{lesson_functions_and_relations}{12}
\setcounter{lesson_evaluating_functions}{13}
\setcounter{lesson_finding_domain_and_range_graphically}{14}
\setcounter{lesson_fundamental_functions}{15}
\setcounter{lesson_finding_domain_algebraically}{30}
\setcounter{lesson_solving_functions}{31}
\setcounter{lesson_function_arithmetic}{32}
\setcounter{lesson_composite_functions}{33}
\setcounter{lesson_inverse_functions_definition_and_HLT}{34}
\setcounter{lesson_finding_an_inverse_function}{35}
\setcounter{lesson_transformations_translations}{36}
\setcounter{lesson_transformations_reflections}{37}
\setcounter{lesson_transformations_scalings}{38}
\setcounter{lesson_transformations_summary}{39}
\setcounter{lesson_piecewise_functions}{40}
\setcounter{lesson_functions_containing_absolute_values}{41}
\setcounter{lesson_absolute_as_piecewise}{42}
\setcounter{lesson_polynomials_introduction}{43}
\setcounter{lesson_sign_diagrams_polynomials}{44}
\setcounter{lesson_factoring_quadratic_type}{46}
\setcounter{lesson_factoring_summary}{45}
\setcounter{lesson_polynomial_division}{47}
\setcounter{lesson_synthetic_division}{48}
\setcounter{lesson_end_behavior_polynomials}{49}
\setcounter{lesson_local_behavior_polynomials}{50}
\setcounter{lesson_rational_root_theorem}{51}
\setcounter{lesson_polynomials_graphing_summary}{52}
\setcounter{lesson_polynomial_inequalities}{53}
\setcounter{lesson_rationals_introduction_and_terminology}{54}
\setcounter{lesson_sign_diagrams_rationals}{55}
\setcounter{lesson_horizontal_asymptotes}{56}
\setcounter{lesson_slant_and_curvilinear_asymptotes}{57}
\setcounter{lesson_vertical_asymptotes}{58}
\setcounter{lesson_holes}{59}
\setcounter{lesson_rationals_graphing_summary}{60}

\begin{document}
{\bf \large Lesson \arabic{lesson_quadratics_introduction}: Introduction to Quadratics}\phantomsection\label{les:quadratics_introduction}
%\\ CC attribute: \href{http://www.wallace.ccfaculty.org/book/book.html}{\it{Beginning and Intermediate Algebra}} by T. Wallace. 
%\\ CC attribute: \href{http://www.stitz-zeager.com}{\it{College Algebra}} by C. Stitz and J. Zeager. 
\hfill \doclicenseImage[imagewidth=5em]\\
\par
{\bf Objective:} Recognize a quadratic equation and graphically.\\
\par
{\bf Students will be able to:}
\begin{itemize}
	\item Determine if an equation is linear or quadratic.
	\item Determine if the corresponding graph of a quadratic is concave up or down.
	\item Identify the $y-$intercept for the graph of a quadratic.
	\item Recognize the vertex form of a quadratic.
\end{itemize}
{\bf Prerequisite Knowledge:}
\begin{itemize}
	\item Application of the distributive property.
	\item Order of operations and combining like terms.
\end{itemize}
\hrulefill

{\bf Lesson:}\\
\ \par
A quadratic equation is an equation of the form $$y=ax^2+bx+c,$$ where the {\it coefficients} of $a$,$b$, and $c$ are real numbers and $a\neq 0$. This form is most commonly referred to as the {\it standard form} of a quadratic.  We call $a$ the {\it leading coefficient}, $ax^2$ the {\it leading term} (also known as the {\it quadratic term}), $bx$ the {\it linear term} and $c$ the {\it constant term}.  The quadratic term $ax^2$, must have a nonzero coefficient in order for the equation to be a quadratic (otherwise $f$ would be linear, in slope-intercept form). The most fundamental quadratic equation is $y=x^2$ and its graph, like all quadratics, is known as a {\it parabola}.\\
\ \par
The most useful form for graphing a quadratic equation is the \textit{vertex form}.  A quadratic equation is said to be in vertex form if it is represented as $$y=a(x-h)^2+k,$$ where $h$ and $k$ are real numbers.  The vertex form, unlike the standard form, allows us to immediately identify the vertex of the resulting parabola, which will be the point ($h,k$).\\
\newpage
{\bf I - Motivating Example(s):}
\begin{multicols}{2}
{\bf Example:} $y=x^2$\\

From the standard form, since $a>0$, the graph opens upwards and is said to be {\it concave up}.\\ \\
 As a result, there is a minimum point, known as the {\it vertex}, located at the origin, $(0,0)$. \\  \\
Notice the symmetry over the $y$-axis.

\columnbreak

\begin{center}
\begin{tikzpicture}[xscale=0.75,yscale=0.75]
	\draw [<->](-2.5,0) -- coordinate (x axis mid) (2.5,0) node[below right] {$x$};
	\draw [<->](0,-0.5) -- coordinate (x axis mid) (0,4.5) node[above right] {$y$};
	\draw [<->] plot [domain=-2:2, samples=100] (\x,{(\x)^2});
	\foreach \x in {1,2}
		\draw (\x,2pt) -- (\x,-2pt)	node[anchor=north] {\scriptsize \x};
	\foreach \x in {-2,-1}
		\draw (\x,2pt) -- (\x,-2pt)	node[anchor=north] {\scriptsize \x};
	\foreach \y in {1,2,...,4}
		\draw (2pt,\y) -- (-2pt,\y)	node[anchor=east] {\scriptsize \y}; 
 \draw[fill] (0,0) ellipse (0.075 and 0.075);
\end{tikzpicture}
\end{center}
\end{multicols}

\begin{multicols}{2}
{\bf Example:} $y=-x^2$\\
\ \par
Since $a = -1$, the graph opens downward or we say that it is {\it concave down}.\\
\\
Every parabola with a negative leading coefficient ($a<0$) will be concave down with a maximum value at its vertex.\\
%\columnbreak

\begin{center}
\begin{tikzpicture}[xscale=0.75,yscale=0.75]
	\draw [<->](-2.5,0) -- coordinate (x axis mid) (2.5,0) node[below right] {$x$};
	\draw [<->](0,-4.5) -- coordinate (x axis mid) (0,0.5) node[above right] {$y$};
	\draw [<->] plot [domain=-2:2, samples=100] (\x,{-1*(\x)^2});
	\foreach \x in {1,2}
		\draw (\x,2pt) -- (\x,-2pt)	node[anchor=north] {\scriptsize \x};
	\foreach \x in {-2,-1}
		\draw (\x,2pt) -- (\x,-2pt)	node[anchor=north] {\scriptsize \x};
	\foreach \y in {-4,-3,...,-1}
		\draw (2pt,\y) -- (-2pt,\y)	node[anchor=east] {\scriptsize \y}; 
 \draw[fill] (0,0) ellipse (0.075 and 0.075);
\end{tikzpicture}
\end{center}
\end{multicols}

\begin{multicols}{2}
{\bf Example:} $y = (x+3)^2-9$\\

The vertex is at $(-3,-9)$ and the graph can be realized as the graph of $y=x^2$ shifted left 3 units and down 9 units from the origin.\\ \\  
Since our graph is concave up there will be two $x$-intercepts as the curve opens upward from below the $x$-axis.

\columnbreak

\begin{center}
\begin{tikzpicture}[xscale=0.667,yscale=0.333]
	\draw [<->](-7.5,0) -- coordinate (x axis mid) (1.5,0) node[below right] {$x$};
	\draw [<->](0,-10) -- coordinate (x axis mid) (0,2) node[above right] {$y$};
	\draw [<->] plot [domain=-6.2:0.2, samples=100] (\x,{(\x+3)^2-9});
	\foreach \x in {1}
		\draw (\x,2pt) -- (\x,-2pt)	node[anchor=south] {\scriptsize \x};
	\foreach \x in {-7,-6,...,-1}
		\draw (\x,2pt) -- (\x,-2pt)	node[anchor=south] {\scriptsize \x};
	\foreach \y in {1}
		\draw (2pt,\y) -- (-2pt,\y)	node[anchor=west] {\scriptsize \y}; 
	\foreach \y in {-9,-7,...,-1}
		\draw (2pt,\y) -- (-2pt,\y)	node[anchor=west] {\scriptsize \y}; 
	\draw[fill] (-3,-9) ellipse (0.075 and 0.15);
	\draw[fill] (-6,0) ellipse (0.075 and 0.15);
	\draw[fill] (0,0) ellipse (0.075 and 0.15);
\end{tikzpicture}
\end{center}
\end{multicols}
{\bf II - Demo/Discussion Problems:}\\
\ \par
Describe the following equations as either linear or quadratic.  If quadratic, identify the $y-$intercept and determine whether the corresponding graph is concave up or down.
\begin{enumerate}
	\item $y=x^2-4x+2x^2-3x+5$
	\item $y=-2(x-3)(x+5)-6$
	\item $y=-2x^2-4x+2x^2-7x+7$
\end{enumerate}
\newpage
Identify the vertex, $y-$intercept, and concavity of each of the following quadratic equations.  Find the standard form for each quadratic.
\begin{multicols}{3}
\begin{enumerate}
	\item[4.] $y=-3(x-1)^2+2$
	\item[5.] $y=-\frac{1}{2}(x+2)^2$
	\item[6.] $y=x^2+3$
\end{enumerate}
\end{multicols}
{\bf III - Practice Problems:}\\
\ \par
Simplifying each of the following equations and classify each as linear, quadratic, or neither. If the equation is a quadratic, identify its $y-$intercept and concavity.
\begin{multicols}{2}
\begin{enumerate}
  \item $y=x^2 + 9$
  \item $y=5-2x+x^2$
  \item $y=x+6-3x$
  \item $y=5x+x^2-3x-3x^2$
  \item $y=-5x+3+2x-3x^2+6$
  \item $y=3x^2+-x+x-3x^2+6$
  \item $y=(x-1)(x+2)+3$
  \item $y=(x-5)(2x+3)-2(x-3)$
  \item $y=(x-4)(x+4)-(x+1)^2$
  \item $y=(2x-4)(x-1)-2(x+3)^2+3x^2$
\end{enumerate}
\end{multicols}

Identify the vertex and concavity of each of the following quadratics.

\begin{multicols}{3}
\begin{enumerate}
\setcounter{enumi}{10}
	\item $y=(x-3)^2+4$
  \item $y=(x-2)^2+5$
  \item $y=6(x+3)^2+4$
  \item $y=-2(x-3)^2+4$
  \item $y=-2(x-1)^2-7$
  \item $y=-(x+1)^2$
  \item $y=7x^2+4$
  \item $y=-\frac{1}{2}(x-8)^2+5$
  \item $y=x^2+4$
  \item $y=5x^2+23$
\end{enumerate}
\end{multicols}
\newpage
\ \newpage
\end{document}