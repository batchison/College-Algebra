\documentclass[12pt]{article}
\usepackage[top=1in,left=1in,bottom=1in,right=1in,headsep=2pt]{geometry}	
\usepackage{amssymb,amsmath,amsthm,amsfonts}
\usepackage{chapterfolder,docmute,setspace}
\usepackage{cancel,multicol,tikz,verbatim,framed,polynom,enumitem}
\usepackage[colorlinks, hyperindex, plainpages=false, linkcolor=blue, urlcolor=blue, pdfpagelabels]{hyperref}
% Use the cc-by-nc-sa license for any content linked with Stitz and Zeager's text.  Otherwise, use the cc-by-sa license.
%\usepackage[type={CC},modifier={by-sa},version={4.0},]{doclicense}
\usepackage[type={CC},modifier={by-nc-sa},version={4.0},]{doclicense}

\theoremstyle{definition}
\newtheorem{example}{Example}
\newcommand{\Desmos}{\href{https://www.desmos.com/}{Desmos}}
\setlength{\parindent}{0em}
\setlist{itemsep=0em}
\setlength{\parskip}{0.1em}
% This document is used for ordering of lessons.  If an instructor wishes to change the ordering of assessments, the following steps must be taken:

% 1) Reassign the appropriate numbers for each lesson in the \setcounter commands included in this file.
% 2) Rearrange the \include commands in the master file (the file with 'Course Pack' in the name) to accurately reflect the changes.  
% 3) Rearrange the \items in the measureable_outcomes file to accurately reflect the changes.  Be mindful of page breaks when moving items.
% 4) Re-build all affected files (master file, measureable_outcomes file, and any lesson whose numbering has changed).

%Note: The placement of each \newcounter and \setcounter command reflects the original/default ordering of topics (linears, systems, quadratics, functions, polynomials, rationals).

\newcounter{lesson_solving_linear_equations}
\newcounter{lesson_equations_containing_absolute_values}
\newcounter{lesson_graphing_lines}
\newcounter{lesson_two_forms_of_a_linear_equation}
\newcounter{lesson_parallel_and_perpendicular_lines}
\newcounter{lesson_linear_inequalities}
\newcounter{lesson_compound_inequalities}
\newcounter{lesson_inequalities_containing_absolute_values}
\newcounter{lesson_graphing_systems}
\newcounter{lesson_substitution}
\newcounter{lesson_elimination}
\newcounter{lesson_quadratics_introduction}
\newcounter{lesson_factoring_GCF}
\newcounter{lesson_factoring_grouping}
\newcounter{lesson_factoring_trinomials_a_is_1}
\newcounter{lesson_factoring_trinomials_a_neq_1}
\newcounter{lesson_solving_by_factoring}
\newcounter{lesson_square_roots}
\newcounter{lesson_i_and_complex_numbers}
\newcounter{lesson_vertex_form_and_graphing}
\newcounter{lesson_solve_by_square_roots}
\newcounter{lesson_extracting_square_roots}
\newcounter{lesson_the_discriminant}
\newcounter{lesson_the_quadratic_formula}
\newcounter{lesson_quadratic_inequalities}
\newcounter{lesson_functions_and_relations}
\newcounter{lesson_evaluating_functions}
\newcounter{lesson_finding_domain_and_range_graphically}
\newcounter{lesson_fundamental_functions}
\newcounter{lesson_finding_domain_algebraically}
\newcounter{lesson_solving_functions}
\newcounter{lesson_function_arithmetic}
\newcounter{lesson_composite_functions}
\newcounter{lesson_inverse_functions_definition_and_HLT}
\newcounter{lesson_finding_an_inverse_function}
\newcounter{lesson_transformations_translations}
\newcounter{lesson_transformations_reflections}
\newcounter{lesson_transformations_scalings}
\newcounter{lesson_transformations_summary}
\newcounter{lesson_piecewise_functions}
\newcounter{lesson_functions_containing_absolute_values}
\newcounter{lesson_absolute_as_piecewise}
\newcounter{lesson_polynomials_introduction}
\newcounter{lesson_sign_diagrams_polynomials}
\newcounter{lesson_factoring_quadratic_type}
\newcounter{lesson_factoring_summary}
\newcounter{lesson_polynomial_division}
\newcounter{lesson_synthetic_division}
\newcounter{lesson_end_behavior_polynomials}
\newcounter{lesson_local_behavior_polynomials}
\newcounter{lesson_rational_root_theorem}
\newcounter{lesson_polynomials_graphing_summary}
\newcounter{lesson_polynomial_inequalities}
\newcounter{lesson_rationals_introduction_and_terminology}
\newcounter{lesson_sign_diagrams_rationals}
\newcounter{lesson_horizontal_asymptotes}
\newcounter{lesson_slant_and_curvilinear_asymptotes}
\newcounter{lesson_vertical_asymptotes}
\newcounter{lesson_holes}
\newcounter{lesson_rationals_graphing_summary}

\setcounter{lesson_solving_linear_equations}{1}
\setcounter{lesson_equations_containing_absolute_values}{2}
\setcounter{lesson_graphing_lines}{3}
\setcounter{lesson_two_forms_of_a_linear_equation}{4}
\setcounter{lesson_parallel_and_perpendicular_lines}{5}
\setcounter{lesson_linear_inequalities}{6}
\setcounter{lesson_compound_inequalities}{7}
\setcounter{lesson_inequalities_containing_absolute_values}{8}
\setcounter{lesson_graphing_systems}{9}
\setcounter{lesson_substitution}{10}
\setcounter{lesson_elimination}{11}
\setcounter{lesson_quadratics_introduction}{16}
\setcounter{lesson_factoring_GCF}{17}
\setcounter{lesson_factoring_grouping}{18}
\setcounter{lesson_factoring_trinomials_a_is_1}{19}
\setcounter{lesson_factoring_trinomials_a_neq_1}{20}
\setcounter{lesson_solving_by_factoring}{21}
\setcounter{lesson_square_roots}{22}
\setcounter{lesson_i_and_complex_numbers}{23}
\setcounter{lesson_vertex_form_and_graphing}{24}
\setcounter{lesson_solve_by_square_roots}{25}
\setcounter{lesson_extracting_square_roots}{26}
\setcounter{lesson_the_discriminant}{27}
\setcounter{lesson_the_quadratic_formula}{28}
\setcounter{lesson_quadratic_inequalities}{29}
\setcounter{lesson_functions_and_relations}{12}
\setcounter{lesson_evaluating_functions}{13}
\setcounter{lesson_finding_domain_and_range_graphically}{14}
\setcounter{lesson_fundamental_functions}{15}
\setcounter{lesson_finding_domain_algebraically}{30}
\setcounter{lesson_solving_functions}{31}
\setcounter{lesson_function_arithmetic}{32}
\setcounter{lesson_composite_functions}{33}
\setcounter{lesson_inverse_functions_definition_and_HLT}{34}
\setcounter{lesson_finding_an_inverse_function}{35}
\setcounter{lesson_transformations_translations}{36}
\setcounter{lesson_transformations_reflections}{37}
\setcounter{lesson_transformations_scalings}{38}
\setcounter{lesson_transformations_summary}{39}
\setcounter{lesson_piecewise_functions}{40}
\setcounter{lesson_functions_containing_absolute_values}{41}
\setcounter{lesson_absolute_as_piecewise}{42}
\setcounter{lesson_polynomials_introduction}{43}
\setcounter{lesson_sign_diagrams_polynomials}{44}
\setcounter{lesson_factoring_quadratic_type}{46}
\setcounter{lesson_factoring_summary}{45}
\setcounter{lesson_polynomial_division}{47}
\setcounter{lesson_synthetic_division}{48}
\setcounter{lesson_end_behavior_polynomials}{49}
\setcounter{lesson_local_behavior_polynomials}{50}
\setcounter{lesson_rational_root_theorem}{51}
\setcounter{lesson_polynomials_graphing_summary}{52}
\setcounter{lesson_polynomial_inequalities}{53}
\setcounter{lesson_rationals_introduction_and_terminology}{54}
\setcounter{lesson_sign_diagrams_rationals}{55}
\setcounter{lesson_horizontal_asymptotes}{56}
\setcounter{lesson_slant_and_curvilinear_asymptotes}{57}
\setcounter{lesson_vertical_asymptotes}{58}
\setcounter{lesson_holes}{59}
\setcounter{lesson_rationals_graphing_summary}{60}

\begin{document}
{\bf \large Lesson \arabic{lesson_functions_and_relations}: Functions and Relations}
%\\ CC attribute: \href{http://www.wallace.ccfaculty.org/book/book.html}{\it{Beginning and Intermediate Algebra}} by T. Wallace. 
\\ CC attribute: \href{http://www.stitz-zeager.com}{\it{College Algebra}} by C. Stitz and J. Zeager. 
\hfill \doclicenseImage[imagewidth=5em]\\
\par
{\bf Objective:} Define a relation and a function; determine if a relation is a function.\\
\par
{\bf Students will be able to:}
\begin{itemize}
	\item Apply the Vertical Line Test to a graph to determine whether a relation represents a function.
	\item Determine whether a relation given as either a set, an equation, or a table of points represents a function.
\end{itemize}
{\bf Prerequisite Knowledge:}
\begin{itemize}
	\item Graph a set or table of points on the coordinate plane.
	\item Isolate a variable in an equation.
\end{itemize}
\hrulefill

{\bf Lesson:}\\
\ \par
A {\bf relation} $R$ is a set of points in the $xy$-plane.  A relation in which each $x$-coordinate is paired with exactly one $y$-coordinate is said to describe $y$ as a {\bf function} of $x$.  Relations which represent functions of $x$ will often be denoted by $f$, or $f(x)$, rather than $R$.  The set of all $x$-coordinates of the points in a function $f$ is called the {\bf domain} of $f$, and the set of all $y$-coordinates of the points in $f$ is called the {\bf range} of $f$.\\
\ \par
One major test that is used to determine whether or not a graph of a relation represents $y$ as a function of $x$ is known as the Vertical Line Test.  We will now state the Vertical Line Test as a mathematical theorem and then demonstrate its use.\\
\ \par
{\bf Vertical Line Test}: A set of points in the $xy$-plane represents $y$ as a function of $x$ if and only if no two points lie on the same vertical line.\\
\ \par
Alternatively stated, if a graph is known to represent $y$ as a function of $x$, then there can be no vertical line that intersects the graph in more than one point.  Conversely, if a known graph has the property that no vertical line intersects it in more than one point, then the given graph represents $y$ as a function of $x$.\\
\ \par
When we are presented with an equation, instead of a graph, we can still determine whether or not the equation for our relation represents $y$ as a function of $x$ by solving the equation for $y$ and carefully considering the result.  When solving for $y,$ the existence of the $\pm$ in our solution will cause our corresponding graph to fail the VLT.

\newpage

{\bf I - Motivating Example(s):}\\
\ \par
{\bf Example:} Use the Vertical Line Test to determine whether each of the following graphs represent $y$ as a function of $x$.

\begin{center}
\begin{multicols}{2}
\begin{tikzpicture}[xscale=0.8,yscale=0.8]
	\draw [<->](-4.25,0) -- coordinate (x axis mid) (4.25,0) node[below right] {$x$};
	\draw [<->](0,-4.25) -- coordinate (y axis mid) (0,4.25) node[above right] {$y$};
	\draw [<->] plot [domain=0.25:4, samples=100] (\x,{1/\x});
	\draw [<->] plot [domain=4:0.25, samples=100] (\x,{-1/\x});
%	\foreach \x in {1,...,4}
%		\draw (\x,2pt) -- (\x,-2pt)	node[anchor=north] {\scriptsize \x};
%	\foreach \x in {-4,...,-1}
%		\draw (\x,2pt) -- (\x,-2pt)	node[anchor=south] {\scriptsize \x};
%	\foreach \y in {1,...,4}
%		\draw (2pt,\y) -- (-2pt,\y)	node[anchor=east] {\scriptsize \y}; 
%	\foreach \y in {-4,...,-1}
%		\draw (2pt,\y) -- (-2pt,\y)	node[anchor=west] {\scriptsize \y}; 
	\draw (-3.75,-3) node[above] {\normalsize (a)};
	\end{tikzpicture}

\columnbreak

\begin{tikzpicture}[xscale=0.4, yscale=0.8]
  \draw [<->](-8.5,0) -- coordinate (x axis mid) (8.5,0) node[below right] {$x$};
	\draw [<->](0,-4.25) -- coordinate (y axis mid) (0,4.25) node[above right] {$y$};
	\draw [<->] plot [domain=8.2:0.2,samples=100] (\x,{log2(\x)});
	\draw [<->] plot [domain=8.2:0.2,samples=100] (-\x,{log2(\x)});
%	\foreach \x in {1,...,8}
%	\draw (\x,1pt) -- (\x,-3pt) node[anchor=north] {\scriptsize \x};
%	\foreach \y in {-3,...,-1}
%	\draw (1pt,\y) -- (-3pt,\y) node[anchor=east] {\scriptsize \y}; 
%	\foreach \y in {1,...,3}
%	\draw (1pt,\y) -- (-3pt,\y) node[anchor=east] {\scriptsize \y}; 
\draw (-7.5,-3) node[above] {\normalsize (b)};
\end{tikzpicture}
\end{multicols}
\end{center}
Graph (a) above fails the VLT, since any vertical line drawn in the right half-plane (where $x>0$) intersects the relation at two points.  Graph (b) passes the VLT, since no vertical line intersects the graph at more than one point.\\
\ \par
{\bf Example:} Determine whether the following equation represents $y$ as a function of $x$.
$$x^2+y^2=9$$
Solve the equation for $y$.
  \begin{eqnarray*}
    x^2+y^2=9~~~~~~~~~~~~  & & \text{Solve~for~} y\\
    \underline{-\cancel{x^2}}~~~~~~~~~~\underline{-x^2}~~~~~~~~~  & & \text{Subtract~} x^2\\
    y^2=9-x^2~~~~~  & & \\
	  \sqrt{y^2}=\pm\sqrt{9-x^2}  & & \text{Introduce~a~square~root}\\
		& & \text{~~~include a~} \pm \text{~on~right~side}\\
	  y=\pm\sqrt{9-x^2}  & & y \text{~is~not~a~function~of~}x
	\end{eqnarray*}
Due to the $\pm$, we can conclude that the equation does \textit{not} represent $y$ as a function of $x$.
\newpage

{\bf II - Demo/Discussion Problems:}\\
\ \par
Determine whether each of the following relations represent $y$ as a function of $x$.  Use \Desmos \ to sketch a graph of each relation.
\begin{center}
\begin{multicols}{2}
\begin{enumerate}
\item $\{(1,1),(2,-3),(2,0),(0,3),(-2,1/2)\}$
\item $\{(x,y)~|~x>3 \text{~and~} y\leq 2\}$
\item $x^2=1-y^2$
\item $x=y^2$
\item $y=x^2$
\item $y=3-2x$
\end{enumerate}
\end{multicols}
\end{center}

{\bf III - Practice Problems:}\\
\ \par
Determine if the following relations represent $y$ as a function of $x$ by making a table of values and graphing.  Explain your reasoning.  Use \Desmos \ to confirm your results.
\begin{center}
\begin{multicols}{3}
\begin{enumerate}
	\item $x=y^3$
	\item $y=x$
	\item $xy=1$
	\item $y=(x-3)^2$
	\item $x=(y-3)^2$
	\item $y<2x-5$
\end{enumerate}
\end{multicols}
\end{center}
Circle the letter of each graph/table below that represents $y$ as a function of $x$.
\begin{multicols}{2}
\begin{tikzpicture}[domain=0:3.7, scale=0.7]
    %\draw[very thin,color=gray] (-5,-3) grid (5,3);
		\draw [<->](-4.5,0) -- coordinate (x axis mid) (4.5,0) node[below right] {$x$};
		\draw [<->](0,-3.5) -- coordinate (y axis mid) (0,3.5) node[above right] {$y$};
		\node at (-2,3) {A};
			\foreach \x in {-4,...,-1}
			\draw (\x,1pt) -- (\x,-3pt)
			node[anchor=north] {\scriptsize \x};
			\foreach \x in {1,...,4}
			\draw (\x,1pt) -- (\x,-3pt)
			node[anchor=north] {\scriptsize \x};
			\foreach \y in {-3,...,-1}
			\draw (1pt,\y) -- (-3pt,\y) 
			node[anchor=west] {\scriptsize \y}; 
			\foreach \y in {1,...,3}
			\draw (1pt,\y) -- (-3pt,\y) 
			node[anchor=east] {\scriptsize \y}; 
    \draw[] (2,2) circle (0.125);
		\draw[fill] (3,2) circle (0.125);
		\draw[-] [ultra thick]  (2.1,2)--(2.9,2);
    \draw[] (1,1) circle (0.125);
		\draw[fill] (2,1) circle (0.125);
		\draw[-] [ultra thick]  (1.1,1)--(1.9,1);
    \draw[] (0,0) circle (0.125);
		\draw[fill] (1,0) circle (0.125);
		\draw[-] [ultra thick]  (0.1,0)--(0.9,0);
    \draw[] (-1,-1) circle (0.125);
		\draw[fill] (0,-1) circle (0.125);
		\draw[-] [ultra thick]  (-0.9,-1)--(-0.1,-1);
    \draw[] (-2,-2) circle (0.125);
		\draw[fill] (-1,-2) circle (0.125);
		\draw[-] [ultra thick]  (-1.9,-2)--(-1.1,-2);
\end{tikzpicture}
\hspace{1in}
\begin{tikzpicture}[domain=0:3.7, scale=0.7]
    %\draw[very thin,color=gray] (-5,-3) grid (5,3);
		\draw [<->](-4.5,0) -- coordinate (x axis mid) (4.5,0) node[below right] {$x$};
		\draw [<->](0,-3.5) -- coordinate (y axis mid) (0,3.5) node[above right] {$y$};
		\node at (-2,3) {B};
			\foreach \x in {-4,...,-1}
			\draw (\x,1pt) -- (\x,-3pt)
			node[anchor=north] {\scriptsize \x};
			\foreach \x in {1,...,4}
			\draw (\x,1pt) -- (\x,-3pt)
			node[anchor=north] {\scriptsize \x};
			\foreach \y in {-3,...,-1}
			\draw (1pt,\y) -- (-3pt,\y) 
			node[anchor=east] {\scriptsize \y}; 
			\foreach \y in {1,...,3}
			\draw (1pt,\y) -- (-3pt,\y) 
			node[anchor=east] {\scriptsize \y}; 
    \draw[->,domain=0:2]   plot ({\x^2},\x);
		\draw[->,domain=0:2]   plot ({\x^2},-\x);
%    \draw[->]   plot (\x,{sin(2*\x r)});
%    \draw[->]   plot (-\x,{-sin(2*\x r)});
\end{tikzpicture}
\end{multicols}
\vspace{0.25in}
\begin{multicols}{2}
\begin{tikzpicture}[domain=1.022:3.6, scale=0.7]
    %\draw[very thin,color=gray] (-5,-3) grid (5,3);
		\draw [<->](-4,0) -- coordinate (x axis mid) (4,0) node[below right] {$x$};
		\draw [<->](0,-3.5) -- coordinate (y axis mid) (0,3.5) node[above right] {$y$};
		\node at (-2,3) {C};
			\foreach \x in {-3,...,-1}
			\draw (\x,0.8pt) -- (\x,-0.8pt)
			node[anchor=north] {\scriptsize \x};
			\foreach \x in {1,...,3}
			\draw (\x,0.8pt) -- (\x,-0.8pt)
			node[anchor=north] {\scriptsize \x};
			%\foreach \y in {-3,...,-1}
			%\draw (1pt,\y) -- (-3pt,\y) 
			%node[anchor=east] {\scriptsize \y}; 
			\foreach \y in {1,...,3}
			\draw (1pt,\y) -- (-3pt,\y) 
			node[anchor=east] {\scriptsize \y}; 
			\foreach \y in {-3,...,-1}
			\draw (1pt,\y) -- (-3pt,\y) 
			node[anchor=west] {\scriptsize \y}; 
    \draw[->, domain=0:1.5]   plot (\x,{\x^2+1});
    \draw[->, domain=0:1.5]   plot (-\x,{-\x^2-1)});
		\draw[fill] (0,1) circle (0.13);
		\draw[fill] (0,-1) circle (0.13);
\end{tikzpicture}	
\hspace{1.3in}
\begin{tikzpicture}[domain=2:3.6, scale=0.8]
    %\draw[very thin,color=gray] (-5,-3) grid (5,3);
		\draw [<->](-4,0) -- coordinate (x axis mid) (4,0) node[below right] {$x$};
		\draw [<->](0,-1) -- coordinate (y axis mid) (0,3.5) node[above right] {$y$};
		\node at (-2,3) {D};
			\foreach \x in {-3,...,-1}
			\draw (\x,1pt) -- (\x,-3pt)
			node[anchor=north] {\scriptsize \x};
			\foreach \x in {1,...,3}
			\draw (\x,1pt) -- (\x,-3pt)
			node[anchor=north] {\scriptsize \x};
			%\foreach \y in {-3,...,-1}
			%\draw (1pt,\y) -- (-3pt,\y) 
			%node[anchor=east] {\scriptsize \y}; 
			\foreach \y in {1,...,3}
			\draw (1pt,\y) -- (-3pt,\y) 
			node[anchor=east] {\scriptsize \y}; 
    \draw[->,domain=1:1.75]   plot (\x,{\x^2-1});
    \draw[-,domain=0:1]   plot (\x,{-\x^2+1});
    \draw[-,domain=0:1]   plot (-\x,{-\x^2+1});
    \draw[->,domain=1:1.75]   plot (-\x,{\x^2-1});
    %\draw[->]   plot (\x,{sqrt(\x-2)});
    %\draw[->]   plot (-\x,{sqrt(\x-2)});
		%\draw[fill] (2,0) circle (0.13);
		%\draw[fill] (-2,0) circle (0.13);
\end{tikzpicture}
\end{multicols}
\vspace{0.5in}
E\hspace{1.5in} F\hspace{1.5in} G\hspace{1.5in} H 
%\begin{center}
\begin{multicols}{4}
\begin{tabular}{r|r}
$x$ & $y$\\
\hline
%&\\
$3$ & $-3$\\
%&\\
$2$ & $-2$\\
%&\\
$1$ & $-1$\\
%&\\
$0$ & $0$\\
%&\\
$1$ & $1$\\
%&\\
$2$ & $2$\\
%&\\
$3$ & $3$
\end{tabular}
\hspace{1in}

\begin{tabular}{r|r}
$x$ & $y$\\
\hline
%&\\
$-3$ & $3$\\
%&\\
$-2$ & $2$\\
%&\\
$-1$ & $1$\\
%&\\
$0$ & $0$\\
%&\\
$1$ & $1$\\
%&\\
$2$ & $2$\\
%&\\
$3$ & $3$
\end{tabular}
\hspace{1in}

\begin{tabular}{r|r}
$x$ & $y$\\
\hline
%&\\
$-3$ & $0$\\
%&\\
$-2$ & $0$\\
%&\\
$-1$ & $0$\\
%&\\
$0$ & $0$\\
%&\\
$1$ & $0$\\
%&\\
$2$ & $0$\\
%&\\
$3$ & $0$
\end{tabular}
\hspace{1in}
\begin{tabular}{r|c}
$x$ & $y$\\
\hline
%&\\
$3$ & $8$\\
%&\\
$2$ & $4$\\
%&\\
$1$ & $2$\\
%&\\
$0$ & $1$\\
%&\\
$-1$ & $1/2$\\
%&\\
$-2$ & $1/4$\\
%&\\
$-3$ & $1/8$
\end{tabular}
\end{multicols}
\newpage
\end{document}