\documentclass[12pt]{article}
\usepackage[top=1in,left=1in,bottom=1in,right=1in,headsep=2pt]{geometry}	
\usepackage{amssymb,amsmath,amsthm,amsfonts}
\usepackage{chapterfolder,docmute,setspace}
\usepackage{cancel,multicol,tikz,verbatim,framed,polynom,enumitem}
\usepackage[colorlinks, hyperindex, plainpages=false, linkcolor=blue, urlcolor=blue, pdfpagelabels]{hyperref}
\usepackage[type={CC},modifier={by-sa},version={4.0},]{doclicense}

\theoremstyle{definition}
\newtheorem{example}{Example}
\newcommand{\Desmos}{\href{https://www.desmos.com/}{Desmos}}
\setlength{\parindent}{0em}
\setlist{itemsep=0em}
\setlength{\parskip}{0.1em}
% This document is used for ordering of lessons.  If an instructor wishes to change the ordering of assessments, the following steps must be taken:

% 1) Reassign the appropriate numbers for each lesson in the \setcounter commands included in this file.
% 2) Rearrange the \include commands in the master file (the file with 'Course Pack' in the name) to accurately reflect the changes.  
% 3) Rearrange the \items in the measureable_outcomes file to accurately reflect the changes.  Be mindful of page breaks when moving items.
% 4) Re-build all affected files (master file, measureable_outcomes file, and any lesson whose numbering has changed).

%Note: The placement of each \newcounter and \setcounter command reflects the original/default ordering of topics (linears, systems, quadratics, functions, polynomials, rationals).

\newcounter{lesson_solving_linear_equations}
\newcounter{lesson_equations_containing_absolute_values}
\newcounter{lesson_graphing_lines}
\newcounter{lesson_two_forms_of_a_linear_equation}
\newcounter{lesson_parallel_and_perpendicular_lines}
\newcounter{lesson_linear_inequalities}
\newcounter{lesson_compound_inequalities}
\newcounter{lesson_inequalities_containing_absolute_values}
\newcounter{lesson_graphing_systems}
\newcounter{lesson_substitution}
\newcounter{lesson_elimination}
\newcounter{lesson_quadratics_introduction}
\newcounter{lesson_factoring_GCF}
\newcounter{lesson_factoring_grouping}
\newcounter{lesson_factoring_trinomials_a_is_1}
\newcounter{lesson_factoring_trinomials_a_neq_1}
\newcounter{lesson_solving_by_factoring}
\newcounter{lesson_square_roots}
\newcounter{lesson_i_and_complex_numbers}
\newcounter{lesson_vertex_form_and_graphing}
\newcounter{lesson_solve_by_square_roots}
\newcounter{lesson_extracting_square_roots}
\newcounter{lesson_the_discriminant}
\newcounter{lesson_the_quadratic_formula}
\newcounter{lesson_quadratic_inequalities}
\newcounter{lesson_functions_and_relations}
\newcounter{lesson_evaluating_functions}
\newcounter{lesson_finding_domain_and_range_graphically}
\newcounter{lesson_fundamental_functions}
\newcounter{lesson_finding_domain_algebraically}
\newcounter{lesson_solving_functions}
\newcounter{lesson_function_arithmetic}
\newcounter{lesson_composite_functions}
\newcounter{lesson_inverse_functions_definition_and_HLT}
\newcounter{lesson_finding_an_inverse_function}
\newcounter{lesson_transformations_translations}
\newcounter{lesson_transformations_reflections}
\newcounter{lesson_transformations_scalings}
\newcounter{lesson_transformations_summary}
\newcounter{lesson_piecewise_functions}
\newcounter{lesson_functions_containing_absolute_values}
\newcounter{lesson_absolute_as_piecewise}
\newcounter{lesson_polynomials_introduction}
\newcounter{lesson_sign_diagrams_polynomials}
\newcounter{lesson_factoring_quadratic_type}
\newcounter{lesson_factoring_summary}
\newcounter{lesson_polynomial_division}
\newcounter{lesson_synthetic_division}
\newcounter{lesson_end_behavior_polynomials}
\newcounter{lesson_local_behavior_polynomials}
\newcounter{lesson_rational_root_theorem}
\newcounter{lesson_polynomials_graphing_summary}
\newcounter{lesson_polynomial_inequalities}
\newcounter{lesson_rationals_introduction_and_terminology}
\newcounter{lesson_sign_diagrams_rationals}
\newcounter{lesson_horizontal_asymptotes}
\newcounter{lesson_slant_and_curvilinear_asymptotes}
\newcounter{lesson_vertical_asymptotes}
\newcounter{lesson_holes}
\newcounter{lesson_rationals_graphing_summary}

\setcounter{lesson_solving_linear_equations}{1}
\setcounter{lesson_equations_containing_absolute_values}{2}
\setcounter{lesson_graphing_lines}{3}
\setcounter{lesson_two_forms_of_a_linear_equation}{4}
\setcounter{lesson_parallel_and_perpendicular_lines}{5}
\setcounter{lesson_linear_inequalities}{6}
\setcounter{lesson_compound_inequalities}{7}
\setcounter{lesson_inequalities_containing_absolute_values}{8}
\setcounter{lesson_graphing_systems}{9}
\setcounter{lesson_substitution}{10}
\setcounter{lesson_elimination}{11}
\setcounter{lesson_quadratics_introduction}{16}
\setcounter{lesson_factoring_GCF}{17}
\setcounter{lesson_factoring_grouping}{18}
\setcounter{lesson_factoring_trinomials_a_is_1}{19}
\setcounter{lesson_factoring_trinomials_a_neq_1}{20}
\setcounter{lesson_solving_by_factoring}{21}
\setcounter{lesson_square_roots}{22}
\setcounter{lesson_i_and_complex_numbers}{23}
\setcounter{lesson_vertex_form_and_graphing}{24}
\setcounter{lesson_solve_by_square_roots}{25}
\setcounter{lesson_extracting_square_roots}{26}
\setcounter{lesson_the_discriminant}{27}
\setcounter{lesson_the_quadratic_formula}{28}
\setcounter{lesson_quadratic_inequalities}{29}
\setcounter{lesson_functions_and_relations}{12}
\setcounter{lesson_evaluating_functions}{13}
\setcounter{lesson_finding_domain_and_range_graphically}{14}
\setcounter{lesson_fundamental_functions}{15}
\setcounter{lesson_finding_domain_algebraically}{30}
\setcounter{lesson_solving_functions}{31}
\setcounter{lesson_function_arithmetic}{32}
\setcounter{lesson_composite_functions}{33}
\setcounter{lesson_inverse_functions_definition_and_HLT}{34}
\setcounter{lesson_finding_an_inverse_function}{35}
\setcounter{lesson_transformations_translations}{36}
\setcounter{lesson_transformations_reflections}{37}
\setcounter{lesson_transformations_scalings}{38}
\setcounter{lesson_transformations_summary}{39}
\setcounter{lesson_piecewise_functions}{40}
\setcounter{lesson_functions_containing_absolute_values}{41}
\setcounter{lesson_absolute_as_piecewise}{42}
\setcounter{lesson_polynomials_introduction}{43}
\setcounter{lesson_sign_diagrams_polynomials}{44}
\setcounter{lesson_factoring_quadratic_type}{46}
\setcounter{lesson_factoring_summary}{45}
\setcounter{lesson_polynomial_division}{47}
\setcounter{lesson_synthetic_division}{48}
\setcounter{lesson_end_behavior_polynomials}{49}
\setcounter{lesson_local_behavior_polynomials}{50}
\setcounter{lesson_rational_root_theorem}{51}
\setcounter{lesson_polynomials_graphing_summary}{52}
\setcounter{lesson_polynomial_inequalities}{53}
\setcounter{lesson_rationals_introduction_and_terminology}{54}
\setcounter{lesson_sign_diagrams_rationals}{55}
\setcounter{lesson_horizontal_asymptotes}{56}
\setcounter{lesson_slant_and_curvilinear_asymptotes}{57}
\setcounter{lesson_vertical_asymptotes}{58}
\setcounter{lesson_holes}{59}
\setcounter{lesson_rationals_graphing_summary}{60}

\begin{document}
{\bf \large Lesson \arabic{lesson_graphing_systems}: Graphing Systems of Linear Equations}\\
CC attribute: \href{http://www.wallace.ccfaculty.org/book/book.html}{\it{Beginning and Intermediate Algebra}} by T. Wallace. \hfill \doclicenseImage[imagewidth=5em]\\
\par
{\bf Objective:} Solve linear systems by graphing.\\
\par
{\bf Students will be able to:}
\begin{itemize}
	\item Solve linear systems by graphing both equations on one coordinate plane.
	\item Write system solutions as ordered pairs in the form $(x,y)$.
	\item Verify the accuracy of a solution by plugging it into each equation in the system.
\end{itemize}
{\bf Prerequisite Knowledge:}
\begin{itemize}
	\item Find the slope-intercept form of a linear equation.
	\item Graph linear equations in slope-intercept form.
	\item Plot points on the coordinate plane.
\end{itemize}
\hrulefill

{\bf Lesson:}
\par
{\bf I - Motivating Example(s):}\\
\par
Solve the following system of equations.
  \begin{eqnarray*}
%    \begin{array}{l}
 	\begin{cases} 
     y = - \frac{1}{2} x + 3\\
    	y = \frac{3}{4} x - 2
  		\end{cases}
%   \end{array} 
&  & \text{First identify slopes and} \ y- \text{intercepts.}\\
  & & \\ 
	\begin{array}{l}
      \text{Line 1} : \ m = - \frac{1}{2}, \ \ b = 3\\
      \text{Line 2} : \ m = \frac{3}{4}, \ \ \ b = - 2
    \end{array} &  & \text{Next graph both lines on the same plane.}\\
  \end{eqnarray*}
  
	\begin{multicols}{2}
 	\begin{tikzpicture}[xscale=0.4,yscale=0.4]
		\draw[step=1.0,gray,very thin,dotted] (-8.5,-5.5) grid (8.5,5.5);
		\draw [<->](-8.5,0) -- coordinate (x axis mid) (8.5,0) node[below right] {$x$};
		\draw [<->](0,-5.5) -- coordinate (y axis mid) (0,5.5) node[above right] {$y$};
		\draw [<->,line width=0.4mm] plot [domain=-4:8, samples=100] (\x,{-0.5*\x+3});
		\draw [<->,line width=0.4mm] plot [domain=-4:8, samples=100] (\x,{0.75*\x-2});
		\draw[fill] (0,3) circle (0.2);
		\draw[fill] (2,2) circle (0.2);
		\draw[fill] (4,1) circle (0.2);
		\draw[fill] (0,-2) circle (0.2);
		\draw (-3,3) node {$\ell_1$};
		\draw (2,-2) node {$\ell_2$};
		\draw (3.75,2.5) node {$(4,1)$};
	\end{tikzpicture}
   
  \columnbreak
  To graph each equation, we start at the $y$-intercept and use the slope $\left(\dfrac{\text{rise}}{\text{run}}\right)$ to get the next point, then connect the dots.\\
	\ \par
	Remember a line with a negative slope decreases from left to right.\\
	\ \par
	Use slopes to find the intersection point, $(4,1)$.  This is our solution.
  \end{multicols}

Often our equations won't be in slope-intercept form and we will first have to solve both for $y$ so we can identify the slope and $y$-intercept.\\
\newpage
{\bf II - Demo/Discussion Problems:}\\
\ \par
Solve each of the following systems of linear equations by graphing.
\begin{multicols}{3}
	\begin{enumerate}
		\item $\begin{cases}
					y=2x-4\\
			    y=-2x+4
					\end{cases}$
		\item $\begin{cases}
					2y-3x=-8\\
					2y-3x=2
					\end{cases}$
		\item $\begin{cases} 
					6x-3y=-9\\
					2x+2y=-6 
					\end{cases}$
	\end{enumerate}
\end{multicols}
\ \par
{\bf III - Practice Problems:}\\
\ \par
Solve each of the following systems of linear equations by graphing.

\begin{multicols}{3}
	\begin{enumerate}
	\item
	$\begin{cases}
	y = - x + 1\\
	y = - 5 x - 3
  \end{cases}$
	\item
	$\begin{cases}
	y = - \frac{3}{4} x + 1\\
	y = - \frac{3}{4} x + 2
  \end{cases}$
  \item
	$\begin{cases}
	y = \frac{5}{3} x + 4\\
	y = - \frac{2}{3} x - 3
  \end{cases}$
  \item
	$\begin{cases}
	x - y = 4\\
	2 x + y = - 1
  \end{cases}$\\
  \item
	$\begin{cases}
	2 x + y = 2\\
	x - y = 4
  \end{cases}$
  \item
	$\begin{cases}
	9y+6x=36\\
	3y-6x=-12
  \end{cases}$
  \item
	$\begin{cases}
	3 + y = - x\\
	- 4 - 6 x = - y
  \end{cases}$
	\item
	$\begin{cases}
	y = - \frac{5}{4} x - 2\\
	y = - \frac{1}{4} x + 2
  \end{cases}$
  \item
	$\begin{cases}
	y = 2 x + 2\\
	y = - x - 4
  \end{cases}$
  \item
	$\begin{cases}
	y = \frac{1}{2} x + 4\\
	y = \frac{1}{2} x + 1
  \end{cases}$
  \item
	$\begin{cases}
	6 x + y = - 3\\
	x + y = 2
  \end{cases}$
  \item
	$\begin{cases}
	x + 2 y = 6\\
	5 x - 4 y = 16
  \end{cases}$
  \item
	$\begin{cases}
	- 2 y + x = 4\\
	2 = - x + \frac{1}{2} y
  \end{cases}$
  \item
	$\begin{cases}
	16 = - x - 4 y\\
	- 2 x = - 4 - 4 y
  \end{cases}$
	\item
	$\begin{cases}
	y = - 3\\
	y = - x - 4
  \end{cases}$
  \item
	$\begin{cases}
	y = \frac{1}{3} x + 2\\
	y = - \frac{5}{3} x - 4
  \end{cases}$
  \item
	$\begin{cases}
	x + 3 y = - 9\\
	5 x + 3 y = 3
  \end{cases}$
  \item
	$\begin{cases}
	2 x + 3 y = - 6\\
	2 x + y = 2
  \end{cases}$
  \item
	$\begin{cases}
	2 x + y = - 2\\
	x + 3 y = 9
  \end{cases}$
  \item
	$\begin{cases}
	2 x - y = - 1\\
	3 = - 2 x - y
  \end{cases}$
  \item
	$\begin{cases}
	- y + 7 x = 4\\
	- y + 7 x = 3
  \end{cases}$
  \item
	$\begin{cases}
	y = - x - 2\\
	y = \frac{2}{3} x + 3
  \end{cases}$
  \item
	$\begin{cases}
	y = 2 x - 4\\
	y = \frac{1}{2} x + 2
  \end{cases}$
  \item
	$\begin{cases}
	x + 4 y = - 12\\
	2 x + y = 4
  \end{cases}$
  \item
	$\begin{cases}
	3 x + 2 y = 2\\
	3 x + 2 y = - 6
  \end{cases}$
  \item
	$\begin{cases}
	x - y = 3\\
	5 x + 2 y = 8
  \end{cases}$
	\item
	$\begin{cases}
	- 2 y = - 4 - x\\
	- 2 y = - 5 x + 4
  \end{cases}$
  \item
	$\begin{cases}
	- 4 + y = x\\
	x + 2 = - y
	\end{cases}$
	\end{enumerate}
\end{multicols}
\newpage
\end{document}