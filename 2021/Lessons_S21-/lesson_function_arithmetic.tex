\documentclass[12pt]{article}
\usepackage[top=1in,left=1in,bottom=1in,right=1in,headsep=2pt]{geometry}	
\usepackage{amssymb,amsmath,amsthm,amsfonts}
\usepackage{chapterfolder,docmute,setspace}
\usepackage{cancel,multicol,tikz,verbatim,framed,polynom,enumitem}
\usepackage[colorlinks, hyperindex, plainpages=false, linkcolor=blue, urlcolor=blue, pdfpagelabels]{hyperref}
% Use the cc-by-nc-sa license for any content linked with Stitz and Zeager's text.  Otherwise, use the cc-by-sa license.
\usepackage[type={CC},modifier={by-sa},version={4.0},]{doclicense}
%\usepackage[type={CC},modifier={by-nc-sa},version={4.0},]{doclicense}

\theoremstyle{definition}
\newtheorem{example}{Example}
\newcommand{\Desmos}{\href{https://www.desmos.com/}{Desmos}}
\setlength{\parindent}{0em}
\setlist{itemsep=0em}
\setlength{\parskip}{0.1em}
% This document is used for ordering of lessons.  If an instructor wishes to change the ordering of assessments, the following steps must be taken:

% 1) Reassign the appropriate numbers for each lesson in the \setcounter commands included in this file.
% 2) Rearrange the \include commands in the master file (the file with 'Course Pack' in the name) to accurately reflect the changes.  
% 3) Rearrange the \items in the measureable_outcomes file to accurately reflect the changes.  Be mindful of page breaks when moving items.
% 4) Re-build all affected files (master file, measureable_outcomes file, and any lesson whose numbering has changed).

%Note: The placement of each \newcounter and \setcounter command reflects the original/default ordering of topics (linears, systems, quadratics, functions, polynomials, rationals).

\newcounter{lesson_solving_linear_equations}
\newcounter{lesson_equations_containing_absolute_values}
\newcounter{lesson_graphing_lines}
\newcounter{lesson_two_forms_of_a_linear_equation}
\newcounter{lesson_parallel_and_perpendicular_lines}
\newcounter{lesson_linear_inequalities}
\newcounter{lesson_compound_inequalities}
\newcounter{lesson_inequalities_containing_absolute_values}
\newcounter{lesson_graphing_systems}
\newcounter{lesson_substitution}
\newcounter{lesson_elimination}
\newcounter{lesson_quadratics_introduction}
\newcounter{lesson_factoring_GCF}
\newcounter{lesson_factoring_grouping}
\newcounter{lesson_factoring_trinomials_a_is_1}
\newcounter{lesson_factoring_trinomials_a_neq_1}
\newcounter{lesson_solving_by_factoring}
\newcounter{lesson_square_roots}
\newcounter{lesson_i_and_complex_numbers}
\newcounter{lesson_vertex_form_and_graphing}
\newcounter{lesson_solve_by_square_roots}
\newcounter{lesson_extracting_square_roots}
\newcounter{lesson_the_discriminant}
\newcounter{lesson_the_quadratic_formula}
\newcounter{lesson_quadratic_inequalities}
\newcounter{lesson_functions_and_relations}
\newcounter{lesson_evaluating_functions}
\newcounter{lesson_finding_domain_and_range_graphically}
\newcounter{lesson_fundamental_functions}
\newcounter{lesson_finding_domain_algebraically}
\newcounter{lesson_solving_functions}
\newcounter{lesson_function_arithmetic}
\newcounter{lesson_composite_functions}
\newcounter{lesson_inverse_functions_definition_and_HLT}
\newcounter{lesson_finding_an_inverse_function}
\newcounter{lesson_transformations_translations}
\newcounter{lesson_transformations_reflections}
\newcounter{lesson_transformations_scalings}
\newcounter{lesson_transformations_summary}
\newcounter{lesson_piecewise_functions}
\newcounter{lesson_functions_containing_absolute_values}
\newcounter{lesson_absolute_as_piecewise}
\newcounter{lesson_polynomials_introduction}
\newcounter{lesson_sign_diagrams_polynomials}
\newcounter{lesson_factoring_quadratic_type}
\newcounter{lesson_factoring_summary}
\newcounter{lesson_polynomial_division}
\newcounter{lesson_synthetic_division}
\newcounter{lesson_end_behavior_polynomials}
\newcounter{lesson_local_behavior_polynomials}
\newcounter{lesson_rational_root_theorem}
\newcounter{lesson_polynomials_graphing_summary}
\newcounter{lesson_polynomial_inequalities}
\newcounter{lesson_rationals_introduction_and_terminology}
\newcounter{lesson_sign_diagrams_rationals}
\newcounter{lesson_horizontal_asymptotes}
\newcounter{lesson_slant_and_curvilinear_asymptotes}
\newcounter{lesson_vertical_asymptotes}
\newcounter{lesson_holes}
\newcounter{lesson_rationals_graphing_summary}

\setcounter{lesson_solving_linear_equations}{1}
\setcounter{lesson_equations_containing_absolute_values}{2}
\setcounter{lesson_graphing_lines}{3}
\setcounter{lesson_two_forms_of_a_linear_equation}{4}
\setcounter{lesson_parallel_and_perpendicular_lines}{5}
\setcounter{lesson_linear_inequalities}{6}
\setcounter{lesson_compound_inequalities}{7}
\setcounter{lesson_inequalities_containing_absolute_values}{8}
\setcounter{lesson_graphing_systems}{9}
\setcounter{lesson_substitution}{10}
\setcounter{lesson_elimination}{11}
\setcounter{lesson_quadratics_introduction}{16}
\setcounter{lesson_factoring_GCF}{17}
\setcounter{lesson_factoring_grouping}{18}
\setcounter{lesson_factoring_trinomials_a_is_1}{19}
\setcounter{lesson_factoring_trinomials_a_neq_1}{20}
\setcounter{lesson_solving_by_factoring}{21}
\setcounter{lesson_square_roots}{22}
\setcounter{lesson_i_and_complex_numbers}{23}
\setcounter{lesson_vertex_form_and_graphing}{24}
\setcounter{lesson_solve_by_square_roots}{25}
\setcounter{lesson_extracting_square_roots}{26}
\setcounter{lesson_the_discriminant}{27}
\setcounter{lesson_the_quadratic_formula}{28}
\setcounter{lesson_quadratic_inequalities}{29}
\setcounter{lesson_functions_and_relations}{12}
\setcounter{lesson_evaluating_functions}{13}
\setcounter{lesson_finding_domain_and_range_graphically}{14}
\setcounter{lesson_fundamental_functions}{15}
\setcounter{lesson_finding_domain_algebraically}{30}
\setcounter{lesson_solving_functions}{31}
\setcounter{lesson_function_arithmetic}{32}
\setcounter{lesson_composite_functions}{33}
\setcounter{lesson_inverse_functions_definition_and_HLT}{34}
\setcounter{lesson_finding_an_inverse_function}{35}
\setcounter{lesson_transformations_translations}{36}
\setcounter{lesson_transformations_reflections}{37}
\setcounter{lesson_transformations_scalings}{38}
\setcounter{lesson_transformations_summary}{39}
\setcounter{lesson_piecewise_functions}{40}
\setcounter{lesson_functions_containing_absolute_values}{41}
\setcounter{lesson_absolute_as_piecewise}{42}
\setcounter{lesson_polynomials_introduction}{43}
\setcounter{lesson_sign_diagrams_polynomials}{44}
\setcounter{lesson_factoring_quadratic_type}{46}
\setcounter{lesson_factoring_summary}{45}
\setcounter{lesson_polynomial_division}{47}
\setcounter{lesson_synthetic_division}{48}
\setcounter{lesson_end_behavior_polynomials}{49}
\setcounter{lesson_local_behavior_polynomials}{50}
\setcounter{lesson_rational_root_theorem}{51}
\setcounter{lesson_polynomials_graphing_summary}{52}
\setcounter{lesson_polynomial_inequalities}{53}
\setcounter{lesson_rationals_introduction_and_terminology}{54}
\setcounter{lesson_sign_diagrams_rationals}{55}
\setcounter{lesson_horizontal_asymptotes}{56}
\setcounter{lesson_slant_and_curvilinear_asymptotes}{57}
\setcounter{lesson_vertical_asymptotes}{58}
\setcounter{lesson_holes}{59}
\setcounter{lesson_rationals_graphing_summary}{60}

\begin{document}
{\bf \large Lesson \arabic{lesson_function_arithmetic}: Function Arithmetic}\phantomsection\label{les:function_arithmetic}
\\ CC attribute: \href{http://www.wallace.ccfaculty.org/book/book.html}{\it{Beginning and Intermediate Algebra}} by T. Wallace. 
%\\ CC attribute: \href{http://www.stitz-zeager.com}{\it{College Algebra}} by C. Stitz and J. Zeager. 
\hfill \doclicenseImage[imagewidth=5em]\\
\par
{\bf Objective:} Add, subtract, multiply, and divide functions.\\
\par
{\bf Students will be able to:}
\begin{itemize}
	\item Evaluate functions that are added, subtracted, multiplied, or divided by substituting a value into each function, then applying the operation and simplifying.
	\item Apply the four basic operations to functions of the same variable.
\end{itemize}
{\bf Prerequisite Knowledge:}
\begin{itemize}
	\item Order of operations.
	\item Evaluating functions.
	\item Parentheses and grouping.
\end{itemize}
\hrulefill

{\bf Lesson:}\\
\ \par
The notation for the four basic function operations is as follows. 
\begin{eqnarray*}
  \text{Addition} &  & (f + g) (x) = f (x) + g (x)\\
	\text{Subtraction} &  & (f - g) (x) = f (x) - g (x)\\  
	\text{Multiplication} &  & (f \cdot g) (x) = f (x)\cdot g (x)\\
	\text{Division} &  & \left(\dfrac{f}{g}\right) (x) = \dfrac{f (x)}{g (x)}
\end{eqnarray*}
{\bf I - Motivating Example(s):}\\
\ \par
{\bf Example:} Find $f+g$, where $f (x) = x^2 - x - 2$ and $g(x) = x + 1$.
  \begin{eqnarray*}
    (f + g)(x)~~~~~~~~~~ &  & \text{Consider the problem} \\
		f(x) + g(x)~~~~~~~~~ &  & \text{Rewrite as a sum of two functions}\\
		(x^2 - x - 2) + (x + 1) &  & \text{Substitute functions, inserting parentheses}\\
		x^2 - x - 2 + x + 1~~ &  & \text{Simplify; remove the parentheses}\\
		x^2 - x + x -2 + 1~~ &  & \text{Combine like terms}\\
    (f + g)(x)=x^2 - 1~~~~~~~~~~ &  & \text{Our solution}\\
    ~~~~~~~~~~=(x-1)(x+1) &  & \text{Our solution in factored form}
  \end{eqnarray*}
	
Generally, either form (expanded or factored) would be considered acceptable.

{\bf Example:} Find $g-f$, where $f(x) = x^2 - x - 2$ and $g(x) = x + 1$.
  \begin{eqnarray*}
    (g - f)(x)~~~~~~~~~~ &  & \text{Consider the problem} \\
		g(x) - f(x)~~~~~~~~~ &  & \text{Rewrite as a difference of two functions}\\
		(x + 1)-(x^2 - x - 2)  &  & \text{Substitute functions, inserting parentheses}\\
		x + 1 -x^2 + x + 2~~   &  & \text{Simplify; distribute the negative sign}\\
		-x^2 + x + x +1 + 2~~ &  & \text{Combine like terms}\\
(g - f)(x)=-x^2 +2x +3~~~ &  & \text{Our solution}\\
    ~~~~~~~~~~=-(x-3)(x+1)&  & \text{Our solution in factored form}
 \end{eqnarray*}

{\bf Example:} Find $h \cdot k$, where $h(x) = 3x^2 - 4x$ and $k(x) = x - 2$.
 \begin{eqnarray*}
    (h \cdot k)(x)~~~~~~~~~~ &  & \text{Consider the problem} \\
		h(x)\cdot k(x)~~~~~~~~~ &  & \text{Rewrite as a product of two functions}\\
    (3x^2 - 4x)(x - 2)~~  &  & \text{Substitute functions, inserting parentheses}\\
		3x^3  -6x^2 -4x^2 +8x &  & \text{Expand by distributing}\\
		3x^3  -10x^2 + 8x~~~ &  & \text{Combine like terms}\\
    (h \cdot k)(x)=3x^3  -10x^2 + 8x &  & \text{Our solution}\\
    ~~~~~~=x(3x-4)(x-2)&  & \text{Our solution in factored form}
\end{eqnarray*}

{\bf Example:} Find $\dfrac{g}{f}$, where $f(x) = x^2 - x - 2$ and $g(x) = x + 1$.
 \begin{eqnarray*}
    \left(\dfrac{g}{f}\right)(x)~~~~~~~~ &  & \text{Consider the problem} \\
		\dfrac{g(x)}{f(x)}~~~~~~~~~ &  & \text{Rewrite as a quotient of two functions}\\
    \dfrac{x + 1}{x^2 - x - 2}~~~~~  &  & \text{Substitute functions, parentheses unnecessary}\\
  	 \dfrac{x+1}{(x+1)(x-2)}~~~&  & \text{Factor (if possible)}\\
  & & \\ 
		x \neq -1 ~~\text{~and~}~~ x \neq 2 &  & \text{Restrict denominator:} \ g(x)\neq 0\\
  & & \\ 
	\dfrac{\cancel{x + 1}}{\cancel{(x+1)}(x-2)}~~~  & & \text{Simplify: reduce $\dfrac{x+1}{x+1}$}\\
	\left(\dfrac{g}{f}\right)(x)=\dfrac{1}{x-2}, ~~ x \neq -1  & & \text{Our solution with added restriction}
	\end{eqnarray*}
\newpage
{\bf II - Demo/Discussion Problems:}\\
\ \par
Use $h (x) = 2 x - 4$ and $k (x) = - 3 x + 1$ to find each of the following values for $x=3$ in two ways:
\begin{enumerate}
	\item[a.] Evaluate both $h$ and $k$ at $x=3,$ then combine and simplify the two values accordingly.
	\item[b.] Find a simplified expression for the desired combined function, then evaluate it at $x=3$.
\end{enumerate}
\begin{multicols}{3}
\begin{enumerate}
		\item $(h+k)(3)$
		\item $(h\cdot k)(3)$\\
		\item $(h-k)(3)$
		\item $(k-h)(3)$\\
		\item $\left(\dfrac{h}{k}\right)(3)$
		\item $\left(\dfrac{k}{h}\right)(3)$
\end{enumerate}
\end{multicols}
	{\bf III - Practice Problems:}\\
\ \par
In each problem, use the pair of functions $f$ and $g$ to find the following values, if they exist.

\begin{multicols}{3}
\begin{itemize}

\item  $(f+g)(2)$ 
\item  $(f-g)(-1)$
\item  $(g-f)(1)$

\end{itemize}
\end{multicols}

\begin{multicols}{3}
\begin{itemize}

\item  $(fg)\left(\dfrac{1}{2}\right)$
\item  $\left(\dfrac{f}{g}\right)(0)$
\item  $\left(\dfrac{g}{f}\right)\left(-2\right)$

\end{itemize}
\end{multicols}

\begin{multicols}{2}
\begin{enumerate}
\item  $f(x) = 3x+1$~~~  $g(x) = 4-x$
\item  $f(x) = x^2$ ~~~ $g(x) = -2x+1$
\item  $f(x) = x^2 - x$ ~~~  $g(x) = 12-x^2$
\item  $f(x) = 2x^3$ ~ \mbox{$g(x) = -x^2-2x-3$}
\item  $f(x) = \sqrt{x+3}$ ~~~  $g(x) = 2x-1$
\item  $f(x) = \sqrt{4-x}$ ~ $g(x) = \sqrt{x+2}$
\item  $f(x) = 2x$ ~~~  $g(x) = \dfrac{1}{2x+1}$
\item  $f(x) = x^2$ ~~~ $g(x) = \dfrac{3}{2x-3}$
\item  $f(x) = x^2$ ~~~  $g(x) = \dfrac{1}{x^2}$
\item  $f(x) = x^2+1$ ~~~ $g(x) = \dfrac{1}{x^2+1}$
\end{enumerate}
\end{multicols}

In each problem, use the pair of functions $f$ and $g$ to find the domain of the indicated function then find and simplify an expression for it.

\begin{multicols}{4}
\begin{itemize}

\item  $(f+g)(x)$
\item  $(f-g)(x)$
\item  $(fg)(x)$
\item  $\left(\dfrac{f}{g}\right)(x)$

\end{itemize}
\end{multicols}

\begin{multicols}{2}
\begin{enumerate}
\setcounter{enumi}{10}
\item $f(x) = 2x+1$ ~~~ $g(x) = x-2$
\item $f(x) = 1-4x$ ~~~ $g(x) = 2x-1$
\item $f(x) = x^2$ ~~~ $g(x) = 3x-1$
\item $f(x) = x^2-x$ ~~~ $g(x) = 7x$
\item $f(x) = x^2-4$ ~~~ $g(x) = 3x+6$
\item $f(x) = -x^2+x+6$ ~ $g(x) = x^2-9$
\item $f(x) = \dfrac{x}{2}$ ~~~ $g(x) = \dfrac{2}{x}$
\item $f(x) =x-1$ ~~~ $g(x) = \dfrac{1}{x-1}$
\item $f(x) = x$ ~~~ $g(x) = \sqrt{x+1}$
\item $f(x) = g(x) = \sqrt{x-5}$
\end{enumerate}
\end{multicols}
\newpage
\ \newpage
\end{document}