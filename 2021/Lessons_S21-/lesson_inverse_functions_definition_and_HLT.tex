\documentclass[12pt]{article}
\usepackage[top=1in,left=1in,bottom=1in,right=1in,headsep=2pt]{geometry}	
\usepackage{amssymb,amsmath,amsthm,amsfonts}
\usepackage{chapterfolder,docmute,setspace}
\usepackage{cancel,multicol,tikz,verbatim,framed,polynom,enumitem}
\usepackage[colorlinks, hyperindex, plainpages=false, linkcolor=blue, urlcolor=blue, pdfpagelabels]{hyperref}
\usepackage[metapost,truebbox]{mfpic}
% Use the cc-by-nc-sa license for any content linked with Stitz and Zeager's text.  Otherwise, use the cc-by-sa license.
%\usepackage[type={CC},modifier={by-sa},version={4.0},]{doclicense}
\usepackage[type={CC},modifier={by-nc-sa},version={4.0},]{doclicense}

\theoremstyle{definition}
\newtheorem{example}{Example}
\newcommand{\Desmos}{\href{https://www.desmos.com/}{Desmos}}
\setlength{\parindent}{0em}
\setlist{itemsep=0em}
\setlength{\parskip}{0.1em}
% This document is used for ordering of lessons.  If an instructor wishes to change the ordering of assessments, the following steps must be taken:

% 1) Reassign the appropriate numbers for each lesson in the \setcounter commands included in this file.
% 2) Rearrange the \include commands in the master file (the file with 'Course Pack' in the name) to accurately reflect the changes.  
% 3) Rearrange the \items in the measureable_outcomes file to accurately reflect the changes.  Be mindful of page breaks when moving items.
% 4) Re-build all affected files (master file, measureable_outcomes file, and any lesson whose numbering has changed).

%Note: The placement of each \newcounter and \setcounter command reflects the original/default ordering of topics (linears, systems, quadratics, functions, polynomials, rationals).

\newcounter{lesson_solving_linear_equations}
\newcounter{lesson_equations_containing_absolute_values}
\newcounter{lesson_graphing_lines}
\newcounter{lesson_two_forms_of_a_linear_equation}
\newcounter{lesson_parallel_and_perpendicular_lines}
\newcounter{lesson_linear_inequalities}
\newcounter{lesson_compound_inequalities}
\newcounter{lesson_inequalities_containing_absolute_values}
\newcounter{lesson_graphing_systems}
\newcounter{lesson_substitution}
\newcounter{lesson_elimination}
\newcounter{lesson_quadratics_introduction}
\newcounter{lesson_factoring_GCF}
\newcounter{lesson_factoring_grouping}
\newcounter{lesson_factoring_trinomials_a_is_1}
\newcounter{lesson_factoring_trinomials_a_neq_1}
\newcounter{lesson_solving_by_factoring}
\newcounter{lesson_square_roots}
\newcounter{lesson_i_and_complex_numbers}
\newcounter{lesson_vertex_form_and_graphing}
\newcounter{lesson_solve_by_square_roots}
\newcounter{lesson_extracting_square_roots}
\newcounter{lesson_the_discriminant}
\newcounter{lesson_the_quadratic_formula}
\newcounter{lesson_quadratic_inequalities}
\newcounter{lesson_functions_and_relations}
\newcounter{lesson_evaluating_functions}
\newcounter{lesson_finding_domain_and_range_graphically}
\newcounter{lesson_fundamental_functions}
\newcounter{lesson_finding_domain_algebraically}
\newcounter{lesson_solving_functions}
\newcounter{lesson_function_arithmetic}
\newcounter{lesson_composite_functions}
\newcounter{lesson_inverse_functions_definition_and_HLT}
\newcounter{lesson_finding_an_inverse_function}
\newcounter{lesson_transformations_translations}
\newcounter{lesson_transformations_reflections}
\newcounter{lesson_transformations_scalings}
\newcounter{lesson_transformations_summary}
\newcounter{lesson_piecewise_functions}
\newcounter{lesson_functions_containing_absolute_values}
\newcounter{lesson_absolute_as_piecewise}
\newcounter{lesson_polynomials_introduction}
\newcounter{lesson_sign_diagrams_polynomials}
\newcounter{lesson_factoring_quadratic_type}
\newcounter{lesson_factoring_summary}
\newcounter{lesson_polynomial_division}
\newcounter{lesson_synthetic_division}
\newcounter{lesson_end_behavior_polynomials}
\newcounter{lesson_local_behavior_polynomials}
\newcounter{lesson_rational_root_theorem}
\newcounter{lesson_polynomials_graphing_summary}
\newcounter{lesson_polynomial_inequalities}
\newcounter{lesson_rationals_introduction_and_terminology}
\newcounter{lesson_sign_diagrams_rationals}
\newcounter{lesson_horizontal_asymptotes}
\newcounter{lesson_slant_and_curvilinear_asymptotes}
\newcounter{lesson_vertical_asymptotes}
\newcounter{lesson_holes}
\newcounter{lesson_rationals_graphing_summary}

\setcounter{lesson_solving_linear_equations}{1}
\setcounter{lesson_equations_containing_absolute_values}{2}
\setcounter{lesson_graphing_lines}{3}
\setcounter{lesson_two_forms_of_a_linear_equation}{4}
\setcounter{lesson_parallel_and_perpendicular_lines}{5}
\setcounter{lesson_linear_inequalities}{6}
\setcounter{lesson_compound_inequalities}{7}
\setcounter{lesson_inequalities_containing_absolute_values}{8}
\setcounter{lesson_graphing_systems}{9}
\setcounter{lesson_substitution}{10}
\setcounter{lesson_elimination}{11}
\setcounter{lesson_quadratics_introduction}{16}
\setcounter{lesson_factoring_GCF}{17}
\setcounter{lesson_factoring_grouping}{18}
\setcounter{lesson_factoring_trinomials_a_is_1}{19}
\setcounter{lesson_factoring_trinomials_a_neq_1}{20}
\setcounter{lesson_solving_by_factoring}{21}
\setcounter{lesson_square_roots}{22}
\setcounter{lesson_i_and_complex_numbers}{23}
\setcounter{lesson_vertex_form_and_graphing}{24}
\setcounter{lesson_solve_by_square_roots}{25}
\setcounter{lesson_extracting_square_roots}{26}
\setcounter{lesson_the_discriminant}{27}
\setcounter{lesson_the_quadratic_formula}{28}
\setcounter{lesson_quadratic_inequalities}{29}
\setcounter{lesson_functions_and_relations}{12}
\setcounter{lesson_evaluating_functions}{13}
\setcounter{lesson_finding_domain_and_range_graphically}{14}
\setcounter{lesson_fundamental_functions}{15}
\setcounter{lesson_finding_domain_algebraically}{30}
\setcounter{lesson_solving_functions}{31}
\setcounter{lesson_function_arithmetic}{32}
\setcounter{lesson_composite_functions}{33}
\setcounter{lesson_inverse_functions_definition_and_HLT}{34}
\setcounter{lesson_finding_an_inverse_function}{35}
\setcounter{lesson_transformations_translations}{36}
\setcounter{lesson_transformations_reflections}{37}
\setcounter{lesson_transformations_scalings}{38}
\setcounter{lesson_transformations_summary}{39}
\setcounter{lesson_piecewise_functions}{40}
\setcounter{lesson_functions_containing_absolute_values}{41}
\setcounter{lesson_absolute_as_piecewise}{42}
\setcounter{lesson_polynomials_introduction}{43}
\setcounter{lesson_sign_diagrams_polynomials}{44}
\setcounter{lesson_factoring_quadratic_type}{46}
\setcounter{lesson_factoring_summary}{45}
\setcounter{lesson_polynomial_division}{47}
\setcounter{lesson_synthetic_division}{48}
\setcounter{lesson_end_behavior_polynomials}{49}
\setcounter{lesson_local_behavior_polynomials}{50}
\setcounter{lesson_rational_root_theorem}{51}
\setcounter{lesson_polynomials_graphing_summary}{52}
\setcounter{lesson_polynomial_inequalities}{53}
\setcounter{lesson_rationals_introduction_and_terminology}{54}
\setcounter{lesson_sign_diagrams_rationals}{55}
\setcounter{lesson_horizontal_asymptotes}{56}
\setcounter{lesson_slant_and_curvilinear_asymptotes}{57}
\setcounter{lesson_vertical_asymptotes}{58}
\setcounter{lesson_holes}{59}
\setcounter{lesson_rationals_graphing_summary}{60}

\begin{document}
{\bf \large Lesson \arabic{lesson_inverse_functions_definition_and_HLT}
: Inverse Functions - Definition and the HLT}
%\\ CC attribute: \href{http://www.wallace.ccfaculty.org/book/book.html}{\it{Beginning and Intermediate Algebra}} by T. Wallace. 
\\ CC attribute: \href{http://www.stitz-zeager.com}{\it{College Algebra}} by C. Stitz and J. Zeager. 
\hfill \doclicenseImage[imagewidth=5em]\\
\par
{\bf Objective:} 	Understand the definition of an inverse function and graphical implications.  Determine whether a function is invertible.\\  
\par
{\bf Students will be able to:}
\begin{itemize}
	\item Define an inverse function.
	\item Obtain the graph of an inverse function from the graph of a function.
	\item Use the Horizontal Line Test (HLT) to determine if a function is invertible.
\end{itemize}
{\bf Prerequisite Knowledge:}
\begin{itemize}
	\item Plotting points on the Cartesian plane.
	\item Graphing horizontal lines.
	\item Graph functions by creating a table.
\end{itemize}
\hrulefill

{\bf Lesson:}\\
\ \par
One often considers the operations of addition and subtraction to be ``opposites'' of one another, and similarly for multiplication and division.  The reason for this, naturally, is because each of these operations ``undoes'' the other.  In mathematics, since the term ``opposite'' can take on different meanings, we instead consider addition and subtraction (or multiplication and division) to be {\it inverse operations} of one another.  This notion of an inverse can be applied to entire functions, which we will now discuss.\\
\ \par
We start by analyzing a very basic function which is reversible, a linear function.  Consider the function $f(x) = 3x+4$.  Thinking of $f$ as a process, we start with an input $x$ and apply two steps, in order: 

\begin{enumerate}
	\item multiply by $3$ 
	\item add $4$. 
\end{enumerate}

To reverse this process, we seek a function $g$ which will undo each of these steps, by taking the output from $f$, $3x+4$, and returning the original input $x$.  If we think of the real-world reversible two-step process of first putting on socks then putting on shoes, to reverse the process, we first take off the shoes, and then we take off the socks.  In much the same way, the function $g$ should undo the last step of $f$ first.  That is, the function $g$ should:

\begin{enumerate}
	\item subtract  $4$, then 
	\item divide by $3$. 
\end{enumerate}

Following this procedure, we get $g(x) = \dfrac{x-4}{3}$.\\
\ \par
Now we can test our function to see if it conceptually agrees with our ``feet, socks, and shoes'' analogy.  Just as in the first part of the process we began with our bare feet and ended up in shoes, the reverse process brings us back, in the end, to our bare feet.  We can see if this holds for $f$ and $g$ by using what we already know about functions.\\
\ \par
For example, if $x=5$, then $$f(5) = 3(5)+4 = 15+4 = 19.$$
Substituting the output $19$ from $f$ as our new input for $g$, we get our original input for $f$.
$$g(19) = \dfrac{19-4}{3} = \dfrac{15}{3} = 5$$

To check that $g$ does this for all $x$ in the domain of $f$ (not just a single value), we will need to find and simplify the composite function $(g\circ f)(x)=g(f(x))$.

$$g(f(x)) = g(3x+4) = \dfrac{(3x+\cancel{4})-\cancel{4}}{3} = \dfrac{\cancel{3}x}{\cancel{3}} = x$$

Not only does $g$ ``undo'' $f$, but $f$ also undoes $g$, which we can verify by once again looking at a composite function.  This time we will find and simplify $(f\circ g)(x)=f(g(x))$.

$$f(g(x)) = f\left(\dfrac{x-4}{3}\right) = \cancel{3} \left(\dfrac{x-4}{\cancel{3}}\right) + 4 = (x-\cancel{4}) + \cancel{4} = x$$

Two functions $f$ and $g$ which are related in this manner are defined to be {\it inverse functions}, or simply {\it inverses}, of each other.  More precisely, using the language of function composition, two functions $f$ and $g$ are said to be inverses if both:

\begin{itemize}
	\item $g(f(x)) = x$ for all $x$ in the domain of $f$, and 
	\item $f(g(x)) = x$ for all $x$ in the domain of $g$.
\end{itemize}

We say that a function $f$ is {\it invertible} if an inverse function of $f$ exists.  If two functions $g$ and $f$ are inverses of each other, then we denote this by $g(x)=f^{-1}(x)$, and similarly $f(x)=g^{-1}(x)$.  This notation can be a bit ``gnarly'' at first, since an inverse function $f^{-1}$ of $f$ must not be confused with the reciprocal function, $1/f$.  The primary difference between these two functions is that a reciprocal function satisfies the property that $$f(x)\cdot (1/f)(x)=1,$$ whereas for inverses, $$(f\circ f^{-1})(x)=x \text{~~~~and~~~~} (f^{-1}\circ f)(x)=x.$$

%Using our function $f(x)=3x+4,$ we can see this distinction.

%\begin{itemize}
%	\item Original Function: $f(x)=3x+4$
%	\item Inverse Function: $f^{-1}(x)=\dfrac{x-4}{3}$
%	\item Reciprocal Function: $\left(\dfrac{1}{f}\right)(x)=\dfrac{1}{3x+4}$
%\end{itemize}

\begin{center}
\framebox{
\begin{minipage}{0.9\linewidth}
\textbf{Properties of Inverse Functions:}\\ Let $f$ and $f^{-1}$ be inverse functions of one another.
\begin{itemize}
\item  The range of $f$ is the domain of $f^{-1}$ and the domain of $f$ is the range of $f^{-1}$.
\item  $f(a) = b$ if and only if $f^{-1}(b) = a$.
\item  The point $(a,b)$ is on the graph of $f$ if and only if the point $(b,a)$ is on the graph of $f^{-1}$.
\end{itemize}
\end{minipage}
}
\end{center}

\begin{center}
\begin{tikzpicture}[xscale=0.75,yscale=0.75]
\draw [<->](-5,0) -- coordinate (x axis mid) (5,0) node[below right] {$x$};
\draw [<->](0,-5) -- coordinate (y axis mid) (0,5) node[above right] {$y$};
\foreach \x in {-4,...,-1}
\draw (\x,1pt) -- (\x,-3pt)
node[anchor=north] {\scriptsize \x};
\foreach \x in {1,...,4}
\draw (\x,1pt) -- (\x,-3pt)
node[anchor=north] {\scriptsize \x};
\foreach \y in {-4,...,-1}
\draw (1pt,\y) -- (-3pt,\y) 
node[anchor=east] {\scriptsize \y}; 
\foreach \y in {1,...,4}
\draw (1pt,\y) -- (-3pt,\y) 
node[anchor=east] {\scriptsize \y}; 
\draw [<->, line width=1.25pt, domain=-2.5:0.25] plot (\x, {3*(\x)+4});
\draw [<->,gray, domain=-3.5:4.75] plot (\x, {(\x-4)/3});
\draw [<->,dotted, domain=-4:4] plot (\x,{\x});
\node at (4,3) {\scriptsize $y=x$};
\node at (1.75,4.5) {\scriptsize $f(x)=3x+4$};
\node at (4.5,1) {\scriptsize $f^{-1}(x)=\dfrac{x-4}{3}$};
\draw[fill] (0,4) circle (0.075);
\draw[fill, gray] (4,0) circle (0.075);
\draw[fill] (-1.33333,0) circle (0.075);
\draw[fill, gray] (0,-1.33333) circle (0.075);
\end{tikzpicture}
\end{center}

Graphically, we can identify one-to-one functions using the following test.

\begin{center}
\framebox{
\begin{minipage}{0.9\linewidth}
\textbf{The Horizontal Line Test (HLT):}\\
A function $f$ is one-to-one if and only if no horizontal line intersects the graph of $f$ more than once.
\label{HLT}
\end{minipage}
}
\end{center}

We say that the graph of a function {\bf passes} the Horizontal Line Test  if no horizontal line intersects the graph more than once; otherwise, we say the graph of the function {\bf fails} the Horizontal Line Test.\\
\ \par
Lastly, we have argued that if $f$ is invertible, then $f$ must be one-to-one, since otherwise the reflection of the graph of $y = f(x)$ about the line $y = x$ will fail the Vertical Line Test.  It turns out that being one-to-one is also enough to guarantee invertibility of a function $f$.  To see this, we can think of $f$ as the set of ordered pairs which constitute its graph.  If switching the $x$- and $y$-coordinates of the points results in a function (i.e., passes the VLT), then $f$ is invertible and we have found the graph of its inverse, $f^{-1}$. This is precisely what the Horizontal Line Test does for us: it checks to see whether or not a set of points describes $x$ as a function of $y$.\\
\ \par
We can now summarize our results.

\begin{center}
\framebox{
	\begin{minipage}{0.9\linewidth}
	\textbf{Equivalent Conditions for Invertibility:}\\
	Suppose $f$ is a function. The following statements are equivalent.
	\begin{itemize}
	\item $f$ is invertible ($f^{-1}$ exists).	
	\item $f$ is one-to-one.
	\item The graph of $f$ passes the Horizontal Line Test.
	\end{itemize}
	\end{minipage}
}
\end{center}
{\bf II - Demo/Discussion Problems:}\\
\ \par
Graph each function.  Use the HLT to determine if the given function is invertible.  If so, graph both the function and its inverse on the same set of axes.  Identify at least three reference points for your function and its inverse.
\begin{multicols}{2}
\begin{enumerate}
	\item $f(x)=4x-3$\\
	\item $g(x)=\frac{1}{2}|x|$\\
	\item $h(x)=(x+1)^2-4$\\
	\item $k(x)=x^3$
	\item $\ell(x)=\sqrt{x}-5$\\
	\item $m(x)=\dfrac{1}{x-2}$\\
	\item $n(x) = x^2 - 10x$, $x \geq 5$\\
	\item $p(x) = 3(x + 4)^{2} - 5, \; x \leq -4$
\end{enumerate}
\end{multicols}
{\bf III - Practice Problems:}\\
\ \par
Graph each function in \Desmos.  Use the HLT to determine if the given function is invertible.  If so, graph both the function and its inverse on the same set of axes.  Identify at least three reference points for your function and its inverse.
\begin{enumerate}
\begin{multicols}{3}
\item $f(x) = 2-6x$
\item $f(x) = \dfrac{x-2}{3} + 4$
\item $f(x)  = 1 - \dfrac{4+3x}{5}$
\item $f(x) = \sqrt{3x-1}+5$\\
\item $f(x) = 2-\sqrt{x - 5}$\\
\item $f(x) = 3\sqrt{x-1}-4$
\item $f(x) = 1 - 2\sqrt{2x+5}$\\
\item $f(x) = \sqrt[3]{3x-1}$\\
\item $f(x) = 3-\sqrt[3]{x-2}$
\end{multicols}
\begin{multicols}{2}
\item $f(x) = x^2-6x+5, \; x \leq 3$
\item $f(x) = 4x^2 + 4x + 1$, $x < -1$
\end{multicols}
\begin{multicols}{3}
\item $f(x) = \dfrac{3}{4-x}$\\
\item $f(x) = \dfrac{x}{1-3x}$
\item $f(x) = \dfrac{2x-1}{3x+4}$\\
\item $f(x) = \dfrac{4x + 2}{3x - 6}$
\item $f(x) = \dfrac{-3x - 2}{x + 3}$\\ 
\item $f(x) = \dfrac{x-2}{2x-1}$
\end{multicols}
\end{enumerate}
\newpage
\end{document}