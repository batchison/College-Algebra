\documentclass[12pt]{article}
\usepackage[top=1in,left=1in,bottom=1in,right=1in,headsep=2pt]{geometry}	
\usepackage{amssymb,amsmath,amsthm,amsfonts}
\usepackage{chapterfolder,docmute,setspace}
\usepackage{cancel,multicol,tikz,verbatim,framed,polynom,enumitem}
\usepackage[colorlinks, hyperindex, plainpages=false, linkcolor=blue, urlcolor=blue, pdfpagelabels]{hyperref}
% Use the cc-by-nc-sa license for any content linked with Stitz and Zeager's text.  Otherwise, use the cc-by-sa license.
%\usepackage[type={CC},modifier={by-sa},version={4.0},]{doclicense}
\usepackage[type={CC},modifier={by-nc-sa},version={4.0},]{doclicense}

\theoremstyle{definition}
\newtheorem{example}{Example}
\newcommand{\Desmos}{\href{https://www.desmos.com/}{Desmos}}
\setlength{\parindent}{0em}
\setlist{itemsep=0em}
\setlength{\parskip}{0.1em}
% This document is used for ordering of lessons.  If an instructor wishes to change the ordering of assessments, the following steps must be taken:

% 1) Reassign the appropriate numbers for each lesson in the \setcounter commands included in this file.
% 2) Rearrange the \include commands in the master file (the file with 'Course Pack' in the name) to accurately reflect the changes.  
% 3) Rearrange the \items in the measureable_outcomes file to accurately reflect the changes.  Be mindful of page breaks when moving items.
% 4) Re-build all affected files (master file, measureable_outcomes file, and any lesson whose numbering has changed).

%Note: The placement of each \newcounter and \setcounter command reflects the original/default ordering of topics (linears, systems, quadratics, functions, polynomials, rationals).

\newcounter{lesson_solving_linear_equations}
\newcounter{lesson_equations_containing_absolute_values}
\newcounter{lesson_graphing_lines}
\newcounter{lesson_two_forms_of_a_linear_equation}
\newcounter{lesson_parallel_and_perpendicular_lines}
\newcounter{lesson_linear_inequalities}
\newcounter{lesson_compound_inequalities}
\newcounter{lesson_inequalities_containing_absolute_values}
\newcounter{lesson_graphing_systems}
\newcounter{lesson_substitution}
\newcounter{lesson_elimination}
\newcounter{lesson_quadratics_introduction}
\newcounter{lesson_factoring_GCF}
\newcounter{lesson_factoring_grouping}
\newcounter{lesson_factoring_trinomials_a_is_1}
\newcounter{lesson_factoring_trinomials_a_neq_1}
\newcounter{lesson_solving_by_factoring}
\newcounter{lesson_square_roots}
\newcounter{lesson_i_and_complex_numbers}
\newcounter{lesson_vertex_form_and_graphing}
\newcounter{lesson_solve_by_square_roots}
\newcounter{lesson_extracting_square_roots}
\newcounter{lesson_the_discriminant}
\newcounter{lesson_the_quadratic_formula}
\newcounter{lesson_quadratic_inequalities}
\newcounter{lesson_functions_and_relations}
\newcounter{lesson_evaluating_functions}
\newcounter{lesson_finding_domain_and_range_graphically}
\newcounter{lesson_fundamental_functions}
\newcounter{lesson_finding_domain_algebraically}
\newcounter{lesson_solving_functions}
\newcounter{lesson_function_arithmetic}
\newcounter{lesson_composite_functions}
\newcounter{lesson_inverse_functions_definition_and_HLT}
\newcounter{lesson_finding_an_inverse_function}
\newcounter{lesson_transformations_translations}
\newcounter{lesson_transformations_reflections}
\newcounter{lesson_transformations_scalings}
\newcounter{lesson_transformations_summary}
\newcounter{lesson_piecewise_functions}
\newcounter{lesson_functions_containing_absolute_values}
\newcounter{lesson_absolute_as_piecewise}
\newcounter{lesson_polynomials_introduction}
\newcounter{lesson_sign_diagrams_polynomials}
\newcounter{lesson_factoring_quadratic_type}
\newcounter{lesson_factoring_summary}
\newcounter{lesson_polynomial_division}
\newcounter{lesson_synthetic_division}
\newcounter{lesson_end_behavior_polynomials}
\newcounter{lesson_local_behavior_polynomials}
\newcounter{lesson_rational_root_theorem}
\newcounter{lesson_polynomials_graphing_summary}
\newcounter{lesson_polynomial_inequalities}
\newcounter{lesson_rationals_introduction_and_terminology}
\newcounter{lesson_sign_diagrams_rationals}
\newcounter{lesson_horizontal_asymptotes}
\newcounter{lesson_slant_and_curvilinear_asymptotes}
\newcounter{lesson_vertical_asymptotes}
\newcounter{lesson_holes}
\newcounter{lesson_rationals_graphing_summary}

\setcounter{lesson_solving_linear_equations}{1}
\setcounter{lesson_equations_containing_absolute_values}{2}
\setcounter{lesson_graphing_lines}{3}
\setcounter{lesson_two_forms_of_a_linear_equation}{4}
\setcounter{lesson_parallel_and_perpendicular_lines}{5}
\setcounter{lesson_linear_inequalities}{6}
\setcounter{lesson_compound_inequalities}{7}
\setcounter{lesson_inequalities_containing_absolute_values}{8}
\setcounter{lesson_graphing_systems}{9}
\setcounter{lesson_substitution}{10}
\setcounter{lesson_elimination}{11}
\setcounter{lesson_quadratics_introduction}{16}
\setcounter{lesson_factoring_GCF}{17}
\setcounter{lesson_factoring_grouping}{18}
\setcounter{lesson_factoring_trinomials_a_is_1}{19}
\setcounter{lesson_factoring_trinomials_a_neq_1}{20}
\setcounter{lesson_solving_by_factoring}{21}
\setcounter{lesson_square_roots}{22}
\setcounter{lesson_i_and_complex_numbers}{23}
\setcounter{lesson_vertex_form_and_graphing}{24}
\setcounter{lesson_solve_by_square_roots}{25}
\setcounter{lesson_extracting_square_roots}{26}
\setcounter{lesson_the_discriminant}{27}
\setcounter{lesson_the_quadratic_formula}{28}
\setcounter{lesson_quadratic_inequalities}{29}
\setcounter{lesson_functions_and_relations}{12}
\setcounter{lesson_evaluating_functions}{13}
\setcounter{lesson_finding_domain_and_range_graphically}{14}
\setcounter{lesson_fundamental_functions}{15}
\setcounter{lesson_finding_domain_algebraically}{30}
\setcounter{lesson_solving_functions}{31}
\setcounter{lesson_function_arithmetic}{32}
\setcounter{lesson_composite_functions}{33}
\setcounter{lesson_inverse_functions_definition_and_HLT}{34}
\setcounter{lesson_finding_an_inverse_function}{35}
\setcounter{lesson_transformations_translations}{36}
\setcounter{lesson_transformations_reflections}{37}
\setcounter{lesson_transformations_scalings}{38}
\setcounter{lesson_transformations_summary}{39}
\setcounter{lesson_piecewise_functions}{40}
\setcounter{lesson_functions_containing_absolute_values}{41}
\setcounter{lesson_absolute_as_piecewise}{42}
\setcounter{lesson_polynomials_introduction}{43}
\setcounter{lesson_sign_diagrams_polynomials}{44}
\setcounter{lesson_factoring_quadratic_type}{46}
\setcounter{lesson_factoring_summary}{45}
\setcounter{lesson_polynomial_division}{47}
\setcounter{lesson_synthetic_division}{48}
\setcounter{lesson_end_behavior_polynomials}{49}
\setcounter{lesson_local_behavior_polynomials}{50}
\setcounter{lesson_rational_root_theorem}{51}
\setcounter{lesson_polynomials_graphing_summary}{52}
\setcounter{lesson_polynomial_inequalities}{53}
\setcounter{lesson_rationals_introduction_and_terminology}{54}
\setcounter{lesson_sign_diagrams_rationals}{55}
\setcounter{lesson_horizontal_asymptotes}{56}
\setcounter{lesson_slant_and_curvilinear_asymptotes}{57}
\setcounter{lesson_vertical_asymptotes}{58}
\setcounter{lesson_holes}{59}
\setcounter{lesson_rationals_graphing_summary}{60}

\begin{document}
{\bf \large Lesson \arabic{lesson_polynomial_division}: Polynomial Division}
%\\ CC attribute: \href{http://www.wallace.ccfaculty.org/book/book.html}{\it{Beginning and Intermediate Algebra}} by T. Wallace. 
\\ CC attribute: \href{http://www.stitz-zeager.com}{\it{College Algebra}} by C. Stitz and J. Zeager. 
\hfill \doclicenseImage[imagewidth=5em]\\
\par
{\bf Objective:} Apply polynomial division.\\
\par
{\bf Students will be able to:}
\begin{itemize}
	\item Divide polynomials of varying degrees.  
	\item Correctly label a divisor, dividend, quotient, and remainder in a polynomial division equation.
\end{itemize}
{\bf Prerequisite Knowledge:}
\begin{itemize}
	\item Polynomial definition and terminology.
	\item Combining like terms.
	\item Distributive property.
	\item Properties of exponents.
\end{itemize}
\hrulefill

{\bf Lesson:}\\
\ \par
Let's recall the terminology and format associated with division.
$$\dfrac{\text{dividend}}{\text{divisor}}=\text{quotient}+\dfrac{\text{remainder}}{\text{divisor}}$$
Alternatively, multiplying both sides of the above equation by the divisor, we have the following.
$$\dfrac{\text{dividend}}{\cancel{\text{divisor}}}\cdot\cancel{\text{divisor}}=\text{quotient}\cdot\text{divisor}+\dfrac{\text{remainder}}{\cancel{\text{divisor}}}\cdot\cancel{\text{divisor}}$$
$$\text{dividend}=\text{quotient}\cdot\text{divisor}+\text{remainder}$$
The general process for division of polynomials follows closely with that for dividing integers.\\
\ \par
\framebox{
\begin{minipage}{0.9\linewidth}
\begin{center}
{\bf General Steps for Polynomial (Long) Division}
\end{center}
Let $D(x)$ and $d(x)$ represent two nonzero polynomial functions.  The steps for simplifying the rational expression $\dfrac{D(x)}{d(x)}$ are as follows.
\begin{enumerate}
  \item Divide the leading term of the dividend $D$ by the leading term of the divisor $d$.  Label the resulting term $a_nx^n,$ and write it above the dividend.  This will be the leading term of the quotient, $q(x)$.
  \item Multiply $a_nx^n$ by the divisor, distribute, and simplify.  Label this as $d_1(x)$ and write it directly below the dividend, $D,$ making sure to align terms according to exponents.
  \item Subtract the resulting terms from the dividend.  Label the new expression $D_1$.
\end{enumerate}
\end{minipage}
}
\par
\framebox{
\begin{minipage}{0.9\linewidth}
\begin{enumerate}
  \item[4.] Repeat steps (1)-(3) for the divisor $d$ and the new expression $D_i$ until the degree of $D_i$ is \textit{less than} the degree of the divisor.  Relabel the final new dividend as the remainder, $r(x)$.  The entire polynomial expression appearing above the original dividend is the quotient, $q(x)$.
\end{enumerate}
\begin{center}
\begin{tikzpicture}[xscale=1,yscale=1]
	\draw (0.5,0.5) node {$q(x)$};
	\draw (0,0) node {$d(x) \ )\overline{\ \ \ D(x) \ \ \ }$};
	\draw (0.4,-0.5) node {$- \ \underline{\ \ d_1(x)\ \ }$};
	\draw (1,-1) node {$D_1(x)$};
	\draw (0.8,-1.5) node {$- \ \underline{\ \ d_2(x)\ \ }$};
	\draw (1.5,-2) node {$\ddots$};
	\draw (1.8,-2.75) node {$-\underline{\ \ d_i(x) \ \ }$};
	\draw (2.75,-3.25) node {$D_i(x)=r(x)$};
\end{tikzpicture}
\end{center}
Step (3) often tends to pose the greatest challenge for students.  It is important to keep in mind that we are are always subtracting the top term from the bottom term, which is why we must change the signs of the term(s) on the bottom.  In most cases, we will need to utilize the distributive property.
\end{minipage}
}

{\bf I - Motivating Example(s):}\\
\ \par
{\bf Example:} Divide $9x^5+6x^4-18x^3-24x^2$ by $3x^2$.  Simplify and express your answer in the form
$$\frac{D(x)}{d(x)}=q(x)+\dfrac{r(x)}{d(x)}.$$
\begin{multicols}{2}

\polylongdiv{9x^5+6x^4-18x^3-24x^2}{3x^2}

\columnbreak
We set up our division process by first writing the dividend and the divisor in the appropriate locations. Next, we identify the leading term for our quotient, $3x^3$. Multiplying and subtracting produces our new expression, $D_1(x)=6x^4$. While it is perfectly fine to carry down $-18x^3-24x^2,$ it is not necessary until these terms play a role in the subtraction step.
\end{multicols}
  Repeating our division steps gives us the second term in our quotient, $2x^2$.  Multiplying, subtracting, and carrying down the next term gives us our new expression of $D_2(x)=-18x^3$. Repeating our steps again produces the third term in our quotient, $-6x$. Multiplying and subtracting produces another new expression, $D_3(x)=-24x^2$.  Since the degree of $D_3$ equals that of our divisor, $d(x)=3x^2,$ we will need to apply our steps for division one final time.  After our fourth and final round of steps, our new expression produces a remainder of $r(x)=0$.\\
\ \par
We express the results of our division in the required form as follows.
$$\dfrac{9x^5+6x^4-18x^3-24x^2}{3x^2}=3x^3+2x^2-6x-8+\dfrac{0}{3x^2}$$
\ \par
{\bf Example:} Divide $3x^3-5x^2-32x+7$ by $x-4$.  Simplify and express your answer in the form
$$\frac{D(x)}{d(x)}=q(x)+\dfrac{r(x)}{d(x)}.$$

\begin{multicols}{2}
\polylongdiv{3x^3-5x^2-32x+7}{x-4}

\columnbreak

Since our quotient, $3x^2+7x-4,$
is a trinomial, we must apply the steps for division three times.\\
\ \par
In this second example, our remainder is the constant term $r(x)=-9,$ which has one degree less than our linear divisor, $d(x)=x-4$.
\end{multicols}
\ \par
Our answer is $\dfrac{3x^3-5x^2-32x+7}{x-4}=3x^2+7x-4+\dfrac{-9}{x-4}.$\\
\ \par
{\bf II - Demo/Discussion Problems:}\\
\ \par
Use polynomial long division to divide and simplify each of the given expressions.  Express each answer in the form below.
$$\frac{\text{dividend}}{\text{divisor}} \ = \ \text{quotient} \ + \ \frac{\text{remainder}}{\text{divisor}}$$
\begin{enumerate}
	\item $\dfrac{8x^3+4x^2-2x+6}{4x^2}$
	\item $\dfrac{n^2 + 7 n + 15}{n + 4}$
	\item $\dfrac{x^3 - 46 x + 22}{x + 7}$
	\item $\dfrac{6x^3-8x^2+10x+103}{4+2x}$
	\item $\dfrac{2x^3-4x+42}{x+3}$
\end{enumerate}
\newpage
{\bf III - Practice Problems:}\\
\ \par
Use polynomial long division to divide and simplify each of the given expressions.  Express each answer in the form below.
$$\frac{\text{dividend}}{\text{divisor}} \ = \ \text{quotient} \ + \ \frac{\text{remainder}}{\text{divisor}}$$
\begin{multicols}{3}
\begin{enumerate}
  \item $\dfrac{20 x^4 + x^3 + 2 x^2}{4 x^3}$
  \item $\dfrac{5 x^4 + 45 x^3 + 4 x^2}{9 x}$
  \item $\dfrac{20 x^4 + x^3 + 40 x^2}{10 x}$
  \item $\dfrac{3 x^3 + 4 x^2 + 2 x}{8 x}$
  \item $\dfrac{12 x^4 + 24 x^3 + 3 x^2}{6 x}$
  \item $\dfrac{5 x^4 + 16 x^3 + 16 x^2}{4 x}$
  \item $\dfrac{10 x^4 + 50 x^3 + 2 x^2}{10 x^2}$
  \item $\dfrac{3 x^4 + 18 x^3 + 27 x^2}{9 x^2}$
  \item $\dfrac{x^2 - 2 x - 71}{x + 8}$\label{polydiv_one}
  \item $\dfrac{x^2 - 3 x - 53}{x - 9}$
  \item $\dfrac{x^2 + 13 x + 32}{x + 5}$
  \item $\dfrac{x^2 - 10 x + 16}{x - 7}$
  \item $\dfrac{x^2 - 2 x - 89}{x - 10}$
  \item $\dfrac{x^2 + 4 x - 26}{x + 7}$
  \item $\dfrac{x^2 - 4 x - 38}{x - 8}$
  \item $\dfrac{x^2 - 4}{x - 2}$
  \item $\dfrac{x^3 + 15 x^2 + 49 x - 55}{x + 7}$
  \item $\dfrac{x^3 - 26 x - 41}{x + 4}$
  \item $\dfrac{3 x^3 + 9 x^2 - 64 x - 68}{x + 6}$
  \item $\dfrac{9 x^3 + 45 x^2 + 27 x - 5}{9 x + 9}$
  \item $\dfrac{x^3 - x^2 - 16 x + 8}{x - 4}$
  \item $\dfrac{x^2 - 10 x + 22}{x - 4}$
  \item $\dfrac{x^3 - 16 x^2 + 71 x - 56}{x - 8}$
  \item $\dfrac{x^3 - 4 x^2 - 6 x + 4}{x - 1}$
  \item $\dfrac{8 x^3 - 66 x^2 + 12 x + 37}{x - 8}$
  \item $\dfrac{3 x^2 + 9 x - 9}{3 x - 3}$
  \item $\dfrac{2 x^2 - 5 x - 8}{2 x + 3}$
  \item $\dfrac{3 x^2 - 32}{3 x - 9}$
  \item $\dfrac{4 x^2 - 23 x - 38}{4 x + 5}$
  \item $\dfrac{2 x^3 + 21 x^2 + 25 x}{2 x + 3}$
  \item $\dfrac{4 x^3 - 21 x^2 + 6 x + 19}{4 x + 3}$  
  \item $\dfrac{8 x^3 - 57 x^2 + 42}{8 x + 7}$
  \item $\dfrac{2 x^3 + 12 x^2 + 4 x - 37}{2 x + 6}$
  \item $\dfrac{45 x^2 + 56 x + 19}{9 x + 4}$
  \item $\dfrac{10 x^2 - 32 x + 9}{10 x - 2}$
  \item $\dfrac{4 x^2 - x - 1}{4 x + 3}$
  \item $\dfrac{27 x^2 + 87 x + 35}{3x + 8}$
  \item $\dfrac{4 x^2 - 33 x + 28}{4 x - 5}$
  \item $\dfrac{48 x^2 - 70 x + 16}{6 x - 2}$
  \item $\dfrac{12 x^3 + 12 x^2 - 15 x - 4}{2 x + 3}$
  \item $\dfrac{24 x^3 - 38 x^2 + 29 x - 60}{4 x - 7}$\label{polydiv_two}
	\end{enumerate}
\end{multicols}
\newpage
\end{document}