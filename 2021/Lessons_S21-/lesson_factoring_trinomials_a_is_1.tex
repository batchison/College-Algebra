\documentclass[12pt]{article}
\usepackage[top=1in,left=1in,bottom=1in,right=1in,headsep=2pt]{geometry}	
\usepackage{amssymb,amsmath,amsthm,amsfonts}
\usepackage{chapterfolder,docmute,setspace}
\usepackage{cancel,multicol,tikz,verbatim,framed,polynom,enumitem}
\usepackage[colorlinks, hyperindex, plainpages=false, linkcolor=blue, urlcolor=blue, pdfpagelabels]{hyperref}
% Use the cc-by-nc-sa license for any content linked with Stitz and Zeager's text.  Otherwise, use the cc-by-sa license.
\usepackage[type={CC},modifier={by-sa},version={4.0},]{doclicense}
%\usepackage[type={CC},modifier={by-nc-sa},version={4.0},]{doclicense}

\theoremstyle{definition}
\newtheorem{example}{Example}
\newcommand{\Desmos}{\href{https://www.desmos.com/}{Desmos}}
\setlength{\parindent}{0em}
\setlist{itemsep=0em}
\setlength{\parskip}{0.1em}
% This document is used for ordering of lessons.  If an instructor wishes to change the ordering of assessments, the following steps must be taken:

% 1) Reassign the appropriate numbers for each lesson in the \setcounter commands included in this file.
% 2) Rearrange the \include commands in the master file (the file with 'Course Pack' in the name) to accurately reflect the changes.  
% 3) Rearrange the \items in the measureable_outcomes file to accurately reflect the changes.  Be mindful of page breaks when moving items.
% 4) Re-build all affected files (master file, measureable_outcomes file, and any lesson whose numbering has changed).

%Note: The placement of each \newcounter and \setcounter command reflects the original/default ordering of topics (linears, systems, quadratics, functions, polynomials, rationals).

\newcounter{lesson_solving_linear_equations}
\newcounter{lesson_equations_containing_absolute_values}
\newcounter{lesson_graphing_lines}
\newcounter{lesson_two_forms_of_a_linear_equation}
\newcounter{lesson_parallel_and_perpendicular_lines}
\newcounter{lesson_linear_inequalities}
\newcounter{lesson_compound_inequalities}
\newcounter{lesson_inequalities_containing_absolute_values}
\newcounter{lesson_graphing_systems}
\newcounter{lesson_substitution}
\newcounter{lesson_elimination}
\newcounter{lesson_quadratics_introduction}
\newcounter{lesson_factoring_GCF}
\newcounter{lesson_factoring_grouping}
\newcounter{lesson_factoring_trinomials_a_is_1}
\newcounter{lesson_factoring_trinomials_a_neq_1}
\newcounter{lesson_solving_by_factoring}
\newcounter{lesson_square_roots}
\newcounter{lesson_i_and_complex_numbers}
\newcounter{lesson_vertex_form_and_graphing}
\newcounter{lesson_solve_by_square_roots}
\newcounter{lesson_extracting_square_roots}
\newcounter{lesson_the_discriminant}
\newcounter{lesson_the_quadratic_formula}
\newcounter{lesson_quadratic_inequalities}
\newcounter{lesson_functions_and_relations}
\newcounter{lesson_evaluating_functions}
\newcounter{lesson_finding_domain_and_range_graphically}
\newcounter{lesson_fundamental_functions}
\newcounter{lesson_finding_domain_algebraically}
\newcounter{lesson_solving_functions}
\newcounter{lesson_function_arithmetic}
\newcounter{lesson_composite_functions}
\newcounter{lesson_inverse_functions_definition_and_HLT}
\newcounter{lesson_finding_an_inverse_function}
\newcounter{lesson_transformations_translations}
\newcounter{lesson_transformations_reflections}
\newcounter{lesson_transformations_scalings}
\newcounter{lesson_transformations_summary}
\newcounter{lesson_piecewise_functions}
\newcounter{lesson_functions_containing_absolute_values}
\newcounter{lesson_absolute_as_piecewise}
\newcounter{lesson_polynomials_introduction}
\newcounter{lesson_sign_diagrams_polynomials}
\newcounter{lesson_factoring_quadratic_type}
\newcounter{lesson_factoring_summary}
\newcounter{lesson_polynomial_division}
\newcounter{lesson_synthetic_division}
\newcounter{lesson_end_behavior_polynomials}
\newcounter{lesson_local_behavior_polynomials}
\newcounter{lesson_rational_root_theorem}
\newcounter{lesson_polynomials_graphing_summary}
\newcounter{lesson_polynomial_inequalities}
\newcounter{lesson_rationals_introduction_and_terminology}
\newcounter{lesson_sign_diagrams_rationals}
\newcounter{lesson_horizontal_asymptotes}
\newcounter{lesson_slant_and_curvilinear_asymptotes}
\newcounter{lesson_vertical_asymptotes}
\newcounter{lesson_holes}
\newcounter{lesson_rationals_graphing_summary}

\setcounter{lesson_solving_linear_equations}{1}
\setcounter{lesson_equations_containing_absolute_values}{2}
\setcounter{lesson_graphing_lines}{3}
\setcounter{lesson_two_forms_of_a_linear_equation}{4}
\setcounter{lesson_parallel_and_perpendicular_lines}{5}
\setcounter{lesson_linear_inequalities}{6}
\setcounter{lesson_compound_inequalities}{7}
\setcounter{lesson_inequalities_containing_absolute_values}{8}
\setcounter{lesson_graphing_systems}{9}
\setcounter{lesson_substitution}{10}
\setcounter{lesson_elimination}{11}
\setcounter{lesson_quadratics_introduction}{16}
\setcounter{lesson_factoring_GCF}{17}
\setcounter{lesson_factoring_grouping}{18}
\setcounter{lesson_factoring_trinomials_a_is_1}{19}
\setcounter{lesson_factoring_trinomials_a_neq_1}{20}
\setcounter{lesson_solving_by_factoring}{21}
\setcounter{lesson_square_roots}{22}
\setcounter{lesson_i_and_complex_numbers}{23}
\setcounter{lesson_vertex_form_and_graphing}{24}
\setcounter{lesson_solve_by_square_roots}{25}
\setcounter{lesson_extracting_square_roots}{26}
\setcounter{lesson_the_discriminant}{27}
\setcounter{lesson_the_quadratic_formula}{28}
\setcounter{lesson_quadratic_inequalities}{29}
\setcounter{lesson_functions_and_relations}{12}
\setcounter{lesson_evaluating_functions}{13}
\setcounter{lesson_finding_domain_and_range_graphically}{14}
\setcounter{lesson_fundamental_functions}{15}
\setcounter{lesson_finding_domain_algebraically}{30}
\setcounter{lesson_solving_functions}{31}
\setcounter{lesson_function_arithmetic}{32}
\setcounter{lesson_composite_functions}{33}
\setcounter{lesson_inverse_functions_definition_and_HLT}{34}
\setcounter{lesson_finding_an_inverse_function}{35}
\setcounter{lesson_transformations_translations}{36}
\setcounter{lesson_transformations_reflections}{37}
\setcounter{lesson_transformations_scalings}{38}
\setcounter{lesson_transformations_summary}{39}
\setcounter{lesson_piecewise_functions}{40}
\setcounter{lesson_functions_containing_absolute_values}{41}
\setcounter{lesson_absolute_as_piecewise}{42}
\setcounter{lesson_polynomials_introduction}{43}
\setcounter{lesson_sign_diagrams_polynomials}{44}
\setcounter{lesson_factoring_quadratic_type}{46}
\setcounter{lesson_factoring_summary}{45}
\setcounter{lesson_polynomial_division}{47}
\setcounter{lesson_synthetic_division}{48}
\setcounter{lesson_end_behavior_polynomials}{49}
\setcounter{lesson_local_behavior_polynomials}{50}
\setcounter{lesson_rational_root_theorem}{51}
\setcounter{lesson_polynomials_graphing_summary}{52}
\setcounter{lesson_polynomial_inequalities}{53}
\setcounter{lesson_rationals_introduction_and_terminology}{54}
\setcounter{lesson_sign_diagrams_rationals}{55}
\setcounter{lesson_horizontal_asymptotes}{56}
\setcounter{lesson_slant_and_curvilinear_asymptotes}{57}
\setcounter{lesson_vertical_asymptotes}{58}
\setcounter{lesson_holes}{59}
\setcounter{lesson_rationals_graphing_summary}{60}

\begin{document}
{\bf \large Lesson \arabic{lesson_factoring_trinomials_a_is_1}: Factoring Trinomials with a Leading Coefficient of One}\phantomsection\label{les:factoring_trinomials_a_is_1}
\\ CC attribute: \href{http://www.wallace.ccfaculty.org/book/book.html}{\it{Beginning and Intermediate Algebra}} by T. Wallace. 
%\\ CC attribute: \href{http://www.stitz-zeager.com}{\it{College Algebra}} by C. Stitz and J. Zeager. 
\hfill \doclicenseImage[imagewidth=5em]\\
\par
{\bf Objective:} Factor a trinomial with a leading coefficient of one.\\
\par
{\bf Students will be able to:}
\begin{itemize}
	\item Identify two integer values that add to $b$ and multiply to $a\cdot c=c$ in a trinomial expression with ordered coefficients $a,b,$ and $c$.
	\item Multiply binomials to verify the accuracy of a factorization.
	\item Recognize the relationship between factoring and expanding an expression.
\end{itemize}
{\bf Prerequisite Knowledge:}
\begin{itemize}
	\item Identifying a greatest common factor.
	\item Factor by grouping.
	\item Application of the distributive property.
	\item Multiplication and division of algebraic expressions.
\end{itemize}
\hrulefill

{\bf Lesson:}\\
{\bf I - Motivating Example(s):}\\
\ \par
{\bf Example:} Write the expanded form for the given expression.
  \begin{eqnarray*}
    (x + 6) (x - 4) &  & \text{Distribute} \ (x + 6) \ \text{through the second set of parentheses.}\\
    x (x + 6) - 4 (x + 6) &  & \text{Distribute each monomial through parentheses.}\\
    x^2 + 6 x - 4 x - 24 &  & \text{Combine like terms.}\\
    x^2 + 2 x - 24 &  & \text{Our solution.}
  \end{eqnarray*}	

Notice that if we reverse the last three steps of the previous example, the process resembles grouping. This is because it is grouping! In the second-to-last line, the GCF of the first two terms is $x$ and the GCF of the last two terms is $- 4$. In this manner, we will factor trinomials by writing them as a polynomial containing four terms, splitting up the middle term, and then factor by grouping. This is demonstrated in the following example, which is the previous one done in reverse.\\
\ \par
{\bf Example:} Factor the given expression.
  \begin{eqnarray*}
    x^2 + 2 x - 24 &  & \text{Split middle (linear) term into} \ + 6
    x - 4 x,\\
    & & \ \ \text{since} \ 6+ (-4)=2 \ \text{and} \ 6\cdot (-4)=-24. \\
		x^2 + 6 x - 4 x - 24 &  & \text{Grouping: GCF on left is} \ x, \ \text{on right is} \ - 4.\\
    x (x + 6) - 4 (x + 6) &  & (x + 6) \ \text{appears twice, factor out this GCF.}\\
    (x + 6) (x - 4) &  & \text{Our solution.}
  \end{eqnarray*}
{\bf II - Demo/Discussion Problems:}\\
\ \par
Factor each of the given trinomial expressions.
\begin{multicols}{3}
\begin{enumerate}
	\item $x^2+9x+18$
	\item $x^2-4x+3$
	\item $x^2-13x+30$
	\item $x^2+13x-30$
	\item $5x^2-40x-100$
	\item $x^2-9xy+14y^2$
\end{enumerate}
\end{multicols}
{\bf III - Practice Problems:}\\
\ \par
Factor each of the given trinomial expressions.
\begin{multicols}{3}
  \begin{enumerate}
  \item $p^2 + 17 p + 72$
  \item $x^2_{} + x - 72$
  \item $n^2 - 9 n + 8$
  \item $x^2 + x - 30$
  \item $x^2 - 9 x - 10$
  \item $x^2 + 13 x + 40$
  \item $b^2 + 12 b + 32$
  \item $b^2 - 17 b + 70$
  \item $x^2 + 3 x - 70$
  \item $x^2 + 3 x - 18$
  \item $n^2 - 8 n + 15$
  \item $a^2 - 6 a - 27$
  \item $p^2 + 15 p + 54$
  \item $p^2 + 7 p - 30$
  \item $n^2 - 15 n + 56$
  \item $m^2 - 15 m n + 50 n^2$
  \item $u^2 - 8 u v + 15 v^2$
  \item $m^2 - 3 m n - 40 n^2$
  \item $m^2 + 2 m n - 8 n^2$
  \item $x^2 + 10 x y + 16 y^2$
  \item $x^2 - 11 x y + 18 y^2$
  \item $u^2 - 9 u v + 14 v^2$
  \item $x^2 + x y - 12 y^2$
  \item $x^2 + 14 x y + 45 y^2$
  \item $x^2 + 4 x y - 12 y^2$
  \item $4 x^2 + 52 x + 168$
  \item $5 a^2 + 60 a + 100$
  \item $5 n^2 - 45 n + 40$
  \item $6 a^2 + 24 a - 192$
  \item $5 v^2 + 20 v - 25$
  \item $6 x^2_{} + 18 x y + 12 y^2$
  \item $5 m^2 + 35 m n - 90 n^2$
  \item $6 x^2 + 96 x y + 378 y^2$
  \item $6 m^2 - 36 m n - 162 n^2$
	\end{enumerate}
\end{multicols}
\newpage
\end{document}