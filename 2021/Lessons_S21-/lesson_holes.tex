\documentclass[12pt]{article}
\usepackage[top=1in,left=1in,bottom=1in,right=1in,headsep=2pt]{geometry}	
\usepackage{amssymb,amsmath,amsthm,amsfonts}
\usepackage{chapterfolder,docmute,setspace}
\usepackage{cancel,multicol,tikz,verbatim,framed,polynom,enumitem}
\usepackage[colorlinks, hyperindex, plainpages=false, linkcolor=blue, urlcolor=blue, pdfpagelabels]{hyperref}
% Use the cc-by-nc-sa license for any content linked with Stitz and Zeager's text.  Otherwise, use the cc-by-sa license.
%\usepackage[type={CC},modifier={by-sa},version={4.0},]{doclicense}
\usepackage[type={CC},modifier={by-nc-sa},version={4.0},]{doclicense}

\theoremstyle{definition}
\newtheorem{example}{Example}
\newcommand{\Desmos}{\href{https://www.desmos.com/}{Desmos}}
\setlength{\parindent}{0em}
\setlist{itemsep=0em}
\setlength{\parskip}{0.1em}
% This document is used for ordering of lessons.  If an instructor wishes to change the ordering of assessments, the following steps must be taken:

% 1) Reassign the appropriate numbers for each lesson in the \setcounter commands included in this file.
% 2) Rearrange the \include commands in the master file (the file with 'Course Pack' in the name) to accurately reflect the changes.  
% 3) Rearrange the \items in the measureable_outcomes file to accurately reflect the changes.  Be mindful of page breaks when moving items.
% 4) Re-build all affected files (master file, measureable_outcomes file, and any lesson whose numbering has changed).

%Note: The placement of each \newcounter and \setcounter command reflects the original/default ordering of topics (linears, systems, quadratics, functions, polynomials, rationals).

\newcounter{lesson_solving_linear_equations}
\newcounter{lesson_equations_containing_absolute_values}
\newcounter{lesson_graphing_lines}
\newcounter{lesson_two_forms_of_a_linear_equation}
\newcounter{lesson_parallel_and_perpendicular_lines}
\newcounter{lesson_linear_inequalities}
\newcounter{lesson_compound_inequalities}
\newcounter{lesson_inequalities_containing_absolute_values}
\newcounter{lesson_graphing_systems}
\newcounter{lesson_substitution}
\newcounter{lesson_elimination}
\newcounter{lesson_quadratics_introduction}
\newcounter{lesson_factoring_GCF}
\newcounter{lesson_factoring_grouping}
\newcounter{lesson_factoring_trinomials_a_is_1}
\newcounter{lesson_factoring_trinomials_a_neq_1}
\newcounter{lesson_solving_by_factoring}
\newcounter{lesson_square_roots}
\newcounter{lesson_i_and_complex_numbers}
\newcounter{lesson_vertex_form_and_graphing}
\newcounter{lesson_solve_by_square_roots}
\newcounter{lesson_extracting_square_roots}
\newcounter{lesson_the_discriminant}
\newcounter{lesson_the_quadratic_formula}
\newcounter{lesson_quadratic_inequalities}
\newcounter{lesson_functions_and_relations}
\newcounter{lesson_evaluating_functions}
\newcounter{lesson_finding_domain_and_range_graphically}
\newcounter{lesson_fundamental_functions}
\newcounter{lesson_finding_domain_algebraically}
\newcounter{lesson_solving_functions}
\newcounter{lesson_function_arithmetic}
\newcounter{lesson_composite_functions}
\newcounter{lesson_inverse_functions_definition_and_HLT}
\newcounter{lesson_finding_an_inverse_function}
\newcounter{lesson_transformations_translations}
\newcounter{lesson_transformations_reflections}
\newcounter{lesson_transformations_scalings}
\newcounter{lesson_transformations_summary}
\newcounter{lesson_piecewise_functions}
\newcounter{lesson_functions_containing_absolute_values}
\newcounter{lesson_absolute_as_piecewise}
\newcounter{lesson_polynomials_introduction}
\newcounter{lesson_sign_diagrams_polynomials}
\newcounter{lesson_factoring_quadratic_type}
\newcounter{lesson_factoring_summary}
\newcounter{lesson_polynomial_division}
\newcounter{lesson_synthetic_division}
\newcounter{lesson_end_behavior_polynomials}
\newcounter{lesson_local_behavior_polynomials}
\newcounter{lesson_rational_root_theorem}
\newcounter{lesson_polynomials_graphing_summary}
\newcounter{lesson_polynomial_inequalities}
\newcounter{lesson_rationals_introduction_and_terminology}
\newcounter{lesson_sign_diagrams_rationals}
\newcounter{lesson_horizontal_asymptotes}
\newcounter{lesson_slant_and_curvilinear_asymptotes}
\newcounter{lesson_vertical_asymptotes}
\newcounter{lesson_holes}
\newcounter{lesson_rationals_graphing_summary}

\setcounter{lesson_solving_linear_equations}{1}
\setcounter{lesson_equations_containing_absolute_values}{2}
\setcounter{lesson_graphing_lines}{3}
\setcounter{lesson_two_forms_of_a_linear_equation}{4}
\setcounter{lesson_parallel_and_perpendicular_lines}{5}
\setcounter{lesson_linear_inequalities}{6}
\setcounter{lesson_compound_inequalities}{7}
\setcounter{lesson_inequalities_containing_absolute_values}{8}
\setcounter{lesson_graphing_systems}{9}
\setcounter{lesson_substitution}{10}
\setcounter{lesson_elimination}{11}
\setcounter{lesson_quadratics_introduction}{16}
\setcounter{lesson_factoring_GCF}{17}
\setcounter{lesson_factoring_grouping}{18}
\setcounter{lesson_factoring_trinomials_a_is_1}{19}
\setcounter{lesson_factoring_trinomials_a_neq_1}{20}
\setcounter{lesson_solving_by_factoring}{21}
\setcounter{lesson_square_roots}{22}
\setcounter{lesson_i_and_complex_numbers}{23}
\setcounter{lesson_vertex_form_and_graphing}{24}
\setcounter{lesson_solve_by_square_roots}{25}
\setcounter{lesson_extracting_square_roots}{26}
\setcounter{lesson_the_discriminant}{27}
\setcounter{lesson_the_quadratic_formula}{28}
\setcounter{lesson_quadratic_inequalities}{29}
\setcounter{lesson_functions_and_relations}{12}
\setcounter{lesson_evaluating_functions}{13}
\setcounter{lesson_finding_domain_and_range_graphically}{14}
\setcounter{lesson_fundamental_functions}{15}
\setcounter{lesson_finding_domain_algebraically}{30}
\setcounter{lesson_solving_functions}{31}
\setcounter{lesson_function_arithmetic}{32}
\setcounter{lesson_composite_functions}{33}
\setcounter{lesson_inverse_functions_definition_and_HLT}{34}
\setcounter{lesson_finding_an_inverse_function}{35}
\setcounter{lesson_transformations_translations}{36}
\setcounter{lesson_transformations_reflections}{37}
\setcounter{lesson_transformations_scalings}{38}
\setcounter{lesson_transformations_summary}{39}
\setcounter{lesson_piecewise_functions}{40}
\setcounter{lesson_functions_containing_absolute_values}{41}
\setcounter{lesson_absolute_as_piecewise}{42}
\setcounter{lesson_polynomials_introduction}{43}
\setcounter{lesson_sign_diagrams_polynomials}{44}
\setcounter{lesson_factoring_quadratic_type}{46}
\setcounter{lesson_factoring_summary}{45}
\setcounter{lesson_polynomial_division}{47}
\setcounter{lesson_synthetic_division}{48}
\setcounter{lesson_end_behavior_polynomials}{49}
\setcounter{lesson_local_behavior_polynomials}{50}
\setcounter{lesson_rational_root_theorem}{51}
\setcounter{lesson_polynomials_graphing_summary}{52}
\setcounter{lesson_polynomial_inequalities}{53}
\setcounter{lesson_rationals_introduction_and_terminology}{54}
\setcounter{lesson_sign_diagrams_rationals}{55}
\setcounter{lesson_horizontal_asymptotes}{56}
\setcounter{lesson_slant_and_curvilinear_asymptotes}{57}
\setcounter{lesson_vertical_asymptotes}{58}
\setcounter{lesson_holes}{59}
\setcounter{lesson_rationals_graphing_summary}{60}

\begin{document}
{\bf \large Lesson \arabic{lesson_holes}: Holes}\phantomsection\label{les:holes}
%\\ CC attribute: \href{http://www.wallace.ccfaculty.org/book/book.html}{\it{Beginning and Intermediate Algebra}} by T. Wallace. 
\\ CC attribute: \href{http://www.stitz-zeager.com}{\it{College Algebra}} by C. Stitz and J. Zeager. 
\hfill \doclicenseImage[imagewidth=5em]\\
\par
{\bf Objective:} Identify the precise location of one or more holes in the graph of a rational function.\\
\par
{\bf Students will be able to:}
\begin{itemize}
	\item Identify removable discontinuities and their corresponding holes in the graph of a rational function.
\end{itemize}
{\bf Prerequisite Knowledge:}
\begin{itemize}
	\item Evaluating a function.
	\item Factoring.
	\item The Rational Root Theorem.
	\item Polynomial and \slash or Synthetic Division.
 	\item Multiplicative identity \slash inverse.
\end{itemize}
\hrulefill

{\bf Lesson:}\\
\ \par
While vertical asymptotes correspond to infinite discontinuities, a hole corresponds to a {\it removable discontinuity}, since the removal of a single point along a continuous curve creates the hole.\\
\ \par
Suppose that the rational function $f(x)$ has a discontinuity at $x=c,$ i.e., $c$ is not in the domain of $f$.  If $x=c$ is a vertical asymptote of the graph of $f,$ in the last lesson we saw that as $x\rightarrow c,$ $f(x)\rightarrow\pm\infty$.  If $x=c$ represents a hole in the graph of $f,$ however, we will see that as $x\rightarrow c,$ $f(x)\rightarrow k,$ for some real number $k$.  This is the fundamental difference between infinite and removable discontinuities.
\begin{center}
\begin{multicols}{2}
\begin{tikzpicture}[xscale=0.6,yscale=1.2]
	\draw [<->](-4.25,0) -- coordinate (x axis mid) (4.25,0) node[below right] {$x$};
	\draw [-, dashed](0,-0.5) -- coordinate (y axis mid) (0,3.75) node[below] {};
	\draw [<-] plot [domain=0.30:3, samples=100] (\x,{1/\x});
	\draw [->] plot [domain=-3:-0.30, samples=100] (\x,{-1/\x});
	\draw[fill] (1,1) ellipse (0.5mm and 0.25mm);
  \draw[fill] (2,0.5) ellipse (0.5mm and 0.25mm);
	\draw[fill] (0.5,2) ellipse (0.5mm and 0.25mm);
	\draw[fill] (0.333,3) ellipse (0.5mm and 0.25mm);
	\draw[fill] (-1,1) ellipse (0.5mm and 0.25mm);
  \draw[fill] (-2,0.5) ellipse (0.5mm and 0.25mm);
	\draw[fill] (-0.5,2) ellipse (0.5mm and 0.25mm);
	\draw[fill] (-0.333,3) ellipse (0.5mm and 0.25mm);
	\draw (0,-0.75) node {$x=c$};
	\draw (1.7,0.25) node {$c^+\longleftarrow x$};
	\draw (-1.7,0.25) node {$x\longrightarrow c^-$};
	\draw (2,-0.25) node {$x>c$};
	\draw (-2,-0.25) node {$x<c$};
	\draw (1.5,2) node {$f(x)$};
	\draw (1.5,2.3) node {$\uparrow$};
	\draw (1.5,2.6) node {$\infty$};
	\draw (-1.5,2) node {$f(x)$};
	\draw (-1.5,2.3) node {$\uparrow$};
	\draw (-1.5,2.6) node {$\infty$};
	\draw (0,-1.25) node {Infinite Discontinuity};
\end{tikzpicture}

\columnbreak

\begin{tikzpicture}[xscale=0.6,yscale=1.2]
	\draw [<->](-4.25,0) -- coordinate (x axis mid) (4.25,0) node[below right] {$x$};
	\draw [-, dashed](0,-0.5) -- coordinate (y axis mid) (0,1.9) node[below] {};
	\draw [-](-4,-0.25) -- coordinate (y axis mid) (-4,3.75) node[below] {};
	\draw [-, dashed](-3.6,2) -- coordinate (y axis mid) (-0.15,2) node[below] {};
	\draw [-](-4.15,2) -- coordinate (y axis mid) (-3.85,2) node[left] {};
	\draw [<->] plot [domain=-3.75:3.75, samples=100] (\x,{1.5*sin(\x/2.5 r)+2});
	\draw[fill] (1,2.584) ellipse (0.5mm and 0.25mm);
  \draw[fill] (2,3.07) ellipse (0.5mm and 0.25mm);
	\draw[fill] (0.5,2.298) ellipse (0.5mm and 0.25mm);
	\draw[fill] (0.333,2.197) ellipse (0.5mm and 0.25mm);
	\draw[fill] (-1,1.416) ellipse (0.5mm and 0.25mm);
  \draw[fill] (-2,0.924) ellipse (0.5mm and 0.25mm);
	\draw[fill] (-0.5,1.702) ellipse (0.5mm and 0.25mm);
	\draw[fill] (-0.333,1.803) ellipse (0.5mm and 0.25mm);
	\draw[fill, color=white] (0,2) ellipse (1.5mm and 0.75mm);
	\draw[line width=0.1mm] (0,2) ellipse (1.5mm and 0.75mm);
	\draw (0,-0.75) node {$x=c$};
	\draw (1.7,0.25) node {$c^+\longleftarrow x$};
	\draw (-1.7,0.25) node {$x\longrightarrow c^-$};
	\draw (2,-0.25) node {$x>c$};
	\draw (-2,-0.25) node {$x<c$};
	\draw (-4.65,3.1) node {$f(x)$};
	\draw (-4.65,2.5) node {$\downarrow$};
	\draw (-4.65,0.9) node {$f(x)$};
	\draw (-4.65,1.5) node {$\uparrow$};
	\draw (-4.65,2) node {$k$};
	\draw (0,-1.25) node {Removable Discontinuity};
\end{tikzpicture}
\end{multicols}
\end{center}

In the case of the graph of the left, recall that we have the following statements.
$$\text{As} \ x\rightarrow c^+, \ f(x)\rightarrow \infty. \qquad\qquad \text{As} \ x\rightarrow c^-, \ f(x)\rightarrow \infty.$$
Similarly, in the case of the graph on the right, we employ the same idea, using $k^+$ and $k^-$ in order to identify whether or not the graph of $f$ approaches $k$ from {\it above} if $f(x)>k$ and {\it below} if $f(x)<k$.
$$\text{As} \ x\rightarrow c^+, \ f(x)\rightarrow k^+. \qquad\qquad \text{As} \ x\rightarrow c^-, \ f(x)\rightarrow k^-.$$
In virtually all cases, however, it will be sufficient enough to simply state that as $x\rightarrow c, \ f(x)\rightarrow k,$ since further analysis will often prove difficult.\\
\ \par
We now state the requirement for a hole, which, as with vertical asymptotes, depends on both the rational function $f$ and its simplified expression.
\begin{center}
\framebox{
\begin{minipage}{0.8\linewidth}
Let $f(x)$ be a rational function and let $g(x)$ represent the simplified expression for $f$.  If $x=c$ is not in the domain of $f,$ but {\it is} in the domain of $g,$ then the graph of $f$ will have a hole at $(c,g(c))$.
\end{minipage}
}
\end{center}

{\bf II - Demo/Discussion Problems:}\\
\ \par
Identify the coordinates of any holes in the graph of each of the following rational functions.  Use \Desmos \ to verify your answers.
\begin{enumerate}
 \item $f(x)=\dfrac{-2x+4}{4x-8}$
 \item $g(x)=\dfrac{x^2+25}{x^2-10x+25}$
 \item $h(x)=\dfrac{x^2-9x+20}{x^2-3x-10}$
\end{enumerate}
{\bf III - Practice Problems:}\\
\ \par
Identify the coordinates of any holes in the graph of each of the following rational functions.  Use \Desmos \ to verify your answers.
\begin{enumerate}
\begin{multicols}{3}
\item $a(x)=\dfrac{2x+6}{x+3}$
\item $g(x)=\dfrac{x^3-16x}{x^2-4x}$
\item $h(x)=\dfrac{x^2-4x-9}{x+2}$
\item $j(x)=\dfrac{18x^3-4x-1}{x^2-4}$
\item $k(x)=\dfrac{x^2-17x+72}{x-9}$
\item $p(x)=\dfrac{x^2+6x+8}{2x+4}$
\item $q(x)=\dfrac{x^5}{x(x-5)}$
\item $r(x)=\dfrac{x^2-11x+30}{x^2-36}$
\item $t(x)=\dfrac{3x^2-12x+12}{x^2+2x-8}$
\item $v(x)=\dfrac{x^2+2x-8} {3x^2-12x+12}$
\end{multicols}
\end{enumerate}
\newpage
\end{document}