\documentclass[12pt]{article}
\usepackage[top=1in,left=1in,bottom=1in,right=1in,headsep=2pt]{geometry}	
\usepackage{amssymb,amsmath,amsthm,amsfonts}
\usepackage{chapterfolder,docmute,setspace}
\usepackage{cancel,multicol,tikz,verbatim,framed,polynom,enumitem}
\usepackage[colorlinks, hyperindex, plainpages=false, linkcolor=blue, urlcolor=blue, pdfpagelabels]{hyperref}
% Use the cc-by-nc-sa license for any content linked with Stitz and Zeager's text.  Otherwise, use the cc-by-sa license.
\usepackage[type={CC},modifier={by-sa},version={4.0},]{doclicense}
%\usepackage[type={CC},modifier={by-nc-sa},version={4.0},]{doclicense}

\theoremstyle{definition}
\newtheorem{example}{Example}
\newcommand{\Desmos}{\href{https://www.desmos.com/}{Desmos}}
\setlength{\parindent}{0em}
\setlist{itemsep=0em}
\setlength{\parskip}{0.1em}
% This document is used for ordering of lessons.  If an instructor wishes to change the ordering of assessments, the following steps must be taken:

% 1) Reassign the appropriate numbers for each lesson in the \setcounter commands included in this file.
% 2) Rearrange the \include commands in the master file (the file with 'Course Pack' in the name) to accurately reflect the changes.  
% 3) Rearrange the \items in the measureable_outcomes file to accurately reflect the changes.  Be mindful of page breaks when moving items.
% 4) Re-build all affected files (master file, measureable_outcomes file, and any lesson whose numbering has changed).

%Note: The placement of each \newcounter and \setcounter command reflects the original/default ordering of topics (linears, systems, quadratics, functions, polynomials, rationals).

\newcounter{lesson_solving_linear_equations}
\newcounter{lesson_equations_containing_absolute_values}
\newcounter{lesson_graphing_lines}
\newcounter{lesson_two_forms_of_a_linear_equation}
\newcounter{lesson_parallel_and_perpendicular_lines}
\newcounter{lesson_linear_inequalities}
\newcounter{lesson_compound_inequalities}
\newcounter{lesson_inequalities_containing_absolute_values}
\newcounter{lesson_graphing_systems}
\newcounter{lesson_substitution}
\newcounter{lesson_elimination}
\newcounter{lesson_quadratics_introduction}
\newcounter{lesson_factoring_GCF}
\newcounter{lesson_factoring_grouping}
\newcounter{lesson_factoring_trinomials_a_is_1}
\newcounter{lesson_factoring_trinomials_a_neq_1}
\newcounter{lesson_solving_by_factoring}
\newcounter{lesson_square_roots}
\newcounter{lesson_i_and_complex_numbers}
\newcounter{lesson_vertex_form_and_graphing}
\newcounter{lesson_solve_by_square_roots}
\newcounter{lesson_extracting_square_roots}
\newcounter{lesson_the_discriminant}
\newcounter{lesson_the_quadratic_formula}
\newcounter{lesson_quadratic_inequalities}
\newcounter{lesson_functions_and_relations}
\newcounter{lesson_evaluating_functions}
\newcounter{lesson_finding_domain_and_range_graphically}
\newcounter{lesson_fundamental_functions}
\newcounter{lesson_finding_domain_algebraically}
\newcounter{lesson_solving_functions}
\newcounter{lesson_function_arithmetic}
\newcounter{lesson_composite_functions}
\newcounter{lesson_inverse_functions_definition_and_HLT}
\newcounter{lesson_finding_an_inverse_function}
\newcounter{lesson_transformations_translations}
\newcounter{lesson_transformations_reflections}
\newcounter{lesson_transformations_scalings}
\newcounter{lesson_transformations_summary}
\newcounter{lesson_piecewise_functions}
\newcounter{lesson_functions_containing_absolute_values}
\newcounter{lesson_absolute_as_piecewise}
\newcounter{lesson_polynomials_introduction}
\newcounter{lesson_sign_diagrams_polynomials}
\newcounter{lesson_factoring_quadratic_type}
\newcounter{lesson_factoring_summary}
\newcounter{lesson_polynomial_division}
\newcounter{lesson_synthetic_division}
\newcounter{lesson_end_behavior_polynomials}
\newcounter{lesson_local_behavior_polynomials}
\newcounter{lesson_rational_root_theorem}
\newcounter{lesson_polynomials_graphing_summary}
\newcounter{lesson_polynomial_inequalities}
\newcounter{lesson_rationals_introduction_and_terminology}
\newcounter{lesson_sign_diagrams_rationals}
\newcounter{lesson_horizontal_asymptotes}
\newcounter{lesson_slant_and_curvilinear_asymptotes}
\newcounter{lesson_vertical_asymptotes}
\newcounter{lesson_holes}
\newcounter{lesson_rationals_graphing_summary}

\setcounter{lesson_solving_linear_equations}{1}
\setcounter{lesson_equations_containing_absolute_values}{2}
\setcounter{lesson_graphing_lines}{3}
\setcounter{lesson_two_forms_of_a_linear_equation}{4}
\setcounter{lesson_parallel_and_perpendicular_lines}{5}
\setcounter{lesson_linear_inequalities}{6}
\setcounter{lesson_compound_inequalities}{7}
\setcounter{lesson_inequalities_containing_absolute_values}{8}
\setcounter{lesson_graphing_systems}{9}
\setcounter{lesson_substitution}{10}
\setcounter{lesson_elimination}{11}
\setcounter{lesson_quadratics_introduction}{16}
\setcounter{lesson_factoring_GCF}{17}
\setcounter{lesson_factoring_grouping}{18}
\setcounter{lesson_factoring_trinomials_a_is_1}{19}
\setcounter{lesson_factoring_trinomials_a_neq_1}{20}
\setcounter{lesson_solving_by_factoring}{21}
\setcounter{lesson_square_roots}{22}
\setcounter{lesson_i_and_complex_numbers}{23}
\setcounter{lesson_vertex_form_and_graphing}{24}
\setcounter{lesson_solve_by_square_roots}{25}
\setcounter{lesson_extracting_square_roots}{26}
\setcounter{lesson_the_discriminant}{27}
\setcounter{lesson_the_quadratic_formula}{28}
\setcounter{lesson_quadratic_inequalities}{29}
\setcounter{lesson_functions_and_relations}{12}
\setcounter{lesson_evaluating_functions}{13}
\setcounter{lesson_finding_domain_and_range_graphically}{14}
\setcounter{lesson_fundamental_functions}{15}
\setcounter{lesson_finding_domain_algebraically}{30}
\setcounter{lesson_solving_functions}{31}
\setcounter{lesson_function_arithmetic}{32}
\setcounter{lesson_composite_functions}{33}
\setcounter{lesson_inverse_functions_definition_and_HLT}{34}
\setcounter{lesson_finding_an_inverse_function}{35}
\setcounter{lesson_transformations_translations}{36}
\setcounter{lesson_transformations_reflections}{37}
\setcounter{lesson_transformations_scalings}{38}
\setcounter{lesson_transformations_summary}{39}
\setcounter{lesson_piecewise_functions}{40}
\setcounter{lesson_functions_containing_absolute_values}{41}
\setcounter{lesson_absolute_as_piecewise}{42}
\setcounter{lesson_polynomials_introduction}{43}
\setcounter{lesson_sign_diagrams_polynomials}{44}
\setcounter{lesson_factoring_quadratic_type}{46}
\setcounter{lesson_factoring_summary}{45}
\setcounter{lesson_polynomial_division}{47}
\setcounter{lesson_synthetic_division}{48}
\setcounter{lesson_end_behavior_polynomials}{49}
\setcounter{lesson_local_behavior_polynomials}{50}
\setcounter{lesson_rational_root_theorem}{51}
\setcounter{lesson_polynomials_graphing_summary}{52}
\setcounter{lesson_polynomial_inequalities}{53}
\setcounter{lesson_rationals_introduction_and_terminology}{54}
\setcounter{lesson_sign_diagrams_rationals}{55}
\setcounter{lesson_horizontal_asymptotes}{56}
\setcounter{lesson_slant_and_curvilinear_asymptotes}{57}
\setcounter{lesson_vertical_asymptotes}{58}
\setcounter{lesson_holes}{59}
\setcounter{lesson_rationals_graphing_summary}{60}

\begin{document}
{\bf \large Lesson \arabic{lesson_graphing_lines}: Graphing Lines}\phantomsection\label{les:graphing_lines}
\\ CC attribute: \href{http://www.wallace.ccfaculty.org/book/book.html}{\it{Beginning and Intermediate Algebra}} by T. Wallace. 
%\\ CC attribute: \href{http://www.stitz-zeager.com}{\it{College Algebra}} by C. Stitz and J. Zeager. 
\hfill \doclicenseImage[imagewidth=5em]\\
\par
{\bf Objective:} Graph a linear equation by creating a table of values for $x$.  Identify the slope of a linear equation both graphically and algebraically.\\
\par
{\bf Students will be able to:}
\begin{itemize}
	\item Create and populate a table of points for a given equation or graph.
	\item Calculate the slope of a line when given two points.
\end{itemize}
{\bf Prerequisite Knowledge:}
\begin{itemize}
	\item Plotting points on a coordinate plane.
	\item Identifying $x-$ and $y-$coordinates.
	\item Conceptually understanding of slope.
\end{itemize}
\hrulefill

{\bf Lesson:}\\
\ \par
Given a linear equation, such as $y = 2 x - 3$, one may be interested in what solution(s) are possible for a given $x$ or $y$.  We can visualize the set of solutions by making a graph of all possible $x$ and  $y$ combinations, or {\it coordinate pairs}, that satisfy this equation.  Our corresponding graph will be a line, and any point on this line will make the equation $y = 2 x - 3$ true.  We will do this using a table of values.\\
\ \par
Additionally, the slope of a line will be extremely useful for drawing conclusions about a linear equation and/or its graph.
\begin{center}
\framebox{
\begin{minipage}{0.8\linewidth}
Given two points $(x_1,y_1)$ and $(x_2,y_2)$, the slope of the line through these points is defined as follows.
\[ m = \frac{\text{rise}}{\text{run}} = \frac{\text{change in} \ y}{\text{change in} \ x} = \frac{y_2 -
   y_1}{x_2 - x_1} \]
\end{minipage}
}
\end{center}
Whenever we calculate a slope, as we subtract the corresponding $y$ and $x$ coordinates from one another, it is important that we subtract them in the correct order.

\newpage

{\bf I - Motivating Example(s):}\\
\ \par
{\bf Example:} 
\begin{eqnarray*}
    \text{Graph} \ y = 2 x - 3. &  & \text{Make a table of values.  Any test values may be used.}\\
    &  & \\
    \begin{array}{|c|c|}
      \hline
      x & y\\
      \hline
      - 1 & - 5\\
      \hline
      0 & - 3\\
      \hline
      1 & - 1\\
      \hline
    \end{array} &  & \begin{array}{l}
      \text{Evaluate each test value by replacing} \ x \ \text{with the given value.}\\
			x = - 1 ~~~~~~~ y = 2 (- 1) - 3 = - 2 - 3 = - 5\\
      x = 0  ~~~~~~~~~y = 2 (0) - 3 = 0 - 3 = - 3\\
      x = 1  ~~~~~~~~~y = 2 (1) - 3 = 2 - 3 = - 1
    \end{array}\\
    &  & \\
    (- 1, - 5), (0, - 3), (1, - 1) &  & \text{These become our points to graph for our equation}.
  \end{eqnarray*}
  \begin{multicols}{2}
 	\begin{tikzpicture}[xscale=0.5,yscale=0.5]
		\draw[step=1.0,gray,very thin,dotted] (-5.5,-5.5) grid (5.5,5.5);
		\draw [<->](-5.5,0) -- coordinate (x axis mid) (5.5,0) node[below right] {$x$};
		\draw [<->](0,-5.5) -- coordinate (y axis mid) (0,5.5) node[above right] {$y$};
		\foreach \x in {1,...,5}
		\draw (\x,2pt) -- (\x,-2pt)	node[anchor=south] {\scriptsize \x};
		\foreach \x in {-1,...,-5}
		\draw (\x,2pt) -- (\x,-2pt)	node[anchor=south] {\scriptsize \x};
		\foreach \y in {1,...,5}
		\draw (2pt,\y) -- (-2pt,\y)	node[anchor=east] {\scriptsize \y}; 
		\foreach \y in {-1,...,-5}
		\draw[fill] (2pt,\y) -- (-2pt,\y)	node[anchor=east] {\scriptsize \y}; 
		\draw [<->] plot [domain=-1.2:4, samples=100] (\x,{2*\x-3});
		\draw[fill] (1,-1) circle (0.1) node[right] {};
		\draw[fill] (0,-3)circle (0.1) node[right] {};
		\draw[fill] (-1,-5) circle (0.1) node[right] {};
	\end{tikzpicture}

\columnbreak
    \ \par 
    $(- 1, - 5), (0, - 3),$ and $(1, - 1)$\par
     These become the points from our equation which we will plot on our graph.\par    
     Once the point are on the graph, connect the dots to make a line.\par
     The graph is our solution.
  \end{multicols}

\ \par
Notice the graph also goes through the point $(2, 1)$.  This means that the pair $(x,y)=(2,1)$ will also satisfy the equation $y=2x-3,$ which one can easily check.\\
\ \par
Notice also that the slope of the line above is $m=2$ or $\frac{2}{1}$.  We can check this by using any two points from our table.  We will use $(-1,-5)$ and $(0,-3)$.
\[ m=\frac{-5-(-3)}{-1-0}=\frac{-5+3}{-1}=\frac{-2}{-1}=2 \ \checkmark\]
	
{\bf Example:} Find the slope of the line through the given points. 
  \begin{eqnarray*} (- 4, 3) \ \text{and} \ (2,- 9). &  & \text{Identify} \ x_1, y_1, x_2,\ \text{and} \ y_2.\\
    (x_1, y_1) \ \text{and} \ (x_2, y_2) &  & \text{Use the slope formula}, m = \frac{y_2 - y_1}{x_2 - x_1}.\\
    m = \frac{- 9 - 3}{2 - (- 4)} &  & \text{Simplify.}\\
    m = \frac{-12}{6} &  & \text{Reduce.}\\
    m = - 2 &  & \text{Our solution.}
  \end{eqnarray*}
\newpage
{\bf II - Demo/Discussion Problems:}
\begin{enumerate}
\item Make a table of points and use it to graph the linear equation $2x-3y=6$.
\item Find the slope of the line through the points $(-4,-1)$ and $(-4,-5)$.
\item Find the slope of the line through the points $(3,1)$ and $(-2,1)$.
\end{enumerate}

{\bf III - Practice Problems:}\\
\ \par
For each linear equation below, make a table of points and use it to graph the equation.
\begin{multicols}{4}
 \begin{enumerate}
	\item $y = - \frac{1}{4} x - 3$
  \item $y = x - 1$
  \item $y = - \frac{5}{4} x - 4$
  \item $y = - \frac{3}{5} x + 1$
  \item $y = - 4 x + 2$
  \item $y = \frac{5}{3} x + 4$
  \item $y = \frac{3}{2} x - 5$
  \item $y = - x - 2$
  \item $y = - \frac{4}{5} x - 3$
  \item $y = \frac{1}{2} x$
  \item $x + 5 y = - 15$
  \item $8 x - y = 5$
  \item $4 x + y = 5$
  \item $3 x + 4 y = 16$
  \item $2 x - y = 2$
  \item $7 x + 3 y = - 12$
  \item $x + y = - 1$
  \item $3 x + 4 y = 8$
  \item $x - y = - 3$
  \item $9 x - y = - 4$
 \end{enumerate}
\end{multicols}
Find the slope of each of the following lines.
\begin{multicols}{2}
  21.\\
	\begin{tikzpicture}[xscale=0.4,yscale=0.4]
		\draw[step=1.0,gray,very thin,dotted] (-8.5,-5.5) grid (8.5,5.5);
		\draw [<->](-8.5,0) -- coordinate (x axis mid) (8.5,0) node[below right] {$x$};
		\draw [<->](0,-5.5) -- coordinate (y axis mid) (0,5.5) node[above right] {$y$};
		\foreach \x in {1,...,8}
		\draw (\x,2pt) -- (\x,-2pt)
			node[anchor=south] {\scriptsize \x}
		;
		\foreach \x in {-1,...,-8}
		\draw (\x,2pt) -- (\x,-2pt)
			node[anchor=north] {\scriptsize \x}
		;
		\foreach \y in {1,...,5}
		\draw (2pt,\y) -- (-2pt,\y)
			node[anchor=east] {\scriptsize \y}
		; 
		\foreach \y in {-1,...,-5}
		\draw (2pt,\y) -- (-2pt,\y)
			node[anchor=west] {\scriptsize \y}
		; 
		\draw [<->] plot [domain=-6.5:0.25, samples=100] (\x,{1.5*\x+5.5});
		\draw[fill] (-1,4) circle (0.1) node[right] {};
		\draw[fill] (-5,-2)circle (0.1) node[right] {};
	\end{tikzpicture}
  
  22.\\
	\begin{tikzpicture}[xscale=0.4,yscale=0.4]
		\draw[step=1.0,gray,very thin,dotted] (-8.5,-5.5) grid (8.5,5.5);
		\draw [<->](-8.5,0) -- coordinate (x axis mid) (8.5,0) node[below right] {$x$};
		\draw [<->](0,-5.5) -- coordinate (y axis mid) (0,5.5) node[above right] {$y$};
		\foreach \x in {1,...,8}
		\draw (\x,2pt) -- (\x,-2pt)
			node[anchor=south] {\scriptsize \x}
		;
		\foreach \x in {-1,...,-8}
		\draw (\x,2pt) -- (\x,-2pt)
			node[anchor=north] {\scriptsize \x}
		;
		\foreach \y in {1,...,5}
		\draw (2pt,\y) -- (-2pt,\y)
			node[anchor=east] {\scriptsize \y}
		; 
		\foreach \y in {-1,...,-5}
		\draw (2pt,\y) -- (-2pt,\y)
			node[anchor=west] {\scriptsize \y}
		; 
		\draw [<->] plot [domain=-4.5:4.5, samples=100] (\x,{-0.667*(\x-1)+1});
		\draw[fill] (1,1) circle (0.1) node[right] {};
		\draw[fill] (-2,3)circle (0.1) node[right] {};
	\end{tikzpicture}
\end{multicols}
\begin{multicols}{2}
  23.\\
	\begin{tikzpicture}[xscale=0.4,yscale=0.4]
		\draw[step=1.0,gray,very thin,dotted] (-8.5,-5.5) grid (8.5,5.5);
		\draw [<->](-8.5,0) -- coordinate (x axis mid) (8.5,0) node[below right] {$x$};
		\draw [<->](0,-5.5) -- coordinate (y axis mid) (0,5.5) node[above right] {$y$};
		\foreach \x in {1,...,8}
		\draw (\x,2pt) -- (\x,-2pt)
			node[anchor=south] {\scriptsize \x}
		;
		\foreach \x in {-1,...,-8}
		\draw (\x,2pt) -- (\x,-2pt)
			node[anchor=north] {\scriptsize \x}
		;
		\foreach \y in {1,...,5}
		\draw (2pt,\y) -- (-2pt,\y)
			node[anchor=east] {\scriptsize \y}
		; 
		\foreach \y in {-1,...,-5}
		\draw (2pt,\y) -- (-2pt,\y)
			node[anchor=west] {\scriptsize \y}
		; 
		\draw [<->] plot [domain=-5.5:5.5, samples=100] (2,{\x});
		\draw[fill] (2,-1) circle (0.1) node[right] {};
		\draw[fill] (2,-2)circle (0.1) node[right] {};
	\end{tikzpicture}
  
  24.\\
	\begin{tikzpicture}[xscale=0.4,yscale=0.4]
		\draw[step=1.0,gray,very thin,dotted] (-8.5,-5.5) grid (8.5,5.5);
		\draw [<->](-8.5,0) -- coordinate (x axis mid) (8.5,0) node[below right] {$x$};
		\draw [<->](0,-5.5) -- coordinate (y axis mid) (0,5.5) node[above right] {$y$};
		\foreach \x in {1,...,8}
		\draw (\x,2pt) -- (\x,-2pt)
			node[anchor=south] {\scriptsize \x}
		;
		\foreach \x in {-1,...,-8}
		\draw (\x,2pt) -- (\x,-2pt)
			node[anchor=north] {\scriptsize \x}
		;
		\foreach \y in {1,...,5}
		\draw (2pt,\y) -- (-2pt,\y)
			node[anchor=east] {\scriptsize \y}
		; 
		\foreach \y in {-1,...,-5}
		\draw (2pt,\y) -- (-2pt,\y)
			node[anchor=west] {\scriptsize \y}
		; 
		\draw [<->] plot [domain=-2.5:4.5, samples=100] (\x,{-1*(\x)+3});
		\draw[fill] (-1,4) circle (0.1) node[right] {};
		\draw[fill] (0,3)circle (0.1) node[right] {};
	\end{tikzpicture}
\end{multicols}
\begin{multicols}{2}
  25.\\
	\begin{tikzpicture}[xscale=0.4,yscale=0.4]
		\draw[step=1.0,gray,very thin,dotted] (-8.5,-5.5) grid (8.5,5.5);
		\draw [<->](-8.5,0) -- coordinate (x axis mid) (8.5,0) node[below right] {$x$};
		\draw [<->](0,-5.5) -- coordinate (y axis mid) (0,5.5) node[above right] {$y$};
		\foreach \x in {1,...,8}
		\draw (\x,2pt) -- (\x,-2pt)
			node[anchor=south] {\scriptsize \x}
		;
		\foreach \x in {-1,...,-8}
		\draw (\x,2pt) -- (\x,-2pt)
			node[anchor=north] {\scriptsize \x}
		;
		\foreach \y in {1,...,5}
		\draw (2pt,\y) -- (-2pt,\y)
			node[anchor=east] {\scriptsize \y}
		; 
		\foreach \y in {-1,...,-5}
		\draw (2pt,\y) -- (-2pt,\y)
			node[anchor=west] {\scriptsize \y}
		; 
		\draw [<->] plot [domain=-6.5:3.5, samples=100] (\x,{-0.5*(\x+1)-2});
		\draw[fill] (-1,-2) circle (0.1) node[right] {};
		\draw[fill] (-3,-1)circle (0.1) node[right] {};
	\end{tikzpicture}
  
  26.\\
	\begin{tikzpicture}[xscale=0.4,yscale=0.4]
		\draw[step=1.0,gray,very thin,dotted] (-8.5,-5.5) grid (8.5,5.5);
		\draw [<->](-8.5,0) -- coordinate (x axis mid) (8.5,0) node[below right] {$x$};
		\draw [<->](0,-5.5) -- coordinate (y axis mid) (0,5.5) node[above right] {$y$};
		\foreach \x in {1,...,8}
		\draw (\x,2pt) -- (\x,-2pt)
			node[anchor=south] {\scriptsize \x}
		;
		\foreach \x in {-1,...,-8}
		\draw (\x,2pt) -- (\x,-2pt)
			node[anchor=north] {\scriptsize \x}
		;
		\foreach \y in {1,...,5}
		\draw (2pt,\y) -- (-2pt,\y)
			node[anchor=east] {\scriptsize \y}
		; 
		\foreach \y in {-1,...,-5}
		\draw (2pt,\y) -- (-2pt,\y)
			node[anchor=west] {\scriptsize \y}
		; 
		\draw [<->] plot [domain=-8.5:8.5, samples=100] (\x,{3});
		\draw[fill] (-4,3) circle (0.1) node[right] {};
		\draw[fill] (-2,3)circle (0.1) node[right] {};
	\end{tikzpicture}
\end{multicols}
\newpage
\end{document}