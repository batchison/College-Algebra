\documentclass[12pt]{article}
\usepackage[top=1in,left=1in,bottom=1in,right=1in,headsep=2pt]{geometry}	
\usepackage{amssymb,amsmath,amsthm,amsfonts}
\usepackage{chapterfolder,docmute,setspace}
\usepackage{cancel,multicol,tikz,verbatim,framed,polynom,enumitem}
\usepackage[colorlinks, hyperindex, plainpages=false, linkcolor=blue, urlcolor=blue, pdfpagelabels]{hyperref}
\usepackage[metapost,truebbox]{mfpic}
% Use the cc-by-nc-sa license for any content linked with Stitz and Zeager's text.  Otherwise, use the cc-by-sa license.
%\usepackage[type={CC},modifier={by-sa},version={4.0},]{doclicense}
\usepackage[type={CC},modifier={by-nc-sa},version={4.0},]{doclicense}

\theoremstyle{definition}
\newtheorem{example}{Example}
\newcommand{\Desmos}{\href{https://www.desmos.com/}{Desmos}}
\setlength{\parindent}{0em}
\setlist{itemsep=0em}
\setlength{\parskip}{0.1em}
% This document is used for ordering of lessons.  If an instructor wishes to change the ordering of assessments, the following steps must be taken:

% 1) Reassign the appropriate numbers for each lesson in the \setcounter commands included in this file.
% 2) Rearrange the \include commands in the master file (the file with 'Course Pack' in the name) to accurately reflect the changes.  
% 3) Rearrange the \items in the measureable_outcomes file to accurately reflect the changes.  Be mindful of page breaks when moving items.
% 4) Re-build all affected files (master file, measureable_outcomes file, and any lesson whose numbering has changed).

%Note: The placement of each \newcounter and \setcounter command reflects the original/default ordering of topics (linears, systems, quadratics, functions, polynomials, rationals).

\newcounter{lesson_solving_linear_equations}
\newcounter{lesson_equations_containing_absolute_values}
\newcounter{lesson_graphing_lines}
\newcounter{lesson_two_forms_of_a_linear_equation}
\newcounter{lesson_parallel_and_perpendicular_lines}
\newcounter{lesson_linear_inequalities}
\newcounter{lesson_compound_inequalities}
\newcounter{lesson_inequalities_containing_absolute_values}
\newcounter{lesson_graphing_systems}
\newcounter{lesson_substitution}
\newcounter{lesson_elimination}
\newcounter{lesson_quadratics_introduction}
\newcounter{lesson_factoring_GCF}
\newcounter{lesson_factoring_grouping}
\newcounter{lesson_factoring_trinomials_a_is_1}
\newcounter{lesson_factoring_trinomials_a_neq_1}
\newcounter{lesson_solving_by_factoring}
\newcounter{lesson_square_roots}
\newcounter{lesson_i_and_complex_numbers}
\newcounter{lesson_vertex_form_and_graphing}
\newcounter{lesson_solve_by_square_roots}
\newcounter{lesson_extracting_square_roots}
\newcounter{lesson_the_discriminant}
\newcounter{lesson_the_quadratic_formula}
\newcounter{lesson_quadratic_inequalities}
\newcounter{lesson_functions_and_relations}
\newcounter{lesson_evaluating_functions}
\newcounter{lesson_finding_domain_and_range_graphically}
\newcounter{lesson_fundamental_functions}
\newcounter{lesson_finding_domain_algebraically}
\newcounter{lesson_solving_functions}
\newcounter{lesson_function_arithmetic}
\newcounter{lesson_composite_functions}
\newcounter{lesson_inverse_functions_definition_and_HLT}
\newcounter{lesson_finding_an_inverse_function}
\newcounter{lesson_transformations_translations}
\newcounter{lesson_transformations_reflections}
\newcounter{lesson_transformations_scalings}
\newcounter{lesson_transformations_summary}
\newcounter{lesson_piecewise_functions}
\newcounter{lesson_functions_containing_absolute_values}
\newcounter{lesson_absolute_as_piecewise}
\newcounter{lesson_polynomials_introduction}
\newcounter{lesson_sign_diagrams_polynomials}
\newcounter{lesson_factoring_quadratic_type}
\newcounter{lesson_factoring_summary}
\newcounter{lesson_polynomial_division}
\newcounter{lesson_synthetic_division}
\newcounter{lesson_end_behavior_polynomials}
\newcounter{lesson_local_behavior_polynomials}
\newcounter{lesson_rational_root_theorem}
\newcounter{lesson_polynomials_graphing_summary}
\newcounter{lesson_polynomial_inequalities}
\newcounter{lesson_rationals_introduction_and_terminology}
\newcounter{lesson_sign_diagrams_rationals}
\newcounter{lesson_horizontal_asymptotes}
\newcounter{lesson_slant_and_curvilinear_asymptotes}
\newcounter{lesson_vertical_asymptotes}
\newcounter{lesson_holes}
\newcounter{lesson_rationals_graphing_summary}

\setcounter{lesson_solving_linear_equations}{1}
\setcounter{lesson_equations_containing_absolute_values}{2}
\setcounter{lesson_graphing_lines}{3}
\setcounter{lesson_two_forms_of_a_linear_equation}{4}
\setcounter{lesson_parallel_and_perpendicular_lines}{5}
\setcounter{lesson_linear_inequalities}{6}
\setcounter{lesson_compound_inequalities}{7}
\setcounter{lesson_inequalities_containing_absolute_values}{8}
\setcounter{lesson_graphing_systems}{9}
\setcounter{lesson_substitution}{10}
\setcounter{lesson_elimination}{11}
\setcounter{lesson_quadratics_introduction}{16}
\setcounter{lesson_factoring_GCF}{17}
\setcounter{lesson_factoring_grouping}{18}
\setcounter{lesson_factoring_trinomials_a_is_1}{19}
\setcounter{lesson_factoring_trinomials_a_neq_1}{20}
\setcounter{lesson_solving_by_factoring}{21}
\setcounter{lesson_square_roots}{22}
\setcounter{lesson_i_and_complex_numbers}{23}
\setcounter{lesson_vertex_form_and_graphing}{24}
\setcounter{lesson_solve_by_square_roots}{25}
\setcounter{lesson_extracting_square_roots}{26}
\setcounter{lesson_the_discriminant}{27}
\setcounter{lesson_the_quadratic_formula}{28}
\setcounter{lesson_quadratic_inequalities}{29}
\setcounter{lesson_functions_and_relations}{12}
\setcounter{lesson_evaluating_functions}{13}
\setcounter{lesson_finding_domain_and_range_graphically}{14}
\setcounter{lesson_fundamental_functions}{15}
\setcounter{lesson_finding_domain_algebraically}{30}
\setcounter{lesson_solving_functions}{31}
\setcounter{lesson_function_arithmetic}{32}
\setcounter{lesson_composite_functions}{33}
\setcounter{lesson_inverse_functions_definition_and_HLT}{34}
\setcounter{lesson_finding_an_inverse_function}{35}
\setcounter{lesson_transformations_translations}{36}
\setcounter{lesson_transformations_reflections}{37}
\setcounter{lesson_transformations_scalings}{38}
\setcounter{lesson_transformations_summary}{39}
\setcounter{lesson_piecewise_functions}{40}
\setcounter{lesson_functions_containing_absolute_values}{41}
\setcounter{lesson_absolute_as_piecewise}{42}
\setcounter{lesson_polynomials_introduction}{43}
\setcounter{lesson_sign_diagrams_polynomials}{44}
\setcounter{lesson_factoring_quadratic_type}{46}
\setcounter{lesson_factoring_summary}{45}
\setcounter{lesson_polynomial_division}{47}
\setcounter{lesson_synthetic_division}{48}
\setcounter{lesson_end_behavior_polynomials}{49}
\setcounter{lesson_local_behavior_polynomials}{50}
\setcounter{lesson_rational_root_theorem}{51}
\setcounter{lesson_polynomials_graphing_summary}{52}
\setcounter{lesson_polynomial_inequalities}{53}
\setcounter{lesson_rationals_introduction_and_terminology}{54}
\setcounter{lesson_sign_diagrams_rationals}{55}
\setcounter{lesson_horizontal_asymptotes}{56}
\setcounter{lesson_slant_and_curvilinear_asymptotes}{57}
\setcounter{lesson_vertical_asymptotes}{58}
\setcounter{lesson_holes}{59}
\setcounter{lesson_rationals_graphing_summary}{60}

\begin{document}
{\bf \large Lesson \arabic{lesson_finding_an_inverse_function}: Inverse Functions - Finding an Inverse Function, $f^{-1}$}\phantomsection\label{les:finding_an_inverse_function}
%\\ CC attribute: \href{http://www.wallace.ccfaculty.org/book/book.html}{\it{Beginning and Intermediate Algebra}} by T. Wallace. 
\\ CC attribute: \href{http://www.stitz-zeager.com}{\it{College Algebra}} by C. Stitz and J. Zeager. 
\hfill \doclicenseImage[imagewidth=5em]\\
\par
{\bf Objective:} 	Find the inverse of a given function.\\  
\par
{\bf Students will be able to:}
\begin{itemize}
	\item Find the inverse of a function algebraically.
\end{itemize}
{\bf Prerequisite Knowledge:}
\begin{itemize}
	\item Definition and properties of inverse function.
\end{itemize}
\hrulefill

{\bf Lesson:}
%\ \par
\begin{center}
\framebox{
\begin{minipage}{0.9\linewidth}
\centerline{\textbf{Steps for finding the Inverse of a Function}} 
\begin{enumerate}
\item  Rewrite $f(x)$ as $y$.
\item  Switch $x$ and $y$.
\item  Solve for $y$.
\item  Rewrite $y$ as $f^{-1}(x)$.
\end{enumerate}
\end{minipage}
}
\end{center}

{\bf I - Motivating Example(s):}\\
\ \par
{\bf Example:}  Find the inverse $f^{-1}$ of the function $f(x)=\dfrac{1-2x}{5}$.\\
\ \par
We replace $f(x)$ with $y$ and proceed to switch $x$ and $y$

\[ \begin{array}{rcll}
y & = &  \dfrac{1-2x}{5} & \\ [6pt]
x & = & \dfrac{1-2y}{5} & \mbox{Switch $x$ and $y$} \\ [6pt]
5x & = & 1 - 2y & \mbox{Solve for $y$} \\ [6pt]
5x-1 & = & -2y & \\ 
\dfrac{5x-1}{-2} & = & y & \\ 
y & = & -\dfrac{5}{2} x + \dfrac{1}{2} & 
\end{array} \]

We have $f^{-1}(x) = -\dfrac{5}{2} x + \dfrac{1}{2}$.\\
\ \par
To verify this answer, we leave it as an exercise to the reader to check that $\left(f^{-1} \circ f \right)(x) = x $ for all $x$ in the domain of $f$, and $\left(f \circ f^{-1} \right)(x) = x$ for all $x$ in the domain of $f^{-1}$.  Note that since $f$ and $f^{-1}$ are both linear functions, the domain and range for each function is $(-\infty,\infty)$.\\
\ \par
{\bf Example:}  Find the inverse $g^{-1}$ of the function $g(x) = \dfrac{2x}{1-x}$.\\
\ \par
Notice that the domain of $g$ is $(-\infty,1) \cup (1, \infty)$.  One can verify graphically, that the range of $g$ is $(-\infty,-2) \cup (-2, \infty)$.\\
\ \par
To find $g^{-1}(x)$, we start by replacing $g(x)$ with $y$.

\[ \begin{array}{rcll}
y & = &  \dfrac{2x}{1-x} & \\ [7pt]
x & = & \dfrac{2y}{1-y} & \mbox{Switch $x$ and $y$} \\ [3pt]
x(1-y) & = & 2y & \mbox{Solve for $y$; clear denominator} \\ [3pt]
x-xy & = & 2y & \mbox{Distribute $x$}\\ [3pt]
x & = & xy + 2y & \mbox{Move $y$ terms to one side}\\ [3pt]
x & = & y(x+2) & \mbox{Factor out $y$}\\ [8pt]
y & = & \dfrac{x}{x+2} & \mbox{Divide by $x+2$}
\end{array} \]

We have $g^{-1}(x) = \dfrac{x}{x+2}$.\\
\ \par
Notice that the domain of $g^{-1}$ matches the range of $g$ from earlier,
$(-\infty,-2) \cup (-2, \infty)$.  Again, we can use the graph of $g^{-1}$ to verify that the range of $g^{-1}$ also matches the domain of $g$, $(-\infty,1) \cup (1, \infty)$.\\
\ \par
We leave it as an exercise to show that $\left(g^{-1} \circ g \right)(x) = x$ and  $\left(g \circ g^{-1} \right)(x) = x$.\\
\ \par
{\bf II - Demo/Discussion Problems:}\\
\ \par
Find the inverse function for each of the following functions. Graph both the original function and your answer using \Desmos \ to confirm your results and compare the domains and ranges for your pair of functions.
\begin{multicols}{2}
\begin{enumerate}
 \item $f(x) = 3\sqrt{x}+4$\\
 \item $g(x)=2\sqrt{x-1}-4$\\
 \item $h(x)=-x^2-1,$ $x\geq 0$
 \item $k(x)=-x^2-1,$ $x\leq 0$\\
 \item $\ell(x) = x^2+10x+15,$ where $x\geq -5$\\
 \item $m(x) = x^2+10x+15,$ where $x\leq -5$
\end{enumerate}
\end{multicols}
\newpage
{\bf III - Practice Problems:}\\
\ \par
Find the inverse function for each of the following functions.  Check your answer algebraically by finding $f\circ f^{-1}$ and $f^{-1}\circ f$.  Graph both the original function and your answer using \Desmos \ to confirm your results and compare the domains and ranges for your pair of functions.
\begin{enumerate}
\begin{multicols}{2}
\item $f(x) = 2-6x$\\
\item $f(x) = \dfrac{x-2}{3} + 4$\\
\item $f(x)  = 1 - \dfrac{4+3x}{5}$\\
\item $f(x) = \sqrt{3x-1}+5$\\
\item $f(x) = 2-\sqrt{x - 5}$\\
\item $f(x) = 3\sqrt{x-1}-4$\\
\item $f(x) = 1 - 2\sqrt{2x+5}$\\
\item $f(x) = \sqrt[3]{3x-1}$\\
\item $f(x) = 3-\sqrt[3]{x-2}$\\
\item $f(x) = 8(x-2)^3$
\item $f(x) = (x+3)^2-6, \; x \geq -3$\\
\item $f(x) = 2(x-1)^2 + 4$, $x < 1$\\
\item $f(x) = x^2-6x+5, \; x \leq 3$\\
\item $f(x) = 4x^2 + 4x + 1$, $x < -1$\\
\item $f(x) = \dfrac{3}{4-x}$\\
\item $f(x) = \dfrac{x}{1-3x}$\\
\item $f(x) = \dfrac{2x-1}{3x+4}$\\
\item $f(x) = \dfrac{4x + 2}{3x - 6}$\\
\item $f(x) = \dfrac{-3x - 2}{x + 3}$\\ 
\item $f(x) = \dfrac{x-2}{2x-1}$
\end{multicols}
\end{enumerate}
\newpage
\ \newpage
\end{document}