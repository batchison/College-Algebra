\documentclass[12pt]{article}
\usepackage[top=1in,left=1in,bottom=1in,right=1in,headsep=2pt]{geometry}	
\usepackage{amssymb,amsmath,amsthm,amsfonts}
\usepackage{chapterfolder,docmute,setspace}
\usepackage{cancel,multicol,tikz,verbatim,framed,polynom,enumitem}
\usepackage[colorlinks, hyperindex, plainpages=false, linkcolor=blue, urlcolor=blue, pdfpagelabels]{hyperref}
% Use the cc-by-nc-sa license for any content linked with Stitz and Zeager's text.  Otherwise, use the cc-by-sa license.
\usepackage[type={CC},modifier={by-sa},version={4.0},]{doclicense}
%\usepackage[type={CC},modifier={by-nc-sa},version={4.0},]{doclicense}

\theoremstyle{definition}
\newtheorem{example}{Example}
\newcommand{\Desmos}{\href{https://www.desmos.com/}{Desmos}}
\setlength{\parindent}{0em}
\setlist{itemsep=0em}
\setlength{\parskip}{0.1em}
% This document is used for ordering of lessons.  If an instructor wishes to change the ordering of assessments, the following steps must be taken:

% 1) Reassign the appropriate numbers for each lesson in the \setcounter commands included in this file.
% 2) Rearrange the \include commands in the master file (the file with 'Course Pack' in the name) to accurately reflect the changes.  
% 3) Rearrange the \items in the measureable_outcomes file to accurately reflect the changes.  Be mindful of page breaks when moving items.
% 4) Re-build all affected files (master file, measureable_outcomes file, and any lesson whose numbering has changed).

%Note: The placement of each \newcounter and \setcounter command reflects the original/default ordering of topics (linears, systems, quadratics, functions, polynomials, rationals).

\newcounter{lesson_solving_linear_equations}
\newcounter{lesson_equations_containing_absolute_values}
\newcounter{lesson_graphing_lines}
\newcounter{lesson_two_forms_of_a_linear_equation}
\newcounter{lesson_parallel_and_perpendicular_lines}
\newcounter{lesson_linear_inequalities}
\newcounter{lesson_compound_inequalities}
\newcounter{lesson_inequalities_containing_absolute_values}
\newcounter{lesson_graphing_systems}
\newcounter{lesson_substitution}
\newcounter{lesson_elimination}
\newcounter{lesson_quadratics_introduction}
\newcounter{lesson_factoring_GCF}
\newcounter{lesson_factoring_grouping}
\newcounter{lesson_factoring_trinomials_a_is_1}
\newcounter{lesson_factoring_trinomials_a_neq_1}
\newcounter{lesson_solving_by_factoring}
\newcounter{lesson_square_roots}
\newcounter{lesson_i_and_complex_numbers}
\newcounter{lesson_vertex_form_and_graphing}
\newcounter{lesson_solve_by_square_roots}
\newcounter{lesson_extracting_square_roots}
\newcounter{lesson_the_discriminant}
\newcounter{lesson_the_quadratic_formula}
\newcounter{lesson_quadratic_inequalities}
\newcounter{lesson_functions_and_relations}
\newcounter{lesson_evaluating_functions}
\newcounter{lesson_finding_domain_and_range_graphically}
\newcounter{lesson_fundamental_functions}
\newcounter{lesson_finding_domain_algebraically}
\newcounter{lesson_solving_functions}
\newcounter{lesson_function_arithmetic}
\newcounter{lesson_composite_functions}
\newcounter{lesson_inverse_functions_definition_and_HLT}
\newcounter{lesson_finding_an_inverse_function}
\newcounter{lesson_transformations_translations}
\newcounter{lesson_transformations_reflections}
\newcounter{lesson_transformations_scalings}
\newcounter{lesson_transformations_summary}
\newcounter{lesson_piecewise_functions}
\newcounter{lesson_functions_containing_absolute_values}
\newcounter{lesson_absolute_as_piecewise}
\newcounter{lesson_polynomials_introduction}
\newcounter{lesson_sign_diagrams_polynomials}
\newcounter{lesson_factoring_quadratic_type}
\newcounter{lesson_factoring_summary}
\newcounter{lesson_polynomial_division}
\newcounter{lesson_synthetic_division}
\newcounter{lesson_end_behavior_polynomials}
\newcounter{lesson_local_behavior_polynomials}
\newcounter{lesson_rational_root_theorem}
\newcounter{lesson_polynomials_graphing_summary}
\newcounter{lesson_polynomial_inequalities}
\newcounter{lesson_rationals_introduction_and_terminology}
\newcounter{lesson_sign_diagrams_rationals}
\newcounter{lesson_horizontal_asymptotes}
\newcounter{lesson_slant_and_curvilinear_asymptotes}
\newcounter{lesson_vertical_asymptotes}
\newcounter{lesson_holes}
\newcounter{lesson_rationals_graphing_summary}

\setcounter{lesson_solving_linear_equations}{1}
\setcounter{lesson_equations_containing_absolute_values}{2}
\setcounter{lesson_graphing_lines}{3}
\setcounter{lesson_two_forms_of_a_linear_equation}{4}
\setcounter{lesson_parallel_and_perpendicular_lines}{5}
\setcounter{lesson_linear_inequalities}{6}
\setcounter{lesson_compound_inequalities}{7}
\setcounter{lesson_inequalities_containing_absolute_values}{8}
\setcounter{lesson_graphing_systems}{9}
\setcounter{lesson_substitution}{10}
\setcounter{lesson_elimination}{11}
\setcounter{lesson_quadratics_introduction}{16}
\setcounter{lesson_factoring_GCF}{17}
\setcounter{lesson_factoring_grouping}{18}
\setcounter{lesson_factoring_trinomials_a_is_1}{19}
\setcounter{lesson_factoring_trinomials_a_neq_1}{20}
\setcounter{lesson_solving_by_factoring}{21}
\setcounter{lesson_square_roots}{22}
\setcounter{lesson_i_and_complex_numbers}{23}
\setcounter{lesson_vertex_form_and_graphing}{24}
\setcounter{lesson_solve_by_square_roots}{25}
\setcounter{lesson_extracting_square_roots}{26}
\setcounter{lesson_the_discriminant}{27}
\setcounter{lesson_the_quadratic_formula}{28}
\setcounter{lesson_quadratic_inequalities}{29}
\setcounter{lesson_functions_and_relations}{12}
\setcounter{lesson_evaluating_functions}{13}
\setcounter{lesson_finding_domain_and_range_graphically}{14}
\setcounter{lesson_fundamental_functions}{15}
\setcounter{lesson_finding_domain_algebraically}{30}
\setcounter{lesson_solving_functions}{31}
\setcounter{lesson_function_arithmetic}{32}
\setcounter{lesson_composite_functions}{33}
\setcounter{lesson_inverse_functions_definition_and_HLT}{34}
\setcounter{lesson_finding_an_inverse_function}{35}
\setcounter{lesson_transformations_translations}{36}
\setcounter{lesson_transformations_reflections}{37}
\setcounter{lesson_transformations_scalings}{38}
\setcounter{lesson_transformations_summary}{39}
\setcounter{lesson_piecewise_functions}{40}
\setcounter{lesson_functions_containing_absolute_values}{41}
\setcounter{lesson_absolute_as_piecewise}{42}
\setcounter{lesson_polynomials_introduction}{43}
\setcounter{lesson_sign_diagrams_polynomials}{44}
\setcounter{lesson_factoring_quadratic_type}{46}
\setcounter{lesson_factoring_summary}{45}
\setcounter{lesson_polynomial_division}{47}
\setcounter{lesson_synthetic_division}{48}
\setcounter{lesson_end_behavior_polynomials}{49}
\setcounter{lesson_local_behavior_polynomials}{50}
\setcounter{lesson_rational_root_theorem}{51}
\setcounter{lesson_polynomials_graphing_summary}{52}
\setcounter{lesson_polynomial_inequalities}{53}
\setcounter{lesson_rationals_introduction_and_terminology}{54}
\setcounter{lesson_sign_diagrams_rationals}{55}
\setcounter{lesson_horizontal_asymptotes}{56}
\setcounter{lesson_slant_and_curvilinear_asymptotes}{57}
\setcounter{lesson_vertical_asymptotes}{58}
\setcounter{lesson_holes}{59}
\setcounter{lesson_rationals_graphing_summary}{60}

\begin{document}
{\bf \large Lesson \arabic{lesson_sign_diagrams_polynomials}: Sign Diagrams for Polynomials}
%\\ CC attribute: \href{http://www.wallace.ccfaculty.org/book/book.html}{\it{Beginning and Intermediate Algebra}} by T. Wallace. 
%\\ CC attribute: \href{http://www.stitz-zeager.com}{\it{College Algebra}} by C. Stitz and J. Zeager. 
\hfill \doclicenseImage[imagewidth=5em]\\
\par
{\bf Objective:} Construct a sign diagram for a given polynomial expression.\\
\par
{\bf Students will be able to:}
\begin{itemize}
	\item Evaluate a factored polynomial expression at specified test values in order to determine its sign.
\end{itemize}
{\bf Prerequisite Knowledge:}
\begin{itemize}
	\item Factoring.
	\item Identifying roots of a factored polynomial expression.
	\item Evaluating functions.
	\item Order of operations.
\end{itemize}
\hrulefill

{\bf Lesson:}\\
\ \par
If a polynomial function or expression is completely factored, it will be beneficial to us to construct a sign diagram for the polynomial, in order to answer questions about its graph and confirm any other findings.  Therefore, we devote this lesson to the construction of a sign diagram for a factored polynomial.  Note that expanded polynomials first require us to find a complete factorization prior to constructing a sign diagram.\\
\ \par
Recall that the roots of a quadratic expression represent the dividers in its corresponding sign diagram.  This carries over directly to a polynomial expression.\\
\ \par
{\bf I - Motivating Example(s):}\\
\ \par
{\bf Example:} Construct a sign diagram for the polynomial function $f(x)=2x^2+3x-20$.\\
\ \par
Although our first example is not factored, we can apply the $ac$-method to quickly factor our function.
\begin{equation*}
\begin{split}
f(x) & = 2x^2+3x-20\\
& = 2x^2+8x-5x-20\\
& = 2x(x+4)-5(x+4)\\
& = (x+4)(2x-5)
\end{split}
\end{equation*}
This gives us two roots, $x=-4$ and $x=\frac{5}{2}$, which serve as the dividers in our accompanying diagram.  For our three test values, we will use $x=-5, 0,$ and $3$.
\begin{center}
\begin{tikzpicture}[xscale=1,yscale=1]
	\draw [<->](-6.25,0) -- coordinate (x axis mid) (4.75,0) node[below right] {$x$};
	\draw [-](-3.5,1) -- coordinate (y axis mid) (-3.5,-0.25) node[below] {$-4$};
	\draw [-](2,1) -- coordinate (y axis mid) (2,-0.25) node[below] {$3$};
	\draw (-5,-1) node {$x=-5$};
	\draw (-0.75,-1) node {$x=0$};
	\draw (3.5,-1) node {$x=3$};
	\draw (-5,0.5) node {$+$};
	\draw (-0.75,0.5) node {$-$};
	\draw (3.5,0.5) node {$+$};
	\draw (-5,-1.5) node {\footnotesize $(-)(-)$};
	\draw (-0.75,-1.5) node {\footnotesize $(+)(-)$};
	\draw (3.5,-1.5) node {\footnotesize $(+)(+)$};
\end{tikzpicture}
\end{center}

{\bf Example:} Construct a sign diagram for the factored polynomial function $$g(x)=(x+2)(3x-1)(5-x).$$
Our roots are $x=-2,\frac{1}{3},$ and $5$.  Consequently, the following diagram shows three dividers.

\begin{center}
\begin{tikzpicture}[xscale=1,yscale=1]
	\draw [<->](-4.25,0) -- coordinate (x axis mid) (7.25,0) node[below right] {$x$};
	\draw [-](-2,1) -- coordinate (y axis mid) (-2,-0.25) node[below] {$-2$};
	\draw [-](0.5,1) -- coordinate (y axis mid) (0.5,-0.25) node[below] {$\frac{1}{3}$};
	\draw [-](5,1) -- coordinate (y axis mid) (5,-0.25) node[below] {$5$};
	\draw (-3,-1) node {$x=-3$};
	\draw (-0.75,-1) node {$x=0$};
	\draw (2.75,-1) node {$x=1$};
	\draw (6,-1) node {$x=6$};
	\draw (-3,0.5) node {$+$};
	\draw (-0.75,0.5) node {$-$};
	\draw (2.75,0.5) node {$+$};
	\draw (6,0.5) node {$-$};
	\draw (-3,-1.75) node {\footnotesize $(-)(-)(+)$};
	\draw (-0.75,-1.75) node {\footnotesize $(+)(-)(+)$};
	\draw (2.75,-1.75) node {\footnotesize $(+)(+)(+)$};
	\draw (6,-1.75) node {\footnotesize $(+)(+)(-)$};
\end{tikzpicture}
\end{center}
{\bf II - Demo/Discussion Problems:}\\
\ \par
Construct a sign diagram for the factored polynomial functions below. Use \Desmos \ to graph each function and check the accuracy of your diagram.  Identify the interval(s) where the function is positive and where it is negative.
\begin{enumerate}
	\item $h(x)=(x+2)^2(3x-1)(5-x)$
	\item $f(x)=x(x+1)(x-2)^2(x^2+4)$
\end{enumerate}	
\ \par
{\bf III - Practice Problems:}\\
\ \par
Construct a sign diagram for the factored polynomial functions below. Use \Desmos \ to graph each function and check the accuracy of your diagram.  Identify the interval(s) where the function is positive and where it is negative.
\begin{multicols}{2}
\begin{enumerate}
  \item $f(x)=x^3(x-2)(x+2)$
	\item $g(x)=(x^2+1)(1-x)$
	\item $h(x)=x(x-3)^2(x+3)$
	\item $k(x)=(3x-4)^3$
  \item $\ell(x)=(x^2+2)(x^2+3)$
  \item $m(x)=-2(x+7)^2(1-2x)^2$
  \item $f(x)=(x^2-1)(x+4)$
	\item $g(x)=(x^2-1)(x^2-16)$
	\item $h(x)=-2x^3(3x-1)(2-x)$
	\item $k(x)=(x^2-4x+1)(x+2)^2$
\end{enumerate}
\end{multicols}
\newpage
\end{document}