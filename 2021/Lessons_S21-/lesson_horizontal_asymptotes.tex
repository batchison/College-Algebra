\documentclass[12pt]{article}
\usepackage[top=1in,left=1in,bottom=1in,right=1in,headsep=2pt]{geometry}	
\usepackage{amssymb,amsmath,amsthm,amsfonts}
\usepackage{chapterfolder,docmute,setspace}
\usepackage{cancel,multicol,tikz,verbatim,framed,polynom,enumitem}
\usepackage[colorlinks, hyperindex, plainpages=false, linkcolor=blue, urlcolor=blue, pdfpagelabels]{hyperref}
% Use the cc-by-nc-sa license for any content linked with Stitz and Zeager's text.  Otherwise, use the cc-by-sa license.
%\usepackage[type={CC},modifier={by-sa},version={4.0},]{doclicense}
\usepackage[type={CC},modifier={by-nc-sa},version={4.0},]{doclicense}

\theoremstyle{definition}
\newtheorem{example}{Example}
\newcommand{\Desmos}{\href{https://www.desmos.com/}{Desmos}}
\setlength{\parindent}{0em}
\setlist{itemsep=0em}
\setlength{\parskip}{0.1em}
% This document is used for ordering of lessons.  If an instructor wishes to change the ordering of assessments, the following steps must be taken:

% 1) Reassign the appropriate numbers for each lesson in the \setcounter commands included in this file.
% 2) Rearrange the \include commands in the master file (the file with 'Course Pack' in the name) to accurately reflect the changes.  
% 3) Rearrange the \items in the measureable_outcomes file to accurately reflect the changes.  Be mindful of page breaks when moving items.
% 4) Re-build all affected files (master file, measureable_outcomes file, and any lesson whose numbering has changed).

%Note: The placement of each \newcounter and \setcounter command reflects the original/default ordering of topics (linears, systems, quadratics, functions, polynomials, rationals).

\newcounter{lesson_solving_linear_equations}
\newcounter{lesson_equations_containing_absolute_values}
\newcounter{lesson_graphing_lines}
\newcounter{lesson_two_forms_of_a_linear_equation}
\newcounter{lesson_parallel_and_perpendicular_lines}
\newcounter{lesson_linear_inequalities}
\newcounter{lesson_compound_inequalities}
\newcounter{lesson_inequalities_containing_absolute_values}
\newcounter{lesson_graphing_systems}
\newcounter{lesson_substitution}
\newcounter{lesson_elimination}
\newcounter{lesson_quadratics_introduction}
\newcounter{lesson_factoring_GCF}
\newcounter{lesson_factoring_grouping}
\newcounter{lesson_factoring_trinomials_a_is_1}
\newcounter{lesson_factoring_trinomials_a_neq_1}
\newcounter{lesson_solving_by_factoring}
\newcounter{lesson_square_roots}
\newcounter{lesson_i_and_complex_numbers}
\newcounter{lesson_vertex_form_and_graphing}
\newcounter{lesson_solve_by_square_roots}
\newcounter{lesson_extracting_square_roots}
\newcounter{lesson_the_discriminant}
\newcounter{lesson_the_quadratic_formula}
\newcounter{lesson_quadratic_inequalities}
\newcounter{lesson_functions_and_relations}
\newcounter{lesson_evaluating_functions}
\newcounter{lesson_finding_domain_and_range_graphically}
\newcounter{lesson_fundamental_functions}
\newcounter{lesson_finding_domain_algebraically}
\newcounter{lesson_solving_functions}
\newcounter{lesson_function_arithmetic}
\newcounter{lesson_composite_functions}
\newcounter{lesson_inverse_functions_definition_and_HLT}
\newcounter{lesson_finding_an_inverse_function}
\newcounter{lesson_transformations_translations}
\newcounter{lesson_transformations_reflections}
\newcounter{lesson_transformations_scalings}
\newcounter{lesson_transformations_summary}
\newcounter{lesson_piecewise_functions}
\newcounter{lesson_functions_containing_absolute_values}
\newcounter{lesson_absolute_as_piecewise}
\newcounter{lesson_polynomials_introduction}
\newcounter{lesson_sign_diagrams_polynomials}
\newcounter{lesson_factoring_quadratic_type}
\newcounter{lesson_factoring_summary}
\newcounter{lesson_polynomial_division}
\newcounter{lesson_synthetic_division}
\newcounter{lesson_end_behavior_polynomials}
\newcounter{lesson_local_behavior_polynomials}
\newcounter{lesson_rational_root_theorem}
\newcounter{lesson_polynomials_graphing_summary}
\newcounter{lesson_polynomial_inequalities}
\newcounter{lesson_rationals_introduction_and_terminology}
\newcounter{lesson_sign_diagrams_rationals}
\newcounter{lesson_horizontal_asymptotes}
\newcounter{lesson_slant_and_curvilinear_asymptotes}
\newcounter{lesson_vertical_asymptotes}
\newcounter{lesson_holes}
\newcounter{lesson_rationals_graphing_summary}

\setcounter{lesson_solving_linear_equations}{1}
\setcounter{lesson_equations_containing_absolute_values}{2}
\setcounter{lesson_graphing_lines}{3}
\setcounter{lesson_two_forms_of_a_linear_equation}{4}
\setcounter{lesson_parallel_and_perpendicular_lines}{5}
\setcounter{lesson_linear_inequalities}{6}
\setcounter{lesson_compound_inequalities}{7}
\setcounter{lesson_inequalities_containing_absolute_values}{8}
\setcounter{lesson_graphing_systems}{9}
\setcounter{lesson_substitution}{10}
\setcounter{lesson_elimination}{11}
\setcounter{lesson_quadratics_introduction}{16}
\setcounter{lesson_factoring_GCF}{17}
\setcounter{lesson_factoring_grouping}{18}
\setcounter{lesson_factoring_trinomials_a_is_1}{19}
\setcounter{lesson_factoring_trinomials_a_neq_1}{20}
\setcounter{lesson_solving_by_factoring}{21}
\setcounter{lesson_square_roots}{22}
\setcounter{lesson_i_and_complex_numbers}{23}
\setcounter{lesson_vertex_form_and_graphing}{24}
\setcounter{lesson_solve_by_square_roots}{25}
\setcounter{lesson_extracting_square_roots}{26}
\setcounter{lesson_the_discriminant}{27}
\setcounter{lesson_the_quadratic_formula}{28}
\setcounter{lesson_quadratic_inequalities}{29}
\setcounter{lesson_functions_and_relations}{12}
\setcounter{lesson_evaluating_functions}{13}
\setcounter{lesson_finding_domain_and_range_graphically}{14}
\setcounter{lesson_fundamental_functions}{15}
\setcounter{lesson_finding_domain_algebraically}{30}
\setcounter{lesson_solving_functions}{31}
\setcounter{lesson_function_arithmetic}{32}
\setcounter{lesson_composite_functions}{33}
\setcounter{lesson_inverse_functions_definition_and_HLT}{34}
\setcounter{lesson_finding_an_inverse_function}{35}
\setcounter{lesson_transformations_translations}{36}
\setcounter{lesson_transformations_reflections}{37}
\setcounter{lesson_transformations_scalings}{38}
\setcounter{lesson_transformations_summary}{39}
\setcounter{lesson_piecewise_functions}{40}
\setcounter{lesson_functions_containing_absolute_values}{41}
\setcounter{lesson_absolute_as_piecewise}{42}
\setcounter{lesson_polynomials_introduction}{43}
\setcounter{lesson_sign_diagrams_polynomials}{44}
\setcounter{lesson_factoring_quadratic_type}{46}
\setcounter{lesson_factoring_summary}{45}
\setcounter{lesson_polynomial_division}{47}
\setcounter{lesson_synthetic_division}{48}
\setcounter{lesson_end_behavior_polynomials}{49}
\setcounter{lesson_local_behavior_polynomials}{50}
\setcounter{lesson_rational_root_theorem}{51}
\setcounter{lesson_polynomials_graphing_summary}{52}
\setcounter{lesson_polynomial_inequalities}{53}
\setcounter{lesson_rationals_introduction_and_terminology}{54}
\setcounter{lesson_sign_diagrams_rationals}{55}
\setcounter{lesson_horizontal_asymptotes}{56}
\setcounter{lesson_slant_and_curvilinear_asymptotes}{57}
\setcounter{lesson_vertical_asymptotes}{58}
\setcounter{lesson_holes}{59}
\setcounter{lesson_rationals_graphing_summary}{60}

\begin{document}
{\bf \large Lesson \arabic{lesson_horizontal_asymptotes}: Horizontal Asymptotes}\phantomsection\label{les:horizontal_asymptotes}
%\\ CC attribute: \href{http://www.wallace.ccfaculty.org/book/book.html}{\it{Beginning and Intermediate Algebra}} by T. Wallace. 
\\ CC attribute: \href{http://www.stitz-zeager.com}{\it{College Algebra}} by C. Stitz and J. Zeager. 
\hfill \doclicenseImage[imagewidth=5em]\\
\par
{\bf Objective:} Identify a horizontal asymptote in the graph of a rational function.\\
\par
{\bf Students will be able to:}
\begin{itemize}
	\item Distinguish between and summarize the three cases for the end behavior of the graph of a rational function. 
	\item Identify the equation of a horizontal asymptote in the graph of a rational function.
	\item Use appropriate notation to describe the end behavior of the graph of a rational function.
\end{itemize}
{\bf Prerequisite Knowledge:}
\begin{itemize}
	\item End behavior of polynomials, including degree and leading coefficient.
\end{itemize}
\hrulefill

{\bf Lesson:}\\
\ \par
In this lesson, we will look at the end (or long run) behavior of the graph of a rational function $f$, as $x\rightarrow\pm\infty$.  
\begin{center}
\framebox{
\begin{minipage}{0.8\linewidth}
Let $f(x)=\dfrac{p(x)}{q(x)}$ be a rational function with leading terms $a_nx^n$ and $b_mx^m$ of $p(x)$ and $q(x),$ respectively.
	\begin{itemize}
		\item If $n=m,$ the graph of $f$ will have a horizontal asymptote at $y=\dfrac{a_n}{b_m}$.
		\item If $n<m,$ the graph of $f$ will have a horizontal asymptote at $y=0$.
		\item If $n>m,$ the graph of $f$ will not have a horizontal asymptote.
	\end{itemize}
\end{minipage}
}
\end{center}
Since any polynomial is, by definition, also a rational function, we will begin by including the possibilities that $f(x)\rightarrow\infty$ or $f(x)\rightarrow -\infty$ for either the left (as $x\rightarrow -\infty$) or right (as $x\rightarrow\infty$) end behavior of the graph of a rational function $f$.\\
\ \par
Recall that we used two aspects of a polynomial to identify the end behavior of its graph:
\begin{enumerate}
	\item the parity of the degree (even or odd), and
	\item the sign of the leading coefficient (positive or negative).
\end{enumerate}
As with polynomials, we will use the degree and leading coefficient of both the numerator and denominator of a rational function $f$, to identify the end behavior of its graph.
\newpage
{\bf I - Motivating Example(s):}\\
\ \par
Let us consider the graph of the reciprocal function $f(x)=\dfrac{1}{x}$, shown below.
\begin{multicols}{2}
\begin{tikzpicture}[xscale=0.75,yscale=0.75]
	\draw [<->](-4.25,0) -- coordinate (x axis mid) (4.25,0) node[below right] {$x$};
	\draw [<->](0,-4.25) -- coordinate (y axis mid) (0,4.25) node[above right] {$y$};
	\draw [<->] plot [domain=0.25:4, samples=100] (\x,{1/\x});
	\draw [<->] plot [domain=-4:-0.25, samples=100] (\x,{1/\x});
	\draw [->, line width=0.75mm] plot [domain=3:4, samples=100] (\x,{1/\x});
	\draw [<-, line width=0.75mm] plot [domain=-4:-3, samples=100] (\x,{1/\x});
%	\draw[fill] (1,1) circle (0.05);
%  \draw[fill] (2,0.5) circle (0.05);
%	\draw[fill] (3,0.333) circle (0.05);
%	\draw[fill] (0.5,2) circle (0.05);
%	\draw[fill] (0.333,3) circle (0.05);
%	\draw[fill] (-1,-1) circle (0.05);
%  \draw[fill] (-2,-0.5) circle (0.05);
%	\draw[fill] (-3,-0.333) circle (0.05);
%	\draw[fill] (-0.5,-2) circle (0.05);
%	\draw[fill] (-0.333,-3) circle (0.05);
	\foreach \x in {1,...,4}
		\draw (\x,2pt) -- (\x,-2pt)	node[anchor=north] {\scriptsize \x};
	\foreach \x in {-4,...,-1}
		\draw (\x,2pt) -- (\x,-2pt)	node[anchor=south] {\scriptsize \x};
	\foreach \y in {1,...,4}
		\draw (2pt,\y) -- (-2pt,\y)	node[anchor=east] {\scriptsize \y}; 
	\foreach \y in {-4,...,-1}
		\draw (2pt,\y) -- (-2pt,\y)	node[anchor=west] {\scriptsize \y}; 
	\draw (2,2) node {$f(x)=\dfrac{1}{x}$};
\end{tikzpicture}
\columnbreak

This example presents us with the first instance in which a graph does not tend towards either $\infty$ or $-\infty$, but instead ``levels off'' as the values of $x$ grow in either the positive (right) or negative (left) direction.
$$\text{As } x\rightarrow\infty,\ f(x)\rightarrow 0^+.$$
$$\text{As } x\rightarrow-\infty,\ f(x)\rightarrow 0^-.$$
Here, we use a $+$ or $-$ in the exponent to further describe how the tails of the graph approach $0$, either from \textit{above} ($+$) or from \textit{below} ($-$).  These identifiers can just as easily be omitted entirely, but provide a bit more insight into the graph of the function $f$.  The tails of the graph are thickened for additional emphasis of this concept.
\end{multicols}
In fact, for any real number $k$, we can transform the graph above, by simply adding $k$ to the function, to produce a new rational function whose graph levels off at $k$.  The resulting graph represents a vertical shift of the graph of $\frac{1}{x}$ by $k$ units.  The shift is up when $k>0$ and down when $k<0$.\\
\ \par
{\bf II - Demo/Discussion Problems:}\\
\ \par
Identify the equation of the horizontal asymptote, if one exists, for each of the following functions, and describe the end behavior of the function (As $x\rightarrow\pm\infty,\ldots$).  Use \Desmos \ to verify your answers.
\begin{enumerate}
\begin{multicols}{2}
\item $f(x)=\dfrac{-2x+4}{x-5}$
\item $g(x)=\dfrac{x-1}{x^2-4}$
\item $h(x)=\dfrac{x^2-4}{x-1}$
\item $j(x)=\dfrac{18x^3-4x-1}{6x^3-2x^2-8x}$
\end{multicols}
\end{enumerate}
\newpage
{\bf III - Practice Problems:}\\
\ \par
Identify the equation of the horizontal asymptote, if one exists, for each of the following functions, and describe the end behavior of the function (As $x\rightarrow\pm\infty,\ldots$).  Use \Desmos \ to verify your answers.
\begin{enumerate}
\begin{multicols}{2}
\item $a(x)=\dfrac{-x+4}{x-9}$
\item $b(x)=\dfrac{x^4-x^3-2x-19}{x^5-4}$
\item $c(x)=\dfrac{x^2-4}{5x^2-1}$
\item $d(x)=\dfrac{18x^7-4x-1}{3x^7-2x^2-8x}$
\item $m(x)=\dfrac{18x^3-4x-1}{3x^6-2x^2-8x}$
\item $n(x)=\dfrac{(x^2-9)(x-7)}{(x+3)(x-4)(2x-5)}$
\item $p(x)=\dfrac{-x^2+4}{x-9}$
\item $q(x)=\dfrac{x^3}{x^4-4}$
\item $r(x)=\dfrac{x^4-4}{x^3}$
\item $t(x)=\dfrac{15x-10}{5x-19}$
\end{multicols}
\end{enumerate}
\newpage
\ \newpage
\end{document}