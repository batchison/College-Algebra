\documentclass[12pt]{article}
\usepackage[top=1in,left=1in,bottom=1in,right=1in,headsep=2pt]{geometry}	
\usepackage{amssymb,amsmath,amsthm,amsfonts}
\usepackage{chapterfolder,docmute,setspace}
\usepackage{cancel,multicol,tikz,verbatim,framed,polynom,enumitem}
\usepackage[colorlinks, hyperindex, plainpages=false, linkcolor=blue, urlcolor=blue, pdfpagelabels]{hyperref}
% Use the cc-by-nc-sa license for any content linked with Stitz and Zeager's text.  Otherwise, use the cc-by-sa license.
%\usepackage[type={CC},modifier={by-sa},version={4.0},]{doclicense}
\usepackage[type={CC},modifier={by-nc-sa},version={4.0},]{doclicense}

\theoremstyle{definition}
\newtheorem{example}{Example}
\newcommand{\Desmos}{\href{https://www.desmos.com/}{Desmos}}
\setlength{\parindent}{0em}
\setlist{itemsep=0em}
\setlength{\parskip}{0.1em}
% This document is used for ordering of lessons.  If an instructor wishes to change the ordering of assessments, the following steps must be taken:

% 1) Reassign the appropriate numbers for each lesson in the \setcounter commands included in this file.
% 2) Rearrange the \include commands in the master file (the file with 'Course Pack' in the name) to accurately reflect the changes.  
% 3) Rearrange the \items in the measureable_outcomes file to accurately reflect the changes.  Be mindful of page breaks when moving items.
% 4) Re-build all affected files (master file, measureable_outcomes file, and any lesson whose numbering has changed).

%Note: The placement of each \newcounter and \setcounter command reflects the original/default ordering of topics (linears, systems, quadratics, functions, polynomials, rationals).

\newcounter{lesson_solving_linear_equations}
\newcounter{lesson_equations_containing_absolute_values}
\newcounter{lesson_graphing_lines}
\newcounter{lesson_two_forms_of_a_linear_equation}
\newcounter{lesson_parallel_and_perpendicular_lines}
\newcounter{lesson_linear_inequalities}
\newcounter{lesson_compound_inequalities}
\newcounter{lesson_inequalities_containing_absolute_values}
\newcounter{lesson_graphing_systems}
\newcounter{lesson_substitution}
\newcounter{lesson_elimination}
\newcounter{lesson_quadratics_introduction}
\newcounter{lesson_factoring_GCF}
\newcounter{lesson_factoring_grouping}
\newcounter{lesson_factoring_trinomials_a_is_1}
\newcounter{lesson_factoring_trinomials_a_neq_1}
\newcounter{lesson_solving_by_factoring}
\newcounter{lesson_square_roots}
\newcounter{lesson_i_and_complex_numbers}
\newcounter{lesson_vertex_form_and_graphing}
\newcounter{lesson_solve_by_square_roots}
\newcounter{lesson_extracting_square_roots}
\newcounter{lesson_the_discriminant}
\newcounter{lesson_the_quadratic_formula}
\newcounter{lesson_quadratic_inequalities}
\newcounter{lesson_functions_and_relations}
\newcounter{lesson_evaluating_functions}
\newcounter{lesson_finding_domain_and_range_graphically}
\newcounter{lesson_fundamental_functions}
\newcounter{lesson_finding_domain_algebraically}
\newcounter{lesson_solving_functions}
\newcounter{lesson_function_arithmetic}
\newcounter{lesson_composite_functions}
\newcounter{lesson_inverse_functions_definition_and_HLT}
\newcounter{lesson_finding_an_inverse_function}
\newcounter{lesson_transformations_translations}
\newcounter{lesson_transformations_reflections}
\newcounter{lesson_transformations_scalings}
\newcounter{lesson_transformations_summary}
\newcounter{lesson_piecewise_functions}
\newcounter{lesson_functions_containing_absolute_values}
\newcounter{lesson_absolute_as_piecewise}
\newcounter{lesson_polynomials_introduction}
\newcounter{lesson_sign_diagrams_polynomials}
\newcounter{lesson_factoring_quadratic_type}
\newcounter{lesson_factoring_summary}
\newcounter{lesson_polynomial_division}
\newcounter{lesson_synthetic_division}
\newcounter{lesson_end_behavior_polynomials}
\newcounter{lesson_local_behavior_polynomials}
\newcounter{lesson_rational_root_theorem}
\newcounter{lesson_polynomials_graphing_summary}
\newcounter{lesson_polynomial_inequalities}
\newcounter{lesson_rationals_introduction_and_terminology}
\newcounter{lesson_sign_diagrams_rationals}
\newcounter{lesson_horizontal_asymptotes}
\newcounter{lesson_slant_and_curvilinear_asymptotes}
\newcounter{lesson_vertical_asymptotes}
\newcounter{lesson_holes}
\newcounter{lesson_rationals_graphing_summary}

\setcounter{lesson_solving_linear_equations}{1}
\setcounter{lesson_equations_containing_absolute_values}{2}
\setcounter{lesson_graphing_lines}{3}
\setcounter{lesson_two_forms_of_a_linear_equation}{4}
\setcounter{lesson_parallel_and_perpendicular_lines}{5}
\setcounter{lesson_linear_inequalities}{6}
\setcounter{lesson_compound_inequalities}{7}
\setcounter{lesson_inequalities_containing_absolute_values}{8}
\setcounter{lesson_graphing_systems}{9}
\setcounter{lesson_substitution}{10}
\setcounter{lesson_elimination}{11}
\setcounter{lesson_quadratics_introduction}{16}
\setcounter{lesson_factoring_GCF}{17}
\setcounter{lesson_factoring_grouping}{18}
\setcounter{lesson_factoring_trinomials_a_is_1}{19}
\setcounter{lesson_factoring_trinomials_a_neq_1}{20}
\setcounter{lesson_solving_by_factoring}{21}
\setcounter{lesson_square_roots}{22}
\setcounter{lesson_i_and_complex_numbers}{23}
\setcounter{lesson_vertex_form_and_graphing}{24}
\setcounter{lesson_solve_by_square_roots}{25}
\setcounter{lesson_extracting_square_roots}{26}
\setcounter{lesson_the_discriminant}{27}
\setcounter{lesson_the_quadratic_formula}{28}
\setcounter{lesson_quadratic_inequalities}{29}
\setcounter{lesson_functions_and_relations}{12}
\setcounter{lesson_evaluating_functions}{13}
\setcounter{lesson_finding_domain_and_range_graphically}{14}
\setcounter{lesson_fundamental_functions}{15}
\setcounter{lesson_finding_domain_algebraically}{30}
\setcounter{lesson_solving_functions}{31}
\setcounter{lesson_function_arithmetic}{32}
\setcounter{lesson_composite_functions}{33}
\setcounter{lesson_inverse_functions_definition_and_HLT}{34}
\setcounter{lesson_finding_an_inverse_function}{35}
\setcounter{lesson_transformations_translations}{36}
\setcounter{lesson_transformations_reflections}{37}
\setcounter{lesson_transformations_scalings}{38}
\setcounter{lesson_transformations_summary}{39}
\setcounter{lesson_piecewise_functions}{40}
\setcounter{lesson_functions_containing_absolute_values}{41}
\setcounter{lesson_absolute_as_piecewise}{42}
\setcounter{lesson_polynomials_introduction}{43}
\setcounter{lesson_sign_diagrams_polynomials}{44}
\setcounter{lesson_factoring_quadratic_type}{46}
\setcounter{lesson_factoring_summary}{45}
\setcounter{lesson_polynomial_division}{47}
\setcounter{lesson_synthetic_division}{48}
\setcounter{lesson_end_behavior_polynomials}{49}
\setcounter{lesson_local_behavior_polynomials}{50}
\setcounter{lesson_rational_root_theorem}{51}
\setcounter{lesson_polynomials_graphing_summary}{52}
\setcounter{lesson_polynomial_inequalities}{53}
\setcounter{lesson_rationals_introduction_and_terminology}{54}
\setcounter{lesson_sign_diagrams_rationals}{55}
\setcounter{lesson_horizontal_asymptotes}{56}
\setcounter{lesson_slant_and_curvilinear_asymptotes}{57}
\setcounter{lesson_vertical_asymptotes}{58}
\setcounter{lesson_holes}{59}
\setcounter{lesson_rationals_graphing_summary}{60}

\begin{document}
{\bf \large Lesson \arabic{lesson_functions_containing_absolute_values}: Functions Containing Absolute Values}\phantomsection\label{les:functions_containing_absolute_values}
%\\ CC attribute: \href{http://www.wallace.ccfaculty.org/book/book.html}{\it{Beginning and Intermediate Algebra}} by T. Wallace. 
\\ CC attribute: \href{http://www.stitz-zeager.com}{\it{College Algebra}} by C. Stitz and J. Zeager. 
\hfill \doclicenseImage[imagewidth=5em]\\
\par
{\bf Objective:} Graph a variety of functions that contain an absolute value.\\
\par
{\bf Students will be able to:}
\begin{itemize}
	\item Create a table of values for a function containing an absolute value.
	\item Identify the intercepts of a function containing an absolute value.
	\item Graph a function that contains an absolute value.
	\item Solve equations containing an absolute value graphically.
\end{itemize}
{\bf Prerequisite Knowledge:}
\begin{itemize}
	\item Creating a table of values for a function.
	\item Definition of an absolute value.
\end{itemize}
\hrulefill

{\bf Lesson:}\\
\ \par
The most basic of functions containing an absolute value is $\ell(x)=|x|$.  Later, when we cover transformations of functions, we will see more general forms of such functions.  In particular, if we consider the function
$$f(x)=a|x-h|+k,$$
we can make some simple observations about the graph of $f$.  For example, if $a>0,$ the graph of $f$ will point upwards.  In this case, the graph of $f$ will have a {\it minimum} at $y=k,$ corresponding to the point $(h,k)$.  Alternatively, if $a<0,$ the graph of $f$ will point downwards, and the graph will achieve its {\it maximum} value at the point $(h,k)$.  The magnitude of the coefficient $a$ (i.e. its absolute value) will also determine whether the graph of $f$ is wide or narrow.

\newpage

{\bf I - Motivating Example(s):}\\
\ \par
{\bf Example:} 
\begin{multicols}{2}
\begin{tabular}{ll}
Function Type: & Absolute Value\\
Representative: & $\ell(x)=|x|$
\end{tabular}
\begin{center}
%\vspace{0.25in}
\begin{tabular}{c|c}
	$x$ & $\ell(x)$\\
	\hline
 & \\
 $-3$ & 3\\
 & \\
 $-2$ & 2\\
 & \\
 $-1$ & 1\\
 & \\
 0 & 0\\
 & \\
 1 & 1\\
 & \\
 2 & 2\\
 & \\
 3 & 3\\
 & \\
\end{tabular}
\end{center}
~\\
\vspace{0.25in}
~\\
\begin{tikzpicture}[xscale=0.65,yscale=0.65]
\draw [<->](-5.5,0) -- coordinate (x axis mid) (5.5,0) node[below right] {$x$};
\draw [<->](0,-0.5) -- coordinate (y axis mid) (0,5.5) node[above right] {$y$};
\foreach \x in {-5,...,-1}
\draw (\x,1pt) -- (\x,-3pt) node[anchor=north] {\scriptsize \x};
\foreach \x in {1,...,5}
\draw (\x,1pt) -- (\x,-3pt) node[anchor=north] {\scriptsize \x};
%\foreach \y in {-3,...,-1}
%\draw (1pt,\y) -- (-3pt,\y); 
%node[anchor=east] {\scriptsize \y}; 
\foreach \y in {1,...,5}
\draw (1pt,\y) -- (-3pt,\y) node[anchor=east] {\scriptsize \y}; 
\draw [<-, domain=-5:0] plot (\x, {-\x});
\draw [->, domain=0:5] plot (\x, {\x});
%\draw[fill] (1,-5) circle (0.075);
%\draw[fill] (-1,3) circle (0.075);
\end{tikzpicture}
\begin{center}
Graph of $\ell(x)=|x|$
\end{center}
\end{multicols}
\begin{multicols}{2}
\begin{tabular}{ll}
$y-$intercept: & $(0,0)$\\
$x-$intercept(s): & $(0,0)$
\end{tabular}

\columnbreak
\begin{tabular}{ll}
Domain: & $(-\infty,\infty)$\\
Range: & $[0,\infty)$
\end{tabular}
\end{multicols}
Notes: The domain of an absolute value function of the form $f(x)=a|x-h|+k$ will remain the same as above.  If $a>0,$ the range of $f$ will be $[k,\infty)$.  If $a<0,$ the range of $f$ will be $(-\infty,k]$.\\
\ \par
{\bf Example:} Use the graph of $\ell(x) = |x|$ to graph the function $g(x) = |x-3|$.\\
\ \par
We begin by graphing $\ell(x) = |x|$ and labeling three reference points: $(-1,1)$, $(0,0)$ and $(1,1)$.

\begin{center}
\begin{tikzpicture}[xscale=0.55,yscale=0.55]
	\draw [<->](-4,0) -- coordinate (x axis mid) (4,0) node[below right] {$x$};
	\draw [<->](0,0) -- coordinate (x axis mid) (0,4.5) node[above right] {$y$};
	\draw [->] plot [domain=0:4, samples=100] (\x,{\x});
	\draw [->] plot [domain=0:-4, samples=100] (\x,{-\x});
	\draw[fill] (-1,1) circle (0.08);
	\draw[fill] (1,1) circle (0.08);
	\draw[fill] (0,0) circle (0.08);
	\foreach \x in {-3,...,-1}
	\draw (\x,1pt) -- (\x,-3pt)
	node[anchor=north] {\scriptsize $\x$};
	\foreach \x in {1,...,3}
	\draw (\x,1pt) -- (\x,-3pt)
	node[anchor=north] {\scriptsize $\x$};
	\foreach \y in {1,...,4}
	\draw (1pt,\y) -- (-3pt,\y) 
	node[anchor=east] {\scriptsize $\y$};
	\draw (1,1) node[right] {\scriptsize $(1,1)$}; 
	\draw (-1,1) node[left] {\scriptsize $(-1,1)$}; 
	\draw (0,-0.3) node[below] {\scriptsize $(0,0)$}; 
	\draw (0,-1) node[below] {\scriptsize $\ell(x)=|x|$};  
\end{tikzpicture}
\end{center}

Since $g(x) = |x-3| = \ell(x-3)$, we will add $3$ to each of the $x-$coordinates of the points on the graph of $y=\ell(x)$ to obtain the graph of $y=g(x)$.   This shifts the graph of $y=\ell(x)$ to the {\it right} by $3$ units and moves the points $(-1,1)$ to $(2,1)$,  $(0,0)$ to $(3,0)$ and $(1,1)$ to $(4,1)$.  Connecting these points in the classic `$\vee$' fashion produces the graph of $y = g(x)$.

\[ \begin{array}{ccc}

\begin{tikzpicture}[xscale=0.55,yscale=0.55]
	\draw [<->](-4,0) -- coordinate (x axis mid) (4,0) node[below right] {$x$};
	\draw [<->](0,0) -- coordinate (x axis mid) (0,4.5) node[above right] {$y$};
	\draw [->] plot [domain=0:4, samples=100] (\x,{\x});
	\draw [->] plot [domain=0:-4, samples=100] (\x,{-\x});
	\draw[fill] (-1,1) circle (0.08);
	\draw[fill] (1,1) circle (0.08);
	\draw[fill] (0,0) circle (0.08);
	\foreach \x in {-3,...,-1}
	\draw (\x,1pt) -- (\x,-3pt)
	node[anchor=north] {\scriptsize $\x$};
	\foreach \x in {1,...,3}
	\draw (\x,1pt) -- (\x,-3pt)
	node[anchor=north] {\scriptsize $\x$};
	\foreach \y in {1,...,4}
	\draw (1pt,\y) -- (-3pt,\y) 
	node[anchor=east] {\scriptsize $\y$};
	\draw (1,1) node[right] {\scriptsize $(1,1)$}; 
	\draw (-1,1) node[left] {\scriptsize $(-1,1)$}; 
	\draw (0,-0.3) node[below] {\scriptsize $(0,0)$}; 
	\draw (0,-1) node[below] {\scriptsize $\ell(x)=|x|$};  
\end{tikzpicture}
&

\stackrel{\stackrel{\mbox{\scriptsize shift right $3$ units}}{\xrightarrow{\hspace{1in}}}}{\stackrel{\mbox{ \scriptsize add $3$ to each}}{\mbox{\scriptsize $x$-coordinate}}} 

&

\begin{tikzpicture}[xscale=0.55,yscale=0.55]
	\draw [<->](-1,0) -- coordinate (x axis mid) (7,0) node[below right] {$x$};
	\draw [->](0,0) -- coordinate (x axis mid) (0,4.5) node[above right] {$y$};
	\draw [->] plot [domain=3:7, samples=100] (\x,{\x-3});
	\draw [->] plot [domain=3:-1, samples=100] (\x,{-\x+3});
	\draw[fill] (2,1) circle (0.08);
	\draw[fill] (4,1) circle (0.08);
	\draw[fill] (3,0) circle (0.08);
	\foreach \x in {0,1,2,4,5,6}
	\draw (\x,1pt) -- (\x,-3pt)
	node[anchor=north] {\scriptsize $\x$};
	\foreach \y in {1,...,4}
	\draw (1pt,\y) -- (-3pt,\y) 
	node[anchor=east] {\scriptsize $\y$};
	\draw (4,1) node[right] {\scriptsize $(4,1)$}; 
	\draw (2,1) node[left] {\scriptsize $(2,1)$}; 
	\draw (3,-0.3) node[below] {\scriptsize $(3,0)$}; 
	\draw (3,-1) node[below] {\scriptsize $g(x)=\ell(x-3)=|x-3|$};  
\end{tikzpicture}
\end{array}\]
\ \par
{\bf Example:} Use the graph of $\ell(x) = |x|$ to graph the function $h(x) = |x|-3$.\\
\ \par

Since $h(x) = |x| - 3 = f(x) -3$, we will subtract $3$ from each of the $y-$coordinates of the points on the graph of $y=\ell(x)$ to obtain the graph of $y = h(x)$.  This shifts the graph of $y=\ell(x)$ {\it down} by $3$ units and moves the points $(-1,1)$ to $(-1,-2)$, $(0,0)$ to $(0,-3)$ and $(1,1)$ to $(1,-2)$.  Connecting these points with the `$\vee$' shape produces our graph of $y=h(x)$. 

\[ \begin{array}{ccc}

\begin{tikzpicture}[xscale=0.55,yscale=0.55]
	\draw [<->](-4,0) -- coordinate (x axis mid) (4,0) node[below right] {$x$};
	\draw [<->](0,0) -- coordinate (x axis mid) (0,4.5) node[above right] {$y$};
	\draw [->] plot [domain=0:4, samples=100] (\x,{\x});
	\draw [->] plot [domain=0:-4, samples=100] (\x,{-\x});
	\draw[fill] (-1,1) circle (0.08);
	\draw[fill] (1,1) circle (0.08);
	\draw[fill] (0,0) circle (0.08);
	\foreach \x in {-3,...,-1}
	\draw (\x,1pt) -- (\x,-3pt)
	node[anchor=north] {\scriptsize $\x$};
	\foreach \x in {1,...,3}
	\draw (\x,1pt) -- (\x,-3pt)
	node[anchor=north] {\scriptsize $\x$};
	\foreach \y in {1,...,4}
	\draw (1pt,\y) -- (-3pt,\y) 
	node[anchor=east] {\scriptsize $\y$};
	\draw (1,1) node[right] {\scriptsize $(1,1)$}; 
	\draw (-1,1) node[left] {\scriptsize $(-1,1)$}; 
	\draw (0,-0.3) node[below] {\scriptsize $(0,0)$}; 
	\draw (0,-1) node[below] {\scriptsize $\ell(x)=|x|$};  
\end{tikzpicture}
&

\stackrel{\stackrel{\mbox{\scriptsize shift down $3$ units}}{\xrightarrow{\hspace{1in}}}}{\stackrel{\mbox{ \scriptsize subtract $3$ from}}{\mbox{\scriptsize each $y$-coordinate}}} 

&

\begin{tikzpicture}[xscale=0.55,yscale=0.55]
	\draw [<->](-4,0) -- coordinate (x axis mid) (4,0) node[below right] {$x$};
	\draw [<->](0,-4) -- coordinate (x axis mid) (0,1.5) node[above right] {$y$};
	\draw [->] plot [domain=0:4, samples=100] (\x,{\x-3});
	\draw [->] plot [domain=0:-4, samples=100] (\x,{-\x-3});
	\draw[fill] (-1,-2) circle (0.08);
	\draw[fill] (1,-2) circle (0.08);
	\draw[fill] (0,-3) circle (0.08);
	\foreach \x in {-3,...,-1}
	\draw (\x,1pt) -- (\x,-3pt)
	node[anchor=north] {\scriptsize $\x$};
	\foreach \x in {1,...,3}
	\draw (\x,1pt) -- (\x,-3pt)
	node[anchor=north] {\scriptsize $\x$};
	\foreach \y in {1}
	\draw (1pt,\y) -- (-3pt,\y) 
	node[anchor=east] {\scriptsize $\y$};
	\foreach \y in {-1,...,-3}
	\draw (1pt,\y) -- (-3pt,\y) 
	node[anchor=east] {\scriptsize $\y$};
	\draw (1,-2) node[right] {\scriptsize $(1,-2)$}; 
	\draw (-1,-2) node[left] {\scriptsize $(-1,-2)$}; 
	\draw (1,-3) node {\scriptsize $(0,-3)$}; 
	\draw (0,-4) node[below] {\scriptsize $h(x)=\ell(x)-3=|x|-3$};  
\end{tikzpicture}
\end{array}\]

{\bf II - Demo/Discussion Problems:}\\
\ \par
Graph each of the following functions.  In each case, make a table of values.  Then graph $\ell(x)=|x|$ and $f$ on \Desmos \ and compare the two graphs.  Find the intercepts, domain, and range of $f$.
\begin{multicols}{3}
\begin{enumerate}
	\item $f(x) = |x+3|$
	\item $f(x) = 2|x+3|$
	\item $f(x) = -2|x-3|$
	\item $f(x) = \frac{1}{2}|3x+1|$
	\item $f(x) = \frac{1}{2}|3x+1|-5$
	\item $f(x) = 4-2|3x+1|$
\end{enumerate}	
\end{multicols}
{\bf III - Practice Problems:}\\
\ \par
Graph each of the following functions.  Make a table of values if necessary.  Find the intercepts, domain, and range of the function.
\begin{multicols}{3}
\begin{enumerate}
	\item $f(x) = -|x|-7$
	\item $g(x) = 4|x|+2$
	\item $h(x) = 5|x-3|+6$
	\item $k(x) = -2|x+1|+10$
	\item $m(x) = |x-3|$
	\item $n(x) = -4|x-1|+1$
\end{enumerate}	
\end{multicols}
\newpage
\ \newpage
\end{document}