\documentclass[12pt]{article}
\usepackage[top=1in,left=1in,bottom=1in,right=1in,headsep=2pt]{geometry}	
\usepackage{amssymb,amsmath,amsthm,amsfonts}
\usepackage{chapterfolder,docmute,setspace}
\usepackage{cancel,multicol,tikz,verbatim,framed,polynom,enumitem}
\usepackage[colorlinks, hyperindex, plainpages=false, linkcolor=blue, urlcolor=blue, pdfpagelabels]{hyperref}
% Use the cc-by-nc-sa license for any content linked with Stitz and Zeager's text.  Otherwise, use the cc-by-sa license.
\usepackage[type={CC},modifier={by-sa},version={4.0},]{doclicense}
%\usepackage[type={CC},modifier={by-nc-sa},version={4.0},]{doclicense}

\theoremstyle{definition}
\newtheorem{example}{Example}
\newcommand{\Desmos}{\href{https://www.desmos.com/}{Desmos}}
\setlength{\parindent}{0em}
\setlist{itemsep=0em}
\setlength{\parskip}{0.1em}
% This document is used for ordering of lessons.  If an instructor wishes to change the ordering of assessments, the following steps must be taken:

% 1) Reassign the appropriate numbers for each lesson in the \setcounter commands included in this file.
% 2) Rearrange the \include commands in the master file (the file with 'Course Pack' in the name) to accurately reflect the changes.  
% 3) Rearrange the \items in the measureable_outcomes file to accurately reflect the changes.  Be mindful of page breaks when moving items.
% 4) Re-build all affected files (master file, measureable_outcomes file, and any lesson whose numbering has changed).

%Note: The placement of each \newcounter and \setcounter command reflects the original/default ordering of topics (linears, systems, quadratics, functions, polynomials, rationals).

\newcounter{lesson_solving_linear_equations}
\newcounter{lesson_equations_containing_absolute_values}
\newcounter{lesson_graphing_lines}
\newcounter{lesson_two_forms_of_a_linear_equation}
\newcounter{lesson_parallel_and_perpendicular_lines}
\newcounter{lesson_linear_inequalities}
\newcounter{lesson_compound_inequalities}
\newcounter{lesson_inequalities_containing_absolute_values}
\newcounter{lesson_graphing_systems}
\newcounter{lesson_substitution}
\newcounter{lesson_elimination}
\newcounter{lesson_quadratics_introduction}
\newcounter{lesson_factoring_GCF}
\newcounter{lesson_factoring_grouping}
\newcounter{lesson_factoring_trinomials_a_is_1}
\newcounter{lesson_factoring_trinomials_a_neq_1}
\newcounter{lesson_solving_by_factoring}
\newcounter{lesson_square_roots}
\newcounter{lesson_i_and_complex_numbers}
\newcounter{lesson_vertex_form_and_graphing}
\newcounter{lesson_solve_by_square_roots}
\newcounter{lesson_extracting_square_roots}
\newcounter{lesson_the_discriminant}
\newcounter{lesson_the_quadratic_formula}
\newcounter{lesson_quadratic_inequalities}
\newcounter{lesson_functions_and_relations}
\newcounter{lesson_evaluating_functions}
\newcounter{lesson_finding_domain_and_range_graphically}
\newcounter{lesson_fundamental_functions}
\newcounter{lesson_finding_domain_algebraically}
\newcounter{lesson_solving_functions}
\newcounter{lesson_function_arithmetic}
\newcounter{lesson_composite_functions}
\newcounter{lesson_inverse_functions_definition_and_HLT}
\newcounter{lesson_finding_an_inverse_function}
\newcounter{lesson_transformations_translations}
\newcounter{lesson_transformations_reflections}
\newcounter{lesson_transformations_scalings}
\newcounter{lesson_transformations_summary}
\newcounter{lesson_piecewise_functions}
\newcounter{lesson_functions_containing_absolute_values}
\newcounter{lesson_absolute_as_piecewise}
\newcounter{lesson_polynomials_introduction}
\newcounter{lesson_sign_diagrams_polynomials}
\newcounter{lesson_factoring_quadratic_type}
\newcounter{lesson_factoring_summary}
\newcounter{lesson_polynomial_division}
\newcounter{lesson_synthetic_division}
\newcounter{lesson_end_behavior_polynomials}
\newcounter{lesson_local_behavior_polynomials}
\newcounter{lesson_rational_root_theorem}
\newcounter{lesson_polynomials_graphing_summary}
\newcounter{lesson_polynomial_inequalities}
\newcounter{lesson_rationals_introduction_and_terminology}
\newcounter{lesson_sign_diagrams_rationals}
\newcounter{lesson_horizontal_asymptotes}
\newcounter{lesson_slant_and_curvilinear_asymptotes}
\newcounter{lesson_vertical_asymptotes}
\newcounter{lesson_holes}
\newcounter{lesson_rationals_graphing_summary}

\setcounter{lesson_solving_linear_equations}{1}
\setcounter{lesson_equations_containing_absolute_values}{2}
\setcounter{lesson_graphing_lines}{3}
\setcounter{lesson_two_forms_of_a_linear_equation}{4}
\setcounter{lesson_parallel_and_perpendicular_lines}{5}
\setcounter{lesson_linear_inequalities}{6}
\setcounter{lesson_compound_inequalities}{7}
\setcounter{lesson_inequalities_containing_absolute_values}{8}
\setcounter{lesson_graphing_systems}{9}
\setcounter{lesson_substitution}{10}
\setcounter{lesson_elimination}{11}
\setcounter{lesson_quadratics_introduction}{16}
\setcounter{lesson_factoring_GCF}{17}
\setcounter{lesson_factoring_grouping}{18}
\setcounter{lesson_factoring_trinomials_a_is_1}{19}
\setcounter{lesson_factoring_trinomials_a_neq_1}{20}
\setcounter{lesson_solving_by_factoring}{21}
\setcounter{lesson_square_roots}{22}
\setcounter{lesson_i_and_complex_numbers}{23}
\setcounter{lesson_vertex_form_and_graphing}{24}
\setcounter{lesson_solve_by_square_roots}{25}
\setcounter{lesson_extracting_square_roots}{26}
\setcounter{lesson_the_discriminant}{27}
\setcounter{lesson_the_quadratic_formula}{28}
\setcounter{lesson_quadratic_inequalities}{29}
\setcounter{lesson_functions_and_relations}{12}
\setcounter{lesson_evaluating_functions}{13}
\setcounter{lesson_finding_domain_and_range_graphically}{14}
\setcounter{lesson_fundamental_functions}{15}
\setcounter{lesson_finding_domain_algebraically}{30}
\setcounter{lesson_solving_functions}{31}
\setcounter{lesson_function_arithmetic}{32}
\setcounter{lesson_composite_functions}{33}
\setcounter{lesson_inverse_functions_definition_and_HLT}{34}
\setcounter{lesson_finding_an_inverse_function}{35}
\setcounter{lesson_transformations_translations}{36}
\setcounter{lesson_transformations_reflections}{37}
\setcounter{lesson_transformations_scalings}{38}
\setcounter{lesson_transformations_summary}{39}
\setcounter{lesson_piecewise_functions}{40}
\setcounter{lesson_functions_containing_absolute_values}{41}
\setcounter{lesson_absolute_as_piecewise}{42}
\setcounter{lesson_polynomials_introduction}{43}
\setcounter{lesson_sign_diagrams_polynomials}{44}
\setcounter{lesson_factoring_quadratic_type}{46}
\setcounter{lesson_factoring_summary}{45}
\setcounter{lesson_polynomial_division}{47}
\setcounter{lesson_synthetic_division}{48}
\setcounter{lesson_end_behavior_polynomials}{49}
\setcounter{lesson_local_behavior_polynomials}{50}
\setcounter{lesson_rational_root_theorem}{51}
\setcounter{lesson_polynomials_graphing_summary}{52}
\setcounter{lesson_polynomial_inequalities}{53}
\setcounter{lesson_rationals_introduction_and_terminology}{54}
\setcounter{lesson_sign_diagrams_rationals}{55}
\setcounter{lesson_horizontal_asymptotes}{56}
\setcounter{lesson_slant_and_curvilinear_asymptotes}{57}
\setcounter{lesson_vertical_asymptotes}{58}
\setcounter{lesson_holes}{59}
\setcounter{lesson_rationals_graphing_summary}{60}

\begin{document}
{\bf \large Lesson \arabic{lesson_factoring_trinomials_a_neq_1}: Factoring Trinomials with a Leading Coefficient of $a\neq 1$}\phantomsection\label{les:factoring_trinomials_a_neq_1}
\\ CC attribute: \href{http://www.wallace.ccfaculty.org/book/book.html}{\it{Beginning and Intermediate Algebra}} by T. Wallace. 
%\\ CC attribute: \href{http://www.stitz-zeager.com}{\it{College Algebra}} by C. Stitz and J. Zeager. 
\hfill \doclicenseImage[imagewidth=5em]\\
\par
{\bf Objective:} Factor a trinomial with a leading coefficient of $a\neq 1$.\\
\par
{\bf Students will be able to:}
\begin{itemize}
	\item Identify two integer values that add to $b$ and multiply to $a\cdot c$ in a trinomial expression with ordered coefficients $a,b,$ and $c$.
	\item Multiply binomials to verify the accuracy of a factorization.
	\item Recognize the relationship between factoring and expanding an expression.
\end{itemize}
{\bf Prerequisite Knowledge:}
\begin{itemize}
	\item Identifying a greatest common factor.
	\item Factor by grouping.
	\item Application of the distributive property.
	\item Multiplication and division of algebraic expressions.
\end{itemize}
\hrulefill

{\bf Lesson:}\\
\ \par
When factoring trinomials we use the $ac$-method to split the middle (or linear) term and then factor by grouping. The $ac$-method gets its name from the general trinomial expression, $a x^2 + b x + c$, where $a, b,$ and $c$ are the leading coefficient, linear coefficient, and constant term, respectively.\\
\ \par
The $ac$-method is named as such because we will use the product $a \cdot c$ to help find out what two numbers we will need for grouping later on. In the previous lesson, we always found two numbers whose product was equal to $c$, since the leading coefficient $a$ was 1 in our expression (so $a\cdot c=1\cdot c=c$).  Now we will be working with trinomials where $a\neq1$, so we will need to identify two numbers that multiply to $ac$ and add to $b$.  Aside from this adjustment, the process will be the same as before.\\
\ \par
When $a = 1$, we were able to use a shortcut, using the numbers that split the middle coefficient for our factors. As we will see in our examples, this shortcut will not work when $a \neq 1$.  Therefore, we must go through all the steps of grouping in order to factor the expression.
\newpage

{\bf I - Motivating Example(s):}\\
\ \par
{\bf Example:} Factor the given expression.
  \begin{eqnarray*}
    3 x^2 + 11 x + 6 &  & \text{Multiply to} \ a\cdot c \ \text{or} \  3 \cdot 6  =18, \ \text{add to} \ b=11.\\
    3 x^2 + 9 x + 2 x + 6 &  & \text{The numbers are} \ 9
\ \text{and} \ 2, \ \text{split the linear term.}\\
    3 x (x + 3) + 2 (x + 3) &  & \text{Factor by grouping.}\\
    (x + 3) (3 x + 2) &  & \text{Our solution.}
  \end{eqnarray*}

{\bf Example:} Factor the given expression.
  \begin{eqnarray*}
    8 x^2 - 2 x - 15 &  & \text{Multiply to} \ a\cdot c \ \text{or} \ 8\cdot (- 15) = - 120, \ \text{add to} \ b=- 2.\\
    8 x^2 - 12 x + 10 x - 15 &  & \text{The numbers are} \ - 12 \ \text{and} \ 10, \ \text{split the linear term.}\\
    4 x (2 x - 3) + 5 (2 x - 3) &  & \text{Factor by grouping.}\\
    (2 x - 3) (4 x + 5) &  & \text{Our solution.}
  \end{eqnarray*}

{\bf II - Demo/Discussion Problems:}\\
\ \par
Factor each of the given trinomial expressions.
\begin{multicols}{3}
\begin{enumerate}
	\item $10x^2-27x+5$
	\item $4x^2-xy-5y^2$
	\item $18x^3+33x^2-30x$
\end{enumerate}
\end{multicols}
{\bf III - Practice Problems:}\\
\ \par
Factor each of the given trinomial expressions.
\begin{multicols}{3}
  \begin{enumerate}
  \item $7 x^2 - 48 x + 36$
  \item $7 n^2 - 44 n + 12$
  \item $7 b^2 + 15 b + 2$
  \item $7 v^2 - 24 v - 16$
  \item $5 a^2 - 13 a - 28$
  \item $5 n^2 - 7 n - 24$
  \item $2 x^2 - 5 x + 2$
  \item $3 r^2 - 4 r - 4$
  \item $2 x^2 + 19 x + 35$
  \item $7 x^2 + 29 x - 30$
  \item $2 b^2 - b - 3$
  \item $5 x^2 - 26 x + 24$
  \item $5 x^2 + 13 x + 6$
  \item $3 r^2 + 16 r + 21$
  \item $3 x^2 - 17 x + 20$
  \item $3 u^2 + 13 u v - 10 v^2$
  \item $3 x^2 + 17 x y + 10 y^2$
  \item $7 x^2 - 2 x y - 5 y^2$
  \item $5 x^2 + 28 x y - 49 y^2$
  \item $5 u^2 + 31 u v - 28 v^2$
  \item $6 x^2 - 39 x - 21$
  \item $10 a^2 - 54 a - 36$
  \item $21 x^2 - 87 x - 90$
  \item $21 n^2 + 45 n - 54$
  \item $14 x^2 - 60 x + 16$
  \item $4 r^2 + r - 3$
  \item $6 x^2 + 29 x + 20$
  \item $6 p^2 + 11 p - 7$
  \item $4 x^2 - 17 x + 4$
  \item $4 r^2 + 3 r - 7$
  \item $4 x^2 + 9 x y + 2 y^2$
  \item $4 m^2 + 6 m n + 6 n^2$
  \item $4 m^2 - 9 m n - 9 n^2$
  \item $4 x^2 - 6 x y + 30 y^2$
  \item $4 x^2 + 13 x y + 3 y^2$
  \item $18 u^2 - 3 u v - 36 v^2$
  \item $12 x^2 + 62 x y + 70 y^2$
  \item $16 x^2 + 60 x y + 36 y^2$
  \item $24 x^2 - 52 x y + 8 y^2$
  \item $12 x^2 + 50 x y + 28 y^2$
	\end{enumerate}
\end{multicols}
\newpage
\end{document}