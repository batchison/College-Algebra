\documentclass[12pt]{article}
\usepackage[top=1in,left=1in,bottom=1in,right=1in,headsep=2pt]{geometry}	
\usepackage{amssymb,amsmath,amsthm,amsfonts}
\usepackage{chapterfolder,docmute,setspace}
\usepackage{cancel,multicol,tikz,verbatim,framed,polynom,enumitem}
\usepackage[colorlinks, hyperindex, plainpages=false, linkcolor=blue, urlcolor=blue, pdfpagelabels]{hyperref}
% Use the cc-by-nc-sa license for any content linked with Stitz and Zeager's text.  Otherwise, use the cc-by-sa license.
%\usepackage[type={CC},modifier={by-sa},version={4.0},]{doclicense}
\usepackage[type={CC},modifier={by-nc-sa},version={4.0},]{doclicense}

\theoremstyle{definition}
\newtheorem{example}{Example}
\newcommand{\Desmos}{\href{https://www.desmos.com/}{Desmos}}
\setlength{\parindent}{0em}
\setlist{itemsep=0em}
\setlength{\parskip}{0.1em}
% This document is used for ordering of lessons.  If an instructor wishes to change the ordering of assessments, the following steps must be taken:

% 1) Reassign the appropriate numbers for each lesson in the \setcounter commands included in this file.
% 2) Rearrange the \include commands in the master file (the file with 'Course Pack' in the name) to accurately reflect the changes.  
% 3) Rearrange the \items in the measureable_outcomes file to accurately reflect the changes.  Be mindful of page breaks when moving items.
% 4) Re-build all affected files (master file, measureable_outcomes file, and any lesson whose numbering has changed).

%Note: The placement of each \newcounter and \setcounter command reflects the original/default ordering of topics (linears, systems, quadratics, functions, polynomials, rationals).

\newcounter{lesson_solving_linear_equations}
\newcounter{lesson_equations_containing_absolute_values}
\newcounter{lesson_graphing_lines}
\newcounter{lesson_two_forms_of_a_linear_equation}
\newcounter{lesson_parallel_and_perpendicular_lines}
\newcounter{lesson_linear_inequalities}
\newcounter{lesson_compound_inequalities}
\newcounter{lesson_inequalities_containing_absolute_values}
\newcounter{lesson_graphing_systems}
\newcounter{lesson_substitution}
\newcounter{lesson_elimination}
\newcounter{lesson_quadratics_introduction}
\newcounter{lesson_factoring_GCF}
\newcounter{lesson_factoring_grouping}
\newcounter{lesson_factoring_trinomials_a_is_1}
\newcounter{lesson_factoring_trinomials_a_neq_1}
\newcounter{lesson_solving_by_factoring}
\newcounter{lesson_square_roots}
\newcounter{lesson_i_and_complex_numbers}
\newcounter{lesson_vertex_form_and_graphing}
\newcounter{lesson_solve_by_square_roots}
\newcounter{lesson_extracting_square_roots}
\newcounter{lesson_the_discriminant}
\newcounter{lesson_the_quadratic_formula}
\newcounter{lesson_quadratic_inequalities}
\newcounter{lesson_functions_and_relations}
\newcounter{lesson_evaluating_functions}
\newcounter{lesson_finding_domain_and_range_graphically}
\newcounter{lesson_fundamental_functions}
\newcounter{lesson_finding_domain_algebraically}
\newcounter{lesson_solving_functions}
\newcounter{lesson_function_arithmetic}
\newcounter{lesson_composite_functions}
\newcounter{lesson_inverse_functions_definition_and_HLT}
\newcounter{lesson_finding_an_inverse_function}
\newcounter{lesson_transformations_translations}
\newcounter{lesson_transformations_reflections}
\newcounter{lesson_transformations_scalings}
\newcounter{lesson_transformations_summary}
\newcounter{lesson_piecewise_functions}
\newcounter{lesson_functions_containing_absolute_values}
\newcounter{lesson_absolute_as_piecewise}
\newcounter{lesson_polynomials_introduction}
\newcounter{lesson_sign_diagrams_polynomials}
\newcounter{lesson_factoring_quadratic_type}
\newcounter{lesson_factoring_summary}
\newcounter{lesson_polynomial_division}
\newcounter{lesson_synthetic_division}
\newcounter{lesson_end_behavior_polynomials}
\newcounter{lesson_local_behavior_polynomials}
\newcounter{lesson_rational_root_theorem}
\newcounter{lesson_polynomials_graphing_summary}
\newcounter{lesson_polynomial_inequalities}
\newcounter{lesson_rationals_introduction_and_terminology}
\newcounter{lesson_sign_diagrams_rationals}
\newcounter{lesson_horizontal_asymptotes}
\newcounter{lesson_slant_and_curvilinear_asymptotes}
\newcounter{lesson_vertical_asymptotes}
\newcounter{lesson_holes}
\newcounter{lesson_rationals_graphing_summary}

\setcounter{lesson_solving_linear_equations}{1}
\setcounter{lesson_equations_containing_absolute_values}{2}
\setcounter{lesson_graphing_lines}{3}
\setcounter{lesson_two_forms_of_a_linear_equation}{4}
\setcounter{lesson_parallel_and_perpendicular_lines}{5}
\setcounter{lesson_linear_inequalities}{6}
\setcounter{lesson_compound_inequalities}{7}
\setcounter{lesson_inequalities_containing_absolute_values}{8}
\setcounter{lesson_graphing_systems}{9}
\setcounter{lesson_substitution}{10}
\setcounter{lesson_elimination}{11}
\setcounter{lesson_quadratics_introduction}{16}
\setcounter{lesson_factoring_GCF}{17}
\setcounter{lesson_factoring_grouping}{18}
\setcounter{lesson_factoring_trinomials_a_is_1}{19}
\setcounter{lesson_factoring_trinomials_a_neq_1}{20}
\setcounter{lesson_solving_by_factoring}{21}
\setcounter{lesson_square_roots}{22}
\setcounter{lesson_i_and_complex_numbers}{23}
\setcounter{lesson_vertex_form_and_graphing}{24}
\setcounter{lesson_solve_by_square_roots}{25}
\setcounter{lesson_extracting_square_roots}{26}
\setcounter{lesson_the_discriminant}{27}
\setcounter{lesson_the_quadratic_formula}{28}
\setcounter{lesson_quadratic_inequalities}{29}
\setcounter{lesson_functions_and_relations}{12}
\setcounter{lesson_evaluating_functions}{13}
\setcounter{lesson_finding_domain_and_range_graphically}{14}
\setcounter{lesson_fundamental_functions}{15}
\setcounter{lesson_finding_domain_algebraically}{30}
\setcounter{lesson_solving_functions}{31}
\setcounter{lesson_function_arithmetic}{32}
\setcounter{lesson_composite_functions}{33}
\setcounter{lesson_inverse_functions_definition_and_HLT}{34}
\setcounter{lesson_finding_an_inverse_function}{35}
\setcounter{lesson_transformations_translations}{36}
\setcounter{lesson_transformations_reflections}{37}
\setcounter{lesson_transformations_scalings}{38}
\setcounter{lesson_transformations_summary}{39}
\setcounter{lesson_piecewise_functions}{40}
\setcounter{lesson_functions_containing_absolute_values}{41}
\setcounter{lesson_absolute_as_piecewise}{42}
\setcounter{lesson_polynomials_introduction}{43}
\setcounter{lesson_sign_diagrams_polynomials}{44}
\setcounter{lesson_factoring_quadratic_type}{46}
\setcounter{lesson_factoring_summary}{45}
\setcounter{lesson_polynomial_division}{47}
\setcounter{lesson_synthetic_division}{48}
\setcounter{lesson_end_behavior_polynomials}{49}
\setcounter{lesson_local_behavior_polynomials}{50}
\setcounter{lesson_rational_root_theorem}{51}
\setcounter{lesson_polynomials_graphing_summary}{52}
\setcounter{lesson_polynomial_inequalities}{53}
\setcounter{lesson_rationals_introduction_and_terminology}{54}
\setcounter{lesson_sign_diagrams_rationals}{55}
\setcounter{lesson_horizontal_asymptotes}{56}
\setcounter{lesson_slant_and_curvilinear_asymptotes}{57}
\setcounter{lesson_vertical_asymptotes}{58}
\setcounter{lesson_holes}{59}
\setcounter{lesson_rationals_graphing_summary}{60}

\begin{document}
{\bf \large Lesson \arabic{lesson_quadratic_inequalities}: Quadratic Inequalities}
%\\ CC attribute: \href{http://www.wallace.ccfaculty.org/book/book.html}{\it{Beginning and Intermediate Algebra}} by T. Wallace. 
\\ CC attribute: \href{http://www.stitz-zeager.com}{\it{College Algebra}} by C. Stitz and J. Zeager. 
\hfill \doclicenseImage[imagewidth=5em]\\
\par
{\bf Objective:} Solve quadratic inequalities using a sign diagram.\\
\par
{\bf Students will be able to:}
\begin{itemize}
	\item Construct a sign diagram for a quadratic expression.
	\item Solve a quadratic inequality.
	\item Represent a solution to a quadratic inequality using interval notation.
\end{itemize}
{\bf Prerequisite Knowledge:}
\begin{itemize}
	\item Solving and factoring quadratic equations.
	\item Finding $x-$intercepts.
	\item Evaluating an expression for $x$.
\end{itemize}
\hrulefill

{\bf Lesson:}\\
\ \par
Recall that the {\it vertex form} for a quadratic equation is $y=a(x-h)^2+k,$ where $a\neq 0$ and ($h,k$) is the {\it vertex} of the corresponding parabola.  If $a>0$, then the parabola opens upward, and if $a<0$, then the parabola opens downward.  With any quadratic equation, we know that there are three possibilities for the number of roots, namely $0,1,$ or $2$.  Assuming $a>0$, we illustrate these possibilities in the graphs below.

\begin{center}
\begin{tikzpicture}[xscale=0.75,yscale=0.75]
	\draw [<->](-9.5,0) -- coordinate (x axis mid) (-4.5,0) node[below right] {$x$};
	\draw [<->](-2.5,0) -- coordinate (x axis mid) (2.5,0) node[below right] {$x$};
	\draw [<->](4.5,0) -- coordinate (x axis mid) (9.5,0) node[below right] {$x$};
	\draw [<->](-7,-1.5) -- coordinate (x axis mid) (-7,2.5) node[above right] {$y$};
	\draw [<->](0,-1.5) -- coordinate (x axis mid) (0,2.5) node[above right] {$y$};
	\draw [<->](7,-1.5) -- coordinate (x axis mid) (7,2.5) node[above right] {$y$};
	\draw [<->] plot [domain=-0.4:1.9, samples=100] (\x,{1.7*(\x-0.75)^2});
	\draw [<->] plot [domain=6.1:8.4, samples=100] (\x,{1.7*(\x-7.25)^2-1});
	\draw [<->] plot [domain=-6.35:-8.65, samples=100] (\x,{1.7*(\x+7.5)^2+0.5});
	\draw[fill] (0.75,0) circle (0.05);
	\draw[fill] (8.017,0) circle (0.05);
	\draw[fill] (6.483,0) circle (0.05);
 \end{tikzpicture}
\end{center}

Notice also that each of these three graphs lie above the $x$-axis over different intervals.  In the case of the parabola on the left, the entire graph lies above the $x$-axis, whereas the middle parabola lies above the $x$-axis everywhere {\it except} at its $x$-intercept (where $y=0$).  Even more interesting is the parabola on the right, which contains two {\it separate} intervals where its graph lies above the $x$-axis.\\
\ \par
Considering the case where $a<0$, we see three similar graphs as those appearing above, with the only major difference being the opening of each parabola downward instead of upward (when $a>0$).  When we consider again those intervals where each graph lies above the $x$-axis, each parabola exhibits a different behavior than those where $a>0$. 

\begin{center}
\begin{tikzpicture}[xscale=0.75,yscale=0.75]
	\draw [<->](-9.5,0) -- coordinate (x axis mid) (-4.5,0) node[below right] {$x$};
	\draw [<->](-2.5,0) -- coordinate (x axis mid) (2.5,0) node[below right] {$x$};
	\draw [<->](4.5,0) -- coordinate (x axis mid) (9.5,0) node[below right] {$x$};
	\draw [<->](-7,-2.5) -- coordinate (x axis mid) (-7,1.5) node[above right] {$y$};
	\draw [<->](0,-2.5) -- coordinate (x axis mid) (0,1.5) node[above right] {$y$};
	\draw [<->](7,-2.5) -- coordinate (x axis mid) (7,1.5) node[above right] {$y$};
	\draw [<->] plot [domain=-0.4:1.9, samples=100] (\x,{-1.7*(\x-0.75)^2});
	\draw [<->] plot [domain=6.1:8.4, samples=100] (\x,{-1.7*(\x-7.25)^2+1});
	\draw [<->] plot [domain=-6.35:-8.65, samples=100] (\x,{-1.7*(\x+7.5)^2-0.5});
	\draw[fill] (0.75,0) circle (0.05);
	\draw[fill] (8.017,0) circle (0.05);
	\draw[fill] (6.483,0) circle (0.05);
 \end{tikzpicture}
\end{center}

Now, each of the first two graphs have no points that lie above the $x$-axis, whereas the last graph, on the right, lies above the $x$-axis over the interval that is between its $x$-intercepts.\\
\ \par
Each of these six graphs exhibit all of the various possibilities for the {\it sign} of a quadratic expression $ax^2+bx+c$, where $a\neq 0$.  We can determine the general shape of the graph of a quadratic equation (or function) through identification of its roots and construction of a sign diagram.  As a consequence, we will also use a sign diagram to solve a quadratic inequality.\\
\ \par  
{\bf I - Motivating Example(s):}\\
\ \par
{\bf Example:}  Solve the quadratic inequality $x^2-1> 0$.\\
\ \par
Unlike with solving linear inequalities, one should not attempt to solve for the variable $x$, but rather set the given inequality equal to zero and attempt to {\it factor} the resulting expression on the other side.  In doing this, we obtain $(x+1)(x-1)>0$.  Recalling that $x=\pm 1$ are roots of the given expression, we can therefore rule them out of our solution.  Next, we will {\it test} the expression on the left by plugging in three values for $x$: one that is less than $-1$, one that is between $-1$ and $1$, and one that is greater than $1$.
\begin{center}
\begin{tabular}{ccccc}
\underline{Case} & \underline{Test Value} & \underline{Unsimplified} & \underline{Simplified} & \underline{Result}\\
i & $x=-2$ & ($-2+1$)($-2-1$) & ($-$)$\cdot$($-$) & ($+$)\\
ii & $x=0$ & ($0+1$)($0-1$) & ($+$)$\cdot$($-$) & ($-$)\\
iii & $x=2$ & ($2+1$)($2-1$) & ($+$)$\cdot$($+$) & ($+$)
\end{tabular}
\end{center}
Our end result can be summarized in the following {\it sign diagram}.

\begin{center}
\begin{tikzpicture}[xscale=1,yscale=1]
	\draw [<->](-5,0) -- coordinate (x axis mid) (5,0) node[below right] {$x$};
	\draw [-](-2,1) -- coordinate (y axis mid) (-2,-0.25) node[below] {$-1$};
	\draw [-](2,1) -- coordinate (y axis mid) (2,-0.25) node[below] {$1$};
	\draw (-4,-1) node {$x=-2$};
	\draw (0,-1) node {$x=0$};
	\draw (4,-1) node {$x=2$};
	\draw (-4,0.5) node {$+$};
	\draw (0,0.5) node {$-$};
    \draw (4,0.5) node {$+$};
\end{tikzpicture}
\end{center} 
From our sign diagram, we can conclude that $x^2-1>0$ when $x<-1$ or $x>1$.  Using interval notation, our answer is $(-\infty,-1)\cup(1,\infty)$.\newpage
{\bf II - Demo/Discussion Problems:}\\
\ \par
Solve each of the following inequalities.  Express your answers using interval notation.
\begin{enumerate}
	\item $-x^2+4x+5\geq 0$
	\item $x^2-17x\geq -60$
	\item $x^2+4x+1>0$
	\item $-(x-1)^2+9\geq 0$
\end{enumerate}
Solve each of the following inequalities.  Express your answers using interval notation.
\begin{enumerate}
	\item[5.] $x^2+4x+4>0$
	\item[6.] $x^2+4x+4\geq 0$
	\item[7.] $x^2+4x+4<0$
	\item[8.] $x^2+4x+4\leq 0$
\end{enumerate}
{\bf III - Practice Problems:}\\
\ \par
Solve each of the following inequalities.  Express your answers using interval notation.
\begin{multicols}{2}
\begin{enumerate}
	\item $x^2-4>0$
	\item $x^2-5x-6\leq 0$
	\item $-(3x-2)(x+4)\geq 0$
	\item $3x^2+2x-8>0$
	\item $2x^2-16<0$
	\item $-x^2-13x+30\geq 0$
	\item $x^2-5x+6>0$
	\item $2x^2-3x-14\leq 0$
	\item $x^2+2x-9>0$
	\item $-2x^2+12x-18<0$
\end{enumerate}
\end{multicols}
\newpage
\ \newpage
\end{document}