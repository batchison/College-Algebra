\documentclass[12pt]{article}
\usepackage[top=1in,left=1in,bottom=1in,right=1in,headsep=2pt]{geometry}	
\usepackage{amssymb,amsmath,amsthm,amsfonts}
\usepackage{chapterfolder,docmute,setspace}
\usepackage{cancel,multicol,tikz,verbatim,framed,polynom,enumitem}
\usepackage[colorlinks, hyperindex, plainpages=false, linkcolor=blue, urlcolor=blue, pdfpagelabels]{hyperref}
% Use the cc-by-nc-sa license for any content linked with Stitz and Zeager's text.  Otherwise, use the cc-by-sa license.
%\usepackage[type={CC},modifier={by-sa},version={4.0},]{doclicense}
\usepackage[type={CC},modifier={by-nc-sa},version={4.0},]{doclicense}

\theoremstyle{definition}
\newtheorem{example}{Example}
\newcommand{\Desmos}{\href{https://www.desmos.com/}{Desmos}}
\setlength{\parindent}{0em}
\setlist{itemsep=0em}
\setlength{\parskip}{0.1em}
% This document is used for ordering of lessons.  If an instructor wishes to change the ordering of assessments, the following steps must be taken:

% 1) Reassign the appropriate numbers for each lesson in the \setcounter commands included in this file.
% 2) Rearrange the \include commands in the master file (the file with 'Course Pack' in the name) to accurately reflect the changes.  
% 3) Rearrange the \items in the measureable_outcomes file to accurately reflect the changes.  Be mindful of page breaks when moving items.
% 4) Re-build all affected files (master file, measureable_outcomes file, and any lesson whose numbering has changed).

%Note: The placement of each \newcounter and \setcounter command reflects the original/default ordering of topics (linears, systems, quadratics, functions, polynomials, rationals).

\newcounter{lesson_solving_linear_equations}
\newcounter{lesson_equations_containing_absolute_values}
\newcounter{lesson_graphing_lines}
\newcounter{lesson_two_forms_of_a_linear_equation}
\newcounter{lesson_parallel_and_perpendicular_lines}
\newcounter{lesson_linear_inequalities}
\newcounter{lesson_compound_inequalities}
\newcounter{lesson_inequalities_containing_absolute_values}
\newcounter{lesson_graphing_systems}
\newcounter{lesson_substitution}
\newcounter{lesson_elimination}
\newcounter{lesson_quadratics_introduction}
\newcounter{lesson_factoring_GCF}
\newcounter{lesson_factoring_grouping}
\newcounter{lesson_factoring_trinomials_a_is_1}
\newcounter{lesson_factoring_trinomials_a_neq_1}
\newcounter{lesson_solving_by_factoring}
\newcounter{lesson_square_roots}
\newcounter{lesson_i_and_complex_numbers}
\newcounter{lesson_vertex_form_and_graphing}
\newcounter{lesson_solve_by_square_roots}
\newcounter{lesson_extracting_square_roots}
\newcounter{lesson_the_discriminant}
\newcounter{lesson_the_quadratic_formula}
\newcounter{lesson_quadratic_inequalities}
\newcounter{lesson_functions_and_relations}
\newcounter{lesson_evaluating_functions}
\newcounter{lesson_finding_domain_and_range_graphically}
\newcounter{lesson_fundamental_functions}
\newcounter{lesson_finding_domain_algebraically}
\newcounter{lesson_solving_functions}
\newcounter{lesson_function_arithmetic}
\newcounter{lesson_composite_functions}
\newcounter{lesson_inverse_functions_definition_and_HLT}
\newcounter{lesson_finding_an_inverse_function}
\newcounter{lesson_transformations_translations}
\newcounter{lesson_transformations_reflections}
\newcounter{lesson_transformations_scalings}
\newcounter{lesson_transformations_summary}
\newcounter{lesson_piecewise_functions}
\newcounter{lesson_functions_containing_absolute_values}
\newcounter{lesson_absolute_as_piecewise}
\newcounter{lesson_polynomials_introduction}
\newcounter{lesson_sign_diagrams_polynomials}
\newcounter{lesson_factoring_quadratic_type}
\newcounter{lesson_factoring_summary}
\newcounter{lesson_polynomial_division}
\newcounter{lesson_synthetic_division}
\newcounter{lesson_end_behavior_polynomials}
\newcounter{lesson_local_behavior_polynomials}
\newcounter{lesson_rational_root_theorem}
\newcounter{lesson_polynomials_graphing_summary}
\newcounter{lesson_polynomial_inequalities}
\newcounter{lesson_rationals_introduction_and_terminology}
\newcounter{lesson_sign_diagrams_rationals}
\newcounter{lesson_horizontal_asymptotes}
\newcounter{lesson_slant_and_curvilinear_asymptotes}
\newcounter{lesson_vertical_asymptotes}
\newcounter{lesson_holes}
\newcounter{lesson_rationals_graphing_summary}

\setcounter{lesson_solving_linear_equations}{1}
\setcounter{lesson_equations_containing_absolute_values}{2}
\setcounter{lesson_graphing_lines}{3}
\setcounter{lesson_two_forms_of_a_linear_equation}{4}
\setcounter{lesson_parallel_and_perpendicular_lines}{5}
\setcounter{lesson_linear_inequalities}{6}
\setcounter{lesson_compound_inequalities}{7}
\setcounter{lesson_inequalities_containing_absolute_values}{8}
\setcounter{lesson_graphing_systems}{9}
\setcounter{lesson_substitution}{10}
\setcounter{lesson_elimination}{11}
\setcounter{lesson_quadratics_introduction}{16}
\setcounter{lesson_factoring_GCF}{17}
\setcounter{lesson_factoring_grouping}{18}
\setcounter{lesson_factoring_trinomials_a_is_1}{19}
\setcounter{lesson_factoring_trinomials_a_neq_1}{20}
\setcounter{lesson_solving_by_factoring}{21}
\setcounter{lesson_square_roots}{22}
\setcounter{lesson_i_and_complex_numbers}{23}
\setcounter{lesson_vertex_form_and_graphing}{24}
\setcounter{lesson_solve_by_square_roots}{25}
\setcounter{lesson_extracting_square_roots}{26}
\setcounter{lesson_the_discriminant}{27}
\setcounter{lesson_the_quadratic_formula}{28}
\setcounter{lesson_quadratic_inequalities}{29}
\setcounter{lesson_functions_and_relations}{12}
\setcounter{lesson_evaluating_functions}{13}
\setcounter{lesson_finding_domain_and_range_graphically}{14}
\setcounter{lesson_fundamental_functions}{15}
\setcounter{lesson_finding_domain_algebraically}{30}
\setcounter{lesson_solving_functions}{31}
\setcounter{lesson_function_arithmetic}{32}
\setcounter{lesson_composite_functions}{33}
\setcounter{lesson_inverse_functions_definition_and_HLT}{34}
\setcounter{lesson_finding_an_inverse_function}{35}
\setcounter{lesson_transformations_translations}{36}
\setcounter{lesson_transformations_reflections}{37}
\setcounter{lesson_transformations_scalings}{38}
\setcounter{lesson_transformations_summary}{39}
\setcounter{lesson_piecewise_functions}{40}
\setcounter{lesson_functions_containing_absolute_values}{41}
\setcounter{lesson_absolute_as_piecewise}{42}
\setcounter{lesson_polynomials_introduction}{43}
\setcounter{lesson_sign_diagrams_polynomials}{44}
\setcounter{lesson_factoring_quadratic_type}{46}
\setcounter{lesson_factoring_summary}{45}
\setcounter{lesson_polynomial_division}{47}
\setcounter{lesson_synthetic_division}{48}
\setcounter{lesson_end_behavior_polynomials}{49}
\setcounter{lesson_local_behavior_polynomials}{50}
\setcounter{lesson_rational_root_theorem}{51}
\setcounter{lesson_polynomials_graphing_summary}{52}
\setcounter{lesson_polynomial_inequalities}{53}
\setcounter{lesson_rationals_introduction_and_terminology}{54}
\setcounter{lesson_sign_diagrams_rationals}{55}
\setcounter{lesson_horizontal_asymptotes}{56}
\setcounter{lesson_slant_and_curvilinear_asymptotes}{57}
\setcounter{lesson_vertical_asymptotes}{58}
\setcounter{lesson_holes}{59}
\setcounter{lesson_rationals_graphing_summary}{60}

\begin{document}
{\bf \large Lesson \arabic{lesson_slant_and_curvilinear_asymptotes}: Slant and Curvilinear Asymptotes}\phantomsection\label{les:slant_and_curvilinear_asymptotes}
%\\ CC attribute: \href{http://www.wallace.ccfaculty.org/book/book.html}{\it{Beginning and Intermediate Algebra}} by T. Wallace. 
\\ CC attribute: \href{http://www.stitz-zeager.com}{\it{College Algebra}} by C. Stitz and J. Zeager. 
\hfill \doclicenseImage[imagewidth=5em]\\
\par
{\bf Objective:} Identify a slant or curvilinear asymptote in the graph of a rational function.\\
\par
{\bf Students will be able to:}
\begin{itemize}
	\item Determine the existence of a slant or curvilinear asymptote in the graph of a rational function.
	\item Identify the equation of a slant or curvilinear asymptote for the graph of a rational function.
\end{itemize}
{\bf Prerequisite Knowledge:}
\begin{itemize}
	\item Rational function end behavior criteria.
	\item Polynomial and \slash or synthetic division.
\end{itemize}
\hrulefill

{\bf Lesson:}\\
\ \par
For a rational function $$f(x)=\frac{p(x)}{q(x)}=\frac{a_{n}x^{n}+a_{n-1}x^{n-1}+\ldots+a_{1}x+a_{0}}{b_{m}x^{m}+b_{m-1}x^{m-1}+\ldots+b_1x+b_{0}},$$
when $n>m,$ we know that the graph of $f$ will have no horizontal asymptotes.  Depending upon the difference between $n$ and $m,$ however, there is more to discover about the nature of the graph of $f,$ as $x\rightarrow\pm\infty$.\\
\ \par
{\bf I - Motivating Example(s):}
\begin{center}
\begin{multicols}{2}
\begin{tikzpicture}[xscale=0.15,yscale=0.008]
	\draw [<->](-22,0) -- coordinate (x axis mid) (22,0) node[below right] {$x$};
	\draw [<->](0,-200) -- coordinate (y axis mid) (0,500) node[above right] {$y$};
	\draw [dashed, <->](4,-200) -- coordinate (y axis mid) (4,500) node[below] {};
	\draw [<->] plot [domain=-20:3.65, samples=100] (\x,{\x^3/(\x-4)});
	\draw [<->] plot [domain=4.35:18, samples=100] (\x,{\x^3/(\x-4)});
	\foreach \x in {4,8,...,20}
		\draw (\x,1pt) -- (\x,-1pt)	node[anchor=north] {\scriptsize \x};
	\foreach \x in {-20,-16,...,-4}
		\draw (\x,1pt) -- (\x,-1pt)	node[anchor=north] {\scriptsize \x};
	\foreach \y in {100,200,...,500}
		\draw (1pt,\y) -- (-1pt,\y)	node[anchor=east] {\scriptsize \y}; 
	\foreach \y in {-100,-200}
		\draw (1pt,\y) -- (-1pt,\y)	node[anchor=east] {\scriptsize \y};
	\draw (0,-300) node {$f(x)=\dfrac{x^3}{x-4}$};
\end{tikzpicture}

\columnbreak

\begin{tikzpicture}[xscale=0.45,yscale=0.45]
	\draw [<->](-8.25,0) -- coordinate (x axis mid) (8.25,0) node[below right] {$x$};
	\draw [<->](0,-6.25) -- coordinate (y axis mid) (0,6.25) node[above right] {$y$};
	\draw [dashed, <->](1.5,-6.25) -- coordinate (y axis mid) (1.5,6.25) node[above right] {};
	\draw [dashed, <->] plot [domain=-8:8, samples=100] (\x,{0.5*\x+0.25});
	\draw [<->] plot [domain=-8:1, samples=100] (\x,{((\x-3)*(\x+2))/(2*\x-3)});
	\draw [<->] plot [domain=1.865:8, samples=100] (\x,{((\x-3)*(\x+2))/(2*\x-3)});
	\foreach \x in {1,...,8}
		\draw (\x,1pt) -- (\x,-1pt)	node[anchor=north] {\scriptsize \x};
	\foreach \x in {-8,...,-1}
		\draw (\x,1pt) -- (\x,-1pt)	node[anchor=south] {\scriptsize \x};
	\foreach \y in {1,...,6}
		\draw (1pt,\y) -- (-1pt,\y)	node[anchor=east] {\scriptsize \y}; 
	\foreach \y in {-6,...,-1}
		\draw (1pt,\y) -- (-1pt,\y)	node[anchor=west] {\scriptsize \y}; 
	\draw (0,-8) node {$g(x)=\dfrac{x^2-x-6}{2x-3}$};
\end{tikzpicture}
\end{multicols}
\end{center}

In the case of $g(x)=\dfrac{x^2-x-6}{2x-3},$ we see that as $x\rightarrow\pm\infty,$ the graph of $g$ actually approaches a linear asymptote.  Whereas horizontal asymptotes are horizontal lines, having a slope of zero, this new type of linear asymptote has a non-zero slope and is consequently {\it slanted}.  Hence, we say that the graph of $g$ contains a {\it slant} or {\it oblique asymptote}. This occurs when the degree of the numerator is {\it one} more than the denominator, resulting in a {\it slanted} or {\it linear} asymptote.\\
\ \par
On the other hand, the graph of $f(x)=\dfrac{x^3}{x-4}$ does not appear to contain a slant asymptote.  In fact, as $x\rightarrow\pm\infty,$ the graph of $f$ resembles a parabola.  In cases such as these, we could say that the graph of $f$ contains a {\it curvilinear asymptote}.  In other words, the graph of $f$ approaches some identifiable non-linear curve, as $x$ approaches $\pm\infty$. This happens because the difference in degree between the numerator and the denominator is $2$, hence the asymptote is quadratic.\\
\ \par
{\bf II - Demo/Discussion Problems:}\\
\ \par
Use division to identify the equation of the slant or curvilinear asymptote for the graph of each of the following rational functions.  Use \Desmos \ to verify your answers.
\begin{enumerate}
\item $g(x)=\dfrac{x^2-x-6}{2x-3}$
\item $f(x)=\dfrac{-2x^3+x^2-2x+3}{x^2+1}$
\item $h(x)=\dfrac{x^2-5x+7}{x-2}$
\end{enumerate}
{\bf III - Practice Problems:}\\
\ \par
Use division to identify the equation of the slant or curvilinear asymptote for the graph of each of the following rational functions.  Use \Desmos \ to verify your answers.
\begin{enumerate}
\begin{multicols}{2}
\item $a(x)=\dfrac{5x^2-1}{x+3}$
\item $g(x)=\dfrac{x^2-1}{x-4}$
\item $h(x)=\dfrac{x^2-4x-9}{x+2}$
\item $j(x)=\dfrac{18x^3-4x-1}{3x-8}$
\item $k(x)=\dfrac{-x^2+4}{x-9}$
\item $p(x)=\dfrac{-x^2+4}{x-9}$
\item $q(x)=\dfrac{x^5}{x^4-4x^3-2x+1}$
\item $r(x)=\dfrac{x^4-4}{x^3}$
\item $t(x)=\dfrac{15x^2-10}{5x-19}$
\end{multicols}
\end{enumerate}
\newpage
\end{document}