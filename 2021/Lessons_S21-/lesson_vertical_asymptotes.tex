\documentclass[12pt]{article}
\usepackage[top=1in,left=1in,bottom=1in,right=1in,headsep=2pt]{geometry}	
\usepackage{amssymb,amsmath,amsthm,amsfonts}
\usepackage{chapterfolder,docmute,setspace}
\usepackage{cancel,multicol,tikz,verbatim,framed,polynom,enumitem}
\usepackage[colorlinks, hyperindex, plainpages=false, linkcolor=blue, urlcolor=blue, pdfpagelabels]{hyperref}
% Use the cc-by-nc-sa license for any content linked with Stitz and Zeager's text.  Otherwise, use the cc-by-sa license.
%\usepackage[type={CC},modifier={by-sa},version={4.0},]{doclicense}
\usepackage[type={CC},modifier={by-nc-sa},version={4.0},]{doclicense}

\theoremstyle{definition}
\newtheorem{example}{Example}
\newcommand{\Desmos}{\href{https://www.desmos.com/}{Desmos}}
\setlength{\parindent}{0em}
\setlist{itemsep=0em}
\setlength{\parskip}{0.1em}
% This document is used for ordering of lessons.  If an instructor wishes to change the ordering of assessments, the following steps must be taken:

% 1) Reassign the appropriate numbers for each lesson in the \setcounter commands included in this file.
% 2) Rearrange the \include commands in the master file (the file with 'Course Pack' in the name) to accurately reflect the changes.  
% 3) Rearrange the \items in the measureable_outcomes file to accurately reflect the changes.  Be mindful of page breaks when moving items.
% 4) Re-build all affected files (master file, measureable_outcomes file, and any lesson whose numbering has changed).

%Note: The placement of each \newcounter and \setcounter command reflects the original/default ordering of topics (linears, systems, quadratics, functions, polynomials, rationals).

\newcounter{lesson_solving_linear_equations}
\newcounter{lesson_equations_containing_absolute_values}
\newcounter{lesson_graphing_lines}
\newcounter{lesson_two_forms_of_a_linear_equation}
\newcounter{lesson_parallel_and_perpendicular_lines}
\newcounter{lesson_linear_inequalities}
\newcounter{lesson_compound_inequalities}
\newcounter{lesson_inequalities_containing_absolute_values}
\newcounter{lesson_graphing_systems}
\newcounter{lesson_substitution}
\newcounter{lesson_elimination}
\newcounter{lesson_quadratics_introduction}
\newcounter{lesson_factoring_GCF}
\newcounter{lesson_factoring_grouping}
\newcounter{lesson_factoring_trinomials_a_is_1}
\newcounter{lesson_factoring_trinomials_a_neq_1}
\newcounter{lesson_solving_by_factoring}
\newcounter{lesson_square_roots}
\newcounter{lesson_i_and_complex_numbers}
\newcounter{lesson_vertex_form_and_graphing}
\newcounter{lesson_solve_by_square_roots}
\newcounter{lesson_extracting_square_roots}
\newcounter{lesson_the_discriminant}
\newcounter{lesson_the_quadratic_formula}
\newcounter{lesson_quadratic_inequalities}
\newcounter{lesson_functions_and_relations}
\newcounter{lesson_evaluating_functions}
\newcounter{lesson_finding_domain_and_range_graphically}
\newcounter{lesson_fundamental_functions}
\newcounter{lesson_finding_domain_algebraically}
\newcounter{lesson_solving_functions}
\newcounter{lesson_function_arithmetic}
\newcounter{lesson_composite_functions}
\newcounter{lesson_inverse_functions_definition_and_HLT}
\newcounter{lesson_finding_an_inverse_function}
\newcounter{lesson_transformations_translations}
\newcounter{lesson_transformations_reflections}
\newcounter{lesson_transformations_scalings}
\newcounter{lesson_transformations_summary}
\newcounter{lesson_piecewise_functions}
\newcounter{lesson_functions_containing_absolute_values}
\newcounter{lesson_absolute_as_piecewise}
\newcounter{lesson_polynomials_introduction}
\newcounter{lesson_sign_diagrams_polynomials}
\newcounter{lesson_factoring_quadratic_type}
\newcounter{lesson_factoring_summary}
\newcounter{lesson_polynomial_division}
\newcounter{lesson_synthetic_division}
\newcounter{lesson_end_behavior_polynomials}
\newcounter{lesson_local_behavior_polynomials}
\newcounter{lesson_rational_root_theorem}
\newcounter{lesson_polynomials_graphing_summary}
\newcounter{lesson_polynomial_inequalities}
\newcounter{lesson_rationals_introduction_and_terminology}
\newcounter{lesson_sign_diagrams_rationals}
\newcounter{lesson_horizontal_asymptotes}
\newcounter{lesson_slant_and_curvilinear_asymptotes}
\newcounter{lesson_vertical_asymptotes}
\newcounter{lesson_holes}
\newcounter{lesson_rationals_graphing_summary}

\setcounter{lesson_solving_linear_equations}{1}
\setcounter{lesson_equations_containing_absolute_values}{2}
\setcounter{lesson_graphing_lines}{3}
\setcounter{lesson_two_forms_of_a_linear_equation}{4}
\setcounter{lesson_parallel_and_perpendicular_lines}{5}
\setcounter{lesson_linear_inequalities}{6}
\setcounter{lesson_compound_inequalities}{7}
\setcounter{lesson_inequalities_containing_absolute_values}{8}
\setcounter{lesson_graphing_systems}{9}
\setcounter{lesson_substitution}{10}
\setcounter{lesson_elimination}{11}
\setcounter{lesson_quadratics_introduction}{16}
\setcounter{lesson_factoring_GCF}{17}
\setcounter{lesson_factoring_grouping}{18}
\setcounter{lesson_factoring_trinomials_a_is_1}{19}
\setcounter{lesson_factoring_trinomials_a_neq_1}{20}
\setcounter{lesson_solving_by_factoring}{21}
\setcounter{lesson_square_roots}{22}
\setcounter{lesson_i_and_complex_numbers}{23}
\setcounter{lesson_vertex_form_and_graphing}{24}
\setcounter{lesson_solve_by_square_roots}{25}
\setcounter{lesson_extracting_square_roots}{26}
\setcounter{lesson_the_discriminant}{27}
\setcounter{lesson_the_quadratic_formula}{28}
\setcounter{lesson_quadratic_inequalities}{29}
\setcounter{lesson_functions_and_relations}{12}
\setcounter{lesson_evaluating_functions}{13}
\setcounter{lesson_finding_domain_and_range_graphically}{14}
\setcounter{lesson_fundamental_functions}{15}
\setcounter{lesson_finding_domain_algebraically}{30}
\setcounter{lesson_solving_functions}{31}
\setcounter{lesson_function_arithmetic}{32}
\setcounter{lesson_composite_functions}{33}
\setcounter{lesson_inverse_functions_definition_and_HLT}{34}
\setcounter{lesson_finding_an_inverse_function}{35}
\setcounter{lesson_transformations_translations}{36}
\setcounter{lesson_transformations_reflections}{37}
\setcounter{lesson_transformations_scalings}{38}
\setcounter{lesson_transformations_summary}{39}
\setcounter{lesson_piecewise_functions}{40}
\setcounter{lesson_functions_containing_absolute_values}{41}
\setcounter{lesson_absolute_as_piecewise}{42}
\setcounter{lesson_polynomials_introduction}{43}
\setcounter{lesson_sign_diagrams_polynomials}{44}
\setcounter{lesson_factoring_quadratic_type}{46}
\setcounter{lesson_factoring_summary}{45}
\setcounter{lesson_polynomial_division}{47}
\setcounter{lesson_synthetic_division}{48}
\setcounter{lesson_end_behavior_polynomials}{49}
\setcounter{lesson_local_behavior_polynomials}{50}
\setcounter{lesson_rational_root_theorem}{51}
\setcounter{lesson_polynomials_graphing_summary}{52}
\setcounter{lesson_polynomial_inequalities}{53}
\setcounter{lesson_rationals_introduction_and_terminology}{54}
\setcounter{lesson_sign_diagrams_rationals}{55}
\setcounter{lesson_horizontal_asymptotes}{56}
\setcounter{lesson_slant_and_curvilinear_asymptotes}{57}
\setcounter{lesson_vertical_asymptotes}{58}
\setcounter{lesson_holes}{59}
\setcounter{lesson_rationals_graphing_summary}{60}

\begin{document}
{\bf \large Lesson \arabic{lesson_vertical_asymptotes}: Vertical Asymptotes}
%\\ CC attribute: \href{http://www.wallace.ccfaculty.org/book/book.html}{\it{Beginning and Intermediate Algebra}} by T. Wallace. 
\\ CC attribute: \href{http://www.stitz-zeager.com}{\it{College Algebra}} by C. Stitz and J. Zeager. 
\hfill \doclicenseImage[imagewidth=5em]\\
\par
{\bf Objective:} Identify one or more vertical asymptotes in the graph of a rational function.\\
\par
{\bf Students will be able to:}
\begin{itemize}
	\item Identify infinite discontinuities and their corresponding vertical asymptotes in the graph of a rational function.
\end{itemize}
{\bf Prerequisite Knowledge:}
\begin{itemize}
  \item Equations of vertical lines.
	\item Factoring.
	\item The Rational Root Theorem.
	\item Polynomial and \slash or Synthetic Division.
\end{itemize}
\hrulefill

{\bf Lesson:}\\
\ \par
The central idea around a vertical asymptote, say $x=c,$ is that as $x$ approaches the value of $c,$ either from the left or the right, the values for the corresponding function $f(x)$ will approach either $\infty$ or $-\infty$.
\begin{center}
\begin{tabular}{ll}
Approaching from the right: & As $x\rightarrow c^+, \ f(x)\rightarrow\pm\infty.$\\
&\\
Approaching from the left: & As $x\rightarrow c^-, \ f(x)\rightarrow\pm\infty.$
\end{tabular}
\end{center}
We should be clear here, in that when we say $x$ approaches $c$ {\it from the right,} what is meant is that we are evaluating the function at values of $x$ that are getting arbitrarily close go $c,$ but are all {\it greater} than $c,$ i.e., $x>c$.  This is precisely why we can write $x\rightarrow c^+$ in the statement above.  The $+$ in the exponent signifies that $x>c.$ The same can be said for when $x$ approaches $c$ from the left.  The following graph further illustrates this point.
\begin{center}
\begin{tikzpicture}[xscale=0.6,yscale=1.2]
	\draw [<->](-4.25,0) -- coordinate (x axis mid) (4.25,0) node[below right] {$x$};
	\draw [-, dashed](0,-0.5) -- coordinate (y axis mid) (0,3.75) node[below] {};
	\draw [<-] plot [domain=0.30:3, samples=100] (\x,{1/\x});
	\draw [->] plot [domain=-3:-0.30, samples=100] (\x,{-1/\x});
	\draw[fill] (1,1) ellipse (0.5mm and 0.25mm);
  \draw[fill] (2,0.5) ellipse (0.5mm and 0.25mm);
	\draw[fill] (0.5,2) ellipse (0.5mm and 0.25mm);
	\draw[fill] (0.333,3) ellipse (0.5mm and 0.25mm);
	\draw[fill] (-1,1) ellipse (0.5mm and 0.25mm);
  \draw[fill] (-2,0.5) ellipse (0.5mm and 0.25mm);
	\draw[fill] (-0.5,2) ellipse (0.5mm and 0.25mm);
	\draw[fill] (-0.333,3) ellipse (0.5mm and 0.25mm);
	\draw (0,-0.75) node {$x=c$};
	\draw (1.7,0.25) node {$c^+\longleftarrow x$};
	\draw (-1.7,0.25) node {$x\longrightarrow c^-$};
	\draw (2,-0.25) node {$x>c$};
	\draw (-2,-0.25) node {$x<c$};
	\draw (1.5,2) node {$f(x)$};
	\draw (1.5,2.3) node {$\uparrow$};
	\draw (1.5,2.6) node {$\infty$};
	\draw (-1.5,2) node {$f(x)$};
	\draw (-1.5,2.3) node {$\uparrow$};
	\draw (-1.5,2.6) node {$\infty$};
\end{tikzpicture}
\end{center}
Our graph above shows that as $x$ approaches $c$ from either direction, the values for $f(x)$ approach $+\infty$.  If, instead, we reflected the right-hand side of the graph across the $x-$axis, we would say that as $x\rightarrow c^+,$ $f(x)\rightarrow -\infty,$ since the right-hand side would now point downwards.\\
\ \par
Up until this point, we have seen several examples of graphs of rational functions that contain vertical asymptotes.  We are now ready to formally state the condition for the existence of a vertical asymptote.
\begin{center}
\framebox{
\begin{minipage}{0.8\linewidth}
Let $f(x)$ be a rational function and let $g(x)$ represent the simplified expression for $f$.  If $x=c$ is not in the domain of {\it both} $f$ and $g,$ then the graph of $f$ will have a vertical asymptote at $x=c$.
\\ \\
Alternatively, we could say that a vertical asymptote exists as a zero of a factor in the denominator of $f(x)$, {\it as long as that zero is it is not also in the numerator}.  We will learn later that this would result in a hole.
\end{minipage}
}
\end{center}
{\bf II - Demo/Discussion Problems:}\\
\ \par
Identify the equation of any vertical asymptotes for the graph of each of the following rational functions.  Use \Desmos \ to verify your answers.
\begin{enumerate}
\item $f(x)=\dfrac{-2x+4}{x-5}=\dfrac{-2(x-2)}{x-5}$\\
\item $g(x)=\dfrac{x^2+25}{x^2-10x+25}$\\
\item $h(x)=\dfrac{x^2-9x+20}{x^2-3x-10}$
\end{enumerate}
{\bf III - Practice Problems:}\\
\ \par
Identify the equation of any vertical asymptotes for the graph of each of the following rational functions.  Use \Desmos \ to verify your answers.
\begin{enumerate}
\begin{multicols}{3}
\item $a(x)=\dfrac{5x^2-1}{x+3}$
\item $g(x)=\dfrac{x^2-1}{x-4}$
\item $h(x)=\dfrac{x^2-4x-9}{x+2}$
\item $j(x)=\dfrac{18x^3-4x-1}{x^2-4}$
\item $k(x)=\dfrac{-x^2+4}{x-9}$
\item $p(x)=\dfrac{-x^2+4}{x^2+9}$
\item $q(x)=\dfrac{x^5}{x(x-5)}$
\item $r(x)=\dfrac{x^2-11x+30}{x^2+10x+24}$
\item $t(x)=\dfrac{15x^2-10}{5x-7}$
\end{multicols}
\end{enumerate}
\newpage
\end{document}