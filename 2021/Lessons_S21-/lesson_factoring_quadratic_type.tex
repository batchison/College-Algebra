\documentclass[12pt]{article}
\usepackage[top=1in,left=1in,bottom=1in,right=1in,headsep=2pt]{geometry}	
\usepackage{amssymb,amsmath,amsthm,amsfonts}
\usepackage{chapterfolder,docmute,setspace}
\usepackage{cancel,multicol,tikz,verbatim,framed,polynom,enumitem}
\usepackage[colorlinks, hyperindex, plainpages=false, linkcolor=blue, urlcolor=blue, pdfpagelabels]{hyperref}
% Use the cc-by-nc-sa license for any content linked with Stitz and Zeager's text.  Otherwise, use the cc-by-sa license.
\usepackage[type={CC},modifier={by-sa},version={4.0},]{doclicense}
%\usepackage[type={CC},modifier={by-nc-sa},version={4.0},]{doclicense}

\theoremstyle{definition}
\newtheorem{example}{Example}
\newcommand{\Desmos}{\href{https://www.desmos.com/}{Desmos}}
\setlength{\parindent}{0em}
\setlist{itemsep=0em}
\setlength{\parskip}{0.1em}
% This document is used for ordering of lessons.  If an instructor wishes to change the ordering of assessments, the following steps must be taken:

% 1) Reassign the appropriate numbers for each lesson in the \setcounter commands included in this file.
% 2) Rearrange the \include commands in the master file (the file with 'Course Pack' in the name) to accurately reflect the changes.  
% 3) Rearrange the \items in the measureable_outcomes file to accurately reflect the changes.  Be mindful of page breaks when moving items.
% 4) Re-build all affected files (master file, measureable_outcomes file, and any lesson whose numbering has changed).

%Note: The placement of each \newcounter and \setcounter command reflects the original/default ordering of topics (linears, systems, quadratics, functions, polynomials, rationals).

\newcounter{lesson_solving_linear_equations}
\newcounter{lesson_equations_containing_absolute_values}
\newcounter{lesson_graphing_lines}
\newcounter{lesson_two_forms_of_a_linear_equation}
\newcounter{lesson_parallel_and_perpendicular_lines}
\newcounter{lesson_linear_inequalities}
\newcounter{lesson_compound_inequalities}
\newcounter{lesson_inequalities_containing_absolute_values}
\newcounter{lesson_graphing_systems}
\newcounter{lesson_substitution}
\newcounter{lesson_elimination}
\newcounter{lesson_quadratics_introduction}
\newcounter{lesson_factoring_GCF}
\newcounter{lesson_factoring_grouping}
\newcounter{lesson_factoring_trinomials_a_is_1}
\newcounter{lesson_factoring_trinomials_a_neq_1}
\newcounter{lesson_solving_by_factoring}
\newcounter{lesson_square_roots}
\newcounter{lesson_i_and_complex_numbers}
\newcounter{lesson_vertex_form_and_graphing}
\newcounter{lesson_solve_by_square_roots}
\newcounter{lesson_extracting_square_roots}
\newcounter{lesson_the_discriminant}
\newcounter{lesson_the_quadratic_formula}
\newcounter{lesson_quadratic_inequalities}
\newcounter{lesson_functions_and_relations}
\newcounter{lesson_evaluating_functions}
\newcounter{lesson_finding_domain_and_range_graphically}
\newcounter{lesson_fundamental_functions}
\newcounter{lesson_finding_domain_algebraically}
\newcounter{lesson_solving_functions}
\newcounter{lesson_function_arithmetic}
\newcounter{lesson_composite_functions}
\newcounter{lesson_inverse_functions_definition_and_HLT}
\newcounter{lesson_finding_an_inverse_function}
\newcounter{lesson_transformations_translations}
\newcounter{lesson_transformations_reflections}
\newcounter{lesson_transformations_scalings}
\newcounter{lesson_transformations_summary}
\newcounter{lesson_piecewise_functions}
\newcounter{lesson_functions_containing_absolute_values}
\newcounter{lesson_absolute_as_piecewise}
\newcounter{lesson_polynomials_introduction}
\newcounter{lesson_sign_diagrams_polynomials}
\newcounter{lesson_factoring_quadratic_type}
\newcounter{lesson_factoring_summary}
\newcounter{lesson_polynomial_division}
\newcounter{lesson_synthetic_division}
\newcounter{lesson_end_behavior_polynomials}
\newcounter{lesson_local_behavior_polynomials}
\newcounter{lesson_rational_root_theorem}
\newcounter{lesson_polynomials_graphing_summary}
\newcounter{lesson_polynomial_inequalities}
\newcounter{lesson_rationals_introduction_and_terminology}
\newcounter{lesson_sign_diagrams_rationals}
\newcounter{lesson_horizontal_asymptotes}
\newcounter{lesson_slant_and_curvilinear_asymptotes}
\newcounter{lesson_vertical_asymptotes}
\newcounter{lesson_holes}
\newcounter{lesson_rationals_graphing_summary}

\setcounter{lesson_solving_linear_equations}{1}
\setcounter{lesson_equations_containing_absolute_values}{2}
\setcounter{lesson_graphing_lines}{3}
\setcounter{lesson_two_forms_of_a_linear_equation}{4}
\setcounter{lesson_parallel_and_perpendicular_lines}{5}
\setcounter{lesson_linear_inequalities}{6}
\setcounter{lesson_compound_inequalities}{7}
\setcounter{lesson_inequalities_containing_absolute_values}{8}
\setcounter{lesson_graphing_systems}{9}
\setcounter{lesson_substitution}{10}
\setcounter{lesson_elimination}{11}
\setcounter{lesson_quadratics_introduction}{16}
\setcounter{lesson_factoring_GCF}{17}
\setcounter{lesson_factoring_grouping}{18}
\setcounter{lesson_factoring_trinomials_a_is_1}{19}
\setcounter{lesson_factoring_trinomials_a_neq_1}{20}
\setcounter{lesson_solving_by_factoring}{21}
\setcounter{lesson_square_roots}{22}
\setcounter{lesson_i_and_complex_numbers}{23}
\setcounter{lesson_vertex_form_and_graphing}{24}
\setcounter{lesson_solve_by_square_roots}{25}
\setcounter{lesson_extracting_square_roots}{26}
\setcounter{lesson_the_discriminant}{27}
\setcounter{lesson_the_quadratic_formula}{28}
\setcounter{lesson_quadratic_inequalities}{29}
\setcounter{lesson_functions_and_relations}{12}
\setcounter{lesson_evaluating_functions}{13}
\setcounter{lesson_finding_domain_and_range_graphically}{14}
\setcounter{lesson_fundamental_functions}{15}
\setcounter{lesson_finding_domain_algebraically}{30}
\setcounter{lesson_solving_functions}{31}
\setcounter{lesson_function_arithmetic}{32}
\setcounter{lesson_composite_functions}{33}
\setcounter{lesson_inverse_functions_definition_and_HLT}{34}
\setcounter{lesson_finding_an_inverse_function}{35}
\setcounter{lesson_transformations_translations}{36}
\setcounter{lesson_transformations_reflections}{37}
\setcounter{lesson_transformations_scalings}{38}
\setcounter{lesson_transformations_summary}{39}
\setcounter{lesson_piecewise_functions}{40}
\setcounter{lesson_functions_containing_absolute_values}{41}
\setcounter{lesson_absolute_as_piecewise}{42}
\setcounter{lesson_polynomials_introduction}{43}
\setcounter{lesson_sign_diagrams_polynomials}{44}
\setcounter{lesson_factoring_quadratic_type}{46}
\setcounter{lesson_factoring_summary}{45}
\setcounter{lesson_polynomial_division}{47}
\setcounter{lesson_synthetic_division}{48}
\setcounter{lesson_end_behavior_polynomials}{49}
\setcounter{lesson_local_behavior_polynomials}{50}
\setcounter{lesson_rational_root_theorem}{51}
\setcounter{lesson_polynomials_graphing_summary}{52}
\setcounter{lesson_polynomial_inequalities}{53}
\setcounter{lesson_rationals_introduction_and_terminology}{54}
\setcounter{lesson_sign_diagrams_rationals}{55}
\setcounter{lesson_horizontal_asymptotes}{56}
\setcounter{lesson_slant_and_curvilinear_asymptotes}{57}
\setcounter{lesson_vertical_asymptotes}{58}
\setcounter{lesson_holes}{59}
\setcounter{lesson_rationals_graphing_summary}{60}

\begin{document}
{\bf \large Lesson \arabic{lesson_factoring_quadratic_type}: Factoring Polynomials of Quadratic Type}\phantomsection\label{les:factoring_quadratic_type}
\\ CC attribute: \href{http://www.wallace.ccfaculty.org/book/book.html}{\it{Beginning and Intermediate Algebra}} by T. Wallace. 
%\\ CC attribute: \href{http://www.stitz-zeager.com}{\it{College Algebra}} by C. Stitz and J. Zeager. 
\hfill \doclicenseImage[imagewidth=5em]\\
\par
{\bf Objective:} Recognize and factor a polynomial expression of quadratic type.\\
\par
{\bf Students will be able to:}
\begin{itemize}
	\item Identify polynomial expressions of quadratic type.
	\item Apply the appropriate substitution, $y=x^n,$ to obtain a factorable quadratic polynomial in terms of $y$.
	\item Find the set of roots of a polynomial expression of quadratic type.
\end{itemize}
{\bf Prerequisite Knowledge:}
\begin{itemize}
	\item Properties of exponents.
	\item Factoring a difference of squares.
	\item The $ac$-method for factoring quadratics.
\end{itemize}
\hrulefill

{\bf Lesson:}\\
\ \par
Recall that a quadratic expression in terms of a variable $x$ is an expression of the form $$ax^2+bx+c.$$
If $y$ is any algebraic expression, we say that the expression $$ay^2+by+c$$ is an expression of {\it quadratic type}.\\
\ \par
In just about every case we will see, we will consider $y$ as a power of $x,$ $y=x^n,$ so that our expression of quadratic type will appear as follows.
\begin{center}
\framebox{
\begin{minipage}{0.75\linewidth}
\begin{center}
Quadratic Type:
$$ax^{2n}+bx^n+c \ = \ a\left[x^n\right]^2+b\left[x^n\right]+c$$
\end{center}
\end{minipage}
}
\end{center}
If $y=x^3,$ then the expression $$ay^2+by+c=ax^6+bx^3+c$$
would be an expression of quadratic type.
\newpage
Similarly, if $y=x^4,$ then the expression $$ay^2+by+c=ax^8+bx^4+c$$
would be an expression of quadratic type.\\
\ \par
In each of these last two examples, notice the exponential pattern, where the middle term has an exponent that is half that of the leading term's.  This will always be apparent, as long as the middle coefficient $b$ is nonzero.\\
\ \par
By viewing certain expressions as quadratic type, we can often apply more familiar methods, such as the $ac-$method, to obtain a complete factorization.\\
\ \par
{\bf I - Motivating Example(s):}\\
\ \par
{\bf Example:} If we let $y=x^2,$ then the difference of fourth powers $x^4-16$ can be rewritten as a difference of squares,
$y^2-4^2,$ leading us to the complete factorization over the real numbers shown below.
\begin{equation*}
	\begin{split}
		x^4-16&=\left(x^2\right)^2-4^2\\
		&=y^2-4^2, \ \ y=x^2\\
		&=\left(y+4\right)\left(y-4\right)\\
		&=\left(x^2+4\right)\left(x^2-4\right)\\
		&=\left(x^2+4\right)\left(x+2\right)\left(x-2\right)
	\end{split}
\end{equation*}

{\bf Example:} The trinomial expression $x^4+2x^2-24$ exhibits quadratic type characteristics, since the degree of four is double the exponent appearing in the middle term.  Consequently, we will let $y=x^2$ and rewrite the expression in terms of $y$.
$$y^2+2y-24$$
Applying the $ac-$method, we see the following.
\begin{equation*}
\begin{split}
y^2+2y-24 & = y^2+6y-4y-24\\
&=y\left(y+6\right)-4\left(y+6\right)\\
&=\left(y+6\right)\left(y-4\right)
\end{split}
\end{equation*}
Substituting back for $x,$ we have $\left(x^2+6\right)\left(x^2-4\right)$.  The first factor is a sum of squares, which is irreducible over the reals.  The second factor of $x^2-4$ is a difference of perfect squares, which we know is factorable as $\left(x+2\right)\left(x-2\right)$. Our final factorization is
$$x^4+2x^2-24=\left(x^2+6\right)\left(x+2\right)\left(x-2\right).$$
\newpage
{\bf II - Demo/Discussion Problems:}\\
\ \par
Completely factor each of the following polynomials over the real numbers and identify the set of all real roots.
\begin{enumerate}
\item $x^4-12x^2+27$
\item $x^8+2x^4-24$
\item $x^6+2x^3-24$
\item $x^4-49$
\item $x^6-4x^3-5$
\end{enumerate}
{\bf III - Practice Problems:}\\
\ \par
Completely factor each of the following polynomials over the real numbers and identify the set of all real roots.
\begin{multicols}{2}
\begin{enumerate}
  \item  $x^4 +13x^2+40$
  \item  $x^4-5x^2+4$
  \item  $x^4 -17x^2+16$
  \item  $x^4 -3x^2-40$
  \item  $3x^4 -32x^2+45$
	\item  $x^4 +x^2-12$
  \item  $x^4 -3x^2-10$
  \item  $x^6 -82x^3+81$
  \item  $8x^4 +2x^2-3$
  \item  $2x^4 -19x^2+9$
\end{enumerate}
\end{multicols}
\newpage
\ \newpage
\end{document}