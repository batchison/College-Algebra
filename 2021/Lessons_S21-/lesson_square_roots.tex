\documentclass[12pt]{article}
\usepackage[top=1in,left=1in,bottom=1in,right=1in,headsep=2pt]{geometry}	
\usepackage{amssymb,amsmath,amsthm,amsfonts}
\usepackage{chapterfolder,docmute,setspace}
\usepackage{cancel,multicol,tikz,verbatim,framed,polynom,enumitem}
\usepackage[colorlinks, hyperindex, plainpages=false, linkcolor=blue, urlcolor=blue, pdfpagelabels]{hyperref}
% Use the cc-by-nc-sa license for any content linked with Stitz and Zeager's text.  Otherwise, use the cc-by-sa license.
\usepackage[type={CC},modifier={by-sa},version={4.0},]{doclicense}
%\usepackage[type={CC},modifier={by-nc-sa},version={4.0},]{doclicense}

\theoremstyle{definition}
\newtheorem{example}{Example}
\newcommand{\Desmos}{\href{https://www.desmos.com/}{Desmos}}
\setlength{\parindent}{0em}
\setlist{itemsep=0em}
\setlength{\parskip}{0.1em}
% This document is used for ordering of lessons.  If an instructor wishes to change the ordering of assessments, the following steps must be taken:

% 1) Reassign the appropriate numbers for each lesson in the \setcounter commands included in this file.
% 2) Rearrange the \include commands in the master file (the file with 'Course Pack' in the name) to accurately reflect the changes.  
% 3) Rearrange the \items in the measureable_outcomes file to accurately reflect the changes.  Be mindful of page breaks when moving items.
% 4) Re-build all affected files (master file, measureable_outcomes file, and any lesson whose numbering has changed).

%Note: The placement of each \newcounter and \setcounter command reflects the original/default ordering of topics (linears, systems, quadratics, functions, polynomials, rationals).

\newcounter{lesson_solving_linear_equations}
\newcounter{lesson_equations_containing_absolute_values}
\newcounter{lesson_graphing_lines}
\newcounter{lesson_two_forms_of_a_linear_equation}
\newcounter{lesson_parallel_and_perpendicular_lines}
\newcounter{lesson_linear_inequalities}
\newcounter{lesson_compound_inequalities}
\newcounter{lesson_inequalities_containing_absolute_values}
\newcounter{lesson_graphing_systems}
\newcounter{lesson_substitution}
\newcounter{lesson_elimination}
\newcounter{lesson_quadratics_introduction}
\newcounter{lesson_factoring_GCF}
\newcounter{lesson_factoring_grouping}
\newcounter{lesson_factoring_trinomials_a_is_1}
\newcounter{lesson_factoring_trinomials_a_neq_1}
\newcounter{lesson_solving_by_factoring}
\newcounter{lesson_square_roots}
\newcounter{lesson_i_and_complex_numbers}
\newcounter{lesson_vertex_form_and_graphing}
\newcounter{lesson_solve_by_square_roots}
\newcounter{lesson_extracting_square_roots}
\newcounter{lesson_the_discriminant}
\newcounter{lesson_the_quadratic_formula}
\newcounter{lesson_quadratic_inequalities}
\newcounter{lesson_functions_and_relations}
\newcounter{lesson_evaluating_functions}
\newcounter{lesson_finding_domain_and_range_graphically}
\newcounter{lesson_fundamental_functions}
\newcounter{lesson_finding_domain_algebraically}
\newcounter{lesson_solving_functions}
\newcounter{lesson_function_arithmetic}
\newcounter{lesson_composite_functions}
\newcounter{lesson_inverse_functions_definition_and_HLT}
\newcounter{lesson_finding_an_inverse_function}
\newcounter{lesson_transformations_translations}
\newcounter{lesson_transformations_reflections}
\newcounter{lesson_transformations_scalings}
\newcounter{lesson_transformations_summary}
\newcounter{lesson_piecewise_functions}
\newcounter{lesson_functions_containing_absolute_values}
\newcounter{lesson_absolute_as_piecewise}
\newcounter{lesson_polynomials_introduction}
\newcounter{lesson_sign_diagrams_polynomials}
\newcounter{lesson_factoring_quadratic_type}
\newcounter{lesson_factoring_summary}
\newcounter{lesson_polynomial_division}
\newcounter{lesson_synthetic_division}
\newcounter{lesson_end_behavior_polynomials}
\newcounter{lesson_local_behavior_polynomials}
\newcounter{lesson_rational_root_theorem}
\newcounter{lesson_polynomials_graphing_summary}
\newcounter{lesson_polynomial_inequalities}
\newcounter{lesson_rationals_introduction_and_terminology}
\newcounter{lesson_sign_diagrams_rationals}
\newcounter{lesson_horizontal_asymptotes}
\newcounter{lesson_slant_and_curvilinear_asymptotes}
\newcounter{lesson_vertical_asymptotes}
\newcounter{lesson_holes}
\newcounter{lesson_rationals_graphing_summary}

\setcounter{lesson_solving_linear_equations}{1}
\setcounter{lesson_equations_containing_absolute_values}{2}
\setcounter{lesson_graphing_lines}{3}
\setcounter{lesson_two_forms_of_a_linear_equation}{4}
\setcounter{lesson_parallel_and_perpendicular_lines}{5}
\setcounter{lesson_linear_inequalities}{6}
\setcounter{lesson_compound_inequalities}{7}
\setcounter{lesson_inequalities_containing_absolute_values}{8}
\setcounter{lesson_graphing_systems}{9}
\setcounter{lesson_substitution}{10}
\setcounter{lesson_elimination}{11}
\setcounter{lesson_quadratics_introduction}{16}
\setcounter{lesson_factoring_GCF}{17}
\setcounter{lesson_factoring_grouping}{18}
\setcounter{lesson_factoring_trinomials_a_is_1}{19}
\setcounter{lesson_factoring_trinomials_a_neq_1}{20}
\setcounter{lesson_solving_by_factoring}{21}
\setcounter{lesson_square_roots}{22}
\setcounter{lesson_i_and_complex_numbers}{23}
\setcounter{lesson_vertex_form_and_graphing}{24}
\setcounter{lesson_solve_by_square_roots}{25}
\setcounter{lesson_extracting_square_roots}{26}
\setcounter{lesson_the_discriminant}{27}
\setcounter{lesson_the_quadratic_formula}{28}
\setcounter{lesson_quadratic_inequalities}{29}
\setcounter{lesson_functions_and_relations}{12}
\setcounter{lesson_evaluating_functions}{13}
\setcounter{lesson_finding_domain_and_range_graphically}{14}
\setcounter{lesson_fundamental_functions}{15}
\setcounter{lesson_finding_domain_algebraically}{30}
\setcounter{lesson_solving_functions}{31}
\setcounter{lesson_function_arithmetic}{32}
\setcounter{lesson_composite_functions}{33}
\setcounter{lesson_inverse_functions_definition_and_HLT}{34}
\setcounter{lesson_finding_an_inverse_function}{35}
\setcounter{lesson_transformations_translations}{36}
\setcounter{lesson_transformations_reflections}{37}
\setcounter{lesson_transformations_scalings}{38}
\setcounter{lesson_transformations_summary}{39}
\setcounter{lesson_piecewise_functions}{40}
\setcounter{lesson_functions_containing_absolute_values}{41}
\setcounter{lesson_absolute_as_piecewise}{42}
\setcounter{lesson_polynomials_introduction}{43}
\setcounter{lesson_sign_diagrams_polynomials}{44}
\setcounter{lesson_factoring_quadratic_type}{46}
\setcounter{lesson_factoring_summary}{45}
\setcounter{lesson_polynomial_division}{47}
\setcounter{lesson_synthetic_division}{48}
\setcounter{lesson_end_behavior_polynomials}{49}
\setcounter{lesson_local_behavior_polynomials}{50}
\setcounter{lesson_rational_root_theorem}{51}
\setcounter{lesson_polynomials_graphing_summary}{52}
\setcounter{lesson_polynomial_inequalities}{53}
\setcounter{lesson_rationals_introduction_and_terminology}{54}
\setcounter{lesson_sign_diagrams_rationals}{55}
\setcounter{lesson_horizontal_asymptotes}{56}
\setcounter{lesson_slant_and_curvilinear_asymptotes}{57}
\setcounter{lesson_vertical_asymptotes}{58}
\setcounter{lesson_holes}{59}
\setcounter{lesson_rationals_graphing_summary}{60}

\begin{document}
{\bf \large Lesson \arabic{lesson_square_roots}: Square Roots}%\\ CC attribute: \href{http://www.wallace.ccfaculty.org/book/book.html}{\it{Beginning and Intermediate Algebra}} by T. Wallace. 
%\\ CC attribute: \href{http://www.stitz-zeager.com}{\it{College Algebra}} by C. Stitz and J. Zeager. 
\hfill \doclicenseImage[imagewidth=5em]\\
\par
{\bf Objective:} Simplify and evaluate expressions involving square roots.\\
\par
{\bf Students will be able to:}
\begin{itemize}
	\item Simplify square roots.
\end{itemize}
{\bf Prerequisite Knowledge:}
\begin{itemize}
	\item Prime factorization of a number.
	\item Perfect square numbers.
	\item Find square roots of square numbers.
\end{itemize}
\hrulefill

{\bf Lesson:}\\
{\bf I - Motivating Example(s):}\\
\ \par
{\bf Example:}
  \[ \begin{array}{|c|c|}
       \hline
       \sqrt{0} = 0 & \sqrt{121} = 11\\
       \hline
       \sqrt{1} = 1 & \sqrt{625} = 25\\
       \hline
       \sqrt{4} = 2 & \sqrt{- 81} = \text{undefined}\\
       \hline
     \end{array} \]

The final example of $\sqrt{- 81}$ is currently considered to be undefined, since the square root of a negative number does not equal a real number. This is because if we square a positive or a negative number, the answer will be positive, not to mention that $0^2=0$. Thus we can only take square roots of nonnegative numbers (positive numbers or zero).  In a future lesson, we will define a method we can use to work with and evaluate negative square roots.  For now we will simply say they are undefined.\\
\ \par
Not all numbers have a ``nice'' (or {\it rational}) square root. For example, if we found
$\sqrt{8}$ on our calculator, the answer would be
2.828427124746190097...%603377448419...
, and even this number is a rounded approximation of the square root. To be as accurate as possible, we will never
use the calculator to find decimal approximations of square roots. Instead we will express roots in simplest radical form. We will do this using a property known as the product rule of radicals (in this case, square roots).
\[ {\bf Product \ Rule \ of \ Square \ Roots: \ }
   \sqrt{a \cdot b} = \sqrt{a} \cdot \sqrt{b} \]
More generally,
\[ {\bf Product \ Rule \ of \ Radicals: \ }
   \sqrt[n]{a \cdot b} = \sqrt[n]{a} \cdot \sqrt[n]{b} \]
We can use the product rule of square roots to simplify an expression such as $\sqrt{180}=\sqrt{36 \cdot
5}$ by splitting it into two roots, $\sqrt{36} \cdot \sqrt{5}$, and simplifying the first root, $6 \sqrt{5}$. The trick in this process is being able to recognize that an expression like $\sqrt{180}$ may be rewritten as $\sqrt{36 \cdot 5}$, since $180=36\cdot 5$. In the case of $\sqrt{8}$, we may write $\sqrt{8}=\sqrt{4}\cdot\sqrt{2}=2\sqrt{2}$.\\
\ \par
There are several ways of applying the product rule of square roots. The most common and, with a bit of practice, fastest method is to find perfect squares that divide nicely into the radicand (the number under the radical). This is demonstrated in the next example.\\
\ \par
{\bf Example:} Completely simplify the given radical.
  \begin{eqnarray*}
    \sqrt{75} &  & 75 \text{\ is divisible by \ } 25, 
    \text{\ a perfect square \ }\\
    \sqrt{25 \cdot 3} &  & \text{Split into factors}\\
    \sqrt{25} \cdot \sqrt{3} &  & \text{Product rule, take the square root of \ } 25\\
    5 \sqrt{3} &  & \text{Our solution}
  \end{eqnarray*}
{\bf II - Demo/Discussion Problems:}\\
\ \par
Completely simplify the given radicals.
\begin{multicols}{4}
\begin{enumerate}
	\item $5\sqrt{63}$
	\item $\sqrt{72}$
  \item $- 5~\sqrt[]{72 x^3 y^4}$
	\item $-5\sqrt{18x^4y^6z^{10}}$
\end{enumerate}
\end{multicols}
\ \par
{\bf III - Practice Problems:}\\
\ \par
Completely simplify each of the following square roots completely.
\begin{multicols}{4}
\begin{enumerate}
	\item $\sqrt[]{245}$
	\item $\sqrt[]{125}$
  \item $\sqrt[]{36}$
  \item $\sqrt[]{196}$
  \item $\sqrt[]{12}$
  \item $\sqrt[]{72}$
  \item $3~\sqrt[]{12}$
  \item $5~\sqrt[]{32}$
  \item $6~\sqrt[]{128}$
  \item $7~\sqrt[]{128}$
  \item $- 8~\sqrt[]{392}$
  \item $- 7~\sqrt[]{63}$
  \item $\sqrt[]{192 n}$
  \item $\sqrt[]{343 b}$
  \item $\sqrt[]{196 v^2}$
  \item $\sqrt[]{100 n^3}$
  \item $\sqrt[]{252 x^2}$
  \item $\sqrt[]{200 a^3}$
  \item $-~\sqrt[]{100 k^4}$
  \item $- 4~\sqrt[]{175 p^4}$
  \item $- 7~\sqrt[]{64 x^4}$
  \item $- 2~\sqrt[]{128 n}$
  \item $- 5~\sqrt[]{36 m}$
  \item $8~\sqrt[]{112 p^2}$
  \item $\sqrt[]{45 x^2 y^2}$
  \item $\sqrt[]{72 a^3 b^4}$
  \item $\sqrt[]{16 x^3 y^3}$
  \item $\sqrt[]{512 a^4 b^2}$
  \item $\sqrt[]{320 x^4 y^4}$
  \item $\sqrt[]{512 m^4 n^3}$
  \item $6~\sqrt[]{80 x^{} y^2}$
  \item $8~\sqrt[]{98 m n}$
  \item $5~\sqrt[]{245 x^2 y^3}$
  \item $2~\sqrt[]{72 x^2 y^2}$
  \item $- 2~\sqrt[]{180 u^3 v}$
  \item $- 5~\sqrt[]{96 x^4 y^3}$
  \item $- 8~\sqrt[]{180 x^4 y^2 z^4}$
  \item $6~\sqrt[]{50 a^4 b c^2}$
  \item $2~\sqrt[]{80 h j^4 k}$
  \item $-~\sqrt[]{32 x y^2 z^3}$
  \item $- 4~\sqrt[]{54 m n p^2}$
  \item $- 8~\sqrt[]{32 m^2 p^4 q}$
\end{enumerate}
\end{multicols}
\newpage
\end{document}