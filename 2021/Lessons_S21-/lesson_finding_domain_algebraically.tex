\documentclass[12pt]{article}
\usepackage[top=1in,left=1in,bottom=1in,right=1in,headsep=2pt]{geometry}	
\usepackage{amssymb,amsmath,amsthm,amsfonts}
\usepackage{chapterfolder,docmute,setspace}
\usepackage{cancel,multicol,tikz,verbatim,framed,polynom,enumitem}
\usepackage[colorlinks, hyperindex, plainpages=false, linkcolor=blue, urlcolor=blue, pdfpagelabels]{hyperref}
% Use the cc-by-nc-sa license for any content linked with Stitz and Zeager's text.  Otherwise, use the cc-by-sa license.
%\usepackage[type={CC},modifier={by-sa},version={4.0},]{doclicense}
\usepackage[type={CC},modifier={by-nc-sa},version={4.0},]{doclicense}

\theoremstyle{definition}
\newtheorem{example}{Example}
\newcommand{\Desmos}{\href{https://www.desmos.com/}{Desmos}}
\setlength{\parindent}{0em}
\setlist{itemsep=0em}
\setlength{\parskip}{0.1em}
% This document is used for ordering of lessons.  If an instructor wishes to change the ordering of assessments, the following steps must be taken:

% 1) Reassign the appropriate numbers for each lesson in the \setcounter commands included in this file.
% 2) Rearrange the \include commands in the master file (the file with 'Course Pack' in the name) to accurately reflect the changes.  
% 3) Rearrange the \items in the measureable_outcomes file to accurately reflect the changes.  Be mindful of page breaks when moving items.
% 4) Re-build all affected files (master file, measureable_outcomes file, and any lesson whose numbering has changed).

%Note: The placement of each \newcounter and \setcounter command reflects the original/default ordering of topics (linears, systems, quadratics, functions, polynomials, rationals).

\newcounter{lesson_solving_linear_equations}
\newcounter{lesson_equations_containing_absolute_values}
\newcounter{lesson_graphing_lines}
\newcounter{lesson_two_forms_of_a_linear_equation}
\newcounter{lesson_parallel_and_perpendicular_lines}
\newcounter{lesson_linear_inequalities}
\newcounter{lesson_compound_inequalities}
\newcounter{lesson_inequalities_containing_absolute_values}
\newcounter{lesson_graphing_systems}
\newcounter{lesson_substitution}
\newcounter{lesson_elimination}
\newcounter{lesson_quadratics_introduction}
\newcounter{lesson_factoring_GCF}
\newcounter{lesson_factoring_grouping}
\newcounter{lesson_factoring_trinomials_a_is_1}
\newcounter{lesson_factoring_trinomials_a_neq_1}
\newcounter{lesson_solving_by_factoring}
\newcounter{lesson_square_roots}
\newcounter{lesson_i_and_complex_numbers}
\newcounter{lesson_vertex_form_and_graphing}
\newcounter{lesson_solve_by_square_roots}
\newcounter{lesson_extracting_square_roots}
\newcounter{lesson_the_discriminant}
\newcounter{lesson_the_quadratic_formula}
\newcounter{lesson_quadratic_inequalities}
\newcounter{lesson_functions_and_relations}
\newcounter{lesson_evaluating_functions}
\newcounter{lesson_finding_domain_and_range_graphically}
\newcounter{lesson_fundamental_functions}
\newcounter{lesson_finding_domain_algebraically}
\newcounter{lesson_solving_functions}
\newcounter{lesson_function_arithmetic}
\newcounter{lesson_composite_functions}
\newcounter{lesson_inverse_functions_definition_and_HLT}
\newcounter{lesson_finding_an_inverse_function}
\newcounter{lesson_transformations_translations}
\newcounter{lesson_transformations_reflections}
\newcounter{lesson_transformations_scalings}
\newcounter{lesson_transformations_summary}
\newcounter{lesson_piecewise_functions}
\newcounter{lesson_functions_containing_absolute_values}
\newcounter{lesson_absolute_as_piecewise}
\newcounter{lesson_polynomials_introduction}
\newcounter{lesson_sign_diagrams_polynomials}
\newcounter{lesson_factoring_quadratic_type}
\newcounter{lesson_factoring_summary}
\newcounter{lesson_polynomial_division}
\newcounter{lesson_synthetic_division}
\newcounter{lesson_end_behavior_polynomials}
\newcounter{lesson_local_behavior_polynomials}
\newcounter{lesson_rational_root_theorem}
\newcounter{lesson_polynomials_graphing_summary}
\newcounter{lesson_polynomial_inequalities}
\newcounter{lesson_rationals_introduction_and_terminology}
\newcounter{lesson_sign_diagrams_rationals}
\newcounter{lesson_horizontal_asymptotes}
\newcounter{lesson_slant_and_curvilinear_asymptotes}
\newcounter{lesson_vertical_asymptotes}
\newcounter{lesson_holes}
\newcounter{lesson_rationals_graphing_summary}

\setcounter{lesson_solving_linear_equations}{1}
\setcounter{lesson_equations_containing_absolute_values}{2}
\setcounter{lesson_graphing_lines}{3}
\setcounter{lesson_two_forms_of_a_linear_equation}{4}
\setcounter{lesson_parallel_and_perpendicular_lines}{5}
\setcounter{lesson_linear_inequalities}{6}
\setcounter{lesson_compound_inequalities}{7}
\setcounter{lesson_inequalities_containing_absolute_values}{8}
\setcounter{lesson_graphing_systems}{9}
\setcounter{lesson_substitution}{10}
\setcounter{lesson_elimination}{11}
\setcounter{lesson_quadratics_introduction}{16}
\setcounter{lesson_factoring_GCF}{17}
\setcounter{lesson_factoring_grouping}{18}
\setcounter{lesson_factoring_trinomials_a_is_1}{19}
\setcounter{lesson_factoring_trinomials_a_neq_1}{20}
\setcounter{lesson_solving_by_factoring}{21}
\setcounter{lesson_square_roots}{22}
\setcounter{lesson_i_and_complex_numbers}{23}
\setcounter{lesson_vertex_form_and_graphing}{24}
\setcounter{lesson_solve_by_square_roots}{25}
\setcounter{lesson_extracting_square_roots}{26}
\setcounter{lesson_the_discriminant}{27}
\setcounter{lesson_the_quadratic_formula}{28}
\setcounter{lesson_quadratic_inequalities}{29}
\setcounter{lesson_functions_and_relations}{12}
\setcounter{lesson_evaluating_functions}{13}
\setcounter{lesson_finding_domain_and_range_graphically}{14}
\setcounter{lesson_fundamental_functions}{15}
\setcounter{lesson_finding_domain_algebraically}{30}
\setcounter{lesson_solving_functions}{31}
\setcounter{lesson_function_arithmetic}{32}
\setcounter{lesson_composite_functions}{33}
\setcounter{lesson_inverse_functions_definition_and_HLT}{34}
\setcounter{lesson_finding_an_inverse_function}{35}
\setcounter{lesson_transformations_translations}{36}
\setcounter{lesson_transformations_reflections}{37}
\setcounter{lesson_transformations_scalings}{38}
\setcounter{lesson_transformations_summary}{39}
\setcounter{lesson_piecewise_functions}{40}
\setcounter{lesson_functions_containing_absolute_values}{41}
\setcounter{lesson_absolute_as_piecewise}{42}
\setcounter{lesson_polynomials_introduction}{43}
\setcounter{lesson_sign_diagrams_polynomials}{44}
\setcounter{lesson_factoring_quadratic_type}{46}
\setcounter{lesson_factoring_summary}{45}
\setcounter{lesson_polynomial_division}{47}
\setcounter{lesson_synthetic_division}{48}
\setcounter{lesson_end_behavior_polynomials}{49}
\setcounter{lesson_local_behavior_polynomials}{50}
\setcounter{lesson_rational_root_theorem}{51}
\setcounter{lesson_polynomials_graphing_summary}{52}
\setcounter{lesson_polynomial_inequalities}{53}
\setcounter{lesson_rationals_introduction_and_terminology}{54}
\setcounter{lesson_sign_diagrams_rationals}{55}
\setcounter{lesson_horizontal_asymptotes}{56}
\setcounter{lesson_slant_and_curvilinear_asymptotes}{57}
\setcounter{lesson_vertical_asymptotes}{58}
\setcounter{lesson_holes}{59}
\setcounter{lesson_rationals_graphing_summary}{60}

\begin{document}
{\bf \large Lesson \arabic{lesson_finding_domain_algebraically}: Finding Domain Algebraically}
%\\ CC attribute: \href{http://www.wallace.ccfaculty.org/book/book.html}{\it{Beginning and Intermediate Algebra}} by T. Wallace. 
\\ CC attribute: \href{http://www.stitz-zeager.com}{\it{College Algebra}} by C. Stitz and J. Zeager. 
\hfill \doclicenseImage[imagewidth=5em]\\
\par
{\bf Objective:} Find the domain of a function by algebraic methods.\\
\par
{\bf Students will be able to:}
\begin{itemize}
	\item Determine the appropriate course of action for identifying the domain of a variety of algebraic functions (polynomial, rational, radical, etc.).
	\item Identify the domain of an arbitrary algebraic function.
\end{itemize}
{\bf Prerequisite Knowledge:}
\begin{itemize}
	\item Solving basic inequalities.
	\item Interval notation.
\end{itemize}
\hrulefill

{\bf Lesson:}\\
\ \par
When trying to identify the domain of a function that has been described algebraically or whose graph is not known, we will often need to consider what is {\it not} permissible for the function, then exclude any values of $x$ that will make the function undefined from the interval $(-\infty,\infty)$.  What is left will be our domain.  With virtually every algebraic function, this amounts to avoiding the following situations.
\begin{itemize}
	\item Negatives under an even radical $\left(\sqrt{~}~,~~\sqrt[4]{~}~,~~\sqrt[6]{~}~,~\ldots\right)$
	\item Zero in a denominator
\end{itemize}
{\bf I - Motivating Example(s):}\\
\ \par
{\bf Example:} Find the domain of $f(x)=\frac{1}{3} x^2-x$.
  \begin{eqnarray*}
    f (x) = \frac{1}{3} x^2 - x & & \text{No radicals or variables in a denominator}\\
		& & \text{No values of} \ x \ \text{need to be excluded}\\
		\text{All real numbers or} \ (-\infty,\infty) & & \text{Our solution}
 \end{eqnarray*}

Our next example will be of a {\it rational function}, which is defined as a ratio of two polynomial functions.  We will explore rational functions and their graphs in a later lesson.  Since rational functions usually include expressions in a denominator, their domains will often require us to exclude one or more values of $x$.\\
\ \par
{\bf Example:} Find the domain of the function $f (x) = \dfrac{3 x - 1}{x^2 + x - 6}$.
  \begin{eqnarray*}
    f (x) = \frac{3 x - 1}{x^2 + x - 6}
    &  & \text{Cannot have zero in a denominator}\\
		& & \\
    x^2 + x - 6 \neq 0 &  & \text{Solve by factoring}\\
    (x + 3) (x - 2) \neq 0 &  & \text{Set each factor not equal to zero}\\
    x + 3 \neq 0 \ \text{and} \ x - 2 \neq 0 &  & \text{Solve each inequality}\\
%    \underline{- 3~~~ - 3} ~~~~~~~ \underline{+ 2~~~ + 2} &  & \\
    x \neq - 3, 2 &  & \text{Our solution as an inequality}\\
		(-\infty,-3)\cup(-3,2)\cup(2,\infty) & & \text{Our solution using interval notation}
  \end{eqnarray*}
%\ \par
Although one can easily see that $x=\frac{1}{3}$ will make the numerator equal zero, since $x=\frac{1}{3}$ does not coincide with the two values obtained above (either -3 or 2), we should not exclude it from our domain.  Whenever we are finding the domain of a rational function, we need not be concerned at all with the numerator, and instead must restrict our domain to exclude any value for $x$ that would make the {\it denominator} equal to zero.\\
\ \par
{\bf Example:} Find the domain of $f (x) = \sqrt[]{-2 x + 3}$.
  \begin{eqnarray*}
    f (x) = \sqrt[]{-2 x + 3} &  & \text{Even radical; cannot have negative underneath}\\
    %& & \\
		-2 x + 3 \geq 0 &  & \text{Set greater than or equal to zero and solve}\\
    -2 x \geq -3 &  & \text{Remember to switch direction of inequality}\\
		& &\\
    x \leq \frac{3}{2} \text{~~or~} \left(-\infty,\frac{3}{2}\right]&  & \text{Our solution as an inequality or an interval}
		 %\left(-\infty,\frac{3}{2}\right] & &\tmop{Our~solution~as~an~interval}
  \end{eqnarray*}
{\bf II - Demo/Discussion Problems:}\\
\ \par
Find the domain of each of the following functions.  Express your answers using interval notation.
\begin{multicols}{3}
\begin{enumerate}
\item $f (x) = |3x-2|$
\item $g (x) = (3x-2)^2$
\item $h(x) = \dfrac{1}{3x-2}$
\item $k (x) = \sqrt[]{3x-2}$
\item $k (x) = \sqrt[3]{3x-2}$
\item $\ell (x) = \sqrt[4]{2-3x}$
\item $m(x) = \dfrac{x-2}{\sqrt[]{3x-2}}$
\item $n(x) = \dfrac{\sqrt[]{3x-2}}{x-2}$
\end{enumerate}
\end{multicols}
\newpage
{\bf III - Practice Problems:}\\
\ \par
Find the domain of each of the following functions.  Express your answers using interval notation.
\begin{multicols}{2}
\begin{enumerate}
\item $g(x)-4x^2$
\item $f(x) = x^{4} - 13x^{3} + 56x^{2} - 19$
\item $g(x) = x^2 - 4$
\item $k(x) = \dfrac{x}{x - 8}$
\item $h(x)=\dfrac{x-5}{x+4}$
\item $h(x) = \dfrac{x-2}{x+1}$
\item $k(x) = \dfrac{x-2}{x-2}$  
\item $k(x) = \dfrac{3x}{x^2+x-2}$
\item $g(x) = \dfrac{2x}{x^2-9}$
\item $f(x) = \dfrac{2x}{x^2+9}$
\item $h(x) = \dfrac{x+4}{x^2 - 36}$
\item $f(x) = \sqrt{3-x}$
\item $g(x) = \sqrt{2x+5}$  
\item $f(x)=5\sqrt{x-1}$
\item $h(x) = 9x\sqrt{x+3}$
\item $k(x) = \dfrac{\sqrt{7-x}}{x^2+1}$  
\item  $f(x) = \sqrt{6x-2}$
\item  $g(x) = \dfrac{6}{\sqrt{6x-2}}$
\item $k(x)=\dfrac{4}{\sqrt{x-3}}$
\item $g(x) = \dfrac{x}{\sqrt{x - 8}}$
\item  $h(x) = \sqrt[3]{6x-2}$
\item  $k(x) = \dfrac{6}{4 - \sqrt{6x-2}}$
\item  $f(x) = \dfrac{\sqrt{6x-2}}{x^2-36}$
\item  $g(x) = \dfrac{\sqrt[3]{6x-2}}{x^2+36}$
\item $h(x) = \sqrt{x - 7} + \sqrt{9 - x}$
\item $h(t) = \dfrac{\sqrt{t} - 8}{5-t}$ 
\item $f(r) = \dfrac{\sqrt{r}}{r - 8}$
\item $k(v) = \dfrac{1}{4 - \dfrac{1}{v^{2}}}$
\item $f(y) = \sqrt[3]{\dfrac{y}{y - 8}}$
\item $k(w) = \dfrac{w - 8}{5 - \sqrt{w}}$
\end{enumerate}
\end{multicols}
\newpage
\ \newpage
\end{document}