\documentclass[12pt]{book}
\raggedbottom
\usepackage[top=1in,left=1in,bottom=1in,right=1in,headsep=0.25in]{geometry}	
\usepackage{amssymb,amsmath,amsthm,amsfonts}
\usepackage{chapterfolder,docmute,setspace}
\usepackage{cancel,multicol,tikz,verbatim,framed,polynom,enumitem,tikzpagenodes}
\usepackage[colorlinks, hyperindex, plainpages=false, linkcolor=blue, urlcolor=blue, pdfpagelabels]{hyperref}
\usepackage[type={CC},modifier={by-sa},version={4.0},]{doclicense}

\theoremstyle{definition}
\newtheorem{example}{Example}
\newcommand{\Desmos}{\href{https://www.desmos.com/}{Desmos}}
\setlength{\parindent}{0in}
\setlist{itemsep=0in}
\setlength{\parskip}{0.1in}
\setcounter{secnumdepth}{0}
% This document is used for ordering of lessons.  If an instructor wishes to change the ordering of assessments, the following steps must be taken:

% 1) Reassign the appropriate numbers for each lesson in the \setcounter commands included in this file.
% 2) Rearrange the \include commands in the master file (the file with 'Course Pack' in the name) to accurately reflect the changes.  
% 3) Rarrange the \items in the measureable_outcomes file to accurately reflect the changes.  Be mindful of page breaks when moving items.
% 4) Re-build all affected files (master file, measureable_outcomes file, and any lessons whose numbering has changed).

%Note: The placement of each \newcounter and \setcounter command reflects the original/default ordering of topics (linears, systems, quadratics, functions, polynomials, rationals).

\newcounter{lesson_solving_linear_equations}
\newcounter{lesson_equations_containing_absolute_values}
\newcounter{lesson_graphing_lines}
\newcounter{lesson_two_forms_of_a_linear_equation}
\newcounter{lesson_parallel_and_perpendicular_lines}
\newcounter{lesson_linear_inequalities}
\newcounter{lesson_compound_inequalities}
\newcounter{lesson_inequalities_containing_absolute_values}
\newcounter{lesson_graphing_systems}
\newcounter{lesson_substitution}
\newcounter{lesson_elimination}
\newcounter{lesson_quadratics_introduction}
\newcounter{lesson_factoring_GCF}
\newcounter{lesson_factoring_grouping}
\newcounter{lesson_factoring_trinomials_a_is_1}
\newcounter{lesson_factoring_trinomials_a_neq_1}
\newcounter{lesson_solving_by_factoring}
\newcounter{lesson_square_roots}
\newcounter{lesson_i_and_complex_numbers}
\newcounter{lesson_vertex_form_and_graphing}
\newcounter{lesson_solve_by_square_roots}
\newcounter{lesson_extracting_square_roots}
\newcounter{lesson_the_discriminant}
\newcounter{lesson_the_quadratic_formula}
\newcounter{lesson_quadratic_inequalities}
\newcounter{lesson_functions_and_relations}
\newcounter{lesson_evaluating_functions}
\newcounter{lesson_finding_domain_and_range_graphically}
\newcounter{lesson_fundamental_functions}
\newcounter{lesson_finding_domain_algebraically}
\newcounter{lesson_solving_functions}
\newcounter{lesson_function_arithmetic}
\newcounter{lesson_composite_functions}
\newcounter{lesson_inverse_functions_definition_and_HLT}
\newcounter{lesson_finding_an_inverse_function}
\newcounter{lesson_transformations_translations}
\newcounter{lesson_transformations_reflections}
\newcounter{lesson_transformations_scalings}
\newcounter{lesson_transformations_summary}
\newcounter{lesson_piecewise_functions}
\newcounter{lesson_functions_containing_absolute_values}
\newcounter{lesson_absolute_as_piecewise}
\newcounter{lesson_polynomials_introduction}
\newcounter{lesson_sign_diagrams_polynomials}
\newcounter{lesson_factoring_quadratic_type}
\newcounter{lesson_factoring_summary}
\newcounter{lesson_polynomial_division}
\newcounter{lesson_synthetic_division}
\newcounter{lesson_end_behavior_polynomials}
\newcounter{lesson_local_behavior_polynomials}
\newcounter{lesson_rational_root_theorem}
\newcounter{lesson_polynomials_graphing_summary}
\newcounter{lesson_polynomial_inequalities}
\newcounter{lesson_rationals_introduction_and_terminology}
\newcounter{lesson_sign_diagrams_rationals}
\newcounter{lesson_horizontal_asymptotes}
\newcounter{lesson_slant_and_curvilinear_asymptotes}
\newcounter{lesson_vertical_asymptotes}
\newcounter{lesson_holes}
\newcounter{lesson_rationals_graphing_summary}

\setcounter{lesson_solving_linear_equations}{1}
\setcounter{lesson_equations_containing_absolute_values}{2}
\setcounter{lesson_graphing_lines}{3}
\setcounter{lesson_two_forms_of_a_linear_equation}{4}
\setcounter{lesson_parallel_and_perpendicular_lines}{5}
\setcounter{lesson_linear_inequalities}{6}
\setcounter{lesson_compound_inequalities}{7}
\setcounter{lesson_inequalities_containing_absolute_values}{8}
\setcounter{lesson_graphing_systems}{9}
\setcounter{lesson_substitution}{10}
\setcounter{lesson_elimination}{11}
\setcounter{lesson_quadratics_introduction}{16}
\setcounter{lesson_factoring_GCF}{17}
\setcounter{lesson_factoring_grouping}{18}
\setcounter{lesson_factoring_trinomials_a_is_1}{19}
\setcounter{lesson_factoring_trinomials_a_neq_1}{20}
\setcounter{lesson_solving_by_factoring}{21}
\setcounter{lesson_square_roots}{22}
\setcounter{lesson_i_and_complex_numbers}{23}
\setcounter{lesson_vertex_form_and_graphing}{24}
\setcounter{lesson_solve_by_square_roots}{25}
\setcounter{lesson_extracting_square_roots}{26}
\setcounter{lesson_the_discriminant}{27}
\setcounter{lesson_the_quadratic_formula}{28}
\setcounter{lesson_quadratic_inequalities}{29}
\setcounter{lesson_functions_and_relations}{12}
\setcounter{lesson_evaluating_functions}{13}
\setcounter{lesson_finding_domain_and_range_graphically}{14}
\setcounter{lesson_fundamental_functions}{15}
\setcounter{lesson_finding_domain_algebraically}{30}
\setcounter{lesson_solving_functions}{31}
\setcounter{lesson_function_arithmetic}{32}
\setcounter{lesson_composite_functions}{33}
\setcounter{lesson_inverse_functions_definition_and_HLT}{34}
\setcounter{lesson_finding_an_inverse_function}{35}
\setcounter{lesson_transformations_translations}{36}
\setcounter{lesson_transformations_reflections}{37}
\setcounter{lesson_transformations_scalings}{38}
\setcounter{lesson_transformations_summary}{39}
\setcounter{lesson_piecewise_functions}{40}
\setcounter{lesson_functions_containing_absolute_values}{41}
\setcounter{lesson_absolute_as_piecewise}{42}
\setcounter{lesson_polynomials_introduction}{43}
\setcounter{lesson_sign_diagrams_polynomials}{44}
\setcounter{lesson_factoring_quadratic_type}{46}
\setcounter{lesson_factoring_summary}{45}
\setcounter{lesson_polynomial_division}{47}
\setcounter{lesson_synthetic_division}{48}
\setcounter{lesson_end_behavior_polynomials}{49}
\setcounter{lesson_local_behavior_polynomials}{50}
\setcounter{lesson_rational_root_theorem}{51}
\setcounter{lesson_polynomials_graphing_summary}{52}
\setcounter{lesson_polynomial_inequalities}{53}
\setcounter{lesson_rationals_introduction_and_terminology}{54}
\setcounter{lesson_sign_diagrams_rationals}{55}
\setcounter{lesson_horizontal_asymptotes}{56}
\setcounter{lesson_slant_and_curvilinear_asymptotes}{57}
\setcounter{lesson_vertical_asymptotes}{58}
\setcounter{lesson_holes}{59}
\setcounter{lesson_rationals_graphing_summary}{60}

\newcommand{\tmmathbf}[1]{\ensuremath{\boldsymbol{#1}}}
\newcommand{\tmop}[1]{\ensuremath{\operatorname{#1}}}

\begin{document}
\section{Introduction and Graphing (L\arabic{lesson_graphing_systems})}
\begin{tikzpicture}[remember picture,overlay,shift=(current page text area.north east),scale=0.5]
\draw[step=1.0,gray,very thin,dotted] (-9.8,-7.8) grid (-0.2,1.8);		
\draw[very thick] (-10,-8) -- (-10,2) -- (0,2) -- (0,-8) -- (-10,-8);
\draw[] (-9.8,-7.8) -- (-9.8,1.8) -- (-0.2,1.8) -- (-0.2,-7.8) -- (-9.8,-7.8);
\draw[-] (-9.8,-3) -- coordinate (x axis mid) (-0.2,-3);
\draw[-] (-5,-7.8) -- coordinate (y axis mid) (-5,1.8);
\draw[<->] plot [domain=-8:-1, samples=100] (\x,{\x+1});
\draw[<->] plot [domain=-9:-1, samples=100] (\x,{-0.5*\x-4});
\end{tikzpicture}%
{\bf Objective: Solve a system of linear equations by graphing and identifying the point of intersection.}\par
We have solved problems like $3 x - 4 = 11$ by adding 4 to both sides and then dividing by 3 (solution is $x = 5$). We also have methods to solve equations with more than one variable in them. It turns out that to solve for more than one variable we will need the same number of equations as variables. For example, to solve for two variables such as $x$ and $y$ we will need two equations. When we have two (or more) equations we are working with, we call the set of equations a {\it system}.  When solving a system of equations we are looking for a solution that satisfies each equation simultaneously. If our system consists of two equations in terms of $x$ and $y$, this solution is usually described as an ordered pair $(x, y)$. The following example demonstrates a solution for a system of two linear equations.

\begin{example}~~~
  Show ($x,y$)=(2,1) is the solution to the system
\begin{center}
	%$\begin{array}{l}
   $ 3 x - y = 5$~~~~~~~$x + y = 3$
  %\end{array}$
\end{center}
	\begin{eqnarray*}
    (x,y)=(2, 1) &  & \tmop{Identify} x \tmop{and} y \tmop{from} \tmop{the}
    \tmop{ordered} \tmop{pair}\\
    x = 2,~y = 1 &  & \tmop{Plug} \tmop{these} \tmop{values} \tmop{into}
    \tmop{each} \tmop{equation}\\
%	\end{eqnarray*}
%	\begin{eqnarray*}
    &  & \\
    3 (2) - (1) = 5 &  & \tmop{First} \tmop{equation}\\
    6 - 1 = 5 &  & \tmop{Evaluate}\\
    5 = 5 &  & \tmop{True}\\
    &  & \\
    (2) + (1) = 3 &  & \tmop{Second} \tmop{equation}, \tmop{evaluate}\\
    3 = 3 &  & \tmop{True}
  \end{eqnarray*}
\end{example}
As we found a true statement for both equations we know $(2,1)$ is the solution to the system. It is in fact the only combination of numbers that works in both equations.  In this section, we will attempt to identify a (simultaneous) solution to two equations, if such a solution exists.  It stands to reason that if we use points to describe the solution, we can use graphs to find the solution.\par
If the graph of a line is a picture of all the solutions to its equation, we can graph two lines on the same coordinate plane to see the solutions of both equations. In particular, we are interested in finding all points that are a solution for both equations.  This will be
the point(s) where the two lines intersect! If we can find the intersection of the lines we have found the solution that works in both equations.
\begin{example}~~~Solve the following system of equations.
  \begin{eqnarray*}
    \begin{array}{l}
      y = - \frac{1}{2} x + 3\\
    	y = \frac{3}{4} x - 2
   \end{array} &  & \tmop{First} \tmop{identify}
    \tmop{slopes} \tmop{and} y - \tmop{intercepts}\\
  & & \\ 
	\begin{array}{l}
      \tmop{First~Line} : ~~~m = - \frac{1}{2}, ~~b = 3\\
      \tmop{Second~Line} : ~m = \frac{3}{4}, ~~~b = - 2
    \end{array} &  & \tmop{Next} \tmop{graph} \tmop{both}
    \tmop{lines} \tmop{on} \tmop{the} \tmop{same} \tmop{plane}\\
  \end{eqnarray*}
  
	\begin{multicols}{2}
	\begin{tikzpicture}[xscale=0.4,yscale=0.4]
		\draw[step=1.0,gray,very thin,dotted] (-8.5,-5.5) grid (8.5,5.5);
		\draw [<->](-8.5,0) -- coordinate (x axis mid) (8.5,0) node[below right] {$x$};
		\draw [<->](0,-5.5) -- coordinate (y axis mid) (0,5.5) node[above right] {$y$};
		\foreach \x in {1,...,8}
		\draw (\x,2pt) -- (\x,-2pt)
			node[anchor=south] {\scriptsize \x}
		;
		\foreach \x in {-1,...,-8}
		\draw (\x,2pt) -- (\x,-2pt)
			node[anchor=north] {\scriptsize \x}
		;
		\foreach \y in {1,...,5}
		\draw (2pt,\y) -- (-2pt,\y)
			node[anchor=east] {\scriptsize \y}
		; 
		\foreach \y in {-1,...,-5}
		\draw (2pt,\y) -- (-2pt,\y)
			node[anchor=west] {\scriptsize \y}
		; 
		\draw [<->] plot [domain=-4:8.5, samples=100] (\x,{-0.5*(\x)+3});
		\draw [<->] plot [domain=-4.5:8.5, samples=100] (\x,{0.75*(\x)-2});
		\draw[fill] (0,3) circle (0.1) node[right] {};
		\draw[fill] (2,2) circle (0.1) node[right] {};
		\draw[fill] (0,-2)circle (0.1) node[right] {};
		\draw[fill] (4,1)circle (0.1) node[right] {};
		\draw[fill] (4,1)circle (0.1) node[above] {\scriptsize $(4,1)$};
	\end{tikzpicture}    
	
	\columnbreak
	
    \ \par
    To graph each equation, we start at the $y$-intercept and use the slope to get the next point, then connect the dots.\par
	Remember a line with a negative slope points downhill from left to right!\par
    Find the intersection point, $(4,1)$.  This is our solution.
  \end{multicols}
\end{example}
Often our equations won't be in slope-intercept form and we will have to solve both equations for $y$ first so we can identify the slope and $y-$intercept.
\begin{example}~~~Solve the following system of equations.
  \begin{eqnarray*}
    \begin{array}{l}
      6 x - 3 y = - 9\\
      2 x + 2 y = - 6
    \end{array} &  & \tmop{Solve} \tmop{each} \tmop{equation} \tmop{for~} y\\
    &  & \\
%	\end{eqnarray*}
%	\begin{eqnarray*}
	6 x - 3 y = - 9~~ & ~~~~~~2 x + 2 y = - 6~~~ &\\
    \tmmathbf{\underline{- 6 x ~~~~~~~- 6 x}} & ~~~\tmmathbf{\underline{- 2 x ~~~~~~~~- 2 x}} &  \tmop{Subtract~} x
    \tmop{~terms}\\
    &  & \\
		- 3 y = - 6 x ~- 9 & ~~~2 y = - 2 x - 6 & \tmop{Rearrange} \tmop{equations}\\
    \tmmathbf{\overline{- 3} ~~~~~ \overline{- 3} ~~ \overline{- 3}} & ~~~\tmmathbf{\overline{2} ~~~~~~~ \overline{2} ~~~~~ \overline{2}} & \tmop{Divide} \tmop{by} \tmop{coefficient} \tmop{of~}
    y
	\end{eqnarray*}
	\begin{eqnarray*}
    y = 2 x + 3~~~ & ~~~y = - x - 3 &\tmop{Identify} \tmop{slope} \tmop{and~} y -
    \tmop{intercepts}\\
		& & \\
    \begin{array}{l}
			\tmop{First~Line} : ~~~~m = 2, ~~~~b = 3\\
      \tmop{Second~Line} : ~m = - 1, ~~b = - 3
    \end{array} &  & \tmop{Next} \tmop{graph} \tmop{both}
    \tmop{lines} \tmop{on} \tmop{the} \tmop{same} \tmop{plane}%\\
		%& & \\
  \end{eqnarray*}
\begin{multicols}{2}
	\begin{tikzpicture}[xscale=0.4,yscale=0.4]
		\draw[step=1.0,gray,very thin,dotted] (-8.5,-5.5) grid (8.5,5.5);
		\draw [<->](-8.5,0) -- coordinate (x axis mid) (8.5,0) node[below right] {$x$};
		\draw [<->](0,-5.5) -- coordinate (y axis mid) (0,5.5) node[above right] {$y$};
		\foreach \x in {1,...,8}
		\draw (\x,2pt) -- (\x,-2pt)
			node[anchor=south] {\scriptsize \x}
		;
		\foreach \x in {-1,...,-8}
		\draw (\x,2pt) -- (\x,-2pt)
			node[anchor=north] {\scriptsize \x}
		;
		\foreach \y in {1,...,5}
		\draw (2pt,\y) -- (-2pt,\y)
			node[anchor=east] {\scriptsize \y}
		; 
		\foreach \y in {-1,...,-5}
		\draw (2pt,\y) -- (-2pt,\y)
			node[anchor=west] {\scriptsize \y}
		; 
		\draw [<->] plot [domain=-4:1.25, samples=100] (\x,{2*(\x)+3});
		\draw [<->] plot [domain=-7.5:2.5, samples=100] (\x,{-1*(\x)-3});
		\draw[fill] (0,3) circle (0.1) node[right] {};
		\draw[fill] (1,5) circle (0.1) node[right] {};
		\draw[fill] (0,-3)circle (0.1) node[right] {};
		\draw[fill] (1,-4)circle (0.1) node[right] {};
		\draw[fill] (-2,-1)circle (0.1);
		\draw (-2.5,-1.5) node[left]{\scriptsize $(-2,-1)$};
	\end{tikzpicture}    
	
	\columnbreak
	
    \ \par
    To graph each equation, we start at the $y$-intercept and use the slope to get the next point, then connect the dots.\par
    Remember a line with a negative slope decreases from left to right!\par
	Using our slopes, we can find the intersection point, $(-2,-1)$.  This is our solution.
  \end{multicols}
\end{example}
As we are graphing our lines, it is possible to have one of two unexpected results. These are shown and discussed in the next two examples.
\begin{example}~~~Solve the following system of equations.
    \begin{center}
			%\begin{array}{l}
      $y = \frac{3}{2} x - 4$ ~~~~~~~~~~~~~~ $y = \frac{3}{2} x + 1$\par
%			\end{array} 
		%&  & \tmop{Identify} \tmop{slope} \tmop{and} y -
    %\tmop{intercept} \tmop{of} \tmop{each} \tmop{equation}\\
    %& & \\
		Identify the slope and $y-$intercept of each equation.
		\end{center}
  \begin{eqnarray*}
		\begin{array}{l}
			\tmop{First~Line} : ~~~~m = \frac{3}{2}, ~~b = -4\\
      \tmop{Second~Line} : ~m = \frac{3}{2}, ~~b = 1
    \end{array} &  & \tmop{Next} \tmop{graph} \tmop{both}
    \tmop{lines} \tmop{on} \tmop{the} \tmop{same} \tmop{plane}\\
		%& & \\
	\end{eqnarray*}
  \begin{multicols}{2}
	\begin{tikzpicture}[xscale=0.4,yscale=0.4]
		\draw[step=1.0,gray,very thin,dotted] (-8.5,-5.5) grid (8.5,5.5);
		\draw [<->](-8.5,0) -- coordinate (x axis mid) (8.5,0) node[below right] {$x$};
		\draw [<->](0,-5.5) -- coordinate (y axis mid) (0,5.5) node[above right] {$y$};
		\foreach \x in {1,...,8}
		\draw (\x,2pt) -- (\x,-2pt)
			node[anchor=south] {\scriptsize \x}
		;
		\foreach \x in {-1,...,-8}
		\draw (\x,2pt) -- (\x,-2pt)
			node[anchor=north] {\scriptsize \x}
		;
		\foreach \y in {1,...,5}
		\draw (2pt,\y) -- (-2pt,\y)
			node[anchor=east] {\scriptsize \y}
		; 
		\foreach \y in {-1,...,-5}
		\draw (2pt,\y) -- (-2pt,\y)
			node[anchor=west] {\scriptsize \y}
		; 
		\draw [<->] plot [domain=-1:6, samples=100] (\x,{1.5*(\x)-4});
		\draw [<->] plot [domain=-4.5:2.5, samples=100] (\x,{1.5*(\x)+1});
		\draw[fill] (0,-4) circle (0.1) node[right] {};
		\draw[fill] (2,-1) circle (0.1) node[right] {};
		\draw[fill] (0,1)circle (0.1) node[right] {};
		\draw[fill] (2,4)circle (0.1) node[right] {};
	\end{tikzpicture}
	
	\columnbreak
	
    To graph each equation, we start at the $y$-intercept and use the slope to get the next point, then connect the dots.\par
    The two lines do not intersect; they are parallel!\par
    Since the lines do not intersect, we know that there is no point that will satisfy both equations.\par
    There is no solution, or $\varnothing$.
	\end{multicols}
\end{example}
Notice that we could also have recognized that both lines had the same slope. Remembering that parallel lines have the same slope one could conclude that there is no solution without having to graph the lines.
\begin{example}~~~Solve the following system of equations.
  \begin{eqnarray*}
    \begin{array}{l}
      2 x - 6 y = 12\\
      3 x - 9 y = 18
    \end{array} &  & \tmop{Solve} \tmop{each} \tmop{equation} \tmop{for~} y\\
    &  & \\
%	\end{eqnarray*}
%	\begin{eqnarray*}
    2 x - 6 y = 12~~~ &~~~3 x - 9 y = 18~~ &\\
    \tmmathbf{\underline{- 2 x ~~~~~~~- 2 x}} &~~~ \tmmathbf{\underline{- 3 x ~~~~~~~- 3 x}} &\tmop{Subtract} x
    \tmop{terms}\\
		& & \\
    - 6 y = - 2 x + 12 &~~~ - 9 y = - 3 x + 18~ & \tmop{Put} x \tmop{terms}
    \tmop{first}\\
    \tmmathbf{\overline{- 6} ~~~~~ \overline{- 6} ~~~ \overline{- 6}} &~~~ \tmmathbf{\overline{- 9}~~~~~ 
    \overline{- 9} ~~~ \overline{- 9}} & \tmop{Divide} \tmop{by}
    \tmop{coefficient} \tmop{of} y\\
    y = \frac{1}{3} x - 2~~~ &~~~ y = \displaystyle\frac{1}{3} x - 2 & \tmop{Identify}
    \tmop{the} \tmop{slopes} \tmop{and} y - \tmop{intercepts}%\\
%		& & \\
	\end{eqnarray*}
	\begin{eqnarray*}
    \begin{array}{l}
			\tmop{First~Line} : ~~~~m = \frac{1}{3}, ~~b = -2\\
      \tmop{Second~Line} : ~m = \frac{1}{3}, ~~b = - 2
    \end{array} &  & \tmop{Next} \tmop{graph} \tmop{both}
    \tmop{lines} \tmop{on} \tmop{the} \tmop{same} \tmop{plane}\\
	%	& & \\
  \end{eqnarray*}
	\begin{multicols}{2}
	\begin{tikzpicture}[xscale=0.4,yscale=0.4]
		\draw[step=1.0,gray,very thin,dotted] (-8.5,-5.5) grid (8.5,5.5);
		\draw [<->](-8.5,0) -- coordinate (x axis mid) (8.5,0) node[below right] {$x$};
		\draw [<->](0,-5.5) -- coordinate (y axis mid) (0,5.5) node[above right] {$y$};
		\foreach \x in {1,...,8}
		\draw (\x,2pt) -- (\x,-2pt)
			node[anchor=south] {\scriptsize \x}
		;
		\foreach \x in {-1,...,-8}
		\draw (\x,2pt) -- (\x,-2pt)
			node[anchor=north] {\scriptsize \x}
		;
		\foreach \y in {1,...,5}
		\draw (2pt,\y) -- (-2pt,\y)
			node[anchor=east] {\scriptsize \y}
		; 
		\foreach \y in {-1,...,-5}
		\draw (2pt,\y) -- (-2pt,\y)
			node[anchor=west] {\scriptsize \y}
		; 
		\draw [<->] plot [domain=-6.5:8.5, samples=100] (\x,{0.333*(\x)-2});
		\draw[fill] (0,-2) circle (0.1) node[right] {};
		\draw[fill] (3,-1) circle (0.1) node[right] {};
	\end{tikzpicture}
	
	\columnbreak
	
    To graph each equation, we start at the $y$-intercept and use the slope to get the next point, then connect the dots.\par
    Both equations are the same line!\par
    As one line is directly on top of the other line, we can say that the lines intersect at every point on the line!\par
    Here we say there are infinitely many solutions.
\end{multicols}
\end{example}
Notice that once we had both equations in slope-intercept form we could have recognized that the equations were the same. At this point one could state that there are infinitely many solutions without having to go through the work of graphing the equations.
\end{document}