\documentclass[12pt]{book}
\raggedbottom
\usepackage[top=1in,left=1in,bottom=1in,right=1in,headsep=0.25in]{geometry}	
\usepackage{amssymb,amsmath,amsthm,amsfonts}
\usepackage{chapterfolder,docmute,setspace}
\usepackage{cancel,multicol,tikz,verbatim,framed,polynom,enumitem,tikzpagenodes}
\usepackage[colorlinks, hyperindex, plainpages=false, linkcolor=blue, urlcolor=blue, pdfpagelabels]{hyperref}
\usepackage[type={CC},modifier={by-sa},version={4.0},]{doclicense}

\theoremstyle{definition}
\newtheorem{example}{Example}
\newcommand{\Desmos}{\href{https://www.desmos.com/}{Desmos}}
\setlength{\parindent}{0in}
\setlist{itemsep=0in}
\setlength{\parskip}{0.1in}
\setcounter{secnumdepth}{0}
\input{lesson_order}

\newcommand{\tmmathbf}[1]{\ensuremath{\boldsymbol{#1}}}
\newcommand{\tmop}[1]{\ensuremath{\operatorname{#1}}}

\begin{document}
\section{Practice Problems}
\subsection*{Notation and Basic Examples}                                
Determine whether or not each relation represents $y$ as a function of $x$.  
\begin{enumerate}
\item[1.]  $\{(-3, 9)$, $\;(-2, 4)$, $\;(-1, 1)$, $\;(0, 0)$, $\;(1, 1)$, $\;(2, 4)$, $\;(3, 9)\}$
\item[2.]  $\left\{ (-3,0), (1,6), (2, -3), (4,2), (-5,6), (4, -9), (6,2) \right\}$
\item[3.]  $\left\{ (-3,0), (-7,6), (5,5), (6,4), (4,9), (3,0) \right\}$
\item[4.]  $\left\{ (1,2), (4,4), (9,6), (16,8), (25,10), (36, 12), \ldots \right\}$
\item[5.]  $\{(x, y) \, | \, x$ is an odd integer, and $y$ is an even integer$\}$
\item[6.]  $\{(x, 1) \, | \, x$ is an irrational number$\}$
\item[7.]  $\{(1, 0)$, $\;(2, 1)$, $\;(4, 2)$, $\;(8, 3)$, $\;(16, 4)$, $\;(32, 5), \;$ $\ldots\}$
\item[8.]  $\{\ldots,$ $\;(-3, 9)$, $\;(-2, 4)$, $\;(-1, 1)$, $\;(0, 0)$, $\;(1, 1)$, $\;(2, 4)$, $\;(3, 9), \;$ $\ldots\}$
\end{enumerate}
\begin{multicols}{2}
\begin{enumerate}
\item[9.] $\{ (-2, y) \, | \, -3 < y < 4\}$
\item[10.] $\{ (x,3) \, | \,  -2 \leq x < 4\}$
\end{enumerate}
\end{multicols}
Determine if the following relations represent $y$ as a function of $x$ by making a table of values and graphing.  Explain your reasoning.  Use \Desmos \ to confirm your results.
\begin{center}
\begin{multicols}{3}
\begin{enumerate}
	\item[11.] $x=y^3$
	\item[12.] $y=x$
	\item[13.] $xy=1$
	\item[14.] $y=(x-3)^2$
	\item[15.] $x=(y-3)^2$
	\item[16.] $y<2x-5$
\end{enumerate}
\end{multicols}
\end{center}

Determine whether each of the following relations represents $y$ as a function of $x$.
\begin{multicols}{2}
\begin{enumerate}
\item[17.] \ \\
\begin{tikzpicture}[domain=0:3.7, scale=0.7]
    %\draw[very thin,color=gray] (-5,-3) grid (5,3);
		\draw [<->](-4.5,0) -- coordinate (x axis mid) (4.5,0) node[below right] {$x$};
		\draw [<->](0,-3.5) -- coordinate (y axis mid) (0,3.5) node[above right] {$y$};
			\foreach \x in {-4,...,-1}
			\draw (\x,1pt) -- (\x,-3pt)
			node[anchor=north] {\scriptsize \x};
			\foreach \x in {1,...,4}
			\draw (\x,1pt) -- (\x,-3pt)
			node[anchor=north] {\scriptsize \x};
			\foreach \y in {-3,...,-1}
			\draw (1pt,\y) -- (-3pt,\y) 
			node[anchor=west] {\scriptsize \y}; 
			\foreach \y in {1,...,3}
			\draw (1pt,\y) -- (-3pt,\y) 
			node[anchor=east] {\scriptsize \y}; 
    	\draw[fill] (3,2) circle (0.125);
		\draw[-] [ultra thick]  (2.1,2)--(2.9,2);
    	\draw[fill] (2,1) circle (0.125);
		\draw[-] [ultra thick]  (1.1,1)--(1.9,1);
    	\draw[fill] (1,0) circle (0.125);
		\draw[-] [ultra thick]  (0.1,0)--(0.9,0);
    	\draw[fill] (0,-1) circle (0.125);
		\draw[-] [ultra thick]  (-0.9,-1)--(-0.1,-1);
    	\draw[fill] (-1,-2) circle (0.125);
		\draw[-] [ultra thick]  (-1.9,-2)--(-1.1,-2);
		\draw[fill,white] (2,2) circle (0.125);
		\draw[fill,white] (1,1) circle (0.125);
		\draw[fill,white] (0,0) circle (0.125);
		\draw[fill,white] (-1,-1) circle (0.125);
		\draw[fill,white] (-2,-2) circle (0.125);
		\draw[line width=0.1mm] (2,2) circle (0.125);
		\draw[line width=0.1mm] (1,1) circle (0.125);
		\draw[line width=0.1mm] (0,0) circle (0.125);
		\draw[line width=0.1mm] (-1,-1) circle (0.125);
		\draw[line width=0.1mm] (-2,-2) circle (0.125);
\end{tikzpicture}
\hspace{1in}
\item[18.] \ \\
\begin{tikzpicture}[domain=0:3.7, scale=0.7]
    %\draw[very thin,color=gray] (-5,-3) grid (5,3);
		\draw [<->](-4.5,0) -- coordinate (x axis mid) (4.5,0) node[below right] {$x$};
		\draw [<->](0,-3.5) -- coordinate (y axis mid) (0,3.5) node[above right] {$y$};
			\foreach \x in {-4,...,-1}
			\draw (\x,1pt) -- (\x,-3pt)
			node[anchor=north] {\scriptsize \x};
			\foreach \x in {1,...,4}
			\draw (\x,1pt) -- (\x,-3pt)
			node[anchor=north] {\scriptsize \x};
			\foreach \y in {-3,...,-1}
			\draw (1pt,\y) -- (-3pt,\y) 
			node[anchor=east] {\scriptsize \y}; 
			\foreach \y in {1,...,3}
			\draw (1pt,\y) -- (-3pt,\y) 
			node[anchor=east] {\scriptsize \y}; 
    \draw[->,domain=0:2]   plot ({\x^2},\x);
		\draw[->,domain=0:2]   plot ({\x^2},-\x);
%    \draw[->]   plot (\x,{sin(2*\x r)});
%    \draw[->]   plot (-\x,{-sin(2*\x r)});
\end{tikzpicture}
\end{enumerate}
\end{multicols}
\vspace{0.25in}
\begin{multicols}{2}
\begin{enumerate}
\item[19.] \ \\
\begin{tikzpicture}[domain=1.022:3.6, scale=0.7]
    %\draw[very thin,color=gray] (-5,-3) grid (5,3);
		\draw [<->](-4,0) -- coordinate (x axis mid) (4,0) node[below right] {$x$};
		\draw [<->](0,-3.5) -- coordinate (y axis mid) (0,3.5) node[above right] {$y$};
			\foreach \x in {-3,...,-1}
			\draw (\x,0.8pt) -- (\x,-0.8pt)
			node[anchor=north] {\scriptsize \x};
			\foreach \x in {1,...,3}
			\draw (\x,0.8pt) -- (\x,-0.8pt)
			node[anchor=north] {\scriptsize \x};
			%\foreach \y in {-3,...,-1}
			%\draw (1pt,\y) -- (-3pt,\y) 
			%node[anchor=east] {\scriptsize \y}; 
			\foreach \y in {1,...,3}
			\draw (1pt,\y) -- (-3pt,\y) 
			node[anchor=east] {\scriptsize \y}; 
			\foreach \y in {-3,...,-1}
			\draw (1pt,\y) -- (-3pt,\y) 
			node[anchor=west] {\scriptsize \y}; 
    \draw[->, domain=0:1.5]   plot (\x,{\x^2+1});
    \draw[->, domain=0:1.5]   plot (-\x,{-\x^2-1)});
		\draw[fill] (0,1) circle (0.10);
		\draw[fill] (0,-1) circle (0.10);
\end{tikzpicture}	
\hspace{1.3in}
\item[20.] \ \\
\begin{tikzpicture}[domain=2:3.6, scale=0.8]
    %\draw[very thin,color=gray] (-5,-3) grid (5,3);
		\draw [<->](-4,0) -- coordinate (x axis mid) (4,0) node[below right] {$x$};
		\draw [<->](0,-1) -- coordinate (y axis mid) (0,3.5) node[above right] {$y$};
			\foreach \x in {-3,...,-1}
			\draw (\x,1pt) -- (\x,-3pt)
			node[anchor=north] {\scriptsize \x};
			\foreach \x in {1,...,3}
			\draw (\x,1pt) -- (\x,-3pt)
			node[anchor=north] {\scriptsize \x};
			%\foreach \y in {-3,...,-1}
			%\draw (1pt,\y) -- (-3pt,\y) 
			%node[anchor=east] {\scriptsize \y}; 
			\foreach \y in {1,...,3}
			\draw (1pt,\y) -- (-3pt,\y) 
			node[anchor=east] {\scriptsize \y}; 
    \draw[->,domain=1:1.75]   plot (\x,{\x^2-1});
    \draw[-,domain=0:1]   plot (\x,{-\x^2+1});
    \draw[-,domain=0:1]   plot (-\x,{-\x^2+1});
    \draw[->,domain=1:1.75]   plot (-\x,{\x^2-1});
    %\draw[->]   plot (\x,{sqrt(\x-2)});
    %\draw[->]   plot (-\x,{sqrt(\x-2)});
		%\draw[fill] (2,0) circle (0.13);
		%\draw[fill] (-2,0) circle (0.13);
\end{tikzpicture}
\end{enumerate}
\end{multicols}
\begin{multicols}{4}
\begin{enumerate}
\item[21.] \ \\
\begin{tabular}{r|r}
$x$ & $y$\\
\hline
%&\\
$3$ & $-3$\\
%&\\
$2$ & $-2$\\
%&\\
$1$ & $-1$\\
%&\\
$0$ & $0$\\
%&\\
$1$ & $1$\\
%&\\
$2$ & $2$\\
%&\\
$3$ & $3$
\end{tabular}
\hspace{1in}
\item[22.] \ \\
\begin{tabular}{r|r}
$x$ & $y$\\
\hline
%&\\
$-3$ & $3$\\
%&\\
$-2$ & $2$\\
%&\\
$-1$ & $1$\\
%&\\
$0$ & $0$\\
%&\\
$1$ & $1$\\
%&\\
$2$ & $2$\\
%&\\
$3$ & $3$
\end{tabular}
\hspace{1in}
\item[23.] \ \\
\begin{tabular}{r|r}
$x$ & $y$\\
\hline
%&\\
$-3$ & $0$\\
%&\\
$-2$ & $0$\\
%&\\
$-1$ & $0$\\
%&\\
$0$ & $0$\\
%&\\
$1$ & $0$\\
%&\\
$2$ & $0$\\
%&\\
$3$ & $0$
\end{tabular}
\hspace{1in}
\item[24.] \ \\
\begin{tabular}{r|c}
$x$ & $y$\\
\hline
%&\\
$3$ & $8$\\
%&\\
$2$ & $4$\\
%&\\
$1$ & $2$\\
%&\\
$0$ & $1$\\
%&\\
$-1$ & $1/2$\\
%&\\
$-2$ & $1/4$\\
%&\\
$-3$ & $1/8$
\end{tabular}
\end{enumerate}
\end{multicols}

Determine whether each of the following equations represents $y$ as a function of $x$.

\begin{enumerate}
\begin{multicols}{3}
\item[25.] $y = x^{3} - x$
\item[26.] $y = \sqrt{x - 2}$
\item[27.] $3x+2y=6$
\end{multicols}

\begin{multicols}{3}
\item[28.] $x^{2} - y^{2} = 1$
\item[29.] $y = \dfrac{x}{x^{2} - 9}$
\item[30.] $x = -6$
\end{multicols}

\begin{multicols}{3}
\item[31.] $x = y^2 + 4$
\item[32.] $y = x^2 + 4$
\item[33.] $x^2 + y^2 = 4$
\end{multicols}
\end{enumerate}

For each of the following statements, find an expression for $f(x)$.

\begin{enumerate}
\item[34.] $f$ is a function that takes a real number $x$ and performs the following three steps in the order given: (1) multiply by 2; (2) add 3; (3) divide by 4.
\item[35.] $f$ is a function that takes a real number $x$ and performs the following three steps in the order given: (1) add 3; (2) multiply by 2; (3) divide by 4. 
\item[36.] $f$ is a function that takes a real number $x$ and performs the following three steps in the order given: (1) divide by 4; (2) add 3; (3) multiply by 2.
\item[37.] $f$ is a function that takes a real number $x$ and performs the following three steps in the order given: (1) multiply by 2; (2) add 3; (3) take the square root.
\item[38.] $f$ is a function that takes a real number $x$ and performs the following three steps in the order given: (1) add 3; (2) multiply by 2; (3) take the square root.
\item[39.] $f$ is a function that takes a real number $x$ and performs the following three steps in the order given: (1) add 3; (2) take the square root; (3) multiply by 2.
\item[40.] $f$ is a function that takes a real number $x$ and performs the following three steps in the order given: (1) take the square root; (2) subtract 13; (3) make the quantity the denominator of a fraction with numerator 4. 
\item[41.] $f$ is a function that takes a real number $x$ and performs the following three steps in the order given: (1) subtract 13; (2) take the square root; (3) make the quantity the denominator of a fraction with numerator 4.  
\item[42.] $f$ is a function that takes a real number $x$ and performs the following three steps in the order given: (1) take the square root; (2) make the quantity the denominator of a fraction with numerator 4; (3) subtract 13. 
\item[43.] $f$ is a function that takes a real number $x$ and performs the following three steps in the order given: (1) make the quantity the denominator of a fraction with numerator 4; (2) take the square root; (3) subtract 13.
\end{enumerate}

For each exercise, use the given function $f$ to find and simplify each of the {\bf nine} related values/expressions listed below.

\begin{multicols}{3}
\begin{itemize}
\item $f(1)$
\item $f(-3)$
\item $f\left(\frac{3}{2} \right)$
\end{itemize}
\end{multicols}

\begin{multicols}{3}
\begin{itemize}
\item $f(4x)$
\item $4f(x)$
\item $f(-x)$
\end{itemize}
\end{multicols}

\begin{multicols}{3}
\begin{itemize}
\item  $f(x-4)$
\item $f(x) - 4$
\item  $f\left(x^2\right)$
\end{itemize}
\end{multicols}

\begin{enumerate}
\begin{multicols}{2}
\item[44.]  $f(x) = 2x+1$
\item[45.]  $f(x) = 3 - 4x$
\item[46.] $f(x) = 2 - x^2$
\item[47.] $f(x) = x^2 - 3x + 2$
\item[48.] $f(x) = \sqrt{x-1}$
\item[49.] $f(x) = \dfrac{x}{x-1}$
\item[50.] $f(x) = 6$
\item[51.] $f(x) = 0$
\end{multicols}
\end{enumerate}

For each exercise, use the given function $f$ to find and simplify each of the {\bf nine} related values/expressions listed below.

\begin{multicols}{3}
\begin{itemize}
\item  $f(2)$
\item  $f(-2)$
\item  $f(2a)$
\end{itemize}
\end{multicols}

\begin{multicols}{3}
\begin{itemize}
\item  $2 f(a)$
\item $f(a+2)$
\item $f(a) + f(2)$
\end{itemize}
\end{multicols}

\begin{multicols}{3}
\begin{itemize}
\item  $f \left( \frac{2}{a} \right)$
\item $\frac{f(a)}{2}$
\item  $f(a + h)$
\end{itemize}
\end{multicols}

\begin{enumerate}
\begin{multicols}{2}
\item[52.] $f(x) = 2x-5$
\item[53.] $f(x) = 5-2x$
\item[54.] $f(x) = 2x^2 - 1$
\item[55.] $f(x) = 3x^2+3x-2$
\item[56.] $f(x) = \sqrt{2x+1}$
\item[57.] $f(x) = 1$
\item[58.] $f(x) = \dfrac{x}{2}$
\item[59.] $f(x) = \dfrac{2}{x}$
\end{multicols}
\end{enumerate}

In each of the following exercises, use the given function $f$ to find $f(0)$ and solve $f(x) = 0$

\begin{enumerate}
\begin{multicols}{2}
\item[60.] $f(x) = 2x - 1$
\item[61.] $f(x) = 3 - \frac{2}{5} x$
\item[62.] $f(x) = 2x^2 - 6$
\item[63.] $f(x) = x^2 - x - 12$
\item[64.] $f(x) = \sqrt{x+4}$
\item[65.] $f(x) = \sqrt{1-2x}$
\item[66.] $f(x) = \dfrac{3}{4-x}$
\item[67.] $f(x) = \dfrac{3x^2-12x}{4-x^2}$
\end{multicols}
\end{enumerate}

\subsection*{Identifying Domain and Range Graphically}

For each of the following graphs, identify the corresponding domain and range.  Express your answers using interval notation.
\begin{center}
\begin{multicols}{3}
\begin{enumerate}
\item[1.]
	\begin{tikzpicture}[xscale=0.4,yscale=0.4]
		\draw[step=1.0,gray,very thin,dotted] (-5,-5) grid (5,5);
		\draw [<->](-5,0) -- coordinate (x axis mid) (5,0) node[below right] {$x$};
		\draw [<->](0,-5) -- coordinate (y axis mid) (0,5) node[above right] {$y$};
		\foreach \x in {1,...,4} \draw (\x,2pt) -- (\x,-2pt) node[anchor=north] {\tiny \x};
		\foreach \x in {-1,...,-4} \draw (\x,2pt) -- (\x,-2pt) node[anchor=south] {\tiny \x};
		\foreach \y in {1,...,4} \draw (2pt,\y) -- (-2pt,\y) node[anchor=west] {\tiny \y}; 
		\foreach \y in {-1,...,-4} \draw (2pt,\y) -- (-2pt,\y) node[anchor=east] {\tiny \y}; 
		\draw [<->] plot [domain=-1.85:1.85, samples=100] (\x,{(\x)^2+1});
	\end{tikzpicture}
\columnbreak
\item[2.]
	\begin{tikzpicture}[xscale=0.4,yscale=0.4]
		\draw[step=1.0,gray,very thin,dotted] (-5,-5) grid (5,5);
		\draw [<->](-5,0) -- coordinate (x axis mid) (5,0) node[below right] {$x$};
		\draw [<->](0,-5) -- coordinate (y axis mid) (0,5) node[above right] {$y$};
		\foreach \x in {1,...,4} \draw (\x,2pt) -- (\x,-2pt) node[anchor=north] {\tiny \x};
		\foreach \x in {-1,...,-4} \draw (\x,2pt) -- (\x,-2pt) node[anchor=south] {\tiny \x};
		\foreach \y in {1,...,4} \draw (2pt,\y) -- (-2pt,\y) node[anchor=west] {\tiny \y}; 
		\foreach \y in {-1,...,-4} \draw (2pt,\y) -- (-2pt,\y) node[anchor=east] {\tiny \y}; 
		\draw [<->] plot [domain=-2.5:1.75, samples=100] (\x,{(\x+2)*(\x)*(\x-1)});
	\end{tikzpicture}
\columnbreak
\item[3.]
	\begin{tikzpicture}[xscale=0.4,yscale=0.4]
		\draw[step=1.0,gray,very thin,dotted] (-5,-5) grid (5,5);
		\draw [<->](-5,0) -- coordinate (x axis mid) (5,0) node[below right] {$x$};
		\draw [<->](0,-5) -- coordinate (y axis mid) (0,5) node[above right] {$y$};
		\foreach \x in {1,...,4} \draw (\x,2pt) -- (\x,-2pt) node[anchor=north] {\tiny \x};
		\foreach \x in {-1,...,-4} \draw (\x,2pt) -- (\x,-2pt) node[anchor=south] {\tiny \x};
		\foreach \y in {1,...,4} \draw (2pt,\y) -- (-2pt,\y) node[anchor=west] {\tiny \y}; 
		\foreach \y in {-1,...,-4} \draw (2pt,\y) -- (-2pt,\y) node[anchor=east] {\tiny \y}; 
		\draw [->] plot [domain=2:5, samples=100] (\x,{2*(\x-2)^0.5});
		\draw[fill] (2,0) circle (0.12) node[above] {};
	\end{tikzpicture}
\end{enumerate}
\end{multicols}
\end{center}
\begin{center}
\begin{multicols}{3}
\begin{enumerate}
\item[4.]
	\begin{tikzpicture}[xscale=0.4,yscale=0.4]
		\draw[step=1.0,gray,very thin,dotted] (-5,-5) grid (5,5);
		\draw [<->](-5,0) -- coordinate (x axis mid) (5,0) node[below right] {$x$};
		\draw [<->](0,-5) -- coordinate (y axis mid) (0,5) node[above right] {$y$};
		\foreach \x in {1,...,4} \draw (\x,2pt) -- (\x,-2pt) node[anchor=north] {\tiny \x};
		\foreach \x in {-1,...,-4} \draw (\x,2pt) -- (\x,-2pt) node[anchor=south] {\tiny \x};
		\foreach \y in {1,...,4} \draw (2pt,\y) -- (-2pt,\y) node[anchor=west] {\tiny \y}; 
		\foreach \y in {-1,...,-4} \draw (2pt,\y) -- (-2pt,\y) node[anchor=east] {\tiny \y}; 
		\draw [<->] plot [domain=-4.9:4.9, samples=100] (\x,{4/((\x)^2+1)});
	\end{tikzpicture}
\columnbreak
\item[5.]
	\begin{tikzpicture}[xscale=0.4,yscale=0.4]
		\draw[step=1.0,gray,very thin,dotted] (-5,-5) grid (5,5);
		\draw [<->](-5,0) -- coordinate (x axis mid) (5,0) node[below right] {$x$};
		\draw [<->](0,-5) -- coordinate (y axis mid) (0,5) node[above right] {$y$};
		\foreach \x in {1,...,4} \draw (\x,2pt) -- (\x,-2pt) node[anchor=north] {\tiny \x};
		\foreach \x in {-1,...,-4} \draw (\x,2pt) -- (\x,-2pt) node[anchor=south] {\tiny \x};
		\foreach \y in {1,...,4} \draw (2pt,\y) -- (-2pt,\y) node[anchor=west] {\tiny \y}; 
		\foreach \y in {-1,...,-4} \draw (2pt,\y) -- (-2pt,\y) node[anchor=east] {\tiny \y}; 
		\draw [-] plot [domain=-5:4, samples=100] (\x,{0.0502*(\x)^3 - 0.0344*(\x)^2 - 0.2010*\x + 2.138});
		\draw[fill] (-5,-4) circle (0.12) node[above] {};
		\draw[fill, white] (4,4) circle (0.12) node[above] {};
		\draw[line width=0.15mm] (4,4) circle (0.12) node[above] {};
	\end{tikzpicture}
\columnbreak
\item[6.]
	\begin{tikzpicture}[xscale=0.4,yscale=0.4]
		\draw[step=1.0,gray,very thin,dotted] (-5,-5) grid (5,5);
		\draw [<->](-5,0) -- coordinate (x axis mid) (5,0) node[below right] {$x$};
		\draw [<->](0,-5) -- coordinate (y axis mid) (0,5) node[above right] {$y$};
		\foreach \x in {1,...,4} \draw (\x,2pt) -- (\x,-2pt) node[anchor=north] {\tiny \x};
		\foreach \x in {-1,...,-4} \draw (\x,2pt) -- (\x,-2pt) node[anchor=south] {\tiny \x};
		\foreach \y in {1,...,4} \draw (2pt,\y) -- (-2pt,\y) node[anchor=west] {\tiny \y}; 
		\foreach \y in {-1,...,-4} \draw (2pt,\y) -- (-2pt,\y) node[anchor=east] {\tiny \y}; 
		\draw [<->] plot [domain=-5:5, samples=100] (\x,{2});
	\end{tikzpicture}
\end{enumerate}
\end{multicols}
\end{center}
\begin{center}
\begin{multicols}{3}
\begin{enumerate}
\item[7.]
	\begin{tikzpicture}[xscale=0.4,yscale=0.4]
		\draw[step=1.0,gray,very thin,dotted] (-5,-5) grid (5,5);
		\draw [<->](-5,0) -- coordinate (x axis mid) (5,0) node[below right] {$x$};
		\draw [<->](0,-5) -- coordinate (y axis mid) (0,5) node[above right] {$y$};
		\foreach \x in {1,...,4} \draw (\x,2pt) -- (\x,-2pt) node[anchor=north] {\tiny \x};
		\foreach \x in {-1,...,-4} \draw (\x,2pt) -- (\x,-2pt) node[anchor=south] {\tiny \x};
		\foreach \y in {1,...,4} \draw (2pt,\y) -- (-2pt,\y) node[anchor=west] {\tiny \y}; 
		\foreach \y in {-1,...,-4} \draw (2pt,\y) -- (-2pt,\y) node[anchor=east] {\tiny \y}; 
		\draw [<->] plot [domain=-1.7:1.7, samples=100] (\x,{(-4/((\x)^2-4))+1});
		\draw [<->] plot [domain=2.2:5, samples=100] (\x,{(-4/((\x)^2-4))+1});
		\draw [<->] plot [domain=-5:-2.2, samples=100] (\x,{(-4/((\x)^2-4))+1});
		\draw [<->, dashed, line width=0.2mm] plot [domain=-5:5, samples=100] (2,{\x});
		\draw [<->, dashed, line width=0.2mm] plot [domain=-5:5, samples=100] (-2,{\x});
		\draw [<->, dashed, line width=0.2mm] plot [domain=-5:5, samples=100] (\x,{1});
	\end{tikzpicture}
\columnbreak
\item[8.]
	\begin{tikzpicture}[xscale=0.4,yscale=0.4]
		\draw[step=1.0,gray,very thin,dotted] (-5,-5) grid (5,5);
		\draw [<->](-5,0) -- coordinate (x axis mid) (5,0) node[below right] {$x$};
		\draw [<->](0,-5) -- coordinate (y axis mid) (0,5) node[above right] {$y$};
		\foreach \x in {1,...,4} \draw (\x,2pt) -- (\x,-2pt) node[anchor=north] {\tiny \x};
		\foreach \x in {-1,...,-4} \draw (\x,2pt) -- (\x,-2pt) node[anchor=south] {\tiny \x};
		\foreach \y in {1,...,4} \draw (2pt,\y) -- (-2pt,\y) node[anchor=west] {\tiny \y}; 
		\foreach \y in {-1,...,-4} \draw (2pt,\y) -- (-2pt,\y) node[anchor=east] {\tiny \y}; 
		\draw [<->] plot [domain=-3.05:3.05, samples=100] (\x,{-0.25*((\x)^2-6.25)*((\x)^2-1)+1.25});
	\end{tikzpicture}
\columnbreak
\item[9.]
	\begin{tikzpicture}[xscale=0.4,yscale=0.4]
		\draw[step=1.0,gray,very thin,dotted] (-5,-5) grid (5,5);
		\draw [<->](-5,0) -- coordinate (x axis mid) (5,0) node[below right] {$x$};
		\draw [<->](0,-5) -- coordinate (y axis mid) (0,5) node[above right] {$y$};
		\foreach \x in {1,...,4} \draw (\x,2pt) -- (\x,-2pt) node[anchor=north] {\tiny \x};
		\foreach \x in {-1,...,-4} \draw (\x,2pt) -- (\x,-2pt) node[anchor=south] {\tiny \x};
		\foreach \y in {1,...,4} \draw (2pt,\y) -- (-2pt,\y) node[anchor=west] {\tiny \y}; 
		\foreach \y in {-1,...,-4} \draw (2pt,\y) -- (-2pt,\y) node[anchor=east] {\tiny \y}; 
		\draw [->] plot [domain=2:5, samples=100] (\x,{-2*(\x-2)^0.5});
		\draw[fill, white] (2,0) circle (0.12) node[above] {};
		\draw[line width=0.15mm] (2,0) circle (0.12) node[above] {};
		\draw [->] plot [domain=-2:-5, samples=100] (\x,{-2*(-\x-2)^0.5});
		\draw[fill, white] (-2,0) circle (0.12) node[above] {};
		\draw[line width=0.15mm] (-2,0) circle (0.12) node[above] {};
	\end{tikzpicture}
\end{enumerate}
\end{multicols}
\end{center}
\begin{center}
\begin{multicols}{3}
\begin{enumerate}
\item[10.]
	\begin{tikzpicture}[xscale=0.4,yscale=0.4]
		\draw[step=1.0,gray,very thin,dotted] (-5,-5) grid (5,5);
		\draw [<->](-5,0) -- coordinate (x axis mid) (5,0) node[below right] {$x$};
		\draw [<->](0,-5) -- coordinate (y axis mid) (0,5) node[above right] {$y$};
		\foreach \x in {1,...,4} \draw (\x,2pt) -- (\x,-2pt) node[anchor=north] {\tiny \x};
		\foreach \x in {-1,...,-4} \draw (\x,2pt) -- (\x,-2pt) node[anchor=south] {\tiny \x};
		\foreach \y in {1,...,4} \draw (2pt,\y) -- (-2pt,\y) node[anchor=west] {\tiny \y}; 
		\foreach \y in {-1,...,-4} \draw (2pt,\y) -- (-2pt,\y) node[anchor=east] {\tiny \y}; 
		\draw [<->] plot [domain=-3:3, samples=100] (\x,{-1*(\x)^2+4});
	\end{tikzpicture}
\columnbreak
\item[11.]
	\begin{tikzpicture}[xscale=0.4,yscale=0.4]
		\draw[step=1.0,gray,very thin,dotted] (-5,-5) grid (5,5);
		\draw [<->](-5,0) -- coordinate (x axis mid) (5,0) node[below right] {$x$};
		\draw [<->](0,-5) -- coordinate (y axis mid) (0,5) node[above right] {$y$};
		\foreach \x in {1,...,4} \draw (\x,2pt) -- (\x,-2pt) node[anchor=north] {\tiny \x};
		\foreach \x in {-1,...,-4} \draw (\x,2pt) -- (\x,-2pt) node[anchor=south] {\tiny \x};
		\foreach \y in {1,...,4} \draw (2pt,\y) -- (-2pt,\y) node[anchor=west] {\tiny \y}; 
		\foreach \y in {-1,...,-4} \draw (2pt,\y) -- (-2pt,\y) node[anchor=east] {\tiny \y}; 
		\draw [<->] plot [domain=-5:-0.25, samples=100] (\x,{1/(\x)});
		\draw [<->] plot [domain=5:0.25, samples=100] (\x,{-1/(\x)});
	\end{tikzpicture}
\columnbreak
\item[12.]
	\begin{tikzpicture}[xscale=0.4,yscale=0.4]
		\draw[step=1.0,gray,very thin,dotted] (-5,-5) grid (5,5);
		\draw [<->](-5,0) -- coordinate (x axis mid) (5,0) node[below right] {$x$};
		\draw [<->](0,-5) -- coordinate (y axis mid) (0,5) node[above right] {$y$};
		\foreach \x in {1,...,4} \draw (\x,2pt) -- (\x,-2pt) node[anchor=north] {\tiny \x};
		\foreach \x in {-1,...,-4} \draw (\x,2pt) -- (\x,-2pt) node[anchor=south] {\tiny \x};
		\foreach \y in {1,...,4} \draw (2pt,\y) -- (-2pt,\y) node[anchor=west] {\tiny \y}; 
		\foreach \y in {-1,...,-4} \draw (2pt,\y) -- (-2pt,\y) node[anchor=east] {\tiny \y}; 
		\draw [->] plot [domain=-2:5, samples=100] (\x,{-3*(\x+2)^0.5+3});
		\draw[fill] (-2,3) circle (0.12) node[above] {};
		\draw (-2,3) circle (0.12) node[above] {};
	\end{tikzpicture}
\end{enumerate}
\end{multicols}
\end{center}
\begin{center}
\begin{multicols}{3}
\begin{enumerate}
\item[13.]
	\begin{tikzpicture}[xscale=0.4,yscale=0.4]
		\draw[step=1.0,gray,very thin,dotted] (-5,-5) grid (5,5);
		\draw [<->](-5,0) -- coordinate (x axis mid) (5,0) node[below right] {$x$};
		\draw [<->](0,-5) -- coordinate (y axis mid) (0,5) node[above right] {$y$};
		\foreach \x in {1,...,4} \draw (\x,2pt) -- (\x,-2pt) node[anchor=north] {\tiny \x};
		\foreach \x in {-1,...,-4} \draw (\x,2pt) -- (\x,-2pt) node[anchor=south] {\tiny \x};
		\foreach \y in {1,...,4} \draw (2pt,\y) -- (-2pt,\y) node[anchor=west] {\tiny \y}; 
		\foreach \y in {-1,...,-4} \draw (2pt,\y) -- (-2pt,\y) node[anchor=east] {\tiny \y}; 
		\draw [->] plot [domain=1:2.7, samples=100] (\x,{-5*(\x-1)+4});
		\draw [<-] plot [domain=-4:1, samples=100] (\x,{1.667*(\x-1)+4});
	\end{tikzpicture}
\columnbreak
\item[14.]
	\begin{tikzpicture}[xscale=0.4,yscale=0.4]
		\draw[step=1.0,gray,very thin,dotted] (-5,-5) grid (5,5);
		\draw [<->](-5,0) -- coordinate (x axis mid) (5,0) node[below right] {$x$};
		\draw [<->](0,-5) -- coordinate (y axis mid) (0,5) node[above right] {$y$};
		\foreach \x in {1,...,4} \draw (\x,2pt) -- (\x,-2pt) node[anchor=north] {\tiny \x};
		\foreach \x in {-1,...,-4} \draw (\x,2pt) -- (\x,-2pt) node[anchor=south] {\tiny \x};
		\foreach \y in {1,...,4} \draw (2pt,\y) -- (-2pt,\y) node[anchor=west] {\tiny \y}; 
		\foreach \y in {-1,...,-4} \draw (2pt,\y) -- (-2pt,\y) node[anchor=east] {\tiny \y}; 
		\draw [->] plot [domain=1:5, samples=100] (\x,{2});
		\draw [<-] plot [domain=-4:0, samples=100] (\x,{1.5*(\x)+1});
		\draw[fill,white] (1,2) circle (0.12) node[above] {};
		\draw[line width=0.15mm] (1,2) circle (0.12) node[above] {};
		\draw[fill] (0,1) circle (0.12) node[above] {};
	\end{tikzpicture}
\columnbreak
\item[15.]
	\begin{tikzpicture}[xscale=0.4,yscale=0.4]
		\draw[step=1.0,gray,very thin,dotted] (-5,-5) grid (5,5);
		\draw [<->](-5,0) -- coordinate (x axis mid) (5,0) node[below right] {$x$};
		\draw [<->](0,-5) -- coordinate (y axis mid) (0,5) node[above right] {$y$};
		\foreach \x in {1,...,4} \draw (\x,2pt) -- (\x,-2pt) node[anchor=north] {\tiny \x};
		\foreach \x in {-1,...,-4} \draw (\x,2pt) -- (\x,-2pt) node[anchor=south] {\tiny \x};
		\foreach \y in {1,...,4} \draw (2pt,\y) -- (-2pt,\y) node[anchor=west] {\tiny \y}; 
		\foreach \y in {-1,...,-4} \draw (2pt,\y) -- (-2pt,\y) node[anchor=east] {\tiny \y}; 
		\draw [-] plot [domain=-3:3, samples=100] (\x,{2*sin(60*\x)});
		\draw[fill] (3,0) circle (0.12) node[above] {};
		\draw[fill] (-3,0) circle (0.12) node[above] {};
	\end{tikzpicture}
\end{enumerate}
\end{multicols}
\end{center}
\begin{center}
\begin{multicols}{3}
\begin{enumerate}
\item[16.]
	\begin{tikzpicture}[xscale=0.4,yscale=0.4]
		\draw[step=1.0,gray,very thin,dotted] (-5,-5) grid (5,5);
		\draw [<->](-5,0) -- coordinate (x axis mid) (5,0) node[below right] {$x$};
		\draw [<->](0,-5) -- coordinate (y axis mid) (0,5) node[above right] {$y$};
		\foreach \x in {1,...,4} \draw (\x,2pt) -- (\x,-2pt) node[anchor=north] {\tiny \x};
		\foreach \x in {-1,...,-4} \draw (\x,2pt) -- (\x,-2pt) node[anchor=south] {\tiny \x};
		\foreach \y in {1,...,4} \draw (2pt,\y) -- (-2pt,\y) node[anchor=west] {\tiny \y}; 
		\foreach \y in {-1,...,-4} \draw (2pt,\y) -- (-2pt,\y) node[anchor=east] {\tiny \y}; 
		\draw [->] plot [domain=-2:3.33, samples=100] (\x,{0.8888*(\x-1)^2-3});
		\draw[fill] (-2,5) circle (0.12) node[above] {};
	\end{tikzpicture}
\columnbreak
\item[17.]
	\begin{tikzpicture}[xscale=0.4,yscale=0.4]
		\draw[step=1.0,gray,very thin,dotted] (-5,-5) grid (5,5);
		\draw [<->](-5,0) -- coordinate (x axis mid) (5,0) node[below right] {$x$};
		\draw [<->](0,-5) -- coordinate (y axis mid) (0,5) node[above right] {$y$};
		\foreach \x in {1,...,4} \draw (\x,2pt) -- (\x,-2pt) node[anchor=north] {\tiny \x};
		\foreach \x in {-1,...,-4} \draw (\x,2pt) -- (\x,-2pt) node[anchor=south] {\tiny \x};
		\foreach \y in {1,...,4} \draw (2pt,\y) -- (-2pt,\y) node[anchor=west] {\tiny \y}; 
		\foreach \y in {-1,...,-4} \draw (2pt,\y) -- (-2pt,\y) node[anchor=east] {\tiny \y}; 
		\draw [-] plot [domain=4:5, samples=100] (\x,{-2*(\x-6)});
		\draw [-] plot [domain=3:4, samples=100] (\x,{3*(\x-4)+4});
		\draw [-] plot [domain=0:3, samples=100] (\x,{-1*(\x)-1});
		\draw[fill,white] (3,1) circle (0.12) node[above] {};
		\draw[line width=0.15mm] (3,1) circle (0.12) node[above] {};
		\draw[fill,white] (3,-4) circle (0.12) node[above] {};
		\draw[line width=0.15mm] (3,-4) circle (0.12) node[above] {};
		\draw[fill] (0,-1) circle (0.12) node[above] {};
		\draw[fill] (4,4) circle (0.12) node[above] {};
		\draw[fill] (5,2) circle (0.12) node[above] {};
	\end{tikzpicture}
\columnbreak
\item[18.]
	\begin{tikzpicture}[xscale=0.4,yscale=0.4]
		\draw[step=1.0,gray,very thin,dotted] (-5,-5) grid (5,5);
		\draw [<->](-5,0) -- coordinate (x axis mid) (5,0) node[below right] {$x$};
		\draw [<->](0,-5) -- coordinate (y axis mid) (0,5) node[above right] {$y$};
		\foreach \x in {1,...,4} \draw (\x,2pt) -- (\x,-2pt) node[anchor=north] {\tiny \x};
		\foreach \x in {-1,...,-4} \draw (\x,2pt) -- (\x,-2pt) node[anchor=south] {\tiny \x};
		\foreach \y in {1,...,4} \draw (2pt,\y) -- (-2pt,\y) node[anchor=west] {\tiny \y}; 
		\foreach \y in {-1,...,-4} \draw (2pt,\y) -- (-2pt,\y) node[anchor=east] {\tiny \y}; 
		\draw [<->] plot [domain=1.4:5, samples=100] (\x,{(1/(\x-1))+2});
		\draw [-] plot [domain=-4:-2, samples=100] (\x,{(-2*(2+\x))/\x-2});
		\draw[fill] (-2,-2) circle (0.12) node[above] {};
		\draw [<->, dashed, line width=0.2mm] plot [domain=-5:5, samples=100] (\x,{2});
		\draw [<->, dashed, line width=0.2mm] plot [domain=-5:5, samples=100] ({1},\x);
		\draw[fill,white] (-4,-3) circle (0.12) node[above] {};
		\draw[] (-4,-3) circle (0.12) node[above] {};
	\end{tikzpicture}
\end{enumerate}
\end{multicols}
\end{center}
\end{document}