\documentclass[12pt]{book}
\raggedbottom
\usepackage[top=1in,left=1in,bottom=1in,right=1in,headsep=0.25in]{geometry}	
\usepackage{amssymb,amsmath,amsthm,amsfonts}
\usepackage{chapterfolder,docmute,setspace}
\usepackage{cancel,multicol,tikz,verbatim,framed,polynom,enumitem,tikzpagenodes}
\usepackage[colorlinks, hyperindex, plainpages=false, linkcolor=blue, urlcolor=blue, pdfpagelabels]{hyperref}
\usepackage[type={CC},modifier={by-sa},version={4.0},]{doclicense}

\theoremstyle{definition}
\newtheorem{example}{Example}
\newcommand{\Desmos}{\href{https://www.desmos.com/}{Desmos}}
\setlength{\parindent}{0in}
\setlist{itemsep=0in}
\setlength{\parskip}{0.1in}
\setcounter{secnumdepth}{0}
\input{lesson_order}

\newcommand{\tmmathbf}[1]{\ensuremath{\boldsymbol{#1}}}
\newcommand{\tmop}[1]{\ensuremath{\operatorname{#1}}}

\begin{document}
\section{Fundamental Functions (L\arabic{lesson_fundamental_functions})}
{\bf Objective: Graph and identify the domain, range, and intercepts of any of the ten fundamental functions.}\par
In this section, we have listed ten fundamental function types which will be referenced throughout the rest of the text, as well as one example of each.  Each type of function represents a ``building block'' for understanding the concepts of a traditional algebra course.\par
Students should be able to both identify and sketch a graph of each function, as well as identify its intercepts, domain (both graphically and algebraically), and range (graphically).  Each representative form in the table below includes some element of generalization to reinforce understanding.\par
\begin{center}
\begin{tabular}{|c|c|l|}
\hline
Function Type & Representative Form & ~~~~~~Example\\
\hline
&&\\
Linear & $mx+b$ & $f(x)=3x-4$\\
&&\\
Quadratic & $ax^2+bx+c$ & $g(x)=x^2$\\
&&\\
Square Root & $\sqrt{x-h}$ & $k(x)=\sqrt{x}$\\
&&\\
Absolute Value & $|x-h|$ & $\ell(x)=|x|$\\
&&\\
Cubic & $(x-h)^3$ & $m(x)=x^3$\\
&&\\
Cube Root & $\sqrt[3]{x-h}$ & $n(x)=\sqrt[3]{x}$\\
&&\\
Reciprocal (Rational) & $\dfrac{1}{x-h}$ & $p(x)=\dfrac{1}{x}$\\
&&\\
Semicircular & $\sqrt{r^2-x^2},~r>0$ & $q(x)=\sqrt{9-x^2}$\\
&&\\
Exponential* & $a^x,~a>0, a\neq 1$ & $r(x)=2^x$\\
&&\\
Logarithmic* & $\log_a(x),~a>0, a\neq1$ & $s(x)=\log_2(x)$\\
&&\\
\hline
\end{tabular}
\end{center}
*We have included Exponential and Logarithmic functions for a more complete list.  These functions are more formally treated in a Precalculus setting. 

\newpage

\begin{multicols}{2}
\begin{center}
\begin{tabular}{ll}
Function Type: &\textbf{Linear} $(m\neq 0)$\\
Example: &$f(x)=3x-4$
\end{tabular}
\\
\vspace{0.25in}
\begin{tabular}{c|c}
	$x$ & $f(x)$\\
	\hline
 & \\
 $-3$ & $-13$\\
 & \\
 $-2$ & $-10$\\
 & \\
 $-1$ & $-7$\\
 & \\
 0 & $-4$\\
 & \\
 1 & $-1$\\
 & \\
 $\dfrac{4}{3}$ & 0\\
 & \\
 2 & 2\\
 & \\
 3 & 5\\
 & \\
\end{tabular}
\end{center}
~\\
\vspace{0.5in}
~\\
\begin{tikzpicture}[xscale=0.75,yscale=0.75]
\draw[step=1.0,gray,very thin,dotted] (-4,-5) grid (4,5);
\draw [<->](-4,0) -- coordinate (x axis mid) (4,0) node[below right] {$x$};
\draw [<->](0,-5.25) -- coordinate (y axis mid) (0,5.25) node[above right] {$y$};
\foreach \x in {-3,...,-1}
\draw (\x,1pt) -- (\x,-3pt)
node[anchor=north] {\scriptsize \x};
\foreach \x in {1,...,3}
\draw (\x,1pt) -- (\x,-3pt)
node[anchor=north] {\scriptsize \x};
\foreach \y in {-5,...,-1}
\draw (1pt,\y) -- (-3pt,\y) 
node[anchor=east] {\scriptsize \y}; 
\foreach \y in {1,...,5}
\draw (1pt,\y) -- (-3pt,\y) 
node[anchor=east] {\scriptsize \y}; 
\draw [<->, domain=-0.33:3] plot (\x, {3*(\x)-4});
\draw[fill] (1.33,0) circle (0.075) node[above right] {}; 
\draw (1.4,0.6) node[right] {\scriptsize $\left(\dfrac{4}{3},0\right)$}; 
\draw[fill] (0,-4) circle (0.075);
\end{tikzpicture}
\begin{center}
Graph of $f(x)=3x-4$
\end{center}
\end{multicols}
\ \par
\begin{tabular}{ll}
$y-$intercept: & $(0,-4)$\\
&\\
$x-$intercept(s): & $\left(\frac{4}{3},0\right)$\\
&\\
Domain: & $(-\infty,\infty)$\\
&\\
Range: & $(-\infty,\infty)$
\end{tabular}
\\
\vspace{0.25in}

Notes: If $m=0$, then the corresponding graph of $f(x)=b$ is a horizontal line.  The domain of $f$ is still $(-\infty,\infty)$, but the range consists of a single value, $\{b\}$. 
\newpage

\begin{multicols}{2}
\begin{center}
\begin{tabular}{ll}
Function Type: &\textbf{Quadratic} \\
Example: &$g(x)=x^2$
\end{tabular}
\\
\vspace{0.25in}
\begin{tabular}{c|c}
	$x$ & $g(x)$\\
	\hline
 & \\
 $-3$ & 9\\
 & \\
 $-2$ & 4\\
 & \\
 $-1$ & 1\\
 & \\
 0 & 0\\
 & \\
 1 & 1\\
 & \\
 2 & 4\\
 & \\
 3 & 9\\
 & \\
\end{tabular}
\end{center}
~\\
~\\
~\\
~\\
~\\
\begin{tikzpicture}[xscale=0.75,yscale=0.75]
\draw[step=1.0,gray,very thin,dotted] (-4,-1) grid (4,10);
\draw [<->](-4,0) -- coordinate (x axis mid) (4,0) node[below right] {$x$};
\draw [<->](0,-1) -- coordinate (y axis mid) (0,10) node[above right] {$y$};
\foreach \x in {-3,...,-1}
\draw (\x,1pt) -- (\x,-3pt)
node[anchor=north] {\scriptsize \x};
\foreach \x in {1,...,3}
\draw (\x,1pt) -- (\x,-3pt)
node[anchor=north] {\scriptsize \x};
%\foreach \y in {-3,...,-1}
%\draw (1pt,\y) -- (-3pt,\y); 
%node[anchor=east] {\scriptsize \y}; 
\foreach \y in {1,...,9}
\draw (1pt,\y) -- (-3pt,\y) 
node[anchor=east] {\scriptsize \y}; 
\draw [<->, domain=-3.1:3.1] plot (\x, {(\x)^2});
\draw[fill] (0,0) circle (0.075);
%\draw[fill] (-1,3) circle (0.075);
\end{tikzpicture}
\begin{center}
Graph of $g(x)=x^2$
\end{center}
\end{multicols}
\ \par
%\vspace{0.25in}
\begin{tabular}{ll}
$y-$intercept: & $(0,0)$\\
&\\
$x-$intercept(s): & $(0,0)$\\
&\\
Domain: & $(-\infty,\infty)$\\
&\\
Range: & $[0,\infty),$ or $y\geq 0$
\end{tabular}
\\
~\\
%\vspace{0.25in}

Notes: The domain of any quadratic function is $(-\infty,\infty)$.  If $g(x)=a(x-h)^2+k$, is a quadratic function in vertex form, then if $a>0$, the corresponding parabola will be concave {\it up}, and the range of $g$ will be $[k,\infty)$.  If $a<0$, then the corresponding parabola will be concave {\it down}, and the range of $g$ will be $(-\infty,k]$.  Quadratics will be covered extensively in the next chapter.
\newpage

\begin{multicols}{2}
\begin{center}
\begin{tabular}{ll}
Function Type: &\textbf{Square Root}\\
Example: &$k(x)=\sqrt{x}$
\end{tabular}
\\
\vspace{0.25in}
\begin{tabular}{c|c}
	$x$ & $k(x)$\\
	\hline
 & \\
$-1$ & undefined\\
 & \\
0 & 0\\
 & \\
1 & 1\\
 & \\
2 & $\sqrt{2}\approx 1.41$\\
 & \\
3 & $\sqrt{3}\approx 1.73$\\
 & \\
4 & 2\\
 & \\
9 & 3\\
 & \\
\end{tabular}
\end{center}

\columnbreak
\
\vspace{1.5in}
\ 
\begin{center}
\begin{tikzpicture}[xscale=0.6,yscale=0.6]
\draw[step=1.0,gray,very thin,dotted] (-1,-1) grid (10,4);
\draw [<->](-1,0) -- coordinate (x axis mid) (10,0) node[below right] {$x$};
\draw [<->](0,-1) -- coordinate (y axis mid) (0,4) node[above right] {$y$};
%\foreach \x in {-3,...,-1}
%\draw (\x,1pt) -- (\x,-3pt)
%node[anchor=north] {\scriptsize \x};
\foreach \x in {1,...,9}
\draw (\x,1pt) -- (\x,-3pt)
node[anchor=north] {\scriptsize \x};
%\foreach \y in {-3,...,-1}
%\draw (1pt,\y) -- (-3pt,\y); 
%node[anchor=east] {\scriptsize \y}; 
\foreach \y in {1,...,3}
\draw (1pt,\y) -- (-3pt,\y) 
node[anchor=east] {\scriptsize \y}; 
\draw[domain=0:3.1,->] plot ({\x^(2)},\x);
\draw[fill] (0,0) circle (0.075);
%\draw[fill] (-1,3) circle (0.075);
\end{tikzpicture}
\end{center}
\begin{center}
Graph of $k(x)=\sqrt{x}$
\end{center}
\end{multicols}
\ \par
\begin{tabular}{ll}
$y-$intercept: & $(0,0)$\\
&\\
$x-$intercept(s): & $(0,0)$\\
&\\
Domain: & $[0,\infty),$ or $x\geq 0$\\
&\\
Range: & $[0,\infty),$ or $y\geq 0$
\end{tabular}
\\
~\\
%\vspace{0.25in}

Notes: The domain of a square root function of the form $k(x)=\sqrt{x-h}$ will be $x>h$.  The range will be the same as in the example, $[0,\infty)$.  The $x-$intercept will be $(h,0)$. 

\newpage

\begin{multicols}{2}
\begin{center}
\begin{tabular}{ll}
Function Type: &\textbf{Absolute Value}\\
Example: &$\ell(x)=|x|$
\end{tabular}
\\
\vspace{0.25in}
\begin{tabular}{c|c}
	$x$ & $\ell(x)$\\
	\hline
 & \\
 $-3$ & 3\\
 & \\
 $-2$ & 2\\
 & \\
 $-1$ & 1\\
 & \\
 0 & 0\\
 & \\
 1 & 1\\
 & \\
 2 & 2\\
 & \\
 3 & 3\\
 & \\
\end{tabular}
\end{center}

\columnbreak
\ 
\vspace{1in}
\ 
\begin{center}
\begin{tikzpicture}[xscale=0.7,yscale=0.7]
\draw[step=1.0,gray,very thin,dotted] (-5,-1) grid (5,5);
\draw [<->](-5,0) -- coordinate (x axis mid) (5,0) node[below right] {$x$};
\draw [<->](0,-1) -- coordinate (y axis mid) (0,5) node[above right] {$y$};
\foreach \x in {-5,...,-1}
\draw (\x,1pt) -- (\x,-3pt) node[anchor=north] {\scriptsize \x};
\foreach \x in {1,...,5}
\draw (\x,1pt) -- (\x,-3pt) node[anchor=north] {\scriptsize \x};
%\foreach \y in {-3,...,-1}
%\draw (1pt,\y) -- (-3pt,\y); 
%node[anchor=east] {\scriptsize \y}; 
\foreach \y in {1,...,5}
\draw (1pt,\y) -- (-3pt,\y) node[anchor=east] {\scriptsize \y}; 
\draw [<-, domain=-5:0] plot (\x, {-\x});
\draw [->, domain=0:5] plot (\x, {\x});
%\draw[fill] (1,-5) circle (0.075);
%\draw[fill] (-1,3) circle (0.075);
\end{tikzpicture}
\end{center}
\begin{center}
Graph of $\ell(x)=|x|$
\end{center}
\end{multicols}
\ \par
\begin{tabular}{ll}
$y-$intercept: & $(0,0)$\\
&\\
$x-$intercept(s): & $(0,0)$\\
&\\
Domain: & $(-\infty,\infty)$ \\
&\\
Range: & $[0,\infty),$ or $y\geq 0$
\end{tabular}
\\
~\\

Notes: The domain and range of an absolute value function of the form $\ell(x)=|x-h|$ will remain the same as above.  The $x-$intercept will be $(h,0)$. 

\newpage

\begin{multicols}{2}
\begin{center}
\begin{tabular}{ll}
Function Type: &\textbf{Cubic}\\
Example: &$m(x)=x^3$
\end{tabular}
\\
\vspace{0.25in}
\begin{tabular}{c|c}
	$x$ & $m(x)$\\
	\hline
 & \\
 $-3$ & $-27$\\
 & \\
 $-2$ & -8\\
 & \\
 $-1$ & $-1$\\
 & \\
 0 & 0\\
 & \\
 1 & 1\\
 & \\
 2 & 8\\
 & \\
 3 & 27\\
 & \\
\end{tabular}
\end{center}
\
\columnbreak
\ 
\begin{center}
\begin{tikzpicture}[xscale=1,yscale=0.15]
\draw[xstep=1.0,ystep=5,gray,very thin,dotted] (-3.5,-31) grid (3.5,31);
\draw [<->](-3.5,0) -- coordinate (x axis mid) (3.5,0) node[below right] {$x$};
\draw [<->](0,-31) -- coordinate (y axis mid) (0,31) node[above right] {$y$};
\foreach \x in {-3,...,-1}
\draw (\x,1pt) -- (\x,-3pt) node[anchor=north] {\scriptsize \x};
\foreach \x in {1,...,3}
\draw (\x,1pt) -- (\x,-3pt) node[anchor=north] {\scriptsize \x};
\foreach \y in {-30,-25,...,-5}
\draw (1pt,\y) -- (-3pt,\y) node[anchor=east] {\scriptsize \y}; 
\foreach \y in {5,10,...,30}
\draw (1pt,\y) -- (-3pt,\y) node[anchor=east] {\scriptsize \y}; 
\draw [<->, domain=-3.1:3.1] plot (\x, {(\x)^3});
%\draw[fill] (1,-5) circle (0.075);
\end{tikzpicture}
\end{center}
\begin{center}
Graph of $m(x)=x^3$
\end{center}
\end{multicols}
\ \par
\begin{tabular}{ll}
$y-$intercept: & $(0,0)$\\
&\\
$x-$intercept(s): & $(0,0)$\\
&\\
Domain: & $(-\infty,\infty)$ \\
&\\
Range: & $(-\infty,\infty)$
\end{tabular}
\\
~\\

Notes: The domain and range of a cubic function of the form $m(x)=(x-h)^3$ will remain the same as above.  The $x-$intercept will be $(h,0)$.  The $y-$intercept will be $(0,-h^3)$. 

\newpage

\begin{multicols}{2}
\begin{center}
\begin{tabular}{ll}
Function Type: &\textbf{Cube Root}\\
Example: &$n(x)=\sqrt[3]{x}$
\end{tabular}
\\
\vspace{0.25in}
\begin{tabular}{c|c}
	$x$ & $n(x)$\\
	\hline
 & \\
 $-27$ & $-3$\\
 & \\
 $-8$ & $-2$\\
 & \\
 $-1$ & $-1$\\
 & \\
 0 & 0\\
 & \\
 1 & 1\\
 & \\
 8 & 2\\
 & \\
 27 & 3\\
 & \\
\end{tabular}
\end{center}
\
\columnbreak
~\\
\vspace{0.5in}
~\\
\begin{center}
\begin{tikzpicture}[xscale=0.2,yscale=0.75]
\draw[xstep=5.0,ystep=1,gray,very thin,dotted] (-21,-3.5) grid (21,3.5);
\draw [<->](-21,0) -- coordinate (x axis mid) (21,0) node[below right] {$x$};
\draw [<->](0,-3.5) -- coordinate (y axis mid) (0,3.5) node[above right] {$y$};
\foreach \x in {-20,-15,...,-5}
\draw (\x,1pt) -- (\x,-3pt) node[anchor=north] {\scriptsize \x};
\foreach \x in {5,10,...,20}
\draw (\x,1pt) -- (\x,-3pt) node[anchor=north] {\scriptsize \x};
\foreach \y in {-3,...,-1}
\draw (1pt,\y) -- (-3pt,\y) node[anchor=east] {\scriptsize \y}; 
\foreach \y in {1,...,3}
\draw (1pt,\y) -- (-3pt,\y) node[anchor=east] {\scriptsize \y}; 
\draw [<->, domain=-2.7:2.7] plot ({(\x)^3},\x);
%\draw[fill] (1,-5) circle (0.075);
\end{tikzpicture}
\end{center}
\begin{center}
Graph of $n(x)=\sqrt[3]{x}$
\end{center}
\end{multicols}
\ \par
\begin{tabular}{ll}
$y-$intercept: & $(0,0)$\\
&\\
$x-$intercept(s): & $(0,0)$\\
&\\
Domain: & $(-\infty,\infty)$ \\
&\\
Range: & $(-\infty,\infty)$
\end{tabular}
\\
~\\

Notes: The domain and range of a cube root function of the form $n(x)=\sqrt[3]{x-h}$ will remain the same as above.  The $x-$intercept will be $(h,0)$.  The $y-$intercept will be $(0,-\sqrt[3]{h})$. 

\newpage

\begin{multicols}{2}
\begin{center}
\begin{tabular}{ll}
Function Type: &\textbf{Reciprocal (Rational)}\\
Example: &$p(x)=\dfrac{1}{x}$
\end{tabular}
\\
\vspace{0.25in}
\begin{tabular}{c|c}
	$x$ & $p(x)$\\
	\hline
 & \\
$-3$& $-\dfrac{1}{3}$\\
 & \\
 $-2$& $-\dfrac{1}{2}$\\
 & \\
 $-1$& $-1$\\
 & \\
 0& undefined\\
 & \\
 1& 1\\
 & \\
 2& $\dfrac{1}{2}$\\
 & \\
 3& $\dfrac{1}{3}$\\
 & \\
\end{tabular}
\end{center}

\columnbreak
~\\
\vspace{0.5in}
~\\
\begin{center}
\begin{tikzpicture}[xscale=0.85,yscale=0.85]
	\draw[xstep=1,ystep=1,gray,very thin,dotted] (-4.25,-4.25) grid (4.25,4.25);
	\draw [<->](-4.25,0) -- coordinate (x axis mid) (4.25,0) node[below right] {$x$};
	\draw [<->](0,-4.25) -- coordinate (y axis mid) (0,4.25) node[above right] {$y$};
	\draw [<->] plot [domain=0.25:4, samples=100] (\x,{1/\x});
	\draw [<->] plot [domain=-4:-0.25, samples=100] (\x,{1/\x});
%	\draw[fill] (1,1) circle (0.05);
%  \draw[fill] (2,0.5) circle (0.05);
%	\draw[fill] (3,0.333) circle (0.05);
%	\draw[fill] (0.5,2) circle (0.05);
%	\draw[fill] (0.333,3) circle (0.05);
%	\draw[fill] (-1,-1) circle (0.05);
%  \draw[fill] (-2,-0.5) circle (0.05);
%	\draw[fill] (-3,-0.333) circle (0.05);
%	\draw[fill] (-0.5,-2) circle (0.05);
%	\draw[fill] (-0.333,-3) circle (0.05);
	\foreach \x in {1,...,4}
		\draw (\x,2pt) -- (\x,-2pt)	node[anchor=north] {\scriptsize \x};
	\foreach \x in {-4,...,-1}
		\draw (\x,2pt) -- (\x,-2pt)	node[anchor=south] {\scriptsize \x};
	\foreach \y in {1,...,4}
		\draw (2pt,\y) -- (-2pt,\y)	node[anchor=east] {\scriptsize \y}; 
	\foreach \y in {-4,...,-1}
		\draw (2pt,\y) -- (-2pt,\y)	node[anchor=west] {\scriptsize \y}; 
\end{tikzpicture}
\end{center}
\begin{center}
Graph of $p(x)=\dfrac{1}{x}$
\end{center}
\end{multicols}
\begin{tabular}{ll}
$y-$intercept: & None\\
&\\
$x-$intercept(s): & None\\
&\\
Domain: & $(-\infty,0)\cup(0,\infty),$ or $x\neq 0$\\
&\\
Range: & $(-\infty,0)\cup(0,\infty),$ or $y\neq 0$
\end{tabular}
\\

Notes: The reciprocal function $\dfrac{1}{x}$ gets its name since each $y-$coordinate is the reciprocal of its corresponding $x-$coordinate, and vice versa.  Although the more general representative function $\dfrac{1}{x-h}$ does not uphold this reciprocal property, we can still categorize both the reciprocal form and the more general form as specific types of {\it rational} functions.  The domain of a function of the form $p(x)=\dfrac{1}{x-h}$ will be $(-\infty,h)\cup(h,\infty)$, or $x\neq h$.  The range, however, will remain the same as the reciprocal function, $\dfrac{1}{x}$.  The graph of $p(x)=\dfrac{1}{x-h}$ will have no $x-$intercept.  The $y-$intercept will be at $\left(0,-\dfrac{1}{h}\right)$. 

\newpage

\begin{multicols}{2}
\begin{center}
\begin{tabular}{ll}
Function Type: &\textbf{Semicircular}\\
Example: &$q(x)=\sqrt{9-x^2}$
\end{tabular}
\\
\vspace{0.25in}
\begin{tabular}{c|c}
	$x$ & $q(x)$\\
	\hline
 & \\
$-3$ & 0\\
 & \\
$-2$ & $\sqrt{5}$\\
 & \\
$-1$ & $\sqrt{8}$\\
 & \\
0 & 0\\
 & \\
1 & $\sqrt{8}$\\
 & \\
2 & $\sqrt{5}$\\
 & \\
9 & 0\\
 & \\
\end{tabular}
\end{center}
~\\
\vspace{1in}
~\\
\begin{center}
\begin{tikzpicture}[xscale=1,yscale=1]
	\draw[xstep=1,ystep=1,gray,very thin,dotted] (-3.5,-0.5) grid (3.5,3.5);
	\draw [<->](-3.5,0) -- coordinate (x axis mid) (3.5,0) node[below right] {$x$};
	\draw [<->](0,-0.5) -- coordinate (y axis mid) (0,3.5) node[above right] {$y$};
	%\draw [-] plot [domain=-3:3, samples=100] (\x,{1/\x});
	\begin{scope}
				\clip (-3,0) rectangle (3,3);
				\draw (0,0) circle(3);
	\end{scope}	
	\draw[fill] (3,0) circle (0.05);
	\draw[fill] (-3,0) circle (0.05);
	\foreach \x in {1,...,3}
		\draw (\x,2pt) -- (\x,-2pt)	node[anchor=north] {\scriptsize \x};
	\foreach \x in {-3,...,-1}
		\draw (\x,2pt) -- (\x,-2pt)	node[anchor=north] {\scriptsize \x};
	\foreach \y in {1,...,3}
		\draw (2pt,\y) -- (-2pt,\y)	node[anchor=east] {\scriptsize \y}; 
%	\foreach \y in {-4,...,-1}
%		\draw (2pt,\y) -- (-2pt,\y)	node[anchor=west] {\scriptsize \y}; 
\end{tikzpicture}
\end{center}
\begin{center}
Graph of $q(x)=\sqrt{9-x^2}$
\end{center}
\end{multicols}
\vspace{0.5in}
\begin{tabular}{ll}
$y-$intercept: & $(0,3)$\\
&\\
$x-$intercept(s): & $(-3,0)$ and $(3,0)$\\
&\\
Domain: & $[-3,3],$ or $-3\leq x\leq 3$\\
&\\
Range: & $[0,3],$ or $0\leq y\leq 3$
\end{tabular}
\\
~\\

Notes: The domain of a semicircular function of the form $q(x)=\sqrt{r^2-x^2}$ will be $[-r,r]$, or $-r\leq x\leq r$.  The range will be $[0,r],$ or $0\leq y\leq r$.  The graph of $q(x)=\sqrt{r^2-x^2}$ will have $x-$intercepts at $(\pm r,0)$ and a $y-$intercept at $(0,r)$. 

\newpage

\begin{multicols}{2}
\begin{center}
\begin{tabular}{ll}
Function Type: &\textbf{Exponential}\\
Example: &$r(x)=2^x$
\end{tabular}
\\
\vspace{0.25in}
\begin{tabular}{c|c}
	$x$ & $r(x)$\\
	\hline
 & \\
 $-3$ & $\dfrac{1}{8}$\\
 & \\
 $-2$ & $\dfrac{1}{4}$\\
 & \\
 $-1$ & $\dfrac{1}{2}$\\
 & \\
 0 & 1\\
 & \\
 1 & 2\\
 & \\
 2 & 4\\
 & \\
 3 & 8\\
 & \\
\end{tabular}
\end{center}
\columnbreak
~\\
\vspace{0.25in}
~\\
\begin{tikzpicture}[xscale=0.9, yscale=0.9]
	\draw[xstep=1,ystep=1,gray,very thin,dotted] (-3.9,-0.5) grid (3.9,8.5);
	\draw [<->](-3.9,0) -- coordinate (x axis mid) (3.9,0) node[below right] {$x$};
	\draw [<->](0,-0.5) -- coordinate (y axis mid) (0,8.5) node[above right] {$y$};
	\draw [<->] plot [domain=-3.1:3.1,samples=100] (\x,{2^(\x)});
	\foreach \x in {-3,...,-1}
	\draw (\x,1pt) -- (\x,-3pt) node[anchor=north] {\scriptsize \x};
	\foreach \x in {1,...,3}
	\draw (\x,1pt) -- (\x,-3pt) node[anchor=north] {\scriptsize \x};
%	\foreach \y in {-3,...,-1}
%	\draw (1pt,\y) -- (-3pt,\y) node[anchor=east] {\scriptsize \y}; 
	\foreach \y in {1,...,8}
	\draw (1pt,\y) -- (-3pt,\y) node[anchor=east] {\scriptsize \y};
\end{tikzpicture}
\begin{center}
Graph of $r(x)=2^x$
\end{center}
\end{multicols}
\vspace{0.5in}
\begin{tabular}{ll}
$y-$intercept: & $(0,1)$\\
&\\
$x-$intercept(s): & None\\
&\\
Domain: & $(-\infty,\infty)$ \\
&\\
Range: & $(0,\infty),$ or $y>0$
\end{tabular}
\\
~\\

Notes: The domain, range, $x-$ and $y-$ intercepts of an exponential function of the form $r(x)=a^x,$ where $a$ is positive ($a\neq 1$) will all be the same as above. 

\newpage

\begin{multicols}{2}
\begin{center}
\begin{tabular}{ll}
Function Type: &\textbf{Logarithmic}\\
Example: &$s(x)=\log_2x$
\end{tabular}
\\
\vspace{0.25in}
\begin{tabular}{c|c}
	$x$ & $s(x)$\\
	\hline
 & \\
$\dfrac{1}{8}$ & $-3$\\
 & \\
$\dfrac{1}{4}$ & $-2$\\
 & \\
$\dfrac{1}{2}$ & $-1$\\
 & \\
1 & 0\\
 & \\
2 & 1\\
 & \\
4 & 2\\
 & \\
8 & 3\\
 & \\
\end{tabular}
\end{center}

\columnbreak
~\\
\vspace{0.5in}
~\\
\begin{tikzpicture}[xscale=0.75, yscale=1]
	\draw[xstep=1,ystep=1,gray,very thin,dotted] (-0.5,-3.9) grid (8.5,3.9);
  \draw [<->](-0.5,0) -- coordinate (x axis mid) (8.5,0) node[below right] {$x$};
	\draw [<->](0,-3.9) -- coordinate (y axis mid) (0,3.9) node[above right] {$y$};
	\draw [<->] plot [domain=8.2:0.1,samples=100] (\x,{log2(\x)});
	%\foreach \x in {-3,...,-1}
	%\draw (\x,1pt) -- (\x,-3pt) node[anchor=north] {\scriptsize \x};
	\foreach \x in {1,...,8}
	\draw (\x,1pt) -- (\x,-3pt) node[anchor=north] {\scriptsize \x};
	\foreach \y in {-3,...,-1}
	\draw (1pt,\y) -- (-3pt,\y) node[anchor=east] {\scriptsize \y}; 
	\foreach \y in {1,...,3}
	\draw (1pt,\y) -- (-3pt,\y) node[anchor=east] {\scriptsize \y}; 
\end{tikzpicture}
\begin{center}
Graph of $s(x)=\log_2x$
\end{center}
\end{multicols}
\vspace{0.5in}
\begin{tabular}{ll}
$y-$intercept: & None\\
&\\
$x-$intercept(s): & $(1,0)$\\
&\\
Domain: & $(0,\infty),$ or $x>0$\\
&\\
Range: & $(-\infty,\infty)$
\end{tabular}
\\
~\\

Notes: The domain, range, $x-$ and $y-$ intercepts of a logarithmic function of the form $s(x)=\log_ax,$ where $a$ is positive ($a\neq 1$) will all be the same as above. 
\end{document}