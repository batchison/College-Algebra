\documentclass[12pt]{book}
\raggedbottom
\usepackage[top=1in,left=1in,bottom=1in,right=1in,headsep=0.25in]{geometry}	
\usepackage{amssymb,amsmath,amsthm,amsfonts}
\usepackage{chapterfolder,docmute,setspace}
\usepackage{cancel,multicol,tikz,verbatim,framed,polynom,enumitem,tikzpagenodes}
\usepackage[colorlinks, hyperindex, plainpages=false, linkcolor=blue, urlcolor=blue, pdfpagelabels]{hyperref}
\usepackage[type={CC},modifier={by-sa},version={4.0},]{doclicense}

\theoremstyle{definition}
\newtheorem{example}{Example}
\newcommand{\Desmos}{\href{https://www.desmos.com/}{Desmos}}
\setlength{\parindent}{0in}
\setlist{itemsep=0in}
\setlength{\parskip}{0.1in}
\setcounter{secnumdepth}{0}
\input{lesson_order}

\newcommand{\tmmathbf}[1]{\ensuremath{\boldsymbol{#1}}}
\newcommand{\tmop}[1]{\ensuremath{\operatorname{#1}}}

\begin{document}
\section{Identifying Domain and Range Graphically (L\arabic{lesson_finding_domain_and_range_graphically})}
{\bf Objective: Identify the domain and range of a function that is described graphically.}
In this section, we will first discuss how one can identify the domain and range of a function using its graph.  Later, we will explore finding the domain of a function using algebraic methods.  As finding the range of a function using algebraic methods can often prove quite challenging, we will postpone this topic for another time.
\begin{example}~~~Find the domain and range of the function $f$ whose graph is given below.
\begin{center}
	\begin{tikzpicture}[xscale=0.75,yscale=0.75]
		\draw[step=1.0,gray,very thin,dotted] (-3,-2) grid (3,5);
		\draw [<->](-3,0) -- coordinate (x axis mid) (3,0) node[below right] {$x$};
		\draw [<->](0,-2) -- coordinate (y axis mid) (0,5) node[above right] {$y$};
		\foreach \x in {1,...,2} \draw (\x,2pt) -- (\x,-2pt) node[anchor=north] {\scriptsize \x};
		\foreach \x in {-1,...,-2} \draw (\x,2pt) -- (\x,-2pt) node[anchor=north] {\scriptsize \x};
		\foreach \y in {1,...,4} \draw (2pt,\y) -- (-2pt,\y) node[anchor=east] {\scriptsize \y}; 
		\foreach \y in {-1} \draw (2pt,\y) -- (-2pt,\y) node[anchor=east] {\scriptsize \y}; 
		\draw [<->] plot [domain=-2.35:1, samples=100] (\x,{-1*(\x)^2+4});
		\draw[fill, white] (1,3) circle (0.08) node[above] {};
		\draw[line width=0.15mm] (1,3) circle (0.08) node[above] {};
		\draw (0,-2.75) node {\scriptsize The graph of $f$};
	\end{tikzpicture}
\end{center}
\end{example}
To determine the domain and range of $f$, we need to determine which $x$ and $y$-values respectively occur as coordinates of points on the given graph.\par
To find the domain, it will be helpful to imagine collapsing the curve onto the $x$-axis and determining the portion of the $x$-axis that gets covered.  This is often described as {\it projecting} the curve onto the $x$-axis.  Before we project, we need to pay attention to two subtle notations on the graph:  the arrowhead on the lower left corner of the graph indicates that the graph continues to curve downwards to the left forever; and the open circle at $(1,3)$ indicates that the point $(1,3)$ is \textit{not} on the graph, but all the points on the curve leading up to $(1,3)$ are on the graph.
\begin{center}
\begin{multicols}{2}
	\begin{tikzpicture}[xscale=0.7,yscale=0.7]
		\draw[step=1.0,gray,very thin,dotted] (-4,-3) grid (4,5);
		\draw [<->](-4,0) -- coordinate (x axis mid) (4,0) node[below right] {$x$};
		\draw [<->](0,-3) -- coordinate (y axis mid) (0,5) node[above right] {$y$};
		\foreach \x in {1,...,3} \draw (\x,2pt) -- (\x,-2pt) node[anchor=north] {\scriptsize \x};
		\foreach \x in {-1,...,-3} \draw (\x,2pt) -- (\x,-2pt) node[anchor=north] {\scriptsize \x};
		\foreach \y in {1,...,4} \draw (2pt,\y) -- (-2pt,\y) node[anchor=east] {\scriptsize \y}; 
		\foreach \y in {-1,-2} \draw (2pt,\y) -- (-2pt,\y) node[anchor=east] {\scriptsize \y}; 
		\draw [<->] plot [domain=-2.35:1, samples=100] (\x,{-1*(\x)^2+4});
		\draw[fill, white] (1,3) circle (0.08) node[above] {};
		\draw[line width=0.15mm] (1,3) circle (0.08) node[above] {};
		\draw [<-] plot [domain=0.5:2.25, samples=100] (2,{\x}) node[above] {\scriptsize project down};
		\draw [<-] plot [domain=-0.75:-2.25, samples=100] (-3,{\x}) node[below] {\scriptsize project up};
	\end{tikzpicture}

\columnbreak

	\begin{tikzpicture}[xscale=0.7,yscale=0.7]
		\draw[step=1.0,gray,very thin,dotted] (-4,-3) grid (4,5);
		\draw [<->](-4,0) -- coordinate (x axis mid) (4,0) node[below right] {$x$};
		\draw [<->](0,-3) -- coordinate (y axis mid) (0,5) node[above right] {$y$};
		\foreach \x in {1,...,3} \draw (\x,2pt) -- (\x,-2pt) node[anchor=north] {\scriptsize \x};
		\foreach \x in {-1,...,-3} \draw (\x,2pt) -- (\x,-2pt) node[anchor=north] {\scriptsize \x};
		\foreach \y in {1,...,4} \draw (2pt,\y) -- (-2pt,\y) node[anchor=east] {\scriptsize \y}; 
		\foreach \y in {-1,-2} \draw (2pt,\y) -- (-2pt,\y) node[anchor=east] {\scriptsize \y}; 
		\draw [<->] plot [domain=-2.35:1, samples=100] (\x,{-1*(\x)^2+4});
		\draw[fill, white] (1,3) circle (0.08) node[above] {};
		\draw[line width=0.15mm] (1,3) circle (0.08) node[above] {};
		\draw [<-, line width=.75mm] plot [domain=-4:1, samples=100] (\x,{0});
		\draw (0.95,0) node {\huge )};
	\end{tikzpicture}
\end{multicols}
\end{center}

We see from the figure that if we project the graph of $f$ to the $x$-axis, we get all real numbers less than $1$.  Using interval notation, we write the domain of $f$ as $(-\infty, 1)$.
\newpage
To determine the range of $f$, we use a similar method, projecting the curve onto the $y$-axis as follows.
\begin{center}
\begin{multicols}{2}
	\begin{tikzpicture}[xscale=0.7,yscale=0.7]
		\draw[step=1.0,gray,very thin,dotted] (-4,-3) grid (4,5);
		\draw [<->](-4,0) -- coordinate (x axis mid) (4,0) node[below right] {$x$};
		\draw [<->](0,-3) -- coordinate (y axis mid) (0,5) node[above right] {$y$};
		\foreach \x in {1,...,3} \draw (\x,2pt) -- (\x,-2pt) node[anchor=north] {\scriptsize \x};
		\foreach \x in {-1,...,-3} \draw (\x,2pt) -- (\x,-2pt) node[anchor=north] {\scriptsize \x};
		\foreach \y in {1,...,4} \draw (2pt,\y) -- (-2pt,\y) node[anchor=east] {\scriptsize \y}; 
		\foreach \y in {-1,-2} \draw (2pt,\y) -- (-2pt,\y) node[anchor=east] {\scriptsize \y}; 
		\draw [<->] plot [domain=-2.35:1, samples=100] (\x,{-1*(\x)^2+4});
		\draw[fill, white] (1,3) circle (0.08) node[above] {};
		\draw[line width=0.15mm] (1,3) circle (0.08) node[above] {};
		\draw [->] plot [domain=2.5:1, samples=100] (\x,{2});
		\draw (1.75,1.5) node {\scriptsize project left};
		\draw [->] plot [domain=-3.5:-2, samples=100] (\x,{2.5});
		\draw (-2.75,2) node {\scriptsize project right};
	\end{tikzpicture}

\columnbreak

	\begin{tikzpicture}[xscale=0.7,yscale=0.7]
		\draw[step=1.0,gray,very thin,dotted] (-4,-3) grid (4,5);
		\draw [<->](-4,0) -- coordinate (x axis mid) (4,0) node[below right] {$x$};
		\draw [<->](0,-3) -- coordinate (y axis mid) (0,5) node[above right] {$y$};
		\foreach \x in {1,...,3} \draw (\x,2pt) -- (\x,-2pt) node[anchor=north] {\scriptsize \x};
		\foreach \x in {-1,...,-3} \draw (\x,2pt) -- (\x,-2pt) node[anchor=north] {\scriptsize \x};
		\foreach \y in {1,...,4} \draw (2pt,\y) -- (-2pt,\y) node[anchor=east] {\scriptsize \y}; 
		\foreach \y in {-1,-2} \draw (2pt,\y) -- (-2pt,\y) node[anchor=east] {\scriptsize \y}; 
		\draw [<->] plot [domain=-2.35:1, samples=100] (\x,{-1*(\x)^2+4});
		\draw[fill, white] (1,3) circle (0.08) node[above] {};
		\draw[line width=0.15mm] (1,3) circle (0.08) node[above] {};
		\draw [->, line width=.75mm] plot [domain=4:-3, samples=100] (0,{\x});
		\node[rotate=90] at (0,4) {\huge ]};
	\end{tikzpicture}
\end{multicols}
\end{center}
Note that even though there is an open circle at $(1,3)$, we still include the $y$ value of $3$ in our range, since the point $(-1,3)$ is on the graph of $f$.  We also include $y=4$ in our answer, since the point $(0,4)$ is also on our graph.  Consequently, the range of $f$ is all real numbers less than or equal to $4$, or $(-\infty, 4]$.
\newpage
\end{document}