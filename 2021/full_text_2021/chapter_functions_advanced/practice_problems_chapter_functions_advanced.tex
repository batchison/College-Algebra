\documentclass[12pt]{book}
\raggedbottom
\usepackage[top=1in,left=1in,bottom=1in,right=1in,headsep=0.25in]{geometry}	
\usepackage{amssymb,amsmath,amsthm,amsfonts}
\usepackage{chapterfolder,docmute,setspace}
\usepackage{cancel,multicol,tikz,verbatim,framed,polynom,enumitem,tikzpagenodes}
\usepackage[colorlinks, hyperindex, plainpages=false, linkcolor=blue, urlcolor=blue, pdfpagelabels]{hyperref}
\usepackage[type={CC},modifier={by-sa},version={4.0},]{doclicense}

\theoremstyle{definition}
\newtheorem{example}{Example}
\newcommand{\Desmos}{\href{https://www.desmos.com/}{Desmos}}
\setlength{\parindent}{0in}
\setlist{itemsep=0in}
\setlength{\parskip}{0.1in}
\setcounter{secnumdepth}{0}
% This document is used for ordering of lessons.  If an instructor wishes to change the ordering of assessments, the following steps must be taken:

% 1) Reassign the appropriate numbers for each lesson in the \setcounter commands included in this file.
% 2) Rearrange the \include commands in the master file (the file with 'Course Pack' in the name) to accurately reflect the changes.  
% 3) Rarrange the \items in the measureable_outcomes file to accurately reflect the changes.  Be mindful of page breaks when moving items.
% 4) Re-build all affected files (master file, measureable_outcomes file, and any lessons whose numbering has changed).

%Note: The placement of each \newcounter and \setcounter command reflects the original/default ordering of topics (linears, systems, quadratics, functions, polynomials, rationals).

\newcounter{lesson_solving_linear_equations}
\newcounter{lesson_equations_containing_absolute_values}
\newcounter{lesson_graphing_lines}
\newcounter{lesson_two_forms_of_a_linear_equation}
\newcounter{lesson_parallel_and_perpendicular_lines}
\newcounter{lesson_linear_inequalities}
\newcounter{lesson_compound_inequalities}
\newcounter{lesson_inequalities_containing_absolute_values}
\newcounter{lesson_graphing_systems}
\newcounter{lesson_substitution}
\newcounter{lesson_elimination}
\newcounter{lesson_quadratics_introduction}
\newcounter{lesson_factoring_GCF}
\newcounter{lesson_factoring_grouping}
\newcounter{lesson_factoring_trinomials_a_is_1}
\newcounter{lesson_factoring_trinomials_a_neq_1}
\newcounter{lesson_solving_by_factoring}
\newcounter{lesson_square_roots}
\newcounter{lesson_i_and_complex_numbers}
\newcounter{lesson_vertex_form_and_graphing}
\newcounter{lesson_solve_by_square_roots}
\newcounter{lesson_extracting_square_roots}
\newcounter{lesson_the_discriminant}
\newcounter{lesson_the_quadratic_formula}
\newcounter{lesson_quadratic_inequalities}
\newcounter{lesson_functions_and_relations}
\newcounter{lesson_evaluating_functions}
\newcounter{lesson_finding_domain_and_range_graphically}
\newcounter{lesson_fundamental_functions}
\newcounter{lesson_finding_domain_algebraically}
\newcounter{lesson_solving_functions}
\newcounter{lesson_function_arithmetic}
\newcounter{lesson_composite_functions}
\newcounter{lesson_inverse_functions_definition_and_HLT}
\newcounter{lesson_finding_an_inverse_function}
\newcounter{lesson_transformations_translations}
\newcounter{lesson_transformations_reflections}
\newcounter{lesson_transformations_scalings}
\newcounter{lesson_transformations_summary}
\newcounter{lesson_piecewise_functions}
\newcounter{lesson_functions_containing_absolute_values}
\newcounter{lesson_absolute_as_piecewise}
\newcounter{lesson_polynomials_introduction}
\newcounter{lesson_sign_diagrams_polynomials}
\newcounter{lesson_factoring_quadratic_type}
\newcounter{lesson_factoring_summary}
\newcounter{lesson_polynomial_division}
\newcounter{lesson_synthetic_division}
\newcounter{lesson_end_behavior_polynomials}
\newcounter{lesson_local_behavior_polynomials}
\newcounter{lesson_rational_root_theorem}
\newcounter{lesson_polynomials_graphing_summary}
\newcounter{lesson_polynomial_inequalities}
\newcounter{lesson_rationals_introduction_and_terminology}
\newcounter{lesson_sign_diagrams_rationals}
\newcounter{lesson_horizontal_asymptotes}
\newcounter{lesson_slant_and_curvilinear_asymptotes}
\newcounter{lesson_vertical_asymptotes}
\newcounter{lesson_holes}
\newcounter{lesson_rationals_graphing_summary}

\setcounter{lesson_solving_linear_equations}{1}
\setcounter{lesson_equations_containing_absolute_values}{2}
\setcounter{lesson_graphing_lines}{3}
\setcounter{lesson_two_forms_of_a_linear_equation}{4}
\setcounter{lesson_parallel_and_perpendicular_lines}{5}
\setcounter{lesson_linear_inequalities}{6}
\setcounter{lesson_compound_inequalities}{7}
\setcounter{lesson_inequalities_containing_absolute_values}{8}
\setcounter{lesson_graphing_systems}{9}
\setcounter{lesson_substitution}{10}
\setcounter{lesson_elimination}{11}
\setcounter{lesson_quadratics_introduction}{16}
\setcounter{lesson_factoring_GCF}{17}
\setcounter{lesson_factoring_grouping}{18}
\setcounter{lesson_factoring_trinomials_a_is_1}{19}
\setcounter{lesson_factoring_trinomials_a_neq_1}{20}
\setcounter{lesson_solving_by_factoring}{21}
\setcounter{lesson_square_roots}{22}
\setcounter{lesson_i_and_complex_numbers}{23}
\setcounter{lesson_vertex_form_and_graphing}{24}
\setcounter{lesson_solve_by_square_roots}{25}
\setcounter{lesson_extracting_square_roots}{26}
\setcounter{lesson_the_discriminant}{27}
\setcounter{lesson_the_quadratic_formula}{28}
\setcounter{lesson_quadratic_inequalities}{29}
\setcounter{lesson_functions_and_relations}{12}
\setcounter{lesson_evaluating_functions}{13}
\setcounter{lesson_finding_domain_and_range_graphically}{14}
\setcounter{lesson_fundamental_functions}{15}
\setcounter{lesson_finding_domain_algebraically}{30}
\setcounter{lesson_solving_functions}{31}
\setcounter{lesson_function_arithmetic}{32}
\setcounter{lesson_composite_functions}{33}
\setcounter{lesson_inverse_functions_definition_and_HLT}{34}
\setcounter{lesson_finding_an_inverse_function}{35}
\setcounter{lesson_transformations_translations}{36}
\setcounter{lesson_transformations_reflections}{37}
\setcounter{lesson_transformations_scalings}{38}
\setcounter{lesson_transformations_summary}{39}
\setcounter{lesson_piecewise_functions}{40}
\setcounter{lesson_functions_containing_absolute_values}{41}
\setcounter{lesson_absolute_as_piecewise}{42}
\setcounter{lesson_polynomials_introduction}{43}
\setcounter{lesson_sign_diagrams_polynomials}{44}
\setcounter{lesson_factoring_quadratic_type}{46}
\setcounter{lesson_factoring_summary}{45}
\setcounter{lesson_polynomial_division}{47}
\setcounter{lesson_synthetic_division}{48}
\setcounter{lesson_end_behavior_polynomials}{49}
\setcounter{lesson_local_behavior_polynomials}{50}
\setcounter{lesson_rational_root_theorem}{51}
\setcounter{lesson_polynomials_graphing_summary}{52}
\setcounter{lesson_polynomial_inequalities}{53}
\setcounter{lesson_rationals_introduction_and_terminology}{54}
\setcounter{lesson_sign_diagrams_rationals}{55}
\setcounter{lesson_horizontal_asymptotes}{56}
\setcounter{lesson_slant_and_curvilinear_asymptotes}{57}
\setcounter{lesson_vertical_asymptotes}{58}
\setcounter{lesson_holes}{59}
\setcounter{lesson_rationals_graphing_summary}{60}

\newcommand{\tmmathbf}[1]{\ensuremath{\boldsymbol{#1}}}
\newcommand{\tmop}[1]{\ensuremath{\operatorname{#1}}}

\begin{document}
\section{Practice Problems}
\subsection*{Identifying Domain Algebraically}

Find the domain of each of the following functions.  Express your answers using interval notation.
\begin{enumerate}
\begin{multicols}{3}
\item $f (x) = |3x-2|$
\item $g (x) = (3x-2)^2$
\item $h(x) = \dfrac{1}{3x-2}$
\item $k (x) = \sqrt[]{3x-2}$
\item $k (x) = \sqrt[3]{3x-2}$
\item $\ell (x) = \sqrt[4]{2-3x}$
\item $m(x) = \dfrac{x-2}{\sqrt[]{3x-2}}$
\item $n(x) = \dfrac{\sqrt[]{3x-2}}{x-2}$
\end{multicols}
\end{enumerate}

Find the domain of each of the following functions.  Express your answers using interval notation.
\begin{enumerate}[resume]
\begin{multicols}{2}
\item $g(x)-4x^2$
\item $f(x) = x^{4} - 13x^{3} + 56x^{2} - 19$
\item $g(x) = x^2 - 4$
\item $k(x) = \dfrac{x}{x - 8}$
\item $h(x)=\dfrac{x-5}{x+4}$
\item $h(x) = \dfrac{x-2}{x+1}$
\item $k(x) = \dfrac{x-2}{x-2}$  
\item $k(x) = \dfrac{3x}{x^2+x-2}$
\item $g(x) = \dfrac{2x}{x^2-9}$
\item $f(x) = \dfrac{2x}{x^2+9}$
\item $h(x) = \dfrac{x+4}{x^2 - 36}$
\item $f(x) = \sqrt{3-x}$
\item $g(x) = \sqrt{2x+5}$  
\item $f(x)=5\sqrt{x-1}$
\item $h(x) = 9x\sqrt{x+3}$
\item $k(x) = \dfrac{\sqrt{7-x}}{x^2+1}$  
\item $f(x) = \sqrt{6x-2}$
\item $g(x) = \dfrac{6}{\sqrt{6x-2}}$
\item $k(x)=\dfrac{4}{\sqrt{x-3}}$
\item $g(x) = \dfrac{x}{\sqrt{x - 8}}$
\item $h(x) = \sqrt[3]{6x-2}$
\item $k(x) = \dfrac{6}{4 - \sqrt{6x-2}}$
\item $f(x) = \dfrac{\sqrt{6x-2}}{x^2-36}$
\item $g(x) = \dfrac{\sqrt[3]{6x-2}}{x^2+36}$
\item $h(x) = \sqrt{x - 7} + \sqrt{9 - x}$
\item $h(t) = \dfrac{\sqrt{t} - 8}{5-t}$ 
\item $f(r) = \dfrac{\sqrt{r}}{r - 8}$
\item $k(v) = \dfrac{1}{4 - \dfrac{1}{v^{2}}}$
\item $f(y) = \sqrt[3]{\dfrac{y}{y - 8}}$
\item $k(w) = \dfrac{w - 8}{5 - \sqrt{w}}$
\end{multicols}
\end{enumerate}
\subsection*{Combining Functions}
\subsubsection{Function Arithmetic}
In each of the following exercises, use the pair of functions $f$ and $g$ to find the following values if they exist.

\begin{multicols}{3}
\begin{itemize}
\item  $(f+g)(2)$ 
\item  $(f-g)(-1)$
\item  $(g-f)(1)$
\end{itemize}
\end{multicols}

\begin{multicols}{3}
\begin{itemize}
\item  $(fg)\left(\frac{1}{2}\right)$
\item  $\left(\frac{f}{g}\right)(0)$
\item  $\left(\frac{g}{f}\right)\left(-2\right)$
\end{itemize}
\end{multicols}

\begin{enumerate}
\begin{multicols}{2}
\item  $f(x) = 3x+1$~~~  $g(x) = 4-x$
\item  $f(x) = x^2$ ~~~ $g(x) = -2x+1$
\end{multicols}

\begin{multicols}{2}
\item  $f(x) = x^2 - x$ ~~~  $g(x) = 12-x^2$
\item  $f(x) = 2x^3$ ~ \mbox{$g(x) = -x^2-2x-3$}
\end{multicols}

\begin{multicols}{2}
\item  $f(x) = \sqrt{x+3}$ ~~~  $g(x) = 2x-1$
\item  $f(x) = \sqrt{4-x}$ ~ $g(x) = \sqrt{x+2}$
\end{multicols}

\begin{multicols}{2}
\item  $f(x) = 2x$ ~~~  $g(x) = \dfrac{1}{2x+1}$
\item  $f(x) = x^2$ ~~~ $g(x) = \dfrac{3}{2x-3}$
\end{multicols}

\begin{multicols}{2}
\item  $f(x) = x^2$ ~~~  $g(x) = \dfrac{1}{x^2}$
\item  $f(x) = x^2+1$ ~~~ $g(x) = \dfrac{1}{x^2+1}$
\end{multicols}
\end{enumerate}

In each of the following exercises, use the pair of functions $f$ and $g$ to find the domain of the indicated function then find and simplify an expression for it.

\begin{multicols}{4}
\begin{itemize}
\item  $(f+g)(x)$
\item  $(f-g)(x)$
\item  $(fg)(x)$
\item  $\left(\frac{f}{g}\right)(x)$
\end{itemize}
\end{multicols}

\begin{enumerate}[resume]
\begin{multicols}{2}
\item $f(x) = 2x+1$ ~~~ $g(x) = x-2$
\item $f(x) = 1-4x$ ~~~ $g(x) = 2x-1$
\end{multicols}

\begin{multicols}{2}
\item $f(x) = x^2$ ~~~ $g(x) = 3x-1$
\item $f(x) = x^2-x$ ~~~ $g(x) = 7x$
\end{multicols}

\begin{multicols}{2}
\item $f(x) = x^2-4$ ~~~ $g(x) = 3x+6$
\item \mbox{$f(x) = -x^2+x+6$ ~ $g(x) = x^2-9$}
\end{multicols}

\begin{multicols}{2}
\item $f(x) = \dfrac{x}{2}$ ~~~ $g(x) = \dfrac{2}{x}$
\item $f(x) =x-1$ ~~~ $g(x) = \dfrac{1}{x-1}$
\end{multicols}

\begin{multicols}{2}
\item $f(x) = x$ ~~~ $g(x) = \sqrt{x+1}$
\item $f(x) = g(x) = \sqrt{x-5}$
\end{multicols}
\end{enumerate}

For each of the following exercises, let $f$ be the function defined by \[f = \{(-3, 4), (-2, 2), (-1, 0), (0, 1), (1, 3), (2, 4), (3, -1)\}\] and let $g$ be the function defined \[g = \{(-3, -2), (-2, 0), (-1, -4), (0, 0), (1, -3), (2, 1), (3, 2)\}.\] Use $f$ and $g$ to compute each of the indicated values if they exist.

\begin{enumerate}[resume]
\begin{multicols}{3}
\item $(f + g)(-3)$
\item $(f - g)(2)$
\item $(fg)(-1)$
\end{multicols}

\begin{multicols}{3}
\item $(g + f)(1)$
\item $(g - f)(3)$
\item $(gf)(-3)$
\end{multicols}

\begin{multicols}{3}
\item $\left(\frac{f}{g}\right)(-2)$
\item $\left(\frac{f}{g}\right)(-1)$
\item $\left(\frac{f}{g}\right)(2)$
\end{multicols}

\begin{multicols}{3}
\item $\left(\frac{g}{f}\right)(-1)$
\item $\left(\frac{g}{f}\right)(3)$
\item $\left(\frac{g}{f}\right)(-3)$
\end{multicols}
\end{enumerate}
\newpage
\subsubsection{Composite Functions}
In each of the following exercises, use the given pair of functions to find the following values if they exist.

\begin{multicols}{3}
\begin{itemize}
\item  $(g\circ f)(0)$
\item  $(f\circ g)(-1)$
\item  $(f \circ f)(2)$
\end{itemize}
\end{multicols}

\begin{multicols}{3}
\begin{itemize}
\item  $(g\circ f)(-3)$
\item  $(f\circ g)\left(\frac{1}{2}\right)$
\item  $(f \circ f)(-2)$
\end{itemize}
\end{multicols}

\begin{enumerate}
\begin{multicols}{2}
\item  $f(x) = x^2$, $g(x) = 2x+1$
\item  $f(x) = 4-x$, $g(x) = 1-x^2$
\end{multicols}

\begin{multicols}{2}
\item  $f(x) = 4-3x$, $g(x) = |x|$
\item  $f(x) = |x-1|$, $g(x) = x^2-5$
\end{multicols}

\begin{multicols}{2}
\item  $f(x) = 4x+5$, $g(x) = \sqrt{x}$
\item  $f(x) = \sqrt{3-x}$, $g(x) = x^2+1$
\end{multicols}

\begin{multicols}{2}
\item  $f(x) = \dfrac{3}{1-x}$, $g(x) = \dfrac{4x}{x^2+1}$
\item  $f(x) = \dfrac{x}{x+5}$, $g(x) = \dfrac{2}{7-x^2}$
\end{multicols}
\end{enumerate}

In each of the following exercises, use the given pair of functions to find and simplify expressions for the following functions and state the domain of each using interval notation.

\begin{multicols}{3}
\begin{itemize}
\item  $(g \circ f)(x)$
\item  $(f \circ g)(x)$
\item  $(f \circ f)(x)$
\end{itemize}
\end{multicols}

\begin{enumerate}[resume]
\begin{multicols}{2}
\item  $f(x) = 2x+3$, $g(x) = x^2-9$
\item  $f(x) = x^2 -x+1$, $g(x) = 3x-5$ 
\end{multicols}

\begin{multicols}{2}
\item  $f(x) = x^2-4$, $g(x) = |x|$
\item  $f(x) = 3x-5$, $g(x) = \sqrt{x}$ 
\end{multicols}

\begin{multicols}{2}
\item $f(x) = |x+1|$, $g(x) = \sqrt{x}$
\item $f(x) = 3-x^2$, $g(x) = \sqrt{x+1}$ 
\end{multicols}

\begin{multicols}{2}
\item  $f(x) = |x|$, $g(x) = \sqrt{4-x}$
\item  \mbox{$f(x) = x^2-x-1$, $g(x) = \sqrt{x-5}$} 
\end{multicols}

\begin{multicols}{2}
\item $f(x) = 3x-1$, $g(x) = \dfrac{1}{x+3}$
\item $f(x) = \dfrac{3x}{x-1}$, $g(x) =\dfrac{x}{x-3}$
\end{multicols}

\begin{multicols}{2}
\item $f(x) = \dfrac{x}{2x+1}$, $g(x) = \dfrac{2x+1}{x}$
\item $f(x) = \dfrac{2x}{x^2-4}$, $g(x) =\sqrt{1-x}$
\end{multicols}
\end{enumerate}

In each of the following exercises, use $f(x) = -2x$, $g(x) = \sqrt{x}$ and $h(x) = |x|$ to find and simplify expressions for the following functions and state the domain of each using interval notation.

\begin{enumerate}[resume]
\begin{multicols}{3}
\item $(h\circ g \circ f)(x)$
\item $(h\circ f \circ g)(x)$
\item $(g\circ f \circ h)(x)$
\end{multicols}

\begin{multicols}{3}
\item $(g\circ h \circ f)(x)$ 
\item $(f\circ h \circ g)(x)$
\item $(f\circ g \circ h)(x)$
\end{multicols}
\end{enumerate}
\newpage
In each of the following exercises,  write the given function as a composition of two or more non-identity functions.  (There are several correct answers, so check your answer using function composition.)

\begin{enumerate}[resume]
\begin{multicols}{2}
\item  $p(x) = (2x+3)^3$
\item  $P(x) = \left(x^2-x+1\right)^5$
\end{multicols}

\begin{multicols}{2}
\item  $h(x) = \sqrt{2x-1}$
\item  $H(x) = |7-3x|$
\end{multicols}

\begin{multicols}{2}
\item  $r(x) = \dfrac{2}{5x+1}$
\item  $R(x) = \dfrac{7}{x^2-1}$
\end{multicols}

\begin{multicols}{2}
\item  $q(x) = \dfrac{|x|+1}{|x|-1}$
\item  $Q(x) = \dfrac{2x^3+1}{x^3-1}$
\end{multicols}

\begin{multicols}{2}
\item  $v(x) = \dfrac{2x+1}{3-4x}$
\item  $w(x) = \dfrac{x^2}{x^4+1}$
\end{multicols}

\item Let $g(x) = -x, \, h(x) = x + 2, \, j(x) = 3x$ and $k(x) = x - 4$.  In what order must these functions be composed with $f(x) = \sqrt{x}$ to create $F(x) = 3\sqrt{-x + 2} - 4$?
\item What linear functions could be used to transform $f(x) = x^{3}$ into\\ $F(x) = -\frac{1}{2}(2x - 7)^{3} + 1$?  What is the proper order of composition?
\end{enumerate}

For each of the following exercises, let $f$ be the function defined by \[f = \{(-3, 4), (-2, 2), (-1, 0), (0, 1), (1, 3), (2, 4), (3, -1)\}\] and let $g$ be the function defined \[g = \{(-3, -2), (-2, 0), (-1, -4), (0, 0), (1, -3), (2, 1), (3, 2)\}.\]  Use $f$ and $g$ to compute each of the indicated values if they exist.

\begin{enumerate}[resume]
\begin{multicols}{3}
\item $(f \circ g)(3)$
\item $f(g(-1))$
\item $(f \circ f)(0)$
\end{multicols}

\begin{multicols}{3}
\item $(f \circ g)(-3)$
\item $(g \circ f)(3)$
\item $g(f(-3))$
\end{multicols}

\begin{multicols}{3}
\item $(g \circ g)(-2)$
\item $(g \circ f)(-2)$
\item $g(f(g(0)))$
\end{multicols}
\end{enumerate}
\newpage
In each of the following exercises, use the graphs of $y=f(x)$ and $y=g(x)$ below to find the function value.

\begin{center}
\begin{multicols}{2}
\begin{tikzpicture}[xscale=0.75,yscale=0.75]
	\draw[xstep=1,ystep=1,gray,very thin,dotted] (0,0) grid (5,5);
	\draw[] (0,4) -- (1,2) -- (2,3) -- (3,3) -- (4,0);
	\draw [->](-0.25,0) -- coordinate (x axis mid) (5,0) node[below right] {$x$};
	\draw [->](0,-0.25) -- coordinate (x axis mid) (0,5) node[above right] {$y$};
	\foreach \x in {1,...,4}
	\draw (\x,1pt) -- (\x,-3pt)
	node[anchor=north] {\scriptsize $\x$};
	\foreach \y in {1,...,4}
	\draw (1pt,\y) -- (-3pt,\y) 
	node[anchor=east] {\scriptsize $\y$};
	\draw (2.5,-1.1) node[below] {\scriptsize $y=f(x)$};  
	\draw[fill] (0,4) circle (0.08);
	\draw[fill] (1,2) circle (0.08);
	\draw[fill] (2,3) circle (0.08);
	\draw[fill] (3,3) circle (0.08);
	\draw[fill] (4,0) circle (0.08);
\end{tikzpicture}
\columnbreak
\begin{tikzpicture}[xscale=0.75,yscale=0.75]
	\draw[xstep=1,ystep=1,gray,very thin,dotted] (0,0) grid (5,5);
	\draw[] (0,0) -- (1,3) -- (2,3) -- (3,0) -- (4,4);
	\draw [->](-0.25,0) -- coordinate (x axis mid) (5,0) node[below right] {$x$};
	\draw [->](0,-0.25) -- coordinate (x axis mid) (0,5) node[above right] {$y$};
	\foreach \x in {1,...,4}
	\draw (\x,1pt) -- (\x,-3pt)
	node[anchor=north] {\scriptsize $\x$};
	\foreach \y in {1,...,4}
	\draw (1pt,\y) -- (-3pt,\y) 
	node[anchor=east] {\scriptsize $\y$};
	\draw (2.5,-1.1) node[below] {\scriptsize $y=g(x)$};  
	\draw[fill] (0,0) circle (0.08);
	\draw[fill] (1,3) circle (0.08);
	\draw[fill] (2,3) circle (0.08);
	\draw[fill] (3,0) circle (0.08);
	\draw[fill] (4,4) circle (0.08);
\end{tikzpicture}
\end{multicols}
\end{center}

\begin{enumerate}[resume]
\begin{multicols}{3}
\item  $(g\circ f)(1)$
\item  $(f \circ g)(3)$
\item  $(g\circ f)(2)$
\end{multicols}

\begin{multicols}{3}
\item  $(f\circ g)(0)$  
\item  $(f\circ f)(1)$
\item  $(g \circ g)(1)$
\end{multicols}
\end{enumerate}

\subsection*{Inverse Functions}
%\subsubsection{Definition and the Horizontal Line Test}
%\subsubsection{Finding Inverses Algebraically}
In each of the following exercises, show that the given function is one-to-one and find its inverse.  Check your answers algebraically and graphically.  Verify that the range of $f$ is the domain of $f^{-1}$ and vice-versa.

\begin{enumerate}
\begin{multicols}{2}
\item $f(x) = 6x - 2$
\item $f(x) = 42-x$
\end{multicols}

\begin{multicols}{2}
\item $f(x) = \dfrac{x-2}{3} + 4$
\item $f(x)  = 1 - \dfrac{4+3x}{5}$
\end{multicols}


\begin{multicols}{2}
\item $f(x) = \sqrt{3x-1}+5$
\item $f(x) = 2-\sqrt{x - 5}$
\end{multicols}

\begin{multicols}{2}
\item $f(x) = 3\sqrt{x-1}-4$
\item $f(x) = 1 - 2\sqrt{2x+5}$
\end{multicols}

\begin{multicols}{2}
\item $f(x) = \sqrt[5]{3x-1}$
\item $f(x) = 3-\sqrt[3]{x-2}$
\end{multicols}

\begin{multicols}{2}
\item $f(x) = x^2 - 10x$, $x \geq 5$
\item $f(x) = 3(x + 4)^{2} - 5, \; x \leq -4$
\end{multicols}

\begin{multicols}{2}
\item $f(x) = x^2-6x+5, \; x \leq 3$
\item $f(x) = 4x^2 + 4x + 1$, $x < -1$
\end{multicols}

\begin{multicols}{2}
\item $f(x) = \dfrac{3}{4-x}$
\item $f(x) = \dfrac{x}{1-3x}$
\end{multicols}

\begin{multicols}{2}
\item $f(x) = \dfrac{2x-1}{3x+4}$
\item $f(x) = \dfrac{4x + 2}{3x - 6}$
\end{multicols}

\begin{multicols}{2}
\item $f(x) = \dfrac{-3x - 2}{x + 3}$ 
\item $f(x) = \dfrac{x-2}{2x-1}$
\end{multicols}
\end{enumerate}

Find the inverses of each of the following functions.

\begin{enumerate}[resume]
\begin{multicols}{2}
\item $f(x) = ax + b, \; a \neq 0$
\item $f(x) = a\sqrt{x - h} + k, \; a \neq 0, x \geq h$
\end{multicols}
\item $f(x) = ax^{2} + bx + c$ where $a \neq 0, \, x \geq -\dfrac{b}{2a}$.
\item $f(x) = \dfrac{ax + b}{cx + d}\;$ where $c$ and $d$ are not both zero.
\end{enumerate}

\subsection*{Transformations}
%\subsubsection{Translations}
%\subsubsection{Reflections}
%\subsubsection{Scalings}
%\subsubsection{Transformations Summary}
Suppose $(2,-3)$ is on the graph of $y = f(x)$.  In each of the following exercises, use the given point to find a point on the graph of the given transformed function.

\begin{enumerate}
\begin{multicols}{3}
\item $g(x) = f(x)+3$
\item $g(x) = f(x+3)$
\item $g(x) = f(x)-1$
\end{multicols}

\begin{multicols}{3}
\item $g(x) = f(x-1)$
\item $g(x) = 3f(x)$
\item $g(x) = f(3x)$
\end{multicols}

\begin{multicols}{3}
\item $g(x) = -f(x)$
\item $g(x) = f(-x)$
\item $g(x) = f(x-3)+1$
\end{multicols}

\begin{multicols}{3}
\item $g(x) = 2f(x+1)$
\item $g(x) = 10 - f(x)$
\item $g(x) = 3f(2x) - 1$
\end{multicols}

\begin{multicols}{3}
\item $g(x) = \frac{1}{2} f(4-x)$
\item \mbox{$g(x) = 5f(2x)+3$}
\item \mbox{$g(x) = 2f(1-x) -1$}
\end{multicols}

\begin{multicols}{2}
%\item $g(x) =f\left(\dfrac{7-2x}{4}\right)$
\item $g(x) = \dfrac{f(3x) - 1}{2}$
\item $g(x) = \dfrac{4-f(3x-1)}{7}$
\end{multicols}
\end{enumerate}

The complete graph of $f(x)=|x|$ is given below.  In each of the following exercises, use it to graph the given transformed function.
\begin{center}
\begin{tikzpicture}[xscale=0.75,yscale=0.75]
	\draw[xstep=1,ystep=1,gray,very thin,dotted] (-3.8,0) grid (3.8,4.5);
	\draw [<->](-4,0) -- coordinate (x axis mid) (4,0) node[below right] {$x$};
	\draw [<->](0,0) -- coordinate (x axis mid) (0,4.5) node[above right] {$y$};
	\draw [->] plot [domain=0:4, samples=100] (\x,{\x});
	\draw [->] plot [domain=0:-4, samples=100] (\x,{-\x});
	\draw[fill] (-1,1) circle (0.08);
	\draw[fill] (1,1) circle (0.08);
	\draw[fill] (0,0) circle (0.08);
	\foreach \x in {-3,...,-1}
	\draw (\x,1pt) -- (\x,-3pt)
	node[anchor=north] {\scriptsize $\x$};
	\foreach \x in {1,...,3}
	\draw (\x,1pt) -- (\x,-3pt)
	node[anchor=north] {\scriptsize $\x$};
	\foreach \y in {1,...,4}
	\draw (1pt,\y) -- (-3pt,\y) 
	node[anchor=east] {\scriptsize $\y$};
	\draw (1,1) node[right] {\scriptsize $(1,1)$}; 
	\draw (-1,1) node[left] {\scriptsize $(-1,1)$}; 
	\draw (0,-0.3) node[below] {\scriptsize $(0,0)$}; 
	\draw (0,-1) node[below] {\scriptsize $f(x)=|x|$};  
\end{tikzpicture}
\end{center}

\begin{enumerate}[resume]
\begin{multicols}{3}
\item $g(x) = f(x) + 1$
\item $g(x) = f(x) - 2$
\item $g(x) = f(x+1)$
\end{multicols}

\begin{multicols}{3}
\item $g(x) = f(x - 2)$
\item $g(x) = 2f(x)$
\item $g(x) = f(2x)$
\end{multicols}

\begin{multicols}{3}
\item $g(x) = 2 - f(x)$
\item $g(x) = f(2-x)$
\item $g(x) = 2-f(2-x)$
\end{multicols}

\item Some of the answers to the previous nine exercises should be equal.  Which ones are equal?  What properties of the graph of $y=f(x)$ contribute to this?
\end{enumerate}
The complete graph of $f(x)=\sqrt{9-x^2}$ is given below.  Use the graph of $f$ to graph the each of the given transformations.
\begin{center}
\begin{tikzpicture}[xscale=.75,yscale=.75]
	\draw[xstep=1,ystep=1,gray,very thin,dotted] (-3.5,0) grid (3.5,3.5);
	\draw [<->](-3.5,0) -- coordinate (x axis mid) (3.5,0) node[below right] {$x$};
	\draw [<->](0,-0.5) -- coordinate (y axis mid) (0,3.5) node[above right] {$y$};
	%\draw [-] plot [domain=-3:3, samples=100] (\x,{1/\x});
	\begin{scope}
				\clip (-3,0) rectangle (3,3);
				\draw (0,0) circle(3);
	\end{scope}	
	\draw[fill] (3,0) circle (0.05);
	\draw[fill] (-3,0) circle (0.05);
	\draw[fill] (0,3) circle (0.05);
	\foreach \x in {1,...,3}
		\draw (\x,2pt) -- (\x,-2pt)	node[anchor=north] {\scriptsize \x};
	\foreach \x in {-3,...,-1}
		\draw (\x,2pt) -- (\x,-2pt)	node[anchor=north] {\scriptsize \x};
	\foreach \y in {1,...,3}
		\draw (2pt,\y) -- (-2pt,\y)	node[anchor=east] {\scriptsize \y}; 
%	\foreach \y in {-4,...,-1}
%		\draw (2pt,\y) -- (-2pt,\y)	node[anchor=west] {\scriptsize \y}; 
	\draw (-3,-1) node[] {\scriptsize $(3,0)$}; 
	\draw (3,-1) node[] {\scriptsize $(-3,0)$}; 
	\draw (0,3) node[above right] {\scriptsize $(0,3)$}; 
	\draw (0,-1.1) node[below] {\scriptsize $f(x)=\sqrt{9-x^2}$};  
\end{tikzpicture}
\end{center}
\begin{enumerate}[resume]
\begin{multicols}{3}
\item $g(x) = f(x) + 3$
\item $h(x) = f(x) - \frac{1}{2}$
\item $j(x) = f\left(x - \frac{2}{3}\right)$
\end{multicols}

\begin{multicols}{3}
\item $a(x) = f(x + 4)$
\item $b(x) = f(x + 1) - 1$ 
\item $c(x) = \frac{3}{5}f(x)$
\end{multicols}

\begin{multicols}{3}
\item $d(x) = -2f(x)$
\item $k(x) = f\left(\frac{2}{3}x\right)$
\item $m(x) = -\frac{1}{4}f(3x)$
\end{multicols}

\begin{multicols}{2}
\item $n(x) = 4f(x - 3) - 6$
\item $p(x) = 4 + f(1 - 2x)$
%\item $q(x) = -\frac{1}{2}f\left(\frac{x + 4}{2}\right) - 3$
\end{multicols}
\newpage
\item The graphs of $y = f(x) = \sqrt[3]{x}$ and $y = g(x)$ are shown below. Find a formula for $g$ based on transformations of the graph of $f$.  Check your answer by confirming that the points shown on the graph of $g$ satisfy the equation $y = g(x)$.
\end{enumerate}
\begin{center}
\begin{tikzpicture}[xscale=0.3,yscale=0.6]
\draw[xstep=3.0,ystep=1,gray,very thin,dotted] (-16,-5.5) grid (16,5.5);
\draw [<->](-16,0) -- coordinate (x axis mid) (16,0) node[below right] {$x$};
\draw [<->](0,-5.5) -- coordinate (y axis mid) (0,5.5) node[above right] {$y$};
\foreach \x in {-15,-12,...,-3}
\draw (\x,3pt) -- (\x,-3pt) node[anchor=north] {\scriptsize \x};
\foreach \x in {3,6,...,15}
\draw (\x,3pt) -- (\x,-3pt) node[anchor=north] {\scriptsize \x};
\foreach \y in {-5,...,-1}
\draw (1pt,\y) -- (-3pt,\y) node[anchor=east] {\scriptsize \y}; 
\foreach \y in {1,...,5}
\draw (1pt,\y) -- (-3pt,\y) node[anchor=east] {\scriptsize \y}; 
\foreach \x in {-15,...,-1}
\draw (\x,2pt) -- (\x,-2pt) node[anchor=north] {};
\foreach \x in {1,...,15}
\draw (\x,2pt) -- (\x,-2pt) node[anchor=north] {};
\draw [<->, domain=-2.15:2.15] plot ({(\x)^3},\x);
\draw (0,-6) node[below] {$f(x)=\sqrt[3]{x}$};  
	\draw[fill] (0,0) ellipse (0.1 and 0.05);
	\draw[fill] (8,2) ellipse (0.1 and 0.05);
	\draw[fill] (1,1) ellipse (0.1 and 0.05);
	\draw[fill] (-1,-1) ellipse (0.1 and 0.05);
	\draw[fill] (-8,-2) ellipse (0.1 and 0.05);
\end{tikzpicture}
\end{center}
\begin{center}
\begin{tikzpicture}[xscale=0.3,yscale=0.6]
\draw[xstep=3.0,ystep=1,gray,very thin,dotted] (-16,-5.5) grid (16,5.5);
\draw [<->](-16,0) -- coordinate (x axis mid) (16,0) node[below right] {$x$};
\draw [<->](0,-5.5) -- coordinate (y axis mid) (0,5.5) node[above right] {$y$};
\foreach \x in {-15,-12,...,-3}
\draw (\x,3pt) -- (\x,-3pt) node[anchor=north] {\scriptsize \x};
\foreach \x in {3,6,...,15}
\draw (\x,3pt) -- (\x,-3pt) node[anchor=north] {\scriptsize \x};
\foreach \y in {-5,...,-1}
\draw (1pt,\y) -- (-3pt,\y) node[anchor=east] {\scriptsize \y}; 
\foreach \y in {1,...,5}
\draw (1pt,\y) -- (-3pt,\y) node[anchor=east] {\scriptsize \y}; 
\foreach \x in {-15,...,-1}
\draw (\x,2pt) -- (\x,-2pt) node[anchor=north] {};
\foreach \x in {1,...,15}
\draw (\x,2pt) -- (\x,-2pt) node[anchor=north] {};
\draw [<->, domain=-5.15:3.15] plot ({-0.125*(\x+1)^3-3},\x);
\draw (0,-6) node[below] {$g(x)$};  
	\draw[fill] (-3,-1) ellipse (0.1 and 0.05);
	\draw[fill] (-4,1) ellipse (0.1 and 0.05);
	\draw[fill] (-2,-3) ellipse (0.1 and 0.05);
	\draw[fill] (5,-5) ellipse (0.1 and 0.05);
	\draw[fill] (-11,3) ellipse (0.1 and 0.05);
\end{tikzpicture}
\end{center}
\newpage
\begin{enumerate}[resume]
\item A function $f$ is said to be {\it even} if $f(x)=f(-x)$.  The graph of an even function will be symmetric about the $y-$axis, since $f(-x)$ represents a reflection of the graph of $f$ about the $y-$axis.  Determine both algebraically (using compositions) and graphically (using transformations) whether each of the following fundamental functions is even.
\begin{enumerate}
\begin{multicols}{2}
	\item $g(x)=x^2$
	\item $k(x)=\sqrt{x}$
\end{multicols}
\begin{multicols}{2}
	\item $\ell(x)=|x|$
	\item $m(x)=x^3$
\end{multicols}
\begin{multicols}{2}
	\item $n(x)=\sqrt[3]{x}$
	\item $p(x)=\dfrac{1}{x}$
	\item $q(x)=\sqrt{9-x^2}$
\end{multicols}
\end{enumerate}
\item A function $f$ is said to be {\it odd} if $-f(x)=f(-x)$.  Since $f(-x)$ represents a reflection of the graph of $f$ about the $y-$axis and $-f(x)$ represents a reflection of the graph of $f$ about the $x-$axis, whenever these two reflections produce the same graph, the corresponding function will be odd.   In this case, the graph of an odd function is said to be \textit{symmetric about the origin}.  Determine both algebraically (using compositions) and graphically (using transformations) whether each of the following fundamental functions is odd.
\begin{enumerate}
\begin{multicols}{2}
	\item $g(x)=x^2$
	\item $k(x)=\sqrt{x}$
\end{multicols}
\begin{multicols}{2}
	\item $\ell(x)=|x|$
	\item $m(x)=x^3$
\end{multicols}
\begin{multicols}{2}
	\item $n(x)=\sqrt[3]{x}$
	\item $p(x)=\dfrac{1}{x}$
	\item $q(x)=\sqrt{9-x^2}$
\end{multicols}
\end{enumerate}
\end{enumerate}
Let $f(x) = \sqrt{x}$.  Find a formula for a function $g$ whose graph is obtained from $f$ from the given sequence of transformations. 
\begin{enumerate}[resume]
\item  (1) shift right 2 units; (2) shift down 3 units
\item  (1) shift down 3 units; (2) shift right 2 units
\item  (1) reflect across the $x$-axis; (2) shift up 1 unit
\item  (1) shift up 1 unit; (2) reflect across the $x$-axis
\item  (1) shift left 1 unit; (2) reflect across the $y$-axis; (3) shift up 2 units
\item  (1) reflect across the $y$-axis;  (2) shift left 1 unit;  (3) shift up 2 units
\item  (1) shift left 3 units; (2) vertical stretch by a factor of 2; (3) shift down 4 units
\item  (1) shift left 3 units; (2) shift down 4 units; (3) vertical stretch by a factor of 2
\item  (1) shift right 3 units; (2) horizontal shrink by a factor of 2; (3) shift up 1 unit
\item  (1) horizontal shrink by a factor of 2; (2) shift right 3 units; (3) shift up 1 unit
\end{enumerate}
\newpage
\subsection*{Piecewise-Defined and Absolute Value Functions}
\subsubsection{Piecewise-Defined Functions}
\begin{enumerate}
\item  Let $f(x) = \left\{  \begin{array}{rcr} x + 5 & \mbox{ if } & x \leq -3 \\ \sqrt{9-x^2} & \mbox{ if } & -3 < x \leq 3 \\ -x+5 & \mbox{ if } & x > 3 \\ \end{array}        \right.$\par
Compute the following function values.

\begin{enumerate}
\begin{multicols}{3}
\item $f(-4)$
\item  $f(-3)$
\item  $f(3)$
\end{multicols}

\begin{multicols}{3}
\item  $f(3.1)$
\item  $f(-3.01)$
\item  $f(2)$
\end{multicols}
\end{enumerate}

\item Let ${\displaystyle f(x) = \left\{ \begin{array}{rcr}
x^{2} & \mbox{ if } & x \leq -1\\
\sqrt{1 - x^{2}} & \mbox{ if } & -1 < x \leq 1\\
x & \mbox{ if } & x > 1  \end{array} \right. }$\par
Compute the following function values.

\begin{enumerate}
\begin{multicols}{3}
\item $f(4)$
\item $f(-3)$
\item $f(1)$
\end{multicols}

\begin{multicols}{3}
\item $f(0)$
\item $f(-1)$
\item $f(-0.99)$
\end{multicols}
\end{enumerate}
\end{enumerate}
In each of the following exercises, find all possible $x$ such that $f(x)=0$.  Then sketch the graph of the given piecewise-defined function.  Use your graph to identify the domain and range of each function.
\begin{enumerate}[resume]
\begin{multicols}{2}
\item ${\displaystyle f(x) = \left\{ \begin{array}{rcl} 4-x & \mbox{ if } &  x \leq 3 \\
                                                            2 & \mbox{ if } & x > 3 
                                     \end{array} \right. }$
\item ${\displaystyle f(x) = \left\{ \begin{array}{rcl} x^2 & \mbox{ if } & x \leq 0 \\
                                                     2x & \mbox{ if } & x > 0
                                  \end{array} \right. }$
\end{multicols}

\begin{multicols}{2}
\item ${\displaystyle f(x) = \left\{ \begin{array}{rcl}  -3 & \mbox{ if } & x < 0 \\
                                                        2x-3 & \mbox{ if } & \scriptsize 0 \leq x \leq 3 \\
                                                            3 & \mbox{ if } & x > 3  
                                     \end{array} \right. }$
\item ${\displaystyle f(x) = \left\{ \begin{array}{rcl} x^2 - 4 & \mbox{ if } &x \leq -2\\
                                                                  4-x^2 & \mbox{ if } & -2 < x < 2 \\
                                                         x^2-4 & \mbox{ if } & x \geq 2 
                                     \end{array} \right. }$
\end{multicols}

\begin{multicols}{2}
\item ${\displaystyle f(x) = \left\{ \begin{array}{rcl} -2x - 4 & \mbox{ if } &  x < 0 \\
                                                             3x & \mbox{ if } & x \geq 0 
                                     \end{array} \right. }$
\item ${\displaystyle f(x) = \left\{ \begin{array}{rcl} x^{2} & \mbox{ if } & x \leq -2 \\
                                                        3 - x & \mbox{ if } & -2 < x < 2 \\
                                                            4 & \mbox{ if } & x \geq 2  
                                     \end{array} \right. }$
\end{multicols}

\item ${\displaystyle f(x) = \left\{ \begin{array}{rcl} \dfrac{1}{x} & \mbox{ if } & -6 < x < -1\\
                                                                  x & \mbox{ if } & -1 < x < 1 \\
                                                           \sqrt{x} & \mbox{ if } & 1 < x < 9  
                                     \end{array} \right. }$
\end{enumerate}
\newpage
\subsubsection{Functions Containing an Absolute Value}
%\subsubsection{Absolute Value as a Piecewise Function}
In each of the following exercises, find the zeros of each function and the $x$- and $y$-intercepts of each graph, if any exist.  Then graph the given absolute value function and express it as a piecewise-defined function.  Use the graph to determine the domain and range of each function.
\begin{enumerate}
\begin{multicols}{3}
\item $f(x) = |x + 4|$
\item $f(x) = |x| + 4$
\item $f(x) = |4x|$
\item $f(x) = -3|x|$ 
\item $f(x) = |2x -5|$ 
\item $f(x) = |-2x+5|$
\item $f(x) = 2|x-\frac{5}{2}|$
\item $f(x) = \dfrac{1}{3}|2x - 1|$
\item $f(x) = 3|x + 4| - 4$ 
\end{multicols}
\end{enumerate}
\end{document}