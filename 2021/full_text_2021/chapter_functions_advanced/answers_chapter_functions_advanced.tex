\documentclass[12pt]{book}
\raggedbottom
\usepackage[top=1in,left=1in,bottom=1in,right=1in,headsep=0.25in]{geometry}	
\usepackage{amssymb,amsmath,amsthm,amsfonts}
\usepackage{chapterfolder,docmute,setspace}
\usepackage{cancel,multicol,tikz,verbatim,framed,polynom,enumitem,tikzpagenodes}
\usepackage[colorlinks, hyperindex, plainpages=false, linkcolor=blue, urlcolor=blue, pdfpagelabels]{hyperref}
\usepackage[type={CC},modifier={by-sa},version={4.0},]{doclicense}

\theoremstyle{definition}
\newtheorem{example}{Example}
\newcommand{\Desmos}{\href{https://www.desmos.com/}{Desmos}}
\setlength{\parindent}{0in}
\setlist{itemsep=0in}
\setlength{\parskip}{0.1in}
\setcounter{secnumdepth}{0}
\input{lesson_order}

\newcommand{\tmmathbf}[1]{\ensuremath{\boldsymbol{#1}}}
\newcommand{\tmop}[1]{\ensuremath{\operatorname{#1}}}

\newlist{oddenumerate}{enumerate}{1}
\setlist[oddenumerate]{start=0,label=\theoddenumeratei.}
\makeatletter
\renewcommand\theoddenumeratei{\@arabic{\numexpr2*\value{oddenumeratei}+1}}
\makeatother

\begin{document}
\section{Selected Answers}
\subsection*{Identifying Domain Algebraically}
\begin{oddenumerate}
\begin{multicols}{2}
\item $(-\infty,\infty)$ 
\item $(-\infty,\frac{2}{3})\cup(\frac{2}{3},\infty)$
\item $(-\infty,\infty)$
\item $(\frac{2}{3},\infty)$
\item $(-\infty,\infty)$
\item $(-\infty,\infty)$
\item $(-\infty,-4)\cup(-4,\infty)$
\item $(-\infty,2)\cup(2,\infty)$
\item $(-\infty,-3)\cup(-3,3)\cup(3,\infty)$
\item $(-\infty,-6)\cup(-6,6)\cup(6,\infty)$
\item $[-\frac{5}{2},\infty)$
\item $[-3,\infty)$
\item $[3,\infty)$
\item $(3,\infty)$
\item $(-\infty,\infty)$
\item $(-\infty,-6)\cup(-6,6)\cup(6,\infty)$
\item $[7,9]$
\item $(-\infty,8)\cup(8,\infty)$
\item $(-\infty,\infty)$
\end{multicols}
\end{oddenumerate}

\subsection*{Combining Functions}
\subsubsection{Function Arithmetic}

\begin{oddenumerate}
\item $f(x)=3x+1,~~~~g(x)=4-x$ 
\begin{multicols}{3}
\begin{itemize}
\item  $(f+g)(2)=9$
\item  $(f-g)(-1)=-7$
\item  $(g-f)(1)=-1$
\end{itemize}
\end{multicols}
\begin{multicols}{3}
\begin{itemize}
\item  $(fg)(\frac{1}{2})=\frac{35}{4}$
\item  $\left(\frac{f}{g}\right)(0)=\frac{1}{4}$
\item  $\left(\frac{g}{f}\right)(-2)=-\frac{6}{5}$
\end{itemize}
\end{multicols}
\newpage
\item $f(x)=x^2-x,~~~~g(x)=12-x^2$ 
\begin{multicols}{3}
\begin{itemize}
\item  $(f+g)(2)=10$
\item  $(f-g)(-1)=-9$
\item  $(g-f)(1)=11$
\end{itemize}
\end{multicols}

\begin{multicols}{3}
\begin{itemize}
\item  $(fg)(\frac{1}{2})=-\frac{47}{16}$
\item  $\left(\frac{f}{g}\right)(0)=0$
\item  $\left(\frac{g}{f}\right)(-2)=\frac{4}{3}$
\end{itemize}
\end{multicols}

\item $f(x)=\sqrt{x+3},~~~~g(x)=2x-1$ 
\begin{multicols}{3}
\begin{itemize}
\item  $(f+g)(2)=3+\sqrt{5}$
\item  $(f-g)(-1)=3+\sqrt{2}$
\item  $(g-f)(1)=-1$
\end{itemize}
\end{multicols}

\begin{multicols}{3}
\begin{itemize}
\item  $(fg)(\frac{1}{2})=0$
\item  $\left(\frac{f}{g}\right)(0)=-\sqrt{3}$
\item  $\left(\frac{g}{f}\right)(-2)=-5$
\end{itemize}
\end{multicols}

\item $f(x)=2x,~~~~g(x)=\dfrac{1}{2x+1}$ 
\begin{multicols}{3}
\begin{itemize}
\item  $(f+g)(2)=\frac{21}{5}$
\item  $(f-g)(-1)=-1$
\item  $(g-f)(1)=-\frac{5}{3}$
\end{itemize}
\end{multicols}

\begin{multicols}{3}
\begin{itemize}
\item  $(fg)(\frac{1}{2})=\frac{1}{2}$
\item  $\left(\frac{f}{g}\right)(0)=0$
\item  $\left(\frac{g}{f}\right)(-2)=\frac{1}{12}$
\end{itemize}
\end{multicols}

\item $f(x)=x^2,~~~~g(x)=\dfrac{1}{x^2}$ 
\begin{multicols}{3}
\begin{itemize}
\item  $(f+g)(2)=\frac{17}{4}$
\item  $(f-g)(-1)=0$
\item  $(g-f)(1)=0$
\end{itemize}
\end{multicols}

\begin{multicols}{3}
\begin{itemize}
\item  $(fg)(\frac{1}{2})=1$
\item  $\left(\frac{f}{g}\right)(0)=$DNE
\item  $\left(\frac{g}{f}\right)(-2)=\frac{1}{16}$
\end{itemize}
\end{multicols}

\item $f(x)=2x+1,~~~~g(x)=x-2$ 
\begin{multicols}{2}
\begin{itemize}
\item  $(f+g)(x)=3x-1,~$all reals
\item  $(f-g)(x)=x+3,~$all reals
\end{itemize}
\end{multicols}
\begin{multicols}{2}
\begin{itemize}
\item  $(fg)(x)=2x^2-3x-2,~$all reals
\item  $\left(\frac{f}{g}\right)(x)=\frac{2x+1}{x-2},~x\neq 2$
\end{itemize}
\end{multicols}

\item $f(x)=x^2,~~~~g(x)=3x-1$ 
\begin{multicols}{2}
\begin{itemize}
\item  $(f+g)(x)=x^2+3x-1,~$all reals
\item  $(f-g)(x)=x^2-3x+1,~$all reals
\end{itemize}
\end{multicols}
\begin{multicols}{2}
\begin{itemize}
\item  $(fg)(x)=3x^3-x^2,~$all reals
\item  $\left(\frac{f}{g}\right)(x)=\frac{x^2}{3x-1},~x\neq\frac{1}{3}$
\end{itemize}
\end{multicols}
\item $f(x)=x^2-4,~~~~g(x)=3x+6$ 
\begin{multicols}{2}
\begin{itemize}
\item  $(f+g)(x)=x^2+3x+2,~$all reals
\item  $(f-g)(x)=x^2-3x-10,~$all reals
\end{itemize}
\end{multicols}
\begin{multicols}{2}
\begin{itemize}
\item  $(fg)(x)=3x^3+6x^2-12x-24,~$\\all reals
\item  $\left(\frac{f}{g}\right)(x)=\frac{x^2-4}{3x+6},~x\neq -2$
\end{itemize}
\end{multicols}
\newpage
\item $f(x)=\dfrac{x}{2},~~~~g(x)=\dfrac{2}{x}$ 
\begin{multicols}{2}
\begin{itemize}
\item  $(f+g)(x)=\frac{x^2+4}{2x},~x\neq 0$
\item  $(f-g)(x)=\frac{x^2-4}{2x},~x\neq 0$
\end{itemize}
\end{multicols}
\begin{multicols}{2}
\begin{itemize}
\item  $(fg)(x)=1,~x\neq 0$
\item  $\left(\frac{f}{g}\right)(x)=\frac{x^2}{4},~x\neq 0$
\end{itemize}
\end{multicols}
\item $f(x)=x,~~~~g(x)=\sqrt{x+1}$ 
\begin{multicols}{2}
\begin{itemize}
\item  $(f+g)(x)=x+\sqrt{x+1},~x\geq-1$
\item  $(f-g)(x)=x-\sqrt{x+1},~x\geq-1$
\end{itemize}
\end{multicols}
\begin{multicols}{2}
\begin{itemize}
\item  $(fg)(x)=x\sqrt{x+1},~x\geq-1$
\item  $\left(\frac{f}{g}\right)(x)=\frac{x}{\sqrt{x+1}},~x>-1$
\end{itemize}
\end{multicols}

\begin{multicols}{2}
\item~~2
\item~~0
\item~~3
\item~~DNE
\item~~4
\item~~-2
\end{multicols}
\end{oddenumerate}

\subsubsection{Composite Functions}

\begin{oddenumerate}
\item $f(x)=x^2,~~~~g(x)=2x+1$
\begin{multicols}{3}
\begin{itemize}
\item  $(g\circ f)(0)=1$
\item  $(f\circ g)(-1)=1$
\item  $(f\circ f)(2)=16$
\end{itemize}
\end{multicols}

\begin{multicols}{3}
\begin{itemize}
\item  $(g\circ f)(-3)=19$
\item  $(f\circ g)(\frac{1}{2})=4$
\item  $(f\circ f)(-2)=16$
\end{itemize}
\end{multicols}

\item $f(x)=4-3x,~~~~g(x)=|x|$
\begin{multicols}{3}
\begin{itemize}
\item  $(g\circ f)(0)=4$
\item  $(f\circ g)(-1)=1$
\item  $(f\circ f)(2)=10$
\end{itemize}
\end{multicols}

\begin{multicols}{3}
\begin{itemize}
\item  $(g\circ f)(-3)=13$
\item  $(f\circ g)(\frac{1}{2})=\frac{5}{2}$
\item  $(f\circ f)(-2)=-26$
\end{itemize}
\end{multicols}

\item $f(x)=4x+5,~~~~g(x)=\sqrt{x}$
\begin{multicols}{3}
\begin{itemize}
\item  $(g\circ f)(0)=\sqrt{5}$
\item  $(f\circ g)(-1)=$DNE
\item  $(f\circ f)(2)=57$
\end{itemize}
\end{multicols}

\begin{multicols}{3}
\begin{itemize}
\item  $(g\circ f)(-3)=$DNE
\item  $(f\circ g)(\frac{1}{2})=4\sqrt{\frac{1}{2}}+5$
\item  $(f\circ f)(-2)=-7$
\end{itemize}
\end{multicols}

\item $f(x)=\dfrac{3}{1-x},~~~~g(x)=\dfrac{4x}{x^2+1}$
\begin{multicols}{3}
\begin{itemize}
\item  $(g\circ f)(0)=\frac{6}{5}$
\item  $(f\circ g)(-1)=1$
\item  $(f\circ f)(2)=\frac{3}{4}$
\end{itemize}
\end{multicols}

\begin{multicols}{3}
\begin{itemize}
\item  $(g\circ f)(-3)=\frac{48}{25}$
\item  $(f\circ g)(\frac{1}{2})=-5$
\item  $(f\circ f)(-2)=$DNE
\end{itemize}
\end{multicols}
\newpage
\item $f(x)=2x+3,~~~~g(x)=x^2-9$
\begin{multicols}{3}
\begin{itemize}
\item  $(g\circ f)(x)=4x^2+12x$
\item  $(f\circ g)(x)=2x^2-15$
\item  $(f\circ f)(x)=4x=9$
\end{itemize}
\end{multicols}

\item $f(x)=x^2-4,~~~~g(x)=|x|$
\begin{multicols}{3}
\begin{itemize}
\item  $(g\circ f)(x)=|x^2-4|$
\item  $(f\circ g)(x)=x^2-4$
\item  \mbox{$(f\circ f)(x)=x^4-8x^2+12$}
\end{itemize}
\end{multicols}

\item $f(x)=|x+1|,~~~~g(x)=\sqrt{x}$
\begin{multicols}{3}
\begin{itemize}
\item  $(g\circ f)(x)=\sqrt{|x+1|}$
\item  $(f\circ g)(x)=|\sqrt{x}+1|$
\item  $(f\circ f)(x)=|x+1|+1$
\end{itemize}
\end{multicols}

\item $f(x)=|x|,~~~~g(x)=\sqrt{4-x}$
\begin{multicols}{3}
\begin{itemize}
\item  $(g\circ f)(x)=\sqrt{4-|x|}$
\item  $(f\circ g)(x)=\sqrt{4-x}$
\item  $(f\circ f)(x)=|x|$
\end{itemize}
\end{multicols}

\item $f(x)=3x-1,~~~~g(x)=\dfrac{1}{x+3}$
\begin{multicols}{3}
\begin{itemize}
\item  $(g\circ f)(x)=\frac{1}{3x+2}$
\item  $(f\circ g)(x)=\frac{x}{x+3}$
\item  $(f\circ f)(x)=9x-4$
\end{itemize}
\end{multicols}

\item $f(x)=\dfrac{x}{2x+1},~~~~g(x)=\dfrac{2x+1}{x}$
\begin{multicols}{3}
\begin{itemize}
\item  $(g\circ f)(x)=\frac{4x+1}{x}$
\item  $(f\circ g)(x)=\frac{2x+1}{5x+2}$
\item  $(f\circ f)(x)=\frac{x}{4x+1}$
\end{itemize}
\end{multicols}

\item $h(g(f(x)))=|\sqrt{-2x}|$
\item $g(f(h(x)))=\sqrt{-2|x|}$
\item $f(h(g(x)))=-2|\sqrt{x}|$

\begin{multicols}{2}
\item $f(x)=x^3,~~~~g(x)=2x+3$
\item $f(x)=\sqrt{x},~~~~g(x)=2x-1$
\item $f(x)=\dfrac{2}{x},~~~~g(x)=5x+1$
\item $f(x)=\dfrac{x+1}{x-1},~~~~g(x)=|x|$
\item $f(x)=\dfrac{x+1}{3-2x},~~~~g(x)=2x$
\item $k\circ j\circ f\circ h\circ g$
\end{multicols}
\begin{multicols}{4}
\item~~4
\item~~3
\item~~-4
\item~~0
\item~~-3
\item~~4
\item~~4
\item~~0
\end{multicols}
\end{oddenumerate}
\newpage
\subsection*{Inverse Functions}

\begin{oddenumerate}
\begin{multicols}{2}
\item $f^{-1}(x) = \dfrac{x + 2}{6}$
\item $f^{-1}(x) = 3x-10$
\item $f^{-1}(x) = \frac{1}{3}(x-5)^2+\frac{1}{3}$, $x \geq 5$
\item $f^{-1}(x) = \frac{1}{9}(x+4)^2+1$, $x \geq -4$
\item $f^{-1}(x) = \frac{1}{3} x^{5} + \frac{1}{3}$
\item $f^{-1}(x) = 5 + \sqrt{x+25}$
\item $f^{-1}(x) = 3 - \sqrt{x+4}$
\item $f^{-1}(x) = \dfrac{4x-3}{x}$
\item $f^{-1}(x) = \dfrac{4x+1}{2-3x}$
\item $f^{-1}(x) = \dfrac{-3x - 2}{x + 3}$
\item $f^{-1}(x)=\dfrac{x-b}{a}$
\item $f^{-1}(x)=\dfrac{-b+\sqrt{b^2-4a(c-x)}}{2a}$
\end{multicols}
\end{oddenumerate}

\subsection*{Transformations}

%Suppose $(2,-3)$ is on the graph of $y = f(x)$.  In Exercises \ref{transformpointfirst} - \ref{transformpointlast}, use the given point to %Theorem \ref{transformationsthm} to find a point on the graph of the given transformed function.

\begin{oddenumerate}
\begin{multicols}{4}
\item $(2,0)$
\item $(2,-4)$
\item $(2,-9)$
\item $(2,3)$
\item $(5,-2)$
\item $(2,13)$
\item $(2,-\frac{3}{2})$
\item $(-1,-7)$
\item $(1,1)$
\end{multicols}
\end{oddenumerate}
Each answer below describes the resulting transformation of the graph of $f(x)=|x|$.
\begin{oddenumerate}[resume]
\item Shift down 2 units
\item Shift right 2 units
\item Vertical stretch (or horizontal shrink) by a factor of 2
\item Shift right 2 units
\item Exercises 22. and 23. agree; 21. and 25. agree. $|kx|=|k|\cdot|x|$, where $k\in\mathbb{R}$
\end{oddenumerate}
Each answer below describes the resulting transformation of the graph of $f(x)=\sqrt{9-x^2}$.
\begin{oddenumerate}[resume]\item Shift down 1/2 units
\item Shift left 4 units
\item Vertical shrink by a factor of 5/3
\item Horizontal stretch by a factor of 3/2
\item Shift right 3 units, vertical stretch by a factor of 4, shift down 6 units
\begin{multicols}{2}
\item $g(x)=-2\sqrt[3]{x+3}-1$
\item (d), (e), and (f)
\item $g(x)=\sqrt{x-2}-3$
\item $g(x)=-\sqrt{x}-1$
\item $g(x)=\sqrt{-x-1}+2$
\item $g(x)=2\sqrt{x+3}-8$
\item $g(x)=\sqrt{2x-6}+1$
\end{multicols}
\end{oddenumerate}
\newpage
\subsection*{Piecewise-Defined and Absolute Value Functions}

\subsubsection{Piecewise-Defined Functions}

%\item  Let $f(x) = \left\{  \begin{array}{rcr} x + 5 & \mbox{ if } & x \leq -3 \\ \sqrt{9-x^2} & \mbox{ if } & -3 < x \leq 3 \\ -x+5 & \mbox{ if } & x > 3 \\ \end{array}        \right.$
%Compute the following function values.

\begin{oddenumerate}
\item $f(x) = \left\{  \begin{array}{rcr} x + 5 & \mbox{ if } & x \leq -3 \\ \sqrt{9-x^2} & \mbox{ if } & -3 < x \leq 3 \\ -x+5 & \mbox{ if } & x > 3 \\ \end{array} \right.$
\normalsize
\begin{multicols}{3}
\begin{enumerate}
\item[(a)] $f(-4)=1$
\item[(b)]  $f(-3)=2$
\item[(c)]  $f(3)=0$
\end{enumerate}
\end{multicols}

\begin{multicols}{3}
\begin{enumerate}
\item[(d)]  $f(3.1)=1.9$
\item[(e)]  $f(-3.01)=1.99$
\item[(f)]  $f(2)=\sqrt{5}$
\end{enumerate}
\end{multicols}

\item D:~$(-\infty,\infty)$; R:~$[1,\infty)$; No zeros
\item D:~$(-\infty,\infty)$; R:~$[-3,3]$; $x=3/2$
\item D:~$(-\infty,\infty)$; R:~$(-4,\infty)$; $x=-2,0$
\item D:~$(-6,-1)\cup(-1,1)\cup(1,9)$; R:~$(-1,1)\cup(1,3)$; $x=0$
\end{oddenumerate}

\subsubsection{Functions Containing an Absolute Value}

%In Exercises \ref{graphabsvalexerfirst} - \ref{graphabsvalexerlast}, find the zeros of each function and the $x$- and $y$-intercepts of each graph, if any exist.  Then graph the given absolute value function and express it as a piecewise-defined function.  Use the graph to determine the domain and range of each function.

\begin{oddenumerate}
\item No zeros;  $y-$int at $(0,4)$; D:~$(-\infty,\infty)$; R:~$[4,\infty)$
$$f(x) = \left\{  \begin{array}{rcr} 
x+4 & \mbox{ if } & x \geq 0 \\
-x+4 & \mbox{ if } & x < 0 \\ \end{array}\right.$$
\item Zero at $x=0$; $y-$int at $(0,0)$; D:~$(-\infty,\infty)$; R:~$[0,\infty)$
$$f(x) = \left\{  \begin{array}{rcr} 
 4x & \mbox{ if } & x \geq 0 \\
 -4x & \mbox{ if } & x < 0 \\ \end{array}\right.$$
\item Zero at $x=\frac{5}{2}$;  $y-$int at $(0,5)$;  D:~$(-\infty,\infty)$; R:~$[0,\infty)$
$$f(x) = \left\{  \begin{array}{rcr} 
 2x-5 & \mbox{ if } & x \geq \frac{5}{2} \\
 -2x+5 & \mbox{ if } & x < \frac{5}{2} \\ \end{array}\right.$$
\item Zero at $x=\frac{5}{2}$;  $y-$int at $(0,5)$;  D:~$(-\infty,\infty)$; R:~$[0,\infty)$
$$f(x) = \left\{  \begin{array}{rcr} 
 2x-5 & \mbox{ if } & x \geq \frac{5}{2} \\
 -2x+5 & \mbox{ if } & x < \frac{5}{2} \\ \end{array}\right.$$
\item Zeros at $x=-\frac{16}{3},~-\frac{8}{3}$; $y-$int at $(0,8)$; D:~$(-\infty,\infty)$; R:~$[-4,\infty)$
$$f(x) = \left\{  \begin{array}{rcr} 
3x+8 & \mbox{ if } & x \geq -4 \\
-3x-16 & \mbox{ if } & x < -4 \\ \end{array}\right.$$
\end{oddenumerate}
\end{document}