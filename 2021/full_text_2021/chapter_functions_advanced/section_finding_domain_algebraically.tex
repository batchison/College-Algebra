\documentclass[12pt]{book}
\raggedbottom
\usepackage[top=1in,left=1in,bottom=1in,right=1in,headsep=0.25in]{geometry}	
\usepackage{amssymb,amsmath,amsthm,amsfonts}
\usepackage{chapterfolder,docmute,setspace}
\usepackage{cancel,multicol,tikz,verbatim,framed,polynom,enumitem,tikzpagenodes}
\usepackage[colorlinks, hyperindex, plainpages=false, linkcolor=blue, urlcolor=blue, pdfpagelabels]{hyperref}
\usepackage[type={CC},modifier={by-sa},version={4.0},]{doclicense}

\theoremstyle{definition}
\newtheorem{example}{Example}
\newcommand{\Desmos}{\href{https://www.desmos.com/}{Desmos}}
\setlength{\parindent}{0in}
\setlist{itemsep=0in}
\setlength{\parskip}{0.1in}
\setcounter{secnumdepth}{0}
\input{lesson_order}

\newcommand{\tmmathbf}[1]{\ensuremath{\boldsymbol{#1}}}
\newcommand{\tmop}[1]{\ensuremath{\operatorname{#1}}}

\begin{document}
\section{Identifying Domain Algebraically (L\arabic{lesson_finding_domain_algebraically})}
\begin{tikzpicture}[remember picture, overlay,shift=(current page text area.north east),scale=0.5]
\draw[step=1.0,gray,very thin,dotted] (-9.8,-7.8) grid (-0.2,1.8);		
\draw[very thick] (-10,-8) -- (-10,2) -- (0,2) -- (0,-8) -- (-10,-8);
\draw[] (-9.8,-7.8) -- (-9.8,1.8) -- (-0.2,1.8) -- (-0.2,-7.8) -- (-9.8,-7.8);
\draw[-] (-9.8,-3) -- coordinate (x axis mid) (-0.2,-3);
\draw[-] (-5,-7.8) -- coordinate (y axis mid) (-5,1.8);
\draw[->] plot [domain=-3.5:-1, samples=100] (\x,{-2*(\x-3)-12});
\draw[<-] plot [domain=-7.5:-3.5, samples=100] (\x,{2*(\x)+8});
%%\draw[<->] plot [domain=-9.1:-1.1, samples=100] (\x,{0.125*(\x+2)*(\x+7)^2-3});
%%%\draw[-,dashed] plot [domain=-9.8:-0.2, samples=100] (\x,{-4});
%%%\draw[-,dashed] plot [domain=-7.8:1.8, samples=100] ({-3},\x);
%%%\draw[<->] plot [domain=-9:-3.3, samples=100] (\x,{(1/(\x+3))-4});
%%%\draw[<->] plot [domain=-2.8:-0.5, samples=100] (\x,{(1/(\x+3))-4});
\end{tikzpicture}%
{\bf Objective: Identify the domain of a function that is described algebraically.}\par
When trying to identify the domain of a function that has been described algebraically or whose graph is not known, we will often need to consider what is {\it not} permissible for the function, then exclude any values of $x$ that will make the function undefined from the interval $(-\infty,\infty)$.  What is left will be our domain.  With virtually every algebraic function, this amounts to avoiding the following situations.
\begin{itemize}
	\item Negatives under an even radical $\left(\sqrt{~}~,~~\sqrt[4]{~}~,~~\sqrt[6]{~}~,~\ldots\right)$
	\item Zero in a denominator
\end{itemize}
In the previous chapters, we dealt exclusively with linear equations.  While equations of the form $y=mx+b$ represent $y$ as a function of $x$, they are also included in a much larger family of functions known as {\it polynomials}.  Polynomials are functions of the form $$f(x)=a_nx^n+a_{n-1}x^{n-1}+\ldots+a_2x+a_1x+a_0,$$ where each of the coefficients $a_i$ represent real numbers (with $a_n\neq 0$) and $n$ represents a nonnegative integer.  These functions include quadratics, which are of the form $y=ax^2+bx+c$.  Since polynomials contain no radicals or variables in a denominator, we can immediately conclude that their domain will always be all real numbers, or $(-\infty,\infty)$.  We reiterate this with our first example.
\begin{example}~~~Find the domain of $f(x)=\frac{1}{3} x^2-x$.
  \begin{eqnarray*}
    f (x) = \frac{1}{3} x^2 - x & & \text{No~radicals~or~variables~in~a~denominator}\\
		& & \text{No~values~of~} x\text{~need~to~be~excluded}\\
		\tmop{All~real~numbers~or~}(-\infty,\infty) & & \tmop{Our} \tmop{solution}
 \end{eqnarray*}
 \end{example}
Our next example will be of a {\it rational~function}, which is defined as a ratio of two polynomial functions.  We will explore rational functions and their graphs in a later chapter.  Since rational functions usually include expressions in a denominator, their domains will often require us to exclude one or more values of $x$.
\begin{example}~~~Find the domain of the function $f (x) = \dfrac{3 x - 1}{x^2 + x - 6}$.
  \begin{eqnarray*}
    f (x) = \frac{3 x - 1}{x^2 + x - 6}
    &  & \tmop{Cannot~have~zero~in~a~denominator}\\
		& & \\
    x^2 + x - 6 \neq 0 &  & \tmop{Solve} \tmop{by} \tmop{factoring}\\
    (x + 3) (x - 2) \neq 0 &  & \tmop{Set} \tmop{each} \tmop{factor}
    \tmop{not} \tmop{equal} \tmop{to} \tmop{zero}\\
    x + 3 \neq 0 \tmop{~and~} x - 2 \neq 0 &  & \tmop{Solve} \tmop{each}
    \tmop{inequality}\\
%    \underline{- 3~~~ - 3} ~~~~~~~ \underline{+ 2~~~ + 2} &  & \\
    x \neq - 3, 2 &  & \tmop{Our~solution~as~an~inequality}\\
		(-\infty,-3)\cup(-3,2)\cup(2,\infty) & & \text{Our~solution~using~interval~notation}
  \end{eqnarray*}
\end{example}
The notation in the previous example tells us that $x$ can be any value except for $- 3$ and $2$. If $x$ were to equal one of those two values, our expression in the denominator would reduce to zero and the function
would consequently be undefined.  Furthermore, although one can easily see that $x=\frac{1}{3}$ will make the numerator equal zero, since $x=\frac{1}{3}$ does not coincide with the two values obtained above (either -3 or 2), we should not exclude it from our domain.\par
This example further illustrates that whenever we are finding the domain of a rational function, we need not be concerned at all with the numerator, and instead must restrict our domain to exclude any value for $x$ that would make the {\it denominator} equal to zero.\par
For our final two examples, we will introduce a square root in our function, first in the numerator and later in the denominator.
\begin{example}~~~Find the domain of $f (x) = \sqrt[]{-2 x + 3}$.
  \begin{eqnarray*}
    f (x) = \sqrt[]{-2 x + 3} &  & \tmop{Even~radical;~cannot~have~negative~underneath}\\
    %& & \\
		-2 x + 3 \geq 0 &  & \tmop{Set~greater~than~or~equal~to~zero~and~solve}\\
    -2 x \geq -3 &  & \tmop{Remember~to~switch~direction~of~inequality}\\
		& &\\
    x \leq \frac{3}{2} \text{~~or~} \left(-\infty,\frac{3}{2}\right]&  & \tmop{Our~solution~as~an~inequality~or~an~interval}
		 %\left(-\infty,\frac{3}{2}\right] & &\tmop{Our~solution~as~an~interval}
  \end{eqnarray*}
\end{example}
The notation in the above example states that our variable can be $\frac{3}{2}$ or any real number less than $\frac{3}{2}$. But any number greater
than $\frac{3}{2}$ would make the function undefined.
\begin{example}~~~Find the domain of $m (x) = \dfrac{-x}{\sqrt[]{7x-3}}$.\par
  The even radical tells us that we cannot have a negative value underneath.  But also, the denominator cannot equal zero.  This results in two inequalities.
%	\begin{eqnarray*}
%    m (x) = \frac{-1}{\sqrt[]{7x-3}} &  & \\
		$$7x - 3 \geq  0 \text{~~AND~~} 7x-3 \neq 0$$
Solving for $x$, we get the following.
$$	x \geq \frac{3}{7} \text{~~AND~~} x \neq \frac{3}{7}$$
Our final solution is $x > \dfrac{3}{7}$, or $\left(\dfrac{3}{7},\infty\right)$ as an interval.  This represents the intersection of both inequalities above.
\end{example}
The previous two examples can be generalized as follows.
\begin{itemize}
	\item In instances where the given function is a square root (or even radical), to find the domain we may set up and solve an inequality in which the entire expression underneath is set $\geq 0$.
	\item In instances where the numerator of a given function is a polynomial and the denominator is a square root (or even radical), to find the domain we may set up and solve an inequality in which the expression underneath is set $> 0$ (strictly positive).
\end{itemize}
Since these two cases certainly do not handle every possible function than we may encounter, one should always be cautious when attempting to find the domain of any function.
\end{document}