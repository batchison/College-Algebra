\documentclass[12pt]{book}
\raggedbottom
\usepackage[top=1in,left=1in,bottom=1in,right=1in,headsep=0.25in]{geometry}	
\usepackage{amssymb,amsmath,amsthm,amsfonts}
\usepackage{chapterfolder,docmute,setspace}
\usepackage{cancel,multicol,tikz,verbatim,framed,polynom,enumitem,tikzpagenodes}
\usepackage[colorlinks, hyperindex, plainpages=false, linkcolor=blue, urlcolor=blue, pdfpagelabels]{hyperref}
\usepackage[type={CC},modifier={by-sa},version={4.0},]{doclicense}

\theoremstyle{definition}
\newtheorem{example}{Example}
\newcommand{\Desmos}{\href{https://www.desmos.com/}{Desmos}}
\setlength{\parindent}{0in}
\setlist{itemsep=0in}
\setlength{\parskip}{0.1in}
\setcounter{secnumdepth}{0}
\input{lesson_order}

\newcommand{\tmmathbf}[1]{\ensuremath{\boldsymbol{#1}}}
\newcommand{\tmop}[1]{\ensuremath{\operatorname{#1}}}

\begin{document}
\section{Introduction and Terminology (L\arabic{lesson_rationals_introduction_and_terminology})}
\begin{tikzpicture}[remember picture, overlay,shift=(current page text area.north east),scale=0.5]
\draw[step=1.0,gray,very thin,dotted] (-9.8,-7.8) grid (-0.2,1.8);		
\draw[very thick] (-10,-8) -- (-10,2) -- (0,2) -- (0,-8) -- (-10,-8);
\draw[] (-9.8,-7.8) -- (-9.8,1.8) -- (-0.2,1.8) -- (-0.2,-7.8) -- (-9.8,-7.8);
\draw[-] (-9.8,-3) -- coordinate (x axis mid) (-0.2,-3);
\draw[-] (-5,-7.8) -- coordinate (y axis mid) (-5,1.8);
\draw[-,dashed] plot [domain=-9.8:-0.2, samples=100] (\x,{-4});
\draw[-,dashed] plot [domain=-7.8:1.8, samples=100] ({-3},\x);
\draw[<->] plot [domain=-9:-3.3, samples=100] (\x,{(1/(\x+3))-4});
\draw[<->] plot [domain=-2.8:-0.5, samples=100] (\x,{(1/(\x+3))-4});
\end{tikzpicture}%
{\bf Objective: Define and identify key features of rational functions}\par
A {\it rational function} is a function that can be represented as a ratio (or fraction) of two polynomials $p$ and $q$.  The general form of a rational function $f$ is $$f(x)=\frac{p(x)}{q(x)}=\frac{a_{n}x^{n}+a_{n-1}x^{n-1}+\ldots+a_{1}x+a_{0}}{b_{m}x^{m}+b_{m-1}x^{m-1}+\ldots+b_1x+b_{0}},$$
where each of the $a_i$ and $b_j$ are real numbers (for $i,j=0,1,2,\ldots$), with $a_n$ and $b_m\neq 0$, and both $m$ and $n$ are nonnegative integers.\par
We have already encountered at least one example of a rational function, namely $f(x)=\dfrac{1}{x}$, whose graph we should also be familiar with.
\begin{example}
\begin{multicols}{2}
$f(x)=\dfrac{1}{x}$
\par
Domain: $x\neq 0$ or $(-\infty,0)\cup(0,\infty)$
\par
Range: $y\neq 0$ or $(-\infty,0)\cup(0,\infty)$
\par
\columnbreak
\begin{tikzpicture}[xscale=0.75,yscale=0.75]
	\draw [<->](-4.25,0) -- coordinate (x axis mid) (4.25,0) node[below right] {$x$};
	\draw [<->](0,-4.25) -- coordinate (y axis mid) (0,4.25) node[above right] {$y$};
	\draw [<->] plot [domain=0.25:4, samples=100] (\x,{1/\x});
	\draw [<->] plot [domain=-4:-0.25, samples=100] (\x,{1/\x});
	\draw[fill] (1,1) circle (0.05);
  \draw[fill] (2,0.5) circle (0.05);
	\draw[fill] (3,0.333) circle (0.05);
	\draw[fill] (0.5,2) circle (0.05);
	\draw[fill] (0.333,3) circle (0.05);
	\draw[fill] (-1,-1) circle (0.05);
  \draw[fill] (-2,-0.5) circle (0.05);
	\draw[fill] (-3,-0.333) circle (0.05);
	\draw[fill] (-0.5,-2) circle (0.05);
	\draw[fill] (-0.333,-3) circle (0.05);
	\foreach \x in {1,...,4}
		\draw (\x,2pt) -- (\x,-2pt)	node[anchor=north] {\scriptsize \x};
	\foreach \x in {-4,...,-1}
		\draw (\x,2pt) -- (\x,-2pt)	node[anchor=south] {\scriptsize \x};
	\foreach \y in {1,...,4}
		\draw (2pt,\y) -- (-2pt,\y)	node[anchor=east] {\scriptsize \y}; 
	\foreach \y in {-4,...,-1}
		\draw (2pt,\y) -- (-2pt,\y)	node[anchor=west] {\scriptsize \y}; 
	\draw (2,2) node {$f(x)=\dfrac{1}{x}$};
\end{tikzpicture}
\end{multicols}
\end{example}
As with any function, we can easily identify the $y-$intercept of the graph of $f$ by evaluating the function at $x=0$.
$$f(0)=\frac{p(0)}{q(0)}=\frac{\cancel{a_n\cdot0^n}+\cancel{a_{n-1}\cdot0^{n-1}}+\ldots+\cancel{a_1\cdot0}+a_0}{\cancel{b_m\cdot0^m}+\cancel{b_{m-1}\cdot0^{m-1}}+\ldots+\cancel{b_1\cdot0}+b_0}=\frac{a_0}{b_0}$$
Hence, the graph of $f$ will have a $y-$intercept at the point $\left(0,\dfrac{a_0}{b_0}\right)$.\par
For rational functions that are already factored, finding the $y-$intercept just requires some simplification after substituting zero for $x$.  The following example demonstrates this.
\begin{example}
\begin{multicols}{2}
$f(x)=\dfrac{x^3-x^2-8x+12}{-2x^2+14x-12}$
\par
$g(x)=\dfrac{(x-2)^2(x+3)}{2(x-6)(1-x)}$
\end{multicols}
The $y-$intercept of the graph of $f$ is $(0,\frac{12}{-12})=(0,-1)$.
\par
To find the $y-$intercept of the graph of $g$, we evaluate $g(0)$ below.
$$g(0)=\dfrac{(0-2)^2(0+3)}{2(0-6)(1-0)}=\dfrac{(-2)^2(3)}{2(-6)(1)}=\dfrac{12}{-12}=-1$$
Hence, as was the case with $f$, the $y-$intercept of the graph of $g$ is $(0,-1)$.  In fact, it is left as an exercise to the reader to show that $f$ and $g$ are the same function.
\end{example}
To identify the domain of a rational function $f(x)=\frac{p(x)}{q(x)}$, we must eliminate all real numbers $x$ which make the denominator equal to zero.  In other words, the domain of $f$ is the set of all $x$ such that $q(x)\neq 0$.  Identifying the domain of a rational function that is given in factored form is relatively straightforward, whereas rational functions that are given in expanded form must first be factored.  Again, we provide an example for each case.
\begin{example}\label{horiz_asym_0}
\begin{multicols}{2}
$f(x)=\dfrac{3(x+4)(x-2)^2}{(x+3)^2(2x-3)}$
\par
$g(x)=\dfrac{-x^2-4x+45}{2x^3-5x^2-18x+45}$
\end{multicols}
The domain of $f$ is $x\neq -3,\frac{3}{2}$, or $(-\infty,-3)\cup(-3,\frac{3}{2})\cup(\frac{3}{2},\infty)$.
\par
To find the domain of $g$, one must first use grouping and the $ac-$method to obtain the following factorization of $g$.
\begin{eqnarray*}
 g(x) & = & \dfrac{-x^2-4x+45}{2x^3-5x^2-18x+45}\\
      & = & \dfrac{-(x+9)(x-5)}{(x^2-9)(2x-5)}\\
      & = & \dfrac{-(x+9)(x-5)}{(x-3)(x+3)(2x-5)}
\end{eqnarray*}
We now can identify three zeros for the denominator of $g$ that must be excluded from our domain.
\par
The domain of $g$ is $x\neq 3,-3,\frac{5}{2},$ or $(-\infty,-3)\cup(-3,\frac{5}{2})\cup(\frac{5}{2},3)\cup(3,\infty).$
\end{example}
Notice that in our previous example, the interval notation for each domain always contains one more interval than the number of values that are excluded from the domain.  This will always be the case, since we can think of our excluded values as partitioning dividers of the $x-$axis, $(-\infty,\infty)$.  In other words, three dividers partition the real number line into four intervals.  The multiplicity of each excluded zero of the denominator $q(x)$ will also play a role in the nature of the graph of $f$, as we will see in a later section.
\par
To find all possible $x-$intercepts for the graph of $f(x)=\frac{p(x)}{q(x)}$, we set the function equal to zero and solve for all possible $x$, keeping \textit{only} those values that are also in our domain.  Since $f(x)$ can only equal zero if its numerator is zero, this amounts to finding all roots of the polynomial $p$.
\begin{eqnarray*}
 f(x) & = & 0\\
\dfrac{p(x)}{q(x)} & = & 0\\
\cancel{q(x)}\cdot\dfrac{p(x)}{\cancel{q(x)}} & = & 0\cdot q(x)
\end{eqnarray*}
$$p(x) = 0, ~ q(x)\neq 0$$
Furthermore, referring to the results from the chapter on polynomials, we can again use the multiplicity of each zero of $p$ to determine whether the corresponding $x-$intercept will represent a crossover or turnaround point, as in the following example.
\begin{example}
$f(x)=\dfrac{(x+3)^2(x-1)(x-4)}{(x-1)^2(x^2+2)}$
\par
The numerator has three zeros ($x=-3,1,$ and $4$), but the corresponding graph only has two $x-$intercepts, since $x=1$ is also a zero of the denominator, and therefore not in the domain of $f$.  The $x-$intercept at $(-3,0)$ is a turnaround point, since the multiplicity of the zero $x=-3$ is even.  The $x-$intercept at $(4,0)$ is a crossover point, since the multiplicity of the zero $x=4$ is odd.
\end{example}
The graph of the reciprocal function $f(x)=\frac{1}{x}$ in our first example of this chapter also has two interesting characteristics, known as {\it asymptotes}.  Asymptotes do not appear in the graph of a polynomial, but often show up when analyzing rational functions and more advanced functions such as exponentials and logarithms.  An asymptote usually appears in the form of a line (horizontal, vertical, or slanted) that the graph of a function $f$ approaches.  The graph of $f(x)=\frac{1}{x}$ has a horizontal asymptote at $y=0$, since the end behavior of the graph (as $x\rightarrow \pm\infty$) approaches zero, as well as a vertical asymptote at $x=0$, since the local behavior of the graph near $x=0$ (as $x\rightarrow 0$) tends towards $\pm\infty$.
\par
Later on, we will outline the procedures for finding both horizontal and vertical asymptotes for a rational function, as well as the case where the graph of $f$ has a slant (or oblique) asymptote.  We will also see an example of the special case of a curvilinear asymptote, in which the graph of $f$ approaches a nonlinear curve, as $x$ approaches $\pm\infty$.  Although all polynomials are, by definition, rational functions (with denominator $q(x)=1$), the concept of an asymptote (horizontal, vertical, or otherwise) demonstrates a critical aspect that separates most rational functions from polynomials.
\pagebreak
We close this section with a few more examples of rational functions and their graphs.  While each function shares a common domain ($x\neq 5$), the corresponding graphs exhibit some clear differences.  Specifically, close attention should be paid to the existence, location, and nature of the graph of each function near the $y-$int and $x-$int(s), as well as the horizontal and vertical asymptotes.
\begin{example}\label{horiz_asym_1}
\ \\
\begin{multicols}{2}
$f(x)=\dfrac{-2x+4}{x-5}=\dfrac{-2(x-2)}{x-5}$\par
\ \\
$y-$intercept at $(0,-\frac{4}{5})$
\par
Domain: $x\neq 5$ or $(-\infty,5)\cup(5,\infty)$
\par
$x-$intercept at $(2,0)$
\par
Horizontal asymptote at $y=-2$
\par
Vertical asymptote at $x=5$

\columnbreak

\begin{tikzpicture}[xscale=0.3,yscale=0.3]
	\draw [<->](-11.5,0) -- coordinate (x axis mid) (11.5,0) node[below right] {$x$};
	\draw [<->](0,-11.5) -- coordinate (y axis mid) (0,11.5) node[above right] {$y$};
	\draw [<->] plot [domain=-10.5:4.5, samples=100] (\x,{(-2*\x+4)/(\x-5)});
	\draw [<->] plot [domain=5.75:11, samples=100] (\x,{(-2*\x+4)/(\x-5)});
	\draw [<->,dashed](5,11) -- (5,-11) node[below] {\scriptsize $x=5$};
	\draw [<->,dashed](11,-2) -- (-11,-2) node[below left] {\scriptsize $y=-2$};
	\foreach \x in {2,4,...,10}
		\draw (\x,2pt) -- (\x,-2pt)	node[anchor=north] {\scriptsize \x};
	\foreach \x in {-10,-8,...,-2}
		\draw (\x,2pt) -- (\x,-2pt)	node[anchor=south] {\scriptsize \x};
	\foreach \y in {2,4,...,10}
		\draw (2pt,\y) -- (-2pt,\y)	node[anchor=east] {\scriptsize \y}; 
	\foreach \y in {-10,-8,...,-2}
		\draw (2pt,\y) -- (-2pt,\y)	node[anchor=west] {\scriptsize \y}; 
\end{tikzpicture}
\end{multicols}
\end{example}
\begin{example}\label{horiz_asym_2}
\ \\
\begin{multicols}{2}
$g(x)=\dfrac{x^2-4x+4}{x-5}=\dfrac{(x-2)^2}{x-5}$\par
\ \\
$y-$intercept at $(0,-\frac{4}{5})$
\par
Domain: $x\neq 5$ or $(-\infty,5)\cup(5,\infty)$
\par
$x-$intercept at $(2,0)$
\par
Slant (oblique) asymptote at $y=x+1$
\par
Vertical asymptote at $x=5$

\columnbreak

\begin{tikzpicture}[xscale=0.3,yscale=0.15]
	\draw [<->](-11.5,0) -- coordinate (x axis mid) (11.5,0) node[below right] {$x$};
	\draw [<->](0,-21.5) -- coordinate (y axis mid) (0,21.5) node[above right] {$y$};
	\draw [<->] plot [domain=-10.5:4.5, samples=100] (\x,{(\x-2)^2/(\x-5)});
	\draw [<->] plot [domain=5.7:10.5, samples=100] (\x,{(\x-2)^2/(\x-5)});
	\draw [<->,dashed](5,20) -- (5,-20) node[below] {\scriptsize $x=5$};
	\draw [<->,dashed] plot [domain=-11:11] (\x,{\x+1});
	\foreach \x in {2,4,...,10}
		\draw (\x,4pt) -- (\x,-4pt)	node[anchor=north] {\scriptsize \x};
	\foreach \x in {-10,-8,...,-2}
		\draw (\x,4pt) -- (\x,-4pt)	node[anchor=south] {\scriptsize \x};
	\foreach \y in {4,8,...,20}
		\draw (4pt,\y) -- (-4pt,\y)	node[anchor=east] {\scriptsize \y}; 
	\foreach \y in {-20,-16,...,-4}
		\draw (4pt,\y) -- (-4pt,\y)	node[anchor=west] {\scriptsize \y}; 
	\draw (-10,-5) node {\scriptsize $y=x+1$};
\end{tikzpicture}
\end{multicols}
\newpage
\end{example}
\begin{example}\label{horiz_asym_3}
\ \\
\begin{multicols}{2}
$h(x)=\dfrac{x^2+25}{x^2-10x+25}=\dfrac{x^2+25}{(x-5)^2}$\par
\ \\
$y-$intercept at $(0,1)$
\par
Domain: $x\neq 5$ or $(-\infty,5)\cup(5,\infty)$
\par
No $x-$intercepts
\par
Horizontal asymptote at $y=1$
\par
Vertical asymptote at $x=5$

\columnbreak

\begin{tikzpicture}[xscale=0.2,yscale=0.4]
	\draw [<->](-19.5,0) -- coordinate (x axis mid) (19.5,0) node[below right] {$x$};
	\draw [<->](0,-3.5) -- coordinate (y axis mid) (0,16.5) node[above right] {$y$};
	\draw [<->] plot [domain=-19:3.45, samples=100] (\x,{1+((10*\x)/(\x-5)^2)});
	\draw [<->] plot [domain=7.25:19, samples=100] (\x,{(\x^2+25)/(\x-5)^2});
	\draw [<->,dashed](5,15) -- (5,-3) node[below] {\scriptsize $x=5$};
	\draw [<->,dashed](-19,1) -- (19,1) node[right] {\scriptsize $y=1$};
	\foreach \x in {4,8,...,16}
		\draw (\x,4pt) -- (\x,-4pt)	node[anchor=north] {\scriptsize \x};
	\foreach \x in {-16,-12,...,-4}
		\draw (\x,4pt) -- (\x,-4pt)	node[anchor=north] {\scriptsize \x};
	\foreach \y in {2,4,...,16}
		\draw (4pt,\y) -- (-4pt,\y)	node[anchor=east] {\scriptsize \y}; 
	\foreach \y in {-2}
		\draw (4pt,\y) -- (-4pt,\y)	node[anchor=west] {\scriptsize \y}; 
\end{tikzpicture}
\end{multicols}
\end{example}
\begin{example}\label{horiz_asym_4}
\ \\
\begin{multicols}{2}
$k(x)=\dfrac{x^3-5x^2}{10x-50}=\dfrac{x^2(x-5)}{10(x-5)}$\par
\ \\
$y-$intercept at $(0,0)$
\par
Domain: $x\neq 5$ or $(-\infty,5)\cup(5,\infty)$
\par
$x-$intercept at $(0,0)$
\par
No horizontal or vertical asymptotes
\par
Hole at $(5,\frac{5}{2})$

\columnbreak

\begin{tikzpicture}[xscale=0.5,yscale=0.5]
	\draw [<->](-7.5,0) -- coordinate (x axis mid) (7.5,0) node[below right] {$x$};
	\draw [<->](0,-1.5) -- coordinate (y axis mid) (0,6.5) node[above right] {$y$};
	\draw [<-] plot [domain=-7:4.9, samples=100] (\x,{(\x)^2/10});
	\draw [->] plot [domain=5.1:7, samples=100] (\x,{(\x)^2/10});
	\draw (5,2.5) circle (0.16);
	\foreach \x in {1,2,...,7}
		\draw (\x,2pt) -- (\x,-2pt)	node[anchor=north] {\scriptsize \x};
	\foreach \x in {-7,-6,...,-1}
		\draw (\x,2pt) -- (\x,-2pt)	node[anchor=north] {\scriptsize \x};
	\foreach \y in {1,2,...,6}
		\draw (2pt,\y) -- (-2pt,\y)	node[anchor=east] {\scriptsize \y}; 
	\foreach \y in {-1}
		\draw (2pt,\y) -- (-2pt,\y)	node[anchor=west] {\scriptsize \y}; 
\end{tikzpicture}
\end{multicols}
\end{example}
\end{document}