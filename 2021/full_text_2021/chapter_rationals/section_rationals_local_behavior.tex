\documentclass[12pt]{book}
\raggedbottom
\usepackage[top=1in,left=1in,bottom=1in,right=1in,headsep=0.25in]{geometry}	
\usepackage{amssymb,amsmath,amsthm,amsfonts}
\usepackage{chapterfolder,docmute,setspace}
\usepackage{cancel,multicol,tikz,verbatim,framed,polynom,enumitem,tikzpagenodes}
\usepackage[colorlinks, hyperindex, plainpages=false, linkcolor=blue, urlcolor=blue, pdfpagelabels]{hyperref}
\usepackage[type={CC},modifier={by-sa},version={4.0},]{doclicense}

\theoremstyle{definition}
\newtheorem{example}{Example}
\newcommand{\Desmos}{\href{https://www.desmos.com/}{Desmos}}
\setlength{\parindent}{0in}
\setlist{itemsep=0in}
\setlength{\parskip}{0.1in}
\setcounter{secnumdepth}{0}
% This document is used for ordering of lessons.  If an instructor wishes to change the ordering of assessments, the following steps must be taken:

% 1) Reassign the appropriate numbers for each lesson in the \setcounter commands included in this file.
% 2) Rearrange the \include commands in the master file (the file with 'Course Pack' in the name) to accurately reflect the changes.  
% 3) Rarrange the \items in the measureable_outcomes file to accurately reflect the changes.  Be mindful of page breaks when moving items.
% 4) Re-build all affected files (master file, measureable_outcomes file, and any lessons whose numbering has changed).

%Note: The placement of each \newcounter and \setcounter command reflects the original/default ordering of topics (linears, systems, quadratics, functions, polynomials, rationals).

\newcounter{lesson_solving_linear_equations}
\newcounter{lesson_equations_containing_absolute_values}
\newcounter{lesson_graphing_lines}
\newcounter{lesson_two_forms_of_a_linear_equation}
\newcounter{lesson_parallel_and_perpendicular_lines}
\newcounter{lesson_linear_inequalities}
\newcounter{lesson_compound_inequalities}
\newcounter{lesson_inequalities_containing_absolute_values}
\newcounter{lesson_graphing_systems}
\newcounter{lesson_substitution}
\newcounter{lesson_elimination}
\newcounter{lesson_quadratics_introduction}
\newcounter{lesson_factoring_GCF}
\newcounter{lesson_factoring_grouping}
\newcounter{lesson_factoring_trinomials_a_is_1}
\newcounter{lesson_factoring_trinomials_a_neq_1}
\newcounter{lesson_solving_by_factoring}
\newcounter{lesson_square_roots}
\newcounter{lesson_i_and_complex_numbers}
\newcounter{lesson_vertex_form_and_graphing}
\newcounter{lesson_solve_by_square_roots}
\newcounter{lesson_extracting_square_roots}
\newcounter{lesson_the_discriminant}
\newcounter{lesson_the_quadratic_formula}
\newcounter{lesson_quadratic_inequalities}
\newcounter{lesson_functions_and_relations}
\newcounter{lesson_evaluating_functions}
\newcounter{lesson_finding_domain_and_range_graphically}
\newcounter{lesson_fundamental_functions}
\newcounter{lesson_finding_domain_algebraically}
\newcounter{lesson_solving_functions}
\newcounter{lesson_function_arithmetic}
\newcounter{lesson_composite_functions}
\newcounter{lesson_inverse_functions_definition_and_HLT}
\newcounter{lesson_finding_an_inverse_function}
\newcounter{lesson_transformations_translations}
\newcounter{lesson_transformations_reflections}
\newcounter{lesson_transformations_scalings}
\newcounter{lesson_transformations_summary}
\newcounter{lesson_piecewise_functions}
\newcounter{lesson_functions_containing_absolute_values}
\newcounter{lesson_absolute_as_piecewise}
\newcounter{lesson_polynomials_introduction}
\newcounter{lesson_sign_diagrams_polynomials}
\newcounter{lesson_factoring_quadratic_type}
\newcounter{lesson_factoring_summary}
\newcounter{lesson_polynomial_division}
\newcounter{lesson_synthetic_division}
\newcounter{lesson_end_behavior_polynomials}
\newcounter{lesson_local_behavior_polynomials}
\newcounter{lesson_rational_root_theorem}
\newcounter{lesson_polynomials_graphing_summary}
\newcounter{lesson_polynomial_inequalities}
\newcounter{lesson_rationals_introduction_and_terminology}
\newcounter{lesson_sign_diagrams_rationals}
\newcounter{lesson_horizontal_asymptotes}
\newcounter{lesson_slant_and_curvilinear_asymptotes}
\newcounter{lesson_vertical_asymptotes}
\newcounter{lesson_holes}
\newcounter{lesson_rationals_graphing_summary}

\setcounter{lesson_solving_linear_equations}{1}
\setcounter{lesson_equations_containing_absolute_values}{2}
\setcounter{lesson_graphing_lines}{3}
\setcounter{lesson_two_forms_of_a_linear_equation}{4}
\setcounter{lesson_parallel_and_perpendicular_lines}{5}
\setcounter{lesson_linear_inequalities}{6}
\setcounter{lesson_compound_inequalities}{7}
\setcounter{lesson_inequalities_containing_absolute_values}{8}
\setcounter{lesson_graphing_systems}{9}
\setcounter{lesson_substitution}{10}
\setcounter{lesson_elimination}{11}
\setcounter{lesson_quadratics_introduction}{16}
\setcounter{lesson_factoring_GCF}{17}
\setcounter{lesson_factoring_grouping}{18}
\setcounter{lesson_factoring_trinomials_a_is_1}{19}
\setcounter{lesson_factoring_trinomials_a_neq_1}{20}
\setcounter{lesson_solving_by_factoring}{21}
\setcounter{lesson_square_roots}{22}
\setcounter{lesson_i_and_complex_numbers}{23}
\setcounter{lesson_vertex_form_and_graphing}{24}
\setcounter{lesson_solve_by_square_roots}{25}
\setcounter{lesson_extracting_square_roots}{26}
\setcounter{lesson_the_discriminant}{27}
\setcounter{lesson_the_quadratic_formula}{28}
\setcounter{lesson_quadratic_inequalities}{29}
\setcounter{lesson_functions_and_relations}{12}
\setcounter{lesson_evaluating_functions}{13}
\setcounter{lesson_finding_domain_and_range_graphically}{14}
\setcounter{lesson_fundamental_functions}{15}
\setcounter{lesson_finding_domain_algebraically}{30}
\setcounter{lesson_solving_functions}{31}
\setcounter{lesson_function_arithmetic}{32}
\setcounter{lesson_composite_functions}{33}
\setcounter{lesson_inverse_functions_definition_and_HLT}{34}
\setcounter{lesson_finding_an_inverse_function}{35}
\setcounter{lesson_transformations_translations}{36}
\setcounter{lesson_transformations_reflections}{37}
\setcounter{lesson_transformations_scalings}{38}
\setcounter{lesson_transformations_summary}{39}
\setcounter{lesson_piecewise_functions}{40}
\setcounter{lesson_functions_containing_absolute_values}{41}
\setcounter{lesson_absolute_as_piecewise}{42}
\setcounter{lesson_polynomials_introduction}{43}
\setcounter{lesson_sign_diagrams_polynomials}{44}
\setcounter{lesson_factoring_quadratic_type}{46}
\setcounter{lesson_factoring_summary}{45}
\setcounter{lesson_polynomial_division}{47}
\setcounter{lesson_synthetic_division}{48}
\setcounter{lesson_end_behavior_polynomials}{49}
\setcounter{lesson_local_behavior_polynomials}{50}
\setcounter{lesson_rational_root_theorem}{51}
\setcounter{lesson_polynomials_graphing_summary}{52}
\setcounter{lesson_polynomial_inequalities}{53}
\setcounter{lesson_rationals_introduction_and_terminology}{54}
\setcounter{lesson_sign_diagrams_rationals}{55}
\setcounter{lesson_horizontal_asymptotes}{56}
\setcounter{lesson_slant_and_curvilinear_asymptotes}{57}
\setcounter{lesson_vertical_asymptotes}{58}
\setcounter{lesson_holes}{59}
\setcounter{lesson_rationals_graphing_summary}{60}

\newcommand{\tmmathbf}[1]{\ensuremath{\boldsymbol{#1}}}
\newcommand{\tmop}[1]{\ensuremath{\operatorname{#1}}}

\begin{document}
\section{Local Behavior}
Recall that the domain of a rational function $f(x)=\dfrac{p(x)}{q(x)}$ is the set of all $x$ such that $q(x)\neq 0$.  We call those values not in the domain of $f$ {\it discontinuities}, since they will correspond to a break in the graph of $f$.  In this section we will explore what happens to the graph of a rational function near a given discontinuity.
\par
This boils down to two cases:
\begin{enumerate}
\item vertical asymptotes, known as {\it infinite discontinuities,} and
\item holes, known as {\it removable discontinuities}.
\end{enumerate}
In order to discuss either of the aforementioned cases, we will need to consider a ``simplified expression'' for a rational function $f$.
\par
For example, in Example \ref{sign_diag_2}, we looked at the function $h(x)=\dfrac{(x+1)(x-3)^2}{x-3}$.
\par
As long as $x$ does not equal $3,$ we can think of $(x+1)(x-3)$ as a simplified expression for $h,$ since the two expressions will always produce the same value for $x\neq 3$.
\par
For example, when $x=5,$ we get
$$h(5)=\dfrac{(5+1)(5-3)^2}{5-3}=\dfrac{(5+1)(5-3)^{\cancel{2}}}{\cancel{5-3}}=(5+1)(5-3)=6\cdot 2=12$$
Similarly, if we look at Example \ref{sign_diag_3}, a simplified expression for 
$$f(x)=\dfrac{(x-4)(x-5)}{(x+2)(x-5)}$$
would be $\dfrac{x-4}{x+2}$.
\par
It is important to note that one should never assume that a simplified expression for a rational function $f$ will equal $f$ for {\it all} values of $x$.  This is evident in each of our examples, since the domain of our simplified expression does not equal the domain of the given function.  Nevertheless, the two expressions are closely related, and are needed in order to identify the various discontinuities in the graph of a given rational function.
\subsection{Vertical Asymptotes (L\arabic{lesson_vertical_asymptotes})}
{\bf Objective: Identify one or more vertical asymptotes in the graph of a rational function.}\par
The central idea around a vertical asymptote, say $x=c,$ is that as $x$ approaches the value of $c,$ either from the left or the right, the values for the corresponding function $f(x)$ will approach either $\infty$ or $-\infty$.
\begin{center}
\begin{tabular}{ll}
Approaching from the right: & As $x\rightarrow c^+, \ f(x)\rightarrow\pm\infty.$\\
&\\
Approaching from the left: & As $x\rightarrow c^-, \ f(x)\rightarrow\pm\infty.$
\end{tabular}
\end{center}
We should be clear here, in that when we say $x$ approaches $c$ {\it from the right,} what is meant is that we are evaluating the function at values of $x$ that are getting arbitrarily close go $c,$ but are all {\it greater} than $c,$ i.e., $x>c$.  This is precisely why we can write $x\rightarrow c^+$ in the statement above.  The $+$ in the exponent signifies that $x>c.$ The same can be said for when $x$ approaches $c$ from the left.  The following graph further illustrates this point.
\begin{center}
\begin{tikzpicture}[xscale=0.7,yscale=1.4]
	\draw [<->](-4.25,0) -- coordinate (x axis mid) (4.25,0) node[below right] {$x$};
	\draw [-, dashed](0,-0.5) -- coordinate (y axis mid) (0,3.75) node[below] {};
	\draw [<-] plot [domain=0.30:3, samples=100] (\x,{1/\x});
	\draw [->] plot [domain=-3:-0.30, samples=100] (\x,{-1/\x});
	\draw[fill] (1,1) ellipse (0.5mm and 0.25mm);
  \draw[fill] (2,0.5) ellipse (0.5mm and 0.25mm);
	\draw[fill] (0.5,2) ellipse (0.5mm and 0.25mm);
	\draw[fill] (0.333,3) ellipse (0.5mm and 0.25mm);
	\draw[fill] (-1,1) ellipse (0.5mm and 0.25mm);
  \draw[fill] (-2,0.5) ellipse (0.5mm and 0.25mm);
	\draw[fill] (-0.5,2) ellipse (0.5mm and 0.25mm);
	\draw[fill] (-0.333,3) ellipse (0.5mm and 0.25mm);
	\draw (0,-0.75) node {$x=c$};
	\draw (1.7,0.25) node {$c^+\longleftarrow x$};
	\draw (-1.7,0.25) node {$x\longrightarrow c^-$};
	\draw (2,-0.25) node {$x>c$};
	\draw (-2,-0.25) node {$x<c$};
	\draw (1.5,2) node {$f(x)$};
	\draw (1.5,2.3) node {$\uparrow$};
	\draw (1.5,2.6) node {$\infty$};
	\draw (-1.5,2) node {$f(x)$};
	\draw (-1.5,2.3) node {$\uparrow$};
	\draw (-1.5,2.6) node {$\infty$};
\end{tikzpicture}
\end{center}
Our previous graph shows that as $x$ approaches $c$ from either direction, the values for $f(x)$ approach $+\infty$.  If, instead, we reflected the right-hand side of the graph across the $x-$axis, we would say that as $x\rightarrow c^+,$ $f(x)\rightarrow -\infty,$ since the right-hand side would now point downwards.
\par
Up until this point, we have seen several examples of graphs of rational functions that contain vertical asymptotes.  We are now ready to formally state the condition for the existence of a vertical asymptote.
\begin{center}
\framebox{
\begin{minipage}{0.8\linewidth}
Let $f(x)$ be a rational function and let $g(x)$ represent the simplified expression for $f$.  If $x=c$ is not in the domain of {\it both} $f$ and $g,$ then the graph of $f$ will have a vertical asymptote at $x=c$.
\end{minipage}
}
\end{center}
In the case where $f$ cannot be simplified, any value not in the domain will correspond to a vertical asymptote in the graph.  Examples \ref{horiz_asym_1} and \ref{horiz_asym_3} are a good place to start.  We revisit each of them now.
\begin{example}\label{vert_asym_1}
$f(x)=\dfrac{-2x+4}{x-5}=\dfrac{-2(x-2)}{x-5}$
\begin{multicols}{2}
We quickly see that the expression for $f$ is already simplified.  In this case,
$$\text{As }x\rightarrow 5^+, \ f(x)\rightarrow -\infty.$$
$$\text{As }x\rightarrow 5^-, \ f(x)\rightarrow \infty.$$
A sign diagram for $f$ will also help us to confirm whether $f(x)$ will approach $\infty$ or $-\infty,$ as $x$ approaches $5$ from both the right ($x>5$) and the left ($x<5$).
\begin{tikzpicture}[xscale=1,yscale=1]
	\draw [<->](-0.25,0) -- coordinate (x axis mid) (7.25,0) node[below right] {$x$};
	\draw [-](2,1) -- coordinate (y axis mid) (2,-0.25) node[below] {$2$};
	\draw [-, dashed](5,1) -- coordinate (y axis mid) (5,-0.25) node[below] {$5$};
	\draw (1,-1) node {$x=0$};
	\draw (3.5,-1) node {$x=3$};
	\draw (6,-1) node {$x=6$};
	\draw (1,0.5) node {$-$};
	\draw (3.5,0.5) node {$+$};
	\draw (6,0.5) node {$-$};
\end{tikzpicture}
\begin{tikzpicture}[xscale=0.3,yscale=0.3]
	\draw [<->](-11.5,0) -- coordinate (x axis mid) (11.5,0) node[below right] {$x$};
	\draw [<->](0,-11.5) -- coordinate (y axis mid) (0,11.5) node[above right] {$y$};
	\draw [<->] plot [domain=-10.5:4.5, samples=100] (\x,{(-2*\x+4)/(\x-5)});
	\draw [<->] plot [domain=5.75:11, samples=100] (\x,{(-2*\x+4)/(\x-5)});
	\draw [<->,dashed](5,11) -- (5,-11) node[below] {$x=5$};
	\draw [<->,dashed](11,-2) -- (-11,-2) node[below left] {};
	\foreach \x in {2,4,...,10}
		\draw (\x,2pt) -- (\x,-2pt)	node[anchor=north] {\scriptsize \x};
	\foreach \x in {-10,-8,...,-2}
		\draw (\x,2pt) -- (\x,-2pt)	node[anchor=south] {\scriptsize \x};
	\foreach \y in {2,4,...,10}
		\draw (2pt,\y) -- (-2pt,\y)	node[anchor=east] {\scriptsize \y}; 
	\foreach \y in {-10,-8,...,-2}
		\draw (2pt,\y) -- (-2pt,\y)	node[anchor=west] {\scriptsize \y}; 
	\draw (-6,-4) node {$y=-2$};
\end{tikzpicture}
\end{multicols}
\end{example}
\newpage
\begin{example}\label{vert_asym_2}
$h(x)=\dfrac{x^2+25}{x^2-10x+25}=\dfrac{x^2+25}{(x-5)^2}$
\begin{multicols}{2}
Again, the expression for $h$ is already simplified.  In this case,
$$\text{As }x\rightarrow 5^+, \ f(x)\rightarrow \infty.$$
$$\text{As }x\rightarrow 5^-, \ f(x)\rightarrow \infty.$$
The sign diagram for $h$ confirms this result.
\begin{tikzpicture}[xscale=1.2,yscale=1]
	\draw [<->](2,0) -- coordinate (x axis mid) (8,0) node[below right] {$x$};
	\draw [-, dashed](5,1) -- coordinate (y axis mid) (5,-0.25) node[below] {$5$};
	\draw (3.5,-1) node {$x=0$};
	\draw (6.5,-1) node {$x=6$};
	\draw (3.5,0.5) node {$+$};
	\draw (6.5,0.5) node {$+$};
\end{tikzpicture}
\begin{tikzpicture}[xscale=0.2,yscale=0.4]
	\draw [<->](-19.5,0) -- coordinate (x axis mid) (19.5,0) node[below right] {$x$};
	\draw [<->](0,-3.5) -- coordinate (y axis mid) (0,11.5) node[above right] {$y$};
	\draw [<->] plot [domain=-19:3.2, samples=100] (\x,{1+((10*\x)/(\x-5)^2)});
	\draw [<->] plot [domain=7.8:19, samples=100] (\x,{(\x^2+25)/(\x-5)^2});
	\draw [<->,dashed](5,11) -- (5,-3) node[below] {$x=5$};
	\draw [<->,dashed](-19,1) -- (19,1) node[right] {};
	\foreach \x in {4,8,...,16}
		\draw (\x,4pt) -- (\x,-4pt)	node[anchor=north] {\scriptsize \x};
	\foreach \x in {-16,-12,...,-4}
		\draw (\x,4pt) -- (\x,-4pt)	node[anchor=north] {\scriptsize \x};
	\foreach \y in {2,4,...,10}
		\draw (4pt,\y) -- (-4pt,\y)	node[anchor=east] {\scriptsize \y}; 
	\foreach \y in {-2}
		\draw (4pt,\y) -- (-4pt,\y)	node[anchor=west] {\scriptsize \y}; 
	\draw (-12,2.5) node {$y=1$};
\end{tikzpicture}\end{multicols}
\end{example}
If, unlike the previous two examples, we were faced with a rational function whose simplified expression did {\it not} equal the original function, we would still have a vertical asymptote at $x=c,$ as long as a factor of $(x-c)$ remained in the simplified expression.  This is precisely the same as saying that $x=c$ is not in the domain of both the original function and the simplified expression.  A good example of this is Example \ref{sign_diag_3}, which we revisit now.
\begin{example}\label{vert_asym_3}
$f(x)=\dfrac{x^2-9x+20}{x^2-3x-10}=\dfrac{(x-4)(x-5)}{(x+2)(x-5)}$
\par
Domain of $f$: $x\neq -2,5$
\par
Simplified Expression for $f$: $\dfrac{x-4}{x+2}$
\par
Domain of Simplified Expression: $x\neq -2$
\par
It follows that the graph of $f$ has a vertical asymptote at $x=-2$.  Notice, however, that $f$ does {\it not} have a vertical asymptote at $x=5,$ since $x=5$ is in the domain of the simplified expression.  We will revisit this example, yet again, in the very near future, to discuss the nature of the graph of $f$ near $x=5$.
\par
The sign diagram for $f$ (shown below) confirms the following statements.
$$\text{As }x\rightarrow -2^+, \ f(x)\rightarrow -\infty.$$
$$\text{As }x\rightarrow -2^-, \ f(x)\rightarrow \infty.$$
\begin{center}
\begin{tikzpicture}[xscale=1,yscale=1]
	\draw [<->](-4.25,0) -- coordinate (x axis mid) (7.25,0) node[below right] {$x$};
	\draw [-, dashed](-2,1) -- coordinate (y axis mid) (-2,-0.25) node[below] {$-2$};
	\draw [-](3,1) -- coordinate (y axis mid) (3,-0.25) node[below] {$4$};
	\draw [-, dashed](5,1) -- coordinate (y axis mid) (5,-0.25) node[below] {$5$};
	\draw (-3,-1) node {$x=-3$};
	\draw (0.5,-1) node {$x=0$};
	\draw (4,-1) node {$x=4.5$};
	\draw (6,-1) node {$x=6$};
	\draw (-3,0.5) node {$+$};
	\draw (0.5,0.5) node {$-$};
	\draw (4,0.5) node {$+$};
	\draw (6,0.5) node {$+$};
\end{tikzpicture}
\end{center}
\end{example}
If we look more closely at each of our last three examples, we can begin to make sense of why a particular function approaches $+\infty$ or $-\infty,$ as $x$ approaches our particular discontinuity.  The explanation it tied to the notion of {\it multiplicity} from the chapter on polynomial functions.
\par
Recall that the multiplicity of a root $c$ for a polynomial function $p(x)$ is the maximum number of times, $k,$ that the factor of $x-c$ appears in the polynomial's complete factorization.  For example, if we consider the polynomial
$$p(x)=2x^3(x-4)^2(x-5),$$
the roots $x_1=0$ $x_2=4,$ and $x_3=5$ would have respective multiplicities $k_1=3,$ $k_2=2,$ and $k_3=1$.
\par
The multiplicity $k$ of a root $c$ helped us to determine whether the corresponding $x-$intercept was a {\it crossover} point (when $k$ is odd) or a {\it turnaround} point (when $k$ is even).  It turns out that we can use this same idea for visualizing vertical asymptotes, when $x=c$ is a root of the denominator $q(x)$ of a rational function $f(x)=\dfrac{p(x)}{q(x)}$.\par
%\begin{center}
\framebox{
\begin{minipage}{1.0\linewidth}
Let $f$ be a rational function with vertical asymptote at $x=c,$ and let $k$ be the multiplicity of $c$ in the denominator of the {\it simplified expression} for $f$.\par
If $k$ is odd, then the graph of $f$ near $x=c$ will resemble one of the following:
\begin{center}
\begin{multicols}{2}
\begin{tikzpicture}[xscale=0.45,yscale=0.45]
	\draw [-](-4,0) -- coordinate (x axis mid) (4,0) node[below right] {};
	\draw [-,dashed](0,-3.5) -- coordinate (y axis mid) (0,3.5) node[above right] {};
	\draw [<-] plot [domain=0.30:2, samples=100] (\x,{1/\x});
	\draw [->] plot [domain=-2:-0.30, samples=100] (\x,{1/\x});
	\draw (0,-4.5) node {$x=c$};
	\draw (0,-6) node {As $x\rightarrow c^+,\ f(x)\rightarrow \infty$};
	\draw (0,-7.5) node {As $x\rightarrow c^-,\ f(x)\rightarrow -\infty$};
	\draw (-3,3) node {(i)};
\end{tikzpicture}

\begin{tikzpicture}[xscale=0.45,yscale=0.45]
	\draw [<->](-4,0) -- coordinate (x axis mid) (4,0) node[below right] {$x$};
	\draw [-,dashed](0,-3.5) -- coordinate (y axis mid) (0,3.5) node[above right] {};
	\draw [<-] plot [domain=0.30:2, samples=100] (\x,{-1/\x});
	\draw [->] plot [domain=-2:-0.30, samples=100] (\x,{-1/\x});
	\draw (0,-4.5) node {$x=c$};
	\draw (0,-6) node {As $x\rightarrow c^+,\ f(x)\rightarrow -\infty$};
	\draw (0,-7.5) node {As $x\rightarrow c^-,\ f(x)\rightarrow \infty$};
	\draw (-3,3) node {(ii)};
\end{tikzpicture}
\end{multicols}
\end{center}

If $k$ is even, then the graph of $f$ near $x=c$ will resemble one of the following:
\begin{center}
\begin{multicols}{2}
\begin{tikzpicture}[xscale=0.45,yscale=0.45]
	\draw [-](-4,0) -- coordinate (x axis mid) (4,0) node[below right] {};
	\draw [-,dashed](0,-3.5) -- coordinate (y axis mid) (0,3.5) node[above right] {};
	\draw [<-] plot [domain=0.30:2, samples=100] (\x,{1/\x});
	\draw [->] plot [domain=-2:-0.30, samples=100] (\x,{-1/\x});
	\draw (0,-4.5) node {$x=c$};
	\draw (0,-6) node {As $x\rightarrow c^+,\ f(x)\rightarrow \infty$};
	\draw (0,-7.5) node {As $x\rightarrow c^-,\ f(x)\rightarrow \infty$};
	\draw (-3,3) node {(iii)};
\end{tikzpicture}

\begin{tikzpicture}[xscale=0.45,yscale=0.45]
	\draw [<->](-4,0) -- coordinate (x axis mid) (4,0) node[below right] {$x$};
	\draw [-,dashed](0,-3.5) -- coordinate (y axis mid) (0,3.5) node[above right] {};
	\draw [<-] plot [domain=0.30:2, samples=100] (\x,{-1/\x});
	\draw [->] plot [domain=-2:-0.30, samples=100] (\x,{1/\x});
	\draw (0,-4.5) node {$x=c$};
	\draw (0,-6) node {As $x\rightarrow c^+,\ f(x)\rightarrow -\infty$};
	\draw (0,-7.5) node {As $x\rightarrow c^-,\ f(x)\rightarrow -\infty$};
	\draw (-3,3) node {(iv)};
\end{tikzpicture}
\end{multicols}
\end{center}
\end{minipage}
}
%\end{center}
We clearly see this idea at work for $f(x)=\dfrac{-2(x-2)}{x-5}$ (Example \ref{vert_asym_1}) and $h(x)=\dfrac{x^2+25}{(x-5)^2}$ (Example \ref{vert_asym_2}), whose graphs both show a vertical asymptote at $x=5$.  In the case of $f,$ the multiplicity of $x=5$ in our denominator ($k=1$) is odd.  Consequently, the two sides of our graph near $x=5$ approach the vertical asymptote on opposite sides of the $x-$axis.  Alternatively, the multiplicity of $x=5$ in the denominator of $h$ is even ($k=2$).  Consequently, the two sides of our graph near $x=5$ approach the vertical asymptote on the same side of the $x-$axis.
\par
Furthermore, for a given rational function with vertical asymptote at $x=c,$ once we have identified the associated multiplicity $k,$ we can use a sign diagram to determine precisely which case matches our graph: (i) or (ii) when $k$ is odd, and (iii) or (iv) when $k$ is even.
\begin{example}
We already know that $f(x)=\dfrac{(x-4)(x-5)}{(x+2)(x-5)}$ has a vertical asymptote at $x=-2$ from Example \ref{vert_asym_3}.
\par
Since the multiplicity of $x=-2$ in the denominator of our simplified expression $\dfrac{x-4}{x+2}$ is odd ($k=1$), we know that the nature of the graph near $x=-2$ will resemble one of the following cases.
\begin{center}
\begin{multicols}{2}
\begin{tikzpicture}[xscale=0.45,yscale=0.45]
	\draw [-](-4,0) -- coordinate (x axis mid) (4,0) node[below right] {};
	\draw [-,dashed](0,-3.5) -- coordinate (y axis mid) (0,3.5) node[above right] {};
	\draw [<-] plot [domain=0.30:2, samples=100] (\x,{1/\x});
	\draw [->] plot [domain=-2:-0.30, samples=100] (\x,{1/\x});
	\draw (0,-4.5) node {$x=-2$};
	\draw (0,-6) node {As $x\rightarrow -2^+,\ f(x)\rightarrow \infty$};
	\draw (0,-7.5) node {As $x\rightarrow -2^-,\ f(x)\rightarrow -\infty$};
	\draw (-3,3) node {(i)};
\end{tikzpicture}

\begin{tikzpicture}[xscale=0.45,yscale=0.45]
	\draw [<->](-4,0) -- coordinate (x axis mid) (4,0) node[below right] {$x$};
	\draw [-,dashed](0,-3.5) -- coordinate (y axis mid) (0,3.5) node[above right] {};
	\draw [<-] plot [domain=0.30:2, samples=100] (\x,{-1/\x});
	\draw [->] plot [domain=-2:-0.30, samples=100] (\x,{-1/\x});
	\draw (0,-4.5) node {$x=-2$};
	\draw (0,-6) node {As $x\rightarrow -2^+,\ f(x)\rightarrow -\infty$};
	\draw (0,-7.5) node {As $x\rightarrow -2^-,\ f(x)\rightarrow \infty$};
	\draw (-3,3) node {(ii)};
\end{tikzpicture}
\end{multicols}
\end{center}
But recall that our sign diagram for $f$ was as follows.
\begin{center}
\begin{tikzpicture}[xscale=1,yscale=1]
	\draw [<->](-4.25,0) -- coordinate (x axis mid) (7.25,0) node[below right] {$x$};
	\draw [-, dashed](-2,1) -- coordinate (y axis mid) (-2,-0.25) node[below] {$-2$};
	\draw [-](3,1) -- coordinate (y axis mid) (3,-0.25) node[below] {$4$};
	\draw [-, dashed](5,1) -- coordinate (y axis mid) (5,-0.25) node[below] {$5$};
	\draw (-3,-1) node {$x=-3$};
	\draw (0.5,-1) node {$x=0$};
	\draw (4,-1) node {$x=4.5$};
	\draw (6,-1) node {$x=6$};
	\draw (-3,0.5) node {$+$};
	\draw (0.5,0.5) node {$-$};
	\draw (4,0.5) node {$+$};
	\draw (6,0.5) node {$+$};
\end{tikzpicture}
\end{center}
This tells us that our graph will point {\it upwards} to the left of $x=-2$ and {\it downwards} to the right of $x=-2$.  Hence, case (ii) is the correct graph for our function near $x=-2$.  And our corresponding statements match those from before.
$$\text{As }x\rightarrow -2^+, \ f(x)\rightarrow -\infty.$$
$$\text{As }x\rightarrow -2^-, \ f(x)\rightarrow \infty.$$
\end{example}
\subsection{Holes (L\arabic{lesson_holes})}
{\bf Objective: Identify the precise location of one or more holes in the graph of a rational function.}\par
While Vertical Asymptotes correspond to infinite discontinuities, a hole corresponds to a {\it removable discontinuity}, since the removal of a single point along a continuous curve creates the hole.
\par
Suppose that the rational function $f(x)$ has a discontinuity at $x=c,$ i.e., $c$ is not in the domain of $f$.  If $x=c$ is a vertical asympote of the graph of $f,$ we just saw that as $x\rightarrow c,$ $f(x)\rightarrow\pm\infty$.  If $x=c$ represents a hole in the graph of $f,$ however, we will see that as $x\rightarrow c,$ $f(x)\rightarrow k,$ for some real number $k$.  This is the fundamental difference between infinite and removable discontinuities.
\begin{center}
\begin{multicols}{2}
\begin{tikzpicture}[xscale=0.6,yscale=1.2]
	\draw [<->](-4.25,0) -- coordinate (x axis mid) (4.25,0) node[below right] {$x$};
	\draw [-, dashed](0,-0.5) -- coordinate (y axis mid) (0,3.75) node[below] {};
	\draw [<-] plot [domain=0.30:3, samples=100] (\x,{1/\x});
	\draw [->] plot [domain=-3:-0.30, samples=100] (\x,{-1/\x});
	\draw[fill] (1,1) ellipse (0.5mm and 0.25mm);
  \draw[fill] (2,0.5) ellipse (0.5mm and 0.25mm);
	\draw[fill] (0.5,2) ellipse (0.5mm and 0.25mm);
	\draw[fill] (0.333,3) ellipse (0.5mm and 0.25mm);
	\draw[fill] (-1,1) ellipse (0.5mm and 0.25mm);
  \draw[fill] (-2,0.5) ellipse (0.5mm and 0.25mm);
	\draw[fill] (-0.5,2) ellipse (0.5mm and 0.25mm);
	\draw[fill] (-0.333,3) ellipse (0.5mm and 0.25mm);
	\draw (0,-0.75) node {$x=c$};
	\draw (1.7,0.25) node {$c^+\longleftarrow x$};
	\draw (-1.7,0.25) node {$x\longrightarrow c^-$};
	\draw (2,-0.25) node {$x>c$};
	\draw (-2,-0.25) node {$x<c$};
	\draw (1.5,2) node {$f(x)$};
	\draw (1.5,2.3) node {$\uparrow$};
	\draw (1.5,2.6) node {$\infty$};
	\draw (-1.5,2) node {$f(x)$};
	\draw (-1.5,2.3) node {$\uparrow$};
	\draw (-1.5,2.6) node {$\infty$};
	\draw (0,-1.25) node {Infinite Discontinuity};
\end{tikzpicture}

\columnbreak

\begin{tikzpicture}[xscale=0.6,yscale=1.2]
	\draw [<->](-4.25,0) -- coordinate (x axis mid) (4.25,0) node[below right] {$x$};
	\draw [-, dashed](0,-0.5) -- coordinate (y axis mid) (0,1.9) node[below] {};
	\draw [-](-4,-0.25) -- coordinate (y axis mid) (-4,3.75) node[below] {};
	\draw [-, dashed](-3.6,2) -- coordinate (y axis mid) (-0.15,2) node[below] {};
	\draw [-](-4.15,2) -- coordinate (y axis mid) (-3.85,2) node[left] {};
	\draw [<->] plot [domain=-3.75:3.75, samples=100] (\x,{1.5*sin(\x/2.5 r)+2});
	\draw[fill] (1,2.584) ellipse (0.5mm and 0.25mm);
  \draw[fill] (2,3.07) ellipse (0.5mm and 0.25mm);
	\draw[fill] (0.5,2.298) ellipse (0.5mm and 0.25mm);
	\draw[fill] (0.333,2.197) ellipse (0.5mm and 0.25mm);
	\draw[fill] (-1,1.416) ellipse (0.5mm and 0.25mm);
  \draw[fill] (-2,0.924) ellipse (0.5mm and 0.25mm);
	\draw[fill] (-0.5,1.702) ellipse (0.5mm and 0.25mm);
	\draw[fill] (-0.333,1.803) ellipse (0.5mm and 0.25mm);
	\draw[fill, color=white] (0,2) ellipse (1.5mm and 0.75mm);
	\draw[line width=0.1mm] (0,2) ellipse (1.5mm and 0.75mm);
	\draw (0,-0.75) node {$x=c$};
	\draw (1.7,0.25) node {$c^+\longleftarrow x$};
	\draw (-1.7,0.25) node {$x\longrightarrow c^-$};
	\draw (2,-0.25) node {$x>c$};
	\draw (-2,-0.25) node {$x<c$};
	\draw (-4.65,3.1) node {$f(x)$};
	\draw (-4.65,2.5) node {$\downarrow$};
	\draw (-4.65,0.9) node {$f(x)$};
	\draw (-4.65,1.5) node {$\uparrow$};
	\draw (-4.65,2) node {$k$};
	\draw (0,-1.25) node {Removable Discontinuity};
\end{tikzpicture}
\end{multicols}
\end{center}

In the case of the graph of the left, recall that we have the following statements.
$$\text{As} \ x\rightarrow c^+, \ f(x)\rightarrow \infty. \qquad\qquad \text{As} \ x\rightarrow c^-, \ f(x)\rightarrow \infty.$$
Similarly, in the case of the graph on the right, we employ the same idea, using $k^+$ and $k^-$ in order to identify whether or not the graph of $f$ approaches $k$ from {\it above} if $f(x)>k$ and {\it below} if $f(x)<k$.
$$\text{As} \ x\rightarrow c^+, \ f(x)\rightarrow k^+. \qquad\qquad \text{As} \ x\rightarrow c^-, \ f(x)\rightarrow k^-.$$
In virtually all cases, however, it will be sufficient enough to simply state that as $x\rightarrow c, \ f(x)\rightarrow k,$ since further analysis will often prove difficult.
\par
We now state the requirement for a hole, which, as with vertical asymptotes, depends on both the rational function $f$ and its simplified expression.
\begin{center}
\framebox{
\begin{minipage}{0.8\linewidth}
Let $f(x)$ be a rational function and let $g(x)$ represent the simplified expression for $f$.  If $x=c$ is not in the domain of $f,$ but {\it is} in the domain of $g,$ then the graph of $f$ will have a hole at $(c,g(c))$.
\end{minipage}
}
\end{center}
For the existence of a vertical asymptote, the identified discontinuity had to be excluded from the domain of {\it both} $f$ and its simplified expression.  This is not the case for a hole, however, as the simplified expression $g$ is, in fact, defined at $x=c$.  Furthermore, the value $g(c)$ tells us the precise location of our hole along the $y-$axis.

Example \ref{horiz_asym_4} is our first example with a hole, and we revisit it now.

\begin{example}
\begin{multicols}{2}
Original Function:
\par
$k(x)=\dfrac{x^3-5x^2}{10x-50}=\dfrac{x^2(x-5)}{10(x-5)}$
\par
Domain of $k$: $x\neq 5$ or $(-\infty,5)\cup(5,\infty)$
\par
The original function $k$ is undefined at $x=5$

\columnbreak

Simplified Expression:
\par
$g(x)=\dfrac{x^2\cancel{x-5}}{10\cancel{(x-5)}}=\dfrac{x^2}{10}$
\par
Domain of $g$: $(-\infty,\infty)$
\par
$g(5)=\dfrac{25}{10}=\dfrac{5}{2}$
\end{multicols}
\begin{center}
Conclusion: The graph of $k$ has a hole at $(5,g(5))=(5,\frac{5}{2})$.
\begin{multicols}{2}
\begin{tikzpicture}[xscale=0.4,yscale=0.4]
	\draw [<->](-7.5,0) -- coordinate (x axis mid) (7.5,0) node[below right] {$x$};
	\draw [<->](0,-1.5) -- coordinate (y axis mid) (0,6.5) node[above right] {$y$};
	\draw [<-] plot [domain=-7:4.9, samples=100] (\x,{(\x)^2/10});
	\draw [->] plot [domain=5.1:7, samples=100] (\x,{(\x)^2/10});
	\draw (5,2.5) circle (0.18);
	\foreach \x in {1,2,...,7}
		\draw (\x,2pt) -- (\x,-2pt)	node[anchor=north] {\scriptsize \x};
	\foreach \x in {-7,-6,...,-1}
		\draw (\x,2pt) -- (\x,-2pt)	node[anchor=north] {\scriptsize \x};
	\foreach \y in {1,2,...,6}
		\draw (2pt,\y) -- (-2pt,\y)	node[anchor=east] {\scriptsize \y}; 
	\draw (0,-3) node {Graph of $k(x)=\dfrac{x^3-5x^2}{10x-50}$};
\end{tikzpicture}

\columnbreak

\begin{tikzpicture}[xscale=0.4,yscale=0.4]
	\draw [<->](-7.5,0) -- coordinate (x axis mid) (7.5,0) node[below right] {$x$};
	\draw [<->](0,-1.5) -- coordinate (y axis mid) (0,6.5) node[above right] {$y$};
	\draw [<-] plot [domain=-7:4.9, samples=100] (\x,{(\x)^2/10});
	\draw [->] plot [domain=5.1:7, samples=100] (\x,{(\x)^2/10});
	\draw [fill] (5,2.5) circle (0.18);
	\foreach \x in {1,2,...,7}
		\draw (\x,2pt) -- (\x,-2pt)	node[anchor=north] {\scriptsize \x};
	\foreach \x in {-7,-6,...,-1}
		\draw (\x,2pt) -- (\x,-2pt)	node[anchor=north] {\scriptsize \x};
	\foreach \y in {1,2,...,6}
		\draw (2pt,\y) -- (-2pt,\y)	node[anchor=east] {\scriptsize \y}; 
	\draw (0,-3) node {Graph of $g(x)=\dfrac{x^2}{10}$};
\end{tikzpicture}
\end{multicols}
\end{center}
\end{example}
In this first example, we observe a somewhat obvious fact that we have neglected to state until now:
\begin{quote}
The graph of a rational function $f$ will always equal the graph of its simplified expression for any $x$ {\it in the domain} of $f$.
\end{quote}
In this case, the graphs of $k(x)=\dfrac{x^3-5x^2}{10x-50}$ and the familiar quadratic function $g(x)=\dfrac{x^2}{10}$ only differ in their behavior at $x=5$.
\par
We now take one last look at a familiar example (\ref{vert_asym_3}), and include its graph for completeness
\begin{example}
$f(x)=\dfrac{x^2-9x+20}{x^2-3x-10}=\dfrac{(x-4)(x-5)}{(x+2)(x-5)}$
\par
The simplified expression for $f$ is $g(x)=\dfrac{x-4}{x+2}$.
\par
Notice that $x=5$ is in the domain of $g,$ with $g(5)=\frac{5-4}{5+2}=\frac{1}{7}$.
\par
Since $x=5$ is not in the domain of $f,$ the graph of $f$ will have a hole at $(5,\frac{1}{7})$.
\par
Recall that our sign diagram for $f$ is as follows.
\begin{center}
\begin{tikzpicture}[xscale=1,yscale=1]
	\draw [<->](-4.25,0) -- coordinate (x axis mid) (7.25,0) node[below right] {$x$};
	\draw [-, dashed](-2,1) -- coordinate (y axis mid) (-2,-0.25) node[below] {$-2$};
	\draw [-](3,1) -- coordinate (y axis mid) (3,-0.25) node[below] {$4$};
	\draw [-, dashed](5,1) -- coordinate (y axis mid) (5,-0.25) node[below] {$5$};
	\draw (-3,-1) node {$x=-3$};
	\draw (0.5,-1) node {$x=0$};
	\draw (4,-1) node {$x=4.5$};
	\draw (6,-1) node {$x=6$};
	\draw (-3,0.5) node {$+$};
	\draw (0.5,0.5) node {$-$};
	\draw (4,0.5) node {$+$};
	\draw (6,0.5) node {$+$};
\end{tikzpicture}
\end{center}
We now are ready to make sense of the complete graph of $f,$ presented below.
\begin{center}
\begin{tikzpicture}[xscale=0.45, yscale=0.9]
	\draw [<->](-13.25,0) -- coordinate (x axis mid) (13.25,0) node[below right] {$x$};
	\draw [<->](0,-5.25) -- coordinate (x axis mid) (0,5.25) node[above right] {$y$};
	\draw [<->] plot [domain=-13:-3.5, samples=100] (\x,{(\x-4)/(\x+2)});
	\draw [<->] plot [domain=-1:13, samples=100] (\x,{(\x-4)/(\x+2)});
	\draw [<->, dashed](-2,-5) -- coordinate (y axis mid) (-2,5) node[below] {};
	\foreach \x in {2,4,...,12}
		\draw (\x,2pt) -- (\x,-2pt)	node[anchor=north] {\scriptsize \x};
	\foreach \x in {-12,-10,...,-2}
		\draw (\x,2pt) -- (\x,-2pt)	node[anchor=north] {\scriptsize \x};
	\foreach \y in {1,2,...,5}
		\draw (2pt,\y) -- (-2pt,\y)	node[anchor=east] {\scriptsize \y}; 
	\foreach \y in {-1,-2,...,-5}
		\draw (2pt,\y) -- (-2pt,\y)	node[anchor=west] {\scriptsize \y}; 
	\draw [fill, color=white] (5,0.14286) ellipse (2mm and 1mm);
	\draw [line width=0.25mm] (5,0.14286) ellipse (2mm and 1mm);
	\draw (-4,-4.5) node {$x=-2$};
	\draw (10,1.5) node {$y=1$};
	\draw (0,-6.25) node {$f(x)=\dfrac{x^2-9x+20}{x^2-3x-10}=\dfrac{(x-4)(x-5)}{(x+2)(x-5)}=\dfrac{x-4}{x+2},\ x\neq 5$};
	\draw (7,-2) node {Hole at $(5,\frac{1}{7})$};
	\draw [->](6,-1.5) -- coordinate (x axis mid) (5.25,-0.25);
	\draw [<->, dashed](-13,1) -- coordinate (x axis mid) (13,1);
\end{tikzpicture}
\end{center}
\end{example}
In each of our last two examples, the simplified expression for our given rational function has seen a complete elimination of the ``offending factor'' $(x-c)$ from both the numerator and the denominator.  Based upon our criteria for identifying holes, this is certainly a requirement for the denominator.  As we will see with our next example, however, it is possible that not all offending factors will completely disappear from the numerator.  In this situation, our hole will simply reside on the $x-$axis, since $g(c)$ will equal zero in our simplified expression.
\newpage
\begin{example}
Find the domain of $f(x)=\dfrac{4x^2}{x^3+3x^2-4x},$ and identify all discontinuities.  In each case, determine whether the discontinuity is infinite (vertical asymptote) or removable (hole).
Once again, we begin by obtaining a complete factorization of $f$ in order to identify its domain and simplified expression.
\begin{center}
\begin{multicols}{2}
\begin{equation*}
\begin{split}
f(x) & =\dfrac{4x^2}{x^3+3x^2-4x}\\
& = \dfrac{4x^2}{x(x^2+3x-4)}\\
& = \dfrac{4x^2}{x(x+4)(x-1)}
\end{split}
\end{equation*}

\columnbreak

The domain of $f$ is $x\neq -4,0,1$ or
$$(-\infty,-4)\cup(-4,0)\cup(0,1).$$
The simplified expression for $f$ is
$$g(x)=\dfrac{4x}{(x+4)(x-1)}.$$
\end{multicols}
\end{center}
Here, we see that the graph of $f$ has three discontinuities, occurring at those values not in the domain, $x=-4,0,$ and $1$.  Since the factors of $x+4$ and $x-1$ still appear in the denominator of the simplified expression, it follows that the discontinuities at $x=-4$ and $x=1$ will be infinite, and the graph of $f$ will have vertical asymptotes along these lines.
\par
Our third discontinuity at $x=0$ will be removable, since the simplified expression does not contain a factor of $x$ in its denominator.  Hence, the graph of $f$ will have a hole at the point $(0,g(0))=(0,0)$.
\par
We leave it as an exercise for the reader to verify our answer by graphing $f$ using \Desmos.
\end{example}
\end{document}