\documentclass[12pt]{book}
\raggedbottom
\usepackage[top=1in,left=1in,bottom=1in,right=1in,headsep=0.25in]{geometry}	
\usepackage{amssymb,amsmath,amsthm,amsfonts}
\usepackage{chapterfolder,docmute,setspace}
\usepackage{cancel,multicol,tikz,verbatim,framed,polynom,enumitem,tikzpagenodes}
\usepackage[colorlinks, hyperindex, plainpages=false, linkcolor=blue, urlcolor=blue, pdfpagelabels]{hyperref}
\usepackage[type={CC},modifier={by-sa},version={4.0},]{doclicense}

\theoremstyle{definition}
\newtheorem{example}{Example}
\newcommand{\Desmos}{\href{https://www.desmos.com/}{Desmos}}
\setlength{\parindent}{0in}
\setlist{itemsep=0in}
\setlength{\parskip}{0.1in}
\setcounter{secnumdepth}{0}
\input{lesson_order}

\newcommand{\tmmathbf}[1]{\ensuremath{\boldsymbol{#1}}}
\newcommand{\tmop}[1]{\ensuremath{\operatorname{#1}}}

\begin{document}
\section{Graphing Summary (L\arabic{lesson_rationals_graphing_summary})}
{\bf Objective: Graph a rational function in its entirety.}\par
At this point, we have addressed all key features of rational functions individually.  This section pulls each of these aspects together, for a detailed analysis of a rational function, culminating in a complete sketch of its graph.  Along the way, we will need to address each of the following aspects for our rational function $f(x)=\dfrac{p(x)}{q(x)}$.  It is important to note that there is no universally accepted order to this checklist.
\begin{itemize}
	\item Find the $y-$intercept of the graph of $f,$ $(0,f(0)),$ if it exists.
	\item Use the degrees and leading coefficients of $p$ and $q$ to determine whether the graph of $f$ has a horizontal asymptote.  If the graph of $f$ has a slant asymptote, use polynomial division to find where it is located.
	\item Identify a complete factorization of $f,$ and use it to find the domain of the function.  This is the set of all $x,$ such that $q(x)\neq 0$.
	\item Find any $x-$intercepts of the graph of $f$.  This is the set of all $x$ in the domain of $f,$ such that $p(x)=0$.  Using multiplicities, classify each $x-$intercept as a crossover or turnaround (``bounce'') point.
	\item Find the simplified expression $g$ for the given function $f$, and use it to identify any vertical asymptotes or holes in the graph of $f$.  Use multiplicities to help visualize the nature of the graph of $f$ near its vertical asymptotes.  If $f$ has a hole at $x=c,$ use $g$ to help plot the hole's precise location at $(c,g(c))$. 
	\item Using both the $x-$intercepts and the discontinuities (those $x$ not in the domain), construct a sign diagram for $f$.
\end{itemize}
In each example that follows, we will carefully examine the given function, making sure not to omit any of the checklist items above and to compare each item to those that precede it along the way for accuracy.  Although the process will take some time, if we are thorough, our end result should be a complete, accurate sketch of the given rational function.  We will start by revisiting our last example.
\begin{example}Sketch a complete graph of the rational function below, making sure to have a clearly defined scale and label all key features of your graph (intercepts, asymptotes, and holes).
$$f(x)=\dfrac{4x^2}{x^3+3x^2-4x}$$
In this first example, we see that the graph of $f$ will not have a $y-$intercept, since $f(0)=\frac{0}{0},$ which is undefined.
\par
Since the degree of the numerator is less than the degree of the denominator, we conclude that the graph of $f$ has a horizontal asymptote along the $x-$axis, $y=0$.
\par
Our graph also has no $x-$intercepts, since our numerator only equals zero when $x=0,$ which we know is not in our domain of $f$.
\par
Furthermore, from the work in our last example, we know that $f$ has a complete factorization of
$$f(x)=\dfrac{4x^2}{x(x+4)(x-1)},$$
with corresponding domain $x\neq -4,0,1,$ and simplified expression
$$g(x)=\dfrac{4x}{(x+4)(x-1)}.$$
Consequently, the graph of $f$ has vertical asymptotes at $x=-4$ and $x=1$ and a hole at $(0,g(0))=(0,0)$.
\par
Since the multiplicities of both $x=-4$ and $x=1$ in the denominator of $f$ are both one (odd), we know that the graph of $f$ will approach each vertical asymptote from opposite sides of the $x-$axis.  The following sign diagram confirms this observation.
\begin{center}
\begin{tikzpicture}[xscale=1,yscale=1]
	\draw [<->](-6.25,0) -- coordinate (x axis mid) (4.25,0) node[below right] {$x$};
	\draw [-, dashed](-4,1) -- coordinate (y axis mid) (-4,-0.25) node[below] {$-4$};
	\draw [-, dashed](-1,1) -- coordinate (y axis mid) (-1,-0.25) node[below] {$0$};
	\draw [-, dashed](2,1) -- coordinate (y axis mid) (2,-0.25) node[below] {$1$};
	\draw (-5,-1) node {$x=-5$};
	\draw (-2.5,-1) node {$x=-1$};
	\draw (0.5,-1) node {$x=\frac{1}{2}$};
	\draw (3,-1) node {$x=2$};
	\draw (-5,0.5) node {$-$};
	\draw (-2.5,0.5) node {$+$};
	\draw (0.5,0.5) node {$-$};
	\draw (3,0.5) node {$+$};
\end{tikzpicture}
\end{center}
We are now ready to try our hand at graphing $f,$ and begin our graph by defining a scale for both the $x-$ and $y-$axes, and identifying all intercepts, and asymptotes.  This should always be our first step to successfully sketching a decent-looking graph.  To emphasize this point, we first show an initial graph that identifies each of these features, and further shades those areas of the $xy-$plane that correspond to our sign diagram above.
\begin{center}
\begin{tikzpicture}[xscale=0.65,yscale=0.65]
	\fill[color=lightgray] (-8.25,-6.5)--(-8.25,0)--(-4,0)--(-4,-6.5)--cycle;
	\fill[color=lightgray] (-4,0)--(-4,7)--(0,7)--(0,0)--cycle;
	\fill[color=lightgray] (0,0)--(0,-6.5)--(1,-6.5)--(1,0)--cycle;
	\fill[color=lightgray] (1,0)--(1,7)--(8.25,7)--(8.25,0)--cycle;
	\draw [-](-8.25,0) -- coordinate (x axis mid) (8.25,0) node[below right] {};
	\draw [<->,dashed](-10.25,0) -- coordinate (x axis mid) (10.25,0) node[below right] {$x$};
	\draw [<->](0,-7.25) -- coordinate (x axis mid) (0,7.25) node[above right] {$y$};
	\draw [<->,dashed](-4,7) -- (-4,-6.5) node[below] {$x=-4$};
	\draw [<->,dashed](1,7) -- (1,-6.5) node[below] {$x=1$};
	\foreach \x in {2,4,...,8}
		\draw (\x,2pt) -- (\x,-2pt)	node[anchor=north] {\scriptsize \x};
	\foreach \x in {-8,-6,...,-2}
		\draw (\x,2pt) -- (\x,-2pt)	node[anchor=north] {\scriptsize \x};
	\foreach \y in {2,4,...,6}
		\draw (2pt,\y) -- (-2pt,\y)	node[anchor=east] {\scriptsize \y}; 
	\foreach \y in {-6,-4,...,-2}
		\draw (2pt,\y) -- (-2pt,\y)	node[anchor=east] {\scriptsize \y}; 
	\draw[fill, color=white] (0,0) circle (0.15);
	\draw[line width=0.3mm] (0,0) circle (0.15);
\end{tikzpicture}
\end{center}
We now carefully sketch the graph of $f$ based upon our findings.
\begin{center}
\begin{tikzpicture}[xscale=0.75,yscale=0.75]
	\draw [-](-8.25,0) -- coordinate (x axis mid) (8.25,0) node[below right] {};
	\draw [<->,dashed](-10.25,0) -- coordinate (x axis mid) (10.25,0) node[below right] {$x$};
	\draw [<->](0,-7.25) -- coordinate (x axis mid) (0,7.25) node[above right] {$y$};
	\draw [<->,dashed](-4,7) -- (-4,-6.5) node[below] {$x=-4$};
	\draw [<->,dashed](1,7) -- (1,-6.5) node[below] {$x=1$};
	\draw [<->] plot [domain=-9.5:-4.5, samples=100] (\x,{(4*\x)/((\x+4)*(\x-1))});
	\draw [<->] plot [domain=-3.55:0.887, samples=100] (\x,{(4*\x)/((\x+4)*(\x-1))});
	\draw [<->] plot [domain=1.13:9.5, samples=100] (\x,{(4*\x)/((\x+4)*(\x-1))});
	\foreach \x in {2,4,...,8}
		\draw (\x,2pt) -- (\x,-2pt)	node[anchor=north] {\scriptsize \x};
	\foreach \x in {-8,-6,...,-2}
		\draw (\x,2pt) -- (\x,-2pt)	node[anchor=north] {\scriptsize \x};
	\foreach \y in {2,4,...,6}
		\draw (2pt,\y) -- (-2pt,\y)	node[anchor=east] {\scriptsize \y}; 
	\foreach \y in {-6,-4,...,-2}
		\draw (2pt,\y) -- (-2pt,\y)	node[anchor=east] {\scriptsize \y}; 
	\draw[fill, color=white] (0,0) circle (0.15);
	\draw[line width=0.3mm] (0,0) circle (0.15);
	\draw (5,-5.5) node {$f(x)=\dfrac{4x^2}{x^3+3x^2-4x}$};
\end{tikzpicture}
\end{center}
\end{example}
\begin{example}Sketch a complete graph of the rational function below, making sure to have a clearly defined scale and label all key features of your graph (intercepts, asymptotes, and holes).
$$f(x)=\dfrac{x^2-6x+9}{x^2+x-6}$$
Again, we start by evaluating $f$ at $x=0$ to identify the $y-$intercept.  We get $(0,\frac{9}{-6})=(0,-\frac{3}{2})$.
\par
Since the degrees of the numerator and denominator are equal and the leading coefficients are also equal, we know that the graph of $f$ will have a horizontal asymptote at the line $y=1$.
\par
A complete factorization of $f$ gives us
$$f(x)=\dfrac{(x-3)^2}{(x+3)(x-2)},$$
which is also our simplified expression.
\par
Using our factorization, we see that the graph of $f$ will have an $x-$intercept at $(3,0).$  This will be a turnaround point, due to the even multiplicity of the root $x=3$ in the numerator.
\par
Our domain for $f$ is $x\neq -3,2,$ and the corresponding graph will have vertical asymptotes at $x=-3$ and $x=2$.
\par
Our sign diagram is as follows.
\begin{center}
\begin{tikzpicture}[xscale=1,yscale=1]
	\draw [<->](-6.25,0) -- coordinate (x axis mid) (6.25,0) node[below right] {$x$};
	\draw [-, dashed](-3,1) -- coordinate (y axis mid) (-3,-0.25) node[below] {$-3$};
	\draw [-, dashed](1,1) -- coordinate (y axis mid) (1,-0.25) node[below] {$2$};
	\draw [-](3,1) -- coordinate (y axis mid) (3,-0.25) node[below] {$3$};
	\draw (-4.5,-1) node {$x=-4$};
	\draw (-1,-1) node {$x=0$};
	\draw (2,-1) node {$x=\frac{5}{2}$};
	\draw (4.5,-1) node {$x=4$};
	\draw (-4.5,0.5) node {$+$};
	\draw (-1,0.5) node {$-$};
	\draw (2,0.5) node {$+$};
	\draw (4.5,0.5) node {$+$};
\end{tikzpicture}
\end{center}
As with our last example, we will build up to the graph of $f$ by first defining a scale, identifying our intercepts and asymptotes, and shading the corresponding areas that should include our graph.
\begin{center}
\begin{tikzpicture}[xscale=0.5,yscale=0.5]
	\fill[color=lightgray] (-12.25,8.25)--(-12.25,0)--(-3,0)--(-3,8.25)--cycle;
	\fill[color=lightgray] (2,-7.25)--(2,0)--(-3,0)--(-3,-7.25)--cycle;
	\fill[color=lightgray] (2,8.25)--(2,0)--(11.25,0)--(11.25,8.25)--cycle;
	\draw [<->](-12.25,0) -- coordinate (x axis mid) (12.25,0) node[below right] {$x$};
	\draw [<->, dashed](-12.25,1) -- coordinate (x axis mid) (11.25,1) node[right] {$y=1$};
	\draw [<->](0,-8.25) -- coordinate (x axis mid) (0,8.25) node[above right] {$y$};
	\draw [<->,dashed](-3,8.25) -- (-3,-7.25) node[below] {$x=-3$};
	\draw [<->,dashed](2,8.25) -- (2,-7.25) node[below] {$x=2$};
	\foreach \x in {2,4,...,12}
		\draw (\x,2pt) -- (\x,-2pt)	node[anchor=north] {\scriptsize \x};
	\foreach \x in {-12,-8,...,-2}
		\draw (\x,2pt) -- (\x,-2pt)	node[anchor=north] {\scriptsize \x};
	\foreach \y in {2,4,...,8}
		\draw (2pt,\y) -- (-2pt,\y)	node[anchor=east] {\scriptsize \y}; 
	\foreach \y in {-8,-6,...,-2}
		\draw (2pt,\y) -- (-2pt,\y)	node[anchor=east] {\scriptsize \y}; 
	\draw[fill] (3,0) circle (0.1);
	\draw[fill] (0,-1.5) circle (0.1);
\end{tikzpicture}
\end{center}
In this particular example, without further algebraic analysis of our function, we will have to make a choice as to how the graph of $f$ approaches the horizontal asymptote, $y=1$.  For example, does it approach from above or below?  Does the graph ever cross the horizontal asymptote?  In order to answer this last question, one could attempt to solve the equation $f(x)=1$ for all possible $x$ in order to see exactly where our graph crosses the line $y=1$.
\par
Similarly, we cannot know if our $y-$intercept is also a local maximum value for our graph or how exactly the graph intersects this point without using advanced methods that are typically covered in a calculus course.
\par
On account of these subtleties, before we see the complete graph of $f,$ we will take one more step, and begin to sketch the possibilities for our graph near the asymptotes and our intercepts.
\begin{center}
\begin{tikzpicture}[xscale=0.5,yscale=0.5]
	\fill[color=lightgray] (-12.25,8.25)--(-12.25,0)--(-3,0)--(-3,8.25)--cycle;
	\fill[color=lightgray] (2,-7.25)--(2,0)--(-3,0)--(-3,-7.25)--cycle;
	\fill[color=lightgray] (2,8.25)--(2,0)--(11.25,0)--(11.25,8.25)--cycle;
	\draw [<->](-12.25,0) -- coordinate (x axis mid) (12.25,0) node[below right] {$x$};
	\draw [<->, dashed](-12.25,1) -- coordinate (x axis mid) (11.25,1) node[right] {$y=1$};
	\draw [<->](0,-8.25) -- coordinate (x axis mid) (0,8.25) node[above right] {$y$};
	\draw [<->,dashed](-3,8.25) -- (-3,-7.25) node[below] {$x=-3$};
	\draw [<->,dashed](2,8.25) -- (2,-7.25) node[below] {$x=2$};
	\draw [-] plot [domain=2.37:3.7, samples=100] (\x,{(1.5*(\x-3)^2)/(\x)});
	\draw [<-] plot [domain=-12:-10.5, samples=100] (\x,{((\x-3)^2)/((\x+3)*(\x-2))-0.5});
	\draw [<-] plot [domain=-12:-10.5, samples=100] (\x,{-((\x-3)^2)/((\x+3)*(\x-2))+2.5});
	\draw [->] plot [domain=9.6:11, samples=100] (\x,{((\x-3)^2)/((\x+3)*(\x-2))+0.3});
	\draw [->] plot [domain=9.6:11, samples=100] (\x,{-((\x-3)^2)/((\x+3)*(\x-2))+1.8});
	\draw [->] plot [domain=-3.4:-3.3, samples=100] (\x,{12/((\x+3)*(\x-2))});
	\draw [->] plot [domain=-2.52:-2.64, samples=100] (\x,{12/((\x+3)*(\x-2))});
	\draw [->] plot [domain=1.5:1.63, samples=100] (\x,{12/((\x+3)*(\x-2))});
	\draw [->] plot [domain=2.4:2.3, samples=100] (\x,{12/((\x+3)*(\x-2))});
	\draw [-] plot [domain=-0.3:0.2, samples=100] (\x,{(-2*\x+1.5)/(2*\x-1)});
	\draw [-] plot [domain=-0.2:0.3, samples=100] (\x,{-(2*\x+1.5)/(2*\x+1)});
	\draw [-] plot [domain=0:0.35, samples=100] (\x,{(-(\x^2)-1.5)});
	\draw [-] plot [domain=0:0.35, samples=100] (-\x,{(-(\x)^2-1.5)});
	\foreach \x in {2,4,...,12}
		\draw (\x,2pt) -- (\x,-2pt)	node[anchor=north] {\scriptsize \x};
	\foreach \x in {-12,-8,...,-2}
		\draw (\x,2pt) -- (\x,-2pt)	node[anchor=north] {\scriptsize \x};
	\foreach \y in {2,4,...,8}
		\draw (2pt,\y) -- (-2pt,\y)	node[anchor=east] {\scriptsize \y}; 
	\foreach \y in {-8,-6,...,-2}
		\draw (2pt,\y) -- (-2pt,\y)	node[anchor=east] {\scriptsize \y}; 
	\draw[fill] (3,0) circle (0.1);
	\draw[fill] (0,-1.5) circle (0.1);
\end{tikzpicture}
\end{center}
At this point it is important to reinforce the fact that our graph is meant to be a {\it rough sketch} of the actual graph of $f$.  So, we have to make some choices about how to connect everything up properly.  A more complete sketch of $f$ requires advanced techniques and concepts that one would likely see in precalculus or calculus.  Nevertheless, as long as there is a logical, smooth connection to each aspect of our graph, we can rest assured that our analysis of the function is sufficient.
\par
We can now conclude this example with the actual graph of $f$.
\begin{center}
\begin{tikzpicture}[xscale=0.5,yscale=0.5]
	\draw [<->](-12.25,0) -- coordinate (x axis mid) (12.25,0) node[below right] {$x$};
	\draw [<->, dashed](-12.25,1) -- coordinate (x axis mid) (11.25,1) node[right] {$y=1$};
	\draw [<->](0,-8.25) -- coordinate (x axis mid) (0,8.25) node[above right] {$y$};
	\draw [<->,dashed](-3,8.25) -- (-3,-7.25) node[below] {$x=-3$};
	\draw [<->,dashed](2,8.25) -- (2,-7.25) node[below] {$x=2$};
	\draw [<->] plot [domain=-12:-4.02, samples=100] (\x,{((\x-3)^2/((\x+3)*(\x-2))});
	\draw [<->] plot [domain=-2.1:1.87, samples=100] (\x,{((\x-3)^2/((\x+3)*(\x-2))});
	\draw [<->] plot [domain=2.1:11.25, samples=100] (\x,{((\x-3)^2/((\x+3)*(\x-2))});
	\foreach \x in {2,4,...,12}
		\draw (\x,2pt) -- (\x,-2pt)	node[anchor=north] {\scriptsize \x};
	\foreach \x in {-12,-8,...,-2}
		\draw (\x,2pt) -- (\x,-2pt)	node[anchor=north] {\scriptsize \x};
	\foreach \y in {2,4,...,8}
		\draw (2pt,\y) -- (-2pt,\y)	node[anchor=east] {\scriptsize \y}; 
	\foreach \y in {-8,-6,...,-2}
		\draw (2pt,\y) -- (-2pt,\y)	node[anchor=east] {\scriptsize \y}; 
	\draw[fill] (3,0) circle (0.1);
	\draw[fill] (0,-1.5) circle (0.1);
	\draw (7,-6) node {$f(x)=\dfrac{x^2-6x+9}{x^2+x-6}$};
\end{tikzpicture}
\end{center}
As a result, we learn that as $x\rightarrow\infty,$ $f(x)\rightarrow 1^-,$ and consequently, our graph does not cross over the horizontal asymptote to the right of our $x-$intercept.  Furthermore, our $y-$intercept is not a local maximum for the graph of $f,$ as is often the case.  Lastly, as $x\rightarrow 2,$ we see that the graph of $f$ seems to hold more tightly to the vertical asymptote than when $x\rightarrow -3$.
\end{example}
\begin{example}\label{rat_ineq_1}Sketch a complete graph of the rational function below, making sure to have a clearly defined scale and label all key features of your graph (intercepts, asymptotes, and holes).
$$f(x)=\dfrac{x^3-16x}{3x^2-6x-24}$$
We will begin by obtaining a complete factorization of $f$ and a simplified expression.
\begin{multicols}{2}
\begin{equation*}
\begin{split}
f(x) & = %\dfrac{x^3-16x}{3x^2-6x-72}\\
		 %& = 
		\dfrac{x(x^2-16)}{3(x^2-2x-24)}\\
		 & = \dfrac{x(x+4)(x-4)}{3(x+4)(x-6)}\\
\end{split}
\end{equation*}

\columnbreak

The simplified expression for $f$ is 
\begin{equation*}
\begin{split}
g(x) & =\dfrac{x(x-4)}{3(x-6)}\\
& = \dfrac{x^2-4x}{3x-18}\\
\end{split}
\end{equation*}
\end{multicols}
We quickly see that the graph of $f$ has both an $x-$ and $y-$intercept at the origin, since $f(0)=0$.
\par
Since the degree of the numerator is one greater than the degree of the denominator, we know that the graph of $f$ will have a slant asymptote, and we can use polynomial division to find its precise location.  It is worth noting, however, that we can apply polynomial division to {\it either} our original function or its simplified expression in order to find our slant asymptote, as demonstrated below. 
\begin{multicols}{2}
Original Function:
\[
\polylongdiv{x^3-16x}{3x^2-6x-72}
\]

\columnbreak 

Simplified Expression:
\[
\polylongdiv{x^2-4x}{3x-18}
\]
\end{multicols}

Each calculation carries the same number of steps and yields the same equation for our slant asymptote, $y=\frac{1}{3}x+\frac{2}{3}$.  Since dividing the simplified expression for our function should always be a simpler process, it makes sense to use this approach when we encounter similar problems.
\par
Observe that the domain of $f$ is $x\neq -4,6$.  Since the factor of $x+4$ appears in both the numerator and denominator, the graph of $f$ will contain a hole at $(-4,g(-4))=(-4,-\frac{16}{15})$ and a vertical asymptote at $x=6$.  Our graph will also include crossover $x-$intercepts at $x=0$ and $x=4,$ as their multiplicities in the numerator are both odd.
\par
The sign diagram for $f$ is shown below, and we conclude with the graph of $f$.  Though not necessary, we have again included a shading of the regions of our graph that correspond to our sign diagram.
\begin{center}
\begin{tikzpicture}[xscale=1,yscale=1]
	\draw [<->](-6.25,0) -- coordinate (x axis mid) (8.25,0) node[below right] {$x$};
	\draw [-, dashed](-4,1) -- coordinate (y axis mid) (-4,-0.25) node[below] {$-4$};
	\draw [-](0,1) -- coordinate (y axis mid) (0,-0.25) node[below] {$0$};
	\draw [-](4,1) -- coordinate (y axis mid) (4,-0.25) node[below] {$4$};
	\draw [-, dashed](6,1) -- coordinate (y axis mid) (6,-0.25) node[below] {$6$};
	\draw (-5,-1) node {$x=-5$};
	\draw (-2,-1) node {$x=-2$};
	\draw (2,-1) node {$x=2$};
	\draw (5,-1) node {$x=5$};
	\draw (7,-1) node {$x=7$};
	\draw (-5,0.5) node {$-$};
	\draw (-2,0.5) node {$-$};
	\draw (2,0.5) node {$+$};
	\draw (5,0.5) node {$-$};
	\draw (7,0.5) node {$+$};
\end{tikzpicture}

\begin{tikzpicture}[xscale=0.35,yscale=0.35]
	\fill[color=lightgray] (-20,0)--(-4.1,0)--(-4.1,-19)--(-20,-19)--cycle;
	\fill[color=lightgray] (-3.9,0)--(0,0)--(0,-19)--(-3.9,-19)--cycle;
	\fill[color=lightgray] (0,20)--(4,20)--(4,0)--(0,0)--cycle;
	\fill[color=lightgray] (4,0)--(6,0)--(6,-19)--(4,-19)--cycle;
	\fill[color=lightgray] (6,0)--(6,20)--(20,20)--(20,0)--cycle;
	\draw [<->](-21,0) -- coordinate (x axis mid) (21,0) node[below right] {$x$};
	\draw [<->](0,-21) -- coordinate (x axis mid) (0,21) node[above right] {$y$};
	\draw [<->,dashed](6,20) -- (6,-19) node[below] {$x=6$};
	\draw [<->,dashed] plot [domain=-20:20, samples=100] (\x,{0.333*(\x)+0.667});
	\draw [<->] plot [domain=-20:5.815, samples=100] (\x,{(\x*(\x-4))/(3*(\x-6))});
	\draw [<->] plot [domain=6.245:20, samples=100] (\x,{(\x*(\x-4))/(3*(\x-6))});
	\foreach \x in {4,8,...,20}
		\draw (\x,2pt) -- (\x,-2pt)	node[anchor=north] {\scriptsize \x};
	\foreach \x in {-20,-16,...,-4}
		\draw (\x,2pt) -- (\x,-2pt)	node[anchor=south] {\scriptsize \x};
	\foreach \y in {4,8,...,20}
		\draw (2pt,\y) -- (-2pt,\y)	node[anchor=east] {\scriptsize \y}; 
	\foreach \y in {-20,-16,...,-4}
		\draw (2pt,\y) -- (-2pt,\y)	node[anchor=west] {\scriptsize \y}; 
	\draw[fill] (4,0) circle (0.20);
	\draw[fill] (0,0) circle (0.20);
	\draw[fill, color=white] (-4,-1.067) circle (0.20);
	\draw[line width=0.3mm] (-4,-1.067) circle (0.20);
	\draw (16,4) node {$y=\frac{1}{3}x+\frac{2}{3}$};
	\draw (12,-16) node {$f(x)=\dfrac{x^3-16x}{3x^2-2x-24}$};
  \coordinate (v) at (13,4,0);
	\coordinate (w) at (11,4,0);
	\draw [->] (v) to (w);
\end{tikzpicture}
\end{center}
\end{example}
Our graph and, more importantly, its accompanying sign diagram are essential in determining when $f$ is either positive or negative.  For example, one could say that $f(x)>0$ for all $x$ in the set $(0,4)\cup(6,\infty)$, and $f(x)<0$ when $x$ is in the set $(-\infty,-4)\cup(-4,0)\cup(4,6)$.  We will revisit this idea at the beginning of our next section, when we are asked to solve a rational inequality, rather than graph a particular function.\par
While it is essential that we are able to analyze and graph a rational function, it is equally important that we can work backwards, and correctly interpret a graph to identify the key aspects of an otherwise unknown function.  Our last example does just that.
\begin{example}\label{rat_ineq_2}
Find an explicit form for the rational function whose graph is shown below.
\begin{center}
\begin{tikzpicture}[xscale=0.4,yscale=0.6]
	\draw [<->](-15.5,0) -- coordinate (x axis mid) (15.5,0) node[below right] {$x$};
	\draw [<->](0,-7) -- coordinate (x axis mid) (0,7) node[above right] {$y$};
	\draw [<->,dashed](-6,-7) -- (-6,7) node[below] {};
	\draw [<->,dashed](1,-7) -- (1,7) node[below] {};
	\draw [<->,dashed](-15.5,-0.667) -- (15.5,-0.667) node[below] {};
	\draw [<->] plot [domain=-15.5:-6.57, samples=100] (\x,{(-2*(\x+3)*(\x-3)^2)/(3*(\x+6)*(\x-1)^2)});
	\draw [<->] plot [domain=-5.56:0.39, samples=100] (\x,{(-2*(\x+3)*(\x-3)^2)/(3*(\x+6)*(\x-1)^2)});
	\draw [<->] plot [domain=1.395:15.5, samples=100] (\x,{(-2*(\x+3)*(\x-3)^2)/(3*(\x+6)*(\x-1)^2)});
	\foreach \x in {1,2,...,15}
		\draw (\x,2pt) -- (\x,-2pt)	node[anchor=south] {};
	\foreach \x in {-15,-14,...,-1}
		\draw (\x,2pt) -- (\x,-2pt)	node[anchor=south] {};
	\foreach \y in {1,2,...,6}
		\draw (2pt,\y) -- (-2pt,\y)	node[anchor=east] {}; 
	\foreach \y in {-6,-5,...,-1}
		\draw (2pt,\y) -- (-2pt,\y)	node[anchor=east] {}; 
	\foreach \x in {3,6,...,15}
		\draw (\x,2pt) -- (\x,-2pt)	node[anchor=south] {\scriptsize \x};
	\foreach \x in {-15,-12,...,-3}
		\draw (\x,2pt) -- (\x,-2pt)	node[anchor=south] {\scriptsize \x};
	\foreach \y in {1,2,...,6}
		\draw (2pt,\y) -- (-2pt,\y)	node[anchor=east] {\scriptsize \y}; 
	\foreach \y in {-6,-5,...,-1}
		\draw (2pt,\y) -- (-2pt,\y)	node[anchor=east] {\scriptsize \y}; 
	\draw[fill] (3,0) ellipse (0.15 and 0.10);
	\draw[fill] (-3,0) ellipse (0.15 and 0.10);
	\draw[fill] (0,-3) ellipse (0.15 and 0.10);
	\draw (13,-1.5) node {$y=\frac{-2}{3}$};
  \coordinate (v) at (11,-1.5,0);
	\coordinate (w) at (9,-1,0);
	\draw [->] (v) to (w);
\end{tikzpicture}
\end{center}
Although it may be easy to feel overwhelmed at what this problem is asking us to do, at this point we can rest assured that the skills outlined throughout the chapter will enable us to systematically analyze each aspect of the graph above, in order to construct the rational function whose graph matches our graph.
\par
We begin by first making the following observations, in no particular order, along with the implications for our function $f(x)$.\par
\begin{tabular}{lcl}
$y-$intercept at $(0,-3)$ & $\Longrightarrow$ & $f(0)=-3$\\
&&\\
Crossover $x-$intercept at $x=-3$ & $\Longrightarrow$ & $(x+3)^1$ appearing in numerator\\  
&&\\
Turnaround $x-$intercept at $x=3$ & $\Longrightarrow$ & $(x-3)^2$ appearing in numerator\\
&&\\
Vertical asymptote at $x=-6,$ with & $\Longrightarrow$ & $(x+6)^1$ appearing in denominator\\
ends pointing in opposite directions &   & \\
&&\\
Vertical asymptote at $x=1,$ with & $\Longrightarrow$ & $(x-1)^2$ appearing in denominator\\
ends pointing in same direction &   & \\
&&\\
Horizontal asymptote at $y=\frac{-2}{3}$& $\Longrightarrow$ & Numerator and denominator have same degree, $n$\\
&  & $\frac{a_n}{b_n}=\frac{-2}{3}$
\end{tabular}
Since many of our observations involve a {\it factor} of some kind, we will begin construction of $f$ in its factored form.  An expanded form can then be easily obtained by multiplying, if it is required.
\par
We will start by focusing on the implications from our $x-$intercepts and those values not in our domain (corresponding to any vertical asymptotes and holes).  Consequently, an initial candidate for $f$ could be
$$f(x)=\frac{(x+3)(x-3)^2}{(x+6)(x-1)^2}.$$
At this point, we will need to check our candidate against all additional aspects of the graph outlined above, making adjustments as necessary.  But, we soon see that 
$$f(0)=\frac{(0+3)(0-3)^2}{(0+6)(0-1)^2}=\frac{27}{6}\neq -3,$$ as our graph shows.  Additionally, though the degrees of both the numerator and denominator are equal (in this case, three), it is not hard to see that our numerator and denominator will both have a leading coefficient of $1,$ which will not produce the necessary horizontal asymptote at $y=\frac{-2}{3}$.
\par
The solution to this problem is found by making an adjustment to our initial guess, by introducing a {\it scalar multiplier} $k$.
$$f(x)=\frac{k(x+3)(x-3)^2}{(x+6)(x-1)^2}$$
All that remains is to determine what $k$ equals.  Remember that we know it must be negative, in order to produce a horizontal asymptote at $y=-1$.
In order to find the precise multiple that is needed, we must substitute our $y-$intercept into the equation above and solve for $k$. 
\begin{equation*}
	\begin{split}
f(0)	& = \frac{k(0+3)(0-3)^2}{(0+6)(0-1)^2}\\
	& = \frac{27k}{6}\\
	& = \frac{9k}{2}\\
	\end{split}
\end{equation*}
But we know that $f(0)=-3$.  So we are left with having to solve $\dfrac{9k}{2}=-3$.
\par
In this case, we get $k=\dfrac{-2}{3}$.  Our final answer for $f$ is therefore
$$f(x)=\frac{-2(x+3)(x-3)^2}{3(x+6)(x-1)^2},$$
which further satisfies the requirement for our horizontal asymptote at $y=\frac{-2}{3}$.
\par
Expanding our answer gives us $f(x)=\dfrac{-2x^3+6x^2+18x-54}{3x^3+12x^2-33x+18}$.
\end{example}
\newpage
\end{document}