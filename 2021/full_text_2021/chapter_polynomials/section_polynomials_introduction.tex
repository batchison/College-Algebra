\documentclass[12pt]{book}
\raggedbottom
\usepackage[top=1in,left=1in,bottom=1in,right=1in,headsep=0.25in]{geometry}	
\usepackage{amssymb,amsmath,amsthm,amsfonts}
\usepackage{chapterfolder,docmute,setspace}
\usepackage{cancel,multicol,tikz,verbatim,framed,polynom,enumitem,tikzpagenodes}
\usepackage[colorlinks, hyperindex, plainpages=false, linkcolor=blue, urlcolor=blue, pdfpagelabels]{hyperref}
\usepackage[type={CC},modifier={by-sa},version={4.0},]{doclicense}

\theoremstyle{definition}
\newtheorem{example}{Example}
\newcommand{\Desmos}{\href{https://www.desmos.com/}{Desmos}}
\setlength{\parindent}{0in}
\setlist{itemsep=0in}
\setlength{\parskip}{0.1in}
\setcounter{secnumdepth}{0}
% This document is used for ordering of lessons.  If an instructor wishes to change the ordering of assessments, the following steps must be taken:

% 1) Reassign the appropriate numbers for each lesson in the \setcounter commands included in this file.
% 2) Rearrange the \include commands in the master file (the file with 'Course Pack' in the name) to accurately reflect the changes.  
% 3) Rarrange the \items in the measureable_outcomes file to accurately reflect the changes.  Be mindful of page breaks when moving items.
% 4) Re-build all affected files (master file, measureable_outcomes file, and any lessons whose numbering has changed).

%Note: The placement of each \newcounter and \setcounter command reflects the original/default ordering of topics (linears, systems, quadratics, functions, polynomials, rationals).

\newcounter{lesson_solving_linear_equations}
\newcounter{lesson_equations_containing_absolute_values}
\newcounter{lesson_graphing_lines}
\newcounter{lesson_two_forms_of_a_linear_equation}
\newcounter{lesson_parallel_and_perpendicular_lines}
\newcounter{lesson_linear_inequalities}
\newcounter{lesson_compound_inequalities}
\newcounter{lesson_inequalities_containing_absolute_values}
\newcounter{lesson_graphing_systems}
\newcounter{lesson_substitution}
\newcounter{lesson_elimination}
\newcounter{lesson_quadratics_introduction}
\newcounter{lesson_factoring_GCF}
\newcounter{lesson_factoring_grouping}
\newcounter{lesson_factoring_trinomials_a_is_1}
\newcounter{lesson_factoring_trinomials_a_neq_1}
\newcounter{lesson_solving_by_factoring}
\newcounter{lesson_square_roots}
\newcounter{lesson_i_and_complex_numbers}
\newcounter{lesson_vertex_form_and_graphing}
\newcounter{lesson_solve_by_square_roots}
\newcounter{lesson_extracting_square_roots}
\newcounter{lesson_the_discriminant}
\newcounter{lesson_the_quadratic_formula}
\newcounter{lesson_quadratic_inequalities}
\newcounter{lesson_functions_and_relations}
\newcounter{lesson_evaluating_functions}
\newcounter{lesson_finding_domain_and_range_graphically}
\newcounter{lesson_fundamental_functions}
\newcounter{lesson_finding_domain_algebraically}
\newcounter{lesson_solving_functions}
\newcounter{lesson_function_arithmetic}
\newcounter{lesson_composite_functions}
\newcounter{lesson_inverse_functions_definition_and_HLT}
\newcounter{lesson_finding_an_inverse_function}
\newcounter{lesson_transformations_translations}
\newcounter{lesson_transformations_reflections}
\newcounter{lesson_transformations_scalings}
\newcounter{lesson_transformations_summary}
\newcounter{lesson_piecewise_functions}
\newcounter{lesson_functions_containing_absolute_values}
\newcounter{lesson_absolute_as_piecewise}
\newcounter{lesson_polynomials_introduction}
\newcounter{lesson_sign_diagrams_polynomials}
\newcounter{lesson_factoring_quadratic_type}
\newcounter{lesson_factoring_summary}
\newcounter{lesson_polynomial_division}
\newcounter{lesson_synthetic_division}
\newcounter{lesson_end_behavior_polynomials}
\newcounter{lesson_local_behavior_polynomials}
\newcounter{lesson_rational_root_theorem}
\newcounter{lesson_polynomials_graphing_summary}
\newcounter{lesson_polynomial_inequalities}
\newcounter{lesson_rationals_introduction_and_terminology}
\newcounter{lesson_sign_diagrams_rationals}
\newcounter{lesson_horizontal_asymptotes}
\newcounter{lesson_slant_and_curvilinear_asymptotes}
\newcounter{lesson_vertical_asymptotes}
\newcounter{lesson_holes}
\newcounter{lesson_rationals_graphing_summary}

\setcounter{lesson_solving_linear_equations}{1}
\setcounter{lesson_equations_containing_absolute_values}{2}
\setcounter{lesson_graphing_lines}{3}
\setcounter{lesson_two_forms_of_a_linear_equation}{4}
\setcounter{lesson_parallel_and_perpendicular_lines}{5}
\setcounter{lesson_linear_inequalities}{6}
\setcounter{lesson_compound_inequalities}{7}
\setcounter{lesson_inequalities_containing_absolute_values}{8}
\setcounter{lesson_graphing_systems}{9}
\setcounter{lesson_substitution}{10}
\setcounter{lesson_elimination}{11}
\setcounter{lesson_quadratics_introduction}{16}
\setcounter{lesson_factoring_GCF}{17}
\setcounter{lesson_factoring_grouping}{18}
\setcounter{lesson_factoring_trinomials_a_is_1}{19}
\setcounter{lesson_factoring_trinomials_a_neq_1}{20}
\setcounter{lesson_solving_by_factoring}{21}
\setcounter{lesson_square_roots}{22}
\setcounter{lesson_i_and_complex_numbers}{23}
\setcounter{lesson_vertex_form_and_graphing}{24}
\setcounter{lesson_solve_by_square_roots}{25}
\setcounter{lesson_extracting_square_roots}{26}
\setcounter{lesson_the_discriminant}{27}
\setcounter{lesson_the_quadratic_formula}{28}
\setcounter{lesson_quadratic_inequalities}{29}
\setcounter{lesson_functions_and_relations}{12}
\setcounter{lesson_evaluating_functions}{13}
\setcounter{lesson_finding_domain_and_range_graphically}{14}
\setcounter{lesson_fundamental_functions}{15}
\setcounter{lesson_finding_domain_algebraically}{30}
\setcounter{lesson_solving_functions}{31}
\setcounter{lesson_function_arithmetic}{32}
\setcounter{lesson_composite_functions}{33}
\setcounter{lesson_inverse_functions_definition_and_HLT}{34}
\setcounter{lesson_finding_an_inverse_function}{35}
\setcounter{lesson_transformations_translations}{36}
\setcounter{lesson_transformations_reflections}{37}
\setcounter{lesson_transformations_scalings}{38}
\setcounter{lesson_transformations_summary}{39}
\setcounter{lesson_piecewise_functions}{40}
\setcounter{lesson_functions_containing_absolute_values}{41}
\setcounter{lesson_absolute_as_piecewise}{42}
\setcounter{lesson_polynomials_introduction}{43}
\setcounter{lesson_sign_diagrams_polynomials}{44}
\setcounter{lesson_factoring_quadratic_type}{46}
\setcounter{lesson_factoring_summary}{45}
\setcounter{lesson_polynomial_division}{47}
\setcounter{lesson_synthetic_division}{48}
\setcounter{lesson_end_behavior_polynomials}{49}
\setcounter{lesson_local_behavior_polynomials}{50}
\setcounter{lesson_rational_root_theorem}{51}
\setcounter{lesson_polynomials_graphing_summary}{52}
\setcounter{lesson_polynomial_inequalities}{53}
\setcounter{lesson_rationals_introduction_and_terminology}{54}
\setcounter{lesson_sign_diagrams_rationals}{55}
\setcounter{lesson_horizontal_asymptotes}{56}
\setcounter{lesson_slant_and_curvilinear_asymptotes}{57}
\setcounter{lesson_vertical_asymptotes}{58}
\setcounter{lesson_holes}{59}
\setcounter{lesson_rationals_graphing_summary}{60}

\newcommand{\tmmathbf}[1]{\ensuremath{\boldsymbol{#1}}}
\newcommand{\tmop}[1]{\ensuremath{\operatorname{#1}}}

\begin{document}
\section{Introduction and Terminology (L\arabic{lesson_polynomials_introduction})}
\begin{tikzpicture}[remember picture, overlay,shift=(current page text area.north east),scale=0.5]
\draw[step=1.0,gray,very thin,dotted] (-9.8,-7.8) grid (-0.2,1.8);		
\draw[very thick] (-10,-8) -- (-10,2) -- (0,2) -- (0,-8) -- (-10,-8);
\draw[] (-9.8,-7.8) -- (-9.8,1.8) -- (-0.2,1.8) -- (-0.2,-7.8) -- (-9.8,-7.8);
\draw[-] (-9.8,-3) -- coordinate (x axis mid) (-0.2,-3);
\draw[-] (-5,-7.8) -- coordinate (y axis mid) (-5,1.8);
\draw[<->] plot [domain=-9.1:-1.1, samples=100] (\x,{0.125*(\x+2)*(\x+7)^2-3});
\end{tikzpicture}%
{\bf Objective: Identify key features of and classify a polynomial by degree and number of nonzero terms.}\par
A {\it polynomial} in terms of a variable $x$ is a function of the form
$$f(x) = a_{n}x^{n} + a_{n-1}x^{n-1}+ ... + a_{2}x^2 + a_{1}x + a_{0},$$
where each {\it coefficient}, $a_{i}$, is a real number ($a_n\neq 0$) and the exponent, or {\it degree} of the polynomial, $n$, is a nonnegative integer.
\par
Examples of polynomials include: $f(x) = x^2 + 5$, $f(x)=x$ and $f(x) = -3x^7+4x^3-5x$.  Before classifying polynomials, we will take a moment to establish some key terminology. For our general polynomial above, the:
\begin{center}
\begin{tabular}{lcl}
{\it degree} & is & $n$\\
{\it coefficients} & are & $a_n,a_{n-1},\ldots,a_1,a_0$\\
{\it leading coefficient} & is & $a_n$\\
{\it leading term} & is & $a_nx^n$\\
{\it constant term} & is & $a_0x^0=a_0$.
\end{tabular}
\end{center}
A concrete example will help to clarify each of these terms.
\begin{example} Identify the degree, leading coefficient, leading term and constant term for the polynomial
$$f(x) = -19x^5+4x^4-6x+21.$$
The degree of this polynomial is $n=5$, since five is the greatest exponent.
\par
The leading term, which is the term that contains the greatest exponent (degree), is\\ $a_nx^n=-19x^5$.
\par
The leading coefficient is the real number being multiplied by $x^n$ in the leading term, namely $a_n=-19$.
\par
The constant term is $a_0=21$, which also represents the $y-$intercept for the graph of the given polynomial, just as it did in the chapter on quadratics.
\par
The complete set of coefficients for the given polynomial is
$$\{a_5=-19, \ a_4=4, \ a_3=0, \ a_2=0, \ a_1=-6, \ a_0=21\}.$$
\end{example}
It is important to point out the fact that the previous example contains no {\it cubic} or {\it quadratic} terms, since the respective coefficients are both zero.  This example demonstrates that not every polynomial will contain a nonzero coefficient for every term.  As another example, the {\it power function} $f(x)=x^{10}$ is also characterized as a polynomial having degree $n=10$, leading coefficient $a_{10}=1$, and trailing coefficients $a_i=0$ for $i=9,8,\ldots,1,0$.
\par
Before we can identify and begin to classify a polynomial, we may need to simplify the given expression for $x$, by distributing and combining all like terms.  The general form of a polynomial should be reminiscent of the standard form of a quadratic, with possibly more terms.  Hence the name ``polynomial'', meaning ``many terms''.
\par
The following example shows how to identify a polynomial after the necessary simplification has taken place.
\begin{example} Identify the degree, leading coefficient, leading term and constant term for the given polynomial function.
\begin{equation*}
\begin{split}
f(x) &= 3(x+1)(x-1)+4x^3+2x+3\\
& = 3(x^2-1)+4x^3+2x+3\\
& = 3x^2-3+4x^3+2x+3\\
& = 4x^3+3x^2+2x
\end{split}
\end{equation*}
Upon simplifying, we see that $f$ has degree $n=3$, since three is the greatest exponent.
\par
The leading term is $4x^3$ with a leading coefficient of $a_n=4$.
\par
Since no constant term is shown, $a_0=0$ is our constant term.
\end{example}
Now that we can identify the essential components of a polynomial, we will categorize polynomials based upon their degree, as well as the number of terms, after all necessary simplification.
\newpage
\begin{center}
{\bf Types of Polynomials}
\par
\begin{tabular}{ | c | c | c | } 
\hline
Degree & Type & Example \\ 
\hline
0 & Constant & $-1$ \\ 
\hline
1 & Linear & $2x+\sqrt{5}$ \\ 
\hline
2 & Quadratic & $5x^2 - 32x+2$ \\ 
\hline
3 & Cubic & $(-1/2)x^{3}$ \\ 
\hline
4 & Quartic & $-3x^{4} +2x^2+3x + 1$ \\ 
\hline
5  & Quintic & $-2x^5$ \\ 
\hline
6 or more  & $n^{\text{th}}$ Degree & $-2x^{7} + 52x^6 + 12$ \\ 
\hline
\end{tabular}
\end{center}
One point of note in the table above is the appearance of both rational and irrational coefficients $\left(-1/2 \ \text{and} \ \sqrt{5}\right)$.  The appearance of such coefficients is permissible in polynomials, since our coefficients $a_i$ are only required to be real numbers.  A coefficient containing the imaginary number $i=\sqrt{-1}$, on the other hand, is not permitted.
\par
\begin{center}
{\bf Polynomial Characterizations by Number of Nonzero Terms}
\par
\begin{tabular}{ | c | c | c | } 
\hline
Number of Terms & Name & Example \\ 
\hline
1 & Monomial & $4x^5$ \\ 
\hline
2 & Binomial & $2x^3 +1$ \\ 
\hline
3 & Trinomial & $-23x^{18} +4x^2+3x$ \\ 
\hline
4 & Tetranomial & $-23x^{18} +4x^2+3x + 1$ \\ 
\hline
5 or more & Polynomial & $-2x^4 + x^3 +15x^2-41x + 12$ \\ 
\hline
\end{tabular}
\end{center}
\begin{example} Describe the type and characterization (number of terms) of the polynomial function shown below.
$$f(x) = -19x^5+4x^4-6x+21$$
Polynomials are typically named by their degree first and then their number of terms.  The polynomial above is a {\it quintic tetranomial}; quintic because it is degree five and tetranomial because it contains four terms.
\end{example}
\begin{example} Describe the type and characterization (number of terms) of the polynomial function shown below.
$$f(x) = x^3+x^2$$
The polynomial above is a {\it cubic binomial}, since it has degree three and contains two terms. 
\end{example}
\begin{example} Describe the type and characterization (number of terms) of the polynomial function shown below.
$$f(x) = 21x^4+12x^2-3x^2-9x^2-22x^4$$
Upon simplifying, we see that the given polynomial reduces to $f(x)=-x^4$.  As a result, our polynomial is a quartic (degree four) monomial (one term).
\end{example}
This section ``sets the table'' for the basic terminology that will be used throughout the chapter.  In the next section, we will review some additional prerequisite factoring techniques which will be necessary for working with certain polynomials, and provide a brief summary of all factoring methods that have been discussed up to this point.  Once we have finished our review of factoring, we will be ready to begin the natural (albeit lengthy) method of analyzing and graphing a polynomial function.
\end{document}