\documentclass[12pt]{book}
\raggedbottom
\usepackage[top=1in,left=1in,bottom=1in,right=1in,headsep=0.25in]{geometry}	
\usepackage{amssymb,amsmath,amsthm,amsfonts}
\usepackage{chapterfolder,docmute,setspace}
\usepackage{cancel,multicol,tikz,verbatim,framed,polynom,enumitem,tikzpagenodes}
\usepackage[colorlinks, hyperindex, plainpages=false, linkcolor=blue, urlcolor=blue, pdfpagelabels]{hyperref}
\usepackage[type={CC},modifier={by-sa},version={4.0},]{doclicense}

\theoremstyle{definition}
\newtheorem{example}{Example}
\newcommand{\Desmos}{\href{https://www.desmos.com/}{Desmos}}
\setlength{\parindent}{0in}
\setlist{itemsep=0in}
\setlength{\parskip}{0.1in}
\setcounter{secnumdepth}{0}
% This document is used for ordering of lessons.  If an instructor wishes to change the ordering of assessments, the following steps must be taken:

% 1) Reassign the appropriate numbers for each lesson in the \setcounter commands included in this file.
% 2) Rearrange the \include commands in the master file (the file with 'Course Pack' in the name) to accurately reflect the changes.  
% 3) Rarrange the \items in the measureable_outcomes file to accurately reflect the changes.  Be mindful of page breaks when moving items.
% 4) Re-build all affected files (master file, measureable_outcomes file, and any lessons whose numbering has changed).

%Note: The placement of each \newcounter and \setcounter command reflects the original/default ordering of topics (linears, systems, quadratics, functions, polynomials, rationals).

\newcounter{lesson_solving_linear_equations}
\newcounter{lesson_equations_containing_absolute_values}
\newcounter{lesson_graphing_lines}
\newcounter{lesson_two_forms_of_a_linear_equation}
\newcounter{lesson_parallel_and_perpendicular_lines}
\newcounter{lesson_linear_inequalities}
\newcounter{lesson_compound_inequalities}
\newcounter{lesson_inequalities_containing_absolute_values}
\newcounter{lesson_graphing_systems}
\newcounter{lesson_substitution}
\newcounter{lesson_elimination}
\newcounter{lesson_quadratics_introduction}
\newcounter{lesson_factoring_GCF}
\newcounter{lesson_factoring_grouping}
\newcounter{lesson_factoring_trinomials_a_is_1}
\newcounter{lesson_factoring_trinomials_a_neq_1}
\newcounter{lesson_solving_by_factoring}
\newcounter{lesson_square_roots}
\newcounter{lesson_i_and_complex_numbers}
\newcounter{lesson_vertex_form_and_graphing}
\newcounter{lesson_solve_by_square_roots}
\newcounter{lesson_extracting_square_roots}
\newcounter{lesson_the_discriminant}
\newcounter{lesson_the_quadratic_formula}
\newcounter{lesson_quadratic_inequalities}
\newcounter{lesson_functions_and_relations}
\newcounter{lesson_evaluating_functions}
\newcounter{lesson_finding_domain_and_range_graphically}
\newcounter{lesson_fundamental_functions}
\newcounter{lesson_finding_domain_algebraically}
\newcounter{lesson_solving_functions}
\newcounter{lesson_function_arithmetic}
\newcounter{lesson_composite_functions}
\newcounter{lesson_inverse_functions_definition_and_HLT}
\newcounter{lesson_finding_an_inverse_function}
\newcounter{lesson_transformations_translations}
\newcounter{lesson_transformations_reflections}
\newcounter{lesson_transformations_scalings}
\newcounter{lesson_transformations_summary}
\newcounter{lesson_piecewise_functions}
\newcounter{lesson_functions_containing_absolute_values}
\newcounter{lesson_absolute_as_piecewise}
\newcounter{lesson_polynomials_introduction}
\newcounter{lesson_sign_diagrams_polynomials}
\newcounter{lesson_factoring_quadratic_type}
\newcounter{lesson_factoring_summary}
\newcounter{lesson_polynomial_division}
\newcounter{lesson_synthetic_division}
\newcounter{lesson_end_behavior_polynomials}
\newcounter{lesson_local_behavior_polynomials}
\newcounter{lesson_rational_root_theorem}
\newcounter{lesson_polynomials_graphing_summary}
\newcounter{lesson_polynomial_inequalities}
\newcounter{lesson_rationals_introduction_and_terminology}
\newcounter{lesson_sign_diagrams_rationals}
\newcounter{lesson_horizontal_asymptotes}
\newcounter{lesson_slant_and_curvilinear_asymptotes}
\newcounter{lesson_vertical_asymptotes}
\newcounter{lesson_holes}
\newcounter{lesson_rationals_graphing_summary}

\setcounter{lesson_solving_linear_equations}{1}
\setcounter{lesson_equations_containing_absolute_values}{2}
\setcounter{lesson_graphing_lines}{3}
\setcounter{lesson_two_forms_of_a_linear_equation}{4}
\setcounter{lesson_parallel_and_perpendicular_lines}{5}
\setcounter{lesson_linear_inequalities}{6}
\setcounter{lesson_compound_inequalities}{7}
\setcounter{lesson_inequalities_containing_absolute_values}{8}
\setcounter{lesson_graphing_systems}{9}
\setcounter{lesson_substitution}{10}
\setcounter{lesson_elimination}{11}
\setcounter{lesson_quadratics_introduction}{16}
\setcounter{lesson_factoring_GCF}{17}
\setcounter{lesson_factoring_grouping}{18}
\setcounter{lesson_factoring_trinomials_a_is_1}{19}
\setcounter{lesson_factoring_trinomials_a_neq_1}{20}
\setcounter{lesson_solving_by_factoring}{21}
\setcounter{lesson_square_roots}{22}
\setcounter{lesson_i_and_complex_numbers}{23}
\setcounter{lesson_vertex_form_and_graphing}{24}
\setcounter{lesson_solve_by_square_roots}{25}
\setcounter{lesson_extracting_square_roots}{26}
\setcounter{lesson_the_discriminant}{27}
\setcounter{lesson_the_quadratic_formula}{28}
\setcounter{lesson_quadratic_inequalities}{29}
\setcounter{lesson_functions_and_relations}{12}
\setcounter{lesson_evaluating_functions}{13}
\setcounter{lesson_finding_domain_and_range_graphically}{14}
\setcounter{lesson_fundamental_functions}{15}
\setcounter{lesson_finding_domain_algebraically}{30}
\setcounter{lesson_solving_functions}{31}
\setcounter{lesson_function_arithmetic}{32}
\setcounter{lesson_composite_functions}{33}
\setcounter{lesson_inverse_functions_definition_and_HLT}{34}
\setcounter{lesson_finding_an_inverse_function}{35}
\setcounter{lesson_transformations_translations}{36}
\setcounter{lesson_transformations_reflections}{37}
\setcounter{lesson_transformations_scalings}{38}
\setcounter{lesson_transformations_summary}{39}
\setcounter{lesson_piecewise_functions}{40}
\setcounter{lesson_functions_containing_absolute_values}{41}
\setcounter{lesson_absolute_as_piecewise}{42}
\setcounter{lesson_polynomials_introduction}{43}
\setcounter{lesson_sign_diagrams_polynomials}{44}
\setcounter{lesson_factoring_quadratic_type}{46}
\setcounter{lesson_factoring_summary}{45}
\setcounter{lesson_polynomial_division}{47}
\setcounter{lesson_synthetic_division}{48}
\setcounter{lesson_end_behavior_polynomials}{49}
\setcounter{lesson_local_behavior_polynomials}{50}
\setcounter{lesson_rational_root_theorem}{51}
\setcounter{lesson_polynomials_graphing_summary}{52}
\setcounter{lesson_polynomial_inequalities}{53}
\setcounter{lesson_rationals_introduction_and_terminology}{54}
\setcounter{lesson_sign_diagrams_rationals}{55}
\setcounter{lesson_horizontal_asymptotes}{56}
\setcounter{lesson_slant_and_curvilinear_asymptotes}{57}
\setcounter{lesson_vertical_asymptotes}{58}
\setcounter{lesson_holes}{59}
\setcounter{lesson_rationals_graphing_summary}{60}

\newcommand{\tmmathbf}[1]{\ensuremath{\boldsymbol{#1}}}
\newcommand{\tmop}[1]{\ensuremath{\operatorname{#1}}}

\begin{document}
\section{End Behavior (L\arabic{lesson_end_behavior_polynomials})}
{\bf Objective: Identify and describe the end behavior of the graph of a polynomial function.}\par
The {\it end behavior} of any function refers to what happens near the extreme ends of its graph.  We also often refer to these as the ``tails'' of the graph.  The ends of the graph of a function correspond to points having large positive or negative $x-$coordinates.  Because of this, we can associate the expressions
$$x\rightarrow\infty\qquad\text{and}\qquad x\rightarrow -\infty$$
to the end behavior of a function.  For example, the sentence
\begin{center}
As $x\rightarrow\infty,$ \ $f(x)\rightarrow\infty$.
\end{center}
describes a function for which the right-hand side of its graph, i.e. when $x\rightarrow\infty,$ points upward.  Alternatively, the sentence  
\begin{center}
As $x\rightarrow\infty,$ \ $f(x)\rightarrow -\infty$.
\end{center}
describes a function for which the right-hand side of its graph points downward.
\par
In each of the above mathematical statements, we are identifying both a horizontal direction and a vertical direction:
\begin{enumerate}
	\item  the independent variable $x$ getting large (either positively or negatively),
	\item and the effect this has on the values of $f(x)$.
\end{enumerate}

\begin{example}
Describe the end behavior of the function $f$ whose graph is shown below.
\begin{multicols}{2}
\begin{center}
\begin{tikzpicture}[scale=0.85]
	\draw [<->](-4,0) -- coordinate (x axis mid) (4,0) node[below right] {$x$};
	\draw [<->](0,-1) -- coordinate (y axis mid) (0,4.25) node[above right] {$y$};
	\draw [<->] plot [domain=-2:3.25,samples=100] (\x,{0.5^(\x)});
	\foreach \y in {1,2,3,4} \draw (2pt,\y) -- (-2pt,\y)	node[anchor=east] {\scriptsize \y};							
	\foreach \x in {-3,-2,-1} \draw (\x,2pt) -- (\x,-2pt)	node[anchor=north] {\scriptsize \x};							
	\foreach \x in {1,2,3} \draw (\x,2pt) -- (\x,-2pt)	node[anchor=north] {\scriptsize \x};							
	\draw (-3,3) node {$y=f(x)$};
\end{tikzpicture}
\end{center}

\columnbreak

Although this graph is one that we typically see in a precalculus setting (known as an exponential function), we can still discuss its end behavior.  In this case, as the values of $x$ increase, we see that the points on the graph approach the $x-$axis.  This translates to the following statement.
\begin{center}
As $x\rightarrow\infty,$ \ $f(x)\rightarrow 0$.
\end{center}
\end{multicols}
On the other hand, as the values of $x$ tend towards $-\infty,$ we see that the $y-$coordinates for their respective points continue to increase.  Hence, we can say the following.
\begin{center}
As $x\rightarrow -\infty,$ \ $f(x)\rightarrow \infty$.
\end{center}
\end{example}

Prior to this chapter, we have not had much need to discuss end behavior at great length, since most of the functions which we have been exposed to have been relatively easily diagnosed and graphed.  A quadratic function, $f(x)=ax^2+bx+c$, for example, will either open up or down, depending on the sign of the leading coefficient, $a$.  As we begin to graph polynomials, however, we will see our graphs take more than a few turns, which will require us to have a better understanding about the nature of their tails.
\par
For each algebraic function, the corresponding graph will describe two such statements: one for the left-hand side of the graph ($x\rightarrow -\infty$) and one for the right-hand side of the graph ($x\rightarrow\infty$).  In the case of polynomials, there are only four cases for these two statements, summarized as follows.\par
\framebox{
\begin{minipage}{1\linewidth}
Let $$f(x) = a_{n}x^{n} + a_{n-1}x^{n-1}+ ... + a_{2}x^2 + a_{1}x + a_{0}$$ be a polynomial function with degree $n$ and nonzero leading coefficient $a_n$.\\
\par
The end behavior of $f$ is described by one of the following four cases.

\begin{center}
\begin{multicols}{2}
\begin{tikzpicture}[xscale=0.45,yscale=0.45]
	\draw [<->](-4,0) -- coordinate (x axis mid) (4,0) node[below right] {$x$};
	\draw [<->](0,-4) -- coordinate (y axis mid) (0,4) node[above right] {$y$};
	\draw [->] plot [domain=2.5:3.75, samples=100] (\x,{\x^2/4});
	\draw [->] plot [domain=-2.5:-3.75, samples=100] (\x,{-\x^2/4});
	\draw (-4,6.5) node {I. $n$ even, $a_n>0$};
	\draw (0,-5) node {As $x\rightarrow\infty,\ f(x)\rightarrow\infty$};
	\draw (0,-7) node {As $x\rightarrow -\infty,\ f(x)\rightarrow\infty$};
\end{tikzpicture}

\begin{tikzpicture}[xscale=0.45,yscale=0.45]
	\draw [<->](-4,0) -- coordinate (x axis mid) (4,0) node[below right] {$x$};
	\draw [<->](0,-4) -- coordinate (y axis mid) (0,4) node[above right] {$y$};
	\draw [->] plot [domain=2.5:3.75, samples=100] (\x,{-\x^2/4});
	\draw [->] plot [domain=-2.5:-3.75, samples=100] (\x,{\x^2/4});
	\draw (-4,6.5) node {II. $n$ even, $a_n<0$};
	\draw (0,-5) node {As $x\rightarrow\infty,\ f(x)\rightarrow -\infty$};
	\draw (0,-7) node {As $x\rightarrow -\infty,\ f(x)\rightarrow -\infty$};
\end{tikzpicture}
\end{multicols}
\end{center}

\begin{center}\begin{multicols}{2}
\begin{tikzpicture}[xscale=0.45,yscale=0.45]
	\draw [<->](-4,0) -- coordinate (x axis mid) (4,0) node[below right] {$x$};
	\draw [<->](0,-4) -- coordinate (y axis mid) (0,4) node[above right] {$y$};
	\draw [->] plot [domain=2.5:3.75, samples=100] (\x,{\x^2/4});
	\draw [->] plot [domain=-2.5:-3.75, samples=100] (\x,{\x^2/4});
	\draw (-4,6.5) node {III. $n$ odd, $a_n>0$};
	\draw (0,-5) node {As $x\rightarrow\infty,\ f(x)\rightarrow \infty$};
	\draw (0,-7) node {As $x\rightarrow -\infty,\ f(x)\rightarrow -\infty$};
\end{tikzpicture}

\begin{tikzpicture}[xscale=0.45,yscale=0.45]
	\draw [<->](-4,0) -- coordinate (x axis mid) (4,0) node[below right] {$x$};
	\draw [<->](0,-4) -- coordinate (y axis mid) (0,4) node[above right] {$y$};
	\draw [->] plot [domain=2.5:3.75, samples=100] (\x,{-\x^2/4});
	\draw [->] plot [domain=-2.5:-3.75, samples=100] (\x,{-\x^2/4});
	\draw (-4,6.5) node {IV. $n$ odd, $a_n<0$};
	\draw (0,-5) node {As $x\rightarrow\infty,\ f(x)\rightarrow -\infty$};
	\draw (0,-7) node {As $x\rightarrow -\infty,\ f(x)\rightarrow \infty$};
\end{tikzpicture}
\end{multicols}
\end{center}
\end{minipage}
}

An important initial observation of the previous figure is that the cases for the end behavior of a polynomial only depend on its leading term, $a_nx^n$.  More specifically, the end behavior of a polynomial depends only on the parity of its degree $n$ (even or odd) and the sign of its leading coefficient $a_n$ (positive or negative).  Additionally, we can see that cases I and II also include all quadratic functions (when $n=2$).
\par
Identifying the end behavior for an expanded polynomial, is much more straightforward than for a factored polynomial, as we will see in our next example.
\begin{example}
Determine the end behavior of each of the following functions.
\begin{multicols}{2}
\begin{enumerate}
	\item $f(x)=1-3x^4$
	\item $g(x)=-2x^3+10000x^2+1000$
	\item $h(x)=x(2x-1)(x-5)^2$
	\item $k(x)=-2(1-3x)^2(x+1)(x-1)(x^2+1)$
\end{enumerate}
\end{multicols}
\begin{enumerate}
\item The polynomial $f(x)=1-3x^4$ is in expanded form, though not written in descending-power order.  We can easily re-write $f$ as $f(x)=-3x^4+1$.  In this case, the degree $n=4$ is even, and the leading coefficient $a_n=-3$ is negative.  Hence, we are in case II:
\begin{center}
As $x\rightarrow -\infty, \ f(x)\rightarrow -\infty.$ \hspace{1in} As $x\rightarrow \infty, \ f(x)\rightarrow -\infty.$
\end{center}
\item The polynomial $g$ is also in expanded form, with odd degree $n=3$ and negative leading coefficient, $a_n=-2$.  The ``large'' quadratic and  constant terms will not affect the end behavior of $g,$ and so we are in case IV:
\begin{center}
As $x\rightarrow -\infty, \ g(x)\rightarrow \infty.$ \hspace{1in} As $x\rightarrow \infty, \ g(x)\rightarrow -\infty.$
\end{center}
\item The polynomial $h$ is written in factored form, which is helpful for identifying roots/$x-$intercepts, but not necessarily for describing the tails of the graph of $h$.  Although we could, with enough time, expand $h$ completely to describe the function's end behavior, this will quickly prove to be an inefficient strategy.  Recall, however, that for end behavior we need only focus on finding the leading term, $a_nx^n$.  We do this by identifying any parts of $h$ that will contribute to the leading term.  In this case, we identify in boldface font all contributing components to the leading term of $h$ below.
$$h(x)=\mathbf{x}(\mathbf{2x}-1)(\mathbf{x}-5)^{\mathbf{2}}$$
So, when we expand, the leading term of $h$ will be
$$a_nx^n=x(2x)(x)^2=2x^4.$$
We can now see that $h$ has even degree $n=4,$ and positive leading coefficient, $a_n=2$.  Hence, we are in case I:
\begin{center}
As $x\rightarrow -\infty, \ h(x)\rightarrow \infty.$ \hspace{1in} As $x\rightarrow \infty, \ h(x)\rightarrow \infty.$
\end{center}
\item Similarly, the polynomial $k$ is written in factored form, and will require us to find the leading term, $a_nx^n$.  Again, we identify all contributing components to the leading term of $k$ in boldface font below. 
$$k(x)=\mathbf{-2}(1\mathbf{-3x})^{\mathbf{2}}(\mathbf{x}+1)(\mathbf{x}-1)(\mathbf{x^2}+1)$$
So, when we expand, the leading term of $k$ will be
\begin{equation*}
\begin{split}
a_nx^n & =-2(-3x)^2(x)(x)(x^2)\\
& = -2(9x^2)(x^4)\\
& = -18x^6.
\end{split}
\end{equation*}
Through careful analysis, we see that $k$ has an even degree, $n=6$, and a negative leading coefficient, $a_n=-18$.  Hence, we are in case II:
\begin{center}
As $x\rightarrow -\infty, \ k(x)\rightarrow -\infty.$ \hspace{1in} As $x\rightarrow \infty, \ k(x)\rightarrow -\infty.$
\end{center}
\end{enumerate}
\end{example}
In the previous example, we witnessed a new technique to quickly identify the end behavior of a polynomial that is given in factored form. 
The idea behind this technique is to only focus on the contributing components to a polynomial's leading term, $a_nx^n,$ ignoring all others.  This essentially boils down to focusing on three things:
 \begin{itemize}
		\item any constant multiplier,
		\item the leading term of each factor,
		\item and the power associated with each factor.
	\end{itemize}
In general, if we suppose that a polynomial $f$ has the factorization
\begin{center}
$f(x)=c\cdot(\text{Factor 1})^{k_1}\cdot(\text{Factor 2})^{k_2}\cdot\ldots\cdot(\text{Factor m})^{k_m},$
\end{center}
then the leading term for $f$ will equal
\begin{center}
$a_nx^n=c\cdot(\text{Leading Term 1})^{k_1}\cdot(\text{Leading Term 2})^{k_2}\cdot\ldots\cdot(\text{Leading Term m})^{k_m}.$
\end{center}
Note that ``Leading Term 1'' refers to the leading term of Factor 1, and so on for the other factors.
\par
This approach is similar to one that we have likely seen for identifying the constant term for a factored polynomial.  To identify the constant term, $a_0$ of $f$, we would have
\begin{center}
$a_nx^n=c\cdot(\text{Constant Term 1})^{k_1}\cdot(\text{Constant Term 2})^{k_2}\cdot\ldots\cdot(\text{Constant Term m})^{k_m}.$
\end{center}
\begin{example}\label{poly_end_beh_1}
Find the leading and constant terms for the given function, and use them to identify the end behavior and $y-$intercept of its graph.
$$f(x)=3(-2x+1)^2(x-2)^2(x-5)$$
First, we boldface the contributors for the leading term.
$$f(x)=\mathbf{3}(\mathbf{-2x}+1)^{\mathbf{2}}(\mathbf{x}-2)^{\mathbf{2}}(\mathbf{x}-5)$$
This gives us the following.
\begin{equation*}
\begin{split}
a_nx^n & =3(-2x)^{2}(x)^{2}(x)\\
& = 3(4x^2)x^3\\
& = 12x^5
\end{split}
\end{equation*}
Next, we boldface the contributors for the constant term.
$$f(x)=\mathbf{3}(-2x\mathbf{+1})^{\mathbf{2}}(x\mathbf{-2})^{\mathbf{2}}(x\mathbf{-5})$$
This gives us the following.
\begin{equation*}
\begin{split}
a_0 & = 3(1)^{2}(-2)^{2}(-5)\\
& = 3(1)(4)(-5)\\
& = -60
\end{split}
\end{equation*}
Hence, we have that
$$f(x)=12x^5+\ldots +(-60),$$
with middle terms unknown.
\par
Since our degree, $n=5,$ is odd, and our leading coefficient, $a_n=12,$ is positive, we are in case III for end behavior.
\begin{center}
As $x\rightarrow -\infty, \ f(x)\rightarrow -\infty.$ \hspace{1in} As $x\rightarrow \infty, \ f(x)\rightarrow +\infty.$
\end{center}
Our constant term also tells us that the graph of $f$ has a $y-$intercept at $(0,-60)$.
\end{example}
It is natural to ask why the additional terms of a polynomial have no impact on its end behavior.  To address this, let us consider factoring out the leading term from $f$, which will give us the following.
\begin{equation*}
\begin{split}
f(x) & = a_{n}x^{n} + a_{n-1}x^{n-1}+ ... + a_{2}x^2 + a_{1}x + a_{0}\\
 & = a_{n} x^{n} \left( 1 + \frac{a_{n-1}}{a_{n} x}+ \ldots + \frac{a_{2}}{a_{n} x^{n-2}} + \frac{a_1}{a_{n} x^{n-1}}+\frac{a_{0}}{a_{n} x^{n}}\right)
\end{split}
\end{equation*}
If we use $g(x)$ to denote the expression in parentheses,
$$g(x)=1 + \frac{a_{n-1}}{a_{n} x}+ \ldots + \frac{a_{2}}{a_{n} x^{n-2}} + \frac{a_1}{a_{n} x^{n-1}}+\frac{a_{0}}{a_{n} x^{n}},$$
then
\begin{equation*}
\begin{split}
f(x) & = a_{n} x^{n} \underbrace{\left( 1 + \frac{a_{n-1}}{a_{n} x}+ \ldots + \frac{a_{2}}{a_{n} x^{n-2}} + \frac{a_1}{a_{n} x^{n-1}}+\frac{a_{0}}{a_{n} x^{n}}\right)}_{g(x)}\\
& = a_{n} x^{n} \cdot g(x).
\end{split}
\end{equation*}
But recall that the end behavior of a polynomial is determined when $x\rightarrow\pm\infty$.  So, as $x$ gets large (either positively or negatively), with the exception of the first term, all subsequent terms in the expression for $g$ will approach zero.
\begin{center}
As $x\rightarrow\pm\infty,$ \ $g(x)=1 + \cancelto{0}{\frac{a_{n-1}}{\ a_{n} x \ }}+ \ldots + \cancelto{0}{\frac{a_{2}}{a_{n} x^{n-2}}} + \cancelto{0}{\frac{a_1}{a_{n} x^{n-1}}}+\cancelto{0}{\frac{a_{0}}{\ a_{n} x^{n}}}\rightarrow 1.$
\end{center}
Therefore, since $g(x)$ approaches 1, $f(x)= a_{n} x^{n} \cdot g(x)$ will approach its leading term, $a_{n} x^{n}$.  Hence, we conclude that the end behavior of a polynomial $f$ will coincide with the end behavior of its leading term.
\par
Furthermore, for any polynomial $f(x),$ if we were to graph the two curves $y=f(x)$ and $y=a_nx^n$ using \Desmos \ or another graphing utility, and continue to `zoom out', the two graphs would become virtually indistinguishable from one another.  We demonstrate this in our next example.
\begin{example}
Determine the end behavior of the polynomial function below, and graph both the function and its leading term on a single set of axes.
$$f(x)=-x^3+3x-2$$
The leading term of $f$ is $-x^3,$ with odd degree, $n=3,$ and negative leading coefficient, $a_n=-1$.  Hence, we are in case IV:
\begin{center}
As $x\rightarrow -\infty, \ f(x)\rightarrow \infty.$ \hspace{1in} As $x\rightarrow \infty, \ f(x)\rightarrow -\infty.$
\begin{center}
\begin{tikzpicture}[xscale=0.75,yscale=0.15]
	\draw [<->](-5.25,0) -- coordinate (x axis mid) (5.25,0) node[below right] {$x$};
	\draw [<->](0,-26) -- coordinate (x axis mid) (0,26) node[above right] {$y$};
	\draw [<->] plot [domain=-3.25:3.2, samples=100] (\x,{-\x^3+3*\x-2});
	\draw [<->, dashed, line width=0.5mm] plot [domain=-2.9:2.9, samples=100] (\x,{-\x^3});
	\foreach \x in {1,2,...,5}
		\draw (\x,2pt) -- (\x,-2pt)	node[anchor=south] {\scriptsize \x};
	\foreach \x in {-5,-4,...,-1}
		\draw (\x,2pt) -- (\x,-2pt)	node[anchor=north] {\scriptsize \x};
	\foreach \y in {5,10,...,25}
		\draw (2pt,\y) -- (-2pt,\y)	node[anchor=east] {\scriptsize \y}; 
	\foreach \y in {-25,-20,...,-5}
		\draw (2pt,\y) -- (-2pt,\y)	node[anchor=east] {\scriptsize \y}; 
	\draw (-1.75,20) node {\scriptsize $y=-x^3$};
	\draw (-4.75,5) node {\scriptsize $f(x)=-x^3+3x-2$};
\end{tikzpicture}
\end{center}
\end{center}
\end{example}
\end{document}