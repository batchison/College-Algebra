\documentclass[12pt]{book}
\raggedbottom
\usepackage[top=1in,left=1in,bottom=1in,right=1in,headsep=0.25in]{geometry}	
\usepackage{amssymb,amsmath,amsthm,amsfonts}
\usepackage{chapterfolder,docmute,setspace}
\usepackage{cancel,multicol,tikz,verbatim,framed,polynom,enumitem,tikzpagenodes}
\usepackage[colorlinks, hyperindex, plainpages=false, linkcolor=blue, urlcolor=blue, pdfpagelabels]{hyperref}
\usepackage[type={CC},modifier={by-sa},version={4.0},]{doclicense}

\theoremstyle{definition}
\newtheorem{example}{Example}
\newcommand{\Desmos}{\href{https://www.desmos.com/}{Desmos}}
\setlength{\parindent}{0in}
\setlist{itemsep=0in}
\setlength{\parskip}{0.1in}
\setcounter{secnumdepth}{0}
% This document is used for ordering of lessons.  If an instructor wishes to change the ordering of assessments, the following steps must be taken:

% 1) Reassign the appropriate numbers for each lesson in the \setcounter commands included in this file.
% 2) Rearrange the \include commands in the master file (the file with 'Course Pack' in the name) to accurately reflect the changes.  
% 3) Rarrange the \items in the measureable_outcomes file to accurately reflect the changes.  Be mindful of page breaks when moving items.
% 4) Re-build all affected files (master file, measureable_outcomes file, and any lessons whose numbering has changed).

%Note: The placement of each \newcounter and \setcounter command reflects the original/default ordering of topics (linears, systems, quadratics, functions, polynomials, rationals).

\newcounter{lesson_solving_linear_equations}
\newcounter{lesson_equations_containing_absolute_values}
\newcounter{lesson_graphing_lines}
\newcounter{lesson_two_forms_of_a_linear_equation}
\newcounter{lesson_parallel_and_perpendicular_lines}
\newcounter{lesson_linear_inequalities}
\newcounter{lesson_compound_inequalities}
\newcounter{lesson_inequalities_containing_absolute_values}
\newcounter{lesson_graphing_systems}
\newcounter{lesson_substitution}
\newcounter{lesson_elimination}
\newcounter{lesson_quadratics_introduction}
\newcounter{lesson_factoring_GCF}
\newcounter{lesson_factoring_grouping}
\newcounter{lesson_factoring_trinomials_a_is_1}
\newcounter{lesson_factoring_trinomials_a_neq_1}
\newcounter{lesson_solving_by_factoring}
\newcounter{lesson_square_roots}
\newcounter{lesson_i_and_complex_numbers}
\newcounter{lesson_vertex_form_and_graphing}
\newcounter{lesson_solve_by_square_roots}
\newcounter{lesson_extracting_square_roots}
\newcounter{lesson_the_discriminant}
\newcounter{lesson_the_quadratic_formula}
\newcounter{lesson_quadratic_inequalities}
\newcounter{lesson_functions_and_relations}
\newcounter{lesson_evaluating_functions}
\newcounter{lesson_finding_domain_and_range_graphically}
\newcounter{lesson_fundamental_functions}
\newcounter{lesson_finding_domain_algebraically}
\newcounter{lesson_solving_functions}
\newcounter{lesson_function_arithmetic}
\newcounter{lesson_composite_functions}
\newcounter{lesson_inverse_functions_definition_and_HLT}
\newcounter{lesson_finding_an_inverse_function}
\newcounter{lesson_transformations_translations}
\newcounter{lesson_transformations_reflections}
\newcounter{lesson_transformations_scalings}
\newcounter{lesson_transformations_summary}
\newcounter{lesson_piecewise_functions}
\newcounter{lesson_functions_containing_absolute_values}
\newcounter{lesson_absolute_as_piecewise}
\newcounter{lesson_polynomials_introduction}
\newcounter{lesson_sign_diagrams_polynomials}
\newcounter{lesson_factoring_quadratic_type}
\newcounter{lesson_factoring_summary}
\newcounter{lesson_polynomial_division}
\newcounter{lesson_synthetic_division}
\newcounter{lesson_end_behavior_polynomials}
\newcounter{lesson_local_behavior_polynomials}
\newcounter{lesson_rational_root_theorem}
\newcounter{lesson_polynomials_graphing_summary}
\newcounter{lesson_polynomial_inequalities}
\newcounter{lesson_rationals_introduction_and_terminology}
\newcounter{lesson_sign_diagrams_rationals}
\newcounter{lesson_horizontal_asymptotes}
\newcounter{lesson_slant_and_curvilinear_asymptotes}
\newcounter{lesson_vertical_asymptotes}
\newcounter{lesson_holes}
\newcounter{lesson_rationals_graphing_summary}

\setcounter{lesson_solving_linear_equations}{1}
\setcounter{lesson_equations_containing_absolute_values}{2}
\setcounter{lesson_graphing_lines}{3}
\setcounter{lesson_two_forms_of_a_linear_equation}{4}
\setcounter{lesson_parallel_and_perpendicular_lines}{5}
\setcounter{lesson_linear_inequalities}{6}
\setcounter{lesson_compound_inequalities}{7}
\setcounter{lesson_inequalities_containing_absolute_values}{8}
\setcounter{lesson_graphing_systems}{9}
\setcounter{lesson_substitution}{10}
\setcounter{lesson_elimination}{11}
\setcounter{lesson_quadratics_introduction}{16}
\setcounter{lesson_factoring_GCF}{17}
\setcounter{lesson_factoring_grouping}{18}
\setcounter{lesson_factoring_trinomials_a_is_1}{19}
\setcounter{lesson_factoring_trinomials_a_neq_1}{20}
\setcounter{lesson_solving_by_factoring}{21}
\setcounter{lesson_square_roots}{22}
\setcounter{lesson_i_and_complex_numbers}{23}
\setcounter{lesson_vertex_form_and_graphing}{24}
\setcounter{lesson_solve_by_square_roots}{25}
\setcounter{lesson_extracting_square_roots}{26}
\setcounter{lesson_the_discriminant}{27}
\setcounter{lesson_the_quadratic_formula}{28}
\setcounter{lesson_quadratic_inequalities}{29}
\setcounter{lesson_functions_and_relations}{12}
\setcounter{lesson_evaluating_functions}{13}
\setcounter{lesson_finding_domain_and_range_graphically}{14}
\setcounter{lesson_fundamental_functions}{15}
\setcounter{lesson_finding_domain_algebraically}{30}
\setcounter{lesson_solving_functions}{31}
\setcounter{lesson_function_arithmetic}{32}
\setcounter{lesson_composite_functions}{33}
\setcounter{lesson_inverse_functions_definition_and_HLT}{34}
\setcounter{lesson_finding_an_inverse_function}{35}
\setcounter{lesson_transformations_translations}{36}
\setcounter{lesson_transformations_reflections}{37}
\setcounter{lesson_transformations_scalings}{38}
\setcounter{lesson_transformations_summary}{39}
\setcounter{lesson_piecewise_functions}{40}
\setcounter{lesson_functions_containing_absolute_values}{41}
\setcounter{lesson_absolute_as_piecewise}{42}
\setcounter{lesson_polynomials_introduction}{43}
\setcounter{lesson_sign_diagrams_polynomials}{44}
\setcounter{lesson_factoring_quadratic_type}{46}
\setcounter{lesson_factoring_summary}{45}
\setcounter{lesson_polynomial_division}{47}
\setcounter{lesson_synthetic_division}{48}
\setcounter{lesson_end_behavior_polynomials}{49}
\setcounter{lesson_local_behavior_polynomials}{50}
\setcounter{lesson_rational_root_theorem}{51}
\setcounter{lesson_polynomials_graphing_summary}{52}
\setcounter{lesson_polynomial_inequalities}{53}
\setcounter{lesson_rationals_introduction_and_terminology}{54}
\setcounter{lesson_sign_diagrams_rationals}{55}
\setcounter{lesson_horizontal_asymptotes}{56}
\setcounter{lesson_slant_and_curvilinear_asymptotes}{57}
\setcounter{lesson_vertical_asymptotes}{58}
\setcounter{lesson_holes}{59}
\setcounter{lesson_rationals_graphing_summary}{60}

\newcommand{\tmmathbf}[1]{\ensuremath{\boldsymbol{#1}}}
\newcommand{\tmop}[1]{\ensuremath{\operatorname{#1}}}

\begin{document}
\section{The Rational Root Theorem (L\arabic{lesson_rational_root_theorem})}
{\bf Objective: Apply the Rational Root Theorem to determine a set of possible rational roots for and a factorization of a given polynomial.}\par
The Rational Root Theorem is used to identify a list of all possible rational roots for a given polynomial.
\begin{center}
\framebox{
\begin{minipage}{0.9\linewidth}
{\bf Rational Root Theorem:}  Suppose $f(x) = a_{n} x^{n} + a_{n-\mbox{\tiny$1$}}x^{n-\mbox{\tiny$1$}} + \ldots + a_{\mbox{\tiny$1$}} x + a_{\mbox{\tiny$0$}}$ is a polynomial of degree $n$ with $n \geq 1$, and $a_{\mbox{\tiny$0$}}$, $a_{\mbox{\tiny$1$}}$, \ldots $a_{n}$ are integers.  If $r$ is a rational root of $f$, then $r$ is of the form $\pm \frac{p}{q}$, where $p$ is a factor of the constant term $a_{\mbox{\tiny$0$}}$, and $q$ is a factor of the leading coefficient $a_{n}$.
\end{minipage}
}
\end{center}
The Rational Root Theorem gives us a list of numbers to test as roots of a given polynomial using synthetic division, which is a nicer approach than simply guessing at possible roots.  If none of the numbers in the list turn out to be roots, then either the polynomial has no real roots at all, or all of the real roots will be irrational numbers.\par
\begin{example}~~Let $f(x) = 2x^4+4x^3-x^2-6x-3$. Use the Rational Root Theorem to list all of the possible rational roots of $f$.\par
To generate a complete list of rational roots, we need to take each of the factors of the constant term, $a_{\mbox{\tiny$0$}} = -3$, and divide them by each of the factors of the leading coefficient $a_{\mbox{\tiny$4$}} = 2$.\par
The factors of $-3$ are $\pm \, 1$ and $\pm \, 3$.  Since the Rational Root Theorem tacks on a $\pm$ anyway, for the moment, we consider only the positive factors $1$ and $3$.  The factors of $2$ are  $1$ and $2$, so the Rational Root Theorem gives the list $\left\{\pm \, \frac{1}{1}, \pm \, \frac{1}{2},  \pm \, \frac{3}{1}, \pm \, \frac{3}{2}\right\}$ or $\left\{\pm \, \frac{1}{2}, \pm \, 1, \pm \, \frac{3}{2}, \pm \, 3\right\}$.\par
Additionally, we can evaluate $f$ at each of the eight potential rational roots in our list, to see if any of them are indeed roots.  Starting with $\pm 1,$ we see that 
\begin{center}
$f(1)=2+4-1-6-3=-4\neq 0\qquad$ and $\qquad f(-1)=2-4-1+6-3=0.$
\end{center}
\begin{multicols}{2}
Hence, we can conclude that $x=-1$ is a root of $f$ and $x=1$ is not.  Using synthetic division, we can then divide $f$ by the linear factor $x+1$ as follows.
\columnbreak
\begin{center}
$\polyhornerscheme[x=-1,showbase=top,resultstyle=\bf]{2x^4+4x^3-x^2-6x-3}$
\end{center}
\end{multicols}
We can then begin to factor $f,$
$$2x^4+4x^3-x^2-6x-3=(x+1)(2x^3+2x^2-3x-3)$$
The resulting quotient polynomial is then factorable by grouping,
$$2x^3+2x^2-3x-3=(2x^2-3)(x+1).$$
Factoring out a $2$ from the expression $2x^2-3,$ allows us to factor it as the difference of two squares,
\begin{eqnarray*}
2x^2-3 & = & 2\left(x^2-\frac{3}{2}\right)\\
& = & 2\left(x-\sqrt{\frac{3}{2}}\right)\left(x+\sqrt{\frac{3}{2}}\right)\\
& = & 2\left(x-\frac{\sqrt{6}}{2}\right)\left(x+\frac{\sqrt{6}}{2}\right)\\
\end{eqnarray*}
So, a complete factorization for $f$ would be
$$2x^4+4x^3-x^2-6x-3=2\left(x-\frac{\sqrt{6}}{2}\right)\left(x+\frac{\sqrt{6}}{2}\right)\left(x+1\right)^2,$$
and the set of real roots for $f$ is $\left\{-1,\pm\dfrac{\sqrt{6}}{2}\right\}$.
\end{example}
\end{document}