\documentclass[12pt]{book}
\raggedbottom
\usepackage[top=1in,left=1in,bottom=1in,right=1in,headsep=0.25in]{geometry}	
\usepackage{amssymb,amsmath,amsthm,amsfonts}
\usepackage{chapterfolder,docmute,setspace}
\usepackage{cancel,multicol,tikz,verbatim,framed,polynom,enumitem,tikzpagenodes}
\usepackage[colorlinks, hyperindex, plainpages=false, linkcolor=blue, urlcolor=blue, pdfpagelabels]{hyperref}
\usepackage[type={CC},modifier={by-sa},version={4.0},]{doclicense}

\theoremstyle{definition}
\newtheorem{example}{Example}
\newcommand{\Desmos}{\href{https://www.desmos.com/}{Desmos}}
\setlength{\parindent}{0in}
\setlist{itemsep=0in}
\setlength{\parskip}{0.1in}
\setcounter{secnumdepth}{0}
\input{lesson_order}

\newcommand{\tmmathbf}[1]{\ensuremath{\boldsymbol{#1}}}
\newcommand{\tmop}[1]{\ensuremath{\operatorname{#1}}}

\begin{document}
\section{Sign Diagrams (L\arabic{lesson_sign_diagrams_polynomials})}
{\bf Objective: Construct a sign diagram for a given polynomial expression.}\par
If a polynomial function or expression is completely factored, it will be beneficial to us to construct a sign diagram for the polynomial, in order to answer questions about its graph and confirm any other findings.  Therefore, we devote this section to the construction of a sign diagram for a factored polynomial.  Note that expanded polynomials first require us to find a complete factorization prior to constructing a sign diagram.  This will require us to first employ factoring techniques and possibly polynomial division, which we reserve for a later section.
\par
Recall that the roots of a quadratic expression represent the dividers in its corresponding sign diagram.  This carries over directly to a polynomial expression.
\par
We begin with an example for quadratics.
\begin{example}\label{sign_diag_poly_0}
Construct a sign diagram for the polynomial function $f(x)=2x^2+3x-20$.
\par
Although our first example is not factored, we can apply the $ac$-method to quickly factor our function.
\begin{equation*}
\begin{split}
f(x) & = 2x^2+3x-20\\
& = 2x^2+8x-5x-20\\
& = 2x(x+4)-5(x+4)\\
& = (x+4)(2x-5)
\end{split}
\end{equation*}
This gives us two roots, $x=-4$ and $x=\frac{5}{2}$, which serve as the dividers in our accompanying diagram.  For our three test values, we will use $x=-5, 0,$ and $3$.
\begin{center}
\begin{tikzpicture}[xscale=1,yscale=1]
	\draw [<->](-6.25,0) -- coordinate (x axis mid) (4.75,0) node[below right] {$x$};
	\draw [-](-3.5,1) -- coordinate (y axis mid) (-3.5,-0.25) node[below] {$-4$};
	\draw [-](2,1) -- coordinate (y axis mid) (2,-0.25) node[below] {$3$};
	\draw (-5,-1) node {$x=-5$};
	\draw (-0.75,-1) node {$x=0$};
	\draw (3.5,-1) node {$x=3$};
	\draw (-5,0.5) node {$+$};
	\draw (-0.75,0.5) node {$-$};
	\draw (3.5,0.5) node {$+$};
	\draw (-5,-1.5) node {\footnotesize $(-)(-)$};
	\draw (-0.75,-1.5) node {\footnotesize $(+)(-)$};
	\draw (3.5,-1.5) node {\footnotesize $(+)(+)$};
\end{tikzpicture}
\end{center}
\end{example}
The previous example should be a familiar one, and one that we are comfortable with, since it ties in directly with the chapter on quadratics (degree-2 polynomials).  For polynomials with a degree of $n\geq 3,$ our diagram should look similar.  The primary exceptions will be number of factors in our expression, and the number of dividers in our diagram.  Again, we will focus primarily on polynomials which are already factored for our examples.
\begin{example}\label{sign_diag_poly_1}
Construct a sign diagram for the factored polynomial function $$g(x)=(x+2)(3x-1)(5-x).$$
\par
Our roots are $x=-2,\frac{1}{3},$ and $5$.  Consequently, the following diagram shows three dividers.
\begin{center}
\begin{tikzpicture}[xscale=1,yscale=1]
	\draw [<->](-4.25,0) -- coordinate (x axis mid) (7.25,0) node[below right] {$x$};
	\draw [-](-2,1) -- coordinate (y axis mid) (-2,-0.25) node[below] {$-2$};
	\draw [-](0.5,1) -- coordinate (y axis mid) (0.5,-0.25) node[below] {$\frac{1}{3}$};
	\draw [-](5,1) -- coordinate (y axis mid) (5,-0.25) node[below] {$5$};
	\draw (-3,-1) node {$x=-3$};
	\draw (-0.75,-1) node {$x=0$};
	\draw (2.75,-1) node {$x=1$};
	\draw (6,-1) node {$x=6$};
	\draw (-3,0.5) node {$+$};
	\draw (-0.75,0.5) node {$-$};
	\draw (2.75,0.5) node {$+$};
	\draw (6,0.5) node {$-$};
	\draw (-3,-1.75) node {\footnotesize $(-)(-)(+)$};
	\draw (-0.75,-1.75) node {\footnotesize $(+)(-)(+)$};
	\draw (2.75,-1.75) node {\footnotesize $(+)(+)(+)$};
	\draw (6,-1.75) node {\footnotesize $(+)(+)(-)$};
\end{tikzpicture}
\end{center}
\end{example}
For our next example, we will make a slight change to the function $g$ from the previous example, by including an extra factor of $x+2$.
\begin{example}\label{sign_diag_poly_2}
Construct a sign diagram for the factored polynomial function $$h(x)=(x+2)^2(3x-1)(5-x).$$
\par
Since the roots of $h$ equal those from $g$, our diagram will have the same dividers and test values.
\begin{center}
\begin{tikzpicture}[xscale=1,yscale=1]
	\draw [<->](-4.25,0) -- coordinate (x axis mid) (7.25,0) node[below right] {$x$};
	\draw [-](-2,1) -- coordinate (y axis mid) (-2,-0.25) node[below] {$-2$};
	\draw [-](0.5,1) -- coordinate (y axis mid) (0.5,-0.25) node[below] {$\frac{1}{3}$};
	\draw [-](5,1) -- coordinate (y axis mid) (5,-0.25) node[below] {$5$};
	\draw (-3,-1) node {$x=-3$};
	\draw (-0.75,-1) node {$x=0$};
	\draw (2.75,-1) node {$x=1$};
	\draw (6,-1) node {$x=6$};
	\draw (-3,0.5) node {$-$};
	\draw (-0.75,0.5) node {$-$};
	\draw (2.75,0.5) node {$+$};
	\draw (6,0.5) node {$-$};
	\draw (-3,-1.75) node {\footnotesize $(-)(+)$};
	\draw (-0.75,-1.75) node {\footnotesize $(-)(+)$};
	\draw (2.75,-1.75) node {\footnotesize $(+)(+)$};
	\draw (6,-1.75) node {\footnotesize $(+)(-)$};
\end{tikzpicture}
\end{center}
\end{example}
In the previous diagram, we see that each of our sign calculations have excluded the $(x+2)^2$ factor, since it will always contribute a positive sign and therefore has no impact on the end result.  For example, for the test value $x=-3,$ we get
$$(-)^2(-)(+)=\cancel{(-)^2}(-)(+),$$
which reduces to a negative sign.  This simplification in our sign calculation can be employed for any factor that appears in our function with an {\it even} exponent.
\par
Additionally, our last two diagrams look almost identical, with the lone exception being the sign associated with our first interval, $(-\infty,-2)$.  This should make some sense, however, since we only changed the factor associated with the root $x=-2$ from one example to the next.  The reason behind the change in diagram will become more clear to us as we explore polynomials further.
\par
For our last example, we will present both the sign diagram and the accompanying graph for the given polynomial.  Although the techniques to graphing a polynomial have not yet been discussed, for any function it is often helpful to utilize a graphing utility such as \Desmos, in order to better understand the makeup of the function and how its graph is related.
\begin{example}\label{sign_diag_poly_3}
Construct a sign diagram for the factored polynomial function $$f(x)=x(x+1)(x-2)^2(x^2+4).$$
Use \Desmos \ or a similar graphing utility to construct a graph of $f$.
\par
Before we get started, it is important to spend some time discussing the factorization of $f$.  Specifically, the factor of $x$ will contribute a root of $x=0$.  This is the only instance in which our diagram requires a divider at $x=0.$
\par
Additionally, the factor of $x^2+4$ is often misinterpreted.  By setting the expression equal to zero and solving for $x,$ we see that the factor contributes two {\it imaginary} roots at $x=\pm 2i$.  Furthermore, if we look more closely at this factor, we see that for any real number $x$, $x^2+4$ will always be positive.  Hence, this factor will have no impact on our sign diagram calculations, and will be omitted.  One should caution, however, that this factor does have an impact on the graph of $f$.
\par
We can now conclude that the set of roots for $f$ are $x=-1, 0,$ and $2$.  The accompanying diagram and graph are shown below.  As before, we have also omitted the factor of $(x-2)^2,$ since the squared factor will not impact our signs.
\begin{center}
\begin{tikzpicture}[xscale=1.5,yscale=0.5]
	\draw [<->](-3.25,0) -- coordinate (x axis mid) (4.25,0) node[below right] {$x$};
	\draw [<->](0,-8) -- coordinate (y axis mid) (0,11) node[above right] {$y$};
	%\draw [dashed, <->](1.5,-6.25) -- coordinate (y axis mid) (1.5,6.25) node[above right] {};
	\draw [<->] plot [domain=-1.161:2.362, samples=100] (\x,{\x*(\x+1)*(\x-2)^2*((\x)^2+4)});
	\foreach \x in {1,...,4}
		\draw (\x,1pt) -- (\x,-1pt)	node[anchor=north] {\scriptsize \x};
	\foreach \x in {-3,...,-1}
		\draw (\x,1pt) -- (\x,-1pt)	node[anchor=south] {\scriptsize \x};
	\foreach \y in {2,4,...,10}
		\draw (1pt,\y) -- (-1pt,\y)	node[anchor=east] {\scriptsize \y}; 
	\foreach \y in {-6,-4,...,-2}
		\draw (1pt,\y) -- (-1pt,\y)	node[anchor=west] {\scriptsize \y}; 
\end{tikzpicture}
\end{center}
\begin{center}
\begin{tikzpicture}[xscale=1.5,yscale=1]
	\draw [<->](-3.25,0) -- coordinate (x axis mid) (4.25,0) node[below right] {$x$};
	\draw [-](-1,1) -- coordinate (y axis mid) (-1,-0.25) node[below] {$-1$};
	\draw [-](0,1) -- coordinate (y axis mid) (0,-0.25) node[below] {$0$};
	\draw [-](2,1) -- coordinate (y axis mid) (2,-0.25) node[below] {$2$};
	\draw (-2,-1) node {$x=-2$};
	\draw (-0.5,-1) node {$x=-\frac{1}{2}$};
	\draw (1,-1) node {$x=1$};
	\draw (3,-1) node {$x=3$};
	\draw (-2,0.5) node {$+$};
	\draw (-0.5,0.5) node {$-$};
	\draw (1,0.5) node {$+$};
	\draw (3,0.5) node {$+$};
	\draw (-2,-1.75) node {\footnotesize $(-)(-)$};
	\draw (-0.5,-1.75) node {\footnotesize $(-)(+)$};
	\draw (1,-1.75) node {\footnotesize $(+)(+)$};
	\draw (3,-1.75) node {\footnotesize $(+)(+)$};
\end{tikzpicture}
\end{center}
\end{example}
By looking at the graph of our last example, one should begin to notice the relationship that the graph of a polynomial has with its precise makeup and, consequently, its sign diagram.  In particular, close attention should be paid to the nature of the graph of $f$ near its real roots.  In the case of $x=-1$ and $x=0$ in our last example, we see that the graph {\it crosses over} the $x-$axis.  Alternatively, our graph {\it turns around} or ``bounces off'' at the root $x=2$.  This difference in the local behavior of the graph of $f$ at its roots is not just a coincidence, but rather a consequence of the makeup of the function $f,$ as we will see when we discuss the {\it multiplicity} of the root of a polynomial in a later section.
\newpage
\end{document}