\documentclass[12pt]{book}
\raggedbottom
\usepackage[top=1in,left=1in,bottom=1in,right=1in,headsep=0.25in]{geometry}	
\usepackage{amssymb,amsmath,amsthm,amsfonts}
\usepackage{chapterfolder,docmute,setspace}
\usepackage{cancel,multicol,tikz,verbatim,framed,polynom,enumitem,tikzpagenodes}
\usepackage[colorlinks, hyperindex, plainpages=false, linkcolor=blue, urlcolor=blue, pdfpagelabels]{hyperref}
\usepackage[type={CC},modifier={by-sa},version={4.0},]{doclicense}

\theoremstyle{definition}
\newtheorem{example}{Example}
\newcommand{\Desmos}{\href{https://www.desmos.com/}{Desmos}}
\setlength{\parindent}{0in}
\setlist{itemsep=0in}
\setlength{\parskip}{0.1in}
\setcounter{secnumdepth}{0}
\input{lesson_order}

\newcommand{\tmmathbf}[1]{\ensuremath{\boldsymbol{#1}}}
\newcommand{\tmop}[1]{\ensuremath{\operatorname{#1}}}

\begin{document}
\section{Polynomial Inequalities (L\arabic{lesson_polynomial_inequalities})}
{\bf Objective: Solve a polynomial inequality by constructing a sign diagram.}\par
\begin{example}~~Solve the polynomial inequality
$$x^4+6x^2-15x\leq x^4+2x^3-7x^2.$$
Just as with quadratic inequalities, we begin by setting one side equal to zero.  This gives us $$2x^3-13x^2+15x\geq 0.$$
In order to construct a sign diagram, we must find a factorization and identify the roots of the left-hand side of our inequality.
$$2x^3-13x^2+15x=2x\left(x-\dfrac{3}{2}\right)(x-5)$$
So the dividers in our diagram will be the roots $x=0,\frac{3}{2},$ and $5$.  Below is a chart for testing the intervals in our sign diagram, as well as the end result.
\begin{center}
\begin{tabular}{cccc}
\underline{Interval} & \underline{Test Value} & \underline{Signs} & \underline{Result}\\
$(-\infty,0)$ & $x=-1$ & $(-)(-)(-)$ & $-$\\
$(0,\frac{3}{2})$ & $x=1$ & $(+)(-)(-)$ & $+$\\
$(\frac{3}{2},5)$ & $x=3$ & $(+)(+)(-)$ & $-$\\
$(5,\infty)$ &  $x=6$ & $(+)(+)(+)$ & $+$
\end{tabular}
\end{center}
\begin{center}
\begin{tikzpicture}[xscale=1,yscale=1]
	\draw [<->](-4.75,0) -- coordinate (x axis mid) (6.25,0) node[below right] {$x$};
	\draw [-](-1,1) -- coordinate (y axis mid) (-1,-0.25) node[below] {$0$};
	\draw [-](1,1) -- coordinate (y axis mid) (1,-0.25) node[below] {$\frac{3}{2}$};
	\draw [-](4,1) -- coordinate (y axis mid) (4,-0.25) node[below] {$5$};
	\draw (-3,-1) node {$x=-1$};
	\draw (0,-1) node {$x=1$};
	\draw (2.5,-1) node {$x=3$};
	\draw (5,-1) node {$x=6$};
	\draw (-3,0.5) node {$-$};
	\draw (0,0.5) node {$+$};
	\draw (2.5,0.5) node {$-$};
	\draw (5,0.5) node {$+$};
\end{tikzpicture}
\end{center}
So, using our diagram as an aide, we see that the solution to the inequality $$2x^3-13x^2+15x\geq 0,$$
as well as our original inequality
$$x^4+6x^2-15x\leq x^4+2x^3-7x^2,$$
will be $$\left[0,\frac{3}{2}\right]\cup[5,\infty).$$
\begin{multicols}{2}
\begin{center}
\begin{tikzpicture}[xscale=0.5,yscale=0.125]
	\draw [<->](-4.5,0) -- coordinate (x axis mid) (8.5,0) node[below right] {$x$};
	\draw [<->](0,-25) -- coordinate (y axis mid) (0,25) node[above right] {$y$};
	\draw [<->] plot [domain=-0.769:5.461, samples=100] (\x,{2*\x*(\x-1.5)*(\x-5)});
	\foreach \x in {1,...,8}
		\draw (\x,1pt) -- (\x,-1pt)	node[anchor=north] {\scriptsize \x};
	\foreach \x in {-4,...,-1}
		\draw (\x,1pt) -- (\x,-1pt)	node[anchor=south] {\scriptsize \x};
	\foreach \y in {5,10,...,20}
		\draw (1pt,\y) -- (-1pt,\y)	node[anchor=east] {\scriptsize \y}; 
	\foreach \y in {-20,-15,...,-5}
		\draw (1pt,\y) -- (-1pt,\y)	node[anchor=west] {\scriptsize \y}; 
\end{tikzpicture}
\end{center}
\columnbreak
\ \par
Since our given inequality was inclusive\\
($\leq$ or $\geq$), we include the corresponding endpoints in our answer.\\
\ \par
We can verify that our answer is correct by comparing it to the graph of the function $$f(x)=2x^3-13x^2+15x,$$ which lies above (or on) the $x-$axis over the intervals in our answer.
\end{multicols}
\end{example}
\newpage
\end{document}