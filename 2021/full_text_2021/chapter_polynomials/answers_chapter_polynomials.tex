\documentclass[12pt]{book}
\raggedbottom
\usepackage[top=1in,left=1in,bottom=1in,right=1in,headsep=0.25in]{geometry}	
\usepackage{amssymb,amsmath,amsthm,amsfonts}
\usepackage{chapterfolder,docmute,setspace}
\usepackage{cancel,multicol,tikz,verbatim,framed,polynom,enumitem,tikzpagenodes}
\usepackage[colorlinks, hyperindex, plainpages=false, linkcolor=blue, urlcolor=blue, pdfpagelabels]{hyperref}
\usepackage[type={CC},modifier={by-sa},version={4.0},]{doclicense}

\theoremstyle{definition}
\newtheorem{example}{Example}
\newcommand{\Desmos}{\href{https://www.desmos.com/}{Desmos}}
\setlength{\parindent}{0in}
\setlist{itemsep=0in}
\setlength{\parskip}{0.1in}
\setcounter{secnumdepth}{0}
% This document is used for ordering of lessons.  If an instructor wishes to change the ordering of assessments, the following steps must be taken:

% 1) Reassign the appropriate numbers for each lesson in the \setcounter commands included in this file.
% 2) Rearrange the \include commands in the master file (the file with 'Course Pack' in the name) to accurately reflect the changes.  
% 3) Rarrange the \items in the measureable_outcomes file to accurately reflect the changes.  Be mindful of page breaks when moving items.
% 4) Re-build all affected files (master file, measureable_outcomes file, and any lessons whose numbering has changed).

%Note: The placement of each \newcounter and \setcounter command reflects the original/default ordering of topics (linears, systems, quadratics, functions, polynomials, rationals).

\newcounter{lesson_solving_linear_equations}
\newcounter{lesson_equations_containing_absolute_values}
\newcounter{lesson_graphing_lines}
\newcounter{lesson_two_forms_of_a_linear_equation}
\newcounter{lesson_parallel_and_perpendicular_lines}
\newcounter{lesson_linear_inequalities}
\newcounter{lesson_compound_inequalities}
\newcounter{lesson_inequalities_containing_absolute_values}
\newcounter{lesson_graphing_systems}
\newcounter{lesson_substitution}
\newcounter{lesson_elimination}
\newcounter{lesson_quadratics_introduction}
\newcounter{lesson_factoring_GCF}
\newcounter{lesson_factoring_grouping}
\newcounter{lesson_factoring_trinomials_a_is_1}
\newcounter{lesson_factoring_trinomials_a_neq_1}
\newcounter{lesson_solving_by_factoring}
\newcounter{lesson_square_roots}
\newcounter{lesson_i_and_complex_numbers}
\newcounter{lesson_vertex_form_and_graphing}
\newcounter{lesson_solve_by_square_roots}
\newcounter{lesson_extracting_square_roots}
\newcounter{lesson_the_discriminant}
\newcounter{lesson_the_quadratic_formula}
\newcounter{lesson_quadratic_inequalities}
\newcounter{lesson_functions_and_relations}
\newcounter{lesson_evaluating_functions}
\newcounter{lesson_finding_domain_and_range_graphically}
\newcounter{lesson_fundamental_functions}
\newcounter{lesson_finding_domain_algebraically}
\newcounter{lesson_solving_functions}
\newcounter{lesson_function_arithmetic}
\newcounter{lesson_composite_functions}
\newcounter{lesson_inverse_functions_definition_and_HLT}
\newcounter{lesson_finding_an_inverse_function}
\newcounter{lesson_transformations_translations}
\newcounter{lesson_transformations_reflections}
\newcounter{lesson_transformations_scalings}
\newcounter{lesson_transformations_summary}
\newcounter{lesson_piecewise_functions}
\newcounter{lesson_functions_containing_absolute_values}
\newcounter{lesson_absolute_as_piecewise}
\newcounter{lesson_polynomials_introduction}
\newcounter{lesson_sign_diagrams_polynomials}
\newcounter{lesson_factoring_quadratic_type}
\newcounter{lesson_factoring_summary}
\newcounter{lesson_polynomial_division}
\newcounter{lesson_synthetic_division}
\newcounter{lesson_end_behavior_polynomials}
\newcounter{lesson_local_behavior_polynomials}
\newcounter{lesson_rational_root_theorem}
\newcounter{lesson_polynomials_graphing_summary}
\newcounter{lesson_polynomial_inequalities}
\newcounter{lesson_rationals_introduction_and_terminology}
\newcounter{lesson_sign_diagrams_rationals}
\newcounter{lesson_horizontal_asymptotes}
\newcounter{lesson_slant_and_curvilinear_asymptotes}
\newcounter{lesson_vertical_asymptotes}
\newcounter{lesson_holes}
\newcounter{lesson_rationals_graphing_summary}

\setcounter{lesson_solving_linear_equations}{1}
\setcounter{lesson_equations_containing_absolute_values}{2}
\setcounter{lesson_graphing_lines}{3}
\setcounter{lesson_two_forms_of_a_linear_equation}{4}
\setcounter{lesson_parallel_and_perpendicular_lines}{5}
\setcounter{lesson_linear_inequalities}{6}
\setcounter{lesson_compound_inequalities}{7}
\setcounter{lesson_inequalities_containing_absolute_values}{8}
\setcounter{lesson_graphing_systems}{9}
\setcounter{lesson_substitution}{10}
\setcounter{lesson_elimination}{11}
\setcounter{lesson_quadratics_introduction}{16}
\setcounter{lesson_factoring_GCF}{17}
\setcounter{lesson_factoring_grouping}{18}
\setcounter{lesson_factoring_trinomials_a_is_1}{19}
\setcounter{lesson_factoring_trinomials_a_neq_1}{20}
\setcounter{lesson_solving_by_factoring}{21}
\setcounter{lesson_square_roots}{22}
\setcounter{lesson_i_and_complex_numbers}{23}
\setcounter{lesson_vertex_form_and_graphing}{24}
\setcounter{lesson_solve_by_square_roots}{25}
\setcounter{lesson_extracting_square_roots}{26}
\setcounter{lesson_the_discriminant}{27}
\setcounter{lesson_the_quadratic_formula}{28}
\setcounter{lesson_quadratic_inequalities}{29}
\setcounter{lesson_functions_and_relations}{12}
\setcounter{lesson_evaluating_functions}{13}
\setcounter{lesson_finding_domain_and_range_graphically}{14}
\setcounter{lesson_fundamental_functions}{15}
\setcounter{lesson_finding_domain_algebraically}{30}
\setcounter{lesson_solving_functions}{31}
\setcounter{lesson_function_arithmetic}{32}
\setcounter{lesson_composite_functions}{33}
\setcounter{lesson_inverse_functions_definition_and_HLT}{34}
\setcounter{lesson_finding_an_inverse_function}{35}
\setcounter{lesson_transformations_translations}{36}
\setcounter{lesson_transformations_reflections}{37}
\setcounter{lesson_transformations_scalings}{38}
\setcounter{lesson_transformations_summary}{39}
\setcounter{lesson_piecewise_functions}{40}
\setcounter{lesson_functions_containing_absolute_values}{41}
\setcounter{lesson_absolute_as_piecewise}{42}
\setcounter{lesson_polynomials_introduction}{43}
\setcounter{lesson_sign_diagrams_polynomials}{44}
\setcounter{lesson_factoring_quadratic_type}{46}
\setcounter{lesson_factoring_summary}{45}
\setcounter{lesson_polynomial_division}{47}
\setcounter{lesson_synthetic_division}{48}
\setcounter{lesson_end_behavior_polynomials}{49}
\setcounter{lesson_local_behavior_polynomials}{50}
\setcounter{lesson_rational_root_theorem}{51}
\setcounter{lesson_polynomials_graphing_summary}{52}
\setcounter{lesson_polynomial_inequalities}{53}
\setcounter{lesson_rationals_introduction_and_terminology}{54}
\setcounter{lesson_sign_diagrams_rationals}{55}
\setcounter{lesson_horizontal_asymptotes}{56}
\setcounter{lesson_slant_and_curvilinear_asymptotes}{57}
\setcounter{lesson_vertical_asymptotes}{58}
\setcounter{lesson_holes}{59}
\setcounter{lesson_rationals_graphing_summary}{60}

\newcommand{\tmmathbf}[1]{\ensuremath{\boldsymbol{#1}}}
\newcommand{\tmop}[1]{\ensuremath{\operatorname{#1}}}

\newlist{oddenumerate}{enumerate}{1}
\setlist[oddenumerate]{start=0,label=\theoddenumeratei.}
\makeatletter
\renewcommand\theoddenumeratei{\@arabic{\numexpr2*\value{oddenumeratei}+1}}
\makeatother

\newcount\gpten % (global) power-of-ten -- tells which digit we are doing
\countdef\rtot2 % running total -- remainder so far
\countdef\LDscratch4 % scratch

\def\longdiv#1#2{%
 \vtop{\normalbaselines \offinterlineskip
   \setbox\strutbox\hbox{\vrule height 2.1ex depth .5ex width0ex}%
   \def\showdig{$\underline{\the\LDscratch\strut}$\cr\the\rtot\strut\cr
       \noalign{\kern-.2ex}}%
   \global\rtot=#1\relax
   \count0=\rtot\divide\count0by#2\edef\quotient{\the\count0}%\show\quotient
   % make list macro out of digits in quotient:
   \def\temp##1{\ifx##1\temp\else \noexpand\dodig ##1\expandafter\temp\fi}%
   \edef\routine{\expandafter\temp\quotient\temp}%
   % process list to give power-of-ten:
   \def\dodig##1{\global\multiply\gpten by10 }\global\gpten=1 \routine
   % to display effect of one digit in quotient (zero ignored):
   \def\dodig##1{\global\divide\gpten by10
      \LDscratch =\gpten
      \multiply\LDscratch  by##1%
      \multiply\LDscratch  by#2%
      \global\advance\rtot-\LDscratch \relax
      \ifnum\LDscratch>0 \showdig \fi % must hide \cr in a macro to skip it
   }%
   \tabskip=0pt
   \halign{\hfil##\cr % \halign for entire division problem
     $\quotient$\strut\cr
     #2$\,\overline{\vphantom{\big)}%
     \hbox{\smash{\raise3.5\fontdimen8\textfont3\hbox{$\big)$}}}%
     \mkern2mu \the\rtot}$\cr\noalign{\kern-.2ex}
     \routine \cr % do each digit in quotient
}}}

\begin{document}
\section{Selected Answers}
\subsection*{Introduction and Terminology}
%Identify the degree, set of coefficients, leading coefficient, leading term and constant term for each of the polynomials listed.  Classify each polynomial by both degree and number of nonzero terms.  If it is not already provided, write the polynomial in descending-power order.
%\begin{multicols}{2}
\begin{oddenumerate}
\item %$f(x)=-2x^3-1$
$n=3, \ \ a_n=-2, \ \  a_nx^n=-2x^3, \ \  a_0=-1, \ \  \{-2,0,0,-1\}$
%%\item $f(x)=-2x^4 + 4x+1$
\item %$f(x)=40-x^3$
$n=3, \ \  a_n=-1, \ \  a_nx^n=-1x^3, \ \  a_0=40, \ \  \{-1,0,0,40\}$
%%\item $f(x)=(x-1)^2$
\item %$f(x)=32x^5+x^2+x$
$n=5, \ \  a_n=32, \ \  a_nx^n=32x^5, \ \  a_0=0, \ \  \{32,0,0,1,1,0\}$
%%\item $f(x)=4x^2-3x^4$
\item %$f(x)=-2x^4-4x^2-6x-8$
$n=4, \ \  a_n=-2, \ \  a_nx^n=-2x^4, \ \  a_0=-8, \ \  \{-2,0,-4,-6,-8\}$
%%\item $f(x)=5x+3x^2+x^3+\sqrt{3}$
\item %$f(x)=\frac{1}{2}x^4-5x^2-\frac{1}{2}$
$n=4, \ \  a_n=\frac{1}{2}, \ \  a_nx^n=\frac{1}{2}x^4, \ \  a_0=-\frac{1}{2}, \ \  \{\frac{1}{2},0,-5,0,-\frac{1}{2}\}$
%%\item $f(x)=12-6x+3x^2-2x^3-x^6$
\item %$f(x)=-3x^4-12x^3+x-13$
$n=4, \ \  a_n=-3, \ \  a_nx^n=-3x^4, \ \  a_0=-13, \ \  \{-3,-12,0,1,-13\}$
\end{oddenumerate}
%\end{multicols}
\subsection*{Sign Diagrams}
%Construct a sign diagram for the factored polynomial functions below. Use \Desmos \ to graph each function and check the accuracy of your diagram.  Identify the interval(s) where the function is positive and where it is negative.
\begin{multicols}{2}
\begin{oddenumerate}
\item \ \\ %$f(x)=x^3(x-2)(x+2)$ 
\begin{tikzpicture}[xscale=0.5,yscale=0.5]
	\draw [<->](-5,0) -- coordinate (x axis mid) (5,0) node[below right] {$x$};
	\draw [-](-3,1) -- coordinate (y axis mid) (-3,-0.25) node[below] {$-2$};
	\draw [-](0,1) -- coordinate (y axis mid) (0,-0.25) node[below] {$0$};
	\draw [-](3,1) -- coordinate (y axis mid) (3,-0.25) node[below] {$2$};
	\draw (-4,0.5) node {$-$};
	\draw (-1.5,0.5) node {$+$};
	\draw (1.5,0.5) node {$-$};
	\draw (4,0.5) node {$+$};
\end{tikzpicture}
%%\item $g(x)=(x^2+1)(1-x)$
\item \ \\  %$h(x)=x(x-3)^2(x+3)$
\begin{tikzpicture}[xscale=0.5,yscale=0.5]
	\draw [<->](-5,0) -- coordinate (x axis mid) (5,0) node[below right] {$x$};
	\draw [-](-3,1) -- coordinate (y axis mid) (-3,-0.25) node[below] {$-3$};
	\draw [-](0,1) -- coordinate (y axis mid) (0,-0.25) node[below] {$0$};
	\draw [-](3,1) -- coordinate (y axis mid) (3,-0.25) node[below] {$3$};
	\draw (-4,0.5) node {$+$};
	\draw (-1.5,0.5) node {$-$};
	\draw (1.5,0.5) node {$+$};
	\draw (4,0.5) node {$+$};
\end{tikzpicture}
%%\item $k(x)=(3x-4)^3$
\item \ \\  %$\ell(x)=(x^2+2)(x^2+3)$
\begin{tikzpicture}[xscale=0.5,yscale=0.5]
	\draw [<->](-5,0) -- coordinate (x axis mid) (5,0) node[below right] {$x$};
	\draw (0,0.5) node {$+$};
\end{tikzpicture}

\columnbreak

%%\item $m(x)=-2(x+7)^2(1-2x)^2$
\item \ \\  %$f(x)=(x^2-1)(x+4)$
\begin{tikzpicture}[xscale=0.5,yscale=0.5]
	\draw [<->](-5,0) -- coordinate (x axis mid) (5,0) node[below right] {$x$};
	\draw [-](-3,1) -- coordinate (y axis mid) (-3,-0.25) node[below] {$-4$};
	\draw [-](0,1) -- coordinate (y axis mid) (0,-0.25) node[below] {$-1$};
	\draw [-](3,1) -- coordinate (y axis mid) (3,-0.25) node[below] {$1$};
	\draw (-4,0.5) node {$-$};
	\draw (-1.5,0.5) node {$+$};
	\draw (1.5,0.5) node {$-$};
	\draw (4,0.5) node {$+$};
\end{tikzpicture}
%%\item $g(x)=(x^2-1)(x^2-16)$
\item \ \\  %$h(x)=-2x^3(3x-1)(2-x)$
\begin{tikzpicture}[xscale=0.5,yscale=0.5]
	\draw [<->](-5,0) -- coordinate (x axis mid) (5,0) node[below right] {$x$};
	\draw [-](-3,1) -- coordinate (y axis mid) (-3,-0.25) node[below] {$0$};
	\draw [-](0,1) -- coordinate (y axis mid) (0,-0.25) node[below] {$\frac{1}{3}$};
	\draw [-](3,1) -- coordinate (y axis mid) (3,-0.25) node[below] {$2$};
	\draw (-4,0.5) node {$-$};
	\draw (-1.5,0.5) node {$+$};
	\draw (1.5,0.5) node {$-$};
	\draw (4,0.5) node {$+$};
\end{tikzpicture}
%%\item $k(x)=(x^2-4x+1)(x+2)^2$
\end{oddenumerate}
\end{multicols}
\subsection*{Factoring}
\subsubsection{Some Special Cases}
%Completely factor each of the following polynomial expressions.
\begin{multicols}{2}
\begin{oddenumerate}
  \item %$2 x^2 - 11 x + 15$
$(2x-5)(x-3)$
%%  \item $5 n^3 + 7 n^2 - 6 n$
  \item %$54 u^3 - 16$
$2(3x-2)(9x^2+6x+4)$
%%  \item $54 - 128 x^3$
  \item %$n^2 - n$
$n(n-1)$
%%  \item $2x^4 -21x^2-11$
  \item %$24 a z - 18 a h + 60 y z - 45 y h$
$3(2a+15y)(4z-3h)$
%%  \item $5 u^2 - 9 u v + 4 v^2$
  \item %$16 x^2 + 48 x y + 36 y^2$
$4(2x+3y)^2$
%%  \item $- 2 x^3 + 128 y^3$
  \item %$20 u v - 60 u^3 - 5 x v + 15 x u^2$
$-5(4u-x)(3u^2-v)$
%%  \item $2 x^3 + 5 x^2 y + 3 y^2 x$
  \end{oddenumerate}
\end{multicols}
\subsubsection{Quadratic Type}
%Completely factor each of the following polynomials over the real numbers and identify the set of all real roots.
%\begin{multicols}{2}
\begin{oddenumerate}
  \item  %$x^4 +13x^2+40$
$(x^2+8)(x^2+5)$
%%  \item  $x^4-5x^2+4$
  \item  %$x^4 -17x^2+16$
$(x-1)(x+1)(x-4)(x+4)$
%%  \item  $x^4 -3x^2-40$
  \item  %$3x^4 -32x^2+45$
$(3x^2-5)(x^2-9)=3(x-\frac{\sqrt{15}}{3})(x-\frac{\sqrt{15}}{3})(x-3)(x+3)$
%%  \item  $x^4 +x^2-12$
  \item  %$x^4 -3x^2-10$
$(x^2-5)(x^2+2)=(x-\sqrt{5})(x+\sqrt{5})(x^2+2)$
%%  \item  $x^6 -82x^3+81$
  \item  %$8x^4 +2x^2-3$
$(2x^2-\frac{1}{2})(4x^2+3)=2(x-\frac{\sqrt{2}}{2})(x+\frac{\sqrt{2}}{2})(4x^2+3)$
%%  \item  $2x^4 -19x^2+9$
\end{oddenumerate}
%\end{multicols}
\subsection*{Division}
\subsubsection{Polynomial (Long) Division}
%Use polynomial long division to divide and simplify each of the given expressions.  Express each answer in the form below.
%$$\frac{\text{dividend}}{\text{divisor}} \ = \ \text{quotient} \ + \ \frac{\text{remainder}}{\text{divisor}}$$
\begin{multicols}{2}
\begin{oddenumerate}
%%Note: All of these items need to be corrected to be in terms of the variable x for the polylongdiv command to work.
  \item %$\dfrac{20 x^4 + x^3 + 2 x^2}{4 x^3}$
\polylongdiv{20 x^4 + x^3 + 2 x^2}{4 x^3}
%%  \item $\dfrac{5 x^4 + 45 x^3 + 4 x^2}{9 x}$
  \item %$\dfrac{20 n^4 + n^3 + 40 n^2}{10 n}$
\polylongdiv{20 x^4 + x^3 + 40 x^2}{10 x}
%%  \item $\dfrac{3 k^3 + 4 k^2 + 2 k}{8 k}$
  \item %$\dfrac{12 x^4 + 24 x^3 + 3 x^2}{6 x}$
\polylongdiv{12 x^4 + 24 x^3 + 3 x^2}{6 x}
%%  \item $\dfrac{5 p^4 + 16 p^3 + 16 p^2}{4 p}$
  \item %$\dfrac{10 n^4 + 50 n^3 + 2 n^2}{10 n^2}$
\polylongdiv{10 x^4 + 50 x^3 + 2 x^2}{10 x^2}
%%  \item $\dfrac{3 m^4 + 18 m^3 + 27 m^2}{9 m^2}$
  \item %$\dfrac{x^2 - 2 x - 71}{x + 8}$%\label{polydiv_one}
\polylongdiv{x^2 - 2 x - 71}{x + 8}
%%  \item $\dfrac{r^2 - 3 r - 53}{r - 9}$
  \item %$\dfrac{n^2 + 13 n + 32}{n + 5}$
\polylongdiv{x^2 + 13 x + 32}{x + 5}
%%  \item $\dfrac{b^2 - 10 b + 16}{b - 7}$
  \item %$\dfrac{v^2 - 2 v - 89}{v - 10}$
\polylongdiv{x^2 - 2 x - 89}{x - 10}
%%  \item $\dfrac{x^2 + 4 x - 26}{x + 7}$
  \item %$\dfrac{a^2 - 4 a - 38}{a - 8}$
\polylongdiv{x^2 - 4 x - 38}{x - 8}
%%  \item $\dfrac{n^2 - 4}{n - 2}$
  \item %$\dfrac{a^3 + 15 a^2 + 49 a - 55}{a + 7}$
\polylongdiv{x^3 + 15 x^2 + 49 x - 55}{x + 7}
%%  \item $\dfrac{x^3 - 26 x - 41}{x + 4}$
  \item %$\dfrac{3 n^3 + 9 n^2 - 64 n - 68}{n + 6}$
\polylongdiv{3 x^3 + 9 x^2 - 64 x - 68}{x + 6}
%%  \item $\dfrac{9 p^3 + 45 p^2 + 27 p - 5}{9 p + 9}$
  \item %$\dfrac{r^3 - r^2 - 16 r + 8}{r - 4}$
\polylongdiv{x^3 - x^2 - 16 x + 8}{x - 4}
%%  \item $\dfrac{x^2 - 10 x + 22}{x - 4}$
  \item %$\dfrac{x^3 - 16 x^2 + 71 x - 56}{x - 8}$
\polylongdiv{x^3 - 16 x^2 + 71 x - 56}{x - 8}
%%  \item $\dfrac{k^3 - 4 k^2 - 6 k + 4}{k - 1}$
  \item %$\dfrac{8 k^3 - 66 k^2 + 12 k + 37}{k - 8}$
\polylongdiv{8 x^3 - 66 x^2 + 12 x + 37}{x - 8}
%%  \item $\dfrac{3 m^2 + 9 m - 9}{3 m - 3}$
  \item %$\dfrac{2 x^2 - 5 x - 8}{2 x + 3}$
\polylongdiv{2 x^2 - 5 x - 8}{2 x + 3}
%%  \item $\dfrac{3 v^2 - 32}{3 v - 9}$
  \item %$\dfrac{4 n^2 - 23 n - 38}{4 n + 5}$
\polylongdiv{4 x^2 - 23 x - 38}{4 x + 5}
%%  \item $\dfrac{2 n^3 + 21 n^2 + 25 n}{2 n + 3}$
  \item %$\dfrac{4 v^3 - 21 v^2 + 6 v + 19}{4 v + 3}$  
\polylongdiv{4 x^3 - 21 x^2 + 6 x + 19}{4 x + 3}
%%  \item $\dfrac{8 m^3 - 57 m^2 + 42}{8 m + 7}$
  \item %$\dfrac{2 x^3 + 12 x^2 + 4 x - 37}{2 x + 6}$
\polylongdiv{2 x^3 + 12 x^2 + 4 x - 37}{2 x + 6}
%%  \item $\dfrac{45 p^2 + 56 p + 19}{9 p + 4}$
  \item %$\dfrac{10 x^2 - 32 x + 9}{10 x - 2}$
\polylongdiv{10 x^2 - 32 x + 9}{10 x - 2}
%%  \item $\dfrac{4 r^2 - r - 1}{4 r + 3}$
  \item %$\dfrac{27 b^2 + 87 b + 35}{3 b + 8}$
\polylongdiv{27 x^2 + 87 x + 35}{3 x + 8}
%%  \item $\dfrac{4 x^2 - 33 x + 28}{4 x - 5}$
  \item %$\dfrac{48 k^2 - 70 k + 16}{6 k - 2}$
\polylongdiv{48 x^2 - 70 x + 16}{6 x - 2}
%%  \item $\dfrac{12 n^3 + 12 n^2 - 15 n - 4}{2 n + 3}$
  \item %$\dfrac{24 b^3 - 38 b^2 + 29 b - 60}{4 b - 7}$%\label{polydiv_two}
\polylongdiv{24 x^3 - 38 x^2 + 29 x - 60}{4 x - 7}
	\end{oddenumerate}
\end{multicols}
\subsubsection{Synthetic Division}
%Use synthetic division to divide and simplify each of the given expressions.  Express each answer in the form below.
%$$\frac{\text{dividend}}{\text{divisor}} \ = \ \text{quotient} \ + \ \frac{\text{remainder}}{\text{divisor}}$$
%\begin{multicols}{2}
\begin{oddenumerate}
  \item %$\dfrac
\polyhornerscheme[x=-2,showbase=top,resultstyle=\bf]{x^4-4x^3+2x^2-x+1}
%%  \item $\dfrac{x^4-2x^3+7x^2-6x+3}{x-2}$
  \item %$\dfrac
\polyhornerscheme[x=-2,showbase=top,resultstyle=\bf]{2x^4-2x^3-10x^2+1}
%%  \item $\dfrac{5x^4-2x^3+4x^2-5x}{x-1}$
  \item %$\dfrac
  \polyhornerscheme[x=-5,showbase=top,resultstyle=\bf]{-x^4-x^3+x^2+x+1}
%%  \item $\dfrac{x^4-3x^3+2x^2-x+1}{x-4}$
  \item %$\dfrac
  \polyhornerscheme[x=-2,showbase=top,resultstyle=\bf]{12x^4-x^3+x^2-3x+1}
%%  \item $\dfrac{3x^4+3x^3+13x^2-4x+14}{x+1}$
  \item %$\dfrac
  \polyhornerscheme[x=2,showbase=top,resultstyle=\bf]{1x^4-3x^3+5x^2-14x+2}
%%  \item $\dfrac{2x^4-2x+1}{x+3}$
  \item %$\dfrac
  \polyhornerscheme[x=3,showbase=top,resultstyle=\bf]{x^4-3x-4}
%%  \item $\dfrac{x^4-4x^3+13x^2-5x+7}{x-4}$
\end{oddenumerate}
%\end{multicols}
%Use synthetic division to divide and simplify each of the given expression from Exercises \ref{polydiv_one}-\ref{polydiv_two}.
\subsection*{End Behavior}
%Determine the end behavior of each of the following functions.  Write your answers as mathematical sentences.  Graph each function on \Desmos \ to check your answers.
\begin{oddenumerate}
%\begin{multicols}{2}
\item %$f(x)=-2x^3 + 4x+1$
As $x\rightarrow -\infty, \ f(x)\rightarrow \infty$. \ \ \ As $x\rightarrow\infty, \ f(x)\rightarrow -\infty$.
%%\item $g(x)=32x^5+x^2+15$
\item %$h(x)=-3x^4+4x^2$
As $x\rightarrow -\infty, \ h(x)\rightarrow -\infty$. \ \ \ As $x\rightarrow\infty, \ h(x)\rightarrow -\infty$.
%%\item $k(x)=15x^4-32x^2-x-14$
\item %$\ell(x)=x^5+40$
As $x\rightarrow -\infty, \ \ell(x)\rightarrow -\infty$. \ \ \ As $x\rightarrow\infty, \ \ell(x)\rightarrow \infty$.
%%\item $m(x)=5x^5+3x^2+x+14$
\item %$n(x)=123x^4-7x^3-5x^2-3x+1$
As $x\rightarrow -\infty, \ n(x)\rightarrow \infty$. \ \ \ As $x\rightarrow\infty, \ n(x)\rightarrow \infty$.
%%\item $p(x)=x^3-1$
\item %$q(x)=-23x^6+x^3+x^2+x+1$
As $x\rightarrow -\infty, \ q(x)\rightarrow -\infty$. \ \ \ As $x\rightarrow\infty, \ q(x)\rightarrow -\infty$.
%%\end{multicols}
%\end{oddenumerate}
%Identify the degree, leading coefficient, and constant term of each polynomial function below.  Use the degree and leading coefficient to identify the end behavior of the graph of each function.  Write your answers as mathematical sentences.  Graph each function on \Desmos \ to check your answers.
%\begin{oddenumerate}[resume]
%\begin{multicols}{2}
%    \item $f(x)=x^3(x-2)(x+2)$
	\item %$g(x)=(x^2+1)(1-x)$
$a_nx^n=-1x^3, \ \ a_0=1,$ \ \ As $x\rightarrow -\infty, \ f(x)\rightarrow \infty$. \ \ \ As $x\rightarrow\infty, \ f(x)\rightarrow -\infty$.
%	\item $h(x)=x(x-3)^2(x+3)$
	\item %$k(x)=(3x-4)^3$
$a_nx^n=27x^3, \ \ a_0=-64,$ \ \ As $x\rightarrow -\infty, \ k(x)\rightarrow -\infty$. \ \ \ As $x\rightarrow\infty, \ k(x)\rightarrow \infty$.
%    \item $\ell(x)=(x^2+2)(x^2+3)$
    \item %$m(x)=-2(x+7)^2(1-2x)^2$
$a_nx^n=-8x^4, \ \ a_0=-98,$ \ \ As $x\rightarrow -\infty, \ m(x)\rightarrow -\infty$. \ \ \ As $x\rightarrow\infty, \ m(x)\rightarrow -\infty$.
%    \item $f(x)=(x^2-1)(x+4)$
	\item %$g(x)=(x^2-1)(x^2-16)$
$a_nx^n=1x^4, \ \ a_0=16,$ \ \ As $x\rightarrow -\infty, \ g(x)\rightarrow \infty$. \ \ \ As $x\rightarrow\infty, \ g(x)\rightarrow \infty$.
%	\item $h(x)=-2x^3(3x-1)(2-x)$
	\item %$k(x)=(x^2-4x+1)(x+2)^2$
$a_nx^n=1x^4, \ \ a_0=2,$ \ \ As $x\rightarrow -\infty, \ k(x)\rightarrow \infty$. \ \ \ As $x\rightarrow\infty, \ k(x)\rightarrow \infty$.
%\end{multicols}
\end{oddenumerate}
\subsection*{Local Behavior}
%Determine the set of roots and corresponding multiplicities for the following functions.  In each case, classify the corresponding $x-$intercept as either a turnaround or crossover point.  Use \Desmos \ to check your answers.
%\begin{multicols}{2}
\begin{oddenumerate}
\item %$f(x)=x^3(x-2)(x+2)$
$x_1=0, k_1=3,$ Crossover; \ $x_2=2, k_2=1,$ Crossover;  \ $x_3=-2, k_3=1,$ Crossover
%%\item $g(x)=(x^2+1)(1-x)$
\item %$h(x)=x(x-3)^2(x+3)$
$x_1=0, k_1=1,$ Crossover; \ $x_2=3, k_2=2,$ Turnaround;  \ $x_3=-3, k_3=1,$ Crossover
%%\item $k(x)=(3x-4)^3$
\item %$\ell(x)=(x^2+2)(x^2+3)$
No real roots
%%\item $m(x)=-2(x+7)^2(1-2x)^2$
\item %$f(x)=(x^2-1)(x+4)$
$x_1=1, k_1=1,$ Crossover; \ $x_2=-1, k_2=1,$ Crossover;  \ $x_3=-4, k_3=1,$ Crossover
%%\item $g(x)=(x^2-1)(x^2-16)$
\item %$h(x)=-2x^3(3x-1)(2-x)$
$x_1=0, k_1=3,$ Crossover; \ $x_2=\frac{1}{3}, k_2=1,$ Crossover;  \ $x_3=2, k_3=1,$ Crossover 
%%\item $k(x)=(x^2-4x+1)(x+2)^2$
\item %$f(x)=\frac{1}{2}(x-2)^2(x+5)(x-3)$
$x_1=2, k_1=2,$ Turnaround; \ $x_2=-5, k_2=1,$ Crossover;  \ $x_3=3, k_3=1,$ Crossover 
%%\item $g(x)=(x+2)^2(3x-1)(5-x)$
\end{oddenumerate}
%\end{multicols}
\subsection*{The Rational Root Theorem}
%Use the Rational Root Theorem to identify a set of possible rational roots for each of the polynomial functions below.  Evaluate the function at $x=1$.  If $x=1$ is a real root, divide the polynomial by $x-1$ and factor the resulting quotient.  If $x=1$ is not a real root, evaluate the function at at least one of your remaining possible roots, in order to determine if they are actual roots of the polynomial.  If successful, divide your polynomial by the respective factor and factor the remaining quotient.  Use \Desmos \ to help determine the actual set of real roots.
\begin{oddenumerate}
	\item $f(x) = $ \polyfactorize{x^3 - 2x^2 - 5x + 6}\\
	List of possible rational roots: $\{\pm 6, \pm 3, \pm 2, \pm 1\}$
%	\item $f(x) = 2x^4+x^3-7x^2-3x+3$
	\item $f(x) = $ \polyfactorize{x^5-x^4-37x^3+37x^2+36x-36}\\
	List of possible rational roots: $\{\pm 36, \pm 18, \pm 12, \pm 9, \pm 6, \pm 4, \pm 3, \pm 2, \pm 1\}$
%	\item $f(x) = 3x^{3} + 3x^{2} - 11x - 10$
%\end{oddenumerate}
%Use the Rational Root Theorem to identify a set of possible rational roots for each of the polynomial functions below.  Evaluate the function at at least two of your possible roots, in order to determine if they are actual roots of the polynomial.  If successful, divide your polynomial by the respective factor.  Use \Desmos \ to help determine the actual set of real roots.
%\begin{oddenumerate}[resume]
	\item $f(x) = $ \polyfactorize{x^4 -2x^3+ 5x^2 - 8x + 4}\\
	List of possible rational roots: $\{\pm 4, \pm 2, \pm 1 \}$
%	\item $f(x) = x^{3} + 4x^{2} - 11x + 6$
	\item $f(x) = $ \polyfactorize{-2x^3 + 19x^2 - 49x + 20}\\
	List of possible rational roots: $\{\pm 20, \pm 10, \pm 5, \pm 4, \pm{\frac{5}{2}}, \pm 2, \pm 1, \pm{\frac{1}{2}}\}$
%	\item $f(x) = 36x^{4} - 12x^{3} - 11x^{2} + 2x + 1$
	\item $f(x) = $ \polyfactorize{x^4 - 9x^2 - 4x + 12}\\
	List of possible rational roots: $\{\pm 12, \pm 6, \pm 4, \pm 3, \pm 2, \pm 1 \}$
%	\item $f(x) = x^{4} + 2x^{3} - 12x^{2} - 40x - 32$
	\item $f(x) = $ \polyfactorize{6x^3+19x^2-6x-40}\\
	List of possible rational roots:\\
	$\{\pm 40, \pm 20, \pm{\frac{40}{3}}, \pm 10, \pm 8, \pm{\frac{20}{3}}, \pm 5, \pm 4, \pm{\frac{10}{3}}, \pm{\frac{8}{3}}, \pm{\frac{5}{2}}, \pm 2, \pm{\frac{5}{3}}, \pm{\frac{4}{3}}, \pm 1, \pm{\frac{5}{6}}, \pm{\frac{2}{3}}, \pm{\frac{1}{2}}, \pm{\frac{1}{3}}, \pm{\frac{1}{6}} \}$
\end{oddenumerate}
\subsection*{Graphing Summary}
%Factor each polynomial below, and sketch a complete graph of the function, making sure to have a clearly defined scale and label any intercepts.  Use \Desmos \ to compare your results.
\begin{oddenumerate}
%\begin{multicols}{2}
	\item $f(x) = $ \polyfactorize{-17x^3 + 5x^2 + 34x - 10}
%	\item $f(x) = x^4-9x^2+14$
	\item $f(x) = (3x^2+1)$ \polyfactorize{x^2-5}
%	\item $f(x) = 2x^4-7x^2+6$
	\item $f(x) = $ \polyfactorize{x^5-2x^4-x+2}
%	\item $f(x) = 2x^5+3x^4-32x-48$
	\item $f(x) = (x^3-8)(x^3+2)=(x-2)(x^2+2x+4)(x+\sqrt[3]{2})(x^2-\sqrt[3]{2}x+\sqrt[3]{4})$
%	\item $f(x) = 2x^6-7x^3+5$
	\item $f(x) = -(x^2+1)$ \polyfactorize{x - 7}
%	\item $f(x) = 3x^4 - 5x^3 - 12x^2$
	\item $f(x) = $ \polyfactorize{2x^3 - 5x^2 - x}
%	\item $f(x) = -x^4-2x^2 +15$
%\end{multicols}
%\end{oddenumerate}
%Get a complete factorization of each polynomial below by first dividing the function by $x-1$.  Then sketch a graph of the function, making sure to have a clearly defined scale and label any intercepts.  Use \Desmos \ to compare your results.
%\begin{oddenumerate}[resume]
	\item $f(x) = $ \polyfactorize{x^3 - 2x^2 - 5x + 6}
%	\item $f(x) = x^{3} + 4x^{2} - 11x + 6$
	\item $f(x)= $ \polyfactorize{x^5-x^4-37x^3+37x^2+36x-36}
%	\item $f(x) = x^{4} -2x^3+ 5x^{2} - 8x + 4$
%\end{oddenumerate}
%Use the Rational Root Theorem and polynomial division to get a complete factorization of each polynomial function below.  Then sketch a graph of the function, making sure to have a clearly defined scale and label any intercepts.  Use \Desmos \ to compare your results.
%\begin{oddenumerate}[resume]
	\item $f(x) = $ \polyfactorize{x^4 - 9x^2 - 4x + 12}
%	\item $f(x) = x^{4} + 2x^{3} - 12x^{2} - 40x - 32$
	\item $f(x) = $ \polyfactorize{2x^4+x^3-7x^2-3x+3}
%	\item $f(x) = 3x^{3} + 3x^{2} - 11x - 10$
	\item $f(x) = $ \polyfactorize{6x^3+19x^2-6x-40}
%	\item $f(x) = -2x^{3} + 19x^{2} - 49x + 20$
	\item $f(x) = $ \polyfactorize{36x^4 - 12x^3 - 11x^2 + 2x + 1}
%	\item $f(x) = x^4+4x^3-x-4$	
	\item $f(x) = $ \polyfactorize{2x^3-5x^2-52x+60}
%	\item $f(x) = -x^3-x^2+39x+45$
	\item $f(x) = $ \polyfactorize{-2x^4+7x^3+17x^2-28x-36}
%	\item $f(x) = x^7-5x^6-24x^5+120x^4-25x^3+125x^2$
\end{oddenumerate}
\subsection*{Polynomial Inequalities}
%Solve each polynomial inequality below, expressing your answers using interval notation.  Use \Desmos \ to help confirm that each answer is correct.
\begin{multicols}{2}
\begin{oddenumerate}
 \item $(-\infty,\sqrt{2}]\cup [\sqrt{2},\infty)$ %$x^4 + x^2 \geq 6$
% \item $x^{4} - 9x^{2} \leq 4x - 12$
 \item $[1,\infty)$ %$4x^3 \geq 3x+1$
% \item $x^4 \leq 16+4x-x^3$
 \item $(-\infty,0)\cup (2,\infty)$ %$3x^2 + 2x < x^4$
% \item $\dfrac{x^3+2 x^2}{2} < x+2$
 \item $[4,\infty)$ %$\dfrac{x^3+20x}{8} \geq x^2 + 2$
% \item $19x^{2} + 20 > 2x^{3} + 49x $
 \item $(-\infty,4)$ %$x^3<4x^2$
% \item $x^3-7x^2\leq 12x-84$
 \item $(-\infty,-1]\cup [3,\infty)$ %$(x - 1)^{2} \geq 4$
% \item $2x^4>5x^2+3$
\end{oddenumerate}
\end{multicols}
\end{document}