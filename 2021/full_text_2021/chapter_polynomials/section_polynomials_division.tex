\documentclass[12pt]{book}
\raggedbottom
\usepackage[top=1in,left=1in,bottom=1in,right=1in,headsep=0.25in]{geometry}	
\usepackage{amssymb,amsmath,amsthm,amsfonts}
\usepackage{chapterfolder,docmute,setspace}
\usepackage{cancel,multicol,tikz,verbatim,framed,polynom,enumitem,tikzpagenodes}
\usepackage[colorlinks, hyperindex, plainpages=false, linkcolor=blue, urlcolor=blue, pdfpagelabels]{hyperref}
\usepackage[type={CC},modifier={by-sa},version={4.0},]{doclicense}

\theoremstyle{definition}
\newtheorem{example}{Example}
\newcommand{\Desmos}{\href{https://www.desmos.com/}{Desmos}}
\setlength{\parindent}{0in}
\setlist{itemsep=0in}
\setlength{\parskip}{0.1in}
\setcounter{secnumdepth}{0}
\input{lesson_order}

\newcommand{\tmmathbf}[1]{\ensuremath{\boldsymbol{#1}}}
\newcommand{\tmop}[1]{\ensuremath{\operatorname{#1}}}

\newcount\gpten % (global) power-of-ten -- tells which digit we are doing
\countdef\rtot2 % running total -- remainder so far
\countdef\LDscratch4 % scratch

\def\longdiv#1#2{%
 \vtop{\normalbaselines \offinterlineskip
   \setbox\strutbox\hbox{\vrule height 2.1ex depth .5ex width0ex}%
   \def\showdig{$\underline{\the\LDscratch\strut}$\cr\the\rtot\strut\cr
       \noalign{\kern-.2ex}}%
   \global\rtot=#1\relax
   \count0=\rtot\divide\count0by#2\edef\quotient{\the\count0}%\show\quotient
   % make list macro out of digits in quotient:
   \def\temp##1{\ifx##1\temp\else \noexpand\dodig ##1\expandafter\temp\fi}%
   \edef\routine{\expandafter\temp\quotient\temp}%
   % process list to give power-of-ten:
   \def\dodig##1{\global\multiply\gpten by10 }\global\gpten=1 \routine
   % to display effect of one digit in quotient (zero ignored):
   \def\dodig##1{\global\divide\gpten by10
      \LDscratch =\gpten
      \multiply\LDscratch  by##1%
      \multiply\LDscratch  by#2%
      \global\advance\rtot-\LDscratch \relax
      \ifnum\LDscratch>0 \showdig \fi % must hide \cr in a macro to skip it
   }%
   \tabskip=0pt
   \halign{\hfil##\cr % \halign for entire division problem
     $\quotient$\strut\cr
     #2$\,\overline{\vphantom{\big)}%
     \hbox{\smash{\raise3.5\fontdimen8\textfont3\hbox{$\big)$}}}%
     \mkern2mu \the\rtot}$\cr\noalign{\kern-.2ex}
     \routine \cr % do each digit in quotient
}}}

\begin{document}
\section{Division}
\subsection{Polynomial (Long) Division (L\arabic{lesson_polynomial_division})}
{\bf Objective: Apply polynomial division to a rational expression.}\par
Up until this point, every polynomial expression that we have encountered has either already been provided in its factored form or is easily factorable using one or more of the many techniques that we have learned.  We must be careful, however, to consider the very likely possibility that a given polynomial is not factorable using elementary methods (GCF, grouping, $ac-$method, etc.).  In many cases, obtaining a complete factorization can prove almost impossible without the aid of mathematical computating software.  Although, there is still one powerful tool that can help us to dissect certain factorable, yet formidable, polynomials.  This tool is known as the {\it Rational Root Theorem}, and we will see its use in a later section.
\par
In order to successfully employ the Rational Root Theorem, we first must understand polynomial division.  As we will see, dividing polynomials is a process very similar to long division of whole numbers.
\par
Before we begin with our first example, let's recall the terminology and format associated with division.
$$\dfrac{\text{dividend}}{\text{divisor}}=\text{quotient}+\dfrac{\text{remainder}}{\text{divisor}}$$
Alternatively, multiplying both sides of the above equation by the divisor, we have the following.
$$\dfrac{\text{dividend}}{\cancel{\text{divisor}}}\cdot\cancel{\text{divisor}}=\text{quotient}\cdot\text{divisor}+\dfrac{\text{remainder}}{\cancel{\text{divisor}}}\cdot\cancel{\text{divisor}}$$
$$\text{dividend}=\text{quotient}\cdot\text{divisor}+\text{remainder}$$
We begin with dividing a polynomial by a monomial, which simply utilizes the distributive property.  In the following example, we can think of the stated division as a distribution of the divisor (denominator) to each term in the dividend (numerator).  

\begin{example}\label{poly_div_1} Divide and simplify the following expressions.
\begin{multicols}{2}
\begin{enumerate}
\item $\dfrac{9x^5+6x^4-18x^3-24x^2}{3x^2}$
\item $\dfrac{8x^3+4x^2-2x+6}{4x^2}$
\end{enumerate}
\end{multicols}
\begin{enumerate}
	\item By distributing the expression $\dfrac{1}{3x^2}$ to each of the four terms in the numerator, our first expression becomes
$$\frac{9x^5+6x^4-18x^3-24x^2}{3x^2}=\frac{9x^5}{3x^2}+\frac{6x^4}{3x^2}-\frac{18x^3}{3x^2}-\frac{24x^2}{3x^2}.$$
We can then reduce each individual quotient to produce the following expression.
$$3x^3+2x^2-6x-8$$
In this case, our answer is $3x^3+2x^2-6x-8,$ and we can summarize our results as follows.
$$\frac{9x^5+6x^4-18x^3-24x^2}{3x^2}=3x^3+2x^2-6x-8$$
In this first example, our expression reduced completely, producing our quotient polynomial expression and a remainder of zero.
	\item Again, we will begin with the second expression by splitting it up, or distributing the denominator to each of the three terms in the numerator.  Reducing each individual quotient gives us our answer. 
\begin{equation*}
\begin{split}
\frac{8x^3+4x^2-2x+6}{4x^2}&=\frac{8x^3}{4x^2}+\frac{4x^2}{4x^2}-\frac{2x}{4x^2}+\frac{6}{4x^2}\\
&=2x+1-\frac{1}{2x}+\frac{3}{2x^2}
\end{split}
\end{equation*}
Unlike our first expression, here our expression does not reduce completely, i.e., it contains a nonzero remainder.  Because of this, since our answer includes two rational (or fractional) expressions, one could also combine them to form a single rational expression.
\begin{equation*}
\begin{split}
\frac{8x^3+4x^2-2x+6}{4x^2}&=2x+1-\frac{1}{2x}\cdot\frac{x}{x}+\frac{3}{2x^2}\\
&=2x+1-\frac{x}{2x^2}+\frac{3}{2x^2}\\
&=2x+1+\frac{3-x}{2x^2}
\end{split}
\end{equation*}
Furthermore, if we wanted to identify the remainder in this example, we could multiply both sides of the equation by the divisor, $4x^2$.
\begin{equation*}
\begin{split}
\frac{8x^3+4x^2-2x+6}{\cancel{4x^2}}\cdot \cancel{4x^2}&=(2x+1)\cdot{4x^2}+\left(\frac{3-x}{2\cancel{x^2}}\right)\cdot 4\cancel{x^2}\\
\underbrace{8x^3+4x^2-2x+6}_{\text{dividend}}&=\underbrace{(2x+1)}_{\text{quotient}}\underbrace{4x^2}_{\text{divisor}}+\underbrace{6-2x}_{\text{remainder}}
\end{split}
\end{equation*}
Lastly, one final observation to point out from this particular answer is the initial reduction of the second term, $\dfrac{4x^2}{4x^2},$ which equals one (not zero), and therefore does not simply disappear from the expression altogether.
\end{enumerate}
\end{example}

For division by polynomial expressions that contain more than a single term, long division is usually required.  To illustrate the relationship between polynomial division and standard numerical long division, an example with whole numbers is provided in order to review the (general) steps that will also be used for polynomial long division.
\newpage
\begin{example} Divide 631 by 4.
\begin{multicols}{2}

$$\longdiv{631}{4}$$
\columnbreak

Recalling the process associated with long division, we begin to construct our quotient (157) by comparing the divisor (4) with the largest placeholder of our dividend (631), then subtracting, and repeating this process until we have worked our way down to the ones digit.  We know we have finished, once we are left with a remainder (3) that is less than our divisor.
\end{multicols}
Expressing our answer in the form $$\dfrac{\text{dividend}}{\text{divisor}}=\text{quotient}+\dfrac{\text{remainder}}{\text{divisor}},$$
we have $\dfrac{631}{4}=157+\dfrac{3}{4}.$  Or, in the alternate form
$$\text{dividend}=\text{quotient}\cdot\text{divisor}+\text{remainder},$$
we have $631=145\cdot 4 +3.$
\end{example}

The general process for division of polynomials follows closely with that for dividing integers. The only real difference is in the terminology that we use: {\it term} in place of {\it number/digit}, and {\it degree} instead of {\it value}.

\framebox{
\begin{minipage}{0.9\linewidth}
\begin{center}
General Steps for Polynomial (Long) Division
\end{center}
Let $D(x)$ and $d(x)$ represent two nonzero polynomial functions.  The steps for simplifying the rational expression $\dfrac{D(x)}{d(x)}$ are as follows.
\begin{enumerate}
  \item Divide the leading term of the dividend $D$ by the leading term of the divisor $d$.  Label the resulting term $a_nx^n,$ and write it above the dividend.  This will be the leading term of the quotient, $q(x)$.
  \item Multiply $a_nx^n$ by the divisor, distribute, and simplify.  Label this as $d_1(x)$ and write it directly below the dividend, $D,$ making sure to align terms according to exponents.
  \item Subtract the resulting terms from the dividend.  Label the new expression $D_1$.
  \item Repeat steps (1)-(3) for the divisor $d$ and the new expression $D_i$ until the degree of $D_i$ is \textit{less than} the degree of the divisor.  Relabel the final new dividend as the remainder, $r(x)$.  The entire polynomial expression appearing above the original dividend is the quotient, $q(x)$.
\end{enumerate}
\begin{center}
\begin{tikzpicture}[xscale=1,yscale=1]
	\draw (0.5,0.5) node {$q(x)$};
	\draw (0,0) node {$d(x) \ )\overline{\ \ \ D(x) \ \ \ }$};
	\draw (0.4,-0.5) node {$- \ \underline{\ \ d_1(x)\ \ }$};
	\draw (1,-1) node {$D_1(x)$};
	\draw (0.8,-1.5) node {$- \ \underline{\ \ d_2(x)\ \ }$};
	\draw (1.5,-2) node {$\ddots$};
	\draw (1.8,-2.75) node {$-\underline{\ \ d_i(x) \ \ }$};
	\draw (2.75,-3.25) node {$D_i(x)=r(x)$};
\end{tikzpicture}
\end{center}
Step (3) above often tends to pose the greatest challenge for students.  It is important to keep in mind that we are are always subtracting the top term from the bottom term, which is why we must change the signs of the term(s) on the bottom.  In most cases, we will need to utilize the distributive property.
\end{minipage}
}

A basic example should clear up any confusion, and we begin by revisiting Example \ref{poly_div_1}.

\begin{example} Divide $9x^5+6x^4-18x^3-24x^2$ by $3x^2$.  Simplify and express your answer in the form
$$\frac{D(x)}{d(x)}=q(x)+\dfrac{r(x)}{d(x)}.$$
\begin{multicols}{2}
\polylongdiv[stage=1]{9x^5+6x^4-18x^3-24x^2}{3x^2}

\columnbreak

We set up our division process by first writing the dividend and the divisor in the appropriate locations.

\end{multicols}
\begin{multicols}{2}
\polylongdiv[stage=2]{9x^5+6x^4-18x^3-24x^2}{3x^2}

\columnbreak

Next, we identify the leading term for our quotient, $3x^3$.

\end{multicols}
\begin{multicols}{2}
\polylongdiv[stage=4]{9x^5+6x^4-18x^3-24x^2}{3x^2}

\columnbreak

Multiplying and subtracting produces our new expression, $D_1(x)=6x^4$.
\end{multicols}
Notice that in this example we need only carry down the remaining terms from the dividend when our new expression contains terms which are alike to them.  In this case, for example, while it is perfectly fine to carry down $-18x^3-24x^2,$ it is not necessary until these terms play a role in the subtraction step.
\begin{multicols}{2}

\polylongdiv[stage=7]{9x^5+6x^4-18x^3-24x^2}{3x^2}

\polylongdiv[stage=10]{9x^5+6x^4-18x^3-24x^2}{3x^2}

\columnbreak

Repeating our division steps gives us the second term in our quotient, $2x^2$.
\par
Multiplying this term by the divisor, subtracting, and carrying down the next term in the dividend finishes up the second round of our steps for division.  Our new expression is $D_2(x)=-18x^3$.
\par
Again, we repeat our division steps to produce the third term in our quotient, $-6x$.
\par
Multiplying and subtracting produces another new expression, $D_3(x)=-24x^2$.  Since the degree of $D_3$ equals that of our divisor, $d(x)=3x^2,$ we will need to apply our steps for division one final time.
\end{multicols}

\begin{multicols}{2}

\polylongdiv{9x^5+6x^4-18x^3-24x^2}{3x^2}

\columnbreak

After our fourth and final round of steps, our new expression produces a remainder of $r(x)=0$.  This should come as no real surprise, based upon our earlier calculations from Example \ref{poly_div_1}.  Many examples that we will encounter after this first one will not work out as nicely.
\par
We express the results of our division in the required form as follows.
\end{multicols}

$$\dfrac{9x^5+6x^4-18x^3-24x^2}{3x^2}=3x^3+2x^2-6x-8+\dfrac{0}{3x^2}$$

\end{example}

Because our divisor in this last example is a monomial, our steps do not require the distributive property.  The next example includes a binomial divisor, which will emphasize the importance of the distributive property in the division process.

\begin{example} Divide $3x^3-5x^2-32x+7$ by $x-4$.  Simplify and express your answer in the form
$$\frac{D(x)}{d(x)}=q(x)+\dfrac{r(x)}{d(x)}.$$

\begin{multicols}{2}
\polylongdiv{3x^3-5x^2-32x+7}{x-4}

\columnbreak

For this second example, we have opted to show the entire division process all at once.  Since our quotient,
$$3x^2+7x-4,$$
is a trinomial, we must apply the steps for division three times.
\par
In this second example, our remainder is the constant term $r(x)=-9,$ which has one degree less than our linear divisor, $d(x)=x-4$.
\end{multicols}
Our answer is $\dfrac{3x^3-5x^2-32x+7}{x-4}=3x^2+7x-4+\dfrac{-9}{x-4}.$
\par
If we look at the first round of long division steps, we see a leading term in our quotient of $3x^2$.  Multiplying by our divisor produces the following binomial expression.
$$3x^2(x-4)=3x^3-12x^2$$
Since we must subtract {\it each} term of this expression from our dividend, however, we see that the distributive property has already been applied.  Specifically,
$$-(3x^3-12x^2)=-3x^3+12x^2.$$
What remains now is to simply add each term to the respective term in the dividend, which, if employed correctly, should {\it always} eliminate the leading term in each round of steps.
\par
Failure to distribute a negative across {\it all} terms in this critical step will always lead to an incorrect quotient and remainder.  This is the most common mistake in the polynomial division process, and is one that often takes a long amount of time to correct.  Consequently, checking one's work is critical throughout the division process.  We emphasize these points in this first example where distribution of a negative is required, in the hopes that it is not overlooked.
\end{example}
We continue with another example.
\begin{example} Divide $6x^3-8x^2+10x+103$ by $4+2x$.  Simplify and express your answer in the form
$$\frac{D(x)}{d(x)}=q(x)+\dfrac{r(x)}{d(x)}.$$

Before we begin, we wish to point out the important prerequisite for polynomial division that all expressions be written in {\it descending power order}.  In this case, we will start out by rewriting our divisor as $2x+4$.

\begin{multicols}{2}
\polylongdiv{6x^3-8x^2+10x+103}{2x+4}

\columnbreak

This example is similar to the previous one.  Specifically,
the divisor is a linear binomial, and the dividend has a degree of $3$.
\par
Consequently, the steps for polynomial division are employed three times, yielding a constant remainder.
\par
Our answer should be expressed as follows.
\end{multicols}
$$\frac{6x^3-8x^2+10x+103}{2x+4}=3x^2-10x+25+\frac{3}{2x+4}$$
\end{example}

In each of the previous two examples the dividend contained only nonzero coefficients.  In other words, no term was ``skipped over'' in the expression for $D(x)$.  Our last example will address the importance of keeping track of polynomial {\it place holders}, in the event that a specific term carries with it a zero coefficient, and is therefore omitted from the original expression for the dividend, $D(x)$.
\par

Our last example demonstrates the importance of these preliminary steps.
\begin{example}~~~Divide and simplify the given expression.
  \begin{eqnarray*}
    \frac{2 x^4 + 42x - 4 x^2}{x^2 + 3x} &  & \tmop{Reorder} \tmop{dividend};~
    \tmop{need} x^3 \tmop{term}, \tmop{add} 0 x^3
	\end{eqnarray*}
  \begin{eqnarray*}
    x^2 + 3x~ \overline{)~2 x^4 + \tmmathbf{0 x^3} - 4 x^2 + 42x} &  & \tmop{Divide} \tmop{the}
    \tmop{leading} \tmop{terms} : \frac{2 x^4}{x^2} = 2 x^2
	\end{eqnarray*}
	\begin{eqnarray*}
    \tmmathbf{2 x^2}~~~~~~~~~~~~~~~~~~~~~~~~  &  & \\
    x^2 + 3x~ \overline{)~2 x^4 + 0 x^3 - 4 x^2 + 42x} &  & \tmop{Multiply}
    \tmop{this} \tmop{term} \tmop{by} \tmop{divisor} : 2 x^2 (x^2 + 3x) = 2 x^4 +
    6 x^3\\
    \underline{\tmmathbf{- 2 x^4 - 6 x^3} }~~~~~~~~~~~~~~~~ &  & \tmop{Subtract,~changing~terms}\\
    \tmmathbf{- 6 x^3 - 4 x^2}~~~~~~~  &  & \tmop{Bring} \tmop{down} \tmop{the}
    \tmop{next} \tmop{term,~}-4x^2
   \end{eqnarray*}
  \begin{eqnarray*}
	  2 x^2 \tmmathbf{- 6 x}~~~~~~~~~~~~~~~~~~~~~  &  & \\
    x^2 + 3x~ \overline{)~2 x^4 + 0 x^3 - 4 x^2 + 42x}~~~ &  & \tmop{Repeat,}
    \tmop{divide} \tmop{new~leading~term~by~}x^2 : \frac{- 6 x^3}{x^2} = - 6 x\\
    \underline{- 2 x^4 - 6 x^3 }~~~~~~~~~~~~~~~~~~~ &  & \\
    - 6 x^3 - 4 x^2~~~~~~~~~~~ &  & \tmop{Multiply} \tmop{this} \tmop{term} \tmop{by}
    \tmop{divisor} : - 6 x (x^2 + 3x) = - 6 x^3 - 18 x^2\\
    \underline{\tmmathbf{+ 6 x^3 + 18 x^2} }~~~~~~~~~  &  & \tmop{Subtract,~changing~signs}\\
    \tmmathbf{14 x^2 + 42x} &  & \tmop{Bring} \tmop{down} \tmop{the} \tmop{next}
    \tmop{term,} 42x
		\end{eqnarray*}
  \begin{eqnarray*}
    2 x^2 - 6 x ~\tmmathbf{+14}~~~~~~~~~~~~~~  &  & \\
    x^2 + 3x~ \overline{)~2 x^4 + 0 x^3 - 4 x^2 + 42x}~~~ &  & \tmop{Repeat},
    \tmop{divide} \tmop{new~leading~term~by~} x^2: \frac{14 x^2}{x^2} = 14\\
    \underline{- 2 x^4 - 6 x^3 }~~~~~~~~~~~~~~~~~~~ &  & \\
    - 6 x^3 - 4 x^2~~~~~~~~~~~ &  & \\
    \underline{+ 6 x^3 + 18 x^2}~~~~~~~~~~  &  & \\
    14 x^2 + 42x~ &  & \tmop{Multiply} \tmop{this} \tmop{term} \tmop{by} \tmop{the}
    \tmop{divisor} : 14 (x^2 + 3x) = 14 x^2 + 42x\\
    \tmmathbf{\underline{- 14 x^2 - 42x}} &  & \tmop{Subtract,~changing~signs}\\
    0~~~ &  & \tmop{Zero} \tmop{remainder}\\
    &  & \\
    2 x^2 - 6 x + 14 &  & \tmop{Our} \tmop{solution}
  \end{eqnarray*}
\end{example}
So we have,
$$\frac{2 x^4 - 4 x^2 + 42x}{x^2+3x}~=~2 x^2 - 6 x + 14$$ 

It is important to take a moment to check each problem, to verify that the
exponents decrease incrementally and that none are skipped. 

This final example also illustrates that, just as with classic numerical long
division, sometimes our remainder will be zero.

\subsection{Synthetic Division (L\arabic{lesson_synthetic_division})}
{\bf Objective: Apply synthetic division to a rational expression.}\par
Next, we will introduce a method of division that can be used to streamline the polynomial division process and is often preferred over the more traditional long division method.  This method, known as {\it synthetic division}, although usually quicker than traditional polynomial division, this alternative method can only be implemented when the given divisor is {\it linear}.  Specifically, we will require $d(x)$ to be of the form $x-c$.\par
For our first example, we will divide $x^3+4x^2-5x-14$ by $x-2$, which one can check will produce a quotient of $x^2+6x+7$ and a remainder of zero using polynomial long division.
$$\frac{x^3+4x^2-5x-14}{x-2}~=~x^2+6x+7$$
The method of synthetic division focuses primarily on the coefficients of both the divisor and dividend.  We must still pay careful attention, however, to the powers of our exponents, which will serve as placeholders throughout the process.
To start the process, we will write our coefficients in what we will refer to as a {\it synthetic division tableau} prior to dividing.\par  
To divide $x^3+4x^2-5x-14$ by $x-2$, we first write $2$ in the place of the divisor since $2$ is zero of the factor $x-2$ and we write the coefficients of $x^3+4x^2-5x-14$ in for the dividend.  As our next step, we `bring down' the first coefficient of the dividend.
We will then multiply and add repeatedly.
\begin{center}
\begin{multicols}{2}
\polyhornerscheme[x=2,showbase=top,stage=1]{x^3+4x^2-5x-14}\\
\polyhornerscheme[x=2,showbase=top,stage=2]{x^3+4x^2-5x-14}
\end{multicols}
\end{center}

Next, take the $2$ from the divisor and multiply by the $1$ that was brought down to get $2$.  Write this underneath the $4$, then add to get $6$.

\begin{center}
\begin{multicols}{2}
\polyhornerscheme[x=2,showbase=top,stage=3]{x^3+4x^2-5x-14}\\
\polyhornerscheme[x=2,showbase=top,stage=4]{x^3+4x^2-5x-14}
\end{multicols}
\end{center}

Now multiply the $2$ from the divisor by the $6$ to get $12$, and add it to the $-5$ to get $7$.

\begin{center}
\begin{multicols}{2}
\polyhornerscheme[x=2,showbase=top,stage=5]{x^3+4x^2-5x-14}\\
\polyhornerscheme[x=2,showbase=top,stage=6]{x^3+4x^2-5x-14}
\end{multicols}
\end{center}

Finally, multiply the $2$ in the divisor by the $7$ to get $14$, and add it to the $-14$ to get $0$.

\begin{center}
\begin{multicols}{2}
\polyhornerscheme[x=2,showbase=top,stage=7]{x^3+4x^2-5x-14}\\
\polyhornerscheme[x=2,showbase=top,stage=8,resultstyle=\bf]{x^3+4x^2-5x-14}
\end{multicols}
\end{center}
The first three numbers in the last row of our tableau will be the coefficients of the desired quotient polynomial.  Remember, we started with a third degree polynomial and divided by a first degree polynomial, so the quotient will be a second degree polynomial.  Hence the quotient is $x^2+6x+7$.  The number in bold represents the remainder, which is zero in this case.\par
Due in large part to its speed, synthetic division is often a `tool of choice' for dividing polynomials by divisors of the form $x-c$.  It is important to reiterate that synthetic division will \emph{only} work for these kinds of divisors (linear divisors with leading coefficient 1), and we will need to use polynomial long division for divisors having degree larger than 1.\par
Another observation worth mentioning is that when a polynomial (of degree at least $1$) is divided by $x-c$, the result will be a quotient polynomial of exactly one less degree than the original polynomial.  This is a direct result of the divisor being a linear expression.\par
For a more complete understanding of the relationship between long and synthetic division, students are encouraged to trace each step in synthetic division back to its corresponding step in long division.\par
We conclude this section with three examples using synthetic division.  We will summarize each example using the form below.
$$\frac{\text{dividend}}{\text{divisor}}~=~\text{quotient~}+~\frac{\text{remainder}}{\text{divisor}}$$
\begin{example}~~~Use synthetic division to perform the following polynomial division.  Find the quotient and the remainder polynomials.
$$\frac{5x^3 - 2x^2 + 1}{x-3}$$
When setting up the synthetic division tableau, we need to enter $0$ for the coefficient of $x$ in the dividend as a placeholder, just like in polynomial division.
\begin{multicols}{2}
Setting up and working through the tableau gives us the following result.

\columnbreak

\begin{center}
\polyhornerscheme[x=3,showbase=top,resultstyle=\bf]{5x^3-2x^2+1}
\end{center}
\end{multicols}
Since the dividend was a third degree polynomial, the quotient is a quadratic polynomial with coefficients $5$, $13$ and $39$.  Our quotient is then $q(x) = 5x^2+13x+39$ and the remainder is $r(x) = 118$.\par
Putting this all together, we have the following equation.
$$\frac{5x^3 - 2x^2 + 1}{x-3}~=~5x^2+13x+39~+~\frac{118}{x-3}$$
\end{example}
\begin{example}~~~Use synthetic division to perform the following polynomial division.  Find the quotient and the remainder polynomials.
$$\frac{x^3+8}{x+2}$$
For this division, since we have a factor of $x+2$, we must use the zero of $x = -2$ to begin.
Here, we will once again stress that it is critical to take the time in order to ensure we have set the synthetic division tableau up correctly at the onset of the problem.  Failure to do so will result in an incorrect answer, as well as a considerable amount time spent re-doing the problem.
\begin{center}
\polyhornerscheme[x=-2,showbase=top,resultstyle=\bf]{x^3+8}
\end{center}
We then obtain a quotient of $q(x) = x^2-2x+4$ and remainder of $r(x) =0$. This gives us the following equation.
$$\frac{x^3+8}{x+2}~=~x^2-2x+4$$
This answer is a great reminder of the factoring rules for cubic polynomials that we outlined earlier in the chapter.
\end{example}
\begin{example}~~~Use synthetic division to perform the following polynomial division.  Find the quotient and the remainder polynomials.
$$\dfrac{4-8x-12x^2}{2x-3}$$
To divide $4-8x-12x^2$ by $2x-3$, two things must be done.  First, we write the dividend in descending powers of $x$ as $-12x^2-8x+4$.  Second, since synthetic division works only for factors of the form $x-c$, we factor $2x-3$ as $2\left(x-\frac{3}{2}\right)$.  Our strategy is to first divide $-12x^2-8x+4$ by $2$, to get $-6x^2-4x+2$.  Next, we divide by $x-\frac{3}{2}$.  The tableau becomes
\begin{center}
\polyhornerscheme[x=\frac{3}{2},showbase=top,resultstyle=\bf]{-6x^2-4x+2}
\end{center}
From this, we get a quotient of $q(x)=-6 x - 13$ and a remainder of
$r(x)=-\frac{35}{2}$.  This gives us the following equation.
$$\frac{-6x^2-4x+2}{x-\frac{3}{2}}~=~-6 x - 13 + \frac{-\frac{35}{2}}{x-\frac{3}{2}}$$
Multiplying both sides by of our equation by $\frac{2}{2}$ and distributing gives us our desired answer.
$$\frac{-12x^2-8x+4}{2x-3}~=~-6 x - 13 + \frac{-35}{2x-3}$$
\end{example}
Note that we could also multiply both sides of our last equation by $2x-3$ to obtain the following equation.
$$-12x^2-8x+4 = \left(2x-3\right) (-6 x - 13) - 35$$
While both of the forms above are certainly equivalent, the previous one may remind us of the familiar classic division algorithm for integers, shown below.
\begin{center}
dividend $=$ (divisor)$\cdot$(quotient) $+$ remainder
\end{center}
The first form, however, will be particularly useful when we graph more complicated rational functions in the next chapter.
\newpage
\end{document}