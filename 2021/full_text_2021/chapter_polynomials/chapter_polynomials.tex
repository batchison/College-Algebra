\documentclass[12pt]{book}
\raggedbottom
\usepackage[top=1in,left=1in,bottom=1in,right=1in,headsep=0.25in]{geometry}	
\usepackage{amssymb,amsmath,amsthm,amsfonts}
\usepackage{chapterfolder,docmute,setspace}
\usepackage{cancel,multicol,tikz,verbatim,framed,polynom,enumitem,tikzpagenodes}
\usepackage[colorlinks, hyperindex, plainpages=false, linkcolor=blue, urlcolor=blue, pdfpagelabels]{hyperref}
\usepackage[type={CC},modifier={by-nc-sa},version={4.0},]{doclicense}

\theoremstyle{definition}
\newtheorem{example}{Example}
\newcommand{\Desmos}{\href{https://www.desmos.com/}{Desmos}}
\setlength{\parindent}{0in}
\setlist{itemsep=0in}
\setlength{\parskip}{0.1in}
\setcounter{secnumdepth}{0}
% This document is used for ordering of lessons.  If an instructor wishes to change the ordering of assessments, the following steps must be taken:

% 1) Reassign the appropriate numbers for each lesson in the \setcounter commands included in this file.
% 2) Rearrange the \include commands in the master file (the file with 'Course Pack' in the name) to accurately reflect the changes.  
% 3) Rarrange the \items in the measureable_outcomes file to accurately reflect the changes.  Be mindful of page breaks when moving items.
% 4) Re-build all affected files (master file, measureable_outcomes file, and any lessons whose numbering has changed).

%Note: The placement of each \newcounter and \setcounter command reflects the original/default ordering of topics (linears, systems, quadratics, functions, polynomials, rationals).

\newcounter{lesson_solving_linear_equations}
\newcounter{lesson_equations_containing_absolute_values}
\newcounter{lesson_graphing_lines}
\newcounter{lesson_two_forms_of_a_linear_equation}
\newcounter{lesson_parallel_and_perpendicular_lines}
\newcounter{lesson_linear_inequalities}
\newcounter{lesson_compound_inequalities}
\newcounter{lesson_inequalities_containing_absolute_values}
\newcounter{lesson_graphing_systems}
\newcounter{lesson_substitution}
\newcounter{lesson_elimination}
\newcounter{lesson_quadratics_introduction}
\newcounter{lesson_factoring_GCF}
\newcounter{lesson_factoring_grouping}
\newcounter{lesson_factoring_trinomials_a_is_1}
\newcounter{lesson_factoring_trinomials_a_neq_1}
\newcounter{lesson_solving_by_factoring}
\newcounter{lesson_square_roots}
\newcounter{lesson_i_and_complex_numbers}
\newcounter{lesson_vertex_form_and_graphing}
\newcounter{lesson_solve_by_square_roots}
\newcounter{lesson_extracting_square_roots}
\newcounter{lesson_the_discriminant}
\newcounter{lesson_the_quadratic_formula}
\newcounter{lesson_quadratic_inequalities}
\newcounter{lesson_functions_and_relations}
\newcounter{lesson_evaluating_functions}
\newcounter{lesson_finding_domain_and_range_graphically}
\newcounter{lesson_fundamental_functions}
\newcounter{lesson_finding_domain_algebraically}
\newcounter{lesson_solving_functions}
\newcounter{lesson_function_arithmetic}
\newcounter{lesson_composite_functions}
\newcounter{lesson_inverse_functions_definition_and_HLT}
\newcounter{lesson_finding_an_inverse_function}
\newcounter{lesson_transformations_translations}
\newcounter{lesson_transformations_reflections}
\newcounter{lesson_transformations_scalings}
\newcounter{lesson_transformations_summary}
\newcounter{lesson_piecewise_functions}
\newcounter{lesson_functions_containing_absolute_values}
\newcounter{lesson_absolute_as_piecewise}
\newcounter{lesson_polynomials_introduction}
\newcounter{lesson_sign_diagrams_polynomials}
\newcounter{lesson_factoring_quadratic_type}
\newcounter{lesson_factoring_summary}
\newcounter{lesson_polynomial_division}
\newcounter{lesson_synthetic_division}
\newcounter{lesson_end_behavior_polynomials}
\newcounter{lesson_local_behavior_polynomials}
\newcounter{lesson_rational_root_theorem}
\newcounter{lesson_polynomials_graphing_summary}
\newcounter{lesson_polynomial_inequalities}
\newcounter{lesson_rationals_introduction_and_terminology}
\newcounter{lesson_sign_diagrams_rationals}
\newcounter{lesson_horizontal_asymptotes}
\newcounter{lesson_slant_and_curvilinear_asymptotes}
\newcounter{lesson_vertical_asymptotes}
\newcounter{lesson_holes}
\newcounter{lesson_rationals_graphing_summary}

\setcounter{lesson_solving_linear_equations}{1}
\setcounter{lesson_equations_containing_absolute_values}{2}
\setcounter{lesson_graphing_lines}{3}
\setcounter{lesson_two_forms_of_a_linear_equation}{4}
\setcounter{lesson_parallel_and_perpendicular_lines}{5}
\setcounter{lesson_linear_inequalities}{6}
\setcounter{lesson_compound_inequalities}{7}
\setcounter{lesson_inequalities_containing_absolute_values}{8}
\setcounter{lesson_graphing_systems}{9}
\setcounter{lesson_substitution}{10}
\setcounter{lesson_elimination}{11}
\setcounter{lesson_quadratics_introduction}{16}
\setcounter{lesson_factoring_GCF}{17}
\setcounter{lesson_factoring_grouping}{18}
\setcounter{lesson_factoring_trinomials_a_is_1}{19}
\setcounter{lesson_factoring_trinomials_a_neq_1}{20}
\setcounter{lesson_solving_by_factoring}{21}
\setcounter{lesson_square_roots}{22}
\setcounter{lesson_i_and_complex_numbers}{23}
\setcounter{lesson_vertex_form_and_graphing}{24}
\setcounter{lesson_solve_by_square_roots}{25}
\setcounter{lesson_extracting_square_roots}{26}
\setcounter{lesson_the_discriminant}{27}
\setcounter{lesson_the_quadratic_formula}{28}
\setcounter{lesson_quadratic_inequalities}{29}
\setcounter{lesson_functions_and_relations}{12}
\setcounter{lesson_evaluating_functions}{13}
\setcounter{lesson_finding_domain_and_range_graphically}{14}
\setcounter{lesson_fundamental_functions}{15}
\setcounter{lesson_finding_domain_algebraically}{30}
\setcounter{lesson_solving_functions}{31}
\setcounter{lesson_function_arithmetic}{32}
\setcounter{lesson_composite_functions}{33}
\setcounter{lesson_inverse_functions_definition_and_HLT}{34}
\setcounter{lesson_finding_an_inverse_function}{35}
\setcounter{lesson_transformations_translations}{36}
\setcounter{lesson_transformations_reflections}{37}
\setcounter{lesson_transformations_scalings}{38}
\setcounter{lesson_transformations_summary}{39}
\setcounter{lesson_piecewise_functions}{40}
\setcounter{lesson_functions_containing_absolute_values}{41}
\setcounter{lesson_absolute_as_piecewise}{42}
\setcounter{lesson_polynomials_introduction}{43}
\setcounter{lesson_sign_diagrams_polynomials}{44}
\setcounter{lesson_factoring_quadratic_type}{46}
\setcounter{lesson_factoring_summary}{45}
\setcounter{lesson_polynomial_division}{47}
\setcounter{lesson_synthetic_division}{48}
\setcounter{lesson_end_behavior_polynomials}{49}
\setcounter{lesson_local_behavior_polynomials}{50}
\setcounter{lesson_rational_root_theorem}{51}
\setcounter{lesson_polynomials_graphing_summary}{52}
\setcounter{lesson_polynomial_inequalities}{53}
\setcounter{lesson_rationals_introduction_and_terminology}{54}
\setcounter{lesson_sign_diagrams_rationals}{55}
\setcounter{lesson_horizontal_asymptotes}{56}
\setcounter{lesson_slant_and_curvilinear_asymptotes}{57}
\setcounter{lesson_vertical_asymptotes}{58}
\setcounter{lesson_holes}{59}
\setcounter{lesson_rationals_graphing_summary}{60}

\newcommand{\tmmathbf}[1]{\ensuremath{\boldsymbol{#1}}}
\newcommand{\tmop}[1]{\ensuremath{\operatorname{#1}}}

\newlist{oddenumerate}{enumerate}{1}
\setlist[oddenumerate]{start=0,label=\theoddenumeratei.}
\makeatletter
\renewcommand\theoddenumeratei{\@arabic{\numexpr2*\value{oddenumeratei}+1}}
\makeatother

\newcount\gpten % (global) power-of-ten -- tells which digit we are doing
\countdef\rtot2 % running total -- remainder so far
\countdef\LDscratch4 % scratch

\def\longdiv#1#2{%
 \vtop{\normalbaselines \offinterlineskip
   \setbox\strutbox\hbox{\vrule height 2.1ex depth .5ex width0ex}%
   \def\showdig{$\underline{\the\LDscratch\strut}$\cr\the\rtot\strut\cr
       \noalign{\kern-.2ex}}%
   \global\rtot=#1\relax
   \count0=\rtot\divide\count0by#2\edef\quotient{\the\count0}%\show\quotient
   % make list macro out of digits in quotient:
   \def\temp##1{\ifx##1\temp\else \noexpand\dodig ##1\expandafter\temp\fi}%
   \edef\routine{\expandafter\temp\quotient\temp}%
   % process list to give power-of-ten:
   \def\dodig##1{\global\multiply\gpten by10 }\global\gpten=1 \routine
   % to display effect of one digit in quotient (zero ignored):
   \def\dodig##1{\global\divide\gpten by10
      \LDscratch =\gpten
      \multiply\LDscratch  by##1%
      \multiply\LDscratch  by#2%
      \global\advance\rtot-\LDscratch \relax
      \ifnum\LDscratch>0 \showdig \fi % must hide \cr in a macro to skip it
   }%
   \tabskip=0pt
   \halign{\hfil##\cr % \halign for entire division problem
     $\quotient$\strut\cr
     #2$\,\overline{\vphantom{\big)}%
     \hbox{\smash{\raise3.5\fontdimen8\textfont3\hbox{$\big)$}}}%
     \mkern2mu \the\rtot}$\cr\noalign{\kern-.2ex}
     \routine \cr % do each digit in quotient
}}}

\begin{document}
%\tableofcontents
\chapter{Polynomials}
\documentclass[12pt]{book}
\raggedbottom
\usepackage[top=1in,left=1in,bottom=1in,right=1in,headsep=0.25in]{geometry}	
\usepackage{amssymb,amsmath,amsthm,amsfonts}
\usepackage{chapterfolder,docmute,setspace}
\usepackage{cancel,multicol,tikz,verbatim,framed,polynom,enumitem,tikzpagenodes}
\usepackage[colorlinks, hyperindex, plainpages=false, linkcolor=blue, urlcolor=blue, pdfpagelabels]{hyperref}
\usepackage[type={CC},modifier={by-sa},version={4.0},]{doclicense}

\theoremstyle{definition}
\newtheorem{example}{Example}
\newcommand{\Desmos}{\href{https://www.desmos.com/}{Desmos}}
\setlength{\parindent}{0in}
\setlist{itemsep=0in}
\setlength{\parskip}{0.1in}
\setcounter{secnumdepth}{0}
% This document is used for ordering of lessons.  If an instructor wishes to change the ordering of assessments, the following steps must be taken:

% 1) Reassign the appropriate numbers for each lesson in the \setcounter commands included in this file.
% 2) Rearrange the \include commands in the master file (the file with 'Course Pack' in the name) to accurately reflect the changes.  
% 3) Rarrange the \items in the measureable_outcomes file to accurately reflect the changes.  Be mindful of page breaks when moving items.
% 4) Re-build all affected files (master file, measureable_outcomes file, and any lessons whose numbering has changed).

%Note: The placement of each \newcounter and \setcounter command reflects the original/default ordering of topics (linears, systems, quadratics, functions, polynomials, rationals).

\newcounter{lesson_solving_linear_equations}
\newcounter{lesson_equations_containing_absolute_values}
\newcounter{lesson_graphing_lines}
\newcounter{lesson_two_forms_of_a_linear_equation}
\newcounter{lesson_parallel_and_perpendicular_lines}
\newcounter{lesson_linear_inequalities}
\newcounter{lesson_compound_inequalities}
\newcounter{lesson_inequalities_containing_absolute_values}
\newcounter{lesson_graphing_systems}
\newcounter{lesson_substitution}
\newcounter{lesson_elimination}
\newcounter{lesson_quadratics_introduction}
\newcounter{lesson_factoring_GCF}
\newcounter{lesson_factoring_grouping}
\newcounter{lesson_factoring_trinomials_a_is_1}
\newcounter{lesson_factoring_trinomials_a_neq_1}
\newcounter{lesson_solving_by_factoring}
\newcounter{lesson_square_roots}
\newcounter{lesson_i_and_complex_numbers}
\newcounter{lesson_vertex_form_and_graphing}
\newcounter{lesson_solve_by_square_roots}
\newcounter{lesson_extracting_square_roots}
\newcounter{lesson_the_discriminant}
\newcounter{lesson_the_quadratic_formula}
\newcounter{lesson_quadratic_inequalities}
\newcounter{lesson_functions_and_relations}
\newcounter{lesson_evaluating_functions}
\newcounter{lesson_finding_domain_and_range_graphically}
\newcounter{lesson_fundamental_functions}
\newcounter{lesson_finding_domain_algebraically}
\newcounter{lesson_solving_functions}
\newcounter{lesson_function_arithmetic}
\newcounter{lesson_composite_functions}
\newcounter{lesson_inverse_functions_definition_and_HLT}
\newcounter{lesson_finding_an_inverse_function}
\newcounter{lesson_transformations_translations}
\newcounter{lesson_transformations_reflections}
\newcounter{lesson_transformations_scalings}
\newcounter{lesson_transformations_summary}
\newcounter{lesson_piecewise_functions}
\newcounter{lesson_functions_containing_absolute_values}
\newcounter{lesson_absolute_as_piecewise}
\newcounter{lesson_polynomials_introduction}
\newcounter{lesson_sign_diagrams_polynomials}
\newcounter{lesson_factoring_quadratic_type}
\newcounter{lesson_factoring_summary}
\newcounter{lesson_polynomial_division}
\newcounter{lesson_synthetic_division}
\newcounter{lesson_end_behavior_polynomials}
\newcounter{lesson_local_behavior_polynomials}
\newcounter{lesson_rational_root_theorem}
\newcounter{lesson_polynomials_graphing_summary}
\newcounter{lesson_polynomial_inequalities}
\newcounter{lesson_rationals_introduction_and_terminology}
\newcounter{lesson_sign_diagrams_rationals}
\newcounter{lesson_horizontal_asymptotes}
\newcounter{lesson_slant_and_curvilinear_asymptotes}
\newcounter{lesson_vertical_asymptotes}
\newcounter{lesson_holes}
\newcounter{lesson_rationals_graphing_summary}

\setcounter{lesson_solving_linear_equations}{1}
\setcounter{lesson_equations_containing_absolute_values}{2}
\setcounter{lesson_graphing_lines}{3}
\setcounter{lesson_two_forms_of_a_linear_equation}{4}
\setcounter{lesson_parallel_and_perpendicular_lines}{5}
\setcounter{lesson_linear_inequalities}{6}
\setcounter{lesson_compound_inequalities}{7}
\setcounter{lesson_inequalities_containing_absolute_values}{8}
\setcounter{lesson_graphing_systems}{9}
\setcounter{lesson_substitution}{10}
\setcounter{lesson_elimination}{11}
\setcounter{lesson_quadratics_introduction}{16}
\setcounter{lesson_factoring_GCF}{17}
\setcounter{lesson_factoring_grouping}{18}
\setcounter{lesson_factoring_trinomials_a_is_1}{19}
\setcounter{lesson_factoring_trinomials_a_neq_1}{20}
\setcounter{lesson_solving_by_factoring}{21}
\setcounter{lesson_square_roots}{22}
\setcounter{lesson_i_and_complex_numbers}{23}
\setcounter{lesson_vertex_form_and_graphing}{24}
\setcounter{lesson_solve_by_square_roots}{25}
\setcounter{lesson_extracting_square_roots}{26}
\setcounter{lesson_the_discriminant}{27}
\setcounter{lesson_the_quadratic_formula}{28}
\setcounter{lesson_quadratic_inequalities}{29}
\setcounter{lesson_functions_and_relations}{12}
\setcounter{lesson_evaluating_functions}{13}
\setcounter{lesson_finding_domain_and_range_graphically}{14}
\setcounter{lesson_fundamental_functions}{15}
\setcounter{lesson_finding_domain_algebraically}{30}
\setcounter{lesson_solving_functions}{31}
\setcounter{lesson_function_arithmetic}{32}
\setcounter{lesson_composite_functions}{33}
\setcounter{lesson_inverse_functions_definition_and_HLT}{34}
\setcounter{lesson_finding_an_inverse_function}{35}
\setcounter{lesson_transformations_translations}{36}
\setcounter{lesson_transformations_reflections}{37}
\setcounter{lesson_transformations_scalings}{38}
\setcounter{lesson_transformations_summary}{39}
\setcounter{lesson_piecewise_functions}{40}
\setcounter{lesson_functions_containing_absolute_values}{41}
\setcounter{lesson_absolute_as_piecewise}{42}
\setcounter{lesson_polynomials_introduction}{43}
\setcounter{lesson_sign_diagrams_polynomials}{44}
\setcounter{lesson_factoring_quadratic_type}{46}
\setcounter{lesson_factoring_summary}{45}
\setcounter{lesson_polynomial_division}{47}
\setcounter{lesson_synthetic_division}{48}
\setcounter{lesson_end_behavior_polynomials}{49}
\setcounter{lesson_local_behavior_polynomials}{50}
\setcounter{lesson_rational_root_theorem}{51}
\setcounter{lesson_polynomials_graphing_summary}{52}
\setcounter{lesson_polynomial_inequalities}{53}
\setcounter{lesson_rationals_introduction_and_terminology}{54}
\setcounter{lesson_sign_diagrams_rationals}{55}
\setcounter{lesson_horizontal_asymptotes}{56}
\setcounter{lesson_slant_and_curvilinear_asymptotes}{57}
\setcounter{lesson_vertical_asymptotes}{58}
\setcounter{lesson_holes}{59}
\setcounter{lesson_rationals_graphing_summary}{60}

\newcommand{\tmmathbf}[1]{\ensuremath{\boldsymbol{#1}}}
\newcommand{\tmop}[1]{\ensuremath{\operatorname{#1}}}

\begin{document}
\section{Introduction and Terminology (L\arabic{lesson_polynomials_introduction})}
\begin{tikzpicture}[remember picture, overlay,shift=(current page text area.north east),scale=0.5]
\draw[step=1.0,gray,very thin,dotted] (-9.8,-7.8) grid (-0.2,1.8);		
\draw[very thick] (-10,-8) -- (-10,2) -- (0,2) -- (0,-8) -- (-10,-8);
\draw[] (-9.8,-7.8) -- (-9.8,1.8) -- (-0.2,1.8) -- (-0.2,-7.8) -- (-9.8,-7.8);
\draw[-] (-9.8,-3) -- coordinate (x axis mid) (-0.2,-3);
\draw[-] (-5,-7.8) -- coordinate (y axis mid) (-5,1.8);
\draw[<->] plot [domain=-9.1:-1.1, samples=100] (\x,{0.125*(\x+2)*(\x+7)^2-3});
\end{tikzpicture}%
{\bf Objective: Identify key features of and classify a polynomial by degree and number of nonzero terms.}\par
A {\it polynomial} in terms of a variable $x$ is a function of the form
$$f(x) = a_{n}x^{n} + a_{n-1}x^{n-1}+ ... + a_{2}x^2 + a_{1}x + a_{0},$$
where each {\it coefficient}, $a_{i}$, is a real number ($a_n\neq 0$) and the exponent, or {\it degree} of the polynomial, $n$, is a nonnegative integer.
\par
Examples of polynomials include: $f(x) = x^2 + 5$, $f(x)=x$ and $f(x) = -3x^7+4x^3-5x$.  Before classifying polynomials, we will take a moment to establish some key terminology. For our general polynomial above, the:
\begin{center}
\begin{tabular}{lcl}
{\it degree} & is & $n$\\
{\it coefficients} & are & $a_n,a_{n-1},\ldots,a_1,a_0$\\
{\it leading coefficient} & is & $a_n$\\
{\it leading term} & is & $a_nx^n$\\
{\it constant term} & is & $a_0x^0=a_0$.
\end{tabular}
\end{center}
A concrete example will help to clarify each of these terms.
\begin{example} Identify the degree, leading coefficient, leading term and constant term for the polynomial
$$f(x) = -19x^5+4x^4-6x+21.$$
The degree of this polynomial is $n=5$, since five is the greatest exponent.
\par
The leading term, which is the term that contains the greatest exponent (degree), is\\ $a_nx^n=-19x^5$.
\par
The leading coefficient is the real number being multiplied by $x^n$ in the leading term, namely $a_n=-19$.
\par
The constant term is $a_0=21$, which also represents the $y-$intercept for the graph of the given polynomial, just as it did in the chapter on quadratics.
\par
The complete set of coefficients for the given polynomial is
$$\{a_5=-19, \ a_4=4, \ a_3=0, \ a_2=0, \ a_1=-6, \ a_0=21\}.$$
\end{example}
It is important to point out the fact that the previous example contains no {\it cubic} or {\it quadratic} terms, since the respective coefficients are both zero.  This example demonstrates that not every polynomial will contain a nonzero coefficient for every term.  As another example, the {\it power function} $f(x)=x^{10}$ is also characterized as a polynomial having degree $n=10$, leading coefficient $a_{10}=1$, and trailing coefficients $a_i=0$ for $i=9,8,\ldots,1,0$.
\par
Before we can identify and begin to classify a polynomial, we may need to simplify the given expression for $x$, by distributing and combining all like terms.  The general form of a polynomial should be reminiscent of the standard form of a quadratic, with possibly more terms.  Hence the name ``polynomial'', meaning ``many terms''.
\par
The following example shows how to identify a polynomial after the necessary simplification has taken place.
\begin{example} Identify the degree, leading coefficient, leading term and constant term for the given polynomial function.
\begin{equation*}
\begin{split}
f(x) &= 3(x+1)(x-1)+4x^3+2x+3\\
& = 3(x^2-1)+4x^3+2x+3\\
& = 3x^2-3+4x^3+2x+3\\
& = 4x^3+3x^2+2x
\end{split}
\end{equation*}
Upon simplifying, we see that $f$ has degree $n=3$, since three is the greatest exponent.
\par
The leading term is $4x^3$ with a leading coefficient of $a_n=4$.
\par
Since no constant term is shown, $a_0=0$ is our constant term.
\end{example}
Now that we can identify the essential components of a polynomial, we will categorize polynomials based upon their degree, as well as the number of terms, after all necessary simplification.
\newpage
\begin{center}
{\bf Types of Polynomials}
\par
\begin{tabular}{ | c | c | c | } 
\hline
Degree & Type & Example \\ 
\hline
0 & Constant & $-1$ \\ 
\hline
1 & Linear & $2x+\sqrt{5}$ \\ 
\hline
2 & Quadratic & $5x^2 - 32x+2$ \\ 
\hline
3 & Cubic & $(-1/2)x^{3}$ \\ 
\hline
4 & Quartic & $-3x^{4} +2x^2+3x + 1$ \\ 
\hline
5  & Quintic & $-2x^5$ \\ 
\hline
6 or more  & $n^{\text{th}}$ Degree & $-2x^{7} + 52x^6 + 12$ \\ 
\hline
\end{tabular}
\end{center}
One point of note in the table above is the appearance of both rational and irrational coefficients $\left(-1/2 \ \text{and} \ \sqrt{5}\right)$.  The appearance of such coefficients is permissible in polynomials, since our coefficients $a_i$ are only required to be real numbers.  A coefficient containing the imaginary number $i=\sqrt{-1}$, on the other hand, is not permitted.
\par
\begin{center}
{\bf Polynomial Characterizations by Number of Nonzero Terms}
\par
\begin{tabular}{ | c | c | c | } 
\hline
Number of Terms & Name & Example \\ 
\hline
1 & Monomial & $4x^5$ \\ 
\hline
2 & Binomial & $2x^3 +1$ \\ 
\hline
3 & Trinomial & $-23x^{18} +4x^2+3x$ \\ 
\hline
4 & Tetranomial & $-23x^{18} +4x^2+3x + 1$ \\ 
\hline
5 or more & Polynomial & $-2x^4 + x^3 +15x^2-41x + 12$ \\ 
\hline
\end{tabular}
\end{center}
\begin{example} Describe the type and characterization (number of terms) of the polynomial function shown below.
$$f(x) = -19x^5+4x^4-6x+21$$
Polynomials are typically named by their degree first and then their number of terms.  The polynomial above is a {\it quintic tetranomial}; quintic because it is degree five and tetranomial because it contains four terms.
\end{example}
\begin{example} Describe the type and characterization (number of terms) of the polynomial function shown below.
$$f(x) = x^3+x^2$$
The polynomial above is a {\it cubic binomial}, since it has degree three and contains two terms. 
\end{example}
\begin{example} Describe the type and characterization (number of terms) of the polynomial function shown below.
$$f(x) = 21x^4+12x^2-3x^2-9x^2-22x^4$$
Upon simplifying, we see that the given polynomial reduces to $f(x)=-x^4$.  As a result, our polynomial is a quartic (degree four) monomial (one term).
\end{example}
This section ``sets the table'' for the basic terminology that will be used throughout the chapter.  In the next section, we will review some additional prerequisite factoring techniques which will be necessary for working with certain polynomials, and provide a brief summary of all factoring methods that have been discussed up to this point.  Once we have finished our review of factoring, we will be ready to begin the natural (albeit lengthy) method of analyzing and graphing a polynomial function.
\end{document}
\documentclass[12pt]{book}
\raggedbottom
\usepackage[top=1in,left=1in,bottom=1in,right=1in,headsep=0.25in]{geometry}	
\usepackage{amssymb,amsmath,amsthm,amsfonts}
\usepackage{chapterfolder,docmute,setspace}
\usepackage{cancel,multicol,tikz,verbatim,framed,polynom,enumitem,tikzpagenodes}
\usepackage[colorlinks, hyperindex, plainpages=false, linkcolor=blue, urlcolor=blue, pdfpagelabels]{hyperref}
\usepackage[type={CC},modifier={by-sa},version={4.0},]{doclicense}

\theoremstyle{definition}
\newtheorem{example}{Example}
\newcommand{\Desmos}{\href{https://www.desmos.com/}{Desmos}}
\setlength{\parindent}{0in}
\setlist{itemsep=0in}
\setlength{\parskip}{0.1in}
\setcounter{secnumdepth}{0}
% This document is used for ordering of lessons.  If an instructor wishes to change the ordering of assessments, the following steps must be taken:

% 1) Reassign the appropriate numbers for each lesson in the \setcounter commands included in this file.
% 2) Rearrange the \include commands in the master file (the file with 'Course Pack' in the name) to accurately reflect the changes.  
% 3) Rarrange the \items in the measureable_outcomes file to accurately reflect the changes.  Be mindful of page breaks when moving items.
% 4) Re-build all affected files (master file, measureable_outcomes file, and any lessons whose numbering has changed).

%Note: The placement of each \newcounter and \setcounter command reflects the original/default ordering of topics (linears, systems, quadratics, functions, polynomials, rationals).

\newcounter{lesson_solving_linear_equations}
\newcounter{lesson_equations_containing_absolute_values}
\newcounter{lesson_graphing_lines}
\newcounter{lesson_two_forms_of_a_linear_equation}
\newcounter{lesson_parallel_and_perpendicular_lines}
\newcounter{lesson_linear_inequalities}
\newcounter{lesson_compound_inequalities}
\newcounter{lesson_inequalities_containing_absolute_values}
\newcounter{lesson_graphing_systems}
\newcounter{lesson_substitution}
\newcounter{lesson_elimination}
\newcounter{lesson_quadratics_introduction}
\newcounter{lesson_factoring_GCF}
\newcounter{lesson_factoring_grouping}
\newcounter{lesson_factoring_trinomials_a_is_1}
\newcounter{lesson_factoring_trinomials_a_neq_1}
\newcounter{lesson_solving_by_factoring}
\newcounter{lesson_square_roots}
\newcounter{lesson_i_and_complex_numbers}
\newcounter{lesson_vertex_form_and_graphing}
\newcounter{lesson_solve_by_square_roots}
\newcounter{lesson_extracting_square_roots}
\newcounter{lesson_the_discriminant}
\newcounter{lesson_the_quadratic_formula}
\newcounter{lesson_quadratic_inequalities}
\newcounter{lesson_functions_and_relations}
\newcounter{lesson_evaluating_functions}
\newcounter{lesson_finding_domain_and_range_graphically}
\newcounter{lesson_fundamental_functions}
\newcounter{lesson_finding_domain_algebraically}
\newcounter{lesson_solving_functions}
\newcounter{lesson_function_arithmetic}
\newcounter{lesson_composite_functions}
\newcounter{lesson_inverse_functions_definition_and_HLT}
\newcounter{lesson_finding_an_inverse_function}
\newcounter{lesson_transformations_translations}
\newcounter{lesson_transformations_reflections}
\newcounter{lesson_transformations_scalings}
\newcounter{lesson_transformations_summary}
\newcounter{lesson_piecewise_functions}
\newcounter{lesson_functions_containing_absolute_values}
\newcounter{lesson_absolute_as_piecewise}
\newcounter{lesson_polynomials_introduction}
\newcounter{lesson_sign_diagrams_polynomials}
\newcounter{lesson_factoring_quadratic_type}
\newcounter{lesson_factoring_summary}
\newcounter{lesson_polynomial_division}
\newcounter{lesson_synthetic_division}
\newcounter{lesson_end_behavior_polynomials}
\newcounter{lesson_local_behavior_polynomials}
\newcounter{lesson_rational_root_theorem}
\newcounter{lesson_polynomials_graphing_summary}
\newcounter{lesson_polynomial_inequalities}
\newcounter{lesson_rationals_introduction_and_terminology}
\newcounter{lesson_sign_diagrams_rationals}
\newcounter{lesson_horizontal_asymptotes}
\newcounter{lesson_slant_and_curvilinear_asymptotes}
\newcounter{lesson_vertical_asymptotes}
\newcounter{lesson_holes}
\newcounter{lesson_rationals_graphing_summary}

\setcounter{lesson_solving_linear_equations}{1}
\setcounter{lesson_equations_containing_absolute_values}{2}
\setcounter{lesson_graphing_lines}{3}
\setcounter{lesson_two_forms_of_a_linear_equation}{4}
\setcounter{lesson_parallel_and_perpendicular_lines}{5}
\setcounter{lesson_linear_inequalities}{6}
\setcounter{lesson_compound_inequalities}{7}
\setcounter{lesson_inequalities_containing_absolute_values}{8}
\setcounter{lesson_graphing_systems}{9}
\setcounter{lesson_substitution}{10}
\setcounter{lesson_elimination}{11}
\setcounter{lesson_quadratics_introduction}{16}
\setcounter{lesson_factoring_GCF}{17}
\setcounter{lesson_factoring_grouping}{18}
\setcounter{lesson_factoring_trinomials_a_is_1}{19}
\setcounter{lesson_factoring_trinomials_a_neq_1}{20}
\setcounter{lesson_solving_by_factoring}{21}
\setcounter{lesson_square_roots}{22}
\setcounter{lesson_i_and_complex_numbers}{23}
\setcounter{lesson_vertex_form_and_graphing}{24}
\setcounter{lesson_solve_by_square_roots}{25}
\setcounter{lesson_extracting_square_roots}{26}
\setcounter{lesson_the_discriminant}{27}
\setcounter{lesson_the_quadratic_formula}{28}
\setcounter{lesson_quadratic_inequalities}{29}
\setcounter{lesson_functions_and_relations}{12}
\setcounter{lesson_evaluating_functions}{13}
\setcounter{lesson_finding_domain_and_range_graphically}{14}
\setcounter{lesson_fundamental_functions}{15}
\setcounter{lesson_finding_domain_algebraically}{30}
\setcounter{lesson_solving_functions}{31}
\setcounter{lesson_function_arithmetic}{32}
\setcounter{lesson_composite_functions}{33}
\setcounter{lesson_inverse_functions_definition_and_HLT}{34}
\setcounter{lesson_finding_an_inverse_function}{35}
\setcounter{lesson_transformations_translations}{36}
\setcounter{lesson_transformations_reflections}{37}
\setcounter{lesson_transformations_scalings}{38}
\setcounter{lesson_transformations_summary}{39}
\setcounter{lesson_piecewise_functions}{40}
\setcounter{lesson_functions_containing_absolute_values}{41}
\setcounter{lesson_absolute_as_piecewise}{42}
\setcounter{lesson_polynomials_introduction}{43}
\setcounter{lesson_sign_diagrams_polynomials}{44}
\setcounter{lesson_factoring_quadratic_type}{46}
\setcounter{lesson_factoring_summary}{45}
\setcounter{lesson_polynomial_division}{47}
\setcounter{lesson_synthetic_division}{48}
\setcounter{lesson_end_behavior_polynomials}{49}
\setcounter{lesson_local_behavior_polynomials}{50}
\setcounter{lesson_rational_root_theorem}{51}
\setcounter{lesson_polynomials_graphing_summary}{52}
\setcounter{lesson_polynomial_inequalities}{53}
\setcounter{lesson_rationals_introduction_and_terminology}{54}
\setcounter{lesson_sign_diagrams_rationals}{55}
\setcounter{lesson_horizontal_asymptotes}{56}
\setcounter{lesson_slant_and_curvilinear_asymptotes}{57}
\setcounter{lesson_vertical_asymptotes}{58}
\setcounter{lesson_holes}{59}
\setcounter{lesson_rationals_graphing_summary}{60}

\newcommand{\tmmathbf}[1]{\ensuremath{\boldsymbol{#1}}}
\newcommand{\tmop}[1]{\ensuremath{\operatorname{#1}}}

\begin{document}
\section{Sign Diagrams (L\arabic{lesson_sign_diagrams_polynomials})}
{\bf Objective: Construct a sign diagram for a given polynomial expression.}\par
If a polynomial function or expression is completely factored, it will be beneficial to us to construct a sign diagram for the polynomial, in order to answer questions about its graph and confirm any other findings.  Therefore, we devote this section to the construction of a sign diagram for a factored polynomial.  Note that expanded polynomials first require us to find a complete factorization prior to constructing a sign diagram.  This will require us to first employ factoring techniques and possibly polynomial division, which we reserve for a later section.
\par
Recall that the roots of a quadratic expression represent the dividers in its corresponding sign diagram.  This carries over directly to a polynomial expression.
\par
We begin with an example for quadratics.
\begin{example}\label{sign_diag_poly_0}
Construct a sign diagram for the polynomial function $f(x)=2x^2+3x-20$.
\par
Although our first example is not factored, we can apply the $ac$-method to quickly factor our function.
\begin{equation*}
\begin{split}
f(x) & = 2x^2+3x-20\\
& = 2x^2+8x-5x-20\\
& = 2x(x+4)-5(x+4)\\
& = (x+4)(2x-5)
\end{split}
\end{equation*}
This gives us two roots, $x=-4$ and $x=\frac{5}{2}$, which serve as the dividers in our accompanying diagram.  For our three test values, we will use $x=-5, 0,$ and $3$.
\begin{center}
\begin{tikzpicture}[xscale=1,yscale=1]
	\draw [<->](-6.25,0) -- coordinate (x axis mid) (4.75,0) node[below right] {$x$};
	\draw [-](-3.5,1) -- coordinate (y axis mid) (-3.5,-0.25) node[below] {$-4$};
	\draw [-](2,1) -- coordinate (y axis mid) (2,-0.25) node[below] {$3$};
	\draw (-5,-1) node {$x=-5$};
	\draw (-0.75,-1) node {$x=0$};
	\draw (3.5,-1) node {$x=3$};
	\draw (-5,0.5) node {$+$};
	\draw (-0.75,0.5) node {$-$};
	\draw (3.5,0.5) node {$+$};
	\draw (-5,-1.5) node {\footnotesize $(-)(-)$};
	\draw (-0.75,-1.5) node {\footnotesize $(+)(-)$};
	\draw (3.5,-1.5) node {\footnotesize $(+)(+)$};
\end{tikzpicture}
\end{center}
\end{example}
The previous example should be a familiar one, and one that we are comfortable with, since it ties in directly with the chapter on quadratics (degree-2 polynomials).  For polynomials with a degree of $n\geq 3,$ our diagram should look similar.  The primary exceptions will be number of factors in our expression, and the number of dividers in our diagram.  Again, we will focus primarily on polynomials which are already factored for our examples.
\begin{example}\label{sign_diag_poly_1}
Construct a sign diagram for the factored polynomial function $$g(x)=(x+2)(3x-1)(5-x).$$
\par
Our roots are $x=-2,\frac{1}{3},$ and $5$.  Consequently, the following diagram shows three dividers.
\begin{center}
\begin{tikzpicture}[xscale=1,yscale=1]
	\draw [<->](-4.25,0) -- coordinate (x axis mid) (7.25,0) node[below right] {$x$};
	\draw [-](-2,1) -- coordinate (y axis mid) (-2,-0.25) node[below] {$-2$};
	\draw [-](0.5,1) -- coordinate (y axis mid) (0.5,-0.25) node[below] {$\frac{1}{3}$};
	\draw [-](5,1) -- coordinate (y axis mid) (5,-0.25) node[below] {$5$};
	\draw (-3,-1) node {$x=-3$};
	\draw (-0.75,-1) node {$x=0$};
	\draw (2.75,-1) node {$x=1$};
	\draw (6,-1) node {$x=6$};
	\draw (-3,0.5) node {$+$};
	\draw (-0.75,0.5) node {$-$};
	\draw (2.75,0.5) node {$+$};
	\draw (6,0.5) node {$-$};
	\draw (-3,-1.75) node {\footnotesize $(-)(-)(+)$};
	\draw (-0.75,-1.75) node {\footnotesize $(+)(-)(+)$};
	\draw (2.75,-1.75) node {\footnotesize $(+)(+)(+)$};
	\draw (6,-1.75) node {\footnotesize $(+)(+)(-)$};
\end{tikzpicture}
\end{center}
\end{example}
For our next example, we will make a slight change to the function $g$ from the previous example, by including an extra factor of $x+2$.
\begin{example}\label{sign_diag_poly_2}
Construct a sign diagram for the factored polynomial function $$h(x)=(x+2)^2(3x-1)(5-x).$$
\par
Since the roots of $h$ equal those from $g$, our diagram will have the same dividers and test values.
\begin{center}
\begin{tikzpicture}[xscale=1,yscale=1]
	\draw [<->](-4.25,0) -- coordinate (x axis mid) (7.25,0) node[below right] {$x$};
	\draw [-](-2,1) -- coordinate (y axis mid) (-2,-0.25) node[below] {$-2$};
	\draw [-](0.5,1) -- coordinate (y axis mid) (0.5,-0.25) node[below] {$\frac{1}{3}$};
	\draw [-](5,1) -- coordinate (y axis mid) (5,-0.25) node[below] {$5$};
	\draw (-3,-1) node {$x=-3$};
	\draw (-0.75,-1) node {$x=0$};
	\draw (2.75,-1) node {$x=1$};
	\draw (6,-1) node {$x=6$};
	\draw (-3,0.5) node {$-$};
	\draw (-0.75,0.5) node {$-$};
	\draw (2.75,0.5) node {$+$};
	\draw (6,0.5) node {$-$};
	\draw (-3,-1.75) node {\footnotesize $(-)(+)$};
	\draw (-0.75,-1.75) node {\footnotesize $(-)(+)$};
	\draw (2.75,-1.75) node {\footnotesize $(+)(+)$};
	\draw (6,-1.75) node {\footnotesize $(+)(-)$};
\end{tikzpicture}
\end{center}
\end{example}
In the previous diagram, we see that each of our sign calculations have excluded the $(x+2)^2$ factor, since it will always contribute a positive sign and therefore has no impact on the end result.  For example, for the test value $x=-3,$ we get
$$(-)^2(-)(+)=\cancel{(-)^2}(-)(+),$$
which reduces to a negative sign.  This simplification in our sign calculation can be employed for any factor that appears in our function with an {\it even} exponent.
\par
Additionally, our last two diagrams look almost identical, with the lone exception being the sign associated with our first interval, $(-\infty,-2)$.  This should make some sense, however, since we only changed the factor associated with the root $x=-2$ from one example to the next.  The reason behind the change in diagram will become more clear to us as we explore polynomials further.
\par
For our last example, we will present both the sign diagram and the accompanying graph for the given polynomial.  Although the techniques to graphing a polynomial have not yet been discussed, for any function it is often helpful to utilize a graphing utility such as \Desmos, in order to better understand the makeup of the function and how its graph is related.
\begin{example}\label{sign_diag_poly_3}
Construct a sign diagram for the factored polynomial function $$f(x)=x(x+1)(x-2)^2(x^2+4).$$
Use \Desmos \ or a similar graphing utility to construct a graph of $f$.
\par
Before we get started, it is important to spend some time discussing the factorization of $f$.  Specifically, the factor of $x$ will contribute a root of $x=0$.  This is the only instance in which our diagram requires a divider at $x=0.$
\par
Additionally, the factor of $x^2+4$ is often misinterpreted.  By setting the expression equal to zero and solving for $x,$ we see that the factor contributes two {\it imaginary} roots at $x=\pm 2i$.  Furthermore, if we look more closely at this factor, we see that for any real number $x$, $x^2+4$ will always be positive.  Hence, this factor will have no impact on our sign diagram calculations, and will be omitted.  One should caution, however, that this factor does have an impact on the graph of $f$.
\par
We can now conclude that the set of roots for $f$ are $x=-1, 0,$ and $2$.  The accompanying diagram and graph are shown below.  As before, we have also omitted the factor of $(x-2)^2,$ since the squared factor will not impact our signs.
\begin{center}
\begin{tikzpicture}[xscale=1.5,yscale=0.5]
	\draw [<->](-3.25,0) -- coordinate (x axis mid) (4.25,0) node[below right] {$x$};
	\draw [<->](0,-8) -- coordinate (y axis mid) (0,11) node[above right] {$y$};
	%\draw [dashed, <->](1.5,-6.25) -- coordinate (y axis mid) (1.5,6.25) node[above right] {};
	\draw [<->] plot [domain=-1.161:2.362, samples=100] (\x,{\x*(\x+1)*(\x-2)^2*((\x)^2+4)});
	\foreach \x in {1,...,4}
		\draw (\x,1pt) -- (\x,-1pt)	node[anchor=north] {\scriptsize \x};
	\foreach \x in {-3,...,-1}
		\draw (\x,1pt) -- (\x,-1pt)	node[anchor=south] {\scriptsize \x};
	\foreach \y in {2,4,...,10}
		\draw (1pt,\y) -- (-1pt,\y)	node[anchor=east] {\scriptsize \y}; 
	\foreach \y in {-6,-4,...,-2}
		\draw (1pt,\y) -- (-1pt,\y)	node[anchor=west] {\scriptsize \y}; 
\end{tikzpicture}
\end{center}
\begin{center}
\begin{tikzpicture}[xscale=1.5,yscale=1]
	\draw [<->](-3.25,0) -- coordinate (x axis mid) (4.25,0) node[below right] {$x$};
	\draw [-](-1,1) -- coordinate (y axis mid) (-1,-0.25) node[below] {$-1$};
	\draw [-](0,1) -- coordinate (y axis mid) (0,-0.25) node[below] {$0$};
	\draw [-](2,1) -- coordinate (y axis mid) (2,-0.25) node[below] {$2$};
	\draw (-2,-1) node {$x=-2$};
	\draw (-0.5,-1) node {$x=-\frac{1}{2}$};
	\draw (1,-1) node {$x=1$};
	\draw (3,-1) node {$x=3$};
	\draw (-2,0.5) node {$+$};
	\draw (-0.5,0.5) node {$-$};
	\draw (1,0.5) node {$+$};
	\draw (3,0.5) node {$+$};
	\draw (-2,-1.75) node {\footnotesize $(-)(-)$};
	\draw (-0.5,-1.75) node {\footnotesize $(-)(+)$};
	\draw (1,-1.75) node {\footnotesize $(+)(+)$};
	\draw (3,-1.75) node {\footnotesize $(+)(+)$};
\end{tikzpicture}
\end{center}
\end{example}
By looking at the graph of our last example, one should begin to notice the relationship that the graph of a polynomial has with its precise makeup and, consequently, its sign diagram.  In particular, close attention should be paid to the nature of the graph of $f$ near its real roots.  In the case of $x=-1$ and $x=0$ in our last example, we see that the graph {\it crosses over} the $x-$axis.  Alternatively, our graph {\it turns around} or ``bounces off'' at the root $x=2$.  This difference in the local behavior of the graph of $f$ at its roots is not just a coincidence, but rather a consequence of the makeup of the function $f,$ as we will see when we discuss the {\it multiplicity} of the root of a polynomial in a later section.
\newpage
\end{document}
\documentclass[12pt]{book}
\raggedbottom
\usepackage[top=1in,left=1in,bottom=1in,right=1in,headsep=0.25in]{geometry}	
\usepackage{amssymb,amsmath,amsthm,amsfonts}
\usepackage{chapterfolder,docmute,setspace}
\usepackage{cancel,multicol,tikz,verbatim,framed,polynom,enumitem,tikzpagenodes}
\usepackage[colorlinks, hyperindex, plainpages=false, linkcolor=blue, urlcolor=blue, pdfpagelabels]{hyperref}
\usepackage[type={CC},modifier={by-sa},version={4.0},]{doclicense}

\theoremstyle{definition}
\newtheorem{example}{Example}
\newcommand{\Desmos}{\href{https://www.desmos.com/}{Desmos}}
\setlength{\parindent}{0in}
\setlist{itemsep=0in}
\setlength{\parskip}{0.1in}
\setcounter{secnumdepth}{0}
% This document is used for ordering of lessons.  If an instructor wishes to change the ordering of assessments, the following steps must be taken:

% 1) Reassign the appropriate numbers for each lesson in the \setcounter commands included in this file.
% 2) Rearrange the \include commands in the master file (the file with 'Course Pack' in the name) to accurately reflect the changes.  
% 3) Rarrange the \items in the measureable_outcomes file to accurately reflect the changes.  Be mindful of page breaks when moving items.
% 4) Re-build all affected files (master file, measureable_outcomes file, and any lessons whose numbering has changed).

%Note: The placement of each \newcounter and \setcounter command reflects the original/default ordering of topics (linears, systems, quadratics, functions, polynomials, rationals).

\newcounter{lesson_solving_linear_equations}
\newcounter{lesson_equations_containing_absolute_values}
\newcounter{lesson_graphing_lines}
\newcounter{lesson_two_forms_of_a_linear_equation}
\newcounter{lesson_parallel_and_perpendicular_lines}
\newcounter{lesson_linear_inequalities}
\newcounter{lesson_compound_inequalities}
\newcounter{lesson_inequalities_containing_absolute_values}
\newcounter{lesson_graphing_systems}
\newcounter{lesson_substitution}
\newcounter{lesson_elimination}
\newcounter{lesson_quadratics_introduction}
\newcounter{lesson_factoring_GCF}
\newcounter{lesson_factoring_grouping}
\newcounter{lesson_factoring_trinomials_a_is_1}
\newcounter{lesson_factoring_trinomials_a_neq_1}
\newcounter{lesson_solving_by_factoring}
\newcounter{lesson_square_roots}
\newcounter{lesson_i_and_complex_numbers}
\newcounter{lesson_vertex_form_and_graphing}
\newcounter{lesson_solve_by_square_roots}
\newcounter{lesson_extracting_square_roots}
\newcounter{lesson_the_discriminant}
\newcounter{lesson_the_quadratic_formula}
\newcounter{lesson_quadratic_inequalities}
\newcounter{lesson_functions_and_relations}
\newcounter{lesson_evaluating_functions}
\newcounter{lesson_finding_domain_and_range_graphically}
\newcounter{lesson_fundamental_functions}
\newcounter{lesson_finding_domain_algebraically}
\newcounter{lesson_solving_functions}
\newcounter{lesson_function_arithmetic}
\newcounter{lesson_composite_functions}
\newcounter{lesson_inverse_functions_definition_and_HLT}
\newcounter{lesson_finding_an_inverse_function}
\newcounter{lesson_transformations_translations}
\newcounter{lesson_transformations_reflections}
\newcounter{lesson_transformations_scalings}
\newcounter{lesson_transformations_summary}
\newcounter{lesson_piecewise_functions}
\newcounter{lesson_functions_containing_absolute_values}
\newcounter{lesson_absolute_as_piecewise}
\newcounter{lesson_polynomials_introduction}
\newcounter{lesson_sign_diagrams_polynomials}
\newcounter{lesson_factoring_quadratic_type}
\newcounter{lesson_factoring_summary}
\newcounter{lesson_polynomial_division}
\newcounter{lesson_synthetic_division}
\newcounter{lesson_end_behavior_polynomials}
\newcounter{lesson_local_behavior_polynomials}
\newcounter{lesson_rational_root_theorem}
\newcounter{lesson_polynomials_graphing_summary}
\newcounter{lesson_polynomial_inequalities}
\newcounter{lesson_rationals_introduction_and_terminology}
\newcounter{lesson_sign_diagrams_rationals}
\newcounter{lesson_horizontal_asymptotes}
\newcounter{lesson_slant_and_curvilinear_asymptotes}
\newcounter{lesson_vertical_asymptotes}
\newcounter{lesson_holes}
\newcounter{lesson_rationals_graphing_summary}

\setcounter{lesson_solving_linear_equations}{1}
\setcounter{lesson_equations_containing_absolute_values}{2}
\setcounter{lesson_graphing_lines}{3}
\setcounter{lesson_two_forms_of_a_linear_equation}{4}
\setcounter{lesson_parallel_and_perpendicular_lines}{5}
\setcounter{lesson_linear_inequalities}{6}
\setcounter{lesson_compound_inequalities}{7}
\setcounter{lesson_inequalities_containing_absolute_values}{8}
\setcounter{lesson_graphing_systems}{9}
\setcounter{lesson_substitution}{10}
\setcounter{lesson_elimination}{11}
\setcounter{lesson_quadratics_introduction}{16}
\setcounter{lesson_factoring_GCF}{17}
\setcounter{lesson_factoring_grouping}{18}
\setcounter{lesson_factoring_trinomials_a_is_1}{19}
\setcounter{lesson_factoring_trinomials_a_neq_1}{20}
\setcounter{lesson_solving_by_factoring}{21}
\setcounter{lesson_square_roots}{22}
\setcounter{lesson_i_and_complex_numbers}{23}
\setcounter{lesson_vertex_form_and_graphing}{24}
\setcounter{lesson_solve_by_square_roots}{25}
\setcounter{lesson_extracting_square_roots}{26}
\setcounter{lesson_the_discriminant}{27}
\setcounter{lesson_the_quadratic_formula}{28}
\setcounter{lesson_quadratic_inequalities}{29}
\setcounter{lesson_functions_and_relations}{12}
\setcounter{lesson_evaluating_functions}{13}
\setcounter{lesson_finding_domain_and_range_graphically}{14}
\setcounter{lesson_fundamental_functions}{15}
\setcounter{lesson_finding_domain_algebraically}{30}
\setcounter{lesson_solving_functions}{31}
\setcounter{lesson_function_arithmetic}{32}
\setcounter{lesson_composite_functions}{33}
\setcounter{lesson_inverse_functions_definition_and_HLT}{34}
\setcounter{lesson_finding_an_inverse_function}{35}
\setcounter{lesson_transformations_translations}{36}
\setcounter{lesson_transformations_reflections}{37}
\setcounter{lesson_transformations_scalings}{38}
\setcounter{lesson_transformations_summary}{39}
\setcounter{lesson_piecewise_functions}{40}
\setcounter{lesson_functions_containing_absolute_values}{41}
\setcounter{lesson_absolute_as_piecewise}{42}
\setcounter{lesson_polynomials_introduction}{43}
\setcounter{lesson_sign_diagrams_polynomials}{44}
\setcounter{lesson_factoring_quadratic_type}{46}
\setcounter{lesson_factoring_summary}{45}
\setcounter{lesson_polynomial_division}{47}
\setcounter{lesson_synthetic_division}{48}
\setcounter{lesson_end_behavior_polynomials}{49}
\setcounter{lesson_local_behavior_polynomials}{50}
\setcounter{lesson_rational_root_theorem}{51}
\setcounter{lesson_polynomials_graphing_summary}{52}
\setcounter{lesson_polynomial_inequalities}{53}
\setcounter{lesson_rationals_introduction_and_terminology}{54}
\setcounter{lesson_sign_diagrams_rationals}{55}
\setcounter{lesson_horizontal_asymptotes}{56}
\setcounter{lesson_slant_and_curvilinear_asymptotes}{57}
\setcounter{lesson_vertical_asymptotes}{58}
\setcounter{lesson_holes}{59}
\setcounter{lesson_rationals_graphing_summary}{60}

\newcommand{\tmmathbf}[1]{\ensuremath{\boldsymbol{#1}}}
\newcommand{\tmop}[1]{\ensuremath{\operatorname{#1}}}

\begin{document}
\section{Factoring}
\subsection{Some Special Cases (L\arabic{lesson_factoring_summary})}
{\bf Objective: Factor a general polynomial expression using one or more factorization methods.}\par
When factoring polynomials there are a few special products that, if we can recognize them, can be easily broken down. The first is one we have seen before, when factoring some quadratics in which there is no linear term.
\par
When expanding, we know that the product of a sum and difference of the same two terms results in a difference of two squares.
\begin{center}
\framebox{
\begin{minipage}{0.75\linewidth}
\begin{center}
Difference of Two Squares: $a^2-b^2=\left(a+b\right)\left(a-b\right)$
\end{center}
\end{minipage}
}
\end{center}
Consequently, if faced with the difference of two squares, one can conclude that such an expression will always factor as a product of the
sum and difference of their square roots.  Our first four examples demonstrate this fact.
\begin{example}Factor each of the given binomial expressions completely over the real numbers.
	\begin{multicols}{4}
	\begin{enumerate}
		\item $x^2-16$
		\item $9x^2-25y^2$ 
		\item $x^2-24$
		\item $2x^2-5$
	\end{enumerate}
	\end{multicols}
	\begin{enumerate}
	\item In this first example, we see that $a=x$ and $b=4$ for our difference of two squares.
  \begin{equation*}
		\begin{split}
    x^2-16 &= \left(x\right)^2-\left(4\right)^2\\
					 &= \left(x+4\right)\left(x-4\right)
		\end{split}
	\end{equation*}
	\item Taking the square roots of $9x^2$ and $25y^2$ gives us $a=3x$ and $b=5y$ for our second expression.
	\begin{equation*}
		\begin{split}
    9x^2-25y^2 &= \left(3x\right)^2-\left(5y\right)^2\\
					 &= \left(3x+5y\right)\left(3x-5y\right)
		\end{split}
	\end{equation*}
	\item Our third expression poses a bit of a challenge, since it is the first which does not present us with the difference of two {\it perfect} squares.  In this case, $a=x,$ but $b=\sqrt{24}=\sqrt{4\cdot 6}=2\sqrt{6}$.
	\begin{equation*}
		\begin{split}
    x^2-24 &= \left(x\right)^2-\left(2\sqrt{6}\right)^2\\
					 &= \left(x+2\sqrt{6}\right)\left(x-2\sqrt{6}\right)
		\end{split}
	\end{equation*}
\newpage
	\item Similarly, our final expression presents us with two terms, neither of which are perfect squares.  In this case, $a=\sqrt{2x^2}=\sqrt{2}x$ and $b=\sqrt{5}$.
	\begin{equation*}
		\begin{split}
    2x^2-5 &= \left(\sqrt{2}x\right)^2-\left(\sqrt{5}\right)^2\\
					 &= \left(\sqrt{2}x+\sqrt{5}\right)\left(\sqrt{2}x-\sqrt{5}\right)
		\end{split}
	\end{equation*}
	Note that in this last case, we have $\sqrt{2}x$ (or $x\sqrt{2}$) in our factorization, and not $\sqrt{2x}$.
	\end{enumerate}
\end{example}
It is important to note that, unlike differences, a {\it sum} of squares will never factor over the real numbers.  Such expressions only factor over the complex numbers.  Hence, we say that they are {\it irreducible} over the reals. This can be seen in our next example, where we will attempt to employ the $ac-$method to factor.
\begin{example} Factor the expression $x^2+36$ completely over the real numbers and over the complex numbers.
\par
For the expression $x^2+36,$ $ac=36$ and $b=0,$ as we have no linear term.  So we need to identify two integers, $m$ and $n$, such that $m+n=0$ and $m\cdot n=36$.  Our choices for $m\cdot n$ are $1 \cdot 36$, $2 \cdot 18$, $3 \cdot 12$, $4 \cdot 9$ and $6 \cdot 6$.  But, since there are no combinations from these that will both multiply to 36 {\it and} add to 0, we conclude that the given expression is irreducible over the reals.
\par
Notice that $x^2+36$ does, however, factor over the complex numbers.
	\begin{equation*}
	\begin{split}
	x^2+36 &= x^2-\left(-36\right)\\
	&= x^2-\left(\sqrt{-36}\right)^2\\
	&= x^2-\left(\sqrt{36}\sqrt{-1}\right)^2\\
	&= x^2-\left(6i\right)^2\\
	&= \left(x-6i\right)\left(x+6i\right)
	\end{split}
	\end{equation*}
We can further make sense of this result by recalling the methods from the chapter on quadratics.  Since the discriminant of $x^2+36$ is 
\begin{equation*}
	\begin{split}
		b^2-4ac&=0^2-4\left(1\right)\left(36\right)\\
		&=-144\\
		&<0,
	\end{split}
\end{equation*}
we know that the given expression has no real roots.  Hence, any factorization must contain imaginary numbers.  By setting the expression equal to zero and extracting square roots, we get $x=\pm 6i,$ which further supports our factorization.
\end{example}

We present the general factorization for the sum of two squares over the complex numbers below.
\begin{center}
\framebox{
\begin{minipage}{0.75\linewidth}
\begin{center}
Sum of Two Squares: $a^2+b^2=\left(a+bi\right)\left(a-bi\right)$
\end{center}
\end{minipage}
}
\end{center}
\newpage
For graphing purposes, we will primarily be concerned with factorization over the real numbers.
\par 
In many cases, we can also recognize an entire expression as a perfect square (or a squared binomial).
\begin{center}
\framebox{
\begin{minipage}{0.75\linewidth}
\begin{center}
Perfect Square: $a^2+2ab+b^2=\left(a+b\right)^2$
\end{center}
\end{minipage}
}
\end{center}
While it might seem difficult to recognize a perfect square at first glance, by employing the $ac-$method, we can see that in the case where $m=n,$ the resulting factorization will be a perfect square. In this case, we can factor by identifying the square roots of the first and last
terms and using the sign from the middle term. This is demonstrated in the following example.
\begin{example} Factor each of the given trinomial expressions completely over the real numbers.
  \begin{multicols}{2}
	\begin{enumerate}
	\item $x^2-6x+9$
	\item $4x^2+20xy+25y^2$
	\end{enumerate}
	\end{multicols}
	\begin{enumerate}
	\item For our first expression, $a=1, \ b=-6,$ and $c=9$.  So we must find two integers $m$ and $n$ such that $m+n=-6$ and $mn=ac=9$.  In this case, the numbers we need are $-3$ and $-3$.  Consequently, we will have a perfect square.
	\par
	Using the square roots of $a=1$ and $c=9$ and the negative sign from the linear term, our factorization is 
	$$x^2-6x+9=\left(x-3\right)^2.$$
	\item For our second expression, $a=4, \ b=20,$ and $c=25$.  So we are looking for an $m$ and $n$ such that $m+n=20$ and $mn=ac=100$.  Quickly, we see that $m=n=10,$ and again, we have a perfect square.
	\par
	In this case, our factorization is
	$$4x^2+20xy+25y^2=\left(2x+5y\right)^2.$$
	\end{enumerate}
\end{example}
Another factoring shortcut involves sums and differences of cubes.  Both sums and differences of cubes have very similar factorizations.
\begin{center}
\framebox{
\begin{minipage}{0.75\linewidth}
\begin{center}
Sum of Cubes: $a^3+b^3=\left(a+b\right)\left(a^2-ab+b^2\right)$
\par
Difference of Cubes: $a^3-b^3=\left(a-b\right)\left(a^2+ab+b^2\right)$
\end{center}
\end{minipage}
}
\end{center}
As with all of the formulas in this section, we can verify those for a sum and difference of cubes by expanding the right-hand side.  For example, for the difference of cubes, 
\begin{equation*}
\begin{split}
\left(a-b\right)\left(a^2+ab+b^2\right)&= a\left(a^2+ab+b^2\right)-b\left(a^2+ab+b^2\right)\\
&=a^3+\cancel{a^2b}+\bcancel{ab^2}-\cancel{a^2b}-\bcancel{ab^2}-b^3\\
&=a^3-b^3
\end{split}
\end{equation*}
\newpage
Comparing the formulas one may notice that the only difference resides in the signs between the terms. One way to remember these two formulas is to think of ``{\bf SOAP}'':
\begin{center}
\begin{tabular}{cl}
{\bf S} & The first sign in our factorization is the {\bf Same} sign as the given expression.\\
{\bf O} & The second sign in our factorization is the {\bf Opposite} sign as the given expression.\\
{\bf AP} & The last sign in our factorization is {\bf Always Positive}.
\end{tabular}
\end{center}
\begin{example} Factor each of the given binomial expressions completely over the real numbers.
  \begin{multicols}{2}
		\begin{enumerate}
			\item $m^3-27$
			\item $125p^3+8r^3$
		\end{enumerate}
	\end{multicols}
	\begin{enumerate}
		\item In our first expression, our desired cube roots for each term are $a=m$ and $b=3$.  Using the ``Same, Opposite, Always Positive'' acronym, we have the following factorization.
		\begin{equation*}
			\begin{split}
				m^3-27&=\left(m-3\right)\left(\left(m\right)^2+3m+\left(3\right)^2\right)\\
				&=\left(m-3\right)\left(m^2+3m+9\right)
			\end{split}
		\end{equation*}
		\item In second expression, our desired cube roots for each term are 
		\begin{multicols}{2}
			\begin{equation*}
				\begin{split}
					a&=\sqrt[3]{125p^3}\\
					 &=\sqrt[3]{125}\sqrt[3]{p^3}\\
					 &=5p
				\end{split}
			\end{equation*}

			\columnbreak

			\begin{equation*}
				\begin{split}
					b&=\sqrt[3]{8r^3}\\
					 &=\sqrt[3]{8}\sqrt[3]{r^3}\\
					 &=2r
				\end{split}
			\end{equation*}
		\end{multicols}
		Using the ``Same, Opposite, Always Positive'' acronym, we have the following factorization.
		\begin{equation*}
			\begin{split}
				125p^3+8r^3&=\left(5p+2r\right)\left(\left(5p\right)^2-\left(5p\right)\left(2r\right)+\left(2r\right)^2\right)\\
				&=\left(5p+2r\right)\left(25p^2-10pr+4r^2\right)
			\end{split}
		\end{equation*}
	\end{enumerate}
\end{example}
The second expression in our last example illustrates an important point.  When we identify the first and last terms of the trinomial in our factorization, we must square each cube root in its entirety.  In this case, both the coefficients and variables are squared, so that $\left(5p\right)^2$ becomes $25p^2,$ and $\left(2r\right)^2$ becomes $4r^2$.
\par
After factoring a sum or difference of cubes, it should be natural to attempt to factor the resulting trinomial expression (our second factor). As a general rule, however, this factor should always be irreducible over the reals, with the main exception being that of a GCF in the given expression that might have been initially overlooked.
\par
Our last special case comes up frequently enough that we will devote the next subsection to it.
\subsection{Quadratic Type (L\arabic{lesson_factoring_quadratic_type})}
{\bf Objective: Recognize and factor a polynomial expression of quadratic type.}\par
Recall that a quadratic expression in terms of a variable $x$ is an expression of the form $$ax^2+bx+c.$$
If $y$ is any algebraic expression, we say that the expression $$ay^2+by+c$$ is an expression of {\it quadratic type}.
\par
In just about every case we will see, we will consider $y$ as a power of $x,$ $y=x^n,$ so that our expression of quadratic type will appear as follows.
\begin{center}
\framebox{
\begin{minipage}{0.75\linewidth}
\begin{center}
Quadratic Type:
$$ax^{2n}+bx^n+c \ = \ a\left[x^n\right]^2+b\left[x^n\right]+c$$
\end{center}
\end{minipage}
}
\end{center}
If $y=x^3,$ then the expression $$ay^2+by+c=ax^6+bx^3+c$$
would be an expression of quadratic type.
\par
Similarly, if $y=x^4,$ then the expression $$ay^2+by+c=ax^8+bx^4+c$$
would be an expression of quadratic type.
\par
In each of these last two examples, notice the exponential pattern, where the middle term has an exponent that is half that of the leading term's.  This will always be apparent, as long as the middle coefficient $b$ is nonzero.
\par
By viewing certain expressions as quadratic type, we can often apply more familiar methods, such as the $ac-$method, to obtain a complete factorization.
\par
For example, if we let $y=x^2,$ then the difference of fourth powers $x^4-16$ can be rewritten as a difference of squares,
$y^2-4^2,$ leading us to the complete factorization over the real numbers shown below.
\begin{equation*}
	\begin{split}
		x^4-16&=\left(x^2\right)^2-4^2\\
		&=y^2-4^2, \ \ y=x^2\\
		&=\left(y+4\right)\left(y-4\right)\\
		&=\left(x^2+4\right)\left(x^2-4\right)\\
		&=\left(x^2+4\right)\left(x+2\right)\left(x-2\right)
	\end{split}
\end{equation*}
\begin{example}\label{quad_type_1} Factor the trinomial expression $x^4+2x^2-24$ completely over the real numbers.
\par
Notice that the given trinomial exhibits quadratic type characteristics, since the degree of four is double the exponent appearing in the middle term.  Consequently, we will let $y=x^2$ and rewrite the expression in terms of $y$.
$$y^2+2y-24$$
Applying the $ac-$method, we see the following.
\begin{equation*}
\begin{split}
y^2+2y-24 & = y^2+6y-4y-24\\
&=y\left(y+6\right)-4\left(y+6\right)\\
&=\left(y+6\right)\left(y-4\right)
\end{split}
\end{equation*}
Substituting back for $x,$ we have $\left(x^2+6\right)\left(x^2-4\right)$.  The first factor is a sum of squares, which is irreducible over the reals.  The second factor of $x^2-4$ is a difference of perfect squares, which we know is factorable as $\left(x+2\right)\left(x-2\right)$.
\par
Our final factorization is
$$x^4+2x^2-24=\left(x^2+6\right)\left(x+2\right)\left(x-2\right).$$
\end{example}
\begin{example} Find all real roots of the polynomial expression $x^4-12x^2+27$.
\par
In this example, we are not asked to factor the given expression, but instead to solve for when the expression equals zero.  Still, we can start by finding a complete factorization, again substituting $y=x^2$ and employing the $ac-$method.
\begin{equation*}
\begin{split}
x^4-12x^2+27 &= y^2-12y+27\\
&=y^2-3y-9y+27\\
&=y\left(y-3\right)-9\left(y-3\right)\\
&=\left(y-3\right)\left(y-9\right)\\
&=\left(x^2-3\right)\left(x^2-9\right)\\
&=\left(x+\sqrt{3}\right)\left(x-\sqrt{3}\right)\left(x-3\right)\left(x+3\right)
\end{split}
\end{equation*}
Here, we see that after using the $ac-$method and substituting back for $x,$ we end up with two quadratic factors which can {\it both} be factored as a difference of squares.  In the case of the first factor, $x^2-3,$ our factorization requires a square root, since $3$ is not a perfect square.
\par
Setting each of the four factors equal to zero gives us our set of real roots, $\{\pm\sqrt{3}, \pm 3\}.$
\end{example}
In each of our last two examples, we have seen a degree-four polynomial having two and four real roots, respectively.  We can also easily identify degree-four polynomials having no, one, or three real roots.  The expression $x^4+x^2+1,$ for example, factors as $\left(x^2+1\right)^2,$ which has only complex roots at $\pm i$.  In general, a degree-$n$ polynomial can have as few as zero and as many as $n$ unique real roots.  This is a fact which we will more formally state in a later section, once we have discussed the {\it multiplicity} of a root.
\par
The following example should look familiar.
\begin{example} Factor each of the following polynomial expressions completely over the real numbers.
\begin{multicols}{2}
\begin{enumerate}
\item $x^8+2x^4-24$
\item $x^6+2x^3-24$
\end{enumerate}
\end{multicols}
Before we begin, notice that the coefficients for each of the given expressions match those in Example \ref{quad_type_1}, with the only difference being the exponents appearing in each expression.
\begin{enumerate}
\item Despite the fact that the first expression has a higher degree, its factorization will be simpler than the second expression's.  In this case, we will let $y=x^4,$ and apply the $ac-$method as before.
\begin{equation*}
\begin{split}
x^8+2x^4-24&=\left(x^4\right)^2+2\left(x^4\right)-24\\
&=y^2+2y-24\\
&=\left(y+6\right)\left(y-4\right)\\
&=\left(x^4+6\right)\left(x^4-4\right)
\end{split}
\end{equation*}
Though it might not be obvious, our first factor $x^4+6$ is in fact irreducible over the real numbers.  One way we can realize this is to think of $x^4+6$ as a vertical shift of the graph of $x^4$ up six units.  The resulting graph will lie entirely in the upper-half of the $xy-$plane, and therefore will not intersect the $x-$axis.  Hence, the factor of $x^4+6$ will have no real roots, and consequently any factorization will involve the introduction of imaginary numbers.  Alternatively, one might also notice that raising any real number to the fourth power and adding six will never produce an output of zero, leading us to again conclude that the expression has no real roots.  Lastly, we could also recognize $x^4+6$ as a sum of squares, namely $\left(x^2\right)^2+\left(\sqrt{6}\right)^2,$ which we have already discussed as one expression type that is irreducible over the real numbers.
\par
On the other hand, we can view our second factor $x^4-4$ as a difference of two squares, and factor it as follows.
\begin{equation*}
\begin{split}
x^4-4&=\left(x^2\right)^2-2^2\\
&=\left(x^2+2\right)\left(x^2-2\right)\\
&=\left(x^2+2\right)\left(x+\sqrt{2}\right)\left(x-\sqrt{2}\right)
\end{split}
\end{equation*}
Our complete factorization over the reals is then
$$x^8+2x^4-24=\left(x^4+6\right)\left(x^2+2\right)\left(x+\sqrt{2}\right)\left(x-\sqrt{2}\right).$$
\item In the case of the second expression, if we let $y=x^3,$ we start out with the same two factors for $y,$ which we rewrite as $$\left(x^3+6\right)\left(x^3-4\right).$$
Although neither $6$ nor $4$ are perfect cubes, we can still break down each of the factors above by using the formulas for the sum and difference of cubes from earlier in the section.  For our first factor, letting $a=x$ and $b=\sqrt[3]{6},$ we can write $x^3+6$ as 
$$\left(a+b\right)\left(a^2-ab+b^2\right)=\left(x+\sqrt[3]{6}\right)\left(x^2-\sqrt[3]{6}x+\left(\sqrt[3]{6}\right)^2\right).$$
Similarly, for the second factor, if $a=x$ and $b=\sqrt[3]{4},$ we can write $x^3-4$ as
$$\left(a-b\right)\left(a^2+ab+b^2\right)=\left(x-\sqrt[3]{4}\right)\left(x^2+\sqrt[3]{4}x+\left(\sqrt[3]{4}\right)^2\right).$$
Our complete factorization over the reals is then
\begin{equation*}
\begin{split}
x^6+2x^3-24&=\left(x^3+6\right)\left(x^3-4\right)\\
&=\left(x+\sqrt[3]{6}\right)\left(x^2-\sqrt[3]{6}x+\left(\sqrt[3]{6}\right)^2\right)\left(x-\sqrt[3]{4}\right)\left(x^2+\sqrt[3]{4}x+\left(\sqrt[3]{4}\right)^2\right).
\end{split}
\end{equation*}
\end{enumerate}
\end{example}
We end the subsection on quadratic type with one final example.
\begin{example} Factor each polynomial expression completely over the reals and find its set of real roots.
\begin{multicols}{2}
\begin{enumerate}
\item $x^4-49$
\item $x^6-4x^3-5$
\end{enumerate}
\end{multicols}
\begin{enumerate}
\item Setting $y=x^2,$ we can quickly factor our first expression as a difference of squares, breaking down one of its factors in a similar manner.
\begin{equation*}
\begin{split}
x^4-49&=\left(x^2\right)^2-49\\
&=y^2-\left(7\right)^2\\
&=\left(y+7\right)\left(y-7\right)\\
&=\left(x^2+7\right)\left(x^2-7\right)\\
&=\left(x^2+7\right)\left(x+\sqrt{7}\right)\left(x-\sqrt{7}\right)
\end{split}
\end{equation*}
From our two linear factors, we obtain $x=\pm 7$ as our two real roots.
\item For our second expression, setting $y=x^3,$ we apply the $ac-$method.
\begin{equation*}
\begin{split}
x^6-4x^3-5&=\left(x^3\right)^2-4\left(x^3\right)-5\\
&=y^2-4y-5\\
&=\left(y+1\right)\left(y-5\right)\\
&=\left(x^2+1\right)\left(x^2-5\right)\\
&=\left(x^2+1\right)\left(x+\sqrt{5}\right)\left(x-\sqrt{5}\right)
\end{split}
\end{equation*}
Our two linear factors give us $x=\pm \sqrt{5}$ as the real roots for our given expression.
\end{enumerate}
\end{example}
\end{document}
\documentclass[12pt]{book}
\raggedbottom
\usepackage[top=1in,left=1in,bottom=1in,right=1in,headsep=0.25in]{geometry}	
\usepackage{amssymb,amsmath,amsthm,amsfonts}
\usepackage{chapterfolder,docmute,setspace}
\usepackage{cancel,multicol,tikz,verbatim,framed,polynom,enumitem,tikzpagenodes}
\usepackage[colorlinks, hyperindex, plainpages=false, linkcolor=blue, urlcolor=blue, pdfpagelabels]{hyperref}
\usepackage[type={CC},modifier={by-sa},version={4.0},]{doclicense}

\theoremstyle{definition}
\newtheorem{example}{Example}
\newcommand{\Desmos}{\href{https://www.desmos.com/}{Desmos}}
\setlength{\parindent}{0in}
\setlist{itemsep=0in}
\setlength{\parskip}{0.1in}
\setcounter{secnumdepth}{0}
% This document is used for ordering of lessons.  If an instructor wishes to change the ordering of assessments, the following steps must be taken:

% 1) Reassign the appropriate numbers for each lesson in the \setcounter commands included in this file.
% 2) Rearrange the \include commands in the master file (the file with 'Course Pack' in the name) to accurately reflect the changes.  
% 3) Rarrange the \items in the measureable_outcomes file to accurately reflect the changes.  Be mindful of page breaks when moving items.
% 4) Re-build all affected files (master file, measureable_outcomes file, and any lessons whose numbering has changed).

%Note: The placement of each \newcounter and \setcounter command reflects the original/default ordering of topics (linears, systems, quadratics, functions, polynomials, rationals).

\newcounter{lesson_solving_linear_equations}
\newcounter{lesson_equations_containing_absolute_values}
\newcounter{lesson_graphing_lines}
\newcounter{lesson_two_forms_of_a_linear_equation}
\newcounter{lesson_parallel_and_perpendicular_lines}
\newcounter{lesson_linear_inequalities}
\newcounter{lesson_compound_inequalities}
\newcounter{lesson_inequalities_containing_absolute_values}
\newcounter{lesson_graphing_systems}
\newcounter{lesson_substitution}
\newcounter{lesson_elimination}
\newcounter{lesson_quadratics_introduction}
\newcounter{lesson_factoring_GCF}
\newcounter{lesson_factoring_grouping}
\newcounter{lesson_factoring_trinomials_a_is_1}
\newcounter{lesson_factoring_trinomials_a_neq_1}
\newcounter{lesson_solving_by_factoring}
\newcounter{lesson_square_roots}
\newcounter{lesson_i_and_complex_numbers}
\newcounter{lesson_vertex_form_and_graphing}
\newcounter{lesson_solve_by_square_roots}
\newcounter{lesson_extracting_square_roots}
\newcounter{lesson_the_discriminant}
\newcounter{lesson_the_quadratic_formula}
\newcounter{lesson_quadratic_inequalities}
\newcounter{lesson_functions_and_relations}
\newcounter{lesson_evaluating_functions}
\newcounter{lesson_finding_domain_and_range_graphically}
\newcounter{lesson_fundamental_functions}
\newcounter{lesson_finding_domain_algebraically}
\newcounter{lesson_solving_functions}
\newcounter{lesson_function_arithmetic}
\newcounter{lesson_composite_functions}
\newcounter{lesson_inverse_functions_definition_and_HLT}
\newcounter{lesson_finding_an_inverse_function}
\newcounter{lesson_transformations_translations}
\newcounter{lesson_transformations_reflections}
\newcounter{lesson_transformations_scalings}
\newcounter{lesson_transformations_summary}
\newcounter{lesson_piecewise_functions}
\newcounter{lesson_functions_containing_absolute_values}
\newcounter{lesson_absolute_as_piecewise}
\newcounter{lesson_polynomials_introduction}
\newcounter{lesson_sign_diagrams_polynomials}
\newcounter{lesson_factoring_quadratic_type}
\newcounter{lesson_factoring_summary}
\newcounter{lesson_polynomial_division}
\newcounter{lesson_synthetic_division}
\newcounter{lesson_end_behavior_polynomials}
\newcounter{lesson_local_behavior_polynomials}
\newcounter{lesson_rational_root_theorem}
\newcounter{lesson_polynomials_graphing_summary}
\newcounter{lesson_polynomial_inequalities}
\newcounter{lesson_rationals_introduction_and_terminology}
\newcounter{lesson_sign_diagrams_rationals}
\newcounter{lesson_horizontal_asymptotes}
\newcounter{lesson_slant_and_curvilinear_asymptotes}
\newcounter{lesson_vertical_asymptotes}
\newcounter{lesson_holes}
\newcounter{lesson_rationals_graphing_summary}

\setcounter{lesson_solving_linear_equations}{1}
\setcounter{lesson_equations_containing_absolute_values}{2}
\setcounter{lesson_graphing_lines}{3}
\setcounter{lesson_two_forms_of_a_linear_equation}{4}
\setcounter{lesson_parallel_and_perpendicular_lines}{5}
\setcounter{lesson_linear_inequalities}{6}
\setcounter{lesson_compound_inequalities}{7}
\setcounter{lesson_inequalities_containing_absolute_values}{8}
\setcounter{lesson_graphing_systems}{9}
\setcounter{lesson_substitution}{10}
\setcounter{lesson_elimination}{11}
\setcounter{lesson_quadratics_introduction}{16}
\setcounter{lesson_factoring_GCF}{17}
\setcounter{lesson_factoring_grouping}{18}
\setcounter{lesson_factoring_trinomials_a_is_1}{19}
\setcounter{lesson_factoring_trinomials_a_neq_1}{20}
\setcounter{lesson_solving_by_factoring}{21}
\setcounter{lesson_square_roots}{22}
\setcounter{lesson_i_and_complex_numbers}{23}
\setcounter{lesson_vertex_form_and_graphing}{24}
\setcounter{lesson_solve_by_square_roots}{25}
\setcounter{lesson_extracting_square_roots}{26}
\setcounter{lesson_the_discriminant}{27}
\setcounter{lesson_the_quadratic_formula}{28}
\setcounter{lesson_quadratic_inequalities}{29}
\setcounter{lesson_functions_and_relations}{12}
\setcounter{lesson_evaluating_functions}{13}
\setcounter{lesson_finding_domain_and_range_graphically}{14}
\setcounter{lesson_fundamental_functions}{15}
\setcounter{lesson_finding_domain_algebraically}{30}
\setcounter{lesson_solving_functions}{31}
\setcounter{lesson_function_arithmetic}{32}
\setcounter{lesson_composite_functions}{33}
\setcounter{lesson_inverse_functions_definition_and_HLT}{34}
\setcounter{lesson_finding_an_inverse_function}{35}
\setcounter{lesson_transformations_translations}{36}
\setcounter{lesson_transformations_reflections}{37}
\setcounter{lesson_transformations_scalings}{38}
\setcounter{lesson_transformations_summary}{39}
\setcounter{lesson_piecewise_functions}{40}
\setcounter{lesson_functions_containing_absolute_values}{41}
\setcounter{lesson_absolute_as_piecewise}{42}
\setcounter{lesson_polynomials_introduction}{43}
\setcounter{lesson_sign_diagrams_polynomials}{44}
\setcounter{lesson_factoring_quadratic_type}{46}
\setcounter{lesson_factoring_summary}{45}
\setcounter{lesson_polynomial_division}{47}
\setcounter{lesson_synthetic_division}{48}
\setcounter{lesson_end_behavior_polynomials}{49}
\setcounter{lesson_local_behavior_polynomials}{50}
\setcounter{lesson_rational_root_theorem}{51}
\setcounter{lesson_polynomials_graphing_summary}{52}
\setcounter{lesson_polynomial_inequalities}{53}
\setcounter{lesson_rationals_introduction_and_terminology}{54}
\setcounter{lesson_sign_diagrams_rationals}{55}
\setcounter{lesson_horizontal_asymptotes}{56}
\setcounter{lesson_slant_and_curvilinear_asymptotes}{57}
\setcounter{lesson_vertical_asymptotes}{58}
\setcounter{lesson_holes}{59}
\setcounter{lesson_rationals_graphing_summary}{60}

\newcommand{\tmmathbf}[1]{\ensuremath{\boldsymbol{#1}}}
\newcommand{\tmop}[1]{\ensuremath{\operatorname{#1}}}

\newcount\gpten % (global) power-of-ten -- tells which digit we are doing
\countdef\rtot2 % running total -- remainder so far
\countdef\LDscratch4 % scratch

\def\longdiv#1#2{%
 \vtop{\normalbaselines \offinterlineskip
   \setbox\strutbox\hbox{\vrule height 2.1ex depth .5ex width0ex}%
   \def\showdig{$\underline{\the\LDscratch\strut}$\cr\the\rtot\strut\cr
       \noalign{\kern-.2ex}}%
   \global\rtot=#1\relax
   \count0=\rtot\divide\count0by#2\edef\quotient{\the\count0}%\show\quotient
   % make list macro out of digits in quotient:
   \def\temp##1{\ifx##1\temp\else \noexpand\dodig ##1\expandafter\temp\fi}%
   \edef\routine{\expandafter\temp\quotient\temp}%
   % process list to give power-of-ten:
   \def\dodig##1{\global\multiply\gpten by10 }\global\gpten=1 \routine
   % to display effect of one digit in quotient (zero ignored):
   \def\dodig##1{\global\divide\gpten by10
      \LDscratch =\gpten
      \multiply\LDscratch  by##1%
      \multiply\LDscratch  by#2%
      \global\advance\rtot-\LDscratch \relax
      \ifnum\LDscratch>0 \showdig \fi % must hide \cr in a macro to skip it
   }%
   \tabskip=0pt
   \halign{\hfil##\cr % \halign for entire division problem
     $\quotient$\strut\cr
     #2$\,\overline{\vphantom{\big)}%
     \hbox{\smash{\raise3.5\fontdimen8\textfont3\hbox{$\big)$}}}%
     \mkern2mu \the\rtot}$\cr\noalign{\kern-.2ex}
     \routine \cr % do each digit in quotient
}}}

\begin{document}
\section{Division}
\subsection{Polynomial (Long) Division (L\arabic{lesson_polynomial_division})}
{\bf Objective: Apply polynomial division to a rational expression.}\par
Up until this point, every polynomial expression that we have encountered has either already been provided in its factored form or is easily factorable using one or more of the many techniques that we have learned.  We must be careful, however, to consider the very likely possibility that a given polynomial is not factorable using elementary methods (GCF, grouping, $ac-$method, etc.).  In many cases, obtaining a complete factorization can prove almost impossible without the aid of mathematical computating software.  Although, there is still one powerful tool that can help us to dissect certain factorable, yet formidable, polynomials.  This tool is known as the {\it Rational Root Theorem}, and we will see its use in a later section.
\par
In order to successfully employ the Rational Root Theorem, we first must understand polynomial division.  As we will see, dividing polynomials is a process very similar to long division of whole numbers.
\par
Before we begin with our first example, let's recall the terminology and format associated with division.
$$\dfrac{\text{dividend}}{\text{divisor}}=\text{quotient}+\dfrac{\text{remainder}}{\text{divisor}}$$
Alternatively, multiplying both sides of the above equation by the divisor, we have the following.
$$\dfrac{\text{dividend}}{\cancel{\text{divisor}}}\cdot\cancel{\text{divisor}}=\text{quotient}\cdot\text{divisor}+\dfrac{\text{remainder}}{\cancel{\text{divisor}}}\cdot\cancel{\text{divisor}}$$
$$\text{dividend}=\text{quotient}\cdot\text{divisor}+\text{remainder}$$
We begin with dividing a polynomial by a monomial, which simply utilizes the distributive property.  In the following example, we can think of the stated division as a distribution of the divisor (denominator) to each term in the dividend (numerator).  

\begin{example}\label{poly_div_1} Divide and simplify the following expressions.
\begin{multicols}{2}
\begin{enumerate}
\item $\dfrac{9x^5+6x^4-18x^3-24x^2}{3x^2}$
\item $\dfrac{8x^3+4x^2-2x+6}{4x^2}$
\end{enumerate}
\end{multicols}
\begin{enumerate}
	\item By distributing the expression $\dfrac{1}{3x^2}$ to each of the four terms in the numerator, our first expression becomes
$$\frac{9x^5+6x^4-18x^3-24x^2}{3x^2}=\frac{9x^5}{3x^2}+\frac{6x^4}{3x^2}-\frac{18x^3}{3x^2}-\frac{24x^2}{3x^2}.$$
We can then reduce each individual quotient to produce the following expression.
$$3x^3+2x^2-6x-8$$
In this case, our answer is $3x^3+2x^2-6x-8,$ and we can summarize our results as follows.
$$\frac{9x^5+6x^4-18x^3-24x^2}{3x^2}=3x^3+2x^2-6x-8$$
In this first example, our expression reduced completely, producing our quotient polynomial expression and a remainder of zero.
	\item Again, we will begin with the second expression by splitting it up, or distributing the denominator to each of the three terms in the numerator.  Reducing each individual quotient gives us our answer. 
\begin{equation*}
\begin{split}
\frac{8x^3+4x^2-2x+6}{4x^2}&=\frac{8x^3}{4x^2}+\frac{4x^2}{4x^2}-\frac{2x}{4x^2}+\frac{6}{4x^2}\\
&=2x+1-\frac{1}{2x}+\frac{3}{2x^2}
\end{split}
\end{equation*}
Unlike our first expression, here our expression does not reduce completely, i.e., it contains a nonzero remainder.  Because of this, since our answer includes two rational (or fractional) expressions, one could also combine them to form a single rational expression.
\begin{equation*}
\begin{split}
\frac{8x^3+4x^2-2x+6}{4x^2}&=2x+1-\frac{1}{2x}\cdot\frac{x}{x}+\frac{3}{2x^2}\\
&=2x+1-\frac{x}{2x^2}+\frac{3}{2x^2}\\
&=2x+1+\frac{3-x}{2x^2}
\end{split}
\end{equation*}
Furthermore, if we wanted to identify the remainder in this example, we could multiply both sides of the equation by the divisor, $4x^2$.
\begin{equation*}
\begin{split}
\frac{8x^3+4x^2-2x+6}{\cancel{4x^2}}\cdot \cancel{4x^2}&=(2x+1)\cdot{4x^2}+\left(\frac{3-x}{2\cancel{x^2}}\right)\cdot 4\cancel{x^2}\\
\underbrace{8x^3+4x^2-2x+6}_{\text{dividend}}&=\underbrace{(2x+1)}_{\text{quotient}}\underbrace{4x^2}_{\text{divisor}}+\underbrace{6-2x}_{\text{remainder}}
\end{split}
\end{equation*}
Lastly, one final observation to point out from this particular answer is the initial reduction of the second term, $\dfrac{4x^2}{4x^2},$ which equals one (not zero), and therefore does not simply disappear from the expression altogether.
\end{enumerate}
\end{example}

For division by polynomial expressions that contain more than a single term, long division is usually required.  To illustrate the relationship between polynomial division and standard numerical long division, an example with whole numbers is provided in order to review the (general) steps that will also be used for polynomial long division.
\newpage
\begin{example} Divide 631 by 4.
\begin{multicols}{2}

$$\longdiv{631}{4}$$
\columnbreak

Recalling the process associated with long division, we begin to construct our quotient (157) by comparing the divisor (4) with the largest placeholder of our dividend (631), then subtracting, and repeating this process until we have worked our way down to the ones digit.  We know we have finished, once we are left with a remainder (3) that is less than our divisor.
\end{multicols}
Expressing our answer in the form $$\dfrac{\text{dividend}}{\text{divisor}}=\text{quotient}+\dfrac{\text{remainder}}{\text{divisor}},$$
we have $\dfrac{631}{4}=157+\dfrac{3}{4}.$  Or, in the alternate form
$$\text{dividend}=\text{quotient}\cdot\text{divisor}+\text{remainder},$$
we have $631=145\cdot 4 +3.$
\end{example}

The general process for division of polynomials follows closely with that for dividing integers. The only real difference is in the terminology that we use: {\it term} in place of {\it number/digit}, and {\it degree} instead of {\it value}.

\framebox{
\begin{minipage}{0.9\linewidth}
\begin{center}
General Steps for Polynomial (Long) Division
\end{center}
Let $D(x)$ and $d(x)$ represent two nonzero polynomial functions.  The steps for simplifying the rational expression $\dfrac{D(x)}{d(x)}$ are as follows.
\begin{enumerate}
  \item Divide the leading term of the dividend $D$ by the leading term of the divisor $d$.  Label the resulting term $a_nx^n,$ and write it above the dividend.  This will be the leading term of the quotient, $q(x)$.
  \item Multiply $a_nx^n$ by the divisor, distribute, and simplify.  Label this as $d_1(x)$ and write it directly below the dividend, $D,$ making sure to align terms according to exponents.
  \item Subtract the resulting terms from the dividend.  Label the new expression $D_1$.
  \item Repeat steps (1)-(3) for the divisor $d$ and the new expression $D_i$ until the degree of $D_i$ is \textit{less than} the degree of the divisor.  Relabel the final new dividend as the remainder, $r(x)$.  The entire polynomial expression appearing above the original dividend is the quotient, $q(x)$.
\end{enumerate}
\begin{center}
\begin{tikzpicture}[xscale=1,yscale=1]
	\draw (0.5,0.5) node {$q(x)$};
	\draw (0,0) node {$d(x) \ )\overline{\ \ \ D(x) \ \ \ }$};
	\draw (0.4,-0.5) node {$- \ \underline{\ \ d_1(x)\ \ }$};
	\draw (1,-1) node {$D_1(x)$};
	\draw (0.8,-1.5) node {$- \ \underline{\ \ d_2(x)\ \ }$};
	\draw (1.5,-2) node {$\ddots$};
	\draw (1.8,-2.75) node {$-\underline{\ \ d_i(x) \ \ }$};
	\draw (2.75,-3.25) node {$D_i(x)=r(x)$};
\end{tikzpicture}
\end{center}
Step (3) above often tends to pose the greatest challenge for students.  It is important to keep in mind that we are are always subtracting the top term from the bottom term, which is why we must change the signs of the term(s) on the bottom.  In most cases, we will need to utilize the distributive property.
\end{minipage}
}

A basic example should clear up any confusion, and we begin by revisiting Example \ref{poly_div_1}.

\begin{example} Divide $9x^5+6x^4-18x^3-24x^2$ by $3x^2$.  Simplify and express your answer in the form
$$\frac{D(x)}{d(x)}=q(x)+\dfrac{r(x)}{d(x)}.$$
\begin{multicols}{2}
\polylongdiv[stage=1]{9x^5+6x^4-18x^3-24x^2}{3x^2}

\columnbreak

We set up our division process by first writing the dividend and the divisor in the appropriate locations.

\end{multicols}
\begin{multicols}{2}
\polylongdiv[stage=2]{9x^5+6x^4-18x^3-24x^2}{3x^2}

\columnbreak

Next, we identify the leading term for our quotient, $3x^3$.

\end{multicols}
\begin{multicols}{2}
\polylongdiv[stage=4]{9x^5+6x^4-18x^3-24x^2}{3x^2}

\columnbreak

Multiplying and subtracting produces our new expression, $D_1(x)=6x^4$.
\end{multicols}
Notice that in this example we need only carry down the remaining terms from the dividend when our new expression contains terms which are alike to them.  In this case, for example, while it is perfectly fine to carry down $-18x^3-24x^2,$ it is not necessary until these terms play a role in the subtraction step.
\begin{multicols}{2}

\polylongdiv[stage=7]{9x^5+6x^4-18x^3-24x^2}{3x^2}

\polylongdiv[stage=10]{9x^5+6x^4-18x^3-24x^2}{3x^2}

\columnbreak

Repeating our division steps gives us the second term in our quotient, $2x^2$.
\par
Multiplying this term by the divisor, subtracting, and carrying down the next term in the dividend finishes up the second round of our steps for division.  Our new expression is $D_2(x)=-18x^3$.
\par
Again, we repeat our division steps to produce the third term in our quotient, $-6x$.
\par
Multiplying and subtracting produces another new expression, $D_3(x)=-24x^2$.  Since the degree of $D_3$ equals that of our divisor, $d(x)=3x^2,$ we will need to apply our steps for division one final time.
\end{multicols}

\begin{multicols}{2}

\polylongdiv{9x^5+6x^4-18x^3-24x^2}{3x^2}

\columnbreak

After our fourth and final round of steps, our new expression produces a remainder of $r(x)=0$.  This should come as no real surprise, based upon our earlier calculations from Example \ref{poly_div_1}.  Many examples that we will encounter after this first one will not work out as nicely.
\par
We express the results of our division in the required form as follows.
\end{multicols}

$$\dfrac{9x^5+6x^4-18x^3-24x^2}{3x^2}=3x^3+2x^2-6x-8+\dfrac{0}{3x^2}$$

\end{example}

Because our divisor in this last example is a monomial, our steps do not require the distributive property.  The next example includes a binomial divisor, which will emphasize the importance of the distributive property in the division process.

\begin{example} Divide $3x^3-5x^2-32x+7$ by $x-4$.  Simplify and express your answer in the form
$$\frac{D(x)}{d(x)}=q(x)+\dfrac{r(x)}{d(x)}.$$

\begin{multicols}{2}
\polylongdiv{3x^3-5x^2-32x+7}{x-4}

\columnbreak

For this second example, we have opted to show the entire division process all at once.  Since our quotient,
$$3x^2+7x-4,$$
is a trinomial, we must apply the steps for division three times.
\par
In this second example, our remainder is the constant term $r(x)=-9,$ which has one degree less than our linear divisor, $d(x)=x-4$.
\end{multicols}
Our answer is $\dfrac{3x^3-5x^2-32x+7}{x-4}=3x^2+7x-4+\dfrac{-9}{x-4}.$
\par
If we look at the first round of long division steps, we see a leading term in our quotient of $3x^2$.  Multiplying by our divisor produces the following binomial expression.
$$3x^2(x-4)=3x^3-12x^2$$
Since we must subtract {\it each} term of this expression from our dividend, however, we see that the distributive property has already been applied.  Specifically,
$$-(3x^3-12x^2)=-3x^3+12x^2.$$
What remains now is to simply add each term to the respective term in the dividend, which, if employed correctly, should {\it always} eliminate the leading term in each round of steps.
\par
Failure to distribute a negative across {\it all} terms in this critical step will always lead to an incorrect quotient and remainder.  This is the most common mistake in the polynomial division process, and is one that often takes a long amount of time to correct.  Consequently, checking one's work is critical throughout the division process.  We emphasize these points in this first example where distribution of a negative is required, in the hopes that it is not overlooked.
\end{example}
We continue with another example.
\begin{example} Divide $6x^3-8x^2+10x+103$ by $4+2x$.  Simplify and express your answer in the form
$$\frac{D(x)}{d(x)}=q(x)+\dfrac{r(x)}{d(x)}.$$

Before we begin, we wish to point out the important prerequisite for polynomial division that all expressions be written in {\it descending power order}.  In this case, we will start out by rewriting our divisor as $2x+4$.

\begin{multicols}{2}
\polylongdiv{6x^3-8x^2+10x+103}{2x+4}

\columnbreak

This example is similar to the previous one.  Specifically,
the divisor is a linear binomial, and the dividend has a degree of $3$.
\par
Consequently, the steps for polynomial division are employed three times, yielding a constant remainder.
\par
Our answer should be expressed as follows.
\end{multicols}
$$\frac{6x^3-8x^2+10x+103}{2x+4}=3x^2-10x+25+\frac{3}{2x+4}$$
\end{example}

In each of the previous two examples the dividend contained only nonzero coefficients.  In other words, no term was ``skipped over'' in the expression for $D(x)$.  Our last example will address the importance of keeping track of polynomial {\it place holders}, in the event that a specific term carries with it a zero coefficient, and is therefore omitted from the original expression for the dividend, $D(x)$.
\par

Our last example demonstrates the importance of these preliminary steps.
\begin{example}~~~Divide and simplify the given expression.
  \begin{eqnarray*}
    \frac{2 x^4 + 42x - 4 x^2}{x^2 + 3x} &  & \tmop{Reorder} \tmop{dividend};~
    \tmop{need} x^3 \tmop{term}, \tmop{add} 0 x^3
	\end{eqnarray*}
  \begin{eqnarray*}
    x^2 + 3x~ \overline{)~2 x^4 + \tmmathbf{0 x^3} - 4 x^2 + 42x} &  & \tmop{Divide} \tmop{the}
    \tmop{leading} \tmop{terms} : \frac{2 x^4}{x^2} = 2 x^2
	\end{eqnarray*}
	\begin{eqnarray*}
    \tmmathbf{2 x^2}~~~~~~~~~~~~~~~~~~~~~~~~  &  & \\
    x^2 + 3x~ \overline{)~2 x^4 + 0 x^3 - 4 x^2 + 42x} &  & \tmop{Multiply}
    \tmop{this} \tmop{term} \tmop{by} \tmop{divisor} : 2 x^2 (x^2 + 3x) = 2 x^4 +
    6 x^3\\
    \underline{\tmmathbf{- 2 x^4 - 6 x^3} }~~~~~~~~~~~~~~~~ &  & \tmop{Subtract,~changing~terms}\\
    \tmmathbf{- 6 x^3 - 4 x^2}~~~~~~~  &  & \tmop{Bring} \tmop{down} \tmop{the}
    \tmop{next} \tmop{term,~}-4x^2
   \end{eqnarray*}
  \begin{eqnarray*}
	  2 x^2 \tmmathbf{- 6 x}~~~~~~~~~~~~~~~~~~~~~  &  & \\
    x^2 + 3x~ \overline{)~2 x^4 + 0 x^3 - 4 x^2 + 42x}~~~ &  & \tmop{Repeat,}
    \tmop{divide} \tmop{new~leading~term~by~}x^2 : \frac{- 6 x^3}{x^2} = - 6 x\\
    \underline{- 2 x^4 - 6 x^3 }~~~~~~~~~~~~~~~~~~~ &  & \\
    - 6 x^3 - 4 x^2~~~~~~~~~~~ &  & \tmop{Multiply} \tmop{this} \tmop{term} \tmop{by}
    \tmop{divisor} : - 6 x (x^2 + 3x) = - 6 x^3 - 18 x^2\\
    \underline{\tmmathbf{+ 6 x^3 + 18 x^2} }~~~~~~~~~  &  & \tmop{Subtract,~changing~signs}\\
    \tmmathbf{14 x^2 + 42x} &  & \tmop{Bring} \tmop{down} \tmop{the} \tmop{next}
    \tmop{term,} 42x
		\end{eqnarray*}
  \begin{eqnarray*}
    2 x^2 - 6 x ~\tmmathbf{+14}~~~~~~~~~~~~~~  &  & \\
    x^2 + 3x~ \overline{)~2 x^4 + 0 x^3 - 4 x^2 + 42x}~~~ &  & \tmop{Repeat},
    \tmop{divide} \tmop{new~leading~term~by~} x^2: \frac{14 x^2}{x^2} = 14\\
    \underline{- 2 x^4 - 6 x^3 }~~~~~~~~~~~~~~~~~~~ &  & \\
    - 6 x^3 - 4 x^2~~~~~~~~~~~ &  & \\
    \underline{+ 6 x^3 + 18 x^2}~~~~~~~~~~  &  & \\
    14 x^2 + 42x~ &  & \tmop{Multiply} \tmop{this} \tmop{term} \tmop{by} \tmop{the}
    \tmop{divisor} : 14 (x^2 + 3x) = 14 x^2 + 42x\\
    \tmmathbf{\underline{- 14 x^2 - 42x}} &  & \tmop{Subtract,~changing~signs}\\
    0~~~ &  & \tmop{Zero} \tmop{remainder}\\
    &  & \\
    2 x^2 - 6 x + 14 &  & \tmop{Our} \tmop{solution}
  \end{eqnarray*}
\end{example}
So we have,
$$\frac{2 x^4 - 4 x^2 + 42x}{x^2+3x}~=~2 x^2 - 6 x + 14$$ 

It is important to take a moment to check each problem, to verify that the
exponents decrease incrementally and that none are skipped. 

This final example also illustrates that, just as with classic numerical long
division, sometimes our remainder will be zero.

\subsection{Synthetic Division (L\arabic{lesson_synthetic_division})}
{\bf Objective: Apply synthetic division to a rational expression.}\par
Next, we will introduce a method of division that can be used to streamline the polynomial division process and is often preferred over the more traditional long division method.  This method, known as {\it synthetic division}, although usually quicker than traditional polynomial division, this alternative method can only be implemented when the given divisor is {\it linear}.  Specifically, we will require $d(x)$ to be of the form $x-c$.\par
For our first example, we will divide $x^3+4x^2-5x-14$ by $x-2$, which one can check will produce a quotient of $x^2+6x+7$ and a remainder of zero using polynomial long division.
$$\frac{x^3+4x^2-5x-14}{x-2}~=~x^2+6x+7$$
The method of synthetic division focuses primarily on the coefficients of both the divisor and dividend.  We must still pay careful attention, however, to the powers of our exponents, which will serve as placeholders throughout the process.
To start the process, we will write our coefficients in what we will refer to as a {\it synthetic division tableau} prior to dividing.\par  
To divide $x^3+4x^2-5x-14$ by $x-2$, we first write $2$ in the place of the divisor since $2$ is zero of the factor $x-2$ and we write the coefficients of $x^3+4x^2-5x-14$ in for the dividend.  As our next step, we `bring down' the first coefficient of the dividend.
We will then multiply and add repeatedly.
\begin{center}
\begin{multicols}{2}
\polyhornerscheme[x=2,showbase=top,stage=1]{x^3+4x^2-5x-14}\\
\polyhornerscheme[x=2,showbase=top,stage=2]{x^3+4x^2-5x-14}
\end{multicols}
\end{center}

Next, take the $2$ from the divisor and multiply by the $1$ that was brought down to get $2$.  Write this underneath the $4$, then add to get $6$.

\begin{center}
\begin{multicols}{2}
\polyhornerscheme[x=2,showbase=top,stage=3]{x^3+4x^2-5x-14}\\
\polyhornerscheme[x=2,showbase=top,stage=4]{x^3+4x^2-5x-14}
\end{multicols}
\end{center}

Now multiply the $2$ from the divisor by the $6$ to get $12$, and add it to the $-5$ to get $7$.

\begin{center}
\begin{multicols}{2}
\polyhornerscheme[x=2,showbase=top,stage=5]{x^3+4x^2-5x-14}\\
\polyhornerscheme[x=2,showbase=top,stage=6]{x^3+4x^2-5x-14}
\end{multicols}
\end{center}

Finally, multiply the $2$ in the divisor by the $7$ to get $14$, and add it to the $-14$ to get $0$.

\begin{center}
\begin{multicols}{2}
\polyhornerscheme[x=2,showbase=top,stage=7]{x^3+4x^2-5x-14}\\
\polyhornerscheme[x=2,showbase=top,stage=8,resultstyle=\bf]{x^3+4x^2-5x-14}
\end{multicols}
\end{center}
The first three numbers in the last row of our tableau will be the coefficients of the desired quotient polynomial.  Remember, we started with a third degree polynomial and divided by a first degree polynomial, so the quotient will be a second degree polynomial.  Hence the quotient is $x^2+6x+7$.  The number in bold represents the remainder, which is zero in this case.\par
Due in large part to its speed, synthetic division is often a `tool of choice' for dividing polynomials by divisors of the form $x-c$.  It is important to reiterate that synthetic division will \emph{only} work for these kinds of divisors (linear divisors with leading coefficient 1), and we will need to use polynomial long division for divisors having degree larger than 1.\par
Another observation worth mentioning is that when a polynomial (of degree at least $1$) is divided by $x-c$, the result will be a quotient polynomial of exactly one less degree than the original polynomial.  This is a direct result of the divisor being a linear expression.\par
For a more complete understanding of the relationship between long and synthetic division, students are encouraged to trace each step in synthetic division back to its corresponding step in long division.\par
We conclude this section with three examples using synthetic division.  We will summarize each example using the form below.
$$\frac{\text{dividend}}{\text{divisor}}~=~\text{quotient~}+~\frac{\text{remainder}}{\text{divisor}}$$
\begin{example}~~~Use synthetic division to perform the following polynomial division.  Find the quotient and the remainder polynomials.
$$\frac{5x^3 - 2x^2 + 1}{x-3}$$
When setting up the synthetic division tableau, we need to enter $0$ for the coefficient of $x$ in the dividend as a placeholder, just like in polynomial division.
\begin{multicols}{2}
Setting up and working through the tableau gives us the following result.

\columnbreak

\begin{center}
\polyhornerscheme[x=3,showbase=top,resultstyle=\bf]{5x^3-2x^2+1}
\end{center}
\end{multicols}
Since the dividend was a third degree polynomial, the quotient is a quadratic polynomial with coefficients $5$, $13$ and $39$.  Our quotient is then $q(x) = 5x^2+13x+39$ and the remainder is $r(x) = 118$.\par
Putting this all together, we have the following equation.
$$\frac{5x^3 - 2x^2 + 1}{x-3}~=~5x^2+13x+39~+~\frac{118}{x-3}$$
\end{example}
\begin{example}~~~Use synthetic division to perform the following polynomial division.  Find the quotient and the remainder polynomials.
$$\frac{x^3+8}{x+2}$$
For this division, since we have a factor of $x+2$, we must use the zero of $x = -2$ to begin.
Here, we will once again stress that it is critical to take the time in order to ensure we have set the synthetic division tableau up correctly at the onset of the problem.  Failure to do so will result in an incorrect answer, as well as a considerable amount time spent re-doing the problem.
\begin{center}
\polyhornerscheme[x=-2,showbase=top,resultstyle=\bf]{x^3+8}
\end{center}
We then obtain a quotient of $q(x) = x^2-2x+4$ and remainder of $r(x) =0$. This gives us the following equation.
$$\frac{x^3+8}{x+2}~=~x^2-2x+4$$
This answer is a great reminder of the factoring rules for cubic polynomials that we outlined earlier in the chapter.
\end{example}
\begin{example}~~~Use synthetic division to perform the following polynomial division.  Find the quotient and the remainder polynomials.
$$\dfrac{4-8x-12x^2}{2x-3}$$
To divide $4-8x-12x^2$ by $2x-3$, two things must be done.  First, we write the dividend in descending powers of $x$ as $-12x^2-8x+4$.  Second, since synthetic division works only for factors of the form $x-c$, we factor $2x-3$ as $2\left(x-\frac{3}{2}\right)$.  Our strategy is to first divide $-12x^2-8x+4$ by $2$, to get $-6x^2-4x+2$.  Next, we divide by $x-\frac{3}{2}$.  The tableau becomes
\begin{center}
\polyhornerscheme[x=\frac{3}{2},showbase=top,resultstyle=\bf]{-6x^2-4x+2}
\end{center}
From this, we get a quotient of $q(x)=-6 x - 13$ and a remainder of
$r(x)=-\frac{35}{2}$.  This gives us the following equation.
$$\frac{-6x^2-4x+2}{x-\frac{3}{2}}~=~-6 x - 13 + \frac{-\frac{35}{2}}{x-\frac{3}{2}}$$
Multiplying both sides by of our equation by $\frac{2}{2}$ and distributing gives us our desired answer.
$$\frac{-12x^2-8x+4}{2x-3}~=~-6 x - 13 + \frac{-35}{2x-3}$$
\end{example}
Note that we could also multiply both sides of our last equation by $2x-3$ to obtain the following equation.
$$-12x^2-8x+4 = \left(2x-3\right) (-6 x - 13) - 35$$
While both of the forms above are certainly equivalent, the previous one may remind us of the familiar classic division algorithm for integers, shown below.
\begin{center}
dividend $=$ (divisor)$\cdot$(quotient) $+$ remainder
\end{center}
The first form, however, will be particularly useful when we graph more complicated rational functions in the next chapter.
\newpage
\end{document}
\documentclass[12pt]{book}
\raggedbottom
\usepackage[top=1in,left=1in,bottom=1in,right=1in,headsep=0.25in]{geometry}	
\usepackage{amssymb,amsmath,amsthm,amsfonts}
\usepackage{chapterfolder,docmute,setspace}
\usepackage{cancel,multicol,tikz,verbatim,framed,polynom,enumitem,tikzpagenodes}
\usepackage[colorlinks, hyperindex, plainpages=false, linkcolor=blue, urlcolor=blue, pdfpagelabels]{hyperref}
\usepackage[type={CC},modifier={by-sa},version={4.0},]{doclicense}

\theoremstyle{definition}
\newtheorem{example}{Example}
\newcommand{\Desmos}{\href{https://www.desmos.com/}{Desmos}}
\setlength{\parindent}{0in}
\setlist{itemsep=0in}
\setlength{\parskip}{0.1in}
\setcounter{secnumdepth}{0}
% This document is used for ordering of lessons.  If an instructor wishes to change the ordering of assessments, the following steps must be taken:

% 1) Reassign the appropriate numbers for each lesson in the \setcounter commands included in this file.
% 2) Rearrange the \include commands in the master file (the file with 'Course Pack' in the name) to accurately reflect the changes.  
% 3) Rarrange the \items in the measureable_outcomes file to accurately reflect the changes.  Be mindful of page breaks when moving items.
% 4) Re-build all affected files (master file, measureable_outcomes file, and any lessons whose numbering has changed).

%Note: The placement of each \newcounter and \setcounter command reflects the original/default ordering of topics (linears, systems, quadratics, functions, polynomials, rationals).

\newcounter{lesson_solving_linear_equations}
\newcounter{lesson_equations_containing_absolute_values}
\newcounter{lesson_graphing_lines}
\newcounter{lesson_two_forms_of_a_linear_equation}
\newcounter{lesson_parallel_and_perpendicular_lines}
\newcounter{lesson_linear_inequalities}
\newcounter{lesson_compound_inequalities}
\newcounter{lesson_inequalities_containing_absolute_values}
\newcounter{lesson_graphing_systems}
\newcounter{lesson_substitution}
\newcounter{lesson_elimination}
\newcounter{lesson_quadratics_introduction}
\newcounter{lesson_factoring_GCF}
\newcounter{lesson_factoring_grouping}
\newcounter{lesson_factoring_trinomials_a_is_1}
\newcounter{lesson_factoring_trinomials_a_neq_1}
\newcounter{lesson_solving_by_factoring}
\newcounter{lesson_square_roots}
\newcounter{lesson_i_and_complex_numbers}
\newcounter{lesson_vertex_form_and_graphing}
\newcounter{lesson_solve_by_square_roots}
\newcounter{lesson_extracting_square_roots}
\newcounter{lesson_the_discriminant}
\newcounter{lesson_the_quadratic_formula}
\newcounter{lesson_quadratic_inequalities}
\newcounter{lesson_functions_and_relations}
\newcounter{lesson_evaluating_functions}
\newcounter{lesson_finding_domain_and_range_graphically}
\newcounter{lesson_fundamental_functions}
\newcounter{lesson_finding_domain_algebraically}
\newcounter{lesson_solving_functions}
\newcounter{lesson_function_arithmetic}
\newcounter{lesson_composite_functions}
\newcounter{lesson_inverse_functions_definition_and_HLT}
\newcounter{lesson_finding_an_inverse_function}
\newcounter{lesson_transformations_translations}
\newcounter{lesson_transformations_reflections}
\newcounter{lesson_transformations_scalings}
\newcounter{lesson_transformations_summary}
\newcounter{lesson_piecewise_functions}
\newcounter{lesson_functions_containing_absolute_values}
\newcounter{lesson_absolute_as_piecewise}
\newcounter{lesson_polynomials_introduction}
\newcounter{lesson_sign_diagrams_polynomials}
\newcounter{lesson_factoring_quadratic_type}
\newcounter{lesson_factoring_summary}
\newcounter{lesson_polynomial_division}
\newcounter{lesson_synthetic_division}
\newcounter{lesson_end_behavior_polynomials}
\newcounter{lesson_local_behavior_polynomials}
\newcounter{lesson_rational_root_theorem}
\newcounter{lesson_polynomials_graphing_summary}
\newcounter{lesson_polynomial_inequalities}
\newcounter{lesson_rationals_introduction_and_terminology}
\newcounter{lesson_sign_diagrams_rationals}
\newcounter{lesson_horizontal_asymptotes}
\newcounter{lesson_slant_and_curvilinear_asymptotes}
\newcounter{lesson_vertical_asymptotes}
\newcounter{lesson_holes}
\newcounter{lesson_rationals_graphing_summary}

\setcounter{lesson_solving_linear_equations}{1}
\setcounter{lesson_equations_containing_absolute_values}{2}
\setcounter{lesson_graphing_lines}{3}
\setcounter{lesson_two_forms_of_a_linear_equation}{4}
\setcounter{lesson_parallel_and_perpendicular_lines}{5}
\setcounter{lesson_linear_inequalities}{6}
\setcounter{lesson_compound_inequalities}{7}
\setcounter{lesson_inequalities_containing_absolute_values}{8}
\setcounter{lesson_graphing_systems}{9}
\setcounter{lesson_substitution}{10}
\setcounter{lesson_elimination}{11}
\setcounter{lesson_quadratics_introduction}{16}
\setcounter{lesson_factoring_GCF}{17}
\setcounter{lesson_factoring_grouping}{18}
\setcounter{lesson_factoring_trinomials_a_is_1}{19}
\setcounter{lesson_factoring_trinomials_a_neq_1}{20}
\setcounter{lesson_solving_by_factoring}{21}
\setcounter{lesson_square_roots}{22}
\setcounter{lesson_i_and_complex_numbers}{23}
\setcounter{lesson_vertex_form_and_graphing}{24}
\setcounter{lesson_solve_by_square_roots}{25}
\setcounter{lesson_extracting_square_roots}{26}
\setcounter{lesson_the_discriminant}{27}
\setcounter{lesson_the_quadratic_formula}{28}
\setcounter{lesson_quadratic_inequalities}{29}
\setcounter{lesson_functions_and_relations}{12}
\setcounter{lesson_evaluating_functions}{13}
\setcounter{lesson_finding_domain_and_range_graphically}{14}
\setcounter{lesson_fundamental_functions}{15}
\setcounter{lesson_finding_domain_algebraically}{30}
\setcounter{lesson_solving_functions}{31}
\setcounter{lesson_function_arithmetic}{32}
\setcounter{lesson_composite_functions}{33}
\setcounter{lesson_inverse_functions_definition_and_HLT}{34}
\setcounter{lesson_finding_an_inverse_function}{35}
\setcounter{lesson_transformations_translations}{36}
\setcounter{lesson_transformations_reflections}{37}
\setcounter{lesson_transformations_scalings}{38}
\setcounter{lesson_transformations_summary}{39}
\setcounter{lesson_piecewise_functions}{40}
\setcounter{lesson_functions_containing_absolute_values}{41}
\setcounter{lesson_absolute_as_piecewise}{42}
\setcounter{lesson_polynomials_introduction}{43}
\setcounter{lesson_sign_diagrams_polynomials}{44}
\setcounter{lesson_factoring_quadratic_type}{46}
\setcounter{lesson_factoring_summary}{45}
\setcounter{lesson_polynomial_division}{47}
\setcounter{lesson_synthetic_division}{48}
\setcounter{lesson_end_behavior_polynomials}{49}
\setcounter{lesson_local_behavior_polynomials}{50}
\setcounter{lesson_rational_root_theorem}{51}
\setcounter{lesson_polynomials_graphing_summary}{52}
\setcounter{lesson_polynomial_inequalities}{53}
\setcounter{lesson_rationals_introduction_and_terminology}{54}
\setcounter{lesson_sign_diagrams_rationals}{55}
\setcounter{lesson_horizontal_asymptotes}{56}
\setcounter{lesson_slant_and_curvilinear_asymptotes}{57}
\setcounter{lesson_vertical_asymptotes}{58}
\setcounter{lesson_holes}{59}
\setcounter{lesson_rationals_graphing_summary}{60}

\newcommand{\tmmathbf}[1]{\ensuremath{\boldsymbol{#1}}}
\newcommand{\tmop}[1]{\ensuremath{\operatorname{#1}}}

\begin{document}
\section{End Behavior (L\arabic{lesson_end_behavior_polynomials})}
{\bf Objective: Identify and describe the end behavior of the graph of a polynomial function.}\par
The {\it end behavior} of any function refers to what happens near the extreme ends of its graph.  We also often refer to these as the ``tails'' of the graph.  The ends of the graph of a function correspond to points having large positive or negative $x-$coordinates.  Because of this, we can associate the expressions
$$x\rightarrow\infty\qquad\text{and}\qquad x\rightarrow -\infty$$
to the end behavior of a function.  For example, the sentence
\begin{center}
As $x\rightarrow\infty,$ \ $f(x)\rightarrow\infty$.
\end{center}
describes a function for which the right-hand side of its graph, i.e. when $x\rightarrow\infty,$ points upward.  Alternatively, the sentence  
\begin{center}
As $x\rightarrow\infty,$ \ $f(x)\rightarrow -\infty$.
\end{center}
describes a function for which the right-hand side of its graph points downward.
\par
In each of the above mathematical statements, we are identifying both a horizontal direction and a vertical direction:
\begin{enumerate}
	\item  the independent variable $x$ getting large (either positively or negatively),
	\item and the effect this has on the values of $f(x)$.
\end{enumerate}

\begin{example}
Describe the end behavior of the function $f$ whose graph is shown below.
\begin{multicols}{2}
\begin{center}
\begin{tikzpicture}[scale=0.85]
	\draw [<->](-4,0) -- coordinate (x axis mid) (4,0) node[below right] {$x$};
	\draw [<->](0,-1) -- coordinate (y axis mid) (0,4.25) node[above right] {$y$};
	\draw [<->] plot [domain=-2:3.25,samples=100] (\x,{0.5^(\x)});
	\foreach \y in {1,2,3,4} \draw (2pt,\y) -- (-2pt,\y)	node[anchor=east] {\scriptsize \y};							
	\foreach \x in {-3,-2,-1} \draw (\x,2pt) -- (\x,-2pt)	node[anchor=north] {\scriptsize \x};							
	\foreach \x in {1,2,3} \draw (\x,2pt) -- (\x,-2pt)	node[anchor=north] {\scriptsize \x};							
	\draw (-3,3) node {$y=f(x)$};
\end{tikzpicture}
\end{center}

\columnbreak

Although this graph is one that we typically see in a precalculus setting (known as an exponential function), we can still discuss its end behavior.  In this case, as the values of $x$ increase, we see that the points on the graph approach the $x-$axis.  This translates to the following statement.
\begin{center}
As $x\rightarrow\infty,$ \ $f(x)\rightarrow 0$.
\end{center}
\end{multicols}
On the other hand, as the values of $x$ tend towards $-\infty,$ we see that the $y-$coordinates for their respective points continue to increase.  Hence, we can say the following.
\begin{center}
As $x\rightarrow -\infty,$ \ $f(x)\rightarrow \infty$.
\end{center}
\end{example}

Prior to this chapter, we have not had much need to discuss end behavior at great length, since most of the functions which we have been exposed to have been relatively easily diagnosed and graphed.  A quadratic function, $f(x)=ax^2+bx+c$, for example, will either open up or down, depending on the sign of the leading coefficient, $a$.  As we begin to graph polynomials, however, we will see our graphs take more than a few turns, which will require us to have a better understanding about the nature of their tails.
\par
For each algebraic function, the corresponding graph will describe two such statements: one for the left-hand side of the graph ($x\rightarrow -\infty$) and one for the right-hand side of the graph ($x\rightarrow\infty$).  In the case of polynomials, there are only four cases for these two statements, summarized as follows.\par
\framebox{
\begin{minipage}{1\linewidth}
Let $$f(x) = a_{n}x^{n} + a_{n-1}x^{n-1}+ ... + a_{2}x^2 + a_{1}x + a_{0}$$ be a polynomial function with degree $n$ and nonzero leading coefficient $a_n$.\\
\par
The end behavior of $f$ is described by one of the following four cases.

\begin{center}
\begin{multicols}{2}
\begin{tikzpicture}[xscale=0.45,yscale=0.45]
	\draw [<->](-4,0) -- coordinate (x axis mid) (4,0) node[below right] {$x$};
	\draw [<->](0,-4) -- coordinate (y axis mid) (0,4) node[above right] {$y$};
	\draw [->] plot [domain=2.5:3.75, samples=100] (\x,{\x^2/4});
	\draw [->] plot [domain=-2.5:-3.75, samples=100] (\x,{-\x^2/4});
	\draw (-4,6.5) node {I. $n$ even, $a_n>0$};
	\draw (0,-5) node {As $x\rightarrow\infty,\ f(x)\rightarrow\infty$};
	\draw (0,-7) node {As $x\rightarrow -\infty,\ f(x)\rightarrow\infty$};
\end{tikzpicture}

\begin{tikzpicture}[xscale=0.45,yscale=0.45]
	\draw [<->](-4,0) -- coordinate (x axis mid) (4,0) node[below right] {$x$};
	\draw [<->](0,-4) -- coordinate (y axis mid) (0,4) node[above right] {$y$};
	\draw [->] plot [domain=2.5:3.75, samples=100] (\x,{-\x^2/4});
	\draw [->] plot [domain=-2.5:-3.75, samples=100] (\x,{\x^2/4});
	\draw (-4,6.5) node {II. $n$ even, $a_n<0$};
	\draw (0,-5) node {As $x\rightarrow\infty,\ f(x)\rightarrow -\infty$};
	\draw (0,-7) node {As $x\rightarrow -\infty,\ f(x)\rightarrow -\infty$};
\end{tikzpicture}
\end{multicols}
\end{center}

\begin{center}\begin{multicols}{2}
\begin{tikzpicture}[xscale=0.45,yscale=0.45]
	\draw [<->](-4,0) -- coordinate (x axis mid) (4,0) node[below right] {$x$};
	\draw [<->](0,-4) -- coordinate (y axis mid) (0,4) node[above right] {$y$};
	\draw [->] plot [domain=2.5:3.75, samples=100] (\x,{\x^2/4});
	\draw [->] plot [domain=-2.5:-3.75, samples=100] (\x,{\x^2/4});
	\draw (-4,6.5) node {III. $n$ odd, $a_n>0$};
	\draw (0,-5) node {As $x\rightarrow\infty,\ f(x)\rightarrow \infty$};
	\draw (0,-7) node {As $x\rightarrow -\infty,\ f(x)\rightarrow -\infty$};
\end{tikzpicture}

\begin{tikzpicture}[xscale=0.45,yscale=0.45]
	\draw [<->](-4,0) -- coordinate (x axis mid) (4,0) node[below right] {$x$};
	\draw [<->](0,-4) -- coordinate (y axis mid) (0,4) node[above right] {$y$};
	\draw [->] plot [domain=2.5:3.75, samples=100] (\x,{-\x^2/4});
	\draw [->] plot [domain=-2.5:-3.75, samples=100] (\x,{-\x^2/4});
	\draw (-4,6.5) node {IV. $n$ odd, $a_n<0$};
	\draw (0,-5) node {As $x\rightarrow\infty,\ f(x)\rightarrow -\infty$};
	\draw (0,-7) node {As $x\rightarrow -\infty,\ f(x)\rightarrow \infty$};
\end{tikzpicture}
\end{multicols}
\end{center}
\end{minipage}
}

An important initial observation of the previous figure is that the cases for the end behavior of a polynomial only depend on its leading term, $a_nx^n$.  More specifically, the end behavior of a polynomial depends only on the parity of its degree $n$ (even or odd) and the sign of its leading coefficient $a_n$ (positive or negative).  Additionally, we can see that cases I and II also include all quadratic functions (when $n=2$).
\par
Identifying the end behavior for an expanded polynomial, is much more straightforward than for a factored polynomial, as we will see in our next example.
\begin{example}
Determine the end behavior of each of the following functions.
\begin{multicols}{2}
\begin{enumerate}
	\item $f(x)=1-3x^4$
	\item $g(x)=-2x^3+10000x^2+1000$
	\item $h(x)=x(2x-1)(x-5)^2$
	\item $k(x)=-2(1-3x)^2(x+1)(x-1)(x^2+1)$
\end{enumerate}
\end{multicols}
\begin{enumerate}
\item The polynomial $f(x)=1-3x^4$ is in expanded form, though not written in descending-power order.  We can easily re-write $f$ as $f(x)=-3x^4+1$.  In this case, the degree $n=4$ is even, and the leading coefficient $a_n=-3$ is negative.  Hence, we are in case II:
\begin{center}
As $x\rightarrow -\infty, \ f(x)\rightarrow -\infty.$ \hspace{1in} As $x\rightarrow \infty, \ f(x)\rightarrow -\infty.$
\end{center}
\item The polynomial $g$ is also in expanded form, with odd degree $n=3$ and negative leading coefficient, $a_n=-2$.  The ``large'' quadratic and  constant terms will not affect the end behavior of $g,$ and so we are in case IV:
\begin{center}
As $x\rightarrow -\infty, \ g(x)\rightarrow \infty.$ \hspace{1in} As $x\rightarrow \infty, \ g(x)\rightarrow -\infty.$
\end{center}
\item The polynomial $h$ is written in factored form, which is helpful for identifying roots/$x-$intercepts, but not necessarily for describing the tails of the graph of $h$.  Although we could, with enough time, expand $h$ completely to describe the function's end behavior, this will quickly prove to be an inefficient strategy.  Recall, however, that for end behavior we need only focus on finding the leading term, $a_nx^n$.  We do this by identifying any parts of $h$ that will contribute to the leading term.  In this case, we identify in boldface font all contributing components to the leading term of $h$ below.
$$h(x)=\mathbf{x}(\mathbf{2x}-1)(\mathbf{x}-5)^{\mathbf{2}}$$
So, when we expand, the leading term of $h$ will be
$$a_nx^n=x(2x)(x)^2=2x^4.$$
We can now see that $h$ has even degree $n=4,$ and positive leading coefficient, $a_n=2$.  Hence, we are in case I:
\begin{center}
As $x\rightarrow -\infty, \ h(x)\rightarrow \infty.$ \hspace{1in} As $x\rightarrow \infty, \ h(x)\rightarrow \infty.$
\end{center}
\item Similarly, the polynomial $k$ is written in factored form, and will require us to find the leading term, $a_nx^n$.  Again, we identify all contributing components to the leading term of $k$ in boldface font below. 
$$k(x)=\mathbf{-2}(1\mathbf{-3x})^{\mathbf{2}}(\mathbf{x}+1)(\mathbf{x}-1)(\mathbf{x^2}+1)$$
So, when we expand, the leading term of $k$ will be
\begin{equation*}
\begin{split}
a_nx^n & =-2(-3x)^2(x)(x)(x^2)\\
& = -2(9x^2)(x^4)\\
& = -18x^6.
\end{split}
\end{equation*}
Through careful analysis, we see that $k$ has an even degree, $n=6$, and a negative leading coefficient, $a_n=-18$.  Hence, we are in case II:
\begin{center}
As $x\rightarrow -\infty, \ k(x)\rightarrow -\infty.$ \hspace{1in} As $x\rightarrow \infty, \ k(x)\rightarrow -\infty.$
\end{center}
\end{enumerate}
\end{example}
In the previous example, we witnessed a new technique to quickly identify the end behavior of a polynomial that is given in factored form. 
The idea behind this technique is to only focus on the contributing components to a polynomial's leading term, $a_nx^n,$ ignoring all others.  This essentially boils down to focusing on three things:
 \begin{itemize}
		\item any constant multiplier,
		\item the leading term of each factor,
		\item and the power associated with each factor.
	\end{itemize}
In general, if we suppose that a polynomial $f$ has the factorization
\begin{center}
$f(x)=c\cdot(\text{Factor 1})^{k_1}\cdot(\text{Factor 2})^{k_2}\cdot\ldots\cdot(\text{Factor m})^{k_m},$
\end{center}
then the leading term for $f$ will equal
\begin{center}
$a_nx^n=c\cdot(\text{Leading Term 1})^{k_1}\cdot(\text{Leading Term 2})^{k_2}\cdot\ldots\cdot(\text{Leading Term m})^{k_m}.$
\end{center}
Note that ``Leading Term 1'' refers to the leading term of Factor 1, and so on for the other factors.
\par
This approach is similar to one that we have likely seen for identifying the constant term for a factored polynomial.  To identify the constant term, $a_0$ of $f$, we would have
\begin{center}
$a_nx^n=c\cdot(\text{Constant Term 1})^{k_1}\cdot(\text{Constant Term 2})^{k_2}\cdot\ldots\cdot(\text{Constant Term m})^{k_m}.$
\end{center}
\begin{example}\label{poly_end_beh_1}
Find the leading and constant terms for the given function, and use them to identify the end behavior and $y-$intercept of its graph.
$$f(x)=3(-2x+1)^2(x-2)^2(x-5)$$
First, we boldface the contributors for the leading term.
$$f(x)=\mathbf{3}(\mathbf{-2x}+1)^{\mathbf{2}}(\mathbf{x}-2)^{\mathbf{2}}(\mathbf{x}-5)$$
This gives us the following.
\begin{equation*}
\begin{split}
a_nx^n & =3(-2x)^{2}(x)^{2}(x)\\
& = 3(4x^2)x^3\\
& = 12x^5
\end{split}
\end{equation*}
Next, we boldface the contributors for the constant term.
$$f(x)=\mathbf{3}(-2x\mathbf{+1})^{\mathbf{2}}(x\mathbf{-2})^{\mathbf{2}}(x\mathbf{-5})$$
This gives us the following.
\begin{equation*}
\begin{split}
a_0 & = 3(1)^{2}(-2)^{2}(-5)\\
& = 3(1)(4)(-5)\\
& = -60
\end{split}
\end{equation*}
Hence, we have that
$$f(x)=12x^5+\ldots +(-60),$$
with middle terms unknown.
\par
Since our degree, $n=5,$ is odd, and our leading coefficient, $a_n=12,$ is positive, we are in case III for end behavior.
\begin{center}
As $x\rightarrow -\infty, \ f(x)\rightarrow -\infty.$ \hspace{1in} As $x\rightarrow \infty, \ f(x)\rightarrow +\infty.$
\end{center}
Our constant term also tells us that the graph of $f$ has a $y-$intercept at $(0,-60)$.
\end{example}
It is natural to ask why the additional terms of a polynomial have no impact on its end behavior.  To address this, let us consider factoring out the leading term from $f$, which will give us the following.
\begin{equation*}
\begin{split}
f(x) & = a_{n}x^{n} + a_{n-1}x^{n-1}+ ... + a_{2}x^2 + a_{1}x + a_{0}\\
 & = a_{n} x^{n} \left( 1 + \frac{a_{n-1}}{a_{n} x}+ \ldots + \frac{a_{2}}{a_{n} x^{n-2}} + \frac{a_1}{a_{n} x^{n-1}}+\frac{a_{0}}{a_{n} x^{n}}\right)
\end{split}
\end{equation*}
If we use $g(x)$ to denote the expression in parentheses,
$$g(x)=1 + \frac{a_{n-1}}{a_{n} x}+ \ldots + \frac{a_{2}}{a_{n} x^{n-2}} + \frac{a_1}{a_{n} x^{n-1}}+\frac{a_{0}}{a_{n} x^{n}},$$
then
\begin{equation*}
\begin{split}
f(x) & = a_{n} x^{n} \underbrace{\left( 1 + \frac{a_{n-1}}{a_{n} x}+ \ldots + \frac{a_{2}}{a_{n} x^{n-2}} + \frac{a_1}{a_{n} x^{n-1}}+\frac{a_{0}}{a_{n} x^{n}}\right)}_{g(x)}\\
& = a_{n} x^{n} \cdot g(x).
\end{split}
\end{equation*}
But recall that the end behavior of a polynomial is determined when $x\rightarrow\pm\infty$.  So, as $x$ gets large (either positively or negatively), with the exception of the first term, all subsequent terms in the expression for $g$ will approach zero.
\begin{center}
As $x\rightarrow\pm\infty,$ \ $g(x)=1 + \cancelto{0}{\frac{a_{n-1}}{\ a_{n} x \ }}+ \ldots + \cancelto{0}{\frac{a_{2}}{a_{n} x^{n-2}}} + \cancelto{0}{\frac{a_1}{a_{n} x^{n-1}}}+\cancelto{0}{\frac{a_{0}}{\ a_{n} x^{n}}}\rightarrow 1.$
\end{center}
Therefore, since $g(x)$ approaches 1, $f(x)= a_{n} x^{n} \cdot g(x)$ will approach its leading term, $a_{n} x^{n}$.  Hence, we conclude that the end behavior of a polynomial $f$ will coincide with the end behavior of its leading term.
\par
Furthermore, for any polynomial $f(x),$ if we were to graph the two curves $y=f(x)$ and $y=a_nx^n$ using \Desmos \ or another graphing utility, and continue to `zoom out', the two graphs would become virtually indistinguishable from one another.  We demonstrate this in our next example.
\begin{example}
Determine the end behavior of the polynomial function below, and graph both the function and its leading term on a single set of axes.
$$f(x)=-x^3+3x-2$$
The leading term of $f$ is $-x^3,$ with odd degree, $n=3,$ and negative leading coefficient, $a_n=-1$.  Hence, we are in case IV:
\begin{center}
As $x\rightarrow -\infty, \ f(x)\rightarrow \infty.$ \hspace{1in} As $x\rightarrow \infty, \ f(x)\rightarrow -\infty.$
\begin{center}
\begin{tikzpicture}[xscale=0.75,yscale=0.15]
	\draw [<->](-5.25,0) -- coordinate (x axis mid) (5.25,0) node[below right] {$x$};
	\draw [<->](0,-26) -- coordinate (x axis mid) (0,26) node[above right] {$y$};
	\draw [<->] plot [domain=-3.25:3.2, samples=100] (\x,{-\x^3+3*\x-2});
	\draw [<->, dashed, line width=0.5mm] plot [domain=-2.9:2.9, samples=100] (\x,{-\x^3});
	\foreach \x in {1,2,...,5}
		\draw (\x,2pt) -- (\x,-2pt)	node[anchor=south] {\scriptsize \x};
	\foreach \x in {-5,-4,...,-1}
		\draw (\x,2pt) -- (\x,-2pt)	node[anchor=north] {\scriptsize \x};
	\foreach \y in {5,10,...,25}
		\draw (2pt,\y) -- (-2pt,\y)	node[anchor=east] {\scriptsize \y}; 
	\foreach \y in {-25,-20,...,-5}
		\draw (2pt,\y) -- (-2pt,\y)	node[anchor=east] {\scriptsize \y}; 
	\draw (-1.75,20) node {\scriptsize $y=-x^3$};
	\draw (-4.75,5) node {\scriptsize $f(x)=-x^3+3x-2$};
\end{tikzpicture}
\end{center}
\end{center}
\end{example}
\end{document}
\documentclass[12pt]{book}
\raggedbottom
\usepackage[top=1in,left=1in,bottom=1in,right=1in,headsep=0.25in]{geometry}	
\usepackage{amssymb,amsmath,amsthm,amsfonts}
\usepackage{chapterfolder,docmute,setspace}
\usepackage{cancel,multicol,tikz,verbatim,framed,polynom,enumitem,tikzpagenodes}
\usepackage[colorlinks, hyperindex, plainpages=false, linkcolor=blue, urlcolor=blue, pdfpagelabels]{hyperref}
\usepackage[type={CC},modifier={by-sa},version={4.0},]{doclicense}

\theoremstyle{definition}
\newtheorem{example}{Example}
\newcommand{\Desmos}{\href{https://www.desmos.com/}{Desmos}}
\setlength{\parindent}{0in}
\setlist{itemsep=0in}
\setlength{\parskip}{0.1in}
\setcounter{secnumdepth}{0}
% This document is used for ordering of lessons.  If an instructor wishes to change the ordering of assessments, the following steps must be taken:

% 1) Reassign the appropriate numbers for each lesson in the \setcounter commands included in this file.
% 2) Rearrange the \include commands in the master file (the file with 'Course Pack' in the name) to accurately reflect the changes.  
% 3) Rarrange the \items in the measureable_outcomes file to accurately reflect the changes.  Be mindful of page breaks when moving items.
% 4) Re-build all affected files (master file, measureable_outcomes file, and any lessons whose numbering has changed).

%Note: The placement of each \newcounter and \setcounter command reflects the original/default ordering of topics (linears, systems, quadratics, functions, polynomials, rationals).

\newcounter{lesson_solving_linear_equations}
\newcounter{lesson_equations_containing_absolute_values}
\newcounter{lesson_graphing_lines}
\newcounter{lesson_two_forms_of_a_linear_equation}
\newcounter{lesson_parallel_and_perpendicular_lines}
\newcounter{lesson_linear_inequalities}
\newcounter{lesson_compound_inequalities}
\newcounter{lesson_inequalities_containing_absolute_values}
\newcounter{lesson_graphing_systems}
\newcounter{lesson_substitution}
\newcounter{lesson_elimination}
\newcounter{lesson_quadratics_introduction}
\newcounter{lesson_factoring_GCF}
\newcounter{lesson_factoring_grouping}
\newcounter{lesson_factoring_trinomials_a_is_1}
\newcounter{lesson_factoring_trinomials_a_neq_1}
\newcounter{lesson_solving_by_factoring}
\newcounter{lesson_square_roots}
\newcounter{lesson_i_and_complex_numbers}
\newcounter{lesson_vertex_form_and_graphing}
\newcounter{lesson_solve_by_square_roots}
\newcounter{lesson_extracting_square_roots}
\newcounter{lesson_the_discriminant}
\newcounter{lesson_the_quadratic_formula}
\newcounter{lesson_quadratic_inequalities}
\newcounter{lesson_functions_and_relations}
\newcounter{lesson_evaluating_functions}
\newcounter{lesson_finding_domain_and_range_graphically}
\newcounter{lesson_fundamental_functions}
\newcounter{lesson_finding_domain_algebraically}
\newcounter{lesson_solving_functions}
\newcounter{lesson_function_arithmetic}
\newcounter{lesson_composite_functions}
\newcounter{lesson_inverse_functions_definition_and_HLT}
\newcounter{lesson_finding_an_inverse_function}
\newcounter{lesson_transformations_translations}
\newcounter{lesson_transformations_reflections}
\newcounter{lesson_transformations_scalings}
\newcounter{lesson_transformations_summary}
\newcounter{lesson_piecewise_functions}
\newcounter{lesson_functions_containing_absolute_values}
\newcounter{lesson_absolute_as_piecewise}
\newcounter{lesson_polynomials_introduction}
\newcounter{lesson_sign_diagrams_polynomials}
\newcounter{lesson_factoring_quadratic_type}
\newcounter{lesson_factoring_summary}
\newcounter{lesson_polynomial_division}
\newcounter{lesson_synthetic_division}
\newcounter{lesson_end_behavior_polynomials}
\newcounter{lesson_local_behavior_polynomials}
\newcounter{lesson_rational_root_theorem}
\newcounter{lesson_polynomials_graphing_summary}
\newcounter{lesson_polynomial_inequalities}
\newcounter{lesson_rationals_introduction_and_terminology}
\newcounter{lesson_sign_diagrams_rationals}
\newcounter{lesson_horizontal_asymptotes}
\newcounter{lesson_slant_and_curvilinear_asymptotes}
\newcounter{lesson_vertical_asymptotes}
\newcounter{lesson_holes}
\newcounter{lesson_rationals_graphing_summary}

\setcounter{lesson_solving_linear_equations}{1}
\setcounter{lesson_equations_containing_absolute_values}{2}
\setcounter{lesson_graphing_lines}{3}
\setcounter{lesson_two_forms_of_a_linear_equation}{4}
\setcounter{lesson_parallel_and_perpendicular_lines}{5}
\setcounter{lesson_linear_inequalities}{6}
\setcounter{lesson_compound_inequalities}{7}
\setcounter{lesson_inequalities_containing_absolute_values}{8}
\setcounter{lesson_graphing_systems}{9}
\setcounter{lesson_substitution}{10}
\setcounter{lesson_elimination}{11}
\setcounter{lesson_quadratics_introduction}{16}
\setcounter{lesson_factoring_GCF}{17}
\setcounter{lesson_factoring_grouping}{18}
\setcounter{lesson_factoring_trinomials_a_is_1}{19}
\setcounter{lesson_factoring_trinomials_a_neq_1}{20}
\setcounter{lesson_solving_by_factoring}{21}
\setcounter{lesson_square_roots}{22}
\setcounter{lesson_i_and_complex_numbers}{23}
\setcounter{lesson_vertex_form_and_graphing}{24}
\setcounter{lesson_solve_by_square_roots}{25}
\setcounter{lesson_extracting_square_roots}{26}
\setcounter{lesson_the_discriminant}{27}
\setcounter{lesson_the_quadratic_formula}{28}
\setcounter{lesson_quadratic_inequalities}{29}
\setcounter{lesson_functions_and_relations}{12}
\setcounter{lesson_evaluating_functions}{13}
\setcounter{lesson_finding_domain_and_range_graphically}{14}
\setcounter{lesson_fundamental_functions}{15}
\setcounter{lesson_finding_domain_algebraically}{30}
\setcounter{lesson_solving_functions}{31}
\setcounter{lesson_function_arithmetic}{32}
\setcounter{lesson_composite_functions}{33}
\setcounter{lesson_inverse_functions_definition_and_HLT}{34}
\setcounter{lesson_finding_an_inverse_function}{35}
\setcounter{lesson_transformations_translations}{36}
\setcounter{lesson_transformations_reflections}{37}
\setcounter{lesson_transformations_scalings}{38}
\setcounter{lesson_transformations_summary}{39}
\setcounter{lesson_piecewise_functions}{40}
\setcounter{lesson_functions_containing_absolute_values}{41}
\setcounter{lesson_absolute_as_piecewise}{42}
\setcounter{lesson_polynomials_introduction}{43}
\setcounter{lesson_sign_diagrams_polynomials}{44}
\setcounter{lesson_factoring_quadratic_type}{46}
\setcounter{lesson_factoring_summary}{45}
\setcounter{lesson_polynomial_division}{47}
\setcounter{lesson_synthetic_division}{48}
\setcounter{lesson_end_behavior_polynomials}{49}
\setcounter{lesson_local_behavior_polynomials}{50}
\setcounter{lesson_rational_root_theorem}{51}
\setcounter{lesson_polynomials_graphing_summary}{52}
\setcounter{lesson_polynomial_inequalities}{53}
\setcounter{lesson_rationals_introduction_and_terminology}{54}
\setcounter{lesson_sign_diagrams_rationals}{55}
\setcounter{lesson_horizontal_asymptotes}{56}
\setcounter{lesson_slant_and_curvilinear_asymptotes}{57}
\setcounter{lesson_vertical_asymptotes}{58}
\setcounter{lesson_holes}{59}
\setcounter{lesson_rationals_graphing_summary}{60}

\newcommand{\tmmathbf}[1]{\ensuremath{\boldsymbol{#1}}}
\newcommand{\tmop}[1]{\ensuremath{\operatorname{#1}}}

\begin{document}
\section{Local Behavior (L\arabic{lesson_local_behavior_polynomials})}
{\bf Objective: Identify all real roots and their corresponding multiplicities for a polynomial function that is easily factorable.}\par
In College Algebra and Precalculus, when we refer to the {\it local behavior} of a function $f,$ we will be concerned with anything of interest in the interior of the graph of $f,$ and not its end behavior.  For polynomials, this is the $x-$ and $y-$intercepts of the graph.  These points coincide with when $f(x)=0$ for any $x-$intercepts, and when $x=0$ in the case of the $y-$intercept.  In Calculus, local behavior will also include points where the graph changes inflection or achieves a local maximum or minimum value. 
\par
Since we should be very familiar with finding a $y-$intercept at this point, we will start with a simple example.
\begin{example}
Find the $y-$intercept for each of the following polynomials.
\begin{multicols}{2}
\begin{enumerate}
\item $f(x)=5x^3-\frac{1}{2}x^2+6x-18$
\item $g(x)=\frac{1}{2}(x-2)^2(x+5)(x-3)$
\end{enumerate}
\end{multicols}
\begin{enumerate}
\item $f(0)=-18.$  Hence, the graph of $f$ has a $y-$intercept at $(0,-18)$.
\item In the case of $g,$ we have to identify the constant term $a_0$ in the expanded form of the polynomial. Recalling Example \ref{poly_end_beh_1} from our last section, we can easily obtain this value without the need to expand $g$ in its entirety.
\begin{equation*}
\begin{split}
g(0) & =\frac{1}{2}(0-2)^2(0+5)(0-3)\\
&= \frac{1}{2}(-2)^2(5)(-3)\\
&= \frac{1}{2}(4)(-15)\\
&= 2(-15)\\
&= -30
\end{split}
\end{equation*}
Hence, our $y-$intercept for the graph of $g$ is $(0,-30)$.
\end{enumerate}
\end{example}
Although it is certainly important to identify the $y-$intercept of any polynomial, the primary objective of this section will be finding the roots of a polynomial and classifying the respective $x-$intercepts of its graph.  Since roots/$x-$intercepts coincide with when a function equals zero, $f(x)=0,$ this section will depend heavily on working with a polynomial that is either in factored form or for which a complete factorization is easily obtainable.  In a subsequent section of this chapter, we will see more a more advanced technique for finding a complete factorization of a polynomial, using polynomial division and the Rational Root Theorem.
\par
We begin our exploration of $x-$intercepts by revisiting a past example.
\begin{example}
Find all roots of the polynomial function $h(x)=(x+2)^2(3x-1)(5-x),$ and graph $h$ using \Desmos \ or another graphing utility.  For each root, identify whether the graph of $h$ crosses over or turns around at the corresponding $x-$intercept.
\par
For this example, we will first recall the work done Example \ref{sign_diag_poly_2}, where we identified the roots of $h$ to be $x=-2,\frac{1}{3},$ and $5,$ as well as the following sign diagram.
\begin{center}
\begin{tikzpicture}[xscale=1,yscale=1]
	\draw [<->](-4.25,0) -- coordinate (x axis mid) (7.25,0) node[below right] {$x$};
	\draw [-](-2,1) -- coordinate (y axis mid) (-2,-0.25) node[below] {$-2$};
	\draw [-](0.5,1) -- coordinate (y axis mid) (0.5,-0.25) node[below] {$\frac{1}{3}$};
	\draw [-](5,1) -- coordinate (y axis mid) (5,-0.25) node[below] {$5$};
	\draw (-3,-1) node {$x=-3$};
	\draw (-0.75,-1) node {$x=0$};
	\draw (2.75,-1) node {$x=1$};
	\draw (6,-1) node {$x=6$};
	\draw (-3,0.5) node {$-$};
	\draw (-0.75,0.5) node {$-$};
	\draw (2.75,0.5) node {$+$};
	\draw (6,0.5) node {$-$};
\end{tikzpicture}
\end{center}

Our graph of $h$ is shown below.

\begin{center}
\begin{tikzpicture}[xscale=0.75,yscale=0.015]
	\draw [<->](-8.25,0) -- coordinate (x axis mid) (8.25,0) node[below right] {$x$};
	\draw [<->](0,-250) -- coordinate (y axis mid) (0,450) node[above right] {$y$};
	%\draw [dashed, <->](1.5,-6.25) -- coordinate (y axis mid) (1.5,6.25) node[above right] {};
	\draw [<->] plot [domain=-3.513:5.285, samples=100] (\x,{(\x+2)^2*(3*\x-1)*(5-\x)});
	\foreach \x in {2,4,...,8}
		\draw (\x,25pt) -- (\x,-25pt)	node[anchor=south] {\scriptsize \x};
	\foreach \x in {-8,-6,...,-2}
		\draw (\x,25pt) -- (\x,-25pt)	node[anchor=south] {\scriptsize \x};
	\foreach \x in {1,3,...,8}
		\draw (\x,50pt) -- (\x,-50pt)	node[anchor=south] {};
	\foreach \x in {-8,-7,...,-1}
		\draw (\x,50pt) -- (\x,-50pt)	node[anchor=south] {};
	\foreach \y in {100,200,...,400}
		\draw (2pt,\y) -- (-2pt,\y)	node[anchor=east] {\scriptsize \y}; 
	\foreach \y in {10,20,...,440}
		\draw (1pt,\y) -- (-1pt,\y)	node[anchor=east] {}; 
	\foreach \y in {-200,-100}
		\draw (2pt,\y) -- (-2pt,\y)	node[anchor=west] {\scriptsize \y}; 
	\foreach \y in {-240,-230,...,-10}
		\draw (1pt,\y) -- (-1pt,\y)	node[anchor=west] {}; 
\end{tikzpicture}
\end{center}
Based upon our picture, we see that the graph of $h$ crosses over the $x-$axis at $x=\frac{1}{3}$ and $x=5$.  The graph turns around at $x=-2$.  \end{example}
In our last example, we have included our sign diagram to point out a connection.  Our diagram confirms the nature of each $x-$intercept without the need to graph $h$, since both sides of our {\it turnaround point} $x=-2$ show the {\it same} sign (either $+|+$ or $-|-$).  Similarly, the signs {\it change} from either positive to negative ($+|-$) or negative to positive ($-|+$) for each of our {\it crossover points}.
\par
In fact, this idea of turnaround and crossover points can be parsed down to one basic concept, known as the {\it multiplicity} of a root.  We define the multiplicity of a root below, followed immediately by an example for clarification.
\par
Suppose $f$ is a polynomial function with real root $x=c$.  For some positive integer $k,$ if $(x-c)^{k}$ is a factor of $f$ but $(x-c)^{k+1}$ is not, then we say $x=c$ is a root of $f$ having associated multiplicity $k$.
\begin{example}
Determine the set of roots and corresponding multiplicities for the following functions.
\begin{multicols}{2}
\begin{enumerate}
	\item $f(x)=x^6-2x^5-15x^4$
	\item $g(x)=(x-6)^5(x+2)^2(x^2+1)$
\end{enumerate}
\end{multicols}
\begin{enumerate}
	\item Factoring $f$ gives us the following.
	\begin{equation*}
	\begin{split}
		f(x)&=x^6-2x^5-15x^4\\
		&=x^4(x^2-2x-15)\\
		&=x^4(x-5)(x+3)
	\end{split}
	\end{equation*}
We then can easily see that $f$ has a root at $x=0$ with multiplicity four, and roots at $x=5$ and $x=-3,$ each with multiplicity one.
\item Since $g$ is already factored, we see that $x=6$ is a root having multiplicity five, and $x=-2$ is a root having multiplicity two.  The factor of $x^2+1$ is meant to throw us off, since its roots are the imaginary numbers $\pm i$.
\end{enumerate}
\end{example}
Another way of describing the multiplicity $k$ of a root $x=c$ is that $k$ represents the maximum number of factors of $(x-c)$ that divide the polynomial $f$ (with a remainder of $0$).  That is,
$$f(x)=(x-c)^k\cdot q(x),$$
where $(x-c)$ is {\it not} a factor of the quotient $q(x)$.
\par
If we apply this idea to $g$ in our last example, we see that although $(x-6)^4$ divides our polynomial,
$$g(x)=(x-6)^4\cdot \underbrace{(x-6)(x+2)^2(x^2+1)}_{q(x)},$$
the value of four does not represent the {\it maximum} number of factors of $(x-6)$ that divide $g$: 
$$g(x)=(x-6)^5\cdot \underbrace{(x+2)^2(x^2+1)}_{q(x)}.$$
At this point, we are ready to highlight the importance of multiplicities in graphing polynomials.
\begin{center}
\framebox{
\begin{minipage}{1\linewidth}
Let $f$ be a polynomial function with a real root at $x=c$ having multiplicity $k$.
	\begin{itemize}
		\item If $k$ is {\it even}, the corresponding $x-$intercept $(c,0)$ is a {\it turnaround point}.  In other words, the graph of $f$ touches and rebounds from the $x$-axis at $(c,0)$, leaving the $y-$values to maintain the same sign on either side of the root $x=c$.
		\item If $k$ is {\it odd}, the corresponding $x-$intercept $(c,0)$ is a {\it crossover point}.  In other words, the graph of $f$ crosses through the $x$-axis at $(c,0)$, leaving the $y-$values to change signs on either side of the root $x=c$.
	\end{itemize}
\end{minipage}
}
\end{center}
Combining this new notion about multiplicities of roots with all that we have already learned about polynomials will enable us to quickly identify all important aspects of a particular polynomial function, culminating in a sketch of its graph.  We capitalize on this in our next example.
\begin{example}
Construct a sign diagram for the factored polynomial\\ $f(x)=-(x-3)^2(x+1)(x+5)^2$.
\par
The dividers for our sign diagram come from the set of roots of $f,$ namely $\{-5,-1,3\}$.  
\begin{center}
\begin{tikzpicture}[xscale=1,yscale=1]
	\draw [<->](-7.25,0) -- coordinate (x axis mid) (5.25,0) node[below right] {$x$};
	\draw [-](-5,1) -- coordinate (y axis mid) (-5,-0.25) node[below] {$-5$};
	\draw [-](-1,1) -- coordinate (y axis mid) (-1,-0.25) node[below] {$-1$};
	\draw [-](3,1) -- coordinate (y axis mid) (3,-0.25) node[below] {$3$};
\end{tikzpicture}
\end{center}
Instead of assigning test values, however, we will use both multiplicities and end behavior to determine our various signs.
\par
First, we identify the leading term of $f$.
$$a_nx^n=-(x)^2(x)(x)^2=-x^5$$
Since $a_n<0$ and $n=5$ is odd, our end behavior follows case IV:
\begin{center}
As $x\rightarrow -\infty, \ f(x)\rightarrow\infty.$ \hspace{1in} As $x\rightarrow\infty, \ f(x)\rightarrow -\infty.$
\end{center}
This tells us that our diagram will begin with a positive sign and end with a negative sign.
\begin{center}
\begin{tikzpicture}[xscale=1,yscale=1]
	\draw [<->](-7.25,0) -- coordinate (x axis mid) (5.25,0) node[below right] {$x$};
	\draw [-](-5,1) -- coordinate (y axis mid) (-5,-0.25) node[below] {$-5$};
	\draw [-](-1,1) -- coordinate (y axis mid) (-1,-0.25) node[below] {$-1$};
	\draw [-](3,1) -- coordinate (y axis mid) (3,-0.25) node[below] {$3$};
	\draw (-6,0.5) node {$+$};
	\draw (4,0.5) node {$-$};
\end{tikzpicture}
\end{center}
Furthermore, the multiplicity of the root $x=-5$ is two, which is even.  So, our diagram must contain the same signs on either side of $x=-5,$ namely two positive signs.
\begin{center}
\begin{tikzpicture}[xscale=1,yscale=1]
	\draw [<->](-7.25,0) -- coordinate (x axis mid) (5.25,0) node[below right] {$x$};
	\draw [-](-5,1) -- coordinate (y axis mid) (-5,-0.25) node[below] {$-5$};
	\draw [-](-1,1) -- coordinate (y axis mid) (-1,-0.25) node[below] {$-1$};
	\draw [-](3,1) -- coordinate (y axis mid) (3,-0.25) node[below] {$3$};
	\draw (-6,0.5) node {$+$};
	\draw (-3,0.5) node {$+$};
	\draw (4,0.5) node {$-$};
\end{tikzpicture}
\end{center}
Applying this same idea to both of our remaining roots, we get the following diagram, and are done!
\begin{center}
\begin{tikzpicture}[xscale=1,yscale=1]
	\draw [<->](-7.25,0) -- coordinate (x axis mid) (5.25,0) node[below right] {$x$};
	\draw [-](-5,1) -- coordinate (y axis mid) (-5,-0.25) node[below] {$-5$};
	\draw [-](-1,1) -- coordinate (y axis mid) (-1,-0.25) node[below] {$-1$};
	\draw [-](3,1) -- coordinate (y axis mid) (3,-0.25) node[below] {$3$};
	\draw (-6,0.5) node {$+$};
	\draw (-3,0.5) node {$+$};
	\draw (1,0.5) node {$-$};
	\draw (4,0.5) node {$-$};
\end{tikzpicture}
\end{center}
\end{example}
In fact, we can take this last example further, and easily sketch a graph of our polynomial.  All that remains is to identify the $y-$intercept.
\begin{example}
Sketch a graph of the factored polynomial $f(x)=-(x-3)^2(x+1)(x+5)^2,$ making sure to identify a clearly defined scale and any $x-$ and $y-$intercepts.
\par
Since the roots, $x=3$ and $x=-5$ have even multiplicities, their corresponding $x-$intercepts will be turnaround points.  Our sign diagram confirms this, and further shows that the intercept at $x=-5$ will be a local minimum ($+|+$), whereas the intercept at $x=3$ will be a local maximum ($-|-$).  On the other hand, the root at $x=-1$ has an odd multiplicity, and the $x-$intercept at $x=-1$ will be a crossover point.
\par
For a $y-$intercept, we evaluate the function at $x=0$.
\begin{equation*}
\begin{split}
f(0)&=-(0-3)^2(0+1)(0+5)^2\\
&=-(9)(1)(25)\\
&=-225
\end{split}
\end{equation*}
Since our $y-$intercept is a large negative value, we will have to shrink our scale for the $y-$axis accordingly.
\begin{center}
\begin{tikzpicture}[xscale=0.75,yscale=0.012]
	\draw [<->](-8.25,0) -- coordinate (x axis mid) (8.25,0) node[below right] {$x$};
	\draw [<->](0,-350) -- coordinate (y axis mid) (0,350) node[above right] {$y$};
	\draw [<->] plot [domain=-5.913:3.913, samples=100] (\x,{(-1)*(\x+5)^2*(\x+1)*(\x-3)^2});
	\foreach \x in {2,4,...,8}
		\draw (\x,25pt) -- (\x,-25pt)	node[anchor=south] {\scriptsize \x};
	\foreach \x in {-8,-6,...,-2}
		\draw (\x,25pt) -- (\x,-25pt)	node[anchor=south] {\scriptsize \x};
	\foreach \x in {1,2,...,8}
		\draw (\x,50pt) -- (\x,-50pt)	node[anchor=south] {};
	\foreach \x in {-8,-7,...,-1}
		\draw (\x,50pt) -- (\x,-50pt)	node[anchor=south] {};
	\foreach \y in {100,200,...,300}
		\draw (2pt,\y) -- (-2pt,\y)	node[anchor=east] {\scriptsize \y}; 
	\foreach \y in {25,50,...,325}
		\draw (1pt,\y) -- (-1pt,\y)	node[anchor=east] {}; 
	\foreach \y in {-300,-200,...,-100}
		\draw (2pt,\y) -- (-2pt,\y)	node[anchor=west] {\scriptsize \y}; 
	\foreach \y in {-325,-300,...,-25}
		\draw (1pt,\y) -- (-1pt,\y)	node[anchor=west] {}; 
\end{tikzpicture}
\end{center}
\end{example}
This last example achieves what we have sought after since beginning the chapter.  In it, we have identified the end behavior and all intercepts of a completely factored polynomial.  We have further used both a sign diagram and a multiplicity argument in order to graph the given function.
\par
What remains is to take a closer look at more challenging polynomials, for which a factorization may not be given or even easily identifiable.
\end{document}
\documentclass[12pt]{book}
\raggedbottom
\usepackage[top=1in,left=1in,bottom=1in,right=1in,headsep=0.25in]{geometry}	
\usepackage{amssymb,amsmath,amsthm,amsfonts}
\usepackage{chapterfolder,docmute,setspace}
\usepackage{cancel,multicol,tikz,verbatim,framed,polynom,enumitem,tikzpagenodes}
\usepackage[colorlinks, hyperindex, plainpages=false, linkcolor=blue, urlcolor=blue, pdfpagelabels]{hyperref}
\usepackage[type={CC},modifier={by-sa},version={4.0},]{doclicense}

\theoremstyle{definition}
\newtheorem{example}{Example}
\newcommand{\Desmos}{\href{https://www.desmos.com/}{Desmos}}
\setlength{\parindent}{0in}
\setlist{itemsep=0in}
\setlength{\parskip}{0.1in}
\setcounter{secnumdepth}{0}
% This document is used for ordering of lessons.  If an instructor wishes to change the ordering of assessments, the following steps must be taken:

% 1) Reassign the appropriate numbers for each lesson in the \setcounter commands included in this file.
% 2) Rearrange the \include commands in the master file (the file with 'Course Pack' in the name) to accurately reflect the changes.  
% 3) Rarrange the \items in the measureable_outcomes file to accurately reflect the changes.  Be mindful of page breaks when moving items.
% 4) Re-build all affected files (master file, measureable_outcomes file, and any lessons whose numbering has changed).

%Note: The placement of each \newcounter and \setcounter command reflects the original/default ordering of topics (linears, systems, quadratics, functions, polynomials, rationals).

\newcounter{lesson_solving_linear_equations}
\newcounter{lesson_equations_containing_absolute_values}
\newcounter{lesson_graphing_lines}
\newcounter{lesson_two_forms_of_a_linear_equation}
\newcounter{lesson_parallel_and_perpendicular_lines}
\newcounter{lesson_linear_inequalities}
\newcounter{lesson_compound_inequalities}
\newcounter{lesson_inequalities_containing_absolute_values}
\newcounter{lesson_graphing_systems}
\newcounter{lesson_substitution}
\newcounter{lesson_elimination}
\newcounter{lesson_quadratics_introduction}
\newcounter{lesson_factoring_GCF}
\newcounter{lesson_factoring_grouping}
\newcounter{lesson_factoring_trinomials_a_is_1}
\newcounter{lesson_factoring_trinomials_a_neq_1}
\newcounter{lesson_solving_by_factoring}
\newcounter{lesson_square_roots}
\newcounter{lesson_i_and_complex_numbers}
\newcounter{lesson_vertex_form_and_graphing}
\newcounter{lesson_solve_by_square_roots}
\newcounter{lesson_extracting_square_roots}
\newcounter{lesson_the_discriminant}
\newcounter{lesson_the_quadratic_formula}
\newcounter{lesson_quadratic_inequalities}
\newcounter{lesson_functions_and_relations}
\newcounter{lesson_evaluating_functions}
\newcounter{lesson_finding_domain_and_range_graphically}
\newcounter{lesson_fundamental_functions}
\newcounter{lesson_finding_domain_algebraically}
\newcounter{lesson_solving_functions}
\newcounter{lesson_function_arithmetic}
\newcounter{lesson_composite_functions}
\newcounter{lesson_inverse_functions_definition_and_HLT}
\newcounter{lesson_finding_an_inverse_function}
\newcounter{lesson_transformations_translations}
\newcounter{lesson_transformations_reflections}
\newcounter{lesson_transformations_scalings}
\newcounter{lesson_transformations_summary}
\newcounter{lesson_piecewise_functions}
\newcounter{lesson_functions_containing_absolute_values}
\newcounter{lesson_absolute_as_piecewise}
\newcounter{lesson_polynomials_introduction}
\newcounter{lesson_sign_diagrams_polynomials}
\newcounter{lesson_factoring_quadratic_type}
\newcounter{lesson_factoring_summary}
\newcounter{lesson_polynomial_division}
\newcounter{lesson_synthetic_division}
\newcounter{lesson_end_behavior_polynomials}
\newcounter{lesson_local_behavior_polynomials}
\newcounter{lesson_rational_root_theorem}
\newcounter{lesson_polynomials_graphing_summary}
\newcounter{lesson_polynomial_inequalities}
\newcounter{lesson_rationals_introduction_and_terminology}
\newcounter{lesson_sign_diagrams_rationals}
\newcounter{lesson_horizontal_asymptotes}
\newcounter{lesson_slant_and_curvilinear_asymptotes}
\newcounter{lesson_vertical_asymptotes}
\newcounter{lesson_holes}
\newcounter{lesson_rationals_graphing_summary}

\setcounter{lesson_solving_linear_equations}{1}
\setcounter{lesson_equations_containing_absolute_values}{2}
\setcounter{lesson_graphing_lines}{3}
\setcounter{lesson_two_forms_of_a_linear_equation}{4}
\setcounter{lesson_parallel_and_perpendicular_lines}{5}
\setcounter{lesson_linear_inequalities}{6}
\setcounter{lesson_compound_inequalities}{7}
\setcounter{lesson_inequalities_containing_absolute_values}{8}
\setcounter{lesson_graphing_systems}{9}
\setcounter{lesson_substitution}{10}
\setcounter{lesson_elimination}{11}
\setcounter{lesson_quadratics_introduction}{16}
\setcounter{lesson_factoring_GCF}{17}
\setcounter{lesson_factoring_grouping}{18}
\setcounter{lesson_factoring_trinomials_a_is_1}{19}
\setcounter{lesson_factoring_trinomials_a_neq_1}{20}
\setcounter{lesson_solving_by_factoring}{21}
\setcounter{lesson_square_roots}{22}
\setcounter{lesson_i_and_complex_numbers}{23}
\setcounter{lesson_vertex_form_and_graphing}{24}
\setcounter{lesson_solve_by_square_roots}{25}
\setcounter{lesson_extracting_square_roots}{26}
\setcounter{lesson_the_discriminant}{27}
\setcounter{lesson_the_quadratic_formula}{28}
\setcounter{lesson_quadratic_inequalities}{29}
\setcounter{lesson_functions_and_relations}{12}
\setcounter{lesson_evaluating_functions}{13}
\setcounter{lesson_finding_domain_and_range_graphically}{14}
\setcounter{lesson_fundamental_functions}{15}
\setcounter{lesson_finding_domain_algebraically}{30}
\setcounter{lesson_solving_functions}{31}
\setcounter{lesson_function_arithmetic}{32}
\setcounter{lesson_composite_functions}{33}
\setcounter{lesson_inverse_functions_definition_and_HLT}{34}
\setcounter{lesson_finding_an_inverse_function}{35}
\setcounter{lesson_transformations_translations}{36}
\setcounter{lesson_transformations_reflections}{37}
\setcounter{lesson_transformations_scalings}{38}
\setcounter{lesson_transformations_summary}{39}
\setcounter{lesson_piecewise_functions}{40}
\setcounter{lesson_functions_containing_absolute_values}{41}
\setcounter{lesson_absolute_as_piecewise}{42}
\setcounter{lesson_polynomials_introduction}{43}
\setcounter{lesson_sign_diagrams_polynomials}{44}
\setcounter{lesson_factoring_quadratic_type}{46}
\setcounter{lesson_factoring_summary}{45}
\setcounter{lesson_polynomial_division}{47}
\setcounter{lesson_synthetic_division}{48}
\setcounter{lesson_end_behavior_polynomials}{49}
\setcounter{lesson_local_behavior_polynomials}{50}
\setcounter{lesson_rational_root_theorem}{51}
\setcounter{lesson_polynomials_graphing_summary}{52}
\setcounter{lesson_polynomial_inequalities}{53}
\setcounter{lesson_rationals_introduction_and_terminology}{54}
\setcounter{lesson_sign_diagrams_rationals}{55}
\setcounter{lesson_horizontal_asymptotes}{56}
\setcounter{lesson_slant_and_curvilinear_asymptotes}{57}
\setcounter{lesson_vertical_asymptotes}{58}
\setcounter{lesson_holes}{59}
\setcounter{lesson_rationals_graphing_summary}{60}

\newcommand{\tmmathbf}[1]{\ensuremath{\boldsymbol{#1}}}
\newcommand{\tmop}[1]{\ensuremath{\operatorname{#1}}}

\begin{document}
\section{The Rational Root Theorem (L\arabic{lesson_rational_root_theorem})}
{\bf Objective: Apply the Rational Root Theorem to determine a set of possible rational roots for and a factorization of a given polynomial.}\par
The Rational Root Theorem is used to identify a list of all possible rational roots for a given polynomial.
\begin{center}
\framebox{
\begin{minipage}{0.9\linewidth}
{\bf Rational Root Theorem:}  Suppose $f(x) = a_{n} x^{n} + a_{n-\mbox{\tiny$1$}}x^{n-\mbox{\tiny$1$}} + \ldots + a_{\mbox{\tiny$1$}} x + a_{\mbox{\tiny$0$}}$ is a polynomial of degree $n$ with $n \geq 1$, and $a_{\mbox{\tiny$0$}}$, $a_{\mbox{\tiny$1$}}$, \ldots $a_{n}$ are integers.  If $r$ is a rational root of $f$, then $r$ is of the form $\pm \frac{p}{q}$, where $p$ is a factor of the constant term $a_{\mbox{\tiny$0$}}$, and $q$ is a factor of the leading coefficient $a_{n}$.
\end{minipage}
}
\end{center}
The Rational Root Theorem gives us a list of numbers to test as roots of a given polynomial using synthetic division, which is a nicer approach than simply guessing at possible roots.  If none of the numbers in the list turn out to be roots, then either the polynomial has no real roots at all, or all of the real roots will be irrational numbers.\par
\begin{example}~~Let $f(x) = 2x^4+4x^3-x^2-6x-3$. Use the Rational Root Theorem to list all of the possible rational roots of $f$.\par
To generate a complete list of rational roots, we need to take each of the factors of the constant term, $a_{\mbox{\tiny$0$}} = -3$, and divide them by each of the factors of the leading coefficient $a_{\mbox{\tiny$4$}} = 2$.\par
The factors of $-3$ are $\pm \, 1$ and $\pm \, 3$.  Since the Rational Root Theorem tacks on a $\pm$ anyway, for the moment, we consider only the positive factors $1$ and $3$.  The factors of $2$ are  $1$ and $2$, so the Rational Root Theorem gives the list $\left\{\pm \, \frac{1}{1}, \pm \, \frac{1}{2},  \pm \, \frac{3}{1}, \pm \, \frac{3}{2}\right\}$ or $\left\{\pm \, \frac{1}{2}, \pm \, 1, \pm \, \frac{3}{2}, \pm \, 3\right\}$.\par
Additionally, we can evaluate $f$ at each of the eight potential rational roots in our list, to see if any of them are indeed roots.  Starting with $\pm 1,$ we see that 
\begin{center}
$f(1)=2+4-1-6-3=-4\neq 0\qquad$ and $\qquad f(-1)=2-4-1+6-3=0.$
\end{center}
\begin{multicols}{2}
Hence, we can conclude that $x=-1$ is a root of $f$ and $x=1$ is not.  Using synthetic division, we can then divide $f$ by the linear factor $x+1$ as follows.
\columnbreak
\begin{center}
$\polyhornerscheme[x=-1,showbase=top,resultstyle=\bf]{2x^4+4x^3-x^2-6x-3}$
\end{center}
\end{multicols}
We can then begin to factor $f,$
$$2x^4+4x^3-x^2-6x-3=(x+1)(2x^3+2x^2-3x-3)$$
The resulting quotient polynomial is then factorable by grouping,
$$2x^3+2x^2-3x-3=(2x^2-3)(x+1).$$
Factoring out a $2$ from the expression $2x^2-3,$ allows us to factor it as the difference of two squares,
\begin{eqnarray*}
2x^2-3 & = & 2\left(x^2-\frac{3}{2}\right)\\
& = & 2\left(x-\sqrt{\frac{3}{2}}\right)\left(x+\sqrt{\frac{3}{2}}\right)\\
& = & 2\left(x-\frac{\sqrt{6}}{2}\right)\left(x+\frac{\sqrt{6}}{2}\right)\\
\end{eqnarray*}
So, a complete factorization for $f$ would be
$$2x^4+4x^3-x^2-6x-3=2\left(x-\frac{\sqrt{6}}{2}\right)\left(x+\frac{\sqrt{6}}{2}\right)\left(x+1\right)^2,$$
and the set of real roots for $f$ is $\left\{-1,\pm\dfrac{\sqrt{6}}{2}\right\}$.
\end{example}
\end{document}
\documentclass[12pt]{book}
\raggedbottom
\usepackage[top=1in,left=1in,bottom=1in,right=1in,headsep=0.25in]{geometry}	
\usepackage{amssymb,amsmath,amsthm,amsfonts}
\usepackage{chapterfolder,docmute,setspace}
\usepackage{cancel,multicol,tikz,verbatim,framed,polynom,enumitem,tikzpagenodes}
\usepackage[colorlinks, hyperindex, plainpages=false, linkcolor=blue, urlcolor=blue, pdfpagelabels]{hyperref}
\usepackage[type={CC},modifier={by-sa},version={4.0},]{doclicense}

\theoremstyle{definition}
\newtheorem{example}{Example}
\newcommand{\Desmos}{\href{https://www.desmos.com/}{Desmos}}
\setlength{\parindent}{0in}
\setlist{itemsep=0in}
\setlength{\parskip}{0.1in}
\setcounter{secnumdepth}{0}
% This document is used for ordering of lessons.  If an instructor wishes to change the ordering of assessments, the following steps must be taken:

% 1) Reassign the appropriate numbers for each lesson in the \setcounter commands included in this file.
% 2) Rearrange the \include commands in the master file (the file with 'Course Pack' in the name) to accurately reflect the changes.  
% 3) Rarrange the \items in the measureable_outcomes file to accurately reflect the changes.  Be mindful of page breaks when moving items.
% 4) Re-build all affected files (master file, measureable_outcomes file, and any lessons whose numbering has changed).

%Note: The placement of each \newcounter and \setcounter command reflects the original/default ordering of topics (linears, systems, quadratics, functions, polynomials, rationals).

\newcounter{lesson_solving_linear_equations}
\newcounter{lesson_equations_containing_absolute_values}
\newcounter{lesson_graphing_lines}
\newcounter{lesson_two_forms_of_a_linear_equation}
\newcounter{lesson_parallel_and_perpendicular_lines}
\newcounter{lesson_linear_inequalities}
\newcounter{lesson_compound_inequalities}
\newcounter{lesson_inequalities_containing_absolute_values}
\newcounter{lesson_graphing_systems}
\newcounter{lesson_substitution}
\newcounter{lesson_elimination}
\newcounter{lesson_quadratics_introduction}
\newcounter{lesson_factoring_GCF}
\newcounter{lesson_factoring_grouping}
\newcounter{lesson_factoring_trinomials_a_is_1}
\newcounter{lesson_factoring_trinomials_a_neq_1}
\newcounter{lesson_solving_by_factoring}
\newcounter{lesson_square_roots}
\newcounter{lesson_i_and_complex_numbers}
\newcounter{lesson_vertex_form_and_graphing}
\newcounter{lesson_solve_by_square_roots}
\newcounter{lesson_extracting_square_roots}
\newcounter{lesson_the_discriminant}
\newcounter{lesson_the_quadratic_formula}
\newcounter{lesson_quadratic_inequalities}
\newcounter{lesson_functions_and_relations}
\newcounter{lesson_evaluating_functions}
\newcounter{lesson_finding_domain_and_range_graphically}
\newcounter{lesson_fundamental_functions}
\newcounter{lesson_finding_domain_algebraically}
\newcounter{lesson_solving_functions}
\newcounter{lesson_function_arithmetic}
\newcounter{lesson_composite_functions}
\newcounter{lesson_inverse_functions_definition_and_HLT}
\newcounter{lesson_finding_an_inverse_function}
\newcounter{lesson_transformations_translations}
\newcounter{lesson_transformations_reflections}
\newcounter{lesson_transformations_scalings}
\newcounter{lesson_transformations_summary}
\newcounter{lesson_piecewise_functions}
\newcounter{lesson_functions_containing_absolute_values}
\newcounter{lesson_absolute_as_piecewise}
\newcounter{lesson_polynomials_introduction}
\newcounter{lesson_sign_diagrams_polynomials}
\newcounter{lesson_factoring_quadratic_type}
\newcounter{lesson_factoring_summary}
\newcounter{lesson_polynomial_division}
\newcounter{lesson_synthetic_division}
\newcounter{lesson_end_behavior_polynomials}
\newcounter{lesson_local_behavior_polynomials}
\newcounter{lesson_rational_root_theorem}
\newcounter{lesson_polynomials_graphing_summary}
\newcounter{lesson_polynomial_inequalities}
\newcounter{lesson_rationals_introduction_and_terminology}
\newcounter{lesson_sign_diagrams_rationals}
\newcounter{lesson_horizontal_asymptotes}
\newcounter{lesson_slant_and_curvilinear_asymptotes}
\newcounter{lesson_vertical_asymptotes}
\newcounter{lesson_holes}
\newcounter{lesson_rationals_graphing_summary}

\setcounter{lesson_solving_linear_equations}{1}
\setcounter{lesson_equations_containing_absolute_values}{2}
\setcounter{lesson_graphing_lines}{3}
\setcounter{lesson_two_forms_of_a_linear_equation}{4}
\setcounter{lesson_parallel_and_perpendicular_lines}{5}
\setcounter{lesson_linear_inequalities}{6}
\setcounter{lesson_compound_inequalities}{7}
\setcounter{lesson_inequalities_containing_absolute_values}{8}
\setcounter{lesson_graphing_systems}{9}
\setcounter{lesson_substitution}{10}
\setcounter{lesson_elimination}{11}
\setcounter{lesson_quadratics_introduction}{16}
\setcounter{lesson_factoring_GCF}{17}
\setcounter{lesson_factoring_grouping}{18}
\setcounter{lesson_factoring_trinomials_a_is_1}{19}
\setcounter{lesson_factoring_trinomials_a_neq_1}{20}
\setcounter{lesson_solving_by_factoring}{21}
\setcounter{lesson_square_roots}{22}
\setcounter{lesson_i_and_complex_numbers}{23}
\setcounter{lesson_vertex_form_and_graphing}{24}
\setcounter{lesson_solve_by_square_roots}{25}
\setcounter{lesson_extracting_square_roots}{26}
\setcounter{lesson_the_discriminant}{27}
\setcounter{lesson_the_quadratic_formula}{28}
\setcounter{lesson_quadratic_inequalities}{29}
\setcounter{lesson_functions_and_relations}{12}
\setcounter{lesson_evaluating_functions}{13}
\setcounter{lesson_finding_domain_and_range_graphically}{14}
\setcounter{lesson_fundamental_functions}{15}
\setcounter{lesson_finding_domain_algebraically}{30}
\setcounter{lesson_solving_functions}{31}
\setcounter{lesson_function_arithmetic}{32}
\setcounter{lesson_composite_functions}{33}
\setcounter{lesson_inverse_functions_definition_and_HLT}{34}
\setcounter{lesson_finding_an_inverse_function}{35}
\setcounter{lesson_transformations_translations}{36}
\setcounter{lesson_transformations_reflections}{37}
\setcounter{lesson_transformations_scalings}{38}
\setcounter{lesson_transformations_summary}{39}
\setcounter{lesson_piecewise_functions}{40}
\setcounter{lesson_functions_containing_absolute_values}{41}
\setcounter{lesson_absolute_as_piecewise}{42}
\setcounter{lesson_polynomials_introduction}{43}
\setcounter{lesson_sign_diagrams_polynomials}{44}
\setcounter{lesson_factoring_quadratic_type}{46}
\setcounter{lesson_factoring_summary}{45}
\setcounter{lesson_polynomial_division}{47}
\setcounter{lesson_synthetic_division}{48}
\setcounter{lesson_end_behavior_polynomials}{49}
\setcounter{lesson_local_behavior_polynomials}{50}
\setcounter{lesson_rational_root_theorem}{51}
\setcounter{lesson_polynomials_graphing_summary}{52}
\setcounter{lesson_polynomial_inequalities}{53}
\setcounter{lesson_rationals_introduction_and_terminology}{54}
\setcounter{lesson_sign_diagrams_rationals}{55}
\setcounter{lesson_horizontal_asymptotes}{56}
\setcounter{lesson_slant_and_curvilinear_asymptotes}{57}
\setcounter{lesson_vertical_asymptotes}{58}
\setcounter{lesson_holes}{59}
\setcounter{lesson_rationals_graphing_summary}{60}

\newcommand{\tmmathbf}[1]{\ensuremath{\boldsymbol{#1}}}
\newcommand{\tmop}[1]{\ensuremath{\operatorname{#1}}}

\begin{document}
\section{Graphing Summary (L\arabic{lesson_polynomials_graphing_summary})}
{\bf Objective: Graph a polynomial function in its entirety.}\par
At this point, we have addressed all key features of polynomials individually.  This section pulls each of these aspects together, for a detailed analysis of a polynomial, culminating in a complete sketch of its graph.  Along the way, we will need to address each of the following aspects for our polynomial $f(x)=a_nx^n+a_{n-1}x^{n-1}+\ldots+a_1x+a_0$.  It is important to note that there is no universally accepted order to this checklist.
\begin{itemize}
	\item Find the $y-$intercept of the graph of $f,$ $(0,f(0))=(0,a_0)$.
	\item Use the degree $n$ and leading coefficient $a_n$ to determine the end behavior of the graph of $f$.
	\item Identify a complete factorization of $f,$ and use it to find any $x-$intercepts of the graph of $f$.  Using multiplicities, classify each $x-$intercept as a crossover or turnaround (``bounce'') point.
	\item Using the $x-$intercepts, construct a sign diagram for $f$.
\end{itemize}
In each polynomial we encounter, we will carefully examine the function, making sure not to omit any of the checklist items above and to compare each item to those that precede it along the way for accuracy.  Although the process will take some time, if we are thorough, our end result should be a complete, accurate sketch of the given polynomial.\par
\begin{example}~~Sketch a complete graph of $f(x)=14x^4-17x^3-6x^2+7x+2$.\par
We will start with the $y-$intercept, which is $(0,2)$.\par
Next, we see that $f$ has even degree and positive leading coefficient.  So, the tails of the graph of $f$ both point upwards.  In other words, as $x\rightarrow\pm\infty, f(x)\rightarrow\infty$.\par
Since $f$ is degree-4, contains more than four terms, and is not of quadratic type, we will apply the Rational Root Theorem.  In this case, our set of possible rational roots is
$$\left\{\pm 1, \pm 2, \pm \frac{1}{2}, \pm \frac{1}{7}, \pm \frac{1}{14}, \pm \frac{2}{7}\right\}$$
Fortunately, we see that $f(1)=14-17-6+7+2=0$.  So, $x-1$ is a factor of $f$.  Dividing, we get:
\begin{multicols}{2}
\polylongdiv{14x^4-17x^3-6x^2+7x+2}{x-1}

\columnbreak

\polyhornerscheme[x=1,showbase=top,resultstyle=\bf]{14x^4-17x^3-6x^2+7x+2}
\end{multicols}
So, $f(x)=(x-1)(14x^3-3x^2-9x-2).$  Applying the Rational Root Theorem a second time, we can see that $x=1$ is also a root of the cubic factor of $f,$ since $14-3-9-2=0$.  Again, we can divide to factor $f$ further.
\begin{multicols}{2}
\polylongdiv{14x^3-3x^2-9x-2}{x-1}

\columnbreak

\polyhornerscheme[x=1,showbase=top,resultstyle=\bf]{14x^3-3x^2-9x-2}
\end{multicols}
So, $f(x)=(x-1)^2(14x^2+11x+2)$.  Factoring the remaining quadratic, we have
$$f(x)=(x-1)^2(7x+2)(2x+1),$$
with accompanying set of roots $\{1,-\frac{1}{2},-\frac{2}{7}\}$.\par
Using multiplicities, we conclude that the $x-$intercept $(1,0)$ is a turnaround point, and the intercepts $\left(-\frac{1}{2},0\right)$ and $\left(-\frac{2}{7},0\right)$ are crossover points.\par
Though not necessary for graphing, a sign diagram confirms our end and local behavior findings.
\begin{center}
\begin{tikzpicture}[xscale=1,yscale=1]
	\draw [<->](-6.25,0) -- coordinate (x axis mid) (4.25,0) node[below right] {$x$};
	\draw [-](-4,1) -- coordinate (y axis mid) (-4,-0.25) node[below] {$-\frac{1}{2}$};
	\draw [-](-1,1) -- coordinate (y axis mid) (-1,-0.25) node[below] {$-\frac{2}{7}$};
	\draw [-](2,1) -- coordinate (y axis mid) (2,-0.25) node[below] {$1$};
	\draw (-5.5,-1) node {$x=-1$};
	\draw (-2.5,-1) node {$x=-\frac{3}{7}$};
	\draw (0.5,-1) node {$x=0$};
	\draw (3.5,-1) node {$x=2$};
	\draw (-5.5,0.5) node {$+$};
	\draw (-2.5,0.5) node {$-$};
	\draw (0.5,0.5) node {$+$};
	\draw (3.5,0.5) node {$+$};
\end{tikzpicture}
\end{center}
Putting all of this information together results in the following graph.
\begin{center}
\begin{tikzpicture}[xscale=1.75,yscale=1]
	\draw [<->](-2.25,0) -- coordinate (x axis mid) (2.25,0) node[below right] {$x$};
	\draw [<->](0,-1.75) -- coordinate (y axis mid) (0,5.25) node[above right] {$y$};
	\draw [<->] plot [domain=-0.736:1.33, samples=100] (\x,{(2*\x+1)*(7*\x+2)*(\x-1)*(\x-1)});
	\foreach \x in {0.5,1,1.5,2}
		\draw (\x,1pt) -- (\x,-1pt)	node[anchor=north] {\scriptsize \x};
	\foreach \x in {-2,-1.5,-1,-0.5}
		\draw (\x,1pt) -- (\x,-1pt)	node[anchor=north] {\scriptsize \x};
	\foreach \y in {1,...,5}
		\draw (1pt,\y) -- (-1pt,\y)	node[anchor=east] {\scriptsize \y}; 
	\foreach \y in {-1}
		\draw (1pt,\y) -- (-1pt,\y)	node[anchor=west] {\scriptsize \y}; 
\end{tikzpicture}
\end{center}
\end{example}
\end{document}
\documentclass[12pt]{book}
\raggedbottom
\usepackage[top=1in,left=1in,bottom=1in,right=1in,headsep=0.25in]{geometry}	
\usepackage{amssymb,amsmath,amsthm,amsfonts}
\usepackage{chapterfolder,docmute,setspace}
\usepackage{cancel,multicol,tikz,verbatim,framed,polynom,enumitem,tikzpagenodes}
\usepackage[colorlinks, hyperindex, plainpages=false, linkcolor=blue, urlcolor=blue, pdfpagelabels]{hyperref}
\usepackage[type={CC},modifier={by-sa},version={4.0},]{doclicense}

\theoremstyle{definition}
\newtheorem{example}{Example}
\newcommand{\Desmos}{\href{https://www.desmos.com/}{Desmos}}
\setlength{\parindent}{0in}
\setlist{itemsep=0in}
\setlength{\parskip}{0.1in}
\setcounter{secnumdepth}{0}
% This document is used for ordering of lessons.  If an instructor wishes to change the ordering of assessments, the following steps must be taken:

% 1) Reassign the appropriate numbers for each lesson in the \setcounter commands included in this file.
% 2) Rearrange the \include commands in the master file (the file with 'Course Pack' in the name) to accurately reflect the changes.  
% 3) Rarrange the \items in the measureable_outcomes file to accurately reflect the changes.  Be mindful of page breaks when moving items.
% 4) Re-build all affected files (master file, measureable_outcomes file, and any lessons whose numbering has changed).

%Note: The placement of each \newcounter and \setcounter command reflects the original/default ordering of topics (linears, systems, quadratics, functions, polynomials, rationals).

\newcounter{lesson_solving_linear_equations}
\newcounter{lesson_equations_containing_absolute_values}
\newcounter{lesson_graphing_lines}
\newcounter{lesson_two_forms_of_a_linear_equation}
\newcounter{lesson_parallel_and_perpendicular_lines}
\newcounter{lesson_linear_inequalities}
\newcounter{lesson_compound_inequalities}
\newcounter{lesson_inequalities_containing_absolute_values}
\newcounter{lesson_graphing_systems}
\newcounter{lesson_substitution}
\newcounter{lesson_elimination}
\newcounter{lesson_quadratics_introduction}
\newcounter{lesson_factoring_GCF}
\newcounter{lesson_factoring_grouping}
\newcounter{lesson_factoring_trinomials_a_is_1}
\newcounter{lesson_factoring_trinomials_a_neq_1}
\newcounter{lesson_solving_by_factoring}
\newcounter{lesson_square_roots}
\newcounter{lesson_i_and_complex_numbers}
\newcounter{lesson_vertex_form_and_graphing}
\newcounter{lesson_solve_by_square_roots}
\newcounter{lesson_extracting_square_roots}
\newcounter{lesson_the_discriminant}
\newcounter{lesson_the_quadratic_formula}
\newcounter{lesson_quadratic_inequalities}
\newcounter{lesson_functions_and_relations}
\newcounter{lesson_evaluating_functions}
\newcounter{lesson_finding_domain_and_range_graphically}
\newcounter{lesson_fundamental_functions}
\newcounter{lesson_finding_domain_algebraically}
\newcounter{lesson_solving_functions}
\newcounter{lesson_function_arithmetic}
\newcounter{lesson_composite_functions}
\newcounter{lesson_inverse_functions_definition_and_HLT}
\newcounter{lesson_finding_an_inverse_function}
\newcounter{lesson_transformations_translations}
\newcounter{lesson_transformations_reflections}
\newcounter{lesson_transformations_scalings}
\newcounter{lesson_transformations_summary}
\newcounter{lesson_piecewise_functions}
\newcounter{lesson_functions_containing_absolute_values}
\newcounter{lesson_absolute_as_piecewise}
\newcounter{lesson_polynomials_introduction}
\newcounter{lesson_sign_diagrams_polynomials}
\newcounter{lesson_factoring_quadratic_type}
\newcounter{lesson_factoring_summary}
\newcounter{lesson_polynomial_division}
\newcounter{lesson_synthetic_division}
\newcounter{lesson_end_behavior_polynomials}
\newcounter{lesson_local_behavior_polynomials}
\newcounter{lesson_rational_root_theorem}
\newcounter{lesson_polynomials_graphing_summary}
\newcounter{lesson_polynomial_inequalities}
\newcounter{lesson_rationals_introduction_and_terminology}
\newcounter{lesson_sign_diagrams_rationals}
\newcounter{lesson_horizontal_asymptotes}
\newcounter{lesson_slant_and_curvilinear_asymptotes}
\newcounter{lesson_vertical_asymptotes}
\newcounter{lesson_holes}
\newcounter{lesson_rationals_graphing_summary}

\setcounter{lesson_solving_linear_equations}{1}
\setcounter{lesson_equations_containing_absolute_values}{2}
\setcounter{lesson_graphing_lines}{3}
\setcounter{lesson_two_forms_of_a_linear_equation}{4}
\setcounter{lesson_parallel_and_perpendicular_lines}{5}
\setcounter{lesson_linear_inequalities}{6}
\setcounter{lesson_compound_inequalities}{7}
\setcounter{lesson_inequalities_containing_absolute_values}{8}
\setcounter{lesson_graphing_systems}{9}
\setcounter{lesson_substitution}{10}
\setcounter{lesson_elimination}{11}
\setcounter{lesson_quadratics_introduction}{16}
\setcounter{lesson_factoring_GCF}{17}
\setcounter{lesson_factoring_grouping}{18}
\setcounter{lesson_factoring_trinomials_a_is_1}{19}
\setcounter{lesson_factoring_trinomials_a_neq_1}{20}
\setcounter{lesson_solving_by_factoring}{21}
\setcounter{lesson_square_roots}{22}
\setcounter{lesson_i_and_complex_numbers}{23}
\setcounter{lesson_vertex_form_and_graphing}{24}
\setcounter{lesson_solve_by_square_roots}{25}
\setcounter{lesson_extracting_square_roots}{26}
\setcounter{lesson_the_discriminant}{27}
\setcounter{lesson_the_quadratic_formula}{28}
\setcounter{lesson_quadratic_inequalities}{29}
\setcounter{lesson_functions_and_relations}{12}
\setcounter{lesson_evaluating_functions}{13}
\setcounter{lesson_finding_domain_and_range_graphically}{14}
\setcounter{lesson_fundamental_functions}{15}
\setcounter{lesson_finding_domain_algebraically}{30}
\setcounter{lesson_solving_functions}{31}
\setcounter{lesson_function_arithmetic}{32}
\setcounter{lesson_composite_functions}{33}
\setcounter{lesson_inverse_functions_definition_and_HLT}{34}
\setcounter{lesson_finding_an_inverse_function}{35}
\setcounter{lesson_transformations_translations}{36}
\setcounter{lesson_transformations_reflections}{37}
\setcounter{lesson_transformations_scalings}{38}
\setcounter{lesson_transformations_summary}{39}
\setcounter{lesson_piecewise_functions}{40}
\setcounter{lesson_functions_containing_absolute_values}{41}
\setcounter{lesson_absolute_as_piecewise}{42}
\setcounter{lesson_polynomials_introduction}{43}
\setcounter{lesson_sign_diagrams_polynomials}{44}
\setcounter{lesson_factoring_quadratic_type}{46}
\setcounter{lesson_factoring_summary}{45}
\setcounter{lesson_polynomial_division}{47}
\setcounter{lesson_synthetic_division}{48}
\setcounter{lesson_end_behavior_polynomials}{49}
\setcounter{lesson_local_behavior_polynomials}{50}
\setcounter{lesson_rational_root_theorem}{51}
\setcounter{lesson_polynomials_graphing_summary}{52}
\setcounter{lesson_polynomial_inequalities}{53}
\setcounter{lesson_rationals_introduction_and_terminology}{54}
\setcounter{lesson_sign_diagrams_rationals}{55}
\setcounter{lesson_horizontal_asymptotes}{56}
\setcounter{lesson_slant_and_curvilinear_asymptotes}{57}
\setcounter{lesson_vertical_asymptotes}{58}
\setcounter{lesson_holes}{59}
\setcounter{lesson_rationals_graphing_summary}{60}

\newcommand{\tmmathbf}[1]{\ensuremath{\boldsymbol{#1}}}
\newcommand{\tmop}[1]{\ensuremath{\operatorname{#1}}}

\begin{document}
\section{Polynomial Inequalities (L\arabic{lesson_polynomial_inequalities})}
{\bf Objective: Solve a polynomial inequality by constructing a sign diagram.}\par
\begin{example}~~Solve the polynomial inequality
$$x^4+6x^2-15x\leq x^4+2x^3-7x^2.$$
Just as with quadratic inequalities, we begin by setting one side equal to zero.  This gives us $$2x^3-13x^2+15x\geq 0.$$
In order to construct a sign diagram, we must find a factorization and identify the roots of the left-hand side of our inequality.
$$2x^3-13x^2+15x=2x\left(x-\dfrac{3}{2}\right)(x-5)$$
So the dividers in our diagram will be the roots $x=0,\frac{3}{2},$ and $5$.  Below is a chart for testing the intervals in our sign diagram, as well as the end result.
\begin{center}
\begin{tabular}{cccc}
\underline{Interval} & \underline{Test Value} & \underline{Signs} & \underline{Result}\\
$(-\infty,0)$ & $x=-1$ & $(-)(-)(-)$ & $-$\\
$(0,\frac{3}{2})$ & $x=1$ & $(+)(-)(-)$ & $+$\\
$(\frac{3}{2},5)$ & $x=3$ & $(+)(+)(-)$ & $-$\\
$(5,\infty)$ &  $x=6$ & $(+)(+)(+)$ & $+$
\end{tabular}
\end{center}
\begin{center}
\begin{tikzpicture}[xscale=1,yscale=1]
	\draw [<->](-4.75,0) -- coordinate (x axis mid) (6.25,0) node[below right] {$x$};
	\draw [-](-1,1) -- coordinate (y axis mid) (-1,-0.25) node[below] {$0$};
	\draw [-](1,1) -- coordinate (y axis mid) (1,-0.25) node[below] {$\frac{3}{2}$};
	\draw [-](4,1) -- coordinate (y axis mid) (4,-0.25) node[below] {$5$};
	\draw (-3,-1) node {$x=-1$};
	\draw (0,-1) node {$x=1$};
	\draw (2.5,-1) node {$x=3$};
	\draw (5,-1) node {$x=6$};
	\draw (-3,0.5) node {$-$};
	\draw (0,0.5) node {$+$};
	\draw (2.5,0.5) node {$-$};
	\draw (5,0.5) node {$+$};
\end{tikzpicture}
\end{center}
So, using our diagram as an aide, we see that the solution to the inequality $$2x^3-13x^2+15x\geq 0,$$
as well as our original inequality
$$x^4+6x^2-15x\leq x^4+2x^3-7x^2,$$
will be $$\left[0,\frac{3}{2}\right]\cup[5,\infty).$$
\begin{multicols}{2}
\begin{center}
\begin{tikzpicture}[xscale=0.5,yscale=0.125]
	\draw [<->](-4.5,0) -- coordinate (x axis mid) (8.5,0) node[below right] {$x$};
	\draw [<->](0,-25) -- coordinate (y axis mid) (0,25) node[above right] {$y$};
	\draw [<->] plot [domain=-0.769:5.461, samples=100] (\x,{2*\x*(\x-1.5)*(\x-5)});
	\foreach \x in {1,...,8}
		\draw (\x,1pt) -- (\x,-1pt)	node[anchor=north] {\scriptsize \x};
	\foreach \x in {-4,...,-1}
		\draw (\x,1pt) -- (\x,-1pt)	node[anchor=south] {\scriptsize \x};
	\foreach \y in {5,10,...,20}
		\draw (1pt,\y) -- (-1pt,\y)	node[anchor=east] {\scriptsize \y}; 
	\foreach \y in {-20,-15,...,-5}
		\draw (1pt,\y) -- (-1pt,\y)	node[anchor=west] {\scriptsize \y}; 
\end{tikzpicture}
\end{center}
\columnbreak
\ \par
Since our given inequality was inclusive\\
($\leq$ or $\geq$), we include the corresponding endpoints in our answer.\\
\ \par
We can verify that our answer is correct by comparing it to the graph of the function $$f(x)=2x^3-13x^2+15x,$$ which lies above (or on) the $x-$axis over the intervals in our answer.
\end{multicols}
\end{example}
\newpage
\end{document}
\documentclass[12pt]{book}
\raggedbottom
\usepackage[top=1in,left=1in,bottom=1in,right=1in,headsep=0.25in]{geometry}	
\usepackage{amssymb,amsmath,amsthm,amsfonts}
\usepackage{chapterfolder,docmute,setspace}
\usepackage{cancel,multicol,tikz,verbatim,framed,polynom,enumitem,tikzpagenodes}
\usepackage[colorlinks, hyperindex, plainpages=false, linkcolor=blue, urlcolor=blue, pdfpagelabels]{hyperref}
\usepackage[type={CC},modifier={by-sa},version={4.0},]{doclicense}

\theoremstyle{definition}
\newtheorem{example}{Example}
\newcommand{\Desmos}{\href{https://www.desmos.com/}{Desmos}}
\setlength{\parindent}{0in}
\setlist{itemsep=0in}
\setlength{\parskip}{0.1in}
\setcounter{secnumdepth}{0}
% This document is used for ordering of lessons.  If an instructor wishes to change the ordering of assessments, the following steps must be taken:

% 1) Reassign the appropriate numbers for each lesson in the \setcounter commands included in this file.
% 2) Rearrange the \include commands in the master file (the file with 'Course Pack' in the name) to accurately reflect the changes.  
% 3) Rarrange the \items in the measureable_outcomes file to accurately reflect the changes.  Be mindful of page breaks when moving items.
% 4) Re-build all affected files (master file, measureable_outcomes file, and any lessons whose numbering has changed).

%Note: The placement of each \newcounter and \setcounter command reflects the original/default ordering of topics (linears, systems, quadratics, functions, polynomials, rationals).

\newcounter{lesson_solving_linear_equations}
\newcounter{lesson_equations_containing_absolute_values}
\newcounter{lesson_graphing_lines}
\newcounter{lesson_two_forms_of_a_linear_equation}
\newcounter{lesson_parallel_and_perpendicular_lines}
\newcounter{lesson_linear_inequalities}
\newcounter{lesson_compound_inequalities}
\newcounter{lesson_inequalities_containing_absolute_values}
\newcounter{lesson_graphing_systems}
\newcounter{lesson_substitution}
\newcounter{lesson_elimination}
\newcounter{lesson_quadratics_introduction}
\newcounter{lesson_factoring_GCF}
\newcounter{lesson_factoring_grouping}
\newcounter{lesson_factoring_trinomials_a_is_1}
\newcounter{lesson_factoring_trinomials_a_neq_1}
\newcounter{lesson_solving_by_factoring}
\newcounter{lesson_square_roots}
\newcounter{lesson_i_and_complex_numbers}
\newcounter{lesson_vertex_form_and_graphing}
\newcounter{lesson_solve_by_square_roots}
\newcounter{lesson_extracting_square_roots}
\newcounter{lesson_the_discriminant}
\newcounter{lesson_the_quadratic_formula}
\newcounter{lesson_quadratic_inequalities}
\newcounter{lesson_functions_and_relations}
\newcounter{lesson_evaluating_functions}
\newcounter{lesson_finding_domain_and_range_graphically}
\newcounter{lesson_fundamental_functions}
\newcounter{lesson_finding_domain_algebraically}
\newcounter{lesson_solving_functions}
\newcounter{lesson_function_arithmetic}
\newcounter{lesson_composite_functions}
\newcounter{lesson_inverse_functions_definition_and_HLT}
\newcounter{lesson_finding_an_inverse_function}
\newcounter{lesson_transformations_translations}
\newcounter{lesson_transformations_reflections}
\newcounter{lesson_transformations_scalings}
\newcounter{lesson_transformations_summary}
\newcounter{lesson_piecewise_functions}
\newcounter{lesson_functions_containing_absolute_values}
\newcounter{lesson_absolute_as_piecewise}
\newcounter{lesson_polynomials_introduction}
\newcounter{lesson_sign_diagrams_polynomials}
\newcounter{lesson_factoring_quadratic_type}
\newcounter{lesson_factoring_summary}
\newcounter{lesson_polynomial_division}
\newcounter{lesson_synthetic_division}
\newcounter{lesson_end_behavior_polynomials}
\newcounter{lesson_local_behavior_polynomials}
\newcounter{lesson_rational_root_theorem}
\newcounter{lesson_polynomials_graphing_summary}
\newcounter{lesson_polynomial_inequalities}
\newcounter{lesson_rationals_introduction_and_terminology}
\newcounter{lesson_sign_diagrams_rationals}
\newcounter{lesson_horizontal_asymptotes}
\newcounter{lesson_slant_and_curvilinear_asymptotes}
\newcounter{lesson_vertical_asymptotes}
\newcounter{lesson_holes}
\newcounter{lesson_rationals_graphing_summary}

\setcounter{lesson_solving_linear_equations}{1}
\setcounter{lesson_equations_containing_absolute_values}{2}
\setcounter{lesson_graphing_lines}{3}
\setcounter{lesson_two_forms_of_a_linear_equation}{4}
\setcounter{lesson_parallel_and_perpendicular_lines}{5}
\setcounter{lesson_linear_inequalities}{6}
\setcounter{lesson_compound_inequalities}{7}
\setcounter{lesson_inequalities_containing_absolute_values}{8}
\setcounter{lesson_graphing_systems}{9}
\setcounter{lesson_substitution}{10}
\setcounter{lesson_elimination}{11}
\setcounter{lesson_quadratics_introduction}{16}
\setcounter{lesson_factoring_GCF}{17}
\setcounter{lesson_factoring_grouping}{18}
\setcounter{lesson_factoring_trinomials_a_is_1}{19}
\setcounter{lesson_factoring_trinomials_a_neq_1}{20}
\setcounter{lesson_solving_by_factoring}{21}
\setcounter{lesson_square_roots}{22}
\setcounter{lesson_i_and_complex_numbers}{23}
\setcounter{lesson_vertex_form_and_graphing}{24}
\setcounter{lesson_solve_by_square_roots}{25}
\setcounter{lesson_extracting_square_roots}{26}
\setcounter{lesson_the_discriminant}{27}
\setcounter{lesson_the_quadratic_formula}{28}
\setcounter{lesson_quadratic_inequalities}{29}
\setcounter{lesson_functions_and_relations}{12}
\setcounter{lesson_evaluating_functions}{13}
\setcounter{lesson_finding_domain_and_range_graphically}{14}
\setcounter{lesson_fundamental_functions}{15}
\setcounter{lesson_finding_domain_algebraically}{30}
\setcounter{lesson_solving_functions}{31}
\setcounter{lesson_function_arithmetic}{32}
\setcounter{lesson_composite_functions}{33}
\setcounter{lesson_inverse_functions_definition_and_HLT}{34}
\setcounter{lesson_finding_an_inverse_function}{35}
\setcounter{lesson_transformations_translations}{36}
\setcounter{lesson_transformations_reflections}{37}
\setcounter{lesson_transformations_scalings}{38}
\setcounter{lesson_transformations_summary}{39}
\setcounter{lesson_piecewise_functions}{40}
\setcounter{lesson_functions_containing_absolute_values}{41}
\setcounter{lesson_absolute_as_piecewise}{42}
\setcounter{lesson_polynomials_introduction}{43}
\setcounter{lesson_sign_diagrams_polynomials}{44}
\setcounter{lesson_factoring_quadratic_type}{46}
\setcounter{lesson_factoring_summary}{45}
\setcounter{lesson_polynomial_division}{47}
\setcounter{lesson_synthetic_division}{48}
\setcounter{lesson_end_behavior_polynomials}{49}
\setcounter{lesson_local_behavior_polynomials}{50}
\setcounter{lesson_rational_root_theorem}{51}
\setcounter{lesson_polynomials_graphing_summary}{52}
\setcounter{lesson_polynomial_inequalities}{53}
\setcounter{lesson_rationals_introduction_and_terminology}{54}
\setcounter{lesson_sign_diagrams_rationals}{55}
\setcounter{lesson_horizontal_asymptotes}{56}
\setcounter{lesson_slant_and_curvilinear_asymptotes}{57}
\setcounter{lesson_vertical_asymptotes}{58}
\setcounter{lesson_holes}{59}
\setcounter{lesson_rationals_graphing_summary}{60}

\newcommand{\tmmathbf}[1]{\ensuremath{\boldsymbol{#1}}}
\newcommand{\tmop}[1]{\ensuremath{\operatorname{#1}}}

\begin{document}
\section{Practice Problems}
\subsection*{Introduction and Terminology}
Identify the degree, set of coefficients, leading coefficient, leading term and constant term for each of the polynomials listed.  Classify each polynomial by both degree and number of nonzero terms.  If it is not already provided, write the polynomial in descending-power order.
\begin{multicols}{2}
\begin{enumerate}
\item $f(x)=-2x^3-1$
\item $f(x)=-2x^4 + 4x+1$
\item $f(x)=40-x^3$
\item $f(x)=(x-1)^2$
\item $f(x)=32x^5+x^2+x$
\item $f(x)=4x^2-3x^4$
\item $f(x)=-2x^4-4x^2-6x-8$
\item $f(x)=5x+3x^2+x^3+\sqrt{3}$
\item $f(x)=\frac{1}{2}x^4-5x^2-\frac{1}{2}$
\item $f(x)=12-6x+3x^2-2x^3-x^6$
\item $f(x)=-3x^4-12x^3+x-13$
\end{enumerate}
\end{multicols}
\subsection*{Sign Diagrams}
Construct a sign diagram for the factored polynomial functions below. Use \Desmos \ to graph each function and check the accuracy of your diagram.  Identify the interval(s) where the function is positive and where it is negative.
\begin{multicols}{2}
\begin{enumerate}
\item $f(x)=x^3(x-2)(x+2)$
\item $g(x)=(x^2+1)(1-x)$
\item $h(x)=x(x-3)^2(x+3)$
\item $k(x)=(3x-4)^3$
\item $\ell(x)=(x^2+2)(x^2+3)$
\item $m(x)=-2(x+7)^2(1-2x)^2$
\item $f(x)=(x^2-1)(x+4)$
\item $g(x)=(x^2-1)(x^2-16)$
\item $h(x)=-2x^3(3x-1)(2-x)$
\item $k(x)=(x^2-4x+1)(x+2)^2$
\end{enumerate}
\end{multicols}
\subsection*{Factoring}
\subsubsection{Some Special Cases}
Completely factor each of the following polynomial expressions.
\begin{multicols}{2}
\begin{enumerate}
  \item $2 x^2 - 11 x + 15$
  \item $5 n^3 + 7 n^2 - 6 n$
  \item $54 u^3 - 16$
  \item $54 - 128 x^3$
  \item $n^2 - n$
  \item $2x^4 -21x^2-11$
  \item $24 a z - 18 a h + 60 y z - 45 y h$
  \item $5 u^2 - 9 u v + 4 v^2$
  \item $16 x^2 + 48 x y + 36 y^2$
  \item $- 2 x^3 + 128 y^3$
  \item $20 u v - 60 u^3 - 5 x v + 15 x u^2$
  \item $2 x^3 + 5 x^2 y + 3 y^2 x$
  \end{enumerate}
\end{multicols}
\subsubsection{Quadratic Type}
Completely factor each of the following polynomials over the real numbers and identify the set of all real roots.
\begin{multicols}{2}
\begin{enumerate}
  \item  $x^4 +13x^2+40$
  \item  $x^4-5x^2+4$
  \item  $x^4 -17x^2+16$
  \item  $x^4 -3x^2-40$
  \item  $3x^4 -32x^2+45$
  \item  $x^4 +x^2-12$
  \item  $x^4 -3x^2-10$
  \item  $x^6 -82x^3+81$
  \item  $8x^4 +2x^2-3$
  \item  $2x^4 -19x^2+9$
\end{enumerate}
\end{multicols}
\subsection*{Division}
\subsubsection{Polynomial (Long) Division}
Use polynomial long division to divide and simplify each of the given expressions.  Express each answer in the form below.
$$\frac{\text{dividend}}{\text{divisor}} \ = \ \text{quotient} \ + \ \frac{\text{remainder}}{\text{divisor}}$$
\begin{multicols}{3}
\begin{enumerate}
  \item $\dfrac{20 x^4 + x^3 + 2 x^2}{4 x^3}$
  \item $\dfrac{5 x^4 + 45 x^3 + 4 x^2}{9 x}$
  \item $\dfrac{20 x^4 + x^3 + 40 x^2}{10 x}$
  \item $\dfrac{3 x^3 + 4 x^2 + 2 x}{8 x}$
  \item $\dfrac{12 x^4 + 24 x^3 + 3 x^2}{6 x}$
  \item $\dfrac{5 x^4 + 16 x^3 + 16 x^2}{4 x}$
  \item $\dfrac{10 x^4 + 50 x^3 + 2 x^2}{10 x^2}$
  \item $\dfrac{3 x^4 + 18 x^3 + 27 x^2}{9 x^2}$
  \item $\dfrac{x^2 - 2 x - 71}{x + 8}$\label{polydiv_one}
  \item $\dfrac{x^2 - 3 x - 53}{x - 9}$
  \item $\dfrac{x^2 + 13 x + 32}{x + 5}$
  \item $\dfrac{x^2 - 10 x + 16}{x - 7}$
  \item $\dfrac{x^2 - 2 x - 89}{x - 10}$
  \item $\dfrac{x^2 + 4 x - 26}{x + 7}$
  \item $\dfrac{x^2 - 4 x - 38}{x - 8}$
  \item $\dfrac{x^2 - 4}{x - 2}$
  \item $\dfrac{x^3 + 15 x^2 + 49 x - 55}{x + 7}$
  \item $\dfrac{x^3 - 26 x - 41}{x + 4}$
  \item $\dfrac{3 x^3 + 9 x^2 - 64 x - 68}{x + 6}$
  \item $\dfrac{9 x^3 + 45 x^2 + 27 x - 5}{9 x + 9}$
  \item $\dfrac{x^3 - x^2 - 16 x + 8}{x - 4}$
  \item $\dfrac{x^2 - 10 x + 22}{x - 4}$
  \item $\dfrac{x^3 - 16 x^2 + 71 x - 56}{x - 8}$
  \item $\dfrac{x^3 - 4 x^2 - 6 x + 4}{x - 1}$
  \item $\dfrac{8 x^3 - 66 x^2 + 12 x + 37}{x - 8}$
  \item $\dfrac{3 x^2 + 9 x - 9}{3 x - 3}$
  \item $\dfrac{2 x^2 - 5 x - 8}{2 x + 3}$
  \item $\dfrac{3 x^2 - 32}{3 x - 9}$
  \item $\dfrac{4 x^2 - 23 x - 38}{4 x + 5}$
  \item $\dfrac{2 x^3 + 21 x^2 + 25 x}{2 x + 3}$
  \item $\dfrac{4 x^3 - 21 x^2 + 6 x + 19}{4 x + 3}$  
  \item $\dfrac{8 x^3 - 57 x^2 + 42}{8 x + 7}$
  \item $\dfrac{2 x^3 + 12 x^2 + 4 x - 37}{2 x + 6}$
  \item $\dfrac{45 x^2 + 56 x + 19}{9 x + 4}$
  \item $\dfrac{10 x^2 - 32 x + 9}{10 x - 2}$
  \item $\dfrac{4 x^2 - x - 1}{4 x + 3}$
  \item $\dfrac{27 x^2 + 87 x + 35}{3x + 8}$
  \item $\dfrac{4 x^2 - 33 x + 28}{4 x - 5}$
  \item $\dfrac{48 x^2 - 70 x + 16}{6 x - 2}$
  \item $\dfrac{12 x^3 + 12 x^2 - 15 x - 4}{2 x + 3}$
  \item $\dfrac{24 x^3 - 38 x^2 + 29 x - 60}{4 x - 7}$\label{polydiv_two}
	\end{enumerate}
\end{multicols}
\subsubsection{Synthetic Division}
Use synthetic division to divide and simplify each of the given expressions.  Express each answer in the form below.
$$\frac{\text{dividend}}{\text{divisor}} \ = \ \text{quotient} \ + \ \frac{\text{remainder}}{\text{divisor}}$$
\begin{multicols}{2}
\begin{enumerate}
  \item $\dfrac{x^4-4x^3+2x^2-x+1 }{x+2}$
  \item $\dfrac{x^4-2x^3+7x^2-6x+3 }{x-2}$
  \item $\dfrac{2x^4-2x^3-10x^2+1 }{x+2}$
  \item $\dfrac{5x^4-2x^3+4x^2-5x }{x-1}$
  \item $\dfrac{-x^4-x^3+x^2+x+1 }{x+5}$
  \item $\dfrac{x^4-3x^3+2x^2-x+1 }{x-4}$
  \item $\dfrac{12x^4-x^3+x^2-3x+1 }{x+2}$
  \item $\dfrac{3x^4+3x^3+13x^2-4x+14 }{x+1}$
  \item $\dfrac{1x^4-3x^3+5x^2-14x+2 }{x-2}$
  \item $\dfrac{2x^4-2x+1 }{x+3}$
  \item $\dfrac{x^4-3x-4 }{x-3}$
  \item $\dfrac{x^4-4x^3+13x^2-5x+7 }{x-4}$
\end{enumerate}
\end{multicols}
Use synthetic division to divide and simplify each of the given expression from Exercises \ref{polydiv_one}-\ref{polydiv_two}.
\subsection*{End Behavior}
Determine the end behavior of each of the following functions.  Write your answers as mathematical sentences.  Graph each function on \Desmos \ to check your answers.
\begin{enumerate}
\begin{multicols}{2}
\item $f(x)=-2x^3 + 4x+1$
\item $g(x)=32x^5+x^2+15$
\item $h(x)=-3x^4+4x^2$
\item $k(x)=15x^4-32x^2-x-14$
\item $\ell(x)=x^5+40$
\item $m(x)=5x^5+3x^2+x+14$
\item $n(x)=123x^4-7x^3-5x^2-3x+1$
\item $p(x)=x^3-1$
\item $q(x)=-23x^6+x^3+x^2+x+1$
\end{multicols}
\end{enumerate}
Identify the degree, leading coefficient, and constant term of each polynomial function below.  Use the degree and leading coefficient to identify the end behavior of the graph of each function.  Write your answers as mathematical sentences.  Graph each function on \Desmos \ to check your answers.
\begin{enumerate}[resume]
\begin{multicols}{2}
    \item $f(x)=x^3(x-2)(x+2)$
	\item $g(x)=(x^2+1)(1-x)$
	\item $h(x)=x(x-3)^2(x+3)$
	\item $k(x)=(3x-4)^3$
    \item $\ell(x)=(x^2+2)(x^2+3)$
    \item $m(x)=-2(x+7)^2(1-2x)^2$
    \item $f(x)=(x^2-1)(x+4)$
	\item $g(x)=(x^2-1)(x^2-16)$
	\item $h(x)=-2x^3(3x-1)(2-x)$
	\item $k(x)=(x^2-4x+1)(x+2)^2$
\end{multicols}
\end{enumerate}
\subsection*{Local Behavior}
Determine the set of roots and corresponding multiplicities for the following functions.  In each case, classify the corresponding $x-$intercept as either a turnaround or crossover point.  Use \Desmos \ to check your answers.
\begin{multicols}{2}
\begin{enumerate}
\item $f(x)=x^3(x-2)(x+2)$
\item $g(x)=(x^2+1)(1-x)$
\item $h(x)=x(x-3)^2(x+3)$
\item $k(x)=(3x-4)^3$
\item $\ell(x)=(x^2+2)(x^2+3)$
\item $m(x)=-2(x+7)^2(1-2x)^2$
\item $f(x)=(x^2-1)(x+4)$
\item $g(x)=(x^2-1)(x^2-16)$
\item $h(x)=-2x^3(3x-1)(2-x)$
\item $k(x)=(x^2-4x+1)(x+2)^2$
\item $f(x)=\frac{1}{2}(x-2)^2(x+5)(x-3)$
\item $g(x)=(x+2)^2(3x-1)(5-x)$
\end{enumerate}
\end{multicols}
\subsection*{The Rational Root Theorem}
Use the Rational Root Theorem to identify a set of possible rational roots for each of the polynomial functions below.  Evaluate the function at $x=1$.  If $x=1$ is a real root, divide the polynomial by $x-1$ and factor the resulting quotient.  If $x=1$ is not a real root, evaluate the function at at least one of your remaining possible roots, in order to determine if they are actual roots of the polynomial.  If successful, divide your polynomial by the respective factor and factor the remaining quotient.  Use \Desmos \ to help determine the actual set of real roots.
\begin{enumerate}
	\item $f(x) = x^{3} - 2x^{2} - 5x + 6$
	\item $f(x) = 2x^4+x^3-7x^2-3x+3$
	\item $f(x) = x^5-x^4-37x^3+37x^2+36x-36$
	\item $f(x) = 3x^{3} + 3x^{2} - 11x - 10$
\end{enumerate}
Use the Rational Root Theorem to identify a set of possible rational roots for each of the polynomial functions below.  Evaluate the function at at least two of your possible roots, in order to determine if they are actual roots of the polynomial.  If successful, divide your polynomial by the respective factor.  Use \Desmos \ to help determine the actual set of real roots.
\begin{enumerate}[resume]
	\item $f(x) = x^{4} -2x^3+ 5x^{2} - 8x + 4$
	\item $f(x) = x^{3} + 4x^{2} - 11x + 6$
	\item $f(x) = -2x^{3} + 19x^{2} - 49x + 20$
	\item $f(x) = 36x^{4} - 12x^{3} - 11x^{2} + 2x + 1$
	\item $f(x) = x^{4} - 9x^{2} - 4x + 12$
	\item $f(x) = x^{4} + 2x^{3} - 12x^{2} - 40x - 32$
	\item $f(x) = 6x^3+19x^2-6x-40$
\end{enumerate}
\subsection*{Graphing Summary}
Factor each polynomial below, and sketch a complete graph of the function, making sure to have a clearly defined scale and label any intercepts.  Use \Desmos \ to compare your results.
\begin{enumerate}
\begin{multicols}{2}
	\item $f(x) = -17x^{3} + 5x^{2} + 34x - 10$
	\item $f(x) = x^4-9x^2+14$
	\item $f(x) = 3x^4-14x^2-5$
	\item $f(x) = 2x^4-7x^2+6$
	\item $f(x) = x^5-2x^4-x+2$
	\item $f(x) = 2x^5+3x^4-32x-48$
	\item $f(x) = x^6-6x^3-16$
	\item $f(x) = 2x^6-7x^3+5$
	\item $f(x) = -x^{3} + 7x^{2} - x + 7$
	\item $f(x) = 3x^4 - 5x^3 - 12x^2$
	\item $f(x) = 2x^3 - 5x^2 - x$
	\item $f(x) = -x^4-2x^2 +15$
\end{multicols}
\end{enumerate}
Get a complete factorization of each polynomial below by first dividing the function by $x-1$.  Then sketch a graph of the function, making sure to have a clearly defined scale and label any intercepts.  Use \Desmos \ to compare your results.
\begin{enumerate}[resume]
	\item $f(x) = x^{3} - 2x^{2} - 5x + 6$
	\item $f(x) = x^{3} + 4x^{2} - 11x + 6$
	\item $f(x)=x^5-x^4-37x^3+37x^2+36x-36$
	\item $f(x) = x^{4} -2x^3+ 5x^{2} - 8x + 4$
\end{enumerate}
Use the Rational Root Theorem and polynomial division to get a complete factorization of each polynomial function below.  Then sketch a graph of the function, making sure to have a clearly defined scale and label any intercepts.  Use \Desmos \ to compare your results.
\begin{enumerate}[resume]
	\item $f(x) = x^{4} - 9x^{2} - 4x + 12$
	\item $f(x) = x^{4} + 2x^{3} - 12x^{2} - 40x - 32$
	\item $f(x) = 2x^4+x^3-7x^2-3x+3$
	\item $f(x) = 3x^{3} + 3x^{2} - 11x - 10$
	\item $f(x) = 6x^3+19x^2-34x-40$
	\item $f(x) = -2x^{3} + 19x^{2} - 49x + 20$
	\item $f(x) = 36x^{4} - 12x^{3} - 11x^{2} + 2x + 1$
	\item $f(x) = x^4+4x^3-x-4$	
	\item $f(x) = 2x^3-5x^2-52x+60$
	\item $f(x) = -x^3-x^2+39x+45$
	\item $f(x) = -2x^4+7x^3+17x^2-28x-36$
	\item $f(x) = x^7-5x^6-24x^5+120x^4-25x^3+125x^2$
\end{enumerate}
\subsection*{Polynomial Inequalities}
Solve each polynomial inequality below, expressing your answers using interval notation.  Use \Desmos \ to help confirm that each answer is correct.
\begin{multicols}{2}
\begin{enumerate}
 \item $x^4 + x^2 \geq 6$
 \item $x^{4} - 9x^{2} \leq 4x - 12$
 \item $4x^3 \geq 3x+1$
 \item $x^4 \leq 16+4x-x^3$
 \item $3x^2 + 2x < x^4$
 \item $\dfrac{x^3+2 x^2}{2} < x+2$
 \item $\dfrac{x^3+20x}{8} \geq x^2 + 2$
 \item $19x^{2} + 20 > 2x^{3} + 49x $
 \item $x^3<4x^2$
 \item $x^3-7x^2\leq 12x-84$
 \item $(x - 1)^{2} \geq 4$
 \item $2x^4>5x^2+3$
\end{enumerate}
\end{multicols}
\end{document}
\documentclass[12pt]{book}
\raggedbottom
\usepackage[top=1in,left=1in,bottom=1in,right=1in,headsep=0.25in]{geometry}	
\usepackage{amssymb,amsmath,amsthm,amsfonts}
\usepackage{chapterfolder,docmute,setspace}
\usepackage{cancel,multicol,tikz,verbatim,framed,polynom,enumitem,tikzpagenodes}
\usepackage[colorlinks, hyperindex, plainpages=false, linkcolor=blue, urlcolor=blue, pdfpagelabels]{hyperref}
\usepackage[type={CC},modifier={by-sa},version={4.0},]{doclicense}

\theoremstyle{definition}
\newtheorem{example}{Example}
\newcommand{\Desmos}{\href{https://www.desmos.com/}{Desmos}}
\setlength{\parindent}{0in}
\setlist{itemsep=0in}
\setlength{\parskip}{0.1in}
\setcounter{secnumdepth}{0}
% This document is used for ordering of lessons.  If an instructor wishes to change the ordering of assessments, the following steps must be taken:

% 1) Reassign the appropriate numbers for each lesson in the \setcounter commands included in this file.
% 2) Rearrange the \include commands in the master file (the file with 'Course Pack' in the name) to accurately reflect the changes.  
% 3) Rarrange the \items in the measureable_outcomes file to accurately reflect the changes.  Be mindful of page breaks when moving items.
% 4) Re-build all affected files (master file, measureable_outcomes file, and any lessons whose numbering has changed).

%Note: The placement of each \newcounter and \setcounter command reflects the original/default ordering of topics (linears, systems, quadratics, functions, polynomials, rationals).

\newcounter{lesson_solving_linear_equations}
\newcounter{lesson_equations_containing_absolute_values}
\newcounter{lesson_graphing_lines}
\newcounter{lesson_two_forms_of_a_linear_equation}
\newcounter{lesson_parallel_and_perpendicular_lines}
\newcounter{lesson_linear_inequalities}
\newcounter{lesson_compound_inequalities}
\newcounter{lesson_inequalities_containing_absolute_values}
\newcounter{lesson_graphing_systems}
\newcounter{lesson_substitution}
\newcounter{lesson_elimination}
\newcounter{lesson_quadratics_introduction}
\newcounter{lesson_factoring_GCF}
\newcounter{lesson_factoring_grouping}
\newcounter{lesson_factoring_trinomials_a_is_1}
\newcounter{lesson_factoring_trinomials_a_neq_1}
\newcounter{lesson_solving_by_factoring}
\newcounter{lesson_square_roots}
\newcounter{lesson_i_and_complex_numbers}
\newcounter{lesson_vertex_form_and_graphing}
\newcounter{lesson_solve_by_square_roots}
\newcounter{lesson_extracting_square_roots}
\newcounter{lesson_the_discriminant}
\newcounter{lesson_the_quadratic_formula}
\newcounter{lesson_quadratic_inequalities}
\newcounter{lesson_functions_and_relations}
\newcounter{lesson_evaluating_functions}
\newcounter{lesson_finding_domain_and_range_graphically}
\newcounter{lesson_fundamental_functions}
\newcounter{lesson_finding_domain_algebraically}
\newcounter{lesson_solving_functions}
\newcounter{lesson_function_arithmetic}
\newcounter{lesson_composite_functions}
\newcounter{lesson_inverse_functions_definition_and_HLT}
\newcounter{lesson_finding_an_inverse_function}
\newcounter{lesson_transformations_translations}
\newcounter{lesson_transformations_reflections}
\newcounter{lesson_transformations_scalings}
\newcounter{lesson_transformations_summary}
\newcounter{lesson_piecewise_functions}
\newcounter{lesson_functions_containing_absolute_values}
\newcounter{lesson_absolute_as_piecewise}
\newcounter{lesson_polynomials_introduction}
\newcounter{lesson_sign_diagrams_polynomials}
\newcounter{lesson_factoring_quadratic_type}
\newcounter{lesson_factoring_summary}
\newcounter{lesson_polynomial_division}
\newcounter{lesson_synthetic_division}
\newcounter{lesson_end_behavior_polynomials}
\newcounter{lesson_local_behavior_polynomials}
\newcounter{lesson_rational_root_theorem}
\newcounter{lesson_polynomials_graphing_summary}
\newcounter{lesson_polynomial_inequalities}
\newcounter{lesson_rationals_introduction_and_terminology}
\newcounter{lesson_sign_diagrams_rationals}
\newcounter{lesson_horizontal_asymptotes}
\newcounter{lesson_slant_and_curvilinear_asymptotes}
\newcounter{lesson_vertical_asymptotes}
\newcounter{lesson_holes}
\newcounter{lesson_rationals_graphing_summary}

\setcounter{lesson_solving_linear_equations}{1}
\setcounter{lesson_equations_containing_absolute_values}{2}
\setcounter{lesson_graphing_lines}{3}
\setcounter{lesson_two_forms_of_a_linear_equation}{4}
\setcounter{lesson_parallel_and_perpendicular_lines}{5}
\setcounter{lesson_linear_inequalities}{6}
\setcounter{lesson_compound_inequalities}{7}
\setcounter{lesson_inequalities_containing_absolute_values}{8}
\setcounter{lesson_graphing_systems}{9}
\setcounter{lesson_substitution}{10}
\setcounter{lesson_elimination}{11}
\setcounter{lesson_quadratics_introduction}{16}
\setcounter{lesson_factoring_GCF}{17}
\setcounter{lesson_factoring_grouping}{18}
\setcounter{lesson_factoring_trinomials_a_is_1}{19}
\setcounter{lesson_factoring_trinomials_a_neq_1}{20}
\setcounter{lesson_solving_by_factoring}{21}
\setcounter{lesson_square_roots}{22}
\setcounter{lesson_i_and_complex_numbers}{23}
\setcounter{lesson_vertex_form_and_graphing}{24}
\setcounter{lesson_solve_by_square_roots}{25}
\setcounter{lesson_extracting_square_roots}{26}
\setcounter{lesson_the_discriminant}{27}
\setcounter{lesson_the_quadratic_formula}{28}
\setcounter{lesson_quadratic_inequalities}{29}
\setcounter{lesson_functions_and_relations}{12}
\setcounter{lesson_evaluating_functions}{13}
\setcounter{lesson_finding_domain_and_range_graphically}{14}
\setcounter{lesson_fundamental_functions}{15}
\setcounter{lesson_finding_domain_algebraically}{30}
\setcounter{lesson_solving_functions}{31}
\setcounter{lesson_function_arithmetic}{32}
\setcounter{lesson_composite_functions}{33}
\setcounter{lesson_inverse_functions_definition_and_HLT}{34}
\setcounter{lesson_finding_an_inverse_function}{35}
\setcounter{lesson_transformations_translations}{36}
\setcounter{lesson_transformations_reflections}{37}
\setcounter{lesson_transformations_scalings}{38}
\setcounter{lesson_transformations_summary}{39}
\setcounter{lesson_piecewise_functions}{40}
\setcounter{lesson_functions_containing_absolute_values}{41}
\setcounter{lesson_absolute_as_piecewise}{42}
\setcounter{lesson_polynomials_introduction}{43}
\setcounter{lesson_sign_diagrams_polynomials}{44}
\setcounter{lesson_factoring_quadratic_type}{46}
\setcounter{lesson_factoring_summary}{45}
\setcounter{lesson_polynomial_division}{47}
\setcounter{lesson_synthetic_division}{48}
\setcounter{lesson_end_behavior_polynomials}{49}
\setcounter{lesson_local_behavior_polynomials}{50}
\setcounter{lesson_rational_root_theorem}{51}
\setcounter{lesson_polynomials_graphing_summary}{52}
\setcounter{lesson_polynomial_inequalities}{53}
\setcounter{lesson_rationals_introduction_and_terminology}{54}
\setcounter{lesson_sign_diagrams_rationals}{55}
\setcounter{lesson_horizontal_asymptotes}{56}
\setcounter{lesson_slant_and_curvilinear_asymptotes}{57}
\setcounter{lesson_vertical_asymptotes}{58}
\setcounter{lesson_holes}{59}
\setcounter{lesson_rationals_graphing_summary}{60}

\newcommand{\tmmathbf}[1]{\ensuremath{\boldsymbol{#1}}}
\newcommand{\tmop}[1]{\ensuremath{\operatorname{#1}}}

\newlist{oddenumerate}{enumerate}{1}
\setlist[oddenumerate]{start=0,label=\theoddenumeratei.}
\makeatletter
\renewcommand\theoddenumeratei{\@arabic{\numexpr2*\value{oddenumeratei}+1}}
\makeatother

\newcount\gpten % (global) power-of-ten -- tells which digit we are doing
\countdef\rtot2 % running total -- remainder so far
\countdef\LDscratch4 % scratch

\def\longdiv#1#2{%
 \vtop{\normalbaselines \offinterlineskip
   \setbox\strutbox\hbox{\vrule height 2.1ex depth .5ex width0ex}%
   \def\showdig{$\underline{\the\LDscratch\strut}$\cr\the\rtot\strut\cr
       \noalign{\kern-.2ex}}%
   \global\rtot=#1\relax
   \count0=\rtot\divide\count0by#2\edef\quotient{\the\count0}%\show\quotient
   % make list macro out of digits in quotient:
   \def\temp##1{\ifx##1\temp\else \noexpand\dodig ##1\expandafter\temp\fi}%
   \edef\routine{\expandafter\temp\quotient\temp}%
   % process list to give power-of-ten:
   \def\dodig##1{\global\multiply\gpten by10 }\global\gpten=1 \routine
   % to display effect of one digit in quotient (zero ignored):
   \def\dodig##1{\global\divide\gpten by10
      \LDscratch =\gpten
      \multiply\LDscratch  by##1%
      \multiply\LDscratch  by#2%
      \global\advance\rtot-\LDscratch \relax
      \ifnum\LDscratch>0 \showdig \fi % must hide \cr in a macro to skip it
   }%
   \tabskip=0pt
   \halign{\hfil##\cr % \halign for entire division problem
     $\quotient$\strut\cr
     #2$\,\overline{\vphantom{\big)}%
     \hbox{\smash{\raise3.5\fontdimen8\textfont3\hbox{$\big)$}}}%
     \mkern2mu \the\rtot}$\cr\noalign{\kern-.2ex}
     \routine \cr % do each digit in quotient
}}}

\begin{document}
\section{Selected Answers}
\subsection*{Introduction and Terminology}
%Identify the degree, set of coefficients, leading coefficient, leading term and constant term for each of the polynomials listed.  Classify each polynomial by both degree and number of nonzero terms.  If it is not already provided, write the polynomial in descending-power order.
%\begin{multicols}{2}
\begin{oddenumerate}
\item %$f(x)=-2x^3-1$
$n=3, \ \ a_n=-2, \ \  a_nx^n=-2x^3, \ \  a_0=-1, \ \  \{-2,0,0,-1\}$
%%\item $f(x)=-2x^4 + 4x+1$
\item %$f(x)=40-x^3$
$n=3, \ \  a_n=-1, \ \  a_nx^n=-1x^3, \ \  a_0=40, \ \  \{-1,0,0,40\}$
%%\item $f(x)=(x-1)^2$
\item %$f(x)=32x^5+x^2+x$
$n=5, \ \  a_n=32, \ \  a_nx^n=32x^5, \ \  a_0=0, \ \  \{32,0,0,1,1,0\}$
%%\item $f(x)=4x^2-3x^4$
\item %$f(x)=-2x^4-4x^2-6x-8$
$n=4, \ \  a_n=-2, \ \  a_nx^n=-2x^4, \ \  a_0=-8, \ \  \{-2,0,-4,-6,-8\}$
%%\item $f(x)=5x+3x^2+x^3+\sqrt{3}$
\item %$f(x)=\frac{1}{2}x^4-5x^2-\frac{1}{2}$
$n=4, \ \  a_n=\frac{1}{2}, \ \  a_nx^n=\frac{1}{2}x^4, \ \  a_0=-\frac{1}{2}, \ \  \{\frac{1}{2},0,-5,0,-\frac{1}{2}\}$
%%\item $f(x)=12-6x+3x^2-2x^3-x^6$
\item %$f(x)=-3x^4-12x^3+x-13$
$n=4, \ \  a_n=-3, \ \  a_nx^n=-3x^4, \ \  a_0=-13, \ \  \{-3,-12,0,1,-13\}$
\end{oddenumerate}
%\end{multicols}
\subsection*{Sign Diagrams}
%Construct a sign diagram for the factored polynomial functions below. Use \Desmos \ to graph each function and check the accuracy of your diagram.  Identify the interval(s) where the function is positive and where it is negative.
\begin{multicols}{2}
\begin{oddenumerate}
\item \ \\ %$f(x)=x^3(x-2)(x+2)$ 
\begin{tikzpicture}[xscale=0.5,yscale=0.5]
	\draw [<->](-5,0) -- coordinate (x axis mid) (5,0) node[below right] {$x$};
	\draw [-](-3,1) -- coordinate (y axis mid) (-3,-0.25) node[below] {$-2$};
	\draw [-](0,1) -- coordinate (y axis mid) (0,-0.25) node[below] {$0$};
	\draw [-](3,1) -- coordinate (y axis mid) (3,-0.25) node[below] {$2$};
	\draw (-4,0.5) node {$-$};
	\draw (-1.5,0.5) node {$+$};
	\draw (1.5,0.5) node {$-$};
	\draw (4,0.5) node {$+$};
\end{tikzpicture}
%%\item $g(x)=(x^2+1)(1-x)$
\item \ \\  %$h(x)=x(x-3)^2(x+3)$
\begin{tikzpicture}[xscale=0.5,yscale=0.5]
	\draw [<->](-5,0) -- coordinate (x axis mid) (5,0) node[below right] {$x$};
	\draw [-](-3,1) -- coordinate (y axis mid) (-3,-0.25) node[below] {$-3$};
	\draw [-](0,1) -- coordinate (y axis mid) (0,-0.25) node[below] {$0$};
	\draw [-](3,1) -- coordinate (y axis mid) (3,-0.25) node[below] {$3$};
	\draw (-4,0.5) node {$+$};
	\draw (-1.5,0.5) node {$-$};
	\draw (1.5,0.5) node {$+$};
	\draw (4,0.5) node {$+$};
\end{tikzpicture}
%%\item $k(x)=(3x-4)^3$
\item \ \\  %$\ell(x)=(x^2+2)(x^2+3)$
\begin{tikzpicture}[xscale=0.5,yscale=0.5]
	\draw [<->](-5,0) -- coordinate (x axis mid) (5,0) node[below right] {$x$};
	\draw (0,0.5) node {$+$};
\end{tikzpicture}

\columnbreak

%%\item $m(x)=-2(x+7)^2(1-2x)^2$
\item \ \\  %$f(x)=(x^2-1)(x+4)$
\begin{tikzpicture}[xscale=0.5,yscale=0.5]
	\draw [<->](-5,0) -- coordinate (x axis mid) (5,0) node[below right] {$x$};
	\draw [-](-3,1) -- coordinate (y axis mid) (-3,-0.25) node[below] {$-4$};
	\draw [-](0,1) -- coordinate (y axis mid) (0,-0.25) node[below] {$-1$};
	\draw [-](3,1) -- coordinate (y axis mid) (3,-0.25) node[below] {$1$};
	\draw (-4,0.5) node {$-$};
	\draw (-1.5,0.5) node {$+$};
	\draw (1.5,0.5) node {$-$};
	\draw (4,0.5) node {$+$};
\end{tikzpicture}
%%\item $g(x)=(x^2-1)(x^2-16)$
\item \ \\  %$h(x)=-2x^3(3x-1)(2-x)$
\begin{tikzpicture}[xscale=0.5,yscale=0.5]
	\draw [<->](-5,0) -- coordinate (x axis mid) (5,0) node[below right] {$x$};
	\draw [-](-3,1) -- coordinate (y axis mid) (-3,-0.25) node[below] {$0$};
	\draw [-](0,1) -- coordinate (y axis mid) (0,-0.25) node[below] {$\frac{1}{3}$};
	\draw [-](3,1) -- coordinate (y axis mid) (3,-0.25) node[below] {$2$};
	\draw (-4,0.5) node {$-$};
	\draw (-1.5,0.5) node {$+$};
	\draw (1.5,0.5) node {$-$};
	\draw (4,0.5) node {$+$};
\end{tikzpicture}
%%\item $k(x)=(x^2-4x+1)(x+2)^2$
\end{oddenumerate}
\end{multicols}
\subsection*{Factoring}
\subsubsection{Some Special Cases}
%Completely factor each of the following polynomial expressions.
\begin{multicols}{2}
\begin{oddenumerate}
  \item %$2 x^2 - 11 x + 15$
$(2x-5)(x-3)$
%%  \item $5 n^3 + 7 n^2 - 6 n$
  \item %$54 u^3 - 16$
$2(3x-2)(9x^2+6x+4)$
%%  \item $54 - 128 x^3$
  \item %$n^2 - n$
$n(n-1)$
%%  \item $2x^4 -21x^2-11$
  \item %$24 a z - 18 a h + 60 y z - 45 y h$
$3(2a+15y)(4z-3h)$
%%  \item $5 u^2 - 9 u v + 4 v^2$
  \item %$16 x^2 + 48 x y + 36 y^2$
$4(2x+3y)^2$
%%  \item $- 2 x^3 + 128 y^3$
  \item %$20 u v - 60 u^3 - 5 x v + 15 x u^2$
$-5(4u-x)(3u^2-v)$
%%  \item $2 x^3 + 5 x^2 y + 3 y^2 x$
  \end{oddenumerate}
\end{multicols}
\subsubsection{Quadratic Type}
%Completely factor each of the following polynomials over the real numbers and identify the set of all real roots.
%\begin{multicols}{2}
\begin{oddenumerate}
  \item  %$x^4 +13x^2+40$
$(x^2+8)(x^2+5)$
%%  \item  $x^4-5x^2+4$
  \item  %$x^4 -17x^2+16$
$(x-1)(x+1)(x-4)(x+4)$
%%  \item  $x^4 -3x^2-40$
  \item  %$3x^4 -32x^2+45$
$(3x^2-5)(x^2-9)=3(x-\frac{\sqrt{15}}{3})(x-\frac{\sqrt{15}}{3})(x-3)(x+3)$
%%  \item  $x^4 +x^2-12$
  \item  %$x^4 -3x^2-10$
$(x^2-5)(x^2+2)=(x-\sqrt{5})(x+\sqrt{5})(x^2+2)$
%%  \item  $x^6 -82x^3+81$
  \item  %$8x^4 +2x^2-3$
$(2x^2-\frac{1}{2})(4x^2+3)=2(x-\frac{\sqrt{2}}{2})(x+\frac{\sqrt{2}}{2})(4x^2+3)$
%%  \item  $2x^4 -19x^2+9$
\end{oddenumerate}
%\end{multicols}
\subsection*{Division}
\subsubsection{Polynomial (Long) Division}
%Use polynomial long division to divide and simplify each of the given expressions.  Express each answer in the form below.
%$$\frac{\text{dividend}}{\text{divisor}} \ = \ \text{quotient} \ + \ \frac{\text{remainder}}{\text{divisor}}$$
\begin{multicols}{2}
\begin{oddenumerate}
%%Note: All of these items need to be corrected to be in terms of the variable x for the polylongdiv command to work.
  \item %$\dfrac{20 x^4 + x^3 + 2 x^2}{4 x^3}$
\polylongdiv{20 x^4 + x^3 + 2 x^2}{4 x^3}
%%  \item $\dfrac{5 x^4 + 45 x^3 + 4 x^2}{9 x}$
  \item %$\dfrac{20 n^4 + n^3 + 40 n^2}{10 n}$
\polylongdiv{20 x^4 + x^3 + 40 x^2}{10 x}
%%  \item $\dfrac{3 k^3 + 4 k^2 + 2 k}{8 k}$
  \item %$\dfrac{12 x^4 + 24 x^3 + 3 x^2}{6 x}$
\polylongdiv{12 x^4 + 24 x^3 + 3 x^2}{6 x}
%%  \item $\dfrac{5 p^4 + 16 p^3 + 16 p^2}{4 p}$
  \item %$\dfrac{10 n^4 + 50 n^3 + 2 n^2}{10 n^2}$
\polylongdiv{10 x^4 + 50 x^3 + 2 x^2}{10 x^2}
%%  \item $\dfrac{3 m^4 + 18 m^3 + 27 m^2}{9 m^2}$
  \item %$\dfrac{x^2 - 2 x - 71}{x + 8}$%\label{polydiv_one}
\polylongdiv{x^2 - 2 x - 71}{x + 8}
%%  \item $\dfrac{r^2 - 3 r - 53}{r - 9}$
  \item %$\dfrac{n^2 + 13 n + 32}{n + 5}$
\polylongdiv{x^2 + 13 x + 32}{x + 5}
%%  \item $\dfrac{b^2 - 10 b + 16}{b - 7}$
  \item %$\dfrac{v^2 - 2 v - 89}{v - 10}$
\polylongdiv{x^2 - 2 x - 89}{x - 10}
%%  \item $\dfrac{x^2 + 4 x - 26}{x + 7}$
  \item %$\dfrac{a^2 - 4 a - 38}{a - 8}$
\polylongdiv{x^2 - 4 x - 38}{x - 8}
%%  \item $\dfrac{n^2 - 4}{n - 2}$
  \item %$\dfrac{a^3 + 15 a^2 + 49 a - 55}{a + 7}$
\polylongdiv{x^3 + 15 x^2 + 49 x - 55}{x + 7}
%%  \item $\dfrac{x^3 - 26 x - 41}{x + 4}$
  \item %$\dfrac{3 n^3 + 9 n^2 - 64 n - 68}{n + 6}$
\polylongdiv{3 x^3 + 9 x^2 - 64 x - 68}{x + 6}
%%  \item $\dfrac{9 p^3 + 45 p^2 + 27 p - 5}{9 p + 9}$
  \item %$\dfrac{r^3 - r^2 - 16 r + 8}{r - 4}$
\polylongdiv{x^3 - x^2 - 16 x + 8}{x - 4}
%%  \item $\dfrac{x^2 - 10 x + 22}{x - 4}$
  \item %$\dfrac{x^3 - 16 x^2 + 71 x - 56}{x - 8}$
\polylongdiv{x^3 - 16 x^2 + 71 x - 56}{x - 8}
%%  \item $\dfrac{k^3 - 4 k^2 - 6 k + 4}{k - 1}$
  \item %$\dfrac{8 k^3 - 66 k^2 + 12 k + 37}{k - 8}$
\polylongdiv{8 x^3 - 66 x^2 + 12 x + 37}{x - 8}
%%  \item $\dfrac{3 m^2 + 9 m - 9}{3 m - 3}$
  \item %$\dfrac{2 x^2 - 5 x - 8}{2 x + 3}$
\polylongdiv{2 x^2 - 5 x - 8}{2 x + 3}
%%  \item $\dfrac{3 v^2 - 32}{3 v - 9}$
  \item %$\dfrac{4 n^2 - 23 n - 38}{4 n + 5}$
\polylongdiv{4 x^2 - 23 x - 38}{4 x + 5}
%%  \item $\dfrac{2 n^3 + 21 n^2 + 25 n}{2 n + 3}$
  \item %$\dfrac{4 v^3 - 21 v^2 + 6 v + 19}{4 v + 3}$  
\polylongdiv{4 x^3 - 21 x^2 + 6 x + 19}{4 x + 3}
%%  \item $\dfrac{8 m^3 - 57 m^2 + 42}{8 m + 7}$
  \item %$\dfrac{2 x^3 + 12 x^2 + 4 x - 37}{2 x + 6}$
\polylongdiv{2 x^3 + 12 x^2 + 4 x - 37}{2 x + 6}
%%  \item $\dfrac{45 p^2 + 56 p + 19}{9 p + 4}$
  \item %$\dfrac{10 x^2 - 32 x + 9}{10 x - 2}$
\polylongdiv{10 x^2 - 32 x + 9}{10 x - 2}
%%  \item $\dfrac{4 r^2 - r - 1}{4 r + 3}$
  \item %$\dfrac{27 b^2 + 87 b + 35}{3 b + 8}$
\polylongdiv{27 x^2 + 87 x + 35}{3 x + 8}
%%  \item $\dfrac{4 x^2 - 33 x + 28}{4 x - 5}$
  \item %$\dfrac{48 k^2 - 70 k + 16}{6 k - 2}$
\polylongdiv{48 x^2 - 70 x + 16}{6 x - 2}
%%  \item $\dfrac{12 n^3 + 12 n^2 - 15 n - 4}{2 n + 3}$
  \item %$\dfrac{24 b^3 - 38 b^2 + 29 b - 60}{4 b - 7}$%\label{polydiv_two}
\polylongdiv{24 x^3 - 38 x^2 + 29 x - 60}{4 x - 7}
	\end{oddenumerate}
\end{multicols}
\subsubsection{Synthetic Division}
%Use synthetic division to divide and simplify each of the given expressions.  Express each answer in the form below.
%$$\frac{\text{dividend}}{\text{divisor}} \ = \ \text{quotient} \ + \ \frac{\text{remainder}}{\text{divisor}}$$
%\begin{multicols}{2}
\begin{oddenumerate}
  \item %$\dfrac
\polyhornerscheme[x=-2,showbase=top,resultstyle=\bf]{x^4-4x^3+2x^2-x+1}
%%  \item $\dfrac{x^4-2x^3+7x^2-6x+3}{x-2}$
  \item %$\dfrac
\polyhornerscheme[x=-2,showbase=top,resultstyle=\bf]{2x^4-2x^3-10x^2+1}
%%  \item $\dfrac{5x^4-2x^3+4x^2-5x}{x-1}$
  \item %$\dfrac
  \polyhornerscheme[x=-5,showbase=top,resultstyle=\bf]{-x^4-x^3+x^2+x+1}
%%  \item $\dfrac{x^4-3x^3+2x^2-x+1}{x-4}$
  \item %$\dfrac
  \polyhornerscheme[x=-2,showbase=top,resultstyle=\bf]{12x^4-x^3+x^2-3x+1}
%%  \item $\dfrac{3x^4+3x^3+13x^2-4x+14}{x+1}$
  \item %$\dfrac
  \polyhornerscheme[x=2,showbase=top,resultstyle=\bf]{1x^4-3x^3+5x^2-14x+2}
%%  \item $\dfrac{2x^4-2x+1}{x+3}$
  \item %$\dfrac
  \polyhornerscheme[x=3,showbase=top,resultstyle=\bf]{x^4-3x-4}
%%  \item $\dfrac{x^4-4x^3+13x^2-5x+7}{x-4}$
\end{oddenumerate}
%\end{multicols}
%Use synthetic division to divide and simplify each of the given expression from Exercises \ref{polydiv_one}-\ref{polydiv_two}.
\subsection*{End Behavior}
%Determine the end behavior of each of the following functions.  Write your answers as mathematical sentences.  Graph each function on \Desmos \ to check your answers.
\begin{oddenumerate}
%\begin{multicols}{2}
\item %$f(x)=-2x^3 + 4x+1$
As $x\rightarrow -\infty, \ f(x)\rightarrow \infty$. \ \ \ As $x\rightarrow\infty, \ f(x)\rightarrow -\infty$.
%%\item $g(x)=32x^5+x^2+15$
\item %$h(x)=-3x^4+4x^2$
As $x\rightarrow -\infty, \ h(x)\rightarrow -\infty$. \ \ \ As $x\rightarrow\infty, \ h(x)\rightarrow -\infty$.
%%\item $k(x)=15x^4-32x^2-x-14$
\item %$\ell(x)=x^5+40$
As $x\rightarrow -\infty, \ \ell(x)\rightarrow -\infty$. \ \ \ As $x\rightarrow\infty, \ \ell(x)\rightarrow \infty$.
%%\item $m(x)=5x^5+3x^2+x+14$
\item %$n(x)=123x^4-7x^3-5x^2-3x+1$
As $x\rightarrow -\infty, \ n(x)\rightarrow \infty$. \ \ \ As $x\rightarrow\infty, \ n(x)\rightarrow \infty$.
%%\item $p(x)=x^3-1$
\item %$q(x)=-23x^6+x^3+x^2+x+1$
As $x\rightarrow -\infty, \ q(x)\rightarrow -\infty$. \ \ \ As $x\rightarrow\infty, \ q(x)\rightarrow -\infty$.
%%\end{multicols}
%\end{oddenumerate}
%Identify the degree, leading coefficient, and constant term of each polynomial function below.  Use the degree and leading coefficient to identify the end behavior of the graph of each function.  Write your answers as mathematical sentences.  Graph each function on \Desmos \ to check your answers.
%\begin{oddenumerate}[resume]
%\begin{multicols}{2}
%    \item $f(x)=x^3(x-2)(x+2)$
	\item %$g(x)=(x^2+1)(1-x)$
$a_nx^n=-1x^3, \ \ a_0=1,$ \ \ As $x\rightarrow -\infty, \ f(x)\rightarrow \infty$. \ \ \ As $x\rightarrow\infty, \ f(x)\rightarrow -\infty$.
%	\item $h(x)=x(x-3)^2(x+3)$
	\item %$k(x)=(3x-4)^3$
$a_nx^n=27x^3, \ \ a_0=-64,$ \ \ As $x\rightarrow -\infty, \ k(x)\rightarrow -\infty$. \ \ \ As $x\rightarrow\infty, \ k(x)\rightarrow \infty$.
%    \item $\ell(x)=(x^2+2)(x^2+3)$
    \item %$m(x)=-2(x+7)^2(1-2x)^2$
$a_nx^n=-8x^4, \ \ a_0=-98,$ \ \ As $x\rightarrow -\infty, \ m(x)\rightarrow -\infty$. \ \ \ As $x\rightarrow\infty, \ m(x)\rightarrow -\infty$.
%    \item $f(x)=(x^2-1)(x+4)$
	\item %$g(x)=(x^2-1)(x^2-16)$
$a_nx^n=1x^4, \ \ a_0=16,$ \ \ As $x\rightarrow -\infty, \ g(x)\rightarrow \infty$. \ \ \ As $x\rightarrow\infty, \ g(x)\rightarrow \infty$.
%	\item $h(x)=-2x^3(3x-1)(2-x)$
	\item %$k(x)=(x^2-4x+1)(x+2)^2$
$a_nx^n=1x^4, \ \ a_0=2,$ \ \ As $x\rightarrow -\infty, \ k(x)\rightarrow \infty$. \ \ \ As $x\rightarrow\infty, \ k(x)\rightarrow \infty$.
%\end{multicols}
\end{oddenumerate}
\subsection*{Local Behavior}
%Determine the set of roots and corresponding multiplicities for the following functions.  In each case, classify the corresponding $x-$intercept as either a turnaround or crossover point.  Use \Desmos \ to check your answers.
%\begin{multicols}{2}
\begin{oddenumerate}
\item %$f(x)=x^3(x-2)(x+2)$
$x_1=0, k_1=3,$ Crossover; \ $x_2=2, k_2=1,$ Crossover;  \ $x_3=-2, k_3=1,$ Crossover
%%\item $g(x)=(x^2+1)(1-x)$
\item %$h(x)=x(x-3)^2(x+3)$
$x_1=0, k_1=1,$ Crossover; \ $x_2=3, k_2=2,$ Turnaround;  \ $x_3=-3, k_3=1,$ Crossover
%%\item $k(x)=(3x-4)^3$
\item %$\ell(x)=(x^2+2)(x^2+3)$
No real roots
%%\item $m(x)=-2(x+7)^2(1-2x)^2$
\item %$f(x)=(x^2-1)(x+4)$
$x_1=1, k_1=1,$ Crossover; \ $x_2=-1, k_2=1,$ Crossover;  \ $x_3=-4, k_3=1,$ Crossover
%%\item $g(x)=(x^2-1)(x^2-16)$
\item %$h(x)=-2x^3(3x-1)(2-x)$
$x_1=0, k_1=3,$ Crossover; \ $x_2=\frac{1}{3}, k_2=1,$ Crossover;  \ $x_3=2, k_3=1,$ Crossover 
%%\item $k(x)=(x^2-4x+1)(x+2)^2$
\item %$f(x)=\frac{1}{2}(x-2)^2(x+5)(x-3)$
$x_1=2, k_1=2,$ Turnaround; \ $x_2=-5, k_2=1,$ Crossover;  \ $x_3=3, k_3=1,$ Crossover 
%%\item $g(x)=(x+2)^2(3x-1)(5-x)$
\end{oddenumerate}
%\end{multicols}
\subsection*{The Rational Root Theorem}
%Use the Rational Root Theorem to identify a set of possible rational roots for each of the polynomial functions below.  Evaluate the function at $x=1$.  If $x=1$ is a real root, divide the polynomial by $x-1$ and factor the resulting quotient.  If $x=1$ is not a real root, evaluate the function at at least one of your remaining possible roots, in order to determine if they are actual roots of the polynomial.  If successful, divide your polynomial by the respective factor and factor the remaining quotient.  Use \Desmos \ to help determine the actual set of real roots.
\begin{oddenumerate}
	\item $f(x) = $ \polyfactorize{x^3 - 2x^2 - 5x + 6}\\
	List of possible rational roots: $\{\pm 6, \pm 3, \pm 2, \pm 1\}$
%	\item $f(x) = 2x^4+x^3-7x^2-3x+3$
	\item $f(x) = $ \polyfactorize{x^5-x^4-37x^3+37x^2+36x-36}\\
	List of possible rational roots: $\{\pm 36, \pm 18, \pm 12, \pm 9, \pm 6, \pm 4, \pm 3, \pm 2, \pm 1\}$
%	\item $f(x) = 3x^{3} + 3x^{2} - 11x - 10$
%\end{oddenumerate}
%Use the Rational Root Theorem to identify a set of possible rational roots for each of the polynomial functions below.  Evaluate the function at at least two of your possible roots, in order to determine if they are actual roots of the polynomial.  If successful, divide your polynomial by the respective factor.  Use \Desmos \ to help determine the actual set of real roots.
%\begin{oddenumerate}[resume]
	\item $f(x) = $ \polyfactorize{x^4 -2x^3+ 5x^2 - 8x + 4}\\
	List of possible rational roots: $\{\pm 4, \pm 2, \pm 1 \}$
%	\item $f(x) = x^{3} + 4x^{2} - 11x + 6$
	\item $f(x) = $ \polyfactorize{-2x^3 + 19x^2 - 49x + 20}\\
	List of possible rational roots: $\{\pm 20, \pm 10, \pm 5, \pm 4, \pm{\frac{5}{2}}, \pm 2, \pm 1, \pm{\frac{1}{2}}\}$
%	\item $f(x) = 36x^{4} - 12x^{3} - 11x^{2} + 2x + 1$
	\item $f(x) = $ \polyfactorize{x^4 - 9x^2 - 4x + 12}\\
	List of possible rational roots: $\{\pm 12, \pm 6, \pm 4, \pm 3, \pm 2, \pm 1 \}$
%	\item $f(x) = x^{4} + 2x^{3} - 12x^{2} - 40x - 32$
	\item $f(x) = $ \polyfactorize{6x^3+19x^2-6x-40}\\
	List of possible rational roots:\\
	$\{\pm 40, \pm 20, \pm{\frac{40}{3}}, \pm 10, \pm 8, \pm{\frac{20}{3}}, \pm 5, \pm 4, \pm{\frac{10}{3}}, \pm{\frac{8}{3}}, \pm{\frac{5}{2}}, \pm 2, \pm{\frac{5}{3}}, \pm{\frac{4}{3}}, \pm 1, \pm{\frac{5}{6}}, \pm{\frac{2}{3}}, \pm{\frac{1}{2}}, \pm{\frac{1}{3}}, \pm{\frac{1}{6}} \}$
\end{oddenumerate}
\subsection*{Graphing Summary}
%Factor each polynomial below, and sketch a complete graph of the function, making sure to have a clearly defined scale and label any intercepts.  Use \Desmos \ to compare your results.
\begin{oddenumerate}
%\begin{multicols}{2}
	\item $f(x) = $ \polyfactorize{-17x^3 + 5x^2 + 34x - 10}
%	\item $f(x) = x^4-9x^2+14$
	\item $f(x) = (3x^2+1)$ \polyfactorize{x^2-5}
%	\item $f(x) = 2x^4-7x^2+6$
	\item $f(x) = $ \polyfactorize{x^5-2x^4-x+2}
%	\item $f(x) = 2x^5+3x^4-32x-48$
	\item $f(x) = (x^3-8)(x^3+2)=(x-2)(x^2+2x+4)(x+\sqrt[3]{2})(x^2-\sqrt[3]{2}x+\sqrt[3]{4})$
%	\item $f(x) = 2x^6-7x^3+5$
	\item $f(x) = -(x^2+1)$ \polyfactorize{x - 7}
%	\item $f(x) = 3x^4 - 5x^3 - 12x^2$
	\item $f(x) = $ \polyfactorize{2x^3 - 5x^2 - x}
%	\item $f(x) = -x^4-2x^2 +15$
%\end{multicols}
%\end{oddenumerate}
%Get a complete factorization of each polynomial below by first dividing the function by $x-1$.  Then sketch a graph of the function, making sure to have a clearly defined scale and label any intercepts.  Use \Desmos \ to compare your results.
%\begin{oddenumerate}[resume]
	\item $f(x) = $ \polyfactorize{x^3 - 2x^2 - 5x + 6}
%	\item $f(x) = x^{3} + 4x^{2} - 11x + 6$
	\item $f(x)= $ \polyfactorize{x^5-x^4-37x^3+37x^2+36x-36}
%	\item $f(x) = x^{4} -2x^3+ 5x^{2} - 8x + 4$
%\end{oddenumerate}
%Use the Rational Root Theorem and polynomial division to get a complete factorization of each polynomial function below.  Then sketch a graph of the function, making sure to have a clearly defined scale and label any intercepts.  Use \Desmos \ to compare your results.
%\begin{oddenumerate}[resume]
	\item $f(x) = $ \polyfactorize{x^4 - 9x^2 - 4x + 12}
%	\item $f(x) = x^{4} + 2x^{3} - 12x^{2} - 40x - 32$
	\item $f(x) = $ \polyfactorize{2x^4+x^3-7x^2-3x+3}
%	\item $f(x) = 3x^{3} + 3x^{2} - 11x - 10$
	\item $f(x) = $ \polyfactorize{6x^3+19x^2-6x-40}
%	\item $f(x) = -2x^{3} + 19x^{2} - 49x + 20$
	\item $f(x) = $ \polyfactorize{36x^4 - 12x^3 - 11x^2 + 2x + 1}
%	\item $f(x) = x^4+4x^3-x-4$	
	\item $f(x) = $ \polyfactorize{2x^3-5x^2-52x+60}
%	\item $f(x) = -x^3-x^2+39x+45$
	\item $f(x) = $ \polyfactorize{-2x^4+7x^3+17x^2-28x-36}
%	\item $f(x) = x^7-5x^6-24x^5+120x^4-25x^3+125x^2$
\end{oddenumerate}
\subsection*{Polynomial Inequalities}
%Solve each polynomial inequality below, expressing your answers using interval notation.  Use \Desmos \ to help confirm that each answer is correct.
\begin{multicols}{2}
\begin{oddenumerate}
 \item $(-\infty,\sqrt{2}]\cup [\sqrt{2},\infty)$ %$x^4 + x^2 \geq 6$
% \item $x^{4} - 9x^{2} \leq 4x - 12$
 \item $[1,\infty)$ %$4x^3 \geq 3x+1$
% \item $x^4 \leq 16+4x-x^3$
 \item $(-\infty,0)\cup (2,\infty)$ %$3x^2 + 2x < x^4$
% \item $\dfrac{x^3+2 x^2}{2} < x+2$
 \item $[4,\infty)$ %$\dfrac{x^3+20x}{8} \geq x^2 + 2$
% \item $19x^{2} + 20 > 2x^{3} + 49x $
 \item $(-\infty,4)$ %$x^3<4x^2$
% \item $x^3-7x^2\leq 12x-84$
 \item $(-\infty,-1]\cup [3,\infty)$ %$(x - 1)^{2} \geq 4$
% \item $2x^4>5x^2+3$
\end{oddenumerate}
\end{multicols}
\end{document}
\end{document}