\documentclass[12pt]{book}
\raggedbottom
\usepackage[top=1in,left=1in,bottom=1in,right=1in,headsep=0.25in]{geometry}	
\usepackage{amssymb,amsmath,amsthm,amsfonts}
\usepackage{chapterfolder,docmute,setspace}
\usepackage{cancel,multicol,tikz,verbatim,framed,polynom,enumitem,tikzpagenodes}
\usepackage[colorlinks, hyperindex, plainpages=false, linkcolor=blue, urlcolor=blue, pdfpagelabels]{hyperref}
\usepackage[type={CC},modifier={by-nc-sa},version={4.0},]{doclicense}

\usepackage{import}

\theoremstyle{definition}
\newtheorem{example}{Example}
\newcommand{\Desmos}{\href{https://www.desmos.com/}{Desmos}}
\setlength{\parindent}{0in}
\setlist{itemsep=0em}
\setlength{\parskip}{0.1in}
\setcounter{secnumdepth}{0}
% This document is used for ordering of lessons.  If an instructor wishes to change the ordering of assessments, the following steps must be taken:

% 1) Reassign the appropriate numbers for each lesson in the \setcounter commands included in this file.
% 2) Rearrange the \include commands in the master file (the file with 'Course Pack' in the name) to accurately reflect the changes.  
% 3) Rarrange the \items in the measureable_outcomes file to accurately reflect the changes.  Be mindful of page breaks when moving items.
% 4) Re-build all affected files (master file, measureable_outcomes file, and any lessons whose numbering has changed).

%Note: The placement of each \newcounter and \setcounter command reflects the original/default ordering of topics (linears, systems, quadratics, functions, polynomials, rationals).

\newcounter{lesson_solving_linear_equations}
\newcounter{lesson_equations_containing_absolute_values}
\newcounter{lesson_graphing_lines}
\newcounter{lesson_two_forms_of_a_linear_equation}
\newcounter{lesson_parallel_and_perpendicular_lines}
\newcounter{lesson_linear_inequalities}
\newcounter{lesson_compound_inequalities}
\newcounter{lesson_inequalities_containing_absolute_values}
\newcounter{lesson_graphing_systems}
\newcounter{lesson_substitution}
\newcounter{lesson_elimination}
\newcounter{lesson_quadratics_introduction}
\newcounter{lesson_factoring_GCF}
\newcounter{lesson_factoring_grouping}
\newcounter{lesson_factoring_trinomials_a_is_1}
\newcounter{lesson_factoring_trinomials_a_neq_1}
\newcounter{lesson_solving_by_factoring}
\newcounter{lesson_square_roots}
\newcounter{lesson_i_and_complex_numbers}
\newcounter{lesson_vertex_form_and_graphing}
\newcounter{lesson_solve_by_square_roots}
\newcounter{lesson_extracting_square_roots}
\newcounter{lesson_the_discriminant}
\newcounter{lesson_the_quadratic_formula}
\newcounter{lesson_quadratic_inequalities}
\newcounter{lesson_functions_and_relations}
\newcounter{lesson_evaluating_functions}
\newcounter{lesson_finding_domain_and_range_graphically}
\newcounter{lesson_fundamental_functions}
\newcounter{lesson_finding_domain_algebraically}
\newcounter{lesson_solving_functions}
\newcounter{lesson_function_arithmetic}
\newcounter{lesson_composite_functions}
\newcounter{lesson_inverse_functions_definition_and_HLT}
\newcounter{lesson_finding_an_inverse_function}
\newcounter{lesson_transformations_translations}
\newcounter{lesson_transformations_reflections}
\newcounter{lesson_transformations_scalings}
\newcounter{lesson_transformations_summary}
\newcounter{lesson_piecewise_functions}
\newcounter{lesson_functions_containing_absolute_values}
\newcounter{lesson_absolute_as_piecewise}
\newcounter{lesson_polynomials_introduction}
\newcounter{lesson_sign_diagrams_polynomials}
\newcounter{lesson_factoring_quadratic_type}
\newcounter{lesson_factoring_summary}
\newcounter{lesson_polynomial_division}
\newcounter{lesson_synthetic_division}
\newcounter{lesson_end_behavior_polynomials}
\newcounter{lesson_local_behavior_polynomials}
\newcounter{lesson_rational_root_theorem}
\newcounter{lesson_polynomials_graphing_summary}
\newcounter{lesson_polynomial_inequalities}
\newcounter{lesson_rationals_introduction_and_terminology}
\newcounter{lesson_sign_diagrams_rationals}
\newcounter{lesson_horizontal_asymptotes}
\newcounter{lesson_slant_and_curvilinear_asymptotes}
\newcounter{lesson_vertical_asymptotes}
\newcounter{lesson_holes}
\newcounter{lesson_rationals_graphing_summary}

\setcounter{lesson_solving_linear_equations}{1}
\setcounter{lesson_equations_containing_absolute_values}{2}
\setcounter{lesson_graphing_lines}{3}
\setcounter{lesson_two_forms_of_a_linear_equation}{4}
\setcounter{lesson_parallel_and_perpendicular_lines}{5}
\setcounter{lesson_linear_inequalities}{6}
\setcounter{lesson_compound_inequalities}{7}
\setcounter{lesson_inequalities_containing_absolute_values}{8}
\setcounter{lesson_graphing_systems}{9}
\setcounter{lesson_substitution}{10}
\setcounter{lesson_elimination}{11}
\setcounter{lesson_quadratics_introduction}{16}
\setcounter{lesson_factoring_GCF}{17}
\setcounter{lesson_factoring_grouping}{18}
\setcounter{lesson_factoring_trinomials_a_is_1}{19}
\setcounter{lesson_factoring_trinomials_a_neq_1}{20}
\setcounter{lesson_solving_by_factoring}{21}
\setcounter{lesson_square_roots}{22}
\setcounter{lesson_i_and_complex_numbers}{23}
\setcounter{lesson_vertex_form_and_graphing}{24}
\setcounter{lesson_solve_by_square_roots}{25}
\setcounter{lesson_extracting_square_roots}{26}
\setcounter{lesson_the_discriminant}{27}
\setcounter{lesson_the_quadratic_formula}{28}
\setcounter{lesson_quadratic_inequalities}{29}
\setcounter{lesson_functions_and_relations}{12}
\setcounter{lesson_evaluating_functions}{13}
\setcounter{lesson_finding_domain_and_range_graphically}{14}
\setcounter{lesson_fundamental_functions}{15}
\setcounter{lesson_finding_domain_algebraically}{30}
\setcounter{lesson_solving_functions}{31}
\setcounter{lesson_function_arithmetic}{32}
\setcounter{lesson_composite_functions}{33}
\setcounter{lesson_inverse_functions_definition_and_HLT}{34}
\setcounter{lesson_finding_an_inverse_function}{35}
\setcounter{lesson_transformations_translations}{36}
\setcounter{lesson_transformations_reflections}{37}
\setcounter{lesson_transformations_scalings}{38}
\setcounter{lesson_transformations_summary}{39}
\setcounter{lesson_piecewise_functions}{40}
\setcounter{lesson_functions_containing_absolute_values}{41}
\setcounter{lesson_absolute_as_piecewise}{42}
\setcounter{lesson_polynomials_introduction}{43}
\setcounter{lesson_sign_diagrams_polynomials}{44}
\setcounter{lesson_factoring_quadratic_type}{46}
\setcounter{lesson_factoring_summary}{45}
\setcounter{lesson_polynomial_division}{47}
\setcounter{lesson_synthetic_division}{48}
\setcounter{lesson_end_behavior_polynomials}{49}
\setcounter{lesson_local_behavior_polynomials}{50}
\setcounter{lesson_rational_root_theorem}{51}
\setcounter{lesson_polynomials_graphing_summary}{52}
\setcounter{lesson_polynomial_inequalities}{53}
\setcounter{lesson_rationals_introduction_and_terminology}{54}
\setcounter{lesson_sign_diagrams_rationals}{55}
\setcounter{lesson_horizontal_asymptotes}{56}
\setcounter{lesson_slant_and_curvilinear_asymptotes}{57}
\setcounter{lesson_vertical_asymptotes}{58}
\setcounter{lesson_holes}{59}
\setcounter{lesson_rationals_graphing_summary}{60}

\newcommand{\tmmathbf}[1]{\ensuremath{\boldsymbol{#1}}}
\newcommand{\tmop}[1]{\ensuremath{\operatorname{#1}}}
%\usepackage[metapost,truebbox]{mfpic,colortbl}

\newlist{oddenumerate}{enumerate}{1}
\setlist[oddenumerate]{start=0,label=\theoddenumeratei.}
\makeatletter
\renewcommand\theoddenumeratei{\@arabic{\numexpr2*\value{oddenumeratei}+1}}
\makeatother

\newcount\gpten % (global) power-of-ten -- tells which digit we are doing
\countdef\rtot2 % running total -- remainder so far
\countdef\LDscratch4 % scratch

\def\longdiv#1#2{%
 \vtop{\normalbaselines \offinterlineskip
   \setbox\strutbox\hbox{\vrule height 2.1ex depth .5ex width0ex}%
   \def\showdig{$\underline{\the\LDscratch\strut}$\cr\the\rtot\strut\cr
       \noalign{\kern-.2ex}}%
   \global\rtot=#1\relax
   \count0=\rtot\divide\count0by#2\edef\quotient{\the\count0}%\show\quotient
   % make list macro out of digits in quotient:
   \def\temp##1{\ifx##1\temp\else \noexpand\dodig ##1\expandafter\temp\fi}%
   \edef\routine{\expandafter\temp\quotient\temp}%
   % process list to give power-of-ten:
   \def\dodig##1{\global\multiply\gpten by10 }\global\gpten=1 \routine
   % to display effect of one digit in quotient (zero ignored):
   \def\dodig##1{\global\divide\gpten by10
      \LDscratch =\gpten
      \multiply\LDscratch  by##1%
      \multiply\LDscratch  by#2%
      \global\advance\rtot-\LDscratch \relax
      \ifnum\LDscratch>0 \showdig \fi % must hide \cr in a macro to skip it
   }%
   \tabskip=0pt
   \halign{\hfil##\cr % \halign for entire division problem
     $\quotient$\strut\cr
     #2$\,\overline{\vphantom{\big)}%
     \hbox{\smash{\raise3.5\fontdimen8\textfont3\hbox{$\big)$}}}%
     \mkern2mu \the\rtot}$\cr\noalign{\kern-.2ex}
     \routine \cr % do each digit in quotient
}}}

\begin{document}
\pagenumbering{roman}
\documentclass[12pt]{book}
\raggedbottom
\usepackage[top=1in,left=1in,bottom=1in,right=1in,headsep=0.25in]{geometry}	
\usepackage{amssymb,amsmath,amsthm,amsfonts}
\usepackage{chapterfolder,docmute,setspace}
\usepackage{cancel,multicol,tikz,verbatim,framed,polynom,enumitem}
\usepackage[colorlinks, hyperindex, plainpages=false, linkcolor=blue, urlcolor=blue, pdfpagelabels]{hyperref}
\usepackage[type={CC},modifier={by-nc-sa},version={4.0},]{doclicense}

\usepackage{import}

\theoremstyle{definition}
\newtheorem{example}{Example}
\newcommand{\Desmos}{\href{https://www.desmos.com/}{Desmos}}
\setlength{\parindent}{0in}
\setlist{itemsep=0em}
\setlength{\parskip}{0.1in}
\setcounter{secnumdepth}{0}
% This document is used for ordering of lessons.  If an instructor wishes to change the ordering of assessments, the following steps must be taken:

% 1) Reassign the appropriate numbers for each lesson in the \setcounter commands included in this file.
% 2) Rearrange the \include commands in the master file (the file with 'Course Pack' in the name) to accurately reflect the changes.  
% 3) Rarrange the \items in the measureable_outcomes file to accurately reflect the changes.  Be mindful of page breaks when moving items.
% 4) Re-build all affected files (master file, measureable_outcomes file, and any lessons whose numbering has changed).

%Note: The placement of each \newcounter and \setcounter command reflects the original/default ordering of topics (linears, systems, quadratics, functions, polynomials, rationals).

\newcounter{lesson_solving_linear_equations}
\newcounter{lesson_equations_containing_absolute_values}
\newcounter{lesson_graphing_lines}
\newcounter{lesson_two_forms_of_a_linear_equation}
\newcounter{lesson_parallel_and_perpendicular_lines}
\newcounter{lesson_linear_inequalities}
\newcounter{lesson_compound_inequalities}
\newcounter{lesson_inequalities_containing_absolute_values}
\newcounter{lesson_graphing_systems}
\newcounter{lesson_substitution}
\newcounter{lesson_elimination}
\newcounter{lesson_quadratics_introduction}
\newcounter{lesson_factoring_GCF}
\newcounter{lesson_factoring_grouping}
\newcounter{lesson_factoring_trinomials_a_is_1}
\newcounter{lesson_factoring_trinomials_a_neq_1}
\newcounter{lesson_solving_by_factoring}
\newcounter{lesson_square_roots}
\newcounter{lesson_i_and_complex_numbers}
\newcounter{lesson_vertex_form_and_graphing}
\newcounter{lesson_solve_by_square_roots}
\newcounter{lesson_extracting_square_roots}
\newcounter{lesson_the_discriminant}
\newcounter{lesson_the_quadratic_formula}
\newcounter{lesson_quadratic_inequalities}
\newcounter{lesson_functions_and_relations}
\newcounter{lesson_evaluating_functions}
\newcounter{lesson_finding_domain_and_range_graphically}
\newcounter{lesson_fundamental_functions}
\newcounter{lesson_finding_domain_algebraically}
\newcounter{lesson_solving_functions}
\newcounter{lesson_function_arithmetic}
\newcounter{lesson_composite_functions}
\newcounter{lesson_inverse_functions_definition_and_HLT}
\newcounter{lesson_finding_an_inverse_function}
\newcounter{lesson_transformations_translations}
\newcounter{lesson_transformations_reflections}
\newcounter{lesson_transformations_scalings}
\newcounter{lesson_transformations_summary}
\newcounter{lesson_piecewise_functions}
\newcounter{lesson_functions_containing_absolute_values}
\newcounter{lesson_absolute_as_piecewise}
\newcounter{lesson_polynomials_introduction}
\newcounter{lesson_sign_diagrams_polynomials}
\newcounter{lesson_factoring_quadratic_type}
\newcounter{lesson_factoring_summary}
\newcounter{lesson_polynomial_division}
\newcounter{lesson_synthetic_division}
\newcounter{lesson_end_behavior_polynomials}
\newcounter{lesson_local_behavior_polynomials}
\newcounter{lesson_rational_root_theorem}
\newcounter{lesson_polynomials_graphing_summary}
\newcounter{lesson_polynomial_inequalities}
\newcounter{lesson_rationals_introduction_and_terminology}
\newcounter{lesson_sign_diagrams_rationals}
\newcounter{lesson_horizontal_asymptotes}
\newcounter{lesson_slant_and_curvilinear_asymptotes}
\newcounter{lesson_vertical_asymptotes}
\newcounter{lesson_holes}
\newcounter{lesson_rationals_graphing_summary}

\setcounter{lesson_solving_linear_equations}{1}
\setcounter{lesson_equations_containing_absolute_values}{2}
\setcounter{lesson_graphing_lines}{3}
\setcounter{lesson_two_forms_of_a_linear_equation}{4}
\setcounter{lesson_parallel_and_perpendicular_lines}{5}
\setcounter{lesson_linear_inequalities}{6}
\setcounter{lesson_compound_inequalities}{7}
\setcounter{lesson_inequalities_containing_absolute_values}{8}
\setcounter{lesson_graphing_systems}{9}
\setcounter{lesson_substitution}{10}
\setcounter{lesson_elimination}{11}
\setcounter{lesson_quadratics_introduction}{16}
\setcounter{lesson_factoring_GCF}{17}
\setcounter{lesson_factoring_grouping}{18}
\setcounter{lesson_factoring_trinomials_a_is_1}{19}
\setcounter{lesson_factoring_trinomials_a_neq_1}{20}
\setcounter{lesson_solving_by_factoring}{21}
\setcounter{lesson_square_roots}{22}
\setcounter{lesson_i_and_complex_numbers}{23}
\setcounter{lesson_vertex_form_and_graphing}{24}
\setcounter{lesson_solve_by_square_roots}{25}
\setcounter{lesson_extracting_square_roots}{26}
\setcounter{lesson_the_discriminant}{27}
\setcounter{lesson_the_quadratic_formula}{28}
\setcounter{lesson_quadratic_inequalities}{29}
\setcounter{lesson_functions_and_relations}{12}
\setcounter{lesson_evaluating_functions}{13}
\setcounter{lesson_finding_domain_and_range_graphically}{14}
\setcounter{lesson_fundamental_functions}{15}
\setcounter{lesson_finding_domain_algebraically}{30}
\setcounter{lesson_solving_functions}{31}
\setcounter{lesson_function_arithmetic}{32}
\setcounter{lesson_composite_functions}{33}
\setcounter{lesson_inverse_functions_definition_and_HLT}{34}
\setcounter{lesson_finding_an_inverse_function}{35}
\setcounter{lesson_transformations_translations}{36}
\setcounter{lesson_transformations_reflections}{37}
\setcounter{lesson_transformations_scalings}{38}
\setcounter{lesson_transformations_summary}{39}
\setcounter{lesson_piecewise_functions}{40}
\setcounter{lesson_functions_containing_absolute_values}{41}
\setcounter{lesson_absolute_as_piecewise}{42}
\setcounter{lesson_polynomials_introduction}{43}
\setcounter{lesson_sign_diagrams_polynomials}{44}
\setcounter{lesson_factoring_quadratic_type}{46}
\setcounter{lesson_factoring_summary}{45}
\setcounter{lesson_polynomial_division}{47}
\setcounter{lesson_synthetic_division}{48}
\setcounter{lesson_end_behavior_polynomials}{49}
\setcounter{lesson_local_behavior_polynomials}{50}
\setcounter{lesson_rational_root_theorem}{51}
\setcounter{lesson_polynomials_graphing_summary}{52}
\setcounter{lesson_polynomial_inequalities}{53}
\setcounter{lesson_rationals_introduction_and_terminology}{54}
\setcounter{lesson_sign_diagrams_rationals}{55}
\setcounter{lesson_horizontal_asymptotes}{56}
\setcounter{lesson_slant_and_curvilinear_asymptotes}{57}
\setcounter{lesson_vertical_asymptotes}{58}
\setcounter{lesson_holes}{59}
\setcounter{lesson_rationals_graphing_summary}{60}

\begin{document}
\thispagestyle{empty}
\ \vspace{2in}
\begin{center}
{\bf \Huge College Algebra\\
\ \\
Textbook}\\
\vspace{0.5in}
\large {\bf Primary Authors:}\\
\ \\
 Benjamin Atchison
\& Mark Morabito
\end{center}
\newpage
\thispagestyle{empty}
\hfill \doclicenseImage[imagewidth=5em]\\
\par
\normalsize This textbook is designed as the primary resource for instruction of a traditional College Algebra course at
Framingham State University. 
Each section follows closely with its respective lesson(s) in the accompanying course pack, but offers more detailed explanations and additional worked out examples.\\
\par
Although largely free of mathematical errors and ``typos'', students who identify any errors/typos in either the textbook or course pack are encouraged to report them to the instructor, and the reporting of any mathematical errors will be rewarded with small incentives in the form of additional course homework, quiz, or exam points.
\par
\hrulefill
\par
\ \par
The following chapters make up the first half of the course and cover the following content.
\begin{itemize}
	\item Linear Equations and Inequalities
	\item Systems of Linear Equations
	\item Introduction to Functions
	\item Quadratic Equations and Inequalities
\end{itemize}
The following chapters make up the second half of the course and cover the following content.
\begin{itemize}
	\item Advanced Function Concepts
	\item Polynomials
	\item Rational Functions
\end{itemize}
\vfill
%{\bf Note:} This course pack is required only for students whose {\it enrolled course instructor of record} is listed above.
%{\bf Section(s):} 001/002/012 \hfill {\bf CRN(s):} 20604/20605/21048\\
This text contains original content, as well as content adapted from each of the following open-source texts.\par
\href{http://www.wallace.ccfaculty.org/book/book.html}{\it{Beginning and Intermediate Algebra}} by T. Wallace. CC-BY\\
\href{http://www.stitz-zeager.com}{\it{College Algebra}} by C. Stitz and J. Zeager. CC-BY-NC-SA 
\newpage
\end{document}
\tableofcontents
\newpage
\ \\
\thispagestyle{empty}
\ \\
\newpage
\documentclass[12pt]{article}
\usepackage[top=1in,left=1in,bottom=1in,right=1in,headsep=2pt]{geometry}	
\usepackage{amssymb,amsmath,amsthm,amsfonts}
\usepackage{chapterfolder,docmute,setspace}
\usepackage{cancel,multicol,tikz,verbatim,framed,polynom,enumitem}
\usepackage[colorlinks, hyperindex, plainpages=false, linkcolor=blue, urlcolor=blue, pdfpagelabels]{hyperref}
% Use the cc-by-nc-sa license for any content linked with Stitz and Zeager's text.  Otherwise, use the cc-by-sa license.
\usepackage[type={CC},modifier={by-sa},version={4.0},]{doclicense}
%\usepackage[type={CC},modifier={by-nc-sa},version={4.0},]{doclicense}

\theoremstyle{definition}
\newtheorem{example}{Example}
\newcommand{\Desmos}{\href{https://www.desmos.com/}{Desmos}}
\setlength{\parindent}{0em}
\setlist{itemsep=0em}
\setlength{\parskip}{0.1em}

% This document is used for ordering of lessons.  If an instructor wishes to change the ordering of assessments, the following steps must be taken:

% 1) Reassign the appropriate numbers for each lesson in the \setcounter commands included in this file.
% 2) Rearrange the \include commands in the master file (the file with 'Course Pack' in the name) to accurately reflect the changes.  
% 3) Rarrange the \items in the measureable_outcomes file to accurately reflect the changes.  Be mindful of page breaks when moving items.
% 4) Re-build all affected files (master file, measureable_outcomes file, and any lessons whose numbering has changed).

%Note: The placement of each \newcounter and \setcounter command reflects the original/default ordering of topics (linears, systems, quadratics, functions, polynomials, rationals).

\newcounter{lesson_solving_linear_equations}
\newcounter{lesson_equations_containing_absolute_values}
\newcounter{lesson_graphing_lines}
\newcounter{lesson_two_forms_of_a_linear_equation}
\newcounter{lesson_parallel_and_perpendicular_lines}
\newcounter{lesson_linear_inequalities}
\newcounter{lesson_compound_inequalities}
\newcounter{lesson_inequalities_containing_absolute_values}
\newcounter{lesson_graphing_systems}
\newcounter{lesson_substitution}
\newcounter{lesson_elimination}
\newcounter{lesson_quadratics_introduction}
\newcounter{lesson_factoring_GCF}
\newcounter{lesson_factoring_grouping}
\newcounter{lesson_factoring_trinomials_a_is_1}
\newcounter{lesson_factoring_trinomials_a_neq_1}
\newcounter{lesson_solving_by_factoring}
\newcounter{lesson_square_roots}
\newcounter{lesson_i_and_complex_numbers}
\newcounter{lesson_vertex_form_and_graphing}
\newcounter{lesson_solve_by_square_roots}
\newcounter{lesson_extracting_square_roots}
\newcounter{lesson_the_discriminant}
\newcounter{lesson_the_quadratic_formula}
\newcounter{lesson_quadratic_inequalities}
\newcounter{lesson_functions_and_relations}
\newcounter{lesson_evaluating_functions}
\newcounter{lesson_finding_domain_and_range_graphically}
\newcounter{lesson_fundamental_functions}
\newcounter{lesson_finding_domain_algebraically}
\newcounter{lesson_solving_functions}
\newcounter{lesson_function_arithmetic}
\newcounter{lesson_composite_functions}
\newcounter{lesson_inverse_functions_definition_and_HLT}
\newcounter{lesson_finding_an_inverse_function}
\newcounter{lesson_transformations_translations}
\newcounter{lesson_transformations_reflections}
\newcounter{lesson_transformations_scalings}
\newcounter{lesson_transformations_summary}
\newcounter{lesson_piecewise_functions}
\newcounter{lesson_functions_containing_absolute_values}
\newcounter{lesson_absolute_as_piecewise}
\newcounter{lesson_polynomials_introduction}
\newcounter{lesson_sign_diagrams_polynomials}
\newcounter{lesson_factoring_quadratic_type}
\newcounter{lesson_factoring_summary}
\newcounter{lesson_polynomial_division}
\newcounter{lesson_synthetic_division}
\newcounter{lesson_end_behavior_polynomials}
\newcounter{lesson_local_behavior_polynomials}
\newcounter{lesson_rational_root_theorem}
\newcounter{lesson_polynomials_graphing_summary}
\newcounter{lesson_polynomial_inequalities}
\newcounter{lesson_rationals_introduction_and_terminology}
\newcounter{lesson_sign_diagrams_rationals}
\newcounter{lesson_horizontal_asymptotes}
\newcounter{lesson_slant_and_curvilinear_asymptotes}
\newcounter{lesson_vertical_asymptotes}
\newcounter{lesson_holes}
\newcounter{lesson_rationals_graphing_summary}

\setcounter{lesson_solving_linear_equations}{1}
\setcounter{lesson_equations_containing_absolute_values}{2}
\setcounter{lesson_graphing_lines}{3}
\setcounter{lesson_two_forms_of_a_linear_equation}{4}
\setcounter{lesson_parallel_and_perpendicular_lines}{5}
\setcounter{lesson_linear_inequalities}{6}
\setcounter{lesson_compound_inequalities}{7}
\setcounter{lesson_inequalities_containing_absolute_values}{8}
\setcounter{lesson_graphing_systems}{9}
\setcounter{lesson_substitution}{10}
\setcounter{lesson_elimination}{11}
\setcounter{lesson_quadratics_introduction}{16}
\setcounter{lesson_factoring_GCF}{17}
\setcounter{lesson_factoring_grouping}{18}
\setcounter{lesson_factoring_trinomials_a_is_1}{19}
\setcounter{lesson_factoring_trinomials_a_neq_1}{20}
\setcounter{lesson_solving_by_factoring}{21}
\setcounter{lesson_square_roots}{22}
\setcounter{lesson_i_and_complex_numbers}{23}
\setcounter{lesson_vertex_form_and_graphing}{24}
\setcounter{lesson_solve_by_square_roots}{25}
\setcounter{lesson_extracting_square_roots}{26}
\setcounter{lesson_the_discriminant}{27}
\setcounter{lesson_the_quadratic_formula}{28}
\setcounter{lesson_quadratic_inequalities}{29}
\setcounter{lesson_functions_and_relations}{12}
\setcounter{lesson_evaluating_functions}{13}
\setcounter{lesson_finding_domain_and_range_graphically}{14}
\setcounter{lesson_fundamental_functions}{15}
\setcounter{lesson_finding_domain_algebraically}{30}
\setcounter{lesson_solving_functions}{31}
\setcounter{lesson_function_arithmetic}{32}
\setcounter{lesson_composite_functions}{33}
\setcounter{lesson_inverse_functions_definition_and_HLT}{34}
\setcounter{lesson_finding_an_inverse_function}{35}
\setcounter{lesson_transformations_translations}{36}
\setcounter{lesson_transformations_reflections}{37}
\setcounter{lesson_transformations_scalings}{38}
\setcounter{lesson_transformations_summary}{39}
\setcounter{lesson_piecewise_functions}{40}
\setcounter{lesson_functions_containing_absolute_values}{41}
\setcounter{lesson_absolute_as_piecewise}{42}
\setcounter{lesson_polynomials_introduction}{43}
\setcounter{lesson_sign_diagrams_polynomials}{44}
\setcounter{lesson_factoring_quadratic_type}{46}
\setcounter{lesson_factoring_summary}{45}
\setcounter{lesson_polynomial_division}{47}
\setcounter{lesson_synthetic_division}{48}
\setcounter{lesson_end_behavior_polynomials}{49}
\setcounter{lesson_local_behavior_polynomials}{50}
\setcounter{lesson_rational_root_theorem}{51}
\setcounter{lesson_polynomials_graphing_summary}{52}
\setcounter{lesson_polynomial_inequalities}{53}
\setcounter{lesson_rationals_introduction_and_terminology}{54}
\setcounter{lesson_sign_diagrams_rationals}{55}
\setcounter{lesson_horizontal_asymptotes}{56}
\setcounter{lesson_slant_and_curvilinear_asymptotes}{57}
\setcounter{lesson_vertical_asymptotes}{58}
\setcounter{lesson_holes}{59}
\setcounter{lesson_rationals_graphing_summary}{60}
\begin{document}
{\bf \Large Measurable Outcomes}
\hfill \doclicenseImage[imagewidth=5em]\\
\par
Below is a comprehensive list of the anticipated measurable outcomes and some essential prerequisite skills needed for successful completion of the College Algebra course.  This list is based off of the course description and exit list topics of MATH 123 College Algebra at Framingham State University.  Each outcome number aligns to its respective lesson in the accompanying course pack.
\begin{enumerate}
	\item[\arabic{lesson_solving_linear_equations}] Solve general linear equations with variables on both sides of the equation.\\
	\item[\arabic{lesson_equations_containing_absolute_values}] Solve an equation that contains one or more absolute value(s).\\
	\item[\arabic{lesson_graphing_lines}] Graph a linear equation by creating a table of values for $x$.\\  Identify the slope of a linear equation both graphically and algebraically.\\
	\item[\arabic{lesson_two_forms_of_a_linear_equation}] Write the equation of a line in slope-intercept and point-slope form.\\
	\item[\arabic{lesson_parallel_and_perpendicular_lines}] Write the equation of a line given a line parallel or perpendicular.\\
%	\item[\arabic{lesson_}] Solve linear application problems involving consecutive integers, geometry, age, and distance, rate, and time.\\
	\item[\arabic{lesson_linear_inequalities}] Solve, graph, and give interval notation for the solution to a linear inequality.\\  Create a sign diagram to identify those intervals where a linear expression is positive or negative.\\
	\item[\arabic{lesson_compound_inequalities}] Solve, graph, and give interval notation to the solution of a compound inequality.\\
	\item[\arabic{lesson_inequalities_containing_absolute_values}] Solve, graph, and give interval notation to the solution of an inequality containing absolute values.\\
	\item[\arabic{lesson_graphing_systems}] Solve linear systems by graphing.\\
	\item[\arabic{lesson_substitution}] Solve linear systems by substitution.\\
	\item[\arabic{lesson_elimination}] Solve linear systems by addition and elimination.\\
	\item[\arabic{lesson_functions_and_relations}] Define a relation and a function; determine if a relation is a function.\\
	\item[\arabic{lesson_evaluating_functions}] Evaluate functions using appropriate notation.\\
	\item[\arabic{lesson_finding_domain_and_range_graphically}] Find the domain and range of a function from its graph.\\
\newpage
	\item[\arabic{lesson_fundamental_functions}] Graph and identify the domain, range, and intercepts of any of the ten fundamental functions.\\
	\item[\arabic{lesson_quadratics_introduction}] Recognize a quadratic equation in both form and graphically.\\
	\item[\arabic{lesson_factoring_GCF}] Find the greatest common factor (GCF) and factor it out of an expression.\\
	\item[\arabic{lesson_factoring_grouping}] Factor a tetranomial (four-term) expression by grouping.\\
	\item[\arabic{lesson_factoring_trinomials_a_is_1}] Factor a trinomial with a leading coefficient of one.\\
	\item[\arabic{lesson_factoring_trinomials_a_neq_1}] Factor a trinomial with a leading coefficient of $a\neq 1$.\\
	\item[\arabic{lesson_solving_by_factoring}] Solve polynomial equations by factoring and using the Zero Factor Property.\\
	\item[\arabic{lesson_square_roots}] Simplify and evaluate expressions involving square roots.\\
	\item[\arabic{lesson_i_and_complex_numbers}] Simplify expressions involving complex numbers.\\
	\item[\arabic{lesson_vertex_form_and_graphing}] Graph quadratic equations in both standard and vertex forms.\\
	\item[\arabic{lesson_solve_by_square_roots}] Solve quadratic equations of the form $ax^2+c=0$ by introducing a square root.\\
	\item[\arabic{lesson_extracting_square_roots}] Solve quadratic equations using the method of extracting square roots.\\
	\item[\arabic{lesson_the_discriminant}] Use the discriminant to determine the number of real solutions to a quadratic equation.\\
	\item[\arabic{lesson_the_quadratic_formula}] Solve quadratic equations using the Quadratic Formula.\\
	\item[\arabic{lesson_quadratic_inequalities}] Solve quadratic inequalities using a sign diagram.\\
	\item[\arabic{lesson_finding_domain_algebraically}] Find the domain of a function by algebraic methods.
\newpage
	\item[\arabic{lesson_solving_functions}] Solve functions using appropriate notation.\\
	\item[\arabic{lesson_function_arithmetic}] Add, subtract, multiply, and divide functions.\\
	\item[\arabic{lesson_composite_functions}] Construct, evaluate, and interpret composite functions.\\
	\item[\arabic{lesson_inverse_functions_definition_and_HLT}] Understand the definition of an inverse function and graphical implications.  Determine whether a function is invertible.\\  
	\item[\arabic{lesson_finding_an_inverse_function}] Find the inverse of a given function.\\
	\item[\arabic{lesson_transformations_translations}] Recognize and identify vertical and \slash or horizontal translations of a given function.\\
	\item[\arabic{lesson_transformations_reflections}] Recognize and identify reflections over the $x-$ and \slash or $y-$axis of a given function.\\
	\item[\arabic{lesson_transformations_scalings}] Recognize and identify vertical or horizontal scalings of a given function.\\
	\item[\arabic{lesson_transformations_summary}] Recognize and identify functions obtained by applying multiple transformations to a given function.\\
	\item[\arabic{lesson_piecewise_functions}] Define, evaluate, and solve piecewise functions.\\
	\item[\arabic{lesson_functions_containing_absolute_values}] Graph a variety of functions that contain an absolute value.\\
	\item[\arabic{lesson_absolute_as_piecewise}] Interpret a function containing an absolute value as a piecewise-defined function.\\
	\item[\arabic{lesson_polynomials_introduction}] Identify key features of and classify a polynomial by degree and number of nonzero terms.\\
	\item[\arabic{lesson_sign_diagrams_polynomials}] Construct a sign diagram for a given polynomial expression.\\
	\item[\arabic{lesson_factoring_summary}] Factor a general polynomial expression using one or more of factorization methods.
\newpage
	\item[\arabic{lesson_factoring_quadratic_type}] Recognize and factor a polynomial expression of quadratic type.\\
	\item[\arabic{lesson_polynomial_division}] Apply polynomial division.\\
	\item[\arabic{lesson_synthetic_division}] Apply synthetic division.\\
	\item[\arabic{lesson_end_behavior_polynomials}] Determine the end behavior of the graph of a polynomial function.\\
	\item[\arabic{lesson_local_behavior_polynomials}] Identify all real roots and their corresponding multiplicities for a polynomial function (that is easily factorable).\\
	\item[\arabic{lesson_rational_root_theorem}] Apply the Rational Root Theorem to determine a set of possible rational roots for and a factorization of a given polynomial.\\
	\item[\arabic{lesson_polynomials_graphing_summary}] Graph a polynomial function in its entirety.\\
	\item[\arabic{lesson_polynomial_inequalities}] Solve a polynomial inequality by constructing a sign diagram.\\
	\item[\arabic{lesson_rationals_introduction_and_terminology}] Define and identify key features of rational functions.\\
	\item[\arabic{lesson_sign_diagrams_rationals}] Solve rational inequalities by constructing a sign diagram.\\
	\item[\arabic{lesson_horizontal_asymptotes}] Identify a horizontal asymptote in the graph of a rational function.\\ 
	\item[\arabic{lesson_slant_and_curvilinear_asymptotes}] Identify a slant or curvilinear asymptote in the graph of a rational function.\\
	\item[\arabic{lesson_vertical_asymptotes}] Identify one or more vertical asymptotes in the graph of a rational function.\\
	\item[\arabic{lesson_holes}] Identify the precise location of one or more holes in the graph of a rational function.\\ 
	\item[\arabic{lesson_rationals_graphing_summary}] Graph a rational function in its entirety.
\end{enumerate}
\newpage
\end{document}
\pagenumbering{arabic}
\setcounter{page}{13}
\import{./chapter_linears/}{chapter_linears}
\import{./chapter_systems/}{chapter_systems}
\import{./chapter_functions_intro/}{chapter_functions_intro}
\import{./chapter_quadratics/}{chapter_quadratics}
\import{./chapter_functions_advanced/}{chapter_functions_advanced}
\import{./chapter_polynomials/}{chapter_polynomials}
\import{./chapter_rationals/}{chapter_rationals}
\end{document}