\documentclass[12pt]{book}
\raggedbottom
\usepackage[top=1in,left=1in,bottom=1in,right=1in,headsep=0.25in]{geometry}	
\usepackage{amssymb,amsmath,amsthm,amsfonts}
\usepackage{chapterfolder,docmute,setspace}
\usepackage{cancel,multicol,tikz,verbatim,framed,polynom,enumitem,tikzpagenodes}
\usepackage[colorlinks, hyperindex, plainpages=false, linkcolor=blue, urlcolor=blue, pdfpagelabels]{hyperref}
\usepackage[type={CC},modifier={by-sa},version={4.0},]{doclicense}

\theoremstyle{definition}
\newtheorem{example}{Example}
\newcommand{\Desmos}{\href{https://www.desmos.com/}{Desmos}}
\setlength{\parindent}{0in}
\setlist{itemsep=0in}
\setlength{\parskip}{0.1in}
\setcounter{secnumdepth}{0}
\input{lesson_order}

\newcommand{\tmmathbf}[1]{\ensuremath{\boldsymbol{#1}}}
\newcommand{\tmop}[1]{\ensuremath{\operatorname{#1}}}

\begin{document}
\section{The Two Forms of a Linear Equation (L\arabic{lesson_two_forms_of_a_linear_equation})}
\subsection{Slope-Intercept Form}
{\bf Objective: Find the equation of a line with a known slope and $y$-intercept.}\par
When graphing a line we found one method we could use is to make a table of values. However, if we can identify some properties of the line, we may be able to make a graph much quicker and easier. One such method is finding the slope and the $y$-intercept of the equation. The slope can be represented by $m$ and the $y$-intercept, where it crosses the axis and $x = 0$, can be represented by $(0, b)$ where $b$ is the value where the graph crosses the vertical $y$-axis. Any other point on the line can be represented by $(x, y)$. Using this information we will look at the slope formula and solve the formula for $y$.
\begin{example}\label{Lin55}
  \begin{eqnarray*}
    m,~ (0, b),~ (x, y) &  & \tmop{Use} \tmop{the} \tmop{slope} \tmop{formula}\\
    \frac{y - b}{x - 0} = m &  & \tmop{Simplify}\\
    \frac{y - b}{x} = m &  & \tmop{Multiply} \tmop{both} \tmop{sides}
    \tmop{by} x\\
    y - b = m x &  & \tmop{Add} b \tmop{to} \tmop{both} \tmop{sides}\\
    \tmmathbf{\underline{+ b ~~~+ b}} &  & \\
    y = m x + b &  & \tmop{Our} \tmop{solution}
  \end{eqnarray*}
\end{example}
This equation, $y = m x + b$ can be thought of as the equation of any line that has a slope of $m$ and a $y$-intercept of $b$. This formula is known as the slope-intercept formula or equation.\par
\begin{center}
\framebox{
\begin{minipage}{0.8\linewidth}
\begin{center}
 Slope-intercept equation: $y = m x + b$
\end{center}
\end{minipage}
}
\end{center}
If we know the slope and the $y$-intercept we can easily find the equation that represents the line.
\begin{example}\label{Lin56}
  \begin{eqnarray*}
    \tmop{Slope} = \frac{3}{4},~~ y - \tmop{intercept} = - 3 &  & \tmop{Use}
    \tmop{the} \tmop{slope} - \tmop{intercept} \tmop{equation}\\
    y = m x + b &  & m \tmop{is} \tmop{the} \tmop{slope},~~ b \tmop{is}
    \tmop{the} y - \tmop{intercept}\\
    y = \frac{3}{4} x - 3 &  & \tmop{Our} \tmop{solution}
  \end{eqnarray*}
\end{example}
We can also find the equation by looking at a graph and finding the slope and $y$-intercept.
\begin{multicols}{2}
 	\begin{tikzpicture}[xscale=0.5,yscale=0.5]
		\draw[step=1.0,gray,very thin,dotted] (-5.5,-5.5) grid (5.5,5.5);
		\draw [<->](-5.5,0) -- coordinate (x axis mid) (5.5,0) node[below right] {$x$};
		\draw [<->](0,-5.5) -- coordinate (y axis mid) (0,5.5) node[above right] {$y$};
		\draw [->,dashed,line width=0.25mm](0,3) -- (0,1);
		\draw [->,dashed,line width=0.25mm](0,1) -- (3,1);
		\foreach \x in {1,...,5}
		\draw (\x,2pt) -- (\x,-2pt)	node[anchor=south] {\scriptsize \x};
		\foreach \x in {-1,...,-5}
		\draw (\x,2pt) -- (\x,-2pt)	node[anchor=north] {\scriptsize \x};
		\foreach \y in {1,...,5}
		\draw (2pt,\y) -- (-2pt,\y)	node[anchor=east] {\scriptsize \y}; 
		\foreach \y in {-1,...,-5}
		\draw[fill] (2pt,\y) -- (-2pt,\y)	node[anchor=west] {\scriptsize \y}; 
		\draw [<->] plot [domain=-3.25:5, samples=100] (\x,{-0.667*(\x)+3});
		\draw[fill] (0,3) circle (0.1) node[right] {};
		\draw[fill] (3,1)circle (0.1) node[right] {};
	\end{tikzpicture}

	\columnbreak

	\ \par
	Identify the point where the graph crosses the $y$-axis (0,3).\par
	This means the $y$-intercept is 3.\par
    Identify one other point and draw a slope triangle to find the slope.\par
    The slope is $m=- \frac{2}{3}$.\par
	Slope-intercept form: $y=mx+b$\par
	Our Equation: $y = - \frac{2}{3} x + 3$
\end{multicols}
We can also move the opposite direction, using the equation identify the slope and $y$-intercept and graph the equation from this information. However, it will be important for the equation to first be in slope intercept form. If it is not, we will have to solve it for $y$ so we can identify the slope and the $y$-intercept.
\begin{example}\label{Lin58}~~~Write the equation $2x-4y=6$ in slope-intercept form.
  \begin{eqnarray*}
    2 x - 4 y = 6~~ &  & \tmop{Solve} \tmop{for} y\\
    \tmmathbf{\underline{- 2 x ~~~~- 2 x}} &  & \tmop{Subtract} 2 x \tmop{from} \tmop{both}
    \tmop{sides}\\
    - 4 y = - 2 x + 6 &  & \tmop{Put} x \tmop{term} \tmop{first}\\
    \tmmathbf{\overline{- 4} ~~~~ \overline{- 4} ~~ \overline{- 4}} &  & \tmop{Divide~}
    \tmop{each} \tmop{term} \tmop{by} - 4\\
    y = \frac{1}{2} x - \frac{3}{2} &  & \tmop{Our} \tmop{solution}
  \end{eqnarray*}
\end{example}
Once we have an equation in slope-intercept form we can graph it by first plotting the $y$-intercept, then using the slope, finding a second point and connecting the dots.
\begin{example}\label{Lin59}~~~Graph $y=\displaystyle\frac{1}{2}x-4$.
  \begin{eqnarray*}
    y = m x + b&  & \tmop{Slope} - \tmop{intercept} \tmop{equation}\\
    m = \frac{1}{2},~ b = - 4 &  & \tmop{Identify} \tmop{the} \tmop{slope},~ m, \tmop{and} \tmop{the} y - \tmop{intercept},~ b
  \end{eqnarray*}
\newpage
\begin{multicols}{2}
 	\begin{tikzpicture}[xscale=0.5,yscale=0.5]
		\draw[step=1.0,gray,very thin,dotted] (-1.5,-5.5) grid (10.5,5.5);
		\draw [<->](-1.5,0) -- coordinate (x axis mid) (10.5,0) node[below right] {$x$};
		\draw [<->](0,-5.5) -- coordinate (y axis mid) (0,5.5) node[above right] {$y$};
		\foreach \x in {1,...,10}
		\draw (\x,2pt) -- (\x,-2pt)	node[anchor=south] {\scriptsize \x};
		\foreach \y in {1,...,5}
		\draw (2pt,\y) -- (-2pt,\y)	node[anchor=east] {\scriptsize \y}; 
		\foreach \y in {-1,...,-5}
		\draw[fill] (2pt,\y) -- (-2pt,\y)	node[anchor=west] {\scriptsize \y}; 
		\draw [<->] plot [domain=-1:8.5, samples=100] (\x,{0.5*(\x)-4});
		\draw[fill] (0,-4) circle (0.1) node[right] {};
		\draw[fill] (2,-3)circle (0.1) node[right] {};
		\draw[fill] (8,0)circle (0.1) node[right] {};
	\end{tikzpicture}
    
    \columnbreak
    
	\ \par
	Start with a point at the $y$-intercept of $- 4$, $(0,-4)$.\par
	Then use the slope $\frac{\text{rise}}{\text{run}}$ to find the next point, $(2,-3)$.\par
	Once we have both points, connect the dots to get our graph.\par
	Here, we have also identified the $x-$intercept $(8,0)$, by setting $y=0$ and solving for $x$:\par 
	$\frac{1}{2}x-4=0$ when $x=8$.
  \end{multicols}
\end{example}
\begin{example}\label{Lin60}~~~Graph $3x+4y=12$.
  \begin{eqnarray*}
    3 x + 4 y = 12~~ &  & \tmop{Not} \tmop{in} \tmop{slope-intercept} \tmop{form}\\
    \tmmathbf{\underline{- 3 x ~~~~~~- 3 x}} &  & \tmop{Subtract} 3 x \tmop{from} \tmop{both}
    \tmop{sides}\\
    4 y = - 3 x + 12 &  & \tmop{Put} \tmop{the} x \tmop{term} \tmop{first}\\
    \tmmathbf{\overline{4} ~~~~~~~ \overline{4} ~~~~~ \overline{4}}~ &  & \tmop{Divide} \tmop{each}
    \tmop{term} \tmop{by} 4\\
    y = - \frac{3}{4} x + 3 &  & \tmop{Now~in} \tmop{slope} - \tmop{intercept}
    \tmop{form}\\
    m = - \frac{3}{4},~ b = 3 &  & \tmop{Identify} m \tmop{and} b
  \end{eqnarray*}
\end{example}
\begin{multicols}{2}
 	\begin{tikzpicture}[xscale=0.5,yscale=0.5]
		\draw[step=1.0,gray,very thin,dotted] (-5.5,-5.5) grid (5.5,5.5);
		\draw [<->](-5.5,0) -- coordinate (x axis mid) (5.5,0) node[below right] {$x$};
		\draw [<->](0,-5.5) -- coordinate (y axis mid) (0,5.5) node[above right] {$y$};
		\foreach \x in {1,...,5}
		\draw (\x,2pt) -- (\x,-2pt)	node[anchor=south] {\scriptsize \x};
		\foreach \x in {-1,...,-5}
		\draw (\x,2pt) -- (\x,-2pt)	node[anchor=north] {\scriptsize \x};
		\foreach \y in {1,...,5}
		\draw (2pt,\y) -- (-2pt,\y)	node[anchor=east] {\scriptsize \y}; 
		\foreach \y in {-1,...,-5}
		\draw[fill] (2pt,\y) -- (-2pt,\y)	node[anchor=west] {\scriptsize \y}; 
		\draw [<->] plot [domain=-1.5:5.5, samples=100] (\x,{-0.75*(\x)+3});
		\draw[fill] (0,3) circle (0.1) node[right] {};
		\draw[fill] (4,0)circle (0.1) node[right] {};
	\end{tikzpicture}
 
	\columnbreak
	
	\ \par
	Start with a point at the $y-$intercept, $(0,3)$.\par
	Then use the slope $\frac{\text{rise}}{\text{run}}$.  Since the slope is negative, the graph will decrease from left to right.  So we will drop 3 units and run 4 units to the right to find the next point.\par 
	Notice that our next point is also the $x-$intercept, $(4,0)$.\par
	Once we have both points, connect the dots to get our graph.
\end{multicols}
We want to be very careful not to confuse using slope to find the next point with use a coordinate such as $(4, - 2)$ to find an individual point.  Coordinates such as $(4, - 2)$ start from the origin and move horizontally first, and vertically second. Slope starts from a point on the line that could be anywhere on the graph. The numerator is the vertical change and the denominator is the horizontal change.\par
Lines with zero slope or no slope can make a problem seem very different.  Such lines are either horizontal ($m=0$) or vertical ($m$ is undefined).\par
A horizontal line will have a slope of zero which when multiplied by $x$ gives zero. So the equation simply becomes $y = 0x+b$ or just $y=b.$ Remember that in this case, $b$ also refers to where the line crosses the $y-$axis.\par 
If we have no slope ($m=\varnothing$), our line is vertical, and the corresponding equation cannot be written in slope-intercept form.  In this case, there is no $y$ in our equation, and we simply write $x=a,$ where $a$ refers to the $x-$coordinate for the point where the line crosses the $x-$axis.
\begin{multicols}{2}
 	\begin{tikzpicture}[xscale=0.5,yscale=0.5]
		\draw[step=1.0,gray,very thin,dotted] (-5.5,-5.5) grid (5.5,5.5);
		\draw [<->](-5.5,0) -- coordinate (x axis mid) (5.5,0) node[below right] {$x$};
		\draw [<->](0,-5.5) -- coordinate (y axis mid) (0,5.5) node[above right] {$y$};
		\foreach \x in {1,...,5}
		\draw (\x,2pt) -- (\x,-2pt)	node[anchor=south] {\scriptsize \x};
		\foreach \x in {-1,...,-5}
		\draw (\x,2pt) -- (\x,-2pt)	node[anchor=north] {\scriptsize \x};
		\foreach \y in {1,...,5}
		\draw (2pt,\y) -- (-2pt,\y)	node[anchor=west] {\scriptsize \y}; 
		\foreach \y in {-1,...,-5}
		\draw[fill] (2pt,\y) -- (-2pt,\y)	node[anchor=east] {\scriptsize \y}; 
		\draw [<->] plot [domain=-4.5:4.5, samples=100] (-4,\x);
		\draw[fill] (-4,0)circle (0.1) node[right] {};
	\end{tikzpicture}
	
\columnbreak

	\ \par
In this graph, because we have a vertical line ($m$ is undefined), we do not use the slope-intercept form of a linear equation.\par
Rather, we set $x$ equal to the $x-$coordinate of the $x-$intercept.\par 
Our corresponding equation is $x=- 4$.
  \end{multicols}
\subsection{Point-Slope Form}
{\bf Objective: Find the equation of a line with a known slope and passing through a given point.}\par
The slope-intercept form has the advantage of being simple to remember and use, however, it has one major disadvantage: we must know the $y$-intercept in order to use it! Generally we do not know the y-intercept, we only know one or more points (that are not the $y$-intercept). In these cases we can't use the slope intercept equation, so we will use a different, more flexible formula. If we let the slope of an equation be $m$, and a specific point on the line be $(x_1, y_1)$, and any other point on the line be $(x, y)$. We can use the slope formula to make a second equation.
\begin{example}\label{Lin62}
  \begin{eqnarray*}
    m,~ (x_1, y_1),~ (x, y) &  & \tmop{Recall} \tmop{slope} \tmop{formula}\\
    \frac{y_2 - y_1}{x_2 - x_1} = m &  & \tmop{Plug} \tmop{in} \tmop{values}\\
    \frac{y - y_1}{x - x_1} = m &  & \tmop{Multiply} \tmop{both} \tmop{sides}
    \tmop{by} (x - x_1)\\
    y - y_1 = m (x - x_1) &  & \tmop{Our} \tmop{equation}
  \end{eqnarray*}
\end{example}
If we know the slope, $m$ of an equation and any point on the line $(x_1,y_1)$ we can easily plug these values into the equation above which will be called the point-slope formula or equation.
\begin{center}
\framebox{
\begin{minipage}{0.8\linewidth}
\begin{center}
 Point-slope equation: $y - y_1 = m (x - x_1)$
\end{center}
\end{minipage}
}
\end{center}
\begin{example}\label{Lin63} Write the equation of the line through the point $(3, - 4)$ with a slope of $\displaystyle\frac{3}{5}$.
  \begin{eqnarray*}
    y - y_1 = m (x - x_1) &  & \tmop{Plug} \tmop{values} \tmop{into}
    \tmop{point} - \tmop{slope} \tmop{formula}\\
    y - (- 4) = \frac{3}{5} (x - 3) &  & \tmop{Simplify} \tmop{signs}\\
    y + 4 = \frac{3}{5} (x - 3) &  & \tmop{Our} \tmop{solution}
  \end{eqnarray*}
\end{example}
Often, we will prefer final answers be written in slope-intercept form. If the directions ask for the answer in slope-intercept form we will simply distribute the slope, then solve for $y$.

\begin{example}\label{Lin64}\ \par
Write the equation of the line through the point $(- 6, 2)$ with a slope of $- \displaystyle\frac{2}{3}$ in slope-intercept form.
  \begin{eqnarray*}
    y - y_1 = m (x - x_1) &  & \tmop{Plug} \tmop{values} \tmop{into}
    \tmop{point} - \tmop{slope} \tmop{formula}\\
    y - 2 = - \frac{2}{3} \left(x - (- 6)\right) &  & \tmop{Simplify} \tmop{signs}\\
    y - 2 = - \frac{2}{3} (x + 6) &  & \tmop{Distribute~} \tmop{slope}\\
    y - 2 = - \frac{2}{3} x - 4 &  & \tmop{Solve} \tmop{for} y \tmop{by~adding~2~to~both~sides}\\
    \tmmathbf{\underline{+ 2 ~~~~~~~~ + 2}} &  & \\
    y = - \frac{2}{3} x - 2 &  & \tmop{Our} \tmop{solution}
  \end{eqnarray*}
\end{example}
An important thing to observe about the point slope formula is that the operation between the $x$'s and $y$'s is subtraction. This means when you simplify the signs you will have the opposite of the numbers in the point. We need to be very careful with signs as we use the point-slope formula.\par
In order to find the equation of a line we will always need to know the slope. If we don't know the slope to begin with we will have to do some work to find it first before we can get an equation.
\begin{example}\label{Lin65} Find the equation of the line through the points $(- 2, 5)$ and $(4, -3)$.
  \begin{eqnarray*}
    m = \frac{y_2 - y_1}{x_2 - x_1} &  & \tmop{First} \tmop{we} \tmop{must}
    \tmop{find} \tmop{the} \tmop{slope}\\
    m = \frac{- 3 - 5}{4 - (- 2)} = \frac{- 8}{6} = - \frac{4}{3} &  &
    \tmop{Plug} \tmop{values} \tmop{in} \tmop{slope} \tmop{formula} \tmop{and}
    \tmop{evaluate}\\
    y - y_1 = m (x - x_1) &  & \tmop{Use} \tmop{point} - \tmop{slope}
    \tmop{formula},\\
		& & \tmop{~~~plugging~in~slope~and~either~point}\\
  \end{eqnarray*}
  \begin{eqnarray*}
    y - 5 = - \frac{4}{3} (x - (- 2)) &  & \tmop{Simplify} \tmop{signs}\\
    y - 5 = - \frac{4}{3} (x + 2) &  & \tmop{Our} \tmop{solution}
  \end{eqnarray*}
\end{example}

\begin{example}\label{Lin66}\ \par
Find the equation of the line through the points $(- 3, 4)$ and $(- 1,- 2)$ in slope-intercept form.
  \begin{eqnarray*}
    m = \frac{y_2 - y_1}{x_2 - x_1} &  & \tmop{First} \tmop{we} \tmop{must}
    \tmop{find} \tmop{the} \tmop{slope}\\
    m = \frac{- 2 - 4}{- 1 - (- 3)} = \frac{- 6}{2} = - 3 &  & \tmop{Plug}
    \tmop{values} \tmop{in} \tmop{slope} \tmop{formula} \tmop{and}
    \tmop{evaluate}\\
    y - y_1 = m (x - x_1) &  & \tmop{Use} \tmop{point} - \tmop{slope} \tmop{formula},\\
		& &  \tmop{~~~plugging~in~slope~and~either~point}\\
    y - 4 = - 3 (x - (- 3)) &  & \tmop{Simplify} \tmop{signs}\\
    y - 4 = - 3 (x + 3) &  & \tmop{Distribute~} \tmop{slope}\\
    y - 4 = - 3 x - 9 &  & \tmop{Solve} \tmop{for} y \\
    \tmmathbf{\underline{+ 4 ~~~~~~~~~+ 4}} &  & \tmop{Add} 4 \tmop{to} \tmop{both} \tmop{sides}\\
    y = - 3 x - 5 &  & \tmop{Our} \tmop{solution}
  \end{eqnarray*}
\end{example}
\begin{example}\label{Lin67}\ \par
Find the equation of the line through the points $(6, - 2)$ and $(- 4, 1)$ in slope-intercept form.
  \begin{eqnarray*}
    m = \frac{y_2 - y_1}{x_2 - x_1} &  & \tmop{First} \tmop{we} \tmop{must}
    \tmop{find} \tmop{the} \tmop{slope}\\
    m = \frac{1 - (- 2)}{- 4 - 6} = \frac{3}{- 10} = - \frac{3}{10} &  &
    \tmop{Plug} \tmop{values} \tmop{into} \tmop{slope} \tmop{formula}
    \tmop{and} \tmop{evaluate}\\
  \end{eqnarray*}
  \begin{eqnarray*}
    y - y_1 = m (x - x_1) &  & \tmop{Use} \tmop{point} - \tmop{slope}
    \tmop{formula},\\
		& &  \tmop{~~~plugging~in~slope~and~either~point}\\
    y - (- 2) = - \frac{3}{10} (x - 6) &  & \tmop{Simplify} \tmop{signs}\\
    y + 2 = - \frac{3}{10} (x - 6) &  & \tmop{Distribute} \tmop{slope~}\\
    y + 2 = - \frac{3}{10} x + \frac{9}{5} &  & \tmop{Solve} \tmop{for} y,
    \tmop{by~subtracting} 2 \tmop{from} \tmop{both} \tmop{sides}\\
    \tmmathbf{\underline{- 2 ~~~~~~~~~- \frac{10}{5}}} &  & \tmop{Use} \frac{10}{5} \tmop{on}
    \tmop{right} \tmop{so} \tmop{we} \tmop{have} \tmop{a} \tmop{common}
    \tmop{denominator}\\
    & & \\
		y = - \frac{3}{10} x - \frac{1}{5} &  & \tmop{Our} \tmop{solution}
  \end{eqnarray*}
\end{example}
\newpage
\end{document}