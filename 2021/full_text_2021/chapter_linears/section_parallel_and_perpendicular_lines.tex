\documentclass[12pt]{book}
\raggedbottom
\usepackage[top=1in,left=1in,bottom=1in,right=1in,headsep=0.25in]{geometry}	
\usepackage{amssymb,amsmath,amsthm,amsfonts}
\usepackage{chapterfolder,docmute,setspace}
\usepackage{cancel,multicol,tikz,verbatim,framed,polynom,enumitem,tikzpagenodes}
\usepackage[colorlinks, hyperindex, plainpages=false, linkcolor=blue, urlcolor=blue, pdfpagelabels]{hyperref}
\usepackage[type={CC},modifier={by-sa},version={4.0},]{doclicense}

\theoremstyle{definition}
\newtheorem{example}{Example}
\newcommand{\Desmos}{\href{https://www.desmos.com/}{Desmos}}
\setlength{\parindent}{0in}
\setlist{itemsep=0in}
\setlength{\parskip}{0.1in}
\setcounter{secnumdepth}{0}
% This document is used for ordering of lessons.  If an instructor wishes to change the ordering of assessments, the following steps must be taken:

% 1) Reassign the appropriate numbers for each lesson in the \setcounter commands included in this file.
% 2) Rearrange the \include commands in the master file (the file with 'Course Pack' in the name) to accurately reflect the changes.  
% 3) Rarrange the \items in the measureable_outcomes file to accurately reflect the changes.  Be mindful of page breaks when moving items.
% 4) Re-build all affected files (master file, measureable_outcomes file, and any lessons whose numbering has changed).

%Note: The placement of each \newcounter and \setcounter command reflects the original/default ordering of topics (linears, systems, quadratics, functions, polynomials, rationals).

\newcounter{lesson_solving_linear_equations}
\newcounter{lesson_equations_containing_absolute_values}
\newcounter{lesson_graphing_lines}
\newcounter{lesson_two_forms_of_a_linear_equation}
\newcounter{lesson_parallel_and_perpendicular_lines}
\newcounter{lesson_linear_inequalities}
\newcounter{lesson_compound_inequalities}
\newcounter{lesson_inequalities_containing_absolute_values}
\newcounter{lesson_graphing_systems}
\newcounter{lesson_substitution}
\newcounter{lesson_elimination}
\newcounter{lesson_quadratics_introduction}
\newcounter{lesson_factoring_GCF}
\newcounter{lesson_factoring_grouping}
\newcounter{lesson_factoring_trinomials_a_is_1}
\newcounter{lesson_factoring_trinomials_a_neq_1}
\newcounter{lesson_solving_by_factoring}
\newcounter{lesson_square_roots}
\newcounter{lesson_i_and_complex_numbers}
\newcounter{lesson_vertex_form_and_graphing}
\newcounter{lesson_solve_by_square_roots}
\newcounter{lesson_extracting_square_roots}
\newcounter{lesson_the_discriminant}
\newcounter{lesson_the_quadratic_formula}
\newcounter{lesson_quadratic_inequalities}
\newcounter{lesson_functions_and_relations}
\newcounter{lesson_evaluating_functions}
\newcounter{lesson_finding_domain_and_range_graphically}
\newcounter{lesson_fundamental_functions}
\newcounter{lesson_finding_domain_algebraically}
\newcounter{lesson_solving_functions}
\newcounter{lesson_function_arithmetic}
\newcounter{lesson_composite_functions}
\newcounter{lesson_inverse_functions_definition_and_HLT}
\newcounter{lesson_finding_an_inverse_function}
\newcounter{lesson_transformations_translations}
\newcounter{lesson_transformations_reflections}
\newcounter{lesson_transformations_scalings}
\newcounter{lesson_transformations_summary}
\newcounter{lesson_piecewise_functions}
\newcounter{lesson_functions_containing_absolute_values}
\newcounter{lesson_absolute_as_piecewise}
\newcounter{lesson_polynomials_introduction}
\newcounter{lesson_sign_diagrams_polynomials}
\newcounter{lesson_factoring_quadratic_type}
\newcounter{lesson_factoring_summary}
\newcounter{lesson_polynomial_division}
\newcounter{lesson_synthetic_division}
\newcounter{lesson_end_behavior_polynomials}
\newcounter{lesson_local_behavior_polynomials}
\newcounter{lesson_rational_root_theorem}
\newcounter{lesson_polynomials_graphing_summary}
\newcounter{lesson_polynomial_inequalities}
\newcounter{lesson_rationals_introduction_and_terminology}
\newcounter{lesson_sign_diagrams_rationals}
\newcounter{lesson_horizontal_asymptotes}
\newcounter{lesson_slant_and_curvilinear_asymptotes}
\newcounter{lesson_vertical_asymptotes}
\newcounter{lesson_holes}
\newcounter{lesson_rationals_graphing_summary}

\setcounter{lesson_solving_linear_equations}{1}
\setcounter{lesson_equations_containing_absolute_values}{2}
\setcounter{lesson_graphing_lines}{3}
\setcounter{lesson_two_forms_of_a_linear_equation}{4}
\setcounter{lesson_parallel_and_perpendicular_lines}{5}
\setcounter{lesson_linear_inequalities}{6}
\setcounter{lesson_compound_inequalities}{7}
\setcounter{lesson_inequalities_containing_absolute_values}{8}
\setcounter{lesson_graphing_systems}{9}
\setcounter{lesson_substitution}{10}
\setcounter{lesson_elimination}{11}
\setcounter{lesson_quadratics_introduction}{16}
\setcounter{lesson_factoring_GCF}{17}
\setcounter{lesson_factoring_grouping}{18}
\setcounter{lesson_factoring_trinomials_a_is_1}{19}
\setcounter{lesson_factoring_trinomials_a_neq_1}{20}
\setcounter{lesson_solving_by_factoring}{21}
\setcounter{lesson_square_roots}{22}
\setcounter{lesson_i_and_complex_numbers}{23}
\setcounter{lesson_vertex_form_and_graphing}{24}
\setcounter{lesson_solve_by_square_roots}{25}
\setcounter{lesson_extracting_square_roots}{26}
\setcounter{lesson_the_discriminant}{27}
\setcounter{lesson_the_quadratic_formula}{28}
\setcounter{lesson_quadratic_inequalities}{29}
\setcounter{lesson_functions_and_relations}{12}
\setcounter{lesson_evaluating_functions}{13}
\setcounter{lesson_finding_domain_and_range_graphically}{14}
\setcounter{lesson_fundamental_functions}{15}
\setcounter{lesson_finding_domain_algebraically}{30}
\setcounter{lesson_solving_functions}{31}
\setcounter{lesson_function_arithmetic}{32}
\setcounter{lesson_composite_functions}{33}
\setcounter{lesson_inverse_functions_definition_and_HLT}{34}
\setcounter{lesson_finding_an_inverse_function}{35}
\setcounter{lesson_transformations_translations}{36}
\setcounter{lesson_transformations_reflections}{37}
\setcounter{lesson_transformations_scalings}{38}
\setcounter{lesson_transformations_summary}{39}
\setcounter{lesson_piecewise_functions}{40}
\setcounter{lesson_functions_containing_absolute_values}{41}
\setcounter{lesson_absolute_as_piecewise}{42}
\setcounter{lesson_polynomials_introduction}{43}
\setcounter{lesson_sign_diagrams_polynomials}{44}
\setcounter{lesson_factoring_quadratic_type}{46}
\setcounter{lesson_factoring_summary}{45}
\setcounter{lesson_polynomial_division}{47}
\setcounter{lesson_synthetic_division}{48}
\setcounter{lesson_end_behavior_polynomials}{49}
\setcounter{lesson_local_behavior_polynomials}{50}
\setcounter{lesson_rational_root_theorem}{51}
\setcounter{lesson_polynomials_graphing_summary}{52}
\setcounter{lesson_polynomial_inequalities}{53}
\setcounter{lesson_rationals_introduction_and_terminology}{54}
\setcounter{lesson_sign_diagrams_rationals}{55}
\setcounter{lesson_horizontal_asymptotes}{56}
\setcounter{lesson_slant_and_curvilinear_asymptotes}{57}
\setcounter{lesson_vertical_asymptotes}{58}
\setcounter{lesson_holes}{59}
\setcounter{lesson_rationals_graphing_summary}{60}

\newcommand{\tmmathbf}[1]{\ensuremath{\boldsymbol{#1}}}
\newcommand{\tmop}[1]{\ensuremath{\operatorname{#1}}}

\begin{document}
\section{Parallel and Perpendicular Lines (L\arabic{lesson_parallel_and_perpendicular_lines})}
{\bf Objective: Identify the equation of a line that is either parallel or perpendicular to a given line.}\par
There is an interesting connection between the slopes of lines that are parallel, as well as the slopes of lines that are perpendicular (meet at a right angle). This is shown in the following example.
\begin{multicols}{2}
	\begin{tikzpicture}[xscale=0.4,yscale=0.4]
		\draw[step=1.0,gray,very thin,dotted] (-8.5,-5.5) grid (8.5,5.5);
		\draw [<->](-8.5,0) -- coordinate (x axis mid) (8.5,0) node[below right] {$x$};
		\draw [<->](0,-5.5) -- coordinate (y axis mid) (0,5.5) node[above right] {$y$};
		\draw [<->,line width=0.4mm] plot [domain=-4:7, samples=100] (\x,{-0.667*\x+2});
		\draw [<->,line width=0.4mm] plot [domain=-7:4, samples=100] (\x,{-0.667*\x-2.667});
		\draw[fill] (0,2) circle (0.2);
		\draw[fill] (-3,4) circle (0.2);
		\draw[fill] (-4,0) circle (0.2);
		\draw[fill] (-1,-2) circle (0.2);
		\draw (-2,4.5) node {$\ell_1$};
		\draw (-6,2.5) node {$\ell_2$};
	\end{tikzpicture}
\columnbreak
\ \par
\ \par
This graph shows two parallel lines.\\
\par
The slope (rise over run) of each line is ``down 2, right 3,'' or $m_1=m_2=-\frac{2}{3}$.
\end{multicols}

\begin{multicols}{2}
	\begin{tikzpicture}[xscale=0.4,yscale=0.4]
		\draw[step=1.0,gray,very thin,dotted] (-8.5,-5.5) grid (8.5,5.5);
		\draw [<->](-8.5,0) -- coordinate (x axis mid) (8.5,0) node[below right] {$x$};
		\draw [<->](0,-5.5) -- coordinate (y axis mid) (0,5.5) node[above right] {$y$};
		\draw [<->,line width=0.4mm] plot [domain=-8:7, samples=100] (\x,{0.667*\x+0.333});
		\draw [<->,line width=0.4mm] plot [domain=-7:-0.5, samples=100] (\x,{-1.5*\x-6});
		\draw[fill] (1,1) circle (0.2);
		\draw[fill] (-2,-1) circle (0.2);
		\draw[fill] (-4,0) circle (0.2);
		\draw[fill] (-6,3) circle (0.2);
		\draw (6,3) node {$\ell_1$};
		\draw (-4.5,2.5) node {$\ell_2$};
	\end{tikzpicture}
\columnbreak
\ \par
\ \par
This graph shows two perpendicular lines.\\
\par
The slope (rise over run) of the more gradual line is ``up 2, right 3,'' or $m_1=\frac{2}{3}$.\\
\par
The slope of the steeper line is ``down 3, right 2,'' or $m_2=-\frac{3}{2}$.
\end{multicols}

As the first graph above illustrates, parallel lines have the same slope.\par

On the other hand, perpendicular lines are said to have slopes that are {\it negative reciprocals} of one another.   More precisely, if two lines with slopes $m_1$ and $m_2$ are known to be perpendicular, then $m_2=-~\displaystyle\frac{1}{m_1}$ (and so, $m_1m_2=-1$).\par
We can use these properties to make conclusions about parallel and perpendicular lines.
\begin{example}\label{Lin69} Find the slope of a line parallel to $5 y - 2 x = 7$.
  \begin{eqnarray*}
    5 y - 2 x = 7~~~~~ &  & \tmop{To} \tmop{find} \tmop{the} \tmop{slope} \tmop{we}
    \tmop{will} \tmop{put} \tmop{equation} \tmop{in} \tmop{slope} -
    \tmop{intercept} \tmop{form}\\
    \tmmathbf{\underline{+ 2 x ~~ + 2 x}} &  & \tmop{Add} 2 x \tmop{to} \tmop{both}
    \tmop{sides}\\
    5 y = 2 x + 7~~~~~ &  & \tmop{Put} x \tmop{term} \tmop{first}\\
    \tmmathbf{\overline{5} ~~~~ \overline{5} ~~~~~ \overline{5}}~~~~~ &  & \tmop{Divide} \tmop{each}
    \tmop{term} \tmop{by} 5\\
  \end{eqnarray*}
  \begin{eqnarray*}
    y = \frac{2}{5} x + \frac{7}{5}~~~~~ &  & \tmop{The} \tmop{slope} \tmop{is}
    \tmop{the} \tmop{coefficient} \tmop{of} x\\
    m = \frac{2}{5}~~~~~ &  & \tmop{Slope} \tmop{of} \tmop{given} \tmop{line}\\
		& & \tmop{Parallel} \tmop{lines} \tmop{have} \tmop{the} \tmop{same}
    \tmop{slope}\\
    m = \frac{2}{5}~~~~~ &  & \tmop{Our} \tmop{solution}
  \end{eqnarray*}
\end{example}

\begin{example}\label{Lin70} Find the slope of a line perpendicular to $3 x - 4 y = 2$.
  \begin{eqnarray*}
    3 x - 4 y = 2~~~~ &  & \tmop{To} \tmop{find} \tmop{slope} \tmop{we}
    \tmop{will} \tmop{put} \tmop{equation} \tmop{in} \tmop{slope} -
    \tmop{intercept} \tmop{form}\\
    \tmmathbf{\underline{- 3 x ~~~~~~~~- 3 x}} &  & \tmop{Subtract} 3 x \tmop{from} \tmop{both}
    \tmop{sides}\\
    - 4 y = - 3 x + 2~~~~~ &  & \tmop{Put} x \tmop{term} \tmop{first}\\
    \tmmathbf{\overline{- 4} ~~~ \overline{- 4} ~~ \overline{- 4}}~~~~~ &  & \tmop{Divide}
    \tmop{each} \tmop{term} \tmop{by} - 4\\
    y = \frac{3}{4} x - \frac{1}{2}~~~~ &  & \tmop{The} \tmop{slope} \tmop{is}
    \tmop{the} \tmop{coefficient} \tmop{of} x\\
    &  & \\
    m = \frac{3}{4}~~~~ &  & \tmop{Slope} \tmop{of} \tmop{given} \tmop{line}\\
		& & \tmop{Perpendicular} \tmop{lines} \tmop{have} \tmop{negative}
    \tmop{reciprocal} \tmop{slopes}\\
    m = - \frac{4}{3}~~~~ &  & \tmop{Our} \tmop{solution}
  \end{eqnarray*}
\end{example}
Once we have a slope, it is possible to find the complete equation of the desired line, if we know one point on it.

\begin{example}\label{Lin71} Find the equation of a line through $(4, - 5)$ and parallel to $2 x - 3 y =6$.
  \begin{eqnarray*}
    2 x - 3 y = 6~~~~ &  & \tmop{We} \tmop{first} \tmop{need} \tmop{slope}
    \tmop{of} \tmop{parallel} \tmop{line}\\
    \tmmathbf{\underline{- 2 x ~~~~~~~~- 2 x}} &  & \tmop{Subtract} 2 x \tmop{from} \tmop{each}
    \tmop{side}\\
    - 3 y = - 2 x + 6~~~~ &  & \tmop{Put} x \tmop{term} \tmop{first}\\
    \tmmathbf{\overline{- 3} ~~~~ \overline{- 3} ~~ \overline{- 3}}~~~ &  & \tmop{Divide}
    \tmop{each} \tmop{term} \tmop{by} - 3\\
    y = \frac{2}{3} x - 2~~~~ &  & \tmop{Identify} \tmop{the} \tmop{slope},
    \tmop{the} \tmop{coefficient} \tmop{of} x\\
    &  & \\
    m = \frac{2}{3}~~~~ &  & \tmop{Parallel} \tmop{lines} \tmop{have} \tmop{the}
    \tmop{same} \tmop{slope}\\
    m = \frac{2}{3}~~~~ &  & \tmop{We} \tmop{will} \tmop{use} \tmop{this}
    \tmop{slope} \tmop{and} \tmop{our} \tmop{point} (4, - 5)\\
    &  & \\
    y - y_1 = m (x - x_1)~~~~ &  & \tmop{Plug} \tmop{this} \tmop{information}
    \tmop{into} \tmop{point}-\tmop{slope} \tmop{formula}\\
    y - (- 5) = \frac{2}{3} (x - 4)~~~~ &  & \tmop{Simplify} \tmop{signs}\\
    &  & \\
    y + 5 = \frac{2}{3} (x - 4)~~~~ &  & \tmop{Our} \tmop{solution}
  \end{eqnarray*}
\end{example}
\begin{example}\label{Lin72}\ Find the equation of the line through $(6, - 9)$ perpendicular to $y = -\displaystyle\frac{3}{5} x + 4$ in slope-intercept form.
\end{example}
  \begin{eqnarray*}
    y = - \frac{3}{5} x + 4 &  & \tmop{Identify} \tmop{the} \tmop{slope},
    \tmop{coefficient} \tmop{of} x\\
    &  & \\
    m = - \frac{3}{5} &  & \tmop{Perpendicular} \tmop{lines} \tmop{have}
    \tmop{negative} \tmop{reciprocal} \tmop{slopes}\\
    &  & \\
    m = \frac{5}{3} &  & \tmop{We} \tmop{will} \tmop{use} \tmop{this}
    \tmop{slope} \tmop{and} \tmop{our} \tmop{point} (6, - 9) \\
    &  & \\
    y - y_1 = m (x - x_1) &  & \tmop{Plug} \tmop{this} \tmop{information}
    \tmop{into} \tmop{point} - \tmop{slope} \tmop{formula}\\
    &  & \\
    y - (- 9) = \frac{5}{3} (x - 6) &  & \tmop{Simplify} \tmop{signs}\\
    &  & \\
    y + 9 = \frac{5}{3} (x - 6) &  & \tmop{Distribute} \tmop{slope}
  \end{eqnarray*}
  \begin{eqnarray*}
  %  &  & \\
    y + 9 = \frac{5}{3} x - 10 &  & \tmop{Solve} \tmop{for} y\\
    \tmmathbf{\underline{- 9 ~~~~~~~~- 9}} &  & \tmop{Subtract} 9 \tmop{from} \tmop{both}
    \tmop{sides}\\
    y = \frac{5}{3} x - 19 &  & \tmop{Our} \tmop{solution}
  \end{eqnarray*}
Zero slopes and undefined slopes may seem like opposites (one is a horizontal line, one is a vertical line). Because a horizontal line is perpendicular to a vertical line we can say that an undefined slope and a zero slope are actually perpendicular slopes!
\begin{example}\label{Lin73} Find the equation of the line through (3, 4) perpendicular to $x = - 2$.
  \begin{eqnarray*}
    x = - 2~~ &  & \tmop{This} \tmop{equation} \tmop{has} \tmop{an~undefined}
    \tmop{slope}, \tmop{a~vertical} \tmop{line}\\
    \tmop{Undefined} \tmop{slope}~~ &  & \tmop{Perpendicular} \tmop{line} \tmop{then}
    \tmop{would} \tmop{have} \tmop{a~zero} \tmop{slope}\\
    m = 0~~ &  & \tmop{Use} \tmop{this} \tmop{and} \tmop{our} \tmop{point} (3,
    4)\\
    y - y_1 = m (x - x_1)~~ &  & \tmop{Plug} \tmop{this} \tmop{information}
    \tmop{into} \tmop{point} - \tmop{slope} \tmop{formula}\\
    y - 4 = 0 (x - 3)~~ &  & \tmop{Distribute} \tmop{slope}\\
    y - 4 = 0~~ &  & \tmop{Solve} \tmop{for} y\\
    \tmmathbf{\underline{+ 4 ~+ 4}} &  & \tmop{Add} 4 \tmop{to} \tmop{each} \tmop{side}\\
    y = 4~~ &  & \tmop{Our} \tmop{solution}
  \end{eqnarray*}
\end{example}
Being aware that to be perpendicular to a vertical line means we have a horizontal line through a $y$ value of 4, thus we could have jumped from this point right to the solution, $y = 4$.
\end{document}