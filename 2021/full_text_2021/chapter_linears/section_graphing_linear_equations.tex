\documentclass[12pt]{book}
\raggedbottom
\usepackage[top=1in,left=1in,bottom=1in,right=1in,headsep=0.25in]{geometry}	
\usepackage{amssymb,amsmath,amsthm,amsfonts}
\usepackage{chapterfolder,docmute,setspace}
\usepackage{cancel,multicol,tikz,verbatim,framed,polynom,enumitem,tikzpagenodes}
\usepackage[colorlinks, hyperindex, plainpages=false, linkcolor=blue, urlcolor=blue, pdfpagelabels]{hyperref}
\usepackage[type={CC},modifier={by-sa},version={4.0},]{doclicense}

\theoremstyle{definition}
\newtheorem{example}{Example}
\newcommand{\Desmos}{\href{https://www.desmos.com/}{Desmos}}
\setlength{\parindent}{0in}
\setlist{itemsep=0in}
\setlength{\parskip}{0.1in}
\setcounter{secnumdepth}{0}
% This document is used for ordering of lessons.  If an instructor wishes to change the ordering of assessments, the following steps must be taken:

% 1) Reassign the appropriate numbers for each lesson in the \setcounter commands included in this file.
% 2) Rearrange the \include commands in the master file (the file with 'Course Pack' in the name) to accurately reflect the changes.  
% 3) Rarrange the \items in the measureable_outcomes file to accurately reflect the changes.  Be mindful of page breaks when moving items.
% 4) Re-build all affected files (master file, measureable_outcomes file, and any lessons whose numbering has changed).

%Note: The placement of each \newcounter and \setcounter command reflects the original/default ordering of topics (linears, systems, quadratics, functions, polynomials, rationals).

\newcounter{lesson_solving_linear_equations}
\newcounter{lesson_equations_containing_absolute_values}
\newcounter{lesson_graphing_lines}
\newcounter{lesson_two_forms_of_a_linear_equation}
\newcounter{lesson_parallel_and_perpendicular_lines}
\newcounter{lesson_linear_inequalities}
\newcounter{lesson_compound_inequalities}
\newcounter{lesson_inequalities_containing_absolute_values}
\newcounter{lesson_graphing_systems}
\newcounter{lesson_substitution}
\newcounter{lesson_elimination}
\newcounter{lesson_quadratics_introduction}
\newcounter{lesson_factoring_GCF}
\newcounter{lesson_factoring_grouping}
\newcounter{lesson_factoring_trinomials_a_is_1}
\newcounter{lesson_factoring_trinomials_a_neq_1}
\newcounter{lesson_solving_by_factoring}
\newcounter{lesson_square_roots}
\newcounter{lesson_i_and_complex_numbers}
\newcounter{lesson_vertex_form_and_graphing}
\newcounter{lesson_solve_by_square_roots}
\newcounter{lesson_extracting_square_roots}
\newcounter{lesson_the_discriminant}
\newcounter{lesson_the_quadratic_formula}
\newcounter{lesson_quadratic_inequalities}
\newcounter{lesson_functions_and_relations}
\newcounter{lesson_evaluating_functions}
\newcounter{lesson_finding_domain_and_range_graphically}
\newcounter{lesson_fundamental_functions}
\newcounter{lesson_finding_domain_algebraically}
\newcounter{lesson_solving_functions}
\newcounter{lesson_function_arithmetic}
\newcounter{lesson_composite_functions}
\newcounter{lesson_inverse_functions_definition_and_HLT}
\newcounter{lesson_finding_an_inverse_function}
\newcounter{lesson_transformations_translations}
\newcounter{lesson_transformations_reflections}
\newcounter{lesson_transformations_scalings}
\newcounter{lesson_transformations_summary}
\newcounter{lesson_piecewise_functions}
\newcounter{lesson_functions_containing_absolute_values}
\newcounter{lesson_absolute_as_piecewise}
\newcounter{lesson_polynomials_introduction}
\newcounter{lesson_sign_diagrams_polynomials}
\newcounter{lesson_factoring_quadratic_type}
\newcounter{lesson_factoring_summary}
\newcounter{lesson_polynomial_division}
\newcounter{lesson_synthetic_division}
\newcounter{lesson_end_behavior_polynomials}
\newcounter{lesson_local_behavior_polynomials}
\newcounter{lesson_rational_root_theorem}
\newcounter{lesson_polynomials_graphing_summary}
\newcounter{lesson_polynomial_inequalities}
\newcounter{lesson_rationals_introduction_and_terminology}
\newcounter{lesson_sign_diagrams_rationals}
\newcounter{lesson_horizontal_asymptotes}
\newcounter{lesson_slant_and_curvilinear_asymptotes}
\newcounter{lesson_vertical_asymptotes}
\newcounter{lesson_holes}
\newcounter{lesson_rationals_graphing_summary}

\setcounter{lesson_solving_linear_equations}{1}
\setcounter{lesson_equations_containing_absolute_values}{2}
\setcounter{lesson_graphing_lines}{3}
\setcounter{lesson_two_forms_of_a_linear_equation}{4}
\setcounter{lesson_parallel_and_perpendicular_lines}{5}
\setcounter{lesson_linear_inequalities}{6}
\setcounter{lesson_compound_inequalities}{7}
\setcounter{lesson_inequalities_containing_absolute_values}{8}
\setcounter{lesson_graphing_systems}{9}
\setcounter{lesson_substitution}{10}
\setcounter{lesson_elimination}{11}
\setcounter{lesson_quadratics_introduction}{16}
\setcounter{lesson_factoring_GCF}{17}
\setcounter{lesson_factoring_grouping}{18}
\setcounter{lesson_factoring_trinomials_a_is_1}{19}
\setcounter{lesson_factoring_trinomials_a_neq_1}{20}
\setcounter{lesson_solving_by_factoring}{21}
\setcounter{lesson_square_roots}{22}
\setcounter{lesson_i_and_complex_numbers}{23}
\setcounter{lesson_vertex_form_and_graphing}{24}
\setcounter{lesson_solve_by_square_roots}{25}
\setcounter{lesson_extracting_square_roots}{26}
\setcounter{lesson_the_discriminant}{27}
\setcounter{lesson_the_quadratic_formula}{28}
\setcounter{lesson_quadratic_inequalities}{29}
\setcounter{lesson_functions_and_relations}{12}
\setcounter{lesson_evaluating_functions}{13}
\setcounter{lesson_finding_domain_and_range_graphically}{14}
\setcounter{lesson_fundamental_functions}{15}
\setcounter{lesson_finding_domain_algebraically}{30}
\setcounter{lesson_solving_functions}{31}
\setcounter{lesson_function_arithmetic}{32}
\setcounter{lesson_composite_functions}{33}
\setcounter{lesson_inverse_functions_definition_and_HLT}{34}
\setcounter{lesson_finding_an_inverse_function}{35}
\setcounter{lesson_transformations_translations}{36}
\setcounter{lesson_transformations_reflections}{37}
\setcounter{lesson_transformations_scalings}{38}
\setcounter{lesson_transformations_summary}{39}
\setcounter{lesson_piecewise_functions}{40}
\setcounter{lesson_functions_containing_absolute_values}{41}
\setcounter{lesson_absolute_as_piecewise}{42}
\setcounter{lesson_polynomials_introduction}{43}
\setcounter{lesson_sign_diagrams_polynomials}{44}
\setcounter{lesson_factoring_quadratic_type}{46}
\setcounter{lesson_factoring_summary}{45}
\setcounter{lesson_polynomial_division}{47}
\setcounter{lesson_synthetic_division}{48}
\setcounter{lesson_end_behavior_polynomials}{49}
\setcounter{lesson_local_behavior_polynomials}{50}
\setcounter{lesson_rational_root_theorem}{51}
\setcounter{lesson_polynomials_graphing_summary}{52}
\setcounter{lesson_polynomial_inequalities}{53}
\setcounter{lesson_rationals_introduction_and_terminology}{54}
\setcounter{lesson_sign_diagrams_rationals}{55}
\setcounter{lesson_horizontal_asymptotes}{56}
\setcounter{lesson_slant_and_curvilinear_asymptotes}{57}
\setcounter{lesson_vertical_asymptotes}{58}
\setcounter{lesson_holes}{59}
\setcounter{lesson_rationals_graphing_summary}{60}

\newcommand{\tmmathbf}[1]{\ensuremath{\boldsymbol{#1}}}
\newcommand{\tmop}[1]{\ensuremath{\operatorname{#1}}}

\begin{document}
\section{Graphing Linear Equations}
\subsection{The Cartesian Plane}
{\bf Objective: Locate and graph points using $xy$-coordinates}\par
Often, to get an idea of the behavior of an equation we will make a picture that represents the solutions to the equation.  Before we spend much time on making a visual representation of an equation, we first have to understand the basics of graphing.  A {\it graph} is a set of points in the $xy-$plane, also known as the Cartesian plane.  In most cases, a graph can simply be thought of as a ``picture'' of the points.  We will see shortly that the graph of a linear equation is a visualization of the solutions to the equation.  The following is an example of the Cartesian or $xy-$coordinate plane.
\begin{multicols}{2}
 	\begin{tikzpicture}[xscale=0.5,yscale=0.5]
		\draw[step=1.0,gray,very thin,dotted] (-7.5,-5.5) grid (7.5,5.5);
		\draw [<->](-7.5,0) -- coordinate (x axis mid) (7.5,0) node[below right] {$x$};
		\draw [<->](0,-5.5) -- coordinate (y axis mid) (0,5.5) node[above right] {$y$};
		\foreach \x in {1,...,7}
		\draw (\x,2pt) -- (\x,-2pt)	node[anchor=south] {\scriptsize \x};
		\foreach \x in {-1,...,-7}
		\draw (\x,2pt) -- (\x,-2pt)	node[anchor=north] {\scriptsize \x};
		\foreach \y in {1,...,5}
		\draw (2pt,\y) -- (-2pt,\y)	node[anchor=east] {\scriptsize \y}; 
		\foreach \y in {-1,...,-5}
		\draw (2pt,\y) -- (-2pt,\y)	node[anchor=west] {\scriptsize \y}; 
		\draw (3.5,3) node {QI $(+,+)$};
		\draw (-3.5,3) node {QII $(-,+)$};
		\draw (-3.5,-3) node {QIII $(-,-)$};
		\draw (3.5,-3) node {QIV $(+,-)$};
	\end{tikzpicture}

  \columnbreak
  
  The plane is divided into four {\it quadrants}, or sections, by a horizontal number line
  ($x$-axis) and a vertical number line ($y$-axis).\par
Where the two lines, or axes, meet in the center is called the origin. This center origin is where $x$ = 0 and $y$ = 0.\par
  The quadrants are numbered using the roman numerals I, II, III, and IV, beginning with the top-right quadrant (where both $x$ and $y$ are positive) and moving counter-clockwise.
\end{multicols}
As we move to the right the numbers count up from zero, representing $x = 1, 2, 3 \ldots$. To the left the numbers count down from zero, representing \mbox{$x = - 1, - 2, - 3,\ldots$}.
Similarly, as we move up the numbers count up from zero, \mbox{$y = 1, 2, 3,\ldots$}, and as we move down count down from zero, \mbox{$y = - 1, - 2, - 3\ldots$}.\par
We can put dots on the graph which we will call points. Each point has an ``address'' that defines its location. The first number will be the value on the $x -$axis or horizontal number line. This is the distance the point moves left/right from the origin. The second number will represent the value on the $y -$axis or vertical number line. This is the distance the point moves up/down from the origin. The points are given as an ordered pair $(x, y) .$\par
The following example finds the address or coordinate pair for each of several points on the coordinate plane.
\newpage
\begin{example} \label{Lin42}
 Give the coordinates of each point.
  \begin{multicols}{2}
 	\begin{tikzpicture}[xscale=0.4,yscale=0.4]
		\draw[step=1.0,gray,very thin,dotted] (-7.5,-5.5) grid (7.5,5.5);
		\draw [<->](-7.5,0) -- coordinate (x axis mid) (7.5,0) node[below right] {$x$};
		\draw [<->](0,-5.5) -- coordinate (y axis mid) (0,5.5) node[above right] {$y$};
		\foreach \x in {1,...,7}
		\draw (\x,2pt) -- (\x,-2pt)	node[anchor=south] {\scriptsize \x};
		\foreach \x in {-1,...,-7}
		\draw (\x,2pt) -- (\x,-2pt)	node[anchor=south] {\scriptsize \x};
		\foreach \y in {1,...,5}
		\draw (2pt,\y) -- (-2pt,\y)	node[anchor=east] {\scriptsize \y}; 
		\foreach \y in {-1,...,-5}
		\draw[fill] (2pt,\y) -- (-2pt,\y)	node[anchor=east] {\scriptsize \y}; 
		\draw[fill] (1,4) circle (0.15) node[right] {\scriptsize $A(1,4)$};
		\draw[fill] (-5,3)circle (0.15) node[right] {\scriptsize $B(-5,3)$};
		\draw[fill] (0,-2) circle (0.15) node[right] {\scriptsize $C(0,-2)$};
	\end{tikzpicture}

  \columnbreak
  
Tracing from the origin, point $A$ is right 1, up 4. This becomes $A(1, 4)$.\par
Point $B$ is left 5, up 3. Left is backwards or negative so we have $B(- 5, 3)$.\par
Point $C$ is straight down 2 units. There is no left or right. This means we go right zero so the point is $C(0, - 2)$.\par
Our solution is $A(1, 4), B(- 5, 3), C(0, - 2)$.
  \end{multicols}
\end{example}
Just as we can give the coordinates for a set of points, we can take a set of points and plot them on the plane.
\begin{example}\label{Lin43}
   Graph the set of points:\\
    $\{A (3, 2), B (- 2, 1), C (3, - 4), D (- 2, - 3), E (- 3,0), F (0, 2), G (0, 0)\}$
  \begin{multicols}{2}
 	\begin{tikzpicture}[xscale=0.65,yscale=0.65]
		\draw[step=1.0,gray,very thin,dotted] (-4.5,-4.5) grid (4.5,4.5);
		\draw [<->](-4.5,0) -- coordinate (x axis mid) (4.5,0) node[below right] {$x$};
		\draw [<->](0,-4.5) -- coordinate (y axis mid) (0,4.5) node[above right] {$y$};
		\draw[fill] (3,2) circle (0.1) node[above right] {\scriptsize $A(3,2)$};
		\draw[fill] (-2,1) circle (0.1) node[above left] {\scriptsize $B(-2,1)$};
		\draw[fill] (3,-4) circle (0.1) node[below right] {\scriptsize $C(3,-4)$};
		\draw[fill] (-2,-3) circle (0.1) node[above left] {\scriptsize $D(-2,-3)$};
		\draw[->,line width=0.5mm] (0,0) -- (3,0);
		\draw (1.5,0.3)  node {\scriptsize Right 3};
		\draw[->,line width=0.5mm] (3,0) -- (3,2);
		\draw (3.7,1)  node {\scriptsize Up 3};
	\end{tikzpicture}

  \columnbreak
  
     The first point, $A$ is at $(3, 2)$ this means $x = 3$ (right 3) and $y = 2$ (up 2). Following these instructions, starting from the origin, we get our point. This is also illustrated on the graph.\par
     The second point, $B (- 2, 1)$, is left 2 (negative moves backwards), up 1.\par
     The third point, $C (3, - 4)$ is right 3, down 4 (negative moves backwards).\par
     The fourth point, $D(- 2, - 3)$ is left 2, down 3 (both negative, both move backwards).
  \end{multicols}
  \begin{multicols}{2}
 	\begin{tikzpicture}[xscale=0.65,yscale=0.65]
		\draw[step=1.0,gray,very thin,dotted] (-4.5,-4.5) grid (4.5,4.5);
		\draw [<->](-4.5,0) -- coordinate (x axis mid) (4.5,0) node[below right] {$x$};
		\draw [<->](0,-4.5) -- coordinate (y axis mid) (0,4.5) node[above right] {$y$};
		\draw[fill] (3,2) circle (0.1) node[above right] {\scriptsize $A(3,2)$};
		\draw[fill] (-2,1) circle (0.1) node[above left] {\scriptsize $B(-2,1)$};
		\draw[fill] (3,-4) circle (0.1) node[below right] {\scriptsize $C(3,-4)$};
		\draw[fill] (-2,-3) circle (0.1) node[above left] {\scriptsize $D(-2,-3)$};
		\draw[fill] (-3,0) circle (0.1) node[below] {\scriptsize $E(-3,0)$};
		\draw[fill] (0,2) circle (0.1) node[above right] {\scriptsize $F(0,2)$};
		\draw[fill] (0,0) circle (0.1) node[below right] {\scriptsize $G(0,0)$};
	\end{tikzpicture}

  \columnbreak
  
   The last three points have zeros in them. We still treat these points just like the other points. If there is a zero there is just no movement.\par
   First is $E (- 3, 0)$. This is left 3, and up zero, right on the $x -$axis.\par 
   Then is $F (0, 2)$. This is right zero, and up two, right on the $y -$axis.\par
   Finally is $G (0, 0)$. This point has no movement, and thus is right on the origin.
  \end{multicols}
\end{example}
\subsection{Graphing Lines (L\arabic{lesson_graphing_lines})}
{\bf Objective: Graph lines using $xy$-coordinates.}\par
The main purpose of graphs is not to plot random points, but rather to give a picture of the solutions to an equation. We may have an equation such as $y =2 x - 3$. We may be interested in what type of solution are possible in this equation. We can visualize the solution by making a graph of possible $x$ and $y$ combinations that make this equation a true statement. We will have to start by finding possible $x$ and $y$ combinations. We will do this using a table of values.
\begin{example}\label{Lin44}
  \begin{eqnarray*}
    \tmop{Graph} y = 2 x - 3 &  & \tmop{We} \tmop{make~a} \tmop{table}
    \tmop{of} \tmop{values}\\
    &  & \\
    \begin{array}{|c|c|}
      \hline
      x & ~~y~~\\
      \hline
      - 1 & \\
      \hline
      0 & \\
      \hline
      1 & \\
      \hline
    \end{array} &  & \tmop{We} \tmop{will} \tmop{test} \tmop{three}
    \tmop{values} \tmop{for} x. \tmop{Any} \tmop{three} \tmop{can} \tmop{be}
    \tmop{used}\\
    &  & \\
    \begin{array}{|c|c|}
      \hline
      x & y\\
      \hline
      - 1 & - 5\\
      \hline
      0 & - 3\\
      \hline
      1 & - 1\\
      \hline
    \end{array} &  & \begin{array}{l}
      \tmop{Evaluate} \tmop{each} \tmop{by} \tmop{replacing} x \tmop{with}
      \tmop{the} \tmop{given} \tmop{value}\\
      x = - 1 ~~~~~~~ y = 2 (- 1) - 3 = - 2 - 3 = - 5\\
      x = 0  ~~~~~~~~~y = 2 (0) - 3 = 0 - 3 = - 3\\
      x = 1  ~~~~~~~~~y = 2 (1) - 3 = 2 - 3 = - 1
    \end{array}
  \end{eqnarray*}
  \begin{multicols}{2}
 	\begin{tikzpicture}[xscale=0.5,yscale=0.5]
		\draw[step=1.0,gray,very thin,dotted] (-5.5,-5.5) grid (5.5,5.5);
		\draw [<->](-5.5,0) -- coordinate (x axis mid) (5.5,0) node[below right] {$x$};
		\draw [<->](0,-5.5) -- coordinate (y axis mid) (0,5.5) node[above right] {$y$};
		\foreach \x in {1,...,5}
		\draw (\x,2pt) -- (\x,-2pt)	node[anchor=south] {\scriptsize \x};
		\foreach \x in {-1,...,-5}
		\draw (\x,2pt) -- (\x,-2pt)	node[anchor=south] {\scriptsize \x};
		\foreach \y in {1,...,5}
		\draw (2pt,\y) -- (-2pt,\y)	node[anchor=east] {\scriptsize \y}; 
		\foreach \y in {-1,...,-5}
		\draw[fill] (2pt,\y) -- (-2pt,\y)	node[anchor=east] {\scriptsize \y}; 
		\draw [<->] plot [domain=-1.2:4, samples=100] (\x,{2*\x-3});
		\draw[fill] (1,-1) circle (0.1) node[right] {};
		\draw[fill] (0,-3)circle (0.1) node[right] {};
		\draw[fill] (-1,-5) circle (0.1) node[right] {};
	\end{tikzpicture}

\columnbreak
    \ \par 
    $(- 1, - 5), (0, - 3),$ and $(1, - 1)$\par
     These become the points from our equation which we will plot on our graph.\par    
     Once the point are on the graph, connect the dots to make a line.\par
     The graph is our solution.
  \end{multicols}
\end{example}
What this line tells us is that any point on the line will work in the equation $y = 2 x - 3$. For example, notice the graph also goes through the point $(2, 1)$. If we use $x = 2$, we should get $y = 1$. Sure enough, $y = 2 (2) - 3 = 4 - 3 = 1$, just as the graph suggests. Thus we have the line is a picture of all the solutions for $y = 2 x - 3$. We can use this table of values method to draw a graph of any linear equation.
\begin{example}\label{Lin45}  
  \begin{eqnarray*}
    \tmop{Graph} 2 x - 3 y = 6 &  & \tmop{We} \tmop{will} \tmop{use} \tmop{a}
    \tmop{table} \tmop{of} \tmop{values}\\
    &  & \\
    \begin{array}{|c|c|}
      \hline
      x & ~y~~\\
      \hline
      - 3 & \\
      \hline
      0 & \\
      \hline
      3 & \\
      \hline
    \end{array} &  & \tmop{We} \tmop{will} \tmop{test} \tmop{three}
    \tmop{values} \tmop{for} x. \tmop{~Any} \tmop{three} \tmop{can} \tmop{be}
    \tmop{used}.\\
    %&  & \\
%      \end{eqnarray*}
%      \begin{eqnarray*}
		2 (- 3) - 3 y = 6~ &  & \tmop{Substitute} \tmop{each} \tmop{value}
    \tmop{in} \tmop{for} x \tmop{and} \tmop{solve} \tmop{for} y\\
    - 6 - 3 y = 6~ &  & \tmop{Start} \tmop{with} x = - 3, \tmop{multiply}
    \tmop{first}\\
    \bf{\underline{+ 6 ~~~~~~~+ 6}} &  & \tmop{Add} 6 \tmop{to} \tmop{both} \tmop{sides}\\
    - 3 y = 12~ &  & \tmop{Divide} \tmop{both} \tmop{sides} \tmop{by} - 3\\
    \bf{\overline{- 3} ~~~ \overline{- 3}}~ &  & \\
    y = - 4~ &  & \tmop{solution} \tmop{for} y \tmop{when} x = - 3, \tmop{add}
    \tmop{this} \tmop{to} \tmop{table}\\
    &  & \\
    2 (0) - 3 y = 6~ &  & \tmop{Next} x = 0\\
    - 3 y = 6~ &  & \tmop{Multiplying} \tmop{clears} \tmop{the} \tmop{constant}
    \tmop{term}\\
    \bf{\overline{- 3} ~~~~ \overline{- 3}} &  & \tmop{Divide} \tmop{each} \tmop{side}
    \tmop{by} - 3\\
    y = - 2~ &  & \tmop{solution} \tmop{for} y \tmop{when} x = 0, \tmop{add}
    \tmop{this} \tmop{to} \tmop{table}\\
    &  & \\
    2 (3) - 3 y = 6~ &  & \tmop{Next} x = 3\\
    6 - 3 y = 6~ &  & \tmop{Multiply}\\
    \bf{\underline{- 6 ~~~~~~~~- 6}} &  & \tmop{Subtract} 9 \tmop{from} \tmop{both}
    \tmop{sides}\\
    - 3 y = 0~ &  & \tmop{Divide} \tmop{each} \tmop{side} \tmop{by} - 3\\
    \bf{\overline{- 3} ~~~ \overline{- 3}} &  & \\
    y = 0~ &  & \tmop{solution} \tmop{for} y \tmop{when} x = - 3, \tmop{add}
    \tmop{this} \tmop{to} \tmop{table}\\
    &  & \\
%      \end{eqnarray*}
 %     \begin{eqnarray*}
		\begin{array}{|c|c|}
      \hline
      x & y\\
      \hline
      - 3 & - 4\\
      \hline
      0 & - 2\\
      \hline
      3 & 0\\
      \hline
    \end{array} &  & \tmop{Our} \tmop{completed} \tmop{table}
  \end{eqnarray*}
  \begin{multicols}{2}
 	\begin{tikzpicture}[xscale=0.45,yscale=0.45]
		\draw[step=1.0,gray,very thin,dotted] (-5.5,-5.5) grid (5.5,5.5);
		\draw [<->](-5.5,0) -- coordinate (x axis mid) (5.5,0) node[below right] {$x$};
		\draw [<->](0,-5.5) -- coordinate (y axis mid) (0,5.5) node[above right] {$y$};
		\foreach \x in {1,...,5}
		\draw (\x,2pt) -- (\x,-2pt)	node[anchor=south] {\scriptsize \x};
		\foreach \x in {-1,...,-5}
		\draw (\x,2pt) -- (\x,-2pt)	node[anchor=south] {\scriptsize \x};
		\foreach \y in {1,...,5}
		\draw (2pt,\y) -- (-2pt,\y)	node[anchor=east] {\scriptsize \y}; 
		\foreach \y in {-1,...,-5}
		\draw[fill] (2pt,\y) -- (-2pt,\y)	node[anchor=east] {\scriptsize \y}; 
		\draw [<->] plot [domain=-3.5:5, samples=100] (\x,{0.667*\x-2});
		\draw[fill] (-3,-4) circle (0.1) node[right] {};
		\draw[fill] (0,-2)circle (0.1) node[right] {};
		\draw[fill] (3,0) circle (0.1) node[right] {};
	\end{tikzpicture}

\columnbreak
    \ \par 
    The coordinate points from our table are then $(- 3, - 4), (0, -2),$ and $(3, 0)$\par
    \ \par
    After we plot these points, we connect them to form our graph.\par
  \end{multicols}
\end{example}
\subsection{The Slope of a Line}
{\bf Objective: Find the slope of a line given a graph or two points.}\par
As we graph lines, we will want to be able to identify different properties of the lines we graph. One of the most important properties of a line is its slope. {\it Slope} is a measure of steepness.  A line with a large slope, such as 25, is very steep or increases quickly. A line with a small slope, such as $\frac{1}{10}$ is very flat or increases gradually. We will also use slope to describe the direction of the line. A line that goes up from left to right will have a positive slope and a line that goes down from left to right will have a negative slope.\par
As we measure steepness we are interested in how fast the line rises compared to how far the line runs. For this reason we will describe slope as the fraction $\frac{\text{rise}}{\text{run}}$. Rise would be a vertical change, or a change in the $y$-values. Run would be a horizontal change, or a change in the $x$-values. So another way to describe slope would be the fraction $\frac{\text{change in }\ y}{\text{change in }\ x}$. It turns out that if we have a graph we can draw vertical and horizontal lines from one point to another to make what is called a slope triangle. The sides of the slope triangle give us our slope. Using this idea, we find the corresponding slopes for each of the lines that follow.

\begin{multicols}{2}
 	\begin{tikzpicture}[xscale=0.5,yscale=0.5]
		\draw[step=1.0,gray,very thin,dotted] (-5.5,-5.5) grid (5.5,5.5);
		\draw [<->](-5.5,0) -- coordinate (x axis mid) (5.5,0) node[below right] {$x$};
		\draw [<->](0,-5.5) -- coordinate (y axis mid) (0,5.5) node[above right] {$y$};
		\draw [->,dashed,line width=0.25mm](-4,3) -- coordinate (y axis mid) (-4,-1);
		\draw (-5.25,1) node {\scriptsize Rise $-4$};
		\draw [->,dashed,line width=0.25mm](-4,-1) -- (2,-1);
		\draw (-1.5,-1.5) node {\scriptsize Run $6$};
		\foreach \x in {1,...,5}
		\draw (\x,2pt) -- (\x,-2pt)	node[anchor=south] {\scriptsize \x};
%		\foreach \x in {-1,...,-5}
%		\draw (\x,2pt) -- (\x,-2pt)	node[anchor=south] {\scriptsize \x};
		\foreach \y in {1,...,5}
		\draw (2pt,\y) -- (-2pt,\y)	node[anchor=west] {\scriptsize \y}; 
%		\foreach \y in {-1,...,-5}
%		\draw[fill] (2pt,\y) -- (-2pt,\y)	node[anchor=west] {\scriptsize \y}; 
		\draw [<->] plot [domain=-4.5:4.5, samples=100] (\x,{-0.667*(\x-2)-1});
		\draw[fill] (-4,3) circle (0.1) node[right] {};
		\draw[fill] (2,-1)circle (0.1) node[right] {};
	\end{tikzpicture}

\columnbreak

To find the slope of this line we will consider the rise, or vertical change and the run or horizontal change.\par
Drawing these lines in creates a triangle that we can use to count from one point to the next:\par 
the graph goes down 4, right 6. This is a rise of $- 4$ and a run $6$.\par 
As a fraction, we have, $\frac{- 4}{6}$, or $-\frac{2}{3}$ when reduced, which is our slope.
\end{multicols}

\begin{multicols}{2}
 	\begin{tikzpicture}[xscale=0.5,yscale=0.5]
		\draw[step=1.0,gray,very thin,dotted] (-5.5,-5.5) grid (5.5,5.5);
		\draw [<->](-5.5,0) -- coordinate (x axis mid) (5.5,0) node[below right] {$x$};
		\draw [<->](0,-5.5) -- coordinate (y axis mid) (0,5.5) node[above right] {$y$};
		\draw [->,dashed,line width=0.25mm](1,-3) -- (1,3);
		\draw (-1,-1) node {\scriptsize Rise $6$};
		\draw [->,dashed,line width=0.25mm](1,3) -- (4,3);
		\draw (2.5,3.5) node {\scriptsize Run $3$};
		\foreach \x in {1,...,5}
		\draw (\x,2pt) -- (\x,-2pt)	node[anchor=south] {\scriptsize \x};
%		\foreach \x in {-1,...,-5}
%		\draw (\x,2pt) -- (\x,-2pt)	node[anchor=south] {\scriptsize \x};
		\foreach \y in {1,...,5}
		\draw (2pt,\y) -- (-2pt,\y)	node[anchor=west] {\scriptsize \y}; 
%		\foreach \y in {-1,...,-5}
%		\draw[fill] (2pt,\y) -- (-2pt,\y)	node[anchor=west] {\scriptsize \y}; 
		\draw [<->] plot [domain=-0.25:5, samples=100] (\x,{2*(\x)-5});
		\draw[fill] (1,-3) circle (0.1) node[right] {};
		\draw[fill] (4,3)circle (0.1) node[right] {};
	\end{tikzpicture}

\columnbreak
	    
    \ \par
    To find the slope of this line, the rise is up 6, the run is right 3.\par
    Our slope is then written as a fraction:\par 
    $\frac{\text{rise}}{\text{run}}=\frac{6}{3}$.\par
	This fraction reduces to 2.\par
	A slope of $2$ is our solution.
 \end{multicols}

There are two special lines that have unique slopes that we need to be aware of. They are illustrated in the following examples.
  \begin{multicols}{2}
 	\begin{tikzpicture}[xscale=0.5,yscale=0.5]
		\draw[step=1.0,gray,very thin,dotted] (-5.5,-5.5) grid (5.5,5.5);
		\draw [<->](-5.5,0) -- coordinate (x axis mid) (5.5,0) node[below right] {$x$};
		\draw [<->](0,-5.5) -- coordinate (y axis mid) (0,5.5) node[above right] {$y$};
		\foreach \x in {1,...,5}
		\draw (\x,2pt) -- (\x,-2pt)	node[anchor=south] {\scriptsize \x};
		\foreach \x in {-1,...,-5}
		\draw (\x,2pt) -- (\x,-2pt)	node[anchor=north] {\scriptsize \x};
		\foreach \y in {1,...,5}
		\draw (2pt,\y) -- (-2pt,\y)	node[anchor=west] {\scriptsize \y}; 
		\foreach \y in {-1,...,-5}
		\draw[fill] (2pt,\y) -- (-2pt,\y)	node[anchor=east] {\scriptsize \y}; 
		\draw [<->] plot [domain=-4.5:4.5, samples=100] (\x,{2});
		\draw[fill] (1,2) circle (0.1) node[right] {};
		\draw[fill] (-2,2)circle (0.1) node[right] {};
	\end{tikzpicture}

\columnbreak

    \ \par
    In this graph there is no rise, but the run is 3 units.\par
    This slope becomes $\frac{0}{3} = 0$.\par
	\ \par
	This, and all {\it horizontal} lines have a slope of zero.
\end{multicols}
\begin{multicols}{2}
 	\begin{tikzpicture}[xscale=0.5,yscale=0.5]
		\draw[step=1.0,gray,very thin,dotted] (-5.5,-5.5) grid (5.5,5.5);
		\draw [<->](-5.5,0) -- coordinate (x axis mid) (5.5,0) node[below right] {$x$};
		\draw [<->](0,-5.5) -- coordinate (y axis mid) (0,5.5) node[above right] {$y$};
		\foreach \x in {1,...,5}
		\draw (\x,2pt) -- (\x,-2pt)	node[anchor=south] {\scriptsize \x};
		\foreach \x in {-1,...,-5}
		\draw (\x,2pt) -- (\x,-2pt)	node[anchor=north] {\scriptsize \x};
		\foreach \y in {1,...,5}
		\draw (2pt,\y) -- (-2pt,\y)	node[anchor=west] {\scriptsize \y}; 
		\foreach \y in {-1,...,-5}
		\draw[fill] (2pt,\y) -- (-2pt,\y)	node[anchor=east] {\scriptsize \y}; 
		\draw [<->] plot [domain=-4.5:4.5, samples=100] (3,\x);
		\draw[fill] (3,-2) circle (0.1) node[right] {};
		\draw[fill] (3,3)circle (0.1) node[right] {};
	\end{tikzpicture}

\columnbreak

    \ \par
    This line has a rise of 5, but no run.\par
    The slope becomes $\frac{5}{0} =$ undefined, or $\varnothing$.\par
	\ \par
	This, and all {\it vertical} lines have an undefined slope.
  \end{multicols}
As you can see there is a big difference between having a zero slope and having no slope or undefined slope. Remember, slope is a measure of steepness. The first slope is not steep at all, in fact it is flat. Therefore it has a zero slope. The second slope can't get any steeper. It is so steep that there is no number large enough to express how steep it is. This is an undefined slope.\par
We can find the slope of a line through two points without seeing the points on a graph. We can do this using a slope formula. If the rise is the change in $y$ values, we can calculate this by subtracting the $y$ values of a point. Similarly, if run is a change in the $x$ values, we can calculate this by subtracting the $x$ values of a point. In this way we get the following equation for slope.
\begin{center}
The slope of a line through $(x_1,y_1)$ and $(x_2,y_2)$ is $\dfrac{y_2 - y_1}{x_2 - x_1}$.
\end{center}
When mathematicians began working with slope, it was called the modular slope. For this reason we often represent the slope with the variable $m$. Now we have the following for slope.
\begin{center}
\framebox{
\begin{minipage}{0.8\linewidth}
\[ \text{Slope} = m = \frac{\text{rise}}{\text{run}} = \frac{\text{change in} \ y}{\text{change in} \ x} = \frac{y_2 -
   y_1}{x_2 - x_1} \]
\end{minipage}
}
\end{center}
As we subtract the $y$ values and the $x$ values when calculating slope it is important we subtract them in the same order. This process is shown in the following examples.

\begin{example}\label{Lin49}
  \begin{eqnarray*}
    \tmop{Find} \tmop{the} \tmop{slope} \tmop{between} (- 4, 3) \tmop{~and~} (2,
    - 9) &  & \tmop{Identify} x_1, y_1, x_2, y_2\\
    (x_1, y_1) \tmop{~and~} (x_2, y_2) &  & \tmop{Use} \tmop{slope}
    \tmop{formula}, m = \frac{y_2 - y_1}{x_2 - x_1}\\
    m = \frac{- 9 - 3}{2 - (- 4)} &  & \tmop{Simplify}\\
    m = \frac{-12}{6} &  & \tmop{Reduce}\\
    m = - 2 &  & \tmop{Our} \tmop{solution}
  \end{eqnarray*}
\end{example}
\begin{example}\label{Lin50}
  \begin{eqnarray*}
    \tmop{Find} \tmop{the} \tmop{slope} \tmop{between} (4, 6) \tmop{~and~} (2, -
    1) &  & \tmop{Identify} x_1, y_1, x_2, y_2\\
    (x_1, y_1) \tmop{~and~} (x_2, y_2) &  & \tmop{Use} \tmop{slope}
    \tmop{formula}, m = \frac{y_2 - y_1}{x_2 - x_1}\\
    m = \frac{- 1 - 6}{2 - 4} &  & \tmop{Simplify}\\
    m = \frac{- 7}{- 2} &  & \tmop{Reduce}, \tmop{dividing} \tmop{by} - 1\\
    m = \frac{7}{2} &  & \tmop{Our} \tmop{solution}
  \end{eqnarray*}
\end{example}
We may come up against a problem that has a zero slope (horizontal line) or no slope (vertical line) just as with using the graphs.
\begin{example}\label{Lin51}
  \begin{eqnarray*}
    \tmop{Find} \tmop{the} \tmop{slope} \tmop{between~} (- 4, - 1) \tmop{~and~}
    (- 4, - 5) &  & \tmop{Identify} x_1, y_1, x_2, y_2\\
    (x_1, y_1) \tmop{~and~} (x_2, y_2) &  & \tmop{Use} \tmop{slope}
    \tmop{formula}, m = \frac{y_2 - y_1}{x_2 - x_1}\\
    m = \frac{- 5 - (- 1)}{- 4 - (- 4)} &  & \tmop{Simplify}\\
    m = \frac{- 4}{0} &  & \tmop{Can' t} \tmop{divide} \tmop{by} \tmop{zero}\\
    \tmop{Slope} m \tmop{is~undefined} &  & \tmop{Our} \tmop{solution}
  \end{eqnarray*}
\end{example}
\begin{example}\label{Lin52}
  \begin{eqnarray*}
    \tmop{Find} \tmop{the} \tmop{slope} \tmop{between~} (3, 1) \tmop{~and~} (- 2,
    1) &  & \tmop{Identify} x_1, y_1, x_2, y_2\\
    (x_1, y_1) \tmop{~and~} (x_2, y_2) &  & \tmop{Use} \tmop{slope}
    \tmop{formula}, m = \frac{y_2 - y_1}{x_2 - x_1}\\
    m = \frac{1 - 1}{- 2 - 3} &  & \tmop{Simplify}\\
    m = \frac{0}{- 5} &  & \tmop{Reduce}\\
    m = 0 &  & \tmop{Our} \tmop{solution}
  \end{eqnarray*}
\end{example}
Again, there is a big difference between no slope and a zero slope. Zero is an integer and it has a value, the slope of a flat horizontal line. No slope has no value, it is undefined, the slope of a vertical line.\par
Using the slope formula we can also find missing points if we know what the slope is. This is shown in the following two examples.
\begin{example}\label{Lin53}~~
 Find the value of $y$ between the points (2, $y$) and (5, - 1) with slope $- 3$.\\
  \begin{eqnarray*}
    m = \frac{y_2 - y_1}{x_2 - x_1} &  & \tmop{We} \tmop{will} \tmop{plug}
    \tmop{values} \tmop{into~the} \tmop{slope} \tmop{formula}\\
    - 3 = \frac{- 1 - y}{5 - 2} &  & \tmop{Simplify}\\
    - 3 = \frac{- 1 - y}{3} &  & \tmop{Multiply} \tmop{both} \tmop{sides}
    \tmop{by} 3\\
    - 3 {\bf(3)} = \frac{- 1 - y}{3} {\bf(3)} &  & \tmop{Simplify}\\
    - 9 = - 1 - y &  & \tmop{Add} 1 \tmop{to} \tmop{both} \tmop{sides}\\
    \bf{\underline{+ 1 ~~~+ 1}}~~~~  &  & \\
    - 8 = - y &  & \tmop{Divide} \tmop{both} \tmop{sides} \tmop{by} - 1\\
    \bf{\overline{- 1} ~~~~ \overline{- 1}} &  & \\
    8 = y &  & \tmop{Our} \tmop{solution}
  \end{eqnarray*}
\end{example}
\begin{example}\label{Lin54}~~
Find the value of $x$ between the points (- 3, 2) and ($x$, 6) with slope $\displaystyle\frac{2}{5}$.\\
  \begin{eqnarray*}
    m = \frac{y_2 - y_1}{x_2 - x_1} &  & \tmop{We} \tmop{will} \tmop{plug}
    \tmop{values} \tmop{into} \tmop{slope} \tmop{formula}\\
    \frac{2}{5} = \frac{6 - 2}{x - (- 3)} &  & \tmop{Simplify}\\
    \frac{2}{5} = \frac{4}{x + 3} &  & \tmop{Multiply} \tmop{both}
    \tmop{sides} \tmop{by~} (x + 3) \\
    \frac{2}{5} (x + 3) = 4 &  & \tmop{Multiply} \tmop{by} 5 \tmop{to}
    \tmop{clear} \tmop{fraction}\\
  \end{eqnarray*}
  \begin{eqnarray*}
    {\bf(5)} \frac{2}{5} (x + 3) = 4 {\bf(5)} &  & \tmop{Simplify}\\
    2 (x + 3) = 20 &  & \tmop{Distribute}\\
    2 x + 6 = 20 &  &  \\
    \bf{\underline{- 6 ~~- 6}} &  & \tmop{Subtract} 6 \tmop{from} \tmop{both}
    \tmop{sides}\\
    2 x = 14 &  & \tmop{Divide} \tmop{each} \tmop{side} \tmop{by} 2\\
    \bf{\overline{2} ~~~~~ \overline{2}}~ &  & \\
    x = 7 &  & \tmop{Our} \tmop{solution}
  \end{eqnarray*}
\end{example}
\end{document}