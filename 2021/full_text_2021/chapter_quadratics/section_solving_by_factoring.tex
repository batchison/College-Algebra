\documentclass[12pt]{book}
\raggedbottom
\usepackage[top=1in,left=1in,bottom=1in,right=1in,headsep=0.25in]{geometry}	
\usepackage{amssymb,amsmath,amsthm,amsfonts}
\usepackage{chapterfolder,docmute,setspace}
\usepackage{cancel,multicol,tikz,verbatim,framed,polynom,enumitem,tikzpagenodes}
\usepackage[colorlinks, hyperindex, plainpages=false, linkcolor=blue, urlcolor=blue, pdfpagelabels]{hyperref}
\usepackage[type={CC},modifier={by-sa},version={4.0},]{doclicense}

\theoremstyle{definition}
\newtheorem{example}{Example}
\newcommand{\Desmos}{\href{https://www.desmos.com/}{Desmos}}
\setlength{\parindent}{0in}
\setlist{itemsep=0in}
\setlength{\parskip}{0.1in}
\setcounter{secnumdepth}{0}
% This document is used for ordering of lessons.  If an instructor wishes to change the ordering of assessments, the following steps must be taken:

% 1) Reassign the appropriate numbers for each lesson in the \setcounter commands included in this file.
% 2) Rearrange the \include commands in the master file (the file with 'Course Pack' in the name) to accurately reflect the changes.  
% 3) Rarrange the \items in the measureable_outcomes file to accurately reflect the changes.  Be mindful of page breaks when moving items.
% 4) Re-build all affected files (master file, measureable_outcomes file, and any lessons whose numbering has changed).

%Note: The placement of each \newcounter and \setcounter command reflects the original/default ordering of topics (linears, systems, quadratics, functions, polynomials, rationals).

\newcounter{lesson_solving_linear_equations}
\newcounter{lesson_equations_containing_absolute_values}
\newcounter{lesson_graphing_lines}
\newcounter{lesson_two_forms_of_a_linear_equation}
\newcounter{lesson_parallel_and_perpendicular_lines}
\newcounter{lesson_linear_inequalities}
\newcounter{lesson_compound_inequalities}
\newcounter{lesson_inequalities_containing_absolute_values}
\newcounter{lesson_graphing_systems}
\newcounter{lesson_substitution}
\newcounter{lesson_elimination}
\newcounter{lesson_quadratics_introduction}
\newcounter{lesson_factoring_GCF}
\newcounter{lesson_factoring_grouping}
\newcounter{lesson_factoring_trinomials_a_is_1}
\newcounter{lesson_factoring_trinomials_a_neq_1}
\newcounter{lesson_solving_by_factoring}
\newcounter{lesson_square_roots}
\newcounter{lesson_i_and_complex_numbers}
\newcounter{lesson_vertex_form_and_graphing}
\newcounter{lesson_solve_by_square_roots}
\newcounter{lesson_extracting_square_roots}
\newcounter{lesson_the_discriminant}
\newcounter{lesson_the_quadratic_formula}
\newcounter{lesson_quadratic_inequalities}
\newcounter{lesson_functions_and_relations}
\newcounter{lesson_evaluating_functions}
\newcounter{lesson_finding_domain_and_range_graphically}
\newcounter{lesson_fundamental_functions}
\newcounter{lesson_finding_domain_algebraically}
\newcounter{lesson_solving_functions}
\newcounter{lesson_function_arithmetic}
\newcounter{lesson_composite_functions}
\newcounter{lesson_inverse_functions_definition_and_HLT}
\newcounter{lesson_finding_an_inverse_function}
\newcounter{lesson_transformations_translations}
\newcounter{lesson_transformations_reflections}
\newcounter{lesson_transformations_scalings}
\newcounter{lesson_transformations_summary}
\newcounter{lesson_piecewise_functions}
\newcounter{lesson_functions_containing_absolute_values}
\newcounter{lesson_absolute_as_piecewise}
\newcounter{lesson_polynomials_introduction}
\newcounter{lesson_sign_diagrams_polynomials}
\newcounter{lesson_factoring_quadratic_type}
\newcounter{lesson_factoring_summary}
\newcounter{lesson_polynomial_division}
\newcounter{lesson_synthetic_division}
\newcounter{lesson_end_behavior_polynomials}
\newcounter{lesson_local_behavior_polynomials}
\newcounter{lesson_rational_root_theorem}
\newcounter{lesson_polynomials_graphing_summary}
\newcounter{lesson_polynomial_inequalities}
\newcounter{lesson_rationals_introduction_and_terminology}
\newcounter{lesson_sign_diagrams_rationals}
\newcounter{lesson_horizontal_asymptotes}
\newcounter{lesson_slant_and_curvilinear_asymptotes}
\newcounter{lesson_vertical_asymptotes}
\newcounter{lesson_holes}
\newcounter{lesson_rationals_graphing_summary}

\setcounter{lesson_solving_linear_equations}{1}
\setcounter{lesson_equations_containing_absolute_values}{2}
\setcounter{lesson_graphing_lines}{3}
\setcounter{lesson_two_forms_of_a_linear_equation}{4}
\setcounter{lesson_parallel_and_perpendicular_lines}{5}
\setcounter{lesson_linear_inequalities}{6}
\setcounter{lesson_compound_inequalities}{7}
\setcounter{lesson_inequalities_containing_absolute_values}{8}
\setcounter{lesson_graphing_systems}{9}
\setcounter{lesson_substitution}{10}
\setcounter{lesson_elimination}{11}
\setcounter{lesson_quadratics_introduction}{16}
\setcounter{lesson_factoring_GCF}{17}
\setcounter{lesson_factoring_grouping}{18}
\setcounter{lesson_factoring_trinomials_a_is_1}{19}
\setcounter{lesson_factoring_trinomials_a_neq_1}{20}
\setcounter{lesson_solving_by_factoring}{21}
\setcounter{lesson_square_roots}{22}
\setcounter{lesson_i_and_complex_numbers}{23}
\setcounter{lesson_vertex_form_and_graphing}{24}
\setcounter{lesson_solve_by_square_roots}{25}
\setcounter{lesson_extracting_square_roots}{26}
\setcounter{lesson_the_discriminant}{27}
\setcounter{lesson_the_quadratic_formula}{28}
\setcounter{lesson_quadratic_inequalities}{29}
\setcounter{lesson_functions_and_relations}{12}
\setcounter{lesson_evaluating_functions}{13}
\setcounter{lesson_finding_domain_and_range_graphically}{14}
\setcounter{lesson_fundamental_functions}{15}
\setcounter{lesson_finding_domain_algebraically}{30}
\setcounter{lesson_solving_functions}{31}
\setcounter{lesson_function_arithmetic}{32}
\setcounter{lesson_composite_functions}{33}
\setcounter{lesson_inverse_functions_definition_and_HLT}{34}
\setcounter{lesson_finding_an_inverse_function}{35}
\setcounter{lesson_transformations_translations}{36}
\setcounter{lesson_transformations_reflections}{37}
\setcounter{lesson_transformations_scalings}{38}
\setcounter{lesson_transformations_summary}{39}
\setcounter{lesson_piecewise_functions}{40}
\setcounter{lesson_functions_containing_absolute_values}{41}
\setcounter{lesson_absolute_as_piecewise}{42}
\setcounter{lesson_polynomials_introduction}{43}
\setcounter{lesson_sign_diagrams_polynomials}{44}
\setcounter{lesson_factoring_quadratic_type}{46}
\setcounter{lesson_factoring_summary}{45}
\setcounter{lesson_polynomial_division}{47}
\setcounter{lesson_synthetic_division}{48}
\setcounter{lesson_end_behavior_polynomials}{49}
\setcounter{lesson_local_behavior_polynomials}{50}
\setcounter{lesson_rational_root_theorem}{51}
\setcounter{lesson_polynomials_graphing_summary}{52}
\setcounter{lesson_polynomial_inequalities}{53}
\setcounter{lesson_rationals_introduction_and_terminology}{54}
\setcounter{lesson_sign_diagrams_rationals}{55}
\setcounter{lesson_horizontal_asymptotes}{56}
\setcounter{lesson_slant_and_curvilinear_asymptotes}{57}
\setcounter{lesson_vertical_asymptotes}{58}
\setcounter{lesson_holes}{59}
\setcounter{lesson_rationals_graphing_summary}{60}

\newcommand{\tmmathbf}[1]{\ensuremath{\boldsymbol{#1}}}
\newcommand{\tmop}[1]{\ensuremath{\operatorname{#1}}}

\begin{document}
\section{Solving by Factoring (L\arabic{lesson_solving_by_factoring})}
{\bf Objective: Solve polynomial equations by factoring and using the Zero Factor Property.}\par
When solving linear equations such as $2 x - 5 = 21$ we can solve for the variable directly by adding 5 and dividing by 2 to get 13. When working with quadratic equations (or higher degree polynomials), however, we cannot simply isolate the variable as we did with linear equations.  One property that we can use to solve for the variable is known as the zero factor property.
\[ \tmmathbf{\tmop{Zero} \tmop{Factor} \tmop{Property} : \tmop{If} a b = 0
   \tmop{then} \tmop{either} a = 0 \tmop{~or~} b = 0.} \]
The zero factor property tells us that if the product of two factors is zero, then one of the factors must be zero.  We can use this property to help us solve factored polynomials as in the following example.
\begin{example}~~~Solve the given equation for all possible values of $x$.
  \begin{eqnarray*}
    (2 x - 3) (5 x + 1) = 0~~~ &  & \tmop{One} \tmop{factor} \tmop{must}
    \tmop{be} \tmop{zero}\\
    2 x - 3 = 0 \tmop{~or~} 5 x + 1 = 0~~~ &  & \tmop{Set} \tmop{each}
    \tmop{factor} \tmop{equal} \tmop{to} \tmop{zero}\\
    \tmmathbf{\underline{+ 3 ~~+ 3} ~~~~~~~~ \underline{- 1 ~~- 1}} &  & \tmop{Solve} \tmop{each}
    \tmop{equation}\\
    2 x = 3 \tmop{~~~or~~~~} 5 x = - 1~ &  & \\
    \tmmathbf{\overline{2} ~~~~~ \overline{2} ~~~~~~~~~~~ \overline{5} ~~~~~~ \overline{5}}~~ &  & \\
    x = \frac{3}{2} \tmop{~or~} -\frac{1}{5}~~~~~ &  & \tmop{Our} \tmop{solution}
  \end{eqnarray*}
\end{example}
For the zero factor property to work we must have factors to set equal to zero. This means if an expression is not already factored, we must first factor it.
\begin{example}~~~Solve the given equation for all possible values of $x$.
  \begin{eqnarray*}
    4 x^2 + x - 3 = 0 &  & \tmop{Factor} \tmop{using} \tmop{the} ac
    \tmop{-method},\\
		&& \tmop{~~~multiply} \tmop{to} - 12, \tmop{~add} \tmop{to} 1\\
    4 x^2 - 3 x + 4 x - 3 = 0 &  & \tmop{The} \tmop{numbers} \tmop{are} - 3
    \tmop{and} 4, \tmop{~split} \tmop{the} \tmop{linear} \tmop{term}\\
    x (4 x - 3) + 1 (4 x - 3) = 0 &  & \tmop{Factor} \tmop{by}
    \tmop{grouping}
  \end{eqnarray*}
  \begin{eqnarray*}
    (4 x - 3) (x + 1) = 0~~~ &  & \tmop{One} \tmop{factor} \tmop{must} \tmop{be}
    \tmop{zero}\\
    4 x - 3 = 0 \tmop{~or~} x + 1 = 0~~~ &  & \tmop{Set} \tmop{each} \tmop{factor}
    \tmop{equal} \tmop{to} \tmop{zero}\\
    \tmmathbf{\underline{+ 3 ~~+ 3} ~~~~~~ \underline{- 1 ~~- 1}} &  & \tmop{Solve} \tmop{each}
    \tmop{equation}\\
    4 x = 3 \tmop{~~~~or~~~~} x = - 1 &  & \\
    \tmmathbf{\overline{4} ~~~~~ \overline{4}}~~~~~~~~~~~~~~~~~~~~~  &  & \\
    x = \frac{3}{4} \tmop{~or~} - 1~~~~~ &  & \tmop{Our} \tmop{solution}
  \end{eqnarray*}
\end{example}
Another important aspect of the zero factor property is that before we factor, our equation must equal zero. If it does not, we must move terms around so it does equal zero. Although it is not necessary, it will generally be easier to keep our leading term $ax^2$ positive.
\begin{example}~~~Solve the given equation for all possible values of $x$.
  \begin{eqnarray*}
    x^2 = 8 x - 15~~~ &  & \tmop{Set} \tmop{equal} \tmop{to} \tmop{zero}
    \tmop{by} \tmop{moving} \tmop{terms} \tmop{to} \tmop{the} \tmop{left}\\
    x^2 - 8 x + 15 = 0~~~ &  & \tmop{Factor} \tmop{using} \tmop{the} ac
    \tmop{-method},\\
		&&\tmop{~~~multiply} \tmop{to} 15, \tmop{~add} \tmop{to} - 8\\
    (x - 5) (x - 3) = 0~~~ &  & \tmop{The} \tmop{numbers} \tmop{are} - 5
    \tmop{and} - 3\\
    x - 5 = 0 \tmop{~or~} x - 3 = 0~~~ &  & \tmop{Set} \tmop{each} \tmop{factor}
    \tmop{equal} \tmop{to} \tmop{zero}\\
    \tmmathbf{\underline{+ 5 ~~+ 5} ~~~~~~ \underline{+ 3 ~~+ 3}} &  & \tmop{Solve} \tmop{each}
    \tmop{equation}\\
    x = 5 \tmop{~or~} 3~~~~~ &  & \tmop{Our} \tmop{solution}
  \end{eqnarray*}
\end{example}
\begin{example}~~~Solve the given equation for all possible values of $x$.
  \begin{eqnarray*}
    (x - 7) (x + 3) = - 9~~~ &  & \tmop{Not} \tmop{equal} \tmop{to} \tmop{zero},
    \tmop{~multiply} \tmop{first}\\
    x^2 - 7 x + 3 x - 21 = - 9~~~ &  & \tmop{Combine} \tmop{like} \tmop{terms}\\
    x^2 - 4 x - 21 = - 9~~~ &  & \tmop{Move} - 9 \tmop{to} \tmop{other}
    \tmop{side} \tmop{so} \tmop{equation} \tmop{equals} \tmop{zero}\\
    \tmmathbf{\underline{+ 9 ~~+ 9}}~~~ &  & \\
    x^2 - 4 x - 12 = 0~~~~~ &  & \tmop{Factor} \tmop{using} \tmop{the} ac
    \tmop{-method},\\
		&&\tmop{~~~multiply} \tmop{to} - 12, \tmop{~add} \tmop{to} - 4\\
    (x - 6) (x + 2) = 0~~~~~ &  & \tmop{The} \tmop{numbers} \tmop{are} 6 \tmop{and}
    - 2\\
    x - 6 = 0 \tmop{~or~} x + 2 = 0~~~~~ &  & \tmop{Set} \tmop{each} \tmop{factor}
    \tmop{equal} \tmop{to} \tmop{zero}\\
    \tmmathbf{\underline{+ 6 ~~+ 6} ~~~~~~ \underline{- 2 ~~- 2}}~~ &  & \tmop{Solve} \tmop{each}
    \tmop{equation}\\
    x = 6 \tmop{~or~} - 2~~~~~~~ &  & \tmop{Our} \tmop{solution}
  \end{eqnarray*}
\end{example}
\begin{example}~~~Solve the given equation for all possible values of $x$.
  \begin{eqnarray*}
    3 x^2 + 4 x - 5 = 7 x^2 + 4 x - 14~~~ &  & \tmop{Set} \tmop{equal} \tmop{to}
    \tmop{zero} \tmop{by}\\
		&&~~~\tmop{moving} \tmop{terms} \tmop{to} \tmop{the} \tmop{right}\\
    0 = 4 x^2 - 9~~~ &  & \tmop{Factor} \tmop{using} \tmop{difference} \tmop{of}
    \tmop{squares}\\
    0 = (2 x + 3) (2 x - 3)~~~ &  & \tmop{One} \tmop{factor} \tmop{must}
    \tmop{be} \tmop{zero}\\
    2 x + 3 = 0 \tmop{~or~} 2 x - 3 = 0~~~ &  & \tmop{Set} \tmop{each}
    \tmop{factor} \tmop{equal} \tmop{to} \tmop{zero}\\
    \tmmathbf{\underline{- 3 ~- 3} ~~~~~~~~ \underline{+ 3 ~~+ 3}} &  & \tmop{Solve} \tmop{each}
    \tmop{equation}\\
    2 x = - 3 \tmop{~~or~~~} 2 x = 3~~~ &  	& \\
    \tmmathbf{\overline{2} ~~~~~~ \overline{2} ~~~~~~~~~~ \overline{2} ~~~~~ \overline{2}}~~~ &  & \\
    x = -\frac{3}{2} \tmop{~or~} \frac{3}{2}~~~~~ &  & \tmop{Our} \tmop{solution}
  \end{eqnarray*}
\end{example}
Most quadratic equations will have two unique real solutions. It is possible, however, to have only one real solution as the next example illustrates.
\begin{example}~~~Solve the given equation for all possible values of $x$.
  \begin{eqnarray*}
    4 x^2 = 12 x - 9~~~ &  & \tmop{Set} \tmop{equal} \tmop{to} \tmop{zero}
    \tmop{by} \tmop{moving} \tmop{terms} \tmop{to} \tmop{left}\\
    4 x^2 - 12 x + 9 = 0~~~ &  & \tmop{Factor} \tmop{using} \tmop{the} ac
    \tmop{-method},\\
		&& \tmop{~~~multiply} \tmop{to} 36, \tmop{~add} \tmop{to} - 12\\
		4 x^2 - 6 x -6x + 9 = 0~~~ &  & \tmop{Use} - 6 \tmop{and} - 6, \text{~split~the~linear~term}\\ 
    2x(2x-3)-3(2x-3)=0~~~ & & \text{Factor~by~grouping}\\
		(2 x - 3)^2 = 0~~~ &  & \tmop{A~perfect} \tmop{square}!\\
    2 x - 3 = 0~~~ &  & \tmop{Set} \tmop{this} \tmop{factor} \tmop{equal}
    \tmop{to} \tmop{zero}\\
    \tmmathbf{\underline{+ 3 ~~+ 3}} &  & \tmop{Solve} \tmop{the} \tmop{equation}\\
    2 x = 3~~~ &  & \\
    \tmmathbf{\overline{2} ~~~~ \overline{2}}~~~ &  & \\
    x = \frac{3}{2}~~~ &  & \tmop{Our} \tmop{solution}
  \end{eqnarray*}
\end{example}
As always, it will be important to factor out the GCF first if we have one. This GCF is also a factor, and therefore must also be set equal to zero using the zero factor property.  The next example illustrates this.
\begin{example}~~~Solve the given equation for all possible values of $x$.
  \begin{eqnarray*}
    4 x^2 = 8 x~~ &  & \tmop{Set} \tmop{equal} \tmop{to} \tmop{zero} \tmop{by}
    \tmop{moving} \tmop{the} \tmop{terms} \tmop{to} \tmop{left}\\
 &  & \tmop{Be} \tmop{careful}, 4x^2 \tmop{~and~} 8x \tmop{are} \tmop{not} \tmop{like}
    \tmop{terms} !\\
    4 x^2 - 8 x = 0~~~ &  & \tmop{Factor} \tmop{out} \tmop{the} \tmop{GCF}
    \tmop{of} 4 x\\
    4 x (x - 2) = 0~~~ &  & \tmop{One} \tmop{factor} \tmop{must} \tmop{be}
    \tmop{zero}\\
    4 x = 0 \tmop{~or~} x - 2 = 0~~~ &  & \tmop{Set} \tmop{each} \tmop{factor}
    \tmop{equal} \tmop{to} \tmop{zero}\\
    \tmmathbf{\overline{4} ~~~~ \overline{4} ~~~~~~~~~ \underline{+ 2 ~~+ 2}} &  & \tmop{Solve}
    \tmop{each} \tmop{equation}\\
    x = 0 \tmop{~or~} 2~~~~~ &  & \tmop{Our} \tmop{solution}
  \end{eqnarray*}
\end{example}
If our polynomial is not a quadratic, as in the next example, we may end up with more than two solutions.
\begin{example}~~~Solve the given equation for all possible values of $x$.
  \begin{eqnarray*}
    2 x^3 - 14 x^2 + 24 x = 0~~~ &  & \tmop{Factor} \tmop{out} \tmop{the}
    \tmop{GCF} \tmop{of} 2 x\\
    2 x (x^2 - 7 x + 12) = 0~~~ &  & \tmop{Factor} \tmop{with} ac
    \tmop{-method},\\
		&& \tmop{~~~multiply} \tmop{to} 12, \tmop{~add} \tmop{to} - 7\\
    2 x (x - 3) (x - 4) = 0~~~ &  & \tmop{The} \tmop{numbers} \tmop{are} - 3
    \tmop{and} - 4\\
    2 x = 0 \tmop{~or~} x - 3 = 0 \tmop{~or~} x - 4 = 0~~~ &  & \tmop{Set}
    \tmop{each} \tmop{factor} \tmop{equal} \tmop{to} \tmop{zero}\\
    \tmmathbf{\overline{2} ~~~~~ \overline{2} ~~~~~~~~ \underline{+ 3 ~~+ 3} ~~~~~~ \underline{+ 4 ~~+ 4}} &  &
    \tmop{Solve} \tmop{each} \tmop{equation}\\
    x = 0 \tmop{~~or~~} 3 \tmop{~~or~~} 4~~~~~~~ &  & \tmop{Our} \tmop{solution}
  \end{eqnarray*}
\end{example}
\begin{example}~~~Solve the given equation for all possible values of $x$.
  \begin{eqnarray*}
    6 x^2 + 21 x - 27 = 0 &  & \tmop{Factor} \tmop{out} \tmop{the} \tmop{GCF}
    \tmop{of} 3\\
    3 (2 x^2 + 7 x - 9) = 0 &  & \tmop{Factor} \tmop{with} ac
    \tmop{-method},\\
		&&\tmop{~~~multiply} \tmop{to} - 18, \tmop{add} \tmop{to} 7\\
    3 (2 x^2 + 9 x - 2 x - 9) = 0 &  & \tmop{The} \tmop{numbers} \tmop{are} 9
    \tmop{and} - 2\\
    3 [x (2 x + 9) - 1 (2 x + 9)] = 0 &  & \tmop{Factor} \tmop{by}
    \tmop{grouping}\\
    3 (2 x + 9) (x - 1) = 0 &  & \tmop{One} \tmop{factor} \tmop{must}
    \tmop{be} \tmop{zero}
  \end{eqnarray*}
	\begin{eqnarray*}
		3 = 0 \tmop{~or~} 2 x + 9 = 0 \tmop{~or~} x - 1 = 0~~~ &  & \tmop{Set}
    \tmop{each} \tmop{factor} \tmop{equal} \tmop{to} \tmop{zero}\\
    \tmmathbf{3 \neq 0 ~~~~~~~~~\underline{- 9 ~~- 9} ~~~~~~ \underline{+ 1 ~~+ 1}} &  & \tmop{Solve}
    \tmop{each} \tmop{equation}\\
    2 x = - 9 \tmop{~~~or~~} x = 1~~~ &  & \\
    \tmmathbf{\overline{2} ~~~~~~~ \overline{2}}~~~~~~~~~~~~~~~~~~~  &  & \\
    x = - \frac{9}{2} \tmop{~or~} 1~~~~~ &  & \tmop{Our} \tmop{solution}
  \end{eqnarray*}
\end{example}
In the previous example, the GCF did not have a variable in it. When we set this factor equal to zero we got a false statement. No solutions come from this factor. We can only disregard setting the GCF factor equal to zero if it is a constant.\par
Just as not all polynomials can be easily factored, all equations cannot be easily solved by factoring.  If an equation does not factor easily, we will have to solve it using another method. These other methods are saved for another section.\par
It is a common question to ask if it is permissible to get rid of the square on the variable $x^2$ by taking the square root of both sides of the equation. Although it is sometimes possible, there are a few properties of square roots that we have not covered yet, and thus it
is more common to inadvertently break a rule of roots that we may not yet be aware of.  Because of this, we will postpone a discussion of roots until we see how they can be employed properly to solve quadratic equations.  For now, we will advise to {\it not} take the
square root of both sides of an equation!
\end{document}