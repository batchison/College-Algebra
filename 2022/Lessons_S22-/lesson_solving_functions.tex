\documentclass[12pt]{article}
\usepackage[top=1in,left=1in,bottom=1in,right=1in,headsep=2pt]{geometry}	
\usepackage{amssymb,amsmath,amsthm,amsfonts}
\usepackage{chapterfolder,docmute,setspace}
\usepackage{cancel,multicol,tikz,verbatim,framed,polynom,enumitem}
\usepackage[colorlinks, hyperindex, plainpages=false, linkcolor=blue, urlcolor=blue, pdfpagelabels]{hyperref}
% Use the cc-by-nc-sa license for any content linked with Stitz and Zeager's text.  Otherwise, use the cc-by-sa license.
%\usepackage[type={CC},modifier={by-sa},version={4.0},]{doclicense}
\usepackage[type={CC},modifier={by-nc-sa},version={4.0},]{doclicense}

\theoremstyle{definition}
\newtheorem{example}{Example}
\newcommand{\Desmos}{\href{https://www.desmos.com/}{Desmos}}
\setlength{\parindent}{0em}
\setlist{itemsep=0em}
\setlength{\parskip}{0.1em}
% This document is used for ordering of lessons.  If an instructor wishes to change the ordering of assessments, the following steps must be taken:

% 1) Reassign the appropriate numbers for each lesson in the \setcounter commands included in this file.
% 2) Rearrange the \include commands in the master file (the file with 'Course Pack' in the name) to accurately reflect the changes.  
% 3) Rearrange the \items in the measureable_outcomes file to accurately reflect the changes.  Be mindful of page breaks when moving items.
% 4) Re-build all affected files (master file, measureable_outcomes file, and any lesson whose numbering has changed).

%Note: The placement of each \newcounter and \setcounter command reflects the original/default ordering of topics (linears, systems, quadratics, functions, polynomials, rationals).

\newcounter{lesson_solving_linear_equations}
\newcounter{lesson_equations_containing_absolute_values}
\newcounter{lesson_graphing_lines}
\newcounter{lesson_two_forms_of_a_linear_equation}
\newcounter{lesson_parallel_and_perpendicular_lines}
\newcounter{lesson_linear_inequalities}
\newcounter{lesson_compound_inequalities}
\newcounter{lesson_inequalities_containing_absolute_values}
\newcounter{lesson_graphing_systems}
\newcounter{lesson_substitution}
\newcounter{lesson_elimination}
\newcounter{lesson_quadratics_introduction}
\newcounter{lesson_factoring_GCF}
\newcounter{lesson_factoring_grouping}
\newcounter{lesson_factoring_trinomials_a_is_1}
\newcounter{lesson_factoring_trinomials_a_neq_1}
\newcounter{lesson_solving_by_factoring}
\newcounter{lesson_square_roots}
\newcounter{lesson_i_and_complex_numbers}
\newcounter{lesson_vertex_form_and_graphing}
\newcounter{lesson_solve_by_square_roots}
\newcounter{lesson_extracting_square_roots}
\newcounter{lesson_the_discriminant}
\newcounter{lesson_the_quadratic_formula}
\newcounter{lesson_quadratic_inequalities}
\newcounter{lesson_functions_and_relations}
\newcounter{lesson_evaluating_functions}
\newcounter{lesson_finding_domain_and_range_graphically}
\newcounter{lesson_fundamental_functions}
\newcounter{lesson_finding_domain_algebraically}
\newcounter{lesson_solving_functions}
\newcounter{lesson_function_arithmetic}
\newcounter{lesson_composite_functions}
\newcounter{lesson_inverse_functions_definition_and_HLT}
\newcounter{lesson_finding_an_inverse_function}
\newcounter{lesson_transformations_translations}
\newcounter{lesson_transformations_reflections}
\newcounter{lesson_transformations_scalings}
\newcounter{lesson_transformations_summary}
\newcounter{lesson_piecewise_functions}
\newcounter{lesson_functions_containing_absolute_values}
\newcounter{lesson_absolute_as_piecewise}
\newcounter{lesson_polynomials_introduction}
\newcounter{lesson_sign_diagrams_polynomials}
\newcounter{lesson_factoring_quadratic_type}
\newcounter{lesson_factoring_summary}
\newcounter{lesson_polynomial_division}
\newcounter{lesson_synthetic_division}
\newcounter{lesson_end_behavior_polynomials}
\newcounter{lesson_local_behavior_polynomials}
\newcounter{lesson_rational_root_theorem}
\newcounter{lesson_polynomials_graphing_summary}
\newcounter{lesson_polynomial_inequalities}
\newcounter{lesson_rationals_introduction_and_terminology}
\newcounter{lesson_sign_diagrams_rationals}
\newcounter{lesson_horizontal_asymptotes}
\newcounter{lesson_slant_and_curvilinear_asymptotes}
\newcounter{lesson_vertical_asymptotes}
\newcounter{lesson_holes}
\newcounter{lesson_rationals_graphing_summary}

\setcounter{lesson_solving_linear_equations}{1}
\setcounter{lesson_equations_containing_absolute_values}{2}
\setcounter{lesson_graphing_lines}{3}
\setcounter{lesson_two_forms_of_a_linear_equation}{4}
\setcounter{lesson_parallel_and_perpendicular_lines}{5}
\setcounter{lesson_linear_inequalities}{6}
\setcounter{lesson_compound_inequalities}{7}
\setcounter{lesson_inequalities_containing_absolute_values}{8}
\setcounter{lesson_graphing_systems}{9}
\setcounter{lesson_substitution}{10}
\setcounter{lesson_elimination}{11}
\setcounter{lesson_quadratics_introduction}{16}
\setcounter{lesson_factoring_GCF}{17}
\setcounter{lesson_factoring_grouping}{18}
\setcounter{lesson_factoring_trinomials_a_is_1}{19}
\setcounter{lesson_factoring_trinomials_a_neq_1}{20}
\setcounter{lesson_solving_by_factoring}{21}
\setcounter{lesson_square_roots}{22}
\setcounter{lesson_i_and_complex_numbers}{23}
\setcounter{lesson_vertex_form_and_graphing}{24}
\setcounter{lesson_solve_by_square_roots}{25}
\setcounter{lesson_extracting_square_roots}{26}
\setcounter{lesson_the_discriminant}{27}
\setcounter{lesson_the_quadratic_formula}{28}
\setcounter{lesson_quadratic_inequalities}{29}
\setcounter{lesson_functions_and_relations}{12}
\setcounter{lesson_evaluating_functions}{13}
\setcounter{lesson_finding_domain_and_range_graphically}{14}
\setcounter{lesson_fundamental_functions}{15}
\setcounter{lesson_finding_domain_algebraically}{30}
\setcounter{lesson_solving_functions}{31}
\setcounter{lesson_function_arithmetic}{32}
\setcounter{lesson_composite_functions}{33}
\setcounter{lesson_inverse_functions_definition_and_HLT}{34}
\setcounter{lesson_finding_an_inverse_function}{35}
\setcounter{lesson_transformations_translations}{36}
\setcounter{lesson_transformations_reflections}{37}
\setcounter{lesson_transformations_scalings}{38}
\setcounter{lesson_transformations_summary}{39}
\setcounter{lesson_piecewise_functions}{40}
\setcounter{lesson_functions_containing_absolute_values}{41}
\setcounter{lesson_absolute_as_piecewise}{42}
\setcounter{lesson_polynomials_introduction}{43}
\setcounter{lesson_sign_diagrams_polynomials}{44}
\setcounter{lesson_factoring_quadratic_type}{46}
\setcounter{lesson_factoring_summary}{45}
\setcounter{lesson_polynomial_division}{47}
\setcounter{lesson_synthetic_division}{48}
\setcounter{lesson_end_behavior_polynomials}{49}
\setcounter{lesson_local_behavior_polynomials}{50}
\setcounter{lesson_rational_root_theorem}{51}
\setcounter{lesson_polynomials_graphing_summary}{52}
\setcounter{lesson_polynomial_inequalities}{53}
\setcounter{lesson_rationals_introduction_and_terminology}{54}
\setcounter{lesson_sign_diagrams_rationals}{55}
\setcounter{lesson_horizontal_asymptotes}{56}
\setcounter{lesson_slant_and_curvilinear_asymptotes}{57}
\setcounter{lesson_vertical_asymptotes}{58}
\setcounter{lesson_holes}{59}
\setcounter{lesson_rationals_graphing_summary}{60}

\begin{document}
{\bf \large Lesson \arabic{lesson_solving_functions}: Solving Functions}\phantomsection\label{les:solving_functions}
%\\ CC attribute: \href{http://www.wallace.ccfaculty.org/book/book.html}{\it{Beginning and Intermediate Algebra}} by T. Wallace. 
\\ CC attribute: \href{http://www.stitz-zeager.com}{\it{College Algebra}} by C. Stitz and J. Zeager. 
\hfill \doclicenseImage[imagewidth=5em]\\
\par
{\bf Objective:} Solve functions using appropriate notation.\\
\par
{\bf Students will be able to:}
\begin{itemize}
	\item Solve an equation to determine a set of inputs that produce a given output.
\end{itemize}
{\bf Prerequisite Knowledge:}
\begin{itemize}
	\item Order of operations.
	\item Isolating a variable.
\end{itemize}
\hrulefill

{\bf Lesson:}\\
\ \par
In a previous lesson we discussed evaluating functions for a given input, $x=a$.  In this lesson, we focus on {\it solving} functions for a particular output, $y=a$, or $f(x)=a$.  There is a fundamental difference between evaluating, $f(a)$, and solving functions, $f(x)=a$.  To see this, we will pay close attention to when $a=0$ and focus on the graphical implications.\\
\ \par 
When we are evaluating $f(0),$ we are finding the $y-$coordinate associated with when $x=0$.  This corresponds to the $y-$intercept in our graph of $f$.  Alternatively, when we are solving $f(x)=0,$ we are finding {\it any and all} $x-$coordinates associated with when $y=0$.  This corresponds to the (set of) $x-$intercept(s) in the graph of $f$.\\
\ \par
Whereas evaluating functions requires us to use the standard order of operations, PEMDAS, solving a function usually requires us to use the {\it reverse} order of operations, SADMEP, in addition to other methods, such as factoring.  We do this in order to {\it isolate} the variable $x$.\\
\ \par 
{\bf I - Motivating Example(s):}\\
\ \par
{\bf Example:} Given $f(x)=x^2+3x+5$, find all $x$ such that $f(x)=5$.
  \begin{eqnarray*}
			f (x) = x^2+3x+5 & & \text{Substitute 5 in for } f(x)\\
			5 = x^2+3x+5  &  & \text{Solve for } x \text{ by factoring}\\
			0 = x^2+3x  &  & \text{Set equal to 0}\\
			0 = x(x+3)  &  & \text{Factor}\\
			x=0 \text{ or } x= -3  & & \text{Our solutions}
\end{eqnarray*}
The above answer can be verified by checking.  When we input $x=0$ into the function, we simplify to find that $f(0)=5$.  Similarly, we see that when $x=-3$, $f(-3)=5$.
\newpage
{\bf II - Demo/Discussion Problems:}%\\
%\ \par
\begin{enumerate}
	\item Given $h(x)=4x-1,$ find all $x$ such that $h(x)=-3$.\\
	\item Given $g(x)=\dfrac{1}{4x-1},$ find all $x$ such that $g(x)=0$.\\
	\item Given $k(x)=2|2x-3|+3,$ find all $x$ such that $k(x)=11$.\\
	\item Given $\ell(x)=\sqrt{3-5x},$ find all $x$ such that $\ell(x)=0$\\
	\item Given $m(x)=-\sqrt{25-x^2},$ find all $x$ such that $m(x)=0$\\
	\item Given $f(x)=x^2-10x+28,$ find all $x$ such that $f(x)=5$.\\
	{\bf Hint}: Use either the Vertex Form or the Quadratic Formula.
\end{enumerate}
\ \par
{\bf III - Practice Problems:}\\
\ \par
Find $f(0)$ and solve $f(x) = 0$ for each of the given functions.
\begin{center}
\begin{multicols}{2}
\begin{enumerate}
\item $f(x) = 2x - 1$\\
\item $f(x) = 3 - \dfrac{2x}{5}$\\
\item $f(x) = 2x^2 - 6$\\
\item $f(x) = x^2 - x - 12$\\
\item $f(x) = \sqrt{x+4}$\\
\item $f(x) = \dfrac{1}{2}\sqrt{1-2x}$\\
\item $f(x)=\sqrt{20-x^2}$\\
\item $f(x) = \dfrac{3}{4-x}$\\
\item $f(x) = \dfrac{3x^2-12x}{4-x^2}$\\
\end{enumerate}
\end{multicols}
\end{center}
Find when $f(x)=2$ for each of the functions above.
\newpage
\end{document}