\documentclass[12pt]{article}
\usepackage[top=1in,left=1in,bottom=1in,right=1in,headsep=2pt]{geometry}	
\usepackage{amssymb,amsmath,amsthm,amsfonts}
\usepackage{chapterfolder,docmute,setspace}
\usepackage{cancel,multicol,tikz,verbatim,framed,polynom,enumitem}
\usepackage[colorlinks, hyperindex, plainpages=false, linkcolor=blue, urlcolor=blue, pdfpagelabels]{hyperref}
% Use the cc-by-nc-sa license for any content linked with Stitz and Zeager's text.  Otherwise, use the cc-by-sa license.
%\usepackage[type={CC},modifier={by-sa},version={4.0},]{doclicense}
\usepackage[type={CC},modifier={by-nc-sa},version={4.0},]{doclicense}

\theoremstyle{definition}
\newtheorem{example}{Example}
\newcommand{\Desmos}{\href{https://www.desmos.com/}{Desmos}}
\setlength{\parindent}{0em}
\setlist{itemsep=0em}
\setlength{\parskip}{0.1em}
% This document is used for ordering of lessons.  If an instructor wishes to change the ordering of assessments, the following steps must be taken:

% 1) Reassign the appropriate numbers for each lesson in the \setcounter commands included in this file.
% 2) Rearrange the \include commands in the master file (the file with 'Course Pack' in the name) to accurately reflect the changes.  
% 3) Rearrange the \items in the measureable_outcomes file to accurately reflect the changes.  Be mindful of page breaks when moving items.
% 4) Re-build all affected files (master file, measureable_outcomes file, and any lesson whose numbering has changed).

%Note: The placement of each \newcounter and \setcounter command reflects the original/default ordering of topics (linears, systems, quadratics, functions, polynomials, rationals).

\newcounter{lesson_solving_linear_equations}
\newcounter{lesson_equations_containing_absolute_values}
\newcounter{lesson_graphing_lines}
\newcounter{lesson_two_forms_of_a_linear_equation}
\newcounter{lesson_parallel_and_perpendicular_lines}
\newcounter{lesson_linear_inequalities}
\newcounter{lesson_compound_inequalities}
\newcounter{lesson_inequalities_containing_absolute_values}
\newcounter{lesson_graphing_systems}
\newcounter{lesson_substitution}
\newcounter{lesson_elimination}
\newcounter{lesson_quadratics_introduction}
\newcounter{lesson_factoring_GCF}
\newcounter{lesson_factoring_grouping}
\newcounter{lesson_factoring_trinomials_a_is_1}
\newcounter{lesson_factoring_trinomials_a_neq_1}
\newcounter{lesson_solving_by_factoring}
\newcounter{lesson_square_roots}
\newcounter{lesson_i_and_complex_numbers}
\newcounter{lesson_vertex_form_and_graphing}
\newcounter{lesson_solve_by_square_roots}
\newcounter{lesson_extracting_square_roots}
\newcounter{lesson_the_discriminant}
\newcounter{lesson_the_quadratic_formula}
\newcounter{lesson_quadratic_inequalities}
\newcounter{lesson_functions_and_relations}
\newcounter{lesson_evaluating_functions}
\newcounter{lesson_finding_domain_and_range_graphically}
\newcounter{lesson_fundamental_functions}
\newcounter{lesson_finding_domain_algebraically}
\newcounter{lesson_solving_functions}
\newcounter{lesson_function_arithmetic}
\newcounter{lesson_composite_functions}
\newcounter{lesson_inverse_functions_definition_and_HLT}
\newcounter{lesson_finding_an_inverse_function}
\newcounter{lesson_transformations_translations}
\newcounter{lesson_transformations_reflections}
\newcounter{lesson_transformations_scalings}
\newcounter{lesson_transformations_summary}
\newcounter{lesson_piecewise_functions}
\newcounter{lesson_functions_containing_absolute_values}
\newcounter{lesson_absolute_as_piecewise}
\newcounter{lesson_polynomials_introduction}
\newcounter{lesson_sign_diagrams_polynomials}
\newcounter{lesson_factoring_quadratic_type}
\newcounter{lesson_factoring_summary}
\newcounter{lesson_polynomial_division}
\newcounter{lesson_synthetic_division}
\newcounter{lesson_end_behavior_polynomials}
\newcounter{lesson_local_behavior_polynomials}
\newcounter{lesson_rational_root_theorem}
\newcounter{lesson_polynomials_graphing_summary}
\newcounter{lesson_polynomial_inequalities}
\newcounter{lesson_rationals_introduction_and_terminology}
\newcounter{lesson_sign_diagrams_rationals}
\newcounter{lesson_horizontal_asymptotes}
\newcounter{lesson_slant_and_curvilinear_asymptotes}
\newcounter{lesson_vertical_asymptotes}
\newcounter{lesson_holes}
\newcounter{lesson_rationals_graphing_summary}

\setcounter{lesson_solving_linear_equations}{1}
\setcounter{lesson_equations_containing_absolute_values}{2}
\setcounter{lesson_graphing_lines}{3}
\setcounter{lesson_two_forms_of_a_linear_equation}{4}
\setcounter{lesson_parallel_and_perpendicular_lines}{5}
\setcounter{lesson_linear_inequalities}{6}
\setcounter{lesson_compound_inequalities}{7}
\setcounter{lesson_inequalities_containing_absolute_values}{8}
\setcounter{lesson_graphing_systems}{9}
\setcounter{lesson_substitution}{10}
\setcounter{lesson_elimination}{11}
\setcounter{lesson_quadratics_introduction}{16}
\setcounter{lesson_factoring_GCF}{17}
\setcounter{lesson_factoring_grouping}{18}
\setcounter{lesson_factoring_trinomials_a_is_1}{19}
\setcounter{lesson_factoring_trinomials_a_neq_1}{20}
\setcounter{lesson_solving_by_factoring}{21}
\setcounter{lesson_square_roots}{22}
\setcounter{lesson_i_and_complex_numbers}{23}
\setcounter{lesson_vertex_form_and_graphing}{24}
\setcounter{lesson_solve_by_square_roots}{25}
\setcounter{lesson_extracting_square_roots}{26}
\setcounter{lesson_the_discriminant}{27}
\setcounter{lesson_the_quadratic_formula}{28}
\setcounter{lesson_quadratic_inequalities}{29}
\setcounter{lesson_functions_and_relations}{12}
\setcounter{lesson_evaluating_functions}{13}
\setcounter{lesson_finding_domain_and_range_graphically}{14}
\setcounter{lesson_fundamental_functions}{15}
\setcounter{lesson_finding_domain_algebraically}{30}
\setcounter{lesson_solving_functions}{31}
\setcounter{lesson_function_arithmetic}{32}
\setcounter{lesson_composite_functions}{33}
\setcounter{lesson_inverse_functions_definition_and_HLT}{34}
\setcounter{lesson_finding_an_inverse_function}{35}
\setcounter{lesson_transformations_translations}{36}
\setcounter{lesson_transformations_reflections}{37}
\setcounter{lesson_transformations_scalings}{38}
\setcounter{lesson_transformations_summary}{39}
\setcounter{lesson_piecewise_functions}{40}
\setcounter{lesson_functions_containing_absolute_values}{41}
\setcounter{lesson_absolute_as_piecewise}{42}
\setcounter{lesson_polynomials_introduction}{43}
\setcounter{lesson_sign_diagrams_polynomials}{44}
\setcounter{lesson_factoring_quadratic_type}{46}
\setcounter{lesson_factoring_summary}{45}
\setcounter{lesson_polynomial_division}{47}
\setcounter{lesson_synthetic_division}{48}
\setcounter{lesson_end_behavior_polynomials}{49}
\setcounter{lesson_local_behavior_polynomials}{50}
\setcounter{lesson_rational_root_theorem}{51}
\setcounter{lesson_polynomials_graphing_summary}{52}
\setcounter{lesson_polynomial_inequalities}{53}
\setcounter{lesson_rationals_introduction_and_terminology}{54}
\setcounter{lesson_sign_diagrams_rationals}{55}
\setcounter{lesson_horizontal_asymptotes}{56}
\setcounter{lesson_slant_and_curvilinear_asymptotes}{57}
\setcounter{lesson_vertical_asymptotes}{58}
\setcounter{lesson_holes}{59}
\setcounter{lesson_rationals_graphing_summary}{60}

\begin{document}
{\bf \large Lesson \arabic{lesson_rationals_graphing_summary}: Rational Functions Graphing Summary}\phantomsection\label{les:rationals_graphing_summary}
%\\ CC attribute: \href{http://www.wallace.ccfaculty.org/book/book.html}{\it{Beginning and Intermediate Algebra}} by T. Wallace. 
\\ CC attribute: \href{http://www.stitz-zeager.com}{\it{College Algebra}} by C. Stitz and J. Zeager. 
\hfill \doclicenseImage[imagewidth=5em]\\
\par
{\bf Objective:} Graph a rational function in its entirety.\\
\par
{\bf Students will be able to:}
\begin{itemize}
	\item Identify all important aspects of the graph of a rational function: intercepts, holes, and any vertical, horizontal, slant, or curvilinear asymptotes.
	\item Sketch a complete graph of a rational function using a sign diagram.
\end{itemize}
{\bf Prerequisite Knowledge:}
\begin{itemize}
	\item Evaluating a function.
	\item Factoring.
	\item The Rational Root Theorem.
	\item Polynomial and \slash or Synthetic Division.
	\item Multiplicative identity \slash inverse.
	\item Sign Diagrams.
	\item Rational function features.
\end{itemize}
\hrulefill

{\bf Lesson:}\\
\ \par
At this point, we have addressed all key features of rational functions individually.  This section pulls each of these aspects together, for a detailed analysis of a rational function, culminating in a complete sketch of its graph.  Along the way, we will need to address each of the following aspects for our rational function $f(x)=\dfrac{p(x)}{q(x)}$.  It is important to note that there is no universally accepted order to this checklist.
\begin{itemize}
	\item Find the $y-$intercept of the graph of $f,$ $(0,f(0)),$ if it exists.
	\item Use the degrees and leading coefficients of $p$ and $q$ to determine whether the graph of $f$ has a horizontal asymptote.  If the graph of $f$ has a slant asymptote, use polynomial division to find where it is located.
	\item Identify a complete factorization of $f,$ and use it to find the domain of the function.  This is the set of all $x,$ such that $q(x)\neq 0$.
	\item Find any $x-$intercepts of the graph of $f$.  This is the set of all $x$ in the domain of $f,$ such that $p(x)=0$.  Using multiplicities, classify each $x-$intercept as a crossover or turnaround (``bounce'') point.
	\item Find the simplified expression $g$ for the given function $f$, and use it to identify any vertical asymptotes or holes in the graph of $f$.  Use multiplicities to help visualize the nature of the graph of $f$ near its vertical asymptotes.  If $f$ has a hole at $x=c,$ use $g$ to help plot the hole's precise location at $(c,g(c))$. 
	\item Using both the $x-$intercepts and the discontinuities (those $x$ not in the domain), construct a sign diagram for $f$.
\end{itemize}
In each rational function we encounter, we will carefully examine the function, making sure not to omit any of the checklist items above and to compare each item to those that precede it along the way for accuracy.  Although the process will take some time, if we are thorough, our end result should be a complete, accurate sketch of the given rational function.\\
\ \par
{\bf I - Motivating Example(s):}\\
\ \par
{\bf Example:} Sketch a complete graph of the rational function below, making sure to have a clearly defined scale and label all key features of your graph (intercepts, asymptotes, and holes).
$$f(x)=\dfrac{4x^2}{x^3+3x^2-4x}$$
In this example, we see that the graph of $f$ will not have a $y-$intercept, since $f(0)=\frac{0}{0},$ which is undefined.\\
\ \par
Since the degree of the numerator is less than the degree of the denominator, we conclude that the graph of $f$ has a horizontal asymptote along the $x-$axis, $y=0$.\\
\ \par
Our graph also has no $x-$intercepts, since our numerator only equals zero when $x=0,$ which we know is not in our domain of $f$.\\
\ \par
Furthermore, using the knowledge from our previous lessons, we can find the following factorization of $f$.
$$f(x)=\dfrac{4x^2}{x(x+4)(x-1)}$$
Consequently, $f$ has corresponding domain $x\neq -4,0,1,$ and related simplified expression
$$g(x)=\dfrac{4x}{(x+4)(x-1)}.$$
The graph of $f$ has vertical asymptotes at $x=-4$ and $x=1$ and a hole at $(0,g(0))=(0,0)$.\\
\ \par
Since the multiplicities of both $x=-4$ and $x=1$ in the denominator of $f$ are both one (odd), we know that the graph of $f$ will approach each vertical asymptote from opposite sides of the $x-$axis.  The following sign diagram confirms this observation.
\begin{center}
\begin{tikzpicture}[xscale=1,yscale=1]
	\draw [<->](-6.25,0) -- coordinate (x axis mid) (4.25,0) node[below right] {$x$};
	\draw [-, dashed](-4,1) -- coordinate (y axis mid) (-4,-0.25) node[below] {$-4$};
	\draw [-, dashed](-1,1) -- coordinate (y axis mid) (-1,-0.25) node[below] {$0$};
	\draw [-, dashed](2,1) -- coordinate (y axis mid) (2,-0.25) node[below] {$1$};
	\draw (-5,-1) node {$x=-5$};
	\draw (-2.5,-1) node {$x=-1$};
	\draw (0.5,-1) node {$x=\frac{1}{2}$};
	\draw (3,-1) node {$x=2$};
	\draw (-5,0.5) node {$-$};
	\draw (-2.5,0.5) node {$+$};
	\draw (0.5,0.5) node {$-$};
	\draw (3,0.5) node {$+$};
\end{tikzpicture}
\end{center}
We are now ready to try our hand at graphing $f,$ and begin our graph by defining a scale for both the $x-$ and $y-$axes, and identifying all intercepts, and asymptotes.  This should always be our first step to successfully sketching a decent-looking graph.  To emphasize this point, we first show an initial graph that identifies each of these features, and further shades those areas of the $xy-$plane that correspond to our sign diagram above.
\begin{center}
\begin{tikzpicture}[xscale=0.5,yscale=0.5]
	\fill[color=lightgray] (-8.25,-6.5)--(-8.25,0)--(-4,0)--(-4,-6.5)--cycle;
	\fill[color=lightgray] (-4,0)--(-4,7)--(0,7)--(0,0)--cycle;
	\fill[color=lightgray] (0,0)--(0,-6.5)--(1,-6.5)--(1,0)--cycle;
	\fill[color=lightgray] (1,0)--(1,7)--(8.25,7)--(8.25,0)--cycle;
	\draw [-](-8.25,0) -- coordinate (x axis mid) (8.25,0) node[below right] {};
	\draw [<->,dashed](-10.25,0) -- coordinate (x axis mid) (10.25,0) node[below right] {$x$};
	\draw [<->](0,-7.25) -- coordinate (x axis mid) (0,7.25) node[above right] {$y$};
	\draw [<->,dashed](-4,7) -- (-4,-6.5) node[below] {$x=-4$};
	\draw [<->,dashed](1,7) -- (1,-6.5) node[below right] {$x=1$};
	\foreach \x in {2,4,...,8}
		\draw (\x,2pt) -- (\x,-2pt)	node[anchor=north] {\scriptsize \x};
	\foreach \x in {-8,-6,...,-2}
		\draw (\x,2pt) -- (\x,-2pt)	node[anchor=north] {\scriptsize \x};
	\foreach \y in {2,4,...,6}
		\draw (2pt,\y) -- (-2pt,\y)	node[anchor=east] {\scriptsize \y}; 
	\foreach \y in {-6,-4,...,-2}
		\draw (2pt,\y) -- (-2pt,\y)	node[anchor=east] {\scriptsize \y}; 
	\draw[fill, color=white] (0,0) circle (0.15);
	\draw[line width=0.3mm] (0,0) circle (0.15);
\end{tikzpicture}
\end{center}
We now carefully sketch the graph of $f$ based upon our findings.
\begin{center}
\begin{tikzpicture}[xscale=0.5,yscale=0.5]
	\draw [-](-8.25,0) -- coordinate (x axis mid) (8.25,0) node[below right] {};
	\draw [<->,dashed](-10.25,0) -- coordinate (x axis mid) (10.25,0) node[below right] {$x$};
	\draw [<->](0,-7.25) -- coordinate (x axis mid) (0,7.25) node[above right] {$y$};
	\draw [<->,dashed](-4,7) -- (-4,-6.5) node[below] {$x=-4$};
	\draw [<->,dashed](1,7) -- (1,-6.5) node[below right] {$x=1$};
	\draw [<->] plot [domain=-9.5:-4.5, samples=100] (\x,{(4*\x)/((\x+4)*(\x-1))});
	\draw [<->] plot [domain=-3.55:0.887, samples=100] (\x,{(4*\x)/((\x+4)*(\x-1))});
	\draw [<->] plot [domain=1.13:9.5, samples=100] (\x,{(4*\x)/((\x+4)*(\x-1))});
	\foreach \x in {2,4,...,8}
		\draw (\x,2pt) -- (\x,-2pt)	node[anchor=north] {\scriptsize \x};
	\foreach \x in {-8,-6,...,-2}
		\draw (\x,2pt) -- (\x,-2pt)	node[anchor=north] {\scriptsize \x};
	\foreach \y in {2,4,...,6}
		\draw (2pt,\y) -- (-2pt,\y)	node[anchor=east] {\scriptsize \y}; 
	\foreach \y in {-6,-4,...,-2}
		\draw (2pt,\y) -- (-2pt,\y)	node[anchor=east] {\scriptsize \y}; 
	\draw[fill, color=white] (0,0) circle (0.15);
	\draw[line width=0.3mm] (0,0) circle (0.15);
	\draw (7,-5) node {$f(x)=\dfrac{4x^2}{x^3+3x^2-4x}$};
\end{tikzpicture}
\end{center}
{\bf II - Demo/Discussion Problems:}\\
\ \par
Sketch a complete graph of each rational function below, making sure to have a clearly defined scale and label all key features of your graph (intercepts, asymptotes, and holes).  Use \Desmos \ to compare your results.
\begin{enumerate}
\item $f(x)=\dfrac{x^3-16x}{3x^2-6x-24}$
\item $g(x)=\dfrac{4x^2}{x^3+3x^2-4x}$
\end{enumerate}
{\bf III - Practice Problems:}\\
\ \par
Sketch a complete graph of each rational function below, making sure to have a clearly defined scale and label all key features of your graph (intercepts, asymptotes, and holes).  Use \Desmos \ to compare your results.
\begin{enumerate}
\begin{multicols}{2}
\item $a(x)=\dfrac{2x+6}{x+3}$
\item $g(x)=\dfrac{x^3-16x}{x^2-4x}$
\item $h(x)=\dfrac{x^2-4x-9}{x+2}$
\item $j(x)=\dfrac{18x^3-4x-1}{x^2-4}$
\item $k(x) =\dfrac{x^2-17x+72}{x-9}$
\item $p(x)=\dfrac{x^2+6x+8}{2x+4}$
\item $q(x)=\dfrac{x^5}{x(x-5)}$
\item $r(x)=\dfrac{x^2-11x+30}{x^2-36}$
\item $t(x)=\dfrac{3x^2-12x+12}{x^2+2x-8}$
\item $v(x)=\dfrac{x^2+2x-8} {3x^2-12x+12}$
\end{multicols}
\end{enumerate}
\end{document}
