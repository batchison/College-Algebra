\documentclass[12pt]{article}
\usepackage[top=1in,left=1in,bottom=1in,right=1in,headsep=2pt]{geometry}	
\usepackage{amssymb,amsmath,amsthm,amsfonts}
\usepackage{chapterfolder,docmute,setspace}
\usepackage{cancel,multicol,tikz,verbatim,framed,polynom,enumitem}
\usepackage[colorlinks, hyperindex, plainpages=false, linkcolor=blue, urlcolor=blue, pdfpagelabels]{hyperref}
\usepackage[type={CC},modifier={by-sa},version={4.0},]{doclicense}

\theoremstyle{definition}
\newtheorem{example}{Example}
\newcommand{\Desmos}{\href{https://www.desmos.com/}{Desmos}}
\setlength{\parindent}{0em}
\setlist{itemsep=0em}
\setlength{\parskip}{0.1em}
% This document is used for ordering of lessons.  If an instructor wishes to change the ordering of assessments, the following steps must be taken:

% 1) Reassign the appropriate numbers for each lesson in the \setcounter commands included in this file.
% 2) Rearrange the \include commands in the master file (the file with 'Course Pack' in the name) to accurately reflect the changes.  
% 3) Rearrange the \items in the measureable_outcomes file to accurately reflect the changes.  Be mindful of page breaks when moving items.
% 4) Re-build all affected files (master file, measureable_outcomes file, and any lesson whose numbering has changed).

%Note: The placement of each \newcounter and \setcounter command reflects the original/default ordering of topics (linears, systems, quadratics, functions, polynomials, rationals).

\newcounter{lesson_solving_linear_equations}
\newcounter{lesson_equations_containing_absolute_values}
\newcounter{lesson_graphing_lines}
\newcounter{lesson_two_forms_of_a_linear_equation}
\newcounter{lesson_parallel_and_perpendicular_lines}
\newcounter{lesson_linear_inequalities}
\newcounter{lesson_compound_inequalities}
\newcounter{lesson_inequalities_containing_absolute_values}
\newcounter{lesson_graphing_systems}
\newcounter{lesson_substitution}
\newcounter{lesson_elimination}
\newcounter{lesson_quadratics_introduction}
\newcounter{lesson_factoring_GCF}
\newcounter{lesson_factoring_grouping}
\newcounter{lesson_factoring_trinomials_a_is_1}
\newcounter{lesson_factoring_trinomials_a_neq_1}
\newcounter{lesson_solving_by_factoring}
\newcounter{lesson_square_roots}
\newcounter{lesson_i_and_complex_numbers}
\newcounter{lesson_vertex_form_and_graphing}
\newcounter{lesson_solve_by_square_roots}
\newcounter{lesson_extracting_square_roots}
\newcounter{lesson_the_discriminant}
\newcounter{lesson_the_quadratic_formula}
\newcounter{lesson_quadratic_inequalities}
\newcounter{lesson_functions_and_relations}
\newcounter{lesson_evaluating_functions}
\newcounter{lesson_finding_domain_and_range_graphically}
\newcounter{lesson_fundamental_functions}
\newcounter{lesson_finding_domain_algebraically}
\newcounter{lesson_solving_functions}
\newcounter{lesson_function_arithmetic}
\newcounter{lesson_composite_functions}
\newcounter{lesson_inverse_functions_definition_and_HLT}
\newcounter{lesson_finding_an_inverse_function}
\newcounter{lesson_transformations_translations}
\newcounter{lesson_transformations_reflections}
\newcounter{lesson_transformations_scalings}
\newcounter{lesson_transformations_summary}
\newcounter{lesson_piecewise_functions}
\newcounter{lesson_functions_containing_absolute_values}
\newcounter{lesson_absolute_as_piecewise}
\newcounter{lesson_polynomials_introduction}
\newcounter{lesson_sign_diagrams_polynomials}
\newcounter{lesson_factoring_quadratic_type}
\newcounter{lesson_factoring_summary}
\newcounter{lesson_polynomial_division}
\newcounter{lesson_synthetic_division}
\newcounter{lesson_end_behavior_polynomials}
\newcounter{lesson_local_behavior_polynomials}
\newcounter{lesson_rational_root_theorem}
\newcounter{lesson_polynomials_graphing_summary}
\newcounter{lesson_polynomial_inequalities}
\newcounter{lesson_rationals_introduction_and_terminology}
\newcounter{lesson_sign_diagrams_rationals}
\newcounter{lesson_horizontal_asymptotes}
\newcounter{lesson_slant_and_curvilinear_asymptotes}
\newcounter{lesson_vertical_asymptotes}
\newcounter{lesson_holes}
\newcounter{lesson_rationals_graphing_summary}

\setcounter{lesson_solving_linear_equations}{1}
\setcounter{lesson_equations_containing_absolute_values}{2}
\setcounter{lesson_graphing_lines}{3}
\setcounter{lesson_two_forms_of_a_linear_equation}{4}
\setcounter{lesson_parallel_and_perpendicular_lines}{5}
\setcounter{lesson_linear_inequalities}{6}
\setcounter{lesson_compound_inequalities}{7}
\setcounter{lesson_inequalities_containing_absolute_values}{8}
\setcounter{lesson_graphing_systems}{9}
\setcounter{lesson_substitution}{10}
\setcounter{lesson_elimination}{11}
\setcounter{lesson_quadratics_introduction}{16}
\setcounter{lesson_factoring_GCF}{17}
\setcounter{lesson_factoring_grouping}{18}
\setcounter{lesson_factoring_trinomials_a_is_1}{19}
\setcounter{lesson_factoring_trinomials_a_neq_1}{20}
\setcounter{lesson_solving_by_factoring}{21}
\setcounter{lesson_square_roots}{22}
\setcounter{lesson_i_and_complex_numbers}{23}
\setcounter{lesson_vertex_form_and_graphing}{24}
\setcounter{lesson_solve_by_square_roots}{25}
\setcounter{lesson_extracting_square_roots}{26}
\setcounter{lesson_the_discriminant}{27}
\setcounter{lesson_the_quadratic_formula}{28}
\setcounter{lesson_quadratic_inequalities}{29}
\setcounter{lesson_functions_and_relations}{12}
\setcounter{lesson_evaluating_functions}{13}
\setcounter{lesson_finding_domain_and_range_graphically}{14}
\setcounter{lesson_fundamental_functions}{15}
\setcounter{lesson_finding_domain_algebraically}{30}
\setcounter{lesson_solving_functions}{31}
\setcounter{lesson_function_arithmetic}{32}
\setcounter{lesson_composite_functions}{33}
\setcounter{lesson_inverse_functions_definition_and_HLT}{34}
\setcounter{lesson_finding_an_inverse_function}{35}
\setcounter{lesson_transformations_translations}{36}
\setcounter{lesson_transformations_reflections}{37}
\setcounter{lesson_transformations_scalings}{38}
\setcounter{lesson_transformations_summary}{39}
\setcounter{lesson_piecewise_functions}{40}
\setcounter{lesson_functions_containing_absolute_values}{41}
\setcounter{lesson_absolute_as_piecewise}{42}
\setcounter{lesson_polynomials_introduction}{43}
\setcounter{lesson_sign_diagrams_polynomials}{44}
\setcounter{lesson_factoring_quadratic_type}{46}
\setcounter{lesson_factoring_summary}{45}
\setcounter{lesson_polynomial_division}{47}
\setcounter{lesson_synthetic_division}{48}
\setcounter{lesson_end_behavior_polynomials}{49}
\setcounter{lesson_local_behavior_polynomials}{50}
\setcounter{lesson_rational_root_theorem}{51}
\setcounter{lesson_polynomials_graphing_summary}{52}
\setcounter{lesson_polynomial_inequalities}{53}
\setcounter{lesson_rationals_introduction_and_terminology}{54}
\setcounter{lesson_sign_diagrams_rationals}{55}
\setcounter{lesson_horizontal_asymptotes}{56}
\setcounter{lesson_slant_and_curvilinear_asymptotes}{57}
\setcounter{lesson_vertical_asymptotes}{58}
\setcounter{lesson_holes}{59}
\setcounter{lesson_rationals_graphing_summary}{60}

\begin{document}
{\bf \large Lesson \arabic{lesson_substitution}: Solving Systems of Linear Equations by Substitution}\phantomsection\label{les:substitution}\\
CC attribute: \href{http://www.wallace.ccfaculty.org/book/book.html}{\it{Beginning and Intermediate Algebra}} by T. Wallace. \hfill \doclicenseImage[imagewidth=5em]\\
\par
{\bf Objective:} Solve linear systems by substitution.\\
\par
{\bf Students will be able to:}
\begin{itemize}
	\item Identify a lone variable.
	\item Solve linear systems by substitution.
	\item Write system solutions as ordered pairs in the form $(x,y)$.
	\item Verify the accuracy of a solution by plugging it into each equation in the system.
\end{itemize}
{\bf Prerequisite Knowledge:}
\begin{itemize}
	\item Solving a linear equation.
	\item Applying the distributive property.
	\item Combining like terms.
\end{itemize}
\hrulefill

{\bf Lesson:}
\par
{\bf I - Motivating Example(s):}\\
\par
We present the steps for solving a system of linear equations by substitution alongside an example.
\begin{center}
\begin{tabular}{|l|c|}
  \hline
	& \\
  \ \ \ \ \ \ \ \ Steps for Substitution & \begin{tabular}{l}
    System: $\begin{cases}
		4 x - 2 y = 2\\
    2 x + y = - 5\end{cases}$
	\end{tabular}\\
  & \\
  \hline
  1. Identify a lone variable. & \begin{tabular}{l}
    The lone variable is $y$, in the\\
     second equation: $2 x + \framebox{$y$} = - 5$
  \end{tabular}\\
  \hline
  2. Solve for the lone variable. & \begin{tabular}{c}
    Subtract $2x$ from both sides.\\
    $y = - 5 - 2 x$
  \end{tabular}\\
  \hline
  3.~$\begin{array}{l}
	\text{Substitute into the untouched}\\
	\text{equation.}
	\end{array}$& $4 x - 2 (- 5 - 2 x)=2$\\
  \hline
  4. Solve for the remaining variable. & $\begin{array}{c}
 4 x + 10 + 4 x = 2~~~~~~~\\
 8 x + 10 = 2\\
 ~~~~~~~~\underline{- 10 ~~- 10}\\
~~~~~~~~~ 8 x = - 8\\
~~~~~~~~~ \overline{8} ~~~~~~ \overline{8}\\
~~~~~~~~~ x = - 1
  \end{array}$\\
  \hline
  5.~$\begin{array}{l}
	\text{Plug into lone variable}\\
	\text{equation and evaluate.}
	\end{array}$ & $\begin{array}{l}
    y = - 5 - 2 (- 1)\\
    y = - 5 + 2\\
    y = - 3
  \end{array}$\\
  \hline
  Our solution, as a coordinate pair. & $(x,y)=(- 1, - 3)$\\
  \hline
\end{tabular}
\end{center}

{\bf II - Demo/Discussion Problems:}\\
\ \par
Solve each of the following systems of linear equations by substitution.
\begin{multicols}{3}
	\begin{enumerate}
		\item $\begin{cases}
					2x-3y=7\\
			    y=3x-7
					\end{cases}$
		\item $\begin{cases}
					3x+2y=1\\
					x-5y=6
					\end{cases}$
		\item $\begin{cases}
			    x-y=2\\
				  8x-3y=16
					\end{cases}$
	\end{enumerate}
\end{multicols}
\ \par
{\bf III - Practice Problems:}\\
\ \par
Solve each of the following systems of linear equations by substitution.
\begin{multicols}{3}
	\begin{enumerate}
  \item $\begin{cases} 
	y = - 3 x\\
	y = 6 x - 9
  \end{cases}$
  \item $\begin{cases} 
  y = 6 x + 4\\
	y = - 3 x - 5
  \end{cases}$
  \item $\begin{cases} 
  y = 2 x - 3\\
	y = - 2 x + 9
  \end{cases}$
  \item $\begin{cases} 
  y = - 6\\
	3 x - 6 y = 30
  \end{cases}$
  \item $\begin{cases} 
  - 2 x + 2 y = 18\\
  y = 7 x + 15
  \end{cases}$
  \item $\begin{cases} 
  7 x - 2 y = - 7\\
  y = 7
  \end{cases}$
  \item $\begin{cases} 
  - 2 x - y = - 5\\
  x - 8 y = - 23
  \end{cases}$
  \item $\begin{cases} 
  3 x + y = 9\\
	2 x + 8 y = - 16
  \end{cases}$
  \item $\begin{cases} 
  x + 5 y = 15\\
  - 3 x + 2 y = 6
  \end{cases}$
  \item $\begin{cases} 
  y = x + 5\\
	y = - 2 x - 4
  \end{cases}$
  \item $\begin{cases} 
  y = 3 x + 13\\
	y = - 2 x - 22
  \end{cases}$
  \item $\begin{cases} 
  y = 7 x - 24\\
	y = - 3 x + 16
  \end{cases}$
  \item $\begin{cases} 
  6 x - 4 y = - 8\\
  y = - 6 x + 2
  \end{cases}$
  \item $\begin{cases} 
  y = x + 4\\
	3 x - 4 y = - 19
  \end{cases}$
  \item $\begin{cases} 
  x - 2 y = - 13\\
  4 x + 2 y = 18
  \end{cases}$
  \item $\begin{cases} 
  6 x + 4 y = 16\\
  - 2 x + y = - 3
  \end{cases}$
  \item $\begin{cases} 
  - 5 x - 5 y = - 20\\
  - 2 x + y = 7
  \end{cases}$
  \item $\begin{cases} 
  2 x + 3 y = - 10\\
  7 x + y = 3
  \end{cases}$
  \item $\begin{cases} 
  y = - 2 x - 9\\
	y = 2 x - 1
  \end{cases}$
  \item $\begin{cases} 
  y = 3 x + 2\\
	y = - 3 x + 8
  \end{cases}$
  \item $\begin{cases} 
  y = 6 x - 6\\
	- 3 x - 3 y = - 24
  \end{cases}$
  \item $\begin{cases} 
  y = - 5\\
	3 x + 4 y = - 17
  \end{cases}$
  \item $\begin{cases} 
  y = - 8 x + 19\\
	- x + 6 y = 16
  \end{cases}$
  \item $\begin{cases} 
  x - 5 y = 7\\
	2 x + 7 y = - 20
  \end{cases}$
  \item $\begin{cases} 
  - 6 x + y = 20\\
  - 3 x - 3 y = - 18
  \end{cases}$
  \item $\begin{cases} 
  2 x + y = 2\\
  3 x + 7 y = 14
  \end{cases}$
  \item $\begin{cases} 
  - 2 x + 4 y = - 16\\
  y = - 2
  \end{cases}$
  \item $\begin{cases} 
  y = - 6 x + 3\\
	y = 6 x + 3
  \end{cases}$
  \item $\begin{cases} 
  y = - 2 x - 9\\
	y = - 5 x - 21
  \end{cases}$
  \item $\begin{cases} 
  - x + 3 y = 12\\
  y = 6 x + 21
  \end{cases}$
  \item $\begin{cases} 
  7 x + 2 y = - 7\\
  y = 5 x + 5
  \end{cases}$
  \item $\begin{cases} 
  y = - 2 x + 8\\
  - 7 x - 6 y = - 8
  \end{cases}$
  \item $\begin{cases} 
  3 x - 4 y = 15\\
  7 x + y = 4
  \end{cases}$
  \item $\begin{cases} 
  7 x + 5 y = - 13\\
  x - 4 y = - 16
  \end{cases}$
  \item $\begin{cases} 
  2 x + y = - 7\\
  5 x + 3 y = - 21
  \end{cases}$
  \item $\begin{cases} 
  - 2 x + 2 y = - 22\\
  - 5 x - 7 y = - 19
	\end{cases}$
	\end{enumerate}
\end{multicols}
\newpage
\ \newpage
\end{document}