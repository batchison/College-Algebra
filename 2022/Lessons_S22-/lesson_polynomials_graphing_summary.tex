\documentclass[12pt]{article}
\usepackage[top=1in,left=1in,bottom=1in,right=1in,headsep=2pt]{geometry}	
\usepackage{amssymb,amsmath,amsthm,amsfonts}
\usepackage{chapterfolder,docmute,setspace}
\usepackage{cancel,multicol,tikz,verbatim,framed,polynom,enumitem}
\usepackage[colorlinks, hyperindex, plainpages=false, linkcolor=blue, urlcolor=blue, pdfpagelabels]{hyperref}
% Use the cc-by-nc-sa license for any content linked with Stitz and Zeager's text.  Otherwise, use the cc-by-sa license.
%\usepackage[type={CC},modifier={by-sa},version={4.0},]{doclicense}
\usepackage[type={CC},modifier={by-nc-sa},version={4.0},]{doclicense}

\theoremstyle{definition}
\newtheorem{example}{Example}
\newcommand{\Desmos}{\href{https://www.desmos.com/}{Desmos}}
\setlength{\parindent}{0em}
\setlist{itemsep=0em}
\setlength{\parskip}{0.1em}
% This document is used for ordering of lessons.  If an instructor wishes to change the ordering of assessments, the following steps must be taken:

% 1) Reassign the appropriate numbers for each lesson in the \setcounter commands included in this file.
% 2) Rearrange the \include commands in the master file (the file with 'Course Pack' in the name) to accurately reflect the changes.  
% 3) Rearrange the \items in the measureable_outcomes file to accurately reflect the changes.  Be mindful of page breaks when moving items.
% 4) Re-build all affected files (master file, measureable_outcomes file, and any lesson whose numbering has changed).

%Note: The placement of each \newcounter and \setcounter command reflects the original/default ordering of topics (linears, systems, quadratics, functions, polynomials, rationals).

\newcounter{lesson_solving_linear_equations}
\newcounter{lesson_equations_containing_absolute_values}
\newcounter{lesson_graphing_lines}
\newcounter{lesson_two_forms_of_a_linear_equation}
\newcounter{lesson_parallel_and_perpendicular_lines}
\newcounter{lesson_linear_inequalities}
\newcounter{lesson_compound_inequalities}
\newcounter{lesson_inequalities_containing_absolute_values}
\newcounter{lesson_graphing_systems}
\newcounter{lesson_substitution}
\newcounter{lesson_elimination}
\newcounter{lesson_quadratics_introduction}
\newcounter{lesson_factoring_GCF}
\newcounter{lesson_factoring_grouping}
\newcounter{lesson_factoring_trinomials_a_is_1}
\newcounter{lesson_factoring_trinomials_a_neq_1}
\newcounter{lesson_solving_by_factoring}
\newcounter{lesson_square_roots}
\newcounter{lesson_i_and_complex_numbers}
\newcounter{lesson_vertex_form_and_graphing}
\newcounter{lesson_solve_by_square_roots}
\newcounter{lesson_extracting_square_roots}
\newcounter{lesson_the_discriminant}
\newcounter{lesson_the_quadratic_formula}
\newcounter{lesson_quadratic_inequalities}
\newcounter{lesson_functions_and_relations}
\newcounter{lesson_evaluating_functions}
\newcounter{lesson_finding_domain_and_range_graphically}
\newcounter{lesson_fundamental_functions}
\newcounter{lesson_finding_domain_algebraically}
\newcounter{lesson_solving_functions}
\newcounter{lesson_function_arithmetic}
\newcounter{lesson_composite_functions}
\newcounter{lesson_inverse_functions_definition_and_HLT}
\newcounter{lesson_finding_an_inverse_function}
\newcounter{lesson_transformations_translations}
\newcounter{lesson_transformations_reflections}
\newcounter{lesson_transformations_scalings}
\newcounter{lesson_transformations_summary}
\newcounter{lesson_piecewise_functions}
\newcounter{lesson_functions_containing_absolute_values}
\newcounter{lesson_absolute_as_piecewise}
\newcounter{lesson_polynomials_introduction}
\newcounter{lesson_sign_diagrams_polynomials}
\newcounter{lesson_factoring_quadratic_type}
\newcounter{lesson_factoring_summary}
\newcounter{lesson_polynomial_division}
\newcounter{lesson_synthetic_division}
\newcounter{lesson_end_behavior_polynomials}
\newcounter{lesson_local_behavior_polynomials}
\newcounter{lesson_rational_root_theorem}
\newcounter{lesson_polynomials_graphing_summary}
\newcounter{lesson_polynomial_inequalities}
\newcounter{lesson_rationals_introduction_and_terminology}
\newcounter{lesson_sign_diagrams_rationals}
\newcounter{lesson_horizontal_asymptotes}
\newcounter{lesson_slant_and_curvilinear_asymptotes}
\newcounter{lesson_vertical_asymptotes}
\newcounter{lesson_holes}
\newcounter{lesson_rationals_graphing_summary}

\setcounter{lesson_solving_linear_equations}{1}
\setcounter{lesson_equations_containing_absolute_values}{2}
\setcounter{lesson_graphing_lines}{3}
\setcounter{lesson_two_forms_of_a_linear_equation}{4}
\setcounter{lesson_parallel_and_perpendicular_lines}{5}
\setcounter{lesson_linear_inequalities}{6}
\setcounter{lesson_compound_inequalities}{7}
\setcounter{lesson_inequalities_containing_absolute_values}{8}
\setcounter{lesson_graphing_systems}{9}
\setcounter{lesson_substitution}{10}
\setcounter{lesson_elimination}{11}
\setcounter{lesson_quadratics_introduction}{16}
\setcounter{lesson_factoring_GCF}{17}
\setcounter{lesson_factoring_grouping}{18}
\setcounter{lesson_factoring_trinomials_a_is_1}{19}
\setcounter{lesson_factoring_trinomials_a_neq_1}{20}
\setcounter{lesson_solving_by_factoring}{21}
\setcounter{lesson_square_roots}{22}
\setcounter{lesson_i_and_complex_numbers}{23}
\setcounter{lesson_vertex_form_and_graphing}{24}
\setcounter{lesson_solve_by_square_roots}{25}
\setcounter{lesson_extracting_square_roots}{26}
\setcounter{lesson_the_discriminant}{27}
\setcounter{lesson_the_quadratic_formula}{28}
\setcounter{lesson_quadratic_inequalities}{29}
\setcounter{lesson_functions_and_relations}{12}
\setcounter{lesson_evaluating_functions}{13}
\setcounter{lesson_finding_domain_and_range_graphically}{14}
\setcounter{lesson_fundamental_functions}{15}
\setcounter{lesson_finding_domain_algebraically}{30}
\setcounter{lesson_solving_functions}{31}
\setcounter{lesson_function_arithmetic}{32}
\setcounter{lesson_composite_functions}{33}
\setcounter{lesson_inverse_functions_definition_and_HLT}{34}
\setcounter{lesson_finding_an_inverse_function}{35}
\setcounter{lesson_transformations_translations}{36}
\setcounter{lesson_transformations_reflections}{37}
\setcounter{lesson_transformations_scalings}{38}
\setcounter{lesson_transformations_summary}{39}
\setcounter{lesson_piecewise_functions}{40}
\setcounter{lesson_functions_containing_absolute_values}{41}
\setcounter{lesson_absolute_as_piecewise}{42}
\setcounter{lesson_polynomials_introduction}{43}
\setcounter{lesson_sign_diagrams_polynomials}{44}
\setcounter{lesson_factoring_quadratic_type}{46}
\setcounter{lesson_factoring_summary}{45}
\setcounter{lesson_polynomial_division}{47}
\setcounter{lesson_synthetic_division}{48}
\setcounter{lesson_end_behavior_polynomials}{49}
\setcounter{lesson_local_behavior_polynomials}{50}
\setcounter{lesson_rational_root_theorem}{51}
\setcounter{lesson_polynomials_graphing_summary}{52}
\setcounter{lesson_polynomial_inequalities}{53}
\setcounter{lesson_rationals_introduction_and_terminology}{54}
\setcounter{lesson_sign_diagrams_rationals}{55}
\setcounter{lesson_horizontal_asymptotes}{56}
\setcounter{lesson_slant_and_curvilinear_asymptotes}{57}
\setcounter{lesson_vertical_asymptotes}{58}
\setcounter{lesson_holes}{59}
\setcounter{lesson_rationals_graphing_summary}{60}

\begin{document}
{\bf \large Lesson \arabic{lesson_polynomials_graphing_summary}: Polynomials Graphing Summary}\phantomsection\label{les:polynomials_graphing_summary}
%\\ CC attribute: \href{http://www.wallace.ccfaculty.org/book/book.html}{\it{Beginning and Intermediate Algebra}} by T. Wallace. 
\\ CC attribute: \href{http://www.stitz-zeager.com}{\it{College Algebra}} by C. Stitz and J. Zeager. 
\hfill \doclicenseImage[imagewidth=5em]\\
\par
{\bf Objective:} Graph a polynomial function in its entirety.\\
\par
{\bf Students will be able to:}
\begin{itemize}
	\item Identify all important aspects of the graph of a polynomial function: $y-$intercept, $x-$intercepts (including their nature), and end behavior.
	\item Sketch a complete graph of a polynomial function.
\end{itemize}
{\bf Prerequisite Knowledge:}
\begin{itemize}
	\item Evaluating a function.
	\item Polynomial terminology and end behavior.
	\item Factoring.
	\item The Rational Root Theorem.
	\item Polynomial and \slash or Synthetic Division.
	\item Sign Diagrams.
\end{itemize}
\hrulefill

{\bf Lesson:}\\
\ \par
At this point, we have addressed all key features of polynomials individually.  This lesson pulls each of these aspects together, for a detailed analysis of a polynomial, culminating in a complete sketch of its graph.  Along the way, we will need to address each of the following aspects for our polynomial $f(x)=a_nx^n+a_{n-1}x^{n-1}+\ldots+a_1x+a_0$.  It is important to note that there is no universally accepted order to this checklist.
\begin{itemize}
	\item Find the $y-$intercept of the graph of $f,$ $(0,f(0))=(0,a_0)$.
	\item Use the degree $n$ and leading coefficient $a_n$ to determine the end behavior of the graph of $f$.
	\item Identify a complete factorization of $f,$ and use it to find any $x-$intercepts of the graph of $f$.  Using multiplicities, classify each $x-$intercept as a crossover or turnaround (``bounce'') point.
	\item Using the $x-$intercepts, construct a sign diagram for $f$.
\end{itemize}
In each polynomial we encounter, we will carefully examine the function, making sure not to omit any of the checklist items above and to compare each item to those that precede it along the way for accuracy.  Although the process will take some time, if we are thorough, our end result should be a complete, accurate sketch of the given polynomial.\\
\ \par
{\bf I - Motivating Example(s):}\\
\ \par
{\bf Example:}  Sketch a complete graph of $f(x)=14x^4-17x^3-6x^2+7x+2$.\\
\ \par
We will start with the $y-$intercept, which is $(0,2)$.\\
\ \par
Next, we see that $f$ has even degree and positive leading coefficient.  So, the tails of the graph of $f$ both point upwards.  In other words, as $x\rightarrow\pm\infty, f(x)\rightarrow\infty$.\\
\ \par
Since $f$ is degree-4, contains more than four terms, and is not of quadratic type, we will apply the Rational Root Theorem.  In this case, our set of possible rational roots is
$$\left\{\pm 1, \pm 2, \pm \frac{1}{2}, \pm \frac{1}{7}, \pm \frac{1}{14}, \pm \frac{2}{7}\right\}$$
Fortunately, we see that $f(1)=14-17-6+7+2=0$.  So, $x-1$ is a factor of $f$.  Dividing, we get:
\begin{multicols}{2}
\polylongdiv{14x^4-17x^3-6x^2+7x+2}{x-1}

\columnbreak

\polyhornerscheme[x=1,showbase=top,resultstyle=\bf]{14x^4-17x^3-6x^2+7x+2}
\end{multicols}

So, $f(x)=(x-1)(14x^3-3x^2-9x-2).$  Applying the Rational Root Theorem a second time, we can see that $x=1$ is also a root of the cubic factor of $f,$ since $14-3-9-2=0$.  Again, we can divide to factor $f$ further.

\begin{multicols}{2}
\polylongdiv{14x^3-3x^2-9x-2}{x-1}

\columnbreak

\polyhornerscheme[x=1,showbase=top,resultstyle=\bf]{14x^3-3x^2-9x-2}
\end{multicols}

So, $f(x)=(x-1)^2(14x^2+11x+2)$.  Factoring the remaining quadratic, we have\\
$f(x)=(x-1)^2(7x+2)(2x+1),$
with set of roots $\{1,-\frac{1}{2},-\frac{2}{7}\}$.
\newpage
Using multiplicities, we conclude that the $x-$intercept $(1,0)$ is a turnaround point, and the intercepts $\left(-\frac{1}{2},0\right)$ and $\left(-\frac{2}{7},0\right)$ are crossover points.\\
\ \par
Though not necessary for graphing, a sign diagram confirms our end and local behavior findings.

\begin{center}
\begin{tikzpicture}[xscale=1,yscale=1]
	\draw [<->](-6.25,0) -- coordinate (x axis mid) (4.25,0) node[below right] {$x$};
	\draw [-](-4,1) -- coordinate (y axis mid) (-4,-0.25) node[below] {$-\frac{1}{2}$};
	\draw [-](-1,1) -- coordinate (y axis mid) (-1,-0.25) node[below] {$-\frac{2}{7}$};
	\draw [-](2,1) -- coordinate (y axis mid) (2,-0.25) node[below] {$1$};
	\draw (-5.5,-1) node {$x=-1$};
	\draw (-2.5,-1) node {$x=-\frac{3}{7}$};
	\draw (0.5,-1) node {$x=0$};
	\draw (3.5,-1) node {$x=2$};
	\draw (-5.5,0.5) node {$+$};
	\draw (-2.5,0.5) node {$-$};
	\draw (0.5,0.5) node {$+$};
	\draw (3.5,0.5) node {$+$};
\end{tikzpicture}
\end{center}

Putting all of this information together results in the following graph.

\begin{center}
\begin{tikzpicture}[xscale=1.75,yscale=1]
	\draw [<->](-2.25,0) -- coordinate (x axis mid) (2.25,0) node[below right] {$x$};
	\draw [<->](0,-1.75) -- coordinate (y axis mid) (0,5.25) node[above right] {$y$};
	\draw [<->] plot [domain=-0.736:1.33, samples=100] (\x,{(2*\x+1)*(7*\x+2)*(\x-1)*(\x-1)});
	\foreach \x in {0.5,1,1.5,2}
		\draw (\x,1pt) -- (\x,-1pt)	node[anchor=north] {\scriptsize \x};
	\foreach \x in {-2,-1.5,-1,-0.5}
		\draw (\x,1pt) -- (\x,-1pt)	node[anchor=north] {\scriptsize \x};
	\foreach \y in {1,...,5}
		\draw (1pt,\y) -- (-1pt,\y)	node[anchor=east] {\scriptsize \y}; 
	\foreach \y in {-1}
		\draw (1pt,\y) -- (-1pt,\y)	node[anchor=west] {\scriptsize \y}; 
\end{tikzpicture}
\end{center}


{\bf II - Demo/Discussion Problems:}\\
\ \par
Factor each polynomial below, and sketch a complete graph of the function, making sure to have a clearly defined scale and label any intercepts.  Use \Desmos \ to compare your results.
\begin{enumerate}
%\begin{multicols}{2}
	\item $f(x) = -x^{3} + 7x^{2} - x + 7$
	\item $f(x) = 3x^4 - 5x^3 - 12x^2$
	\item $f(x) = 2x^3 - 5x^2 - x$
	\item $f(x) = -x^4-2x^2 +15$
%\end{multicols}
\end{enumerate}
\newpage
Use the Rational Root Theorem and polynomial division to get a complete factorization of each polynomial function below.  Then sketch a graph of the function, making sure to have a clearly defined scale and label any intercepts.  Use \Desmos \ to compare your results.
\begin{enumerate}
	\item[5.] $f(x) = x^4+4x^3-x-4$	
	\item[6.] $f(x) = 2x^3-5x^2-52x+60$
	\item[7.] $f(x) = -x^3-x^2+39x+45$
	\item[8.] $f(x) = -2x^4+7x^3+17x^2-28x-36$
	\item[9.] $f(x) = x^7-5x^6-24x^5+120x^4-25x^3+125x^2$\\
\end{enumerate}
{\bf III - Practice Problems:}\\
\ \par
Factor each polynomial below, and sketch a complete graph of the function, making sure to have a clearly defined scale and label any intercepts.  Use \Desmos \ to compare your results.
\begin{enumerate}
\begin{multicols}{2}
	\item $f(x) = -17x^{3} + 5x^{2} + 34x - 10$
	\item $f(x) = x^4-9x^2+14$
	\item $f(x) = 3x^4-14x^2-5$
	\item $f(x) = 2x^4-7x^2+6$
	\item $f(x) = x^5-2x^4-x+2$
	\item $f(x) = 2x^5+3x^4-32x-48$
	\item $f(x) = x^6-6x^3-16$
	\item $f(x) = 2x^6-7x^3+5$
\end{multicols}
\end{enumerate}
Get a complete factorization of each polynomial below by first dividing the function by $x-1$.  Then sketch a graph of the function, making sure to have a clearly defined scale and label any intercepts.  Use \Desmos \ to compare your results.
\begin{enumerate}
	\item[9.] $f(x) = x^{3} - 2x^{2} - 5x + 6$
	\item[10.] $f(x) = x^{3} + 4x^{2} - 11x + 6$
	\item[11.] $f(x)=x^5-x^4-37x^3+37x^2+36x-36$
	\item[12.] $f(x) = x^{4} -2x^3+ 5x^{2} - 8x + 4$
\end{enumerate}
Use the Rational Root Theorem and polynomial division to get a complete factorization of each polynomial function below.  Then sketch a graph of the function, making sure to have a clearly defined scale and label any intercepts.  Use \Desmos \ to compare your results.
\begin{enumerate}
%\begin{multicols}{2}
	\item[13.] $f(x) = x^{4} - 9x^{2} - 4x + 12$
	\item[14.] $f(x) = x^{4} + 2x^{3} - 12x^{2} - 40x - 32$
	\item[15.] $f(x) = 2x^4+x^3-7x^2-3x+3$
	\item[16.] $f(x) = 3x^{3} + 3x^{2} - 11x - 10$
	\item[17.] $f(x) = 6x^3+19x^2-34x-40$
	\item[18.] $f(x) = -2x^{3} + 19x^{2} - 49x + 20$
	\item[19.] $f(x) = 36x^{4} - 12x^{3} - 11x^{2} + 2x + 1$
%\end{multicols}
\end{enumerate}
\newpage
\end{document}
