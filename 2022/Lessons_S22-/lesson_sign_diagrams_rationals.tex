\documentclass[12pt]{article}
\usepackage[top=1in,left=1in,bottom=1in,right=1in,headsep=2pt]{geometry}	
\usepackage{amssymb,amsmath,amsthm,amsfonts}
\usepackage{chapterfolder,docmute,setspace}
\usepackage{cancel,multicol,tikz,verbatim,framed,polynom,enumitem}
\usepackage[colorlinks, hyperindex, plainpages=false, linkcolor=blue, urlcolor=blue, pdfpagelabels]{hyperref}
% Use the cc-by-nc-sa license for any content linked with Stitz and Zeager's text.  Otherwise, use the cc-by-sa license.
%\usepackage[type={CC},modifier={by-sa},version={4.0},]{doclicense}
\usepackage[type={CC},modifier={by-nc-sa},version={4.0},]{doclicense}

\theoremstyle{definition}
\newtheorem{example}{Example}
\newcommand{\Desmos}{\href{https://www.desmos.com/}{Desmos}}
\setlength{\parindent}{0em}
\setlist{itemsep=0em}
\setlength{\parskip}{0.1em}
% This document is used for ordering of lessons.  If an instructor wishes to change the ordering of assessments, the following steps must be taken:

% 1) Reassign the appropriate numbers for each lesson in the \setcounter commands included in this file.
% 2) Rearrange the \include commands in the master file (the file with 'Course Pack' in the name) to accurately reflect the changes.  
% 3) Rearrange the \items in the measureable_outcomes file to accurately reflect the changes.  Be mindful of page breaks when moving items.
% 4) Re-build all affected files (master file, measureable_outcomes file, and any lesson whose numbering has changed).

%Note: The placement of each \newcounter and \setcounter command reflects the original/default ordering of topics (linears, systems, quadratics, functions, polynomials, rationals).

\newcounter{lesson_solving_linear_equations}
\newcounter{lesson_equations_containing_absolute_values}
\newcounter{lesson_graphing_lines}
\newcounter{lesson_two_forms_of_a_linear_equation}
\newcounter{lesson_parallel_and_perpendicular_lines}
\newcounter{lesson_linear_inequalities}
\newcounter{lesson_compound_inequalities}
\newcounter{lesson_inequalities_containing_absolute_values}
\newcounter{lesson_graphing_systems}
\newcounter{lesson_substitution}
\newcounter{lesson_elimination}
\newcounter{lesson_quadratics_introduction}
\newcounter{lesson_factoring_GCF}
\newcounter{lesson_factoring_grouping}
\newcounter{lesson_factoring_trinomials_a_is_1}
\newcounter{lesson_factoring_trinomials_a_neq_1}
\newcounter{lesson_solving_by_factoring}
\newcounter{lesson_square_roots}
\newcounter{lesson_i_and_complex_numbers}
\newcounter{lesson_vertex_form_and_graphing}
\newcounter{lesson_solve_by_square_roots}
\newcounter{lesson_extracting_square_roots}
\newcounter{lesson_the_discriminant}
\newcounter{lesson_the_quadratic_formula}
\newcounter{lesson_quadratic_inequalities}
\newcounter{lesson_functions_and_relations}
\newcounter{lesson_evaluating_functions}
\newcounter{lesson_finding_domain_and_range_graphically}
\newcounter{lesson_fundamental_functions}
\newcounter{lesson_finding_domain_algebraically}
\newcounter{lesson_solving_functions}
\newcounter{lesson_function_arithmetic}
\newcounter{lesson_composite_functions}
\newcounter{lesson_inverse_functions_definition_and_HLT}
\newcounter{lesson_finding_an_inverse_function}
\newcounter{lesson_transformations_translations}
\newcounter{lesson_transformations_reflections}
\newcounter{lesson_transformations_scalings}
\newcounter{lesson_transformations_summary}
\newcounter{lesson_piecewise_functions}
\newcounter{lesson_functions_containing_absolute_values}
\newcounter{lesson_absolute_as_piecewise}
\newcounter{lesson_polynomials_introduction}
\newcounter{lesson_sign_diagrams_polynomials}
\newcounter{lesson_factoring_quadratic_type}
\newcounter{lesson_factoring_summary}
\newcounter{lesson_polynomial_division}
\newcounter{lesson_synthetic_division}
\newcounter{lesson_end_behavior_polynomials}
\newcounter{lesson_local_behavior_polynomials}
\newcounter{lesson_rational_root_theorem}
\newcounter{lesson_polynomials_graphing_summary}
\newcounter{lesson_polynomial_inequalities}
\newcounter{lesson_rationals_introduction_and_terminology}
\newcounter{lesson_sign_diagrams_rationals}
\newcounter{lesson_horizontal_asymptotes}
\newcounter{lesson_slant_and_curvilinear_asymptotes}
\newcounter{lesson_vertical_asymptotes}
\newcounter{lesson_holes}
\newcounter{lesson_rationals_graphing_summary}

\setcounter{lesson_solving_linear_equations}{1}
\setcounter{lesson_equations_containing_absolute_values}{2}
\setcounter{lesson_graphing_lines}{3}
\setcounter{lesson_two_forms_of_a_linear_equation}{4}
\setcounter{lesson_parallel_and_perpendicular_lines}{5}
\setcounter{lesson_linear_inequalities}{6}
\setcounter{lesson_compound_inequalities}{7}
\setcounter{lesson_inequalities_containing_absolute_values}{8}
\setcounter{lesson_graphing_systems}{9}
\setcounter{lesson_substitution}{10}
\setcounter{lesson_elimination}{11}
\setcounter{lesson_quadratics_introduction}{16}
\setcounter{lesson_factoring_GCF}{17}
\setcounter{lesson_factoring_grouping}{18}
\setcounter{lesson_factoring_trinomials_a_is_1}{19}
\setcounter{lesson_factoring_trinomials_a_neq_1}{20}
\setcounter{lesson_solving_by_factoring}{21}
\setcounter{lesson_square_roots}{22}
\setcounter{lesson_i_and_complex_numbers}{23}
\setcounter{lesson_vertex_form_and_graphing}{24}
\setcounter{lesson_solve_by_square_roots}{25}
\setcounter{lesson_extracting_square_roots}{26}
\setcounter{lesson_the_discriminant}{27}
\setcounter{lesson_the_quadratic_formula}{28}
\setcounter{lesson_quadratic_inequalities}{29}
\setcounter{lesson_functions_and_relations}{12}
\setcounter{lesson_evaluating_functions}{13}
\setcounter{lesson_finding_domain_and_range_graphically}{14}
\setcounter{lesson_fundamental_functions}{15}
\setcounter{lesson_finding_domain_algebraically}{30}
\setcounter{lesson_solving_functions}{31}
\setcounter{lesson_function_arithmetic}{32}
\setcounter{lesson_composite_functions}{33}
\setcounter{lesson_inverse_functions_definition_and_HLT}{34}
\setcounter{lesson_finding_an_inverse_function}{35}
\setcounter{lesson_transformations_translations}{36}
\setcounter{lesson_transformations_reflections}{37}
\setcounter{lesson_transformations_scalings}{38}
\setcounter{lesson_transformations_summary}{39}
\setcounter{lesson_piecewise_functions}{40}
\setcounter{lesson_functions_containing_absolute_values}{41}
\setcounter{lesson_absolute_as_piecewise}{42}
\setcounter{lesson_polynomials_introduction}{43}
\setcounter{lesson_sign_diagrams_polynomials}{44}
\setcounter{lesson_factoring_quadratic_type}{46}
\setcounter{lesson_factoring_summary}{45}
\setcounter{lesson_polynomial_division}{47}
\setcounter{lesson_synthetic_division}{48}
\setcounter{lesson_end_behavior_polynomials}{49}
\setcounter{lesson_local_behavior_polynomials}{50}
\setcounter{lesson_rational_root_theorem}{51}
\setcounter{lesson_polynomials_graphing_summary}{52}
\setcounter{lesson_polynomial_inequalities}{53}
\setcounter{lesson_rationals_introduction_and_terminology}{54}
\setcounter{lesson_sign_diagrams_rationals}{55}
\setcounter{lesson_horizontal_asymptotes}{56}
\setcounter{lesson_slant_and_curvilinear_asymptotes}{57}
\setcounter{lesson_vertical_asymptotes}{58}
\setcounter{lesson_holes}{59}
\setcounter{lesson_rationals_graphing_summary}{60}

\begin{document}
{\bf \large Lesson \arabic{lesson_sign_diagrams_rationals}: Sign Diagrams for Rational Functions}\phantomsection\label{les:sign_diagrams_rationals}
%\\ CC attribute: \href{http://www.wallace.ccfaculty.org/book/book.html}{\it{Beginning and Intermediate Algebra}} by T. Wallace. 
\\ CC attribute: \href{http://www.stitz-zeager.com}{\it{College Algebra}} by C. Stitz and J. Zeager. 
\hfill \doclicenseImage[imagewidth=5em]\\
\par
{\bf Objective:} Solve rational inequalities by constructing a sign diagram.\\
\par
{\bf Students will be able to:}
\begin{itemize}
	\item Create a sign diagram for a rational expression. 
	\item Identify and express the solution to a rational inequality using interval notation.
\end{itemize}
{\bf Prerequisite Knowledge:}
\begin{itemize}
	\item Factoring.
	\item The Rational Root Theorem.
	\item Polynomial and \slash or synthetic division.
	\item Definition and associated terminology of a rational function.
	\item Domain of a rational function.
	\item Interval notation.
\end{itemize}
\hrulefill

{\bf Lesson:}\\
{\bf I - Motivating Example(s):}\\
\par
Whenever we are asked to find when a rational expression or function $f$ is positive, negative, $\geq 0$, or $\leq 0,$ we can always apply the following steps.
\begin{enumerate}
	\item Identify a complete factorization of the expression.
	\item Construct a sign diagram.
	\item Find all intervals that correspond to the desired inequality.
	\item In the case of $\geq$ or $\leq,$ make sure to include any $x-$intercepts.
\end{enumerate}
Solve the inequality $f(x)>0$ for the following function.
$$f(x)=\dfrac{-2x^3+6x^2+18x-54}{3x^3+12x^2-33x+18}$$
Using \Desmos \ to graph our function, we can see that our answer should be $(-6,-3)$.  To reach this result algebraically, we will need to find a factored form for $f$ in order to construct a sign diagram.  Using our prerequisite knowledge from factoring polynomials, specifically the Rational Root Theorem, one can obtain the following factorization.
$$f(x)=\frac{-2(x+3)(x-3)^2}{3(x+6)(x-1)^2}$$
To find the sign diagram for $f,$ we need to identify our $x-$intercepts, as well as those $x$ not in the domain.  From our factorization, we see that this is the set $x=\{-6,-3,1,3\},$ with $x\pm3$ being our intercepts.
\newpage
Our diagram is shown below.
\begin{center}
\begin{tikzpicture}[xscale=1,yscale=1]
	\draw [<->](-8.25,0) -- coordinate (x axis mid) (5.25,0) node[below right] {$x$};
	\draw [-, dashed](-6,1) -- coordinate (y axis mid) (-6,-0.25) node[below] {$-6$};
	\draw [-](-3,1) -- coordinate (y axis mid) (-3,-0.25) node[below] {$-3$};
	\draw [-](3,1) -- coordinate (y axis mid) (3,-0.25) node[below] {$3$};
	\draw [-, dashed](1,1) -- coordinate (y axis mid) (1,-0.25) node[below] {$1$};
	\draw (-7,-1) node {$x=-7$};
	\draw (-4.5,-1) node {$x=-5$};
	\draw (-1,-1) node {$x=0$};
	\draw (2,-1) node {$x=2$};
	\draw (4,-1) node {$x=4$};
	\draw (-7,0.5) node {$-$};
	\draw (-4.5,0.5) node {$+$};
	\draw (-1,0.5) node {$-$};
	\draw (2,0.5) node {$-$};
	\draw (4,0.5) node {$-$};
	\draw (-7,-1.75) node {\footnotesize $\dfrac{(-)(-)}{(+)(-)}$};
	\draw (-4.5,-1.75) node {\footnotesize $\dfrac{(-)(-)}{(+)(+)}$};
	\draw (-1,-1.75) node {\footnotesize $\dfrac{(-)(+)}{(+)(+)}$};
	\draw (2,-1.75) node {\footnotesize $\dfrac{(-)(+)}{(+)(+)}$};
	\draw (4,-1.75) node {\footnotesize $\dfrac{(-)(+)}{(+)(+)}$};
\end{tikzpicture}
\end{center}
One important observation in our diagram is in the calculation of each sign.  For each test value, we have excluded the {\it squared} factors in the numerator and denominator, since both $(x-3)^2$ and $(x-1)^2$ will always contribute a positive sign and not affect the end result.  For example, when $x=0,$ we get $$\dfrac{(-)(+)(-)^2}{(+)(+)(-)^2},$$ which reduces to the result that we see above.  Similarly, we could have excluded the $(+)$ that appears in the denominator of each test value's sign calculation, since the constant multiplier of $3$ will have no impact on sign.\\
\ \par
At this point we are essentially done, since the factorization and construction of our diagram has done the bulk of the work for us.  Since we are asked to find all $x$ such that $f(x)>0,$ we see that this equals to the union of all intervals that correspond to a $+$ sign.  This gives us our anticipated answer of $(-6,-3)$.\\
\ \par
Recalling our discussion of the last example, if we wished to answer the follow-up question of when $f(x)\geq 0,$ we would just need to include all boundary values in our diagram that correspond to $x-$intercepts (when $f(x)=0$).  From our diagram, this would be any value of $x$ that has a {\it solid} divider, remembering that dashed dividers correspond to values not in our domain.  In this case, the function $f(x)\geq 0$ for all $x$ in the set $(-6,-3]\cup\{3\}$.\\
\ \par
{\bf II - Demo/Discussion Problems:}\\
\ \par
Solve each of the following inequalities.  Use \Desmos \ to confirm your answers. 
\begin{enumerate}
\begin{multicols}{2}
\item $\dfrac{4x^2-4x+1}{x^3-x^2-17x-15}\leq 0$
\item $\dfrac{x-6}{x}\geq\dfrac{-2}{x-1}$
\end{multicols}
\end{enumerate}
\newpage
{\bf III - Practice Problems:}\\
\ \par
Solve each of the following inequalities.  Use \Desmos \ to confirm your answers.
\begin{enumerate}
\item $\dfrac{(x-5)(x+4)}{x-1}\leq 0$
\item $\dfrac{x+1}{x-1}\geq\ 0$
\item $\dfrac{(x+1)(x-3)}{x+2}\geq\ 0$
\item $\dfrac{x^2-25}{x^2-1}\leq\ 0$ 
\item $\dfrac{x^2+x-12}{x^3-25x}\leq\ 0$
\end{enumerate}
\newpage
\ \newpage
\end{document}