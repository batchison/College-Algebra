\documentclass[12pt]{article}
\usepackage[top=1in,left=1in,bottom=1in,right=1in,headsep=2pt]{geometry}	
\usepackage{amssymb,amsmath,amsthm,amsfonts}
\usepackage{chapterfolder,docmute,setspace}
\usepackage{cancel,multicol,tikz,verbatim,framed,polynom,enumitem}
\usepackage[colorlinks, hyperindex, plainpages=false, linkcolor=blue, urlcolor=blue, pdfpagelabels]{hyperref}
% Use the cc-by-nc-sa license for any content linked with Stitz and Zeager's text.  Otherwise, use the cc-by-sa license.
%\usepackage[type={CC},modifier={by-sa},version={4.0},]{doclicense}
\usepackage[type={CC},modifier={by-nc-sa},version={4.0},]{doclicense}

\theoremstyle{definition}
\newtheorem{example}{Example}
\newcommand{\Desmos}{\href{https://www.desmos.com/}{Desmos}}
\setlength{\parindent}{0em}
\setlist{itemsep=0em}
\setlength{\parskip}{0.1em}
% This document is used for ordering of lessons.  If an instructor wishes to change the ordering of assessments, the following steps must be taken:

% 1) Reassign the appropriate numbers for each lesson in the \setcounter commands included in this file.
% 2) Rearrange the \include commands in the master file (the file with 'Course Pack' in the name) to accurately reflect the changes.  
% 3) Rearrange the \items in the measureable_outcomes file to accurately reflect the changes.  Be mindful of page breaks when moving items.
% 4) Re-build all affected files (master file, measureable_outcomes file, and any lesson whose numbering has changed).

%Note: The placement of each \newcounter and \setcounter command reflects the original/default ordering of topics (linears, systems, quadratics, functions, polynomials, rationals).

\newcounter{lesson_solving_linear_equations}
\newcounter{lesson_equations_containing_absolute_values}
\newcounter{lesson_graphing_lines}
\newcounter{lesson_two_forms_of_a_linear_equation}
\newcounter{lesson_parallel_and_perpendicular_lines}
\newcounter{lesson_linear_inequalities}
\newcounter{lesson_compound_inequalities}
\newcounter{lesson_inequalities_containing_absolute_values}
\newcounter{lesson_graphing_systems}
\newcounter{lesson_substitution}
\newcounter{lesson_elimination}
\newcounter{lesson_quadratics_introduction}
\newcounter{lesson_factoring_GCF}
\newcounter{lesson_factoring_grouping}
\newcounter{lesson_factoring_trinomials_a_is_1}
\newcounter{lesson_factoring_trinomials_a_neq_1}
\newcounter{lesson_solving_by_factoring}
\newcounter{lesson_square_roots}
\newcounter{lesson_i_and_complex_numbers}
\newcounter{lesson_vertex_form_and_graphing}
\newcounter{lesson_solve_by_square_roots}
\newcounter{lesson_extracting_square_roots}
\newcounter{lesson_the_discriminant}
\newcounter{lesson_the_quadratic_formula}
\newcounter{lesson_quadratic_inequalities}
\newcounter{lesson_functions_and_relations}
\newcounter{lesson_evaluating_functions}
\newcounter{lesson_finding_domain_and_range_graphically}
\newcounter{lesson_fundamental_functions}
\newcounter{lesson_finding_domain_algebraically}
\newcounter{lesson_solving_functions}
\newcounter{lesson_function_arithmetic}
\newcounter{lesson_composite_functions}
\newcounter{lesson_inverse_functions_definition_and_HLT}
\newcounter{lesson_finding_an_inverse_function}
\newcounter{lesson_transformations_translations}
\newcounter{lesson_transformations_reflections}
\newcounter{lesson_transformations_scalings}
\newcounter{lesson_transformations_summary}
\newcounter{lesson_piecewise_functions}
\newcounter{lesson_functions_containing_absolute_values}
\newcounter{lesson_absolute_as_piecewise}
\newcounter{lesson_polynomials_introduction}
\newcounter{lesson_sign_diagrams_polynomials}
\newcounter{lesson_factoring_quadratic_type}
\newcounter{lesson_factoring_summary}
\newcounter{lesson_polynomial_division}
\newcounter{lesson_synthetic_division}
\newcounter{lesson_end_behavior_polynomials}
\newcounter{lesson_local_behavior_polynomials}
\newcounter{lesson_rational_root_theorem}
\newcounter{lesson_polynomials_graphing_summary}
\newcounter{lesson_polynomial_inequalities}
\newcounter{lesson_rationals_introduction_and_terminology}
\newcounter{lesson_sign_diagrams_rationals}
\newcounter{lesson_horizontal_asymptotes}
\newcounter{lesson_slant_and_curvilinear_asymptotes}
\newcounter{lesson_vertical_asymptotes}
\newcounter{lesson_holes}
\newcounter{lesson_rationals_graphing_summary}

\setcounter{lesson_solving_linear_equations}{1}
\setcounter{lesson_equations_containing_absolute_values}{2}
\setcounter{lesson_graphing_lines}{3}
\setcounter{lesson_two_forms_of_a_linear_equation}{4}
\setcounter{lesson_parallel_and_perpendicular_lines}{5}
\setcounter{lesson_linear_inequalities}{6}
\setcounter{lesson_compound_inequalities}{7}
\setcounter{lesson_inequalities_containing_absolute_values}{8}
\setcounter{lesson_graphing_systems}{9}
\setcounter{lesson_substitution}{10}
\setcounter{lesson_elimination}{11}
\setcounter{lesson_quadratics_introduction}{16}
\setcounter{lesson_factoring_GCF}{17}
\setcounter{lesson_factoring_grouping}{18}
\setcounter{lesson_factoring_trinomials_a_is_1}{19}
\setcounter{lesson_factoring_trinomials_a_neq_1}{20}
\setcounter{lesson_solving_by_factoring}{21}
\setcounter{lesson_square_roots}{22}
\setcounter{lesson_i_and_complex_numbers}{23}
\setcounter{lesson_vertex_form_and_graphing}{24}
\setcounter{lesson_solve_by_square_roots}{25}
\setcounter{lesson_extracting_square_roots}{26}
\setcounter{lesson_the_discriminant}{27}
\setcounter{lesson_the_quadratic_formula}{28}
\setcounter{lesson_quadratic_inequalities}{29}
\setcounter{lesson_functions_and_relations}{12}
\setcounter{lesson_evaluating_functions}{13}
\setcounter{lesson_finding_domain_and_range_graphically}{14}
\setcounter{lesson_fundamental_functions}{15}
\setcounter{lesson_finding_domain_algebraically}{30}
\setcounter{lesson_solving_functions}{31}
\setcounter{lesson_function_arithmetic}{32}
\setcounter{lesson_composite_functions}{33}
\setcounter{lesson_inverse_functions_definition_and_HLT}{34}
\setcounter{lesson_finding_an_inverse_function}{35}
\setcounter{lesson_transformations_translations}{36}
\setcounter{lesson_transformations_reflections}{37}
\setcounter{lesson_transformations_scalings}{38}
\setcounter{lesson_transformations_summary}{39}
\setcounter{lesson_piecewise_functions}{40}
\setcounter{lesson_functions_containing_absolute_values}{41}
\setcounter{lesson_absolute_as_piecewise}{42}
\setcounter{lesson_polynomials_introduction}{43}
\setcounter{lesson_sign_diagrams_polynomials}{44}
\setcounter{lesson_factoring_quadratic_type}{46}
\setcounter{lesson_factoring_summary}{45}
\setcounter{lesson_polynomial_division}{47}
\setcounter{lesson_synthetic_division}{48}
\setcounter{lesson_end_behavior_polynomials}{49}
\setcounter{lesson_local_behavior_polynomials}{50}
\setcounter{lesson_rational_root_theorem}{51}
\setcounter{lesson_polynomials_graphing_summary}{52}
\setcounter{lesson_polynomial_inequalities}{53}
\setcounter{lesson_rationals_introduction_and_terminology}{54}
\setcounter{lesson_sign_diagrams_rationals}{55}
\setcounter{lesson_horizontal_asymptotes}{56}
\setcounter{lesson_slant_and_curvilinear_asymptotes}{57}
\setcounter{lesson_vertical_asymptotes}{58}
\setcounter{lesson_holes}{59}
\setcounter{lesson_rationals_graphing_summary}{60}

\begin{document}
{\bf \large Lesson \arabic{lesson_polynomials_introduction}: Polynomials Introduction}\phantomsection\label{les:polynomials_introduction}
%\\ CC attribute: \href{http://www.wallace.ccfaculty.org/book/book.html}{\it{Beginning and Intermediate Algebra}} by T. Wallace. 
\\ CC attribute: \href{http://www.stitz-zeager.com}{\it{College Algebra}} by C. Stitz and J. Zeager. 
\hfill \doclicenseImage[imagewidth=5em]\\
\par
{\bf Objective:} Identify key features of and classify a polynomial by degree and number of nonzero terms.\\
\par
{\bf Students will be able to:}
\begin{itemize}
	\item Identify a polynomial from its definition, and arrange a polynomial in descending-power order.
	\item Identify the degree, set of coefficients, leading coefficient, leading and constant term of a polynomial.
	\item Classify a polynomial by both its degree and number of nonzero terms.
\end{itemize}
{\bf Prerequisite Knowledge:}
\begin{itemize}
	\item Order of operations.
\end{itemize}
\hrulefill

{\bf Lesson:}\\
\ \par
A {\it polynomial} in terms of a variable $x$ is a function of the form
$$f(x) = a_{n}x^{n} + a_{n-1}x^{n-1}+ ... + a_{2}x^2 + a_{1}x + a_{0},$$
where each {\it coefficient}, $a_{i}$, is a real number ($a_n\neq 0$) and the exponent, or {\it degree} of the polynomial, $n$, is a nonnegative integer.
\par
Examples of polynomials include: $f(x) = x^2 + 5$, $f(x)=x$ and $f(x) = -3x^7+4x^3-5x$.  For our general polynomial above, the:
\begin{center}
\begin{tabular}{lcl}
{\it degree} & is & $n$\\
{\it set of coefficients} & is & $\{a_n,a_{n-1},\ldots,a_1,a_0\}$\\
{\it leading coefficient} & is & $a_n$\\
{\it leading term} & is & $a_nx^n$\\
{\it constant term} & is & $a_0x^0=a_0$.
\end{tabular}
\end{center}
We will categorize polynomials based upon their degree, as well as the number of terms, after all necessary simplification.
\begin{center}
{\bf Polynomials Classification by Degree}
\par
\begin{tabular}{ | c | c | c | } 
\hline
Degree & Type & Example \\ 
\hline
0 & Constant & $-1$ \\ 
\hline
1 & Linear & $2x+\sqrt{5}$ \\ 
\hline
2 & Quadratic & $5x^2 - 32x+2$ \\ 
\hline
3 & Cubic & $-0.5x^{3}$ \\ 
\hline
4 & Quartic & $-3x^{4} +2x^2+3x + 1$ \\ 
\hline
5  & Quintic & $-2x^5$ \\ 
\hline
6 or more  & $n^{\text{th}}$ Degree & $-2x^{7} + 52x^6 + 12$ \\ 
\hline
\end{tabular}
\end{center}
One point of note in the table above is the appearance of both rational and irrational coefficients $\left(-0.5 \ \text{and} \ \sqrt{5}\right)$.  The appearance of such coefficients is permissible in polynomials, since our coefficients $a_i$ are only required to be real numbers.  A coefficient containing the imaginary number $i=\sqrt{-1}$, on the other hand, is not permitted.
\par
\begin{center}
{\bf Polynomial Classification by Number of Nonzero Terms}
\par
\begin{tabular}{ | c | c | c | } 
\hline
Number of Terms & Name & Example \\ 
\hline
1 & Monomial & $4x^5$ \\ 
\hline
2 & Binomial & $2x^3 +1$ \\ 
\hline
3 & Trinomial & $-23x^{18} +4x^2+3x$ \\ 
\hline
4 & Tetranomial & $-23x^{18} +4x^2+3x + 1$ \\ 
\hline
5 or more & Polynomial & $-2x^4 + x^3 +15x^2-41x + 12$ \\ 
\hline
\end{tabular}
\end{center}

{\bf I - Motivating Example(s):}\\
\ \par

{\bf Example:} Identify the degree, set of coefficients, leading coefficient, leading term and constant term for the polynomial
$$f(x) = -19x^5+4x^4-6x+21.$$
Classify the polynomial by both degree and number of nonzero terms.
\begin{itemize}
	\item The degree of this polynomial is $n=5$, since five is the greatest exponent.
	\item The leading term, which is the term that contains the greatest exponent (degree), is $a_nx^n=-19x^5$.
	\item The leading coefficient is the real number being multiplied by $x^n$ in the leading term, namely $a_n=-19$.
	\item The constant term is $a_0=21$, which also represents the $y-$intercept for the graph of the given polynomial.
	\item The complete set of coefficients for the given polynomial is
$$\{a_5=-19, \ a_4=4, \ a_3=0, \ a_2=0, \ a_1=-6, \ a_0=21\}.$$
	\item The polynomial is a quintic tetranomial, as it is degree 5 and consists of 4 nonzeroterms.
\end{itemize}
{\bf II - Demo/Discussion Problems:}\\
\ \par
Identify the degree, set of coefficients, leading coefficient, leading term and constant term for each of the polynomials listed.  Classify each polynomial by both degree and number of nonzero terms.  If it is not already provided, write the polynomial in descending-power order.
\begin{multicols}{2}
\begin{enumerate}
\item $f(x)=1$
\item $g(x) = x^3-x^2$
\item $h(x) = x^2-x^3$
\item $k(x)=\sqrt{2}x^4+\pi x^2-e$
\item $\ell(x) = 21x^4+12x^2-3x^2-9x^2-22x^4$
\item $m(x)=3(x+1)(x-1)+2x+4x^3+3$
\end{enumerate}
\end{multicols}
{\bf III - Practice Problems:}\\
\ \par
Identify the degree, set of coefficients, leading coefficient, leading term and constant term for each of the polynomials listed.  Classify each polynomial by both degree and number of nonzero terms.  If it is not already provided, write the polynomial in descending-power order.
\begin{multicols}{2}
\begin{enumerate}
\item $f(x)=-2x^3-1$
\item $f(x)=-2x^4 + 4x+1$
\item $f(x)=40-x^3$
\item $f(x)=(x-1)^2$
\item $f(x)=32x^5+x^2+x$
\item $f(x)=4x^2-3x^4$
\item $f(x)=-2x^4-4x^2-6x-8$
\item $f(x)=5x+3x^2+x^3+\sqrt{3}$
\item $f(x)=\frac{1}{2}x^4-5x^2-\frac{1}{2}$
\item $f(x)=12-6x+3x^2-2x^3-x^6$
\item $f(x)=-3x^4+15x^3+x^2-27x^3-x^2-13$
\end{enumerate}
\end{multicols}
\newpage
\ \newpage
\end{document}