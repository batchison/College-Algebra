\documentclass[12pt]{article}
\usepackage[top=1in,left=1in,bottom=1in,right=1in,headsep=2pt]{geometry}	
\usepackage{amssymb,amsmath,amsthm,amsfonts}
\usepackage{chapterfolder,docmute,setspace}
\usepackage{cancel,multicol,tikz,verbatim,framed,polynom,enumitem}
\usepackage[colorlinks, hyperindex, plainpages=false, linkcolor=blue, urlcolor=blue, pdfpagelabels]{hyperref}
% Use the cc-by-nc-sa license for any content linked with Stitz and Zeager's text.  Otherwise, use the cc-by-sa license.
%\usepackage[type={CC},modifier={by-sa},version={4.0},]{doclicense}
\usepackage[type={CC},modifier={by-nc-sa},version={4.0},]{doclicense}

\theoremstyle{definition}
\newtheorem{example}{Example}
\newcommand{\Desmos}{\href{https://www.desmos.com/}{Desmos}}
\setlength{\parindent}{0em}
\setlist{itemsep=0em}
\setlength{\parskip}{0.1em}
% This document is used for ordering of lessons.  If an instructor wishes to change the ordering of assessments, the following steps must be taken:

% 1) Reassign the appropriate numbers for each lesson in the \setcounter commands included in this file.
% 2) Rearrange the \include commands in the master file (the file with 'Course Pack' in the name) to accurately reflect the changes.  
% 3) Rearrange the \items in the measureable_outcomes file to accurately reflect the changes.  Be mindful of page breaks when moving items.
% 4) Re-build all affected files (master file, measureable_outcomes file, and any lesson whose numbering has changed).

%Note: The placement of each \newcounter and \setcounter command reflects the original/default ordering of topics (linears, systems, quadratics, functions, polynomials, rationals).

\newcounter{lesson_solving_linear_equations}
\newcounter{lesson_equations_containing_absolute_values}
\newcounter{lesson_graphing_lines}
\newcounter{lesson_two_forms_of_a_linear_equation}
\newcounter{lesson_parallel_and_perpendicular_lines}
\newcounter{lesson_linear_inequalities}
\newcounter{lesson_compound_inequalities}
\newcounter{lesson_inequalities_containing_absolute_values}
\newcounter{lesson_graphing_systems}
\newcounter{lesson_substitution}
\newcounter{lesson_elimination}
\newcounter{lesson_quadratics_introduction}
\newcounter{lesson_factoring_GCF}
\newcounter{lesson_factoring_grouping}
\newcounter{lesson_factoring_trinomials_a_is_1}
\newcounter{lesson_factoring_trinomials_a_neq_1}
\newcounter{lesson_solving_by_factoring}
\newcounter{lesson_square_roots}
\newcounter{lesson_i_and_complex_numbers}
\newcounter{lesson_vertex_form_and_graphing}
\newcounter{lesson_solve_by_square_roots}
\newcounter{lesson_extracting_square_roots}
\newcounter{lesson_the_discriminant}
\newcounter{lesson_the_quadratic_formula}
\newcounter{lesson_quadratic_inequalities}
\newcounter{lesson_functions_and_relations}
\newcounter{lesson_evaluating_functions}
\newcounter{lesson_finding_domain_and_range_graphically}
\newcounter{lesson_fundamental_functions}
\newcounter{lesson_finding_domain_algebraically}
\newcounter{lesson_solving_functions}
\newcounter{lesson_function_arithmetic}
\newcounter{lesson_composite_functions}
\newcounter{lesson_inverse_functions_definition_and_HLT}
\newcounter{lesson_finding_an_inverse_function}
\newcounter{lesson_transformations_translations}
\newcounter{lesson_transformations_reflections}
\newcounter{lesson_transformations_scalings}
\newcounter{lesson_transformations_summary}
\newcounter{lesson_piecewise_functions}
\newcounter{lesson_functions_containing_absolute_values}
\newcounter{lesson_absolute_as_piecewise}
\newcounter{lesson_polynomials_introduction}
\newcounter{lesson_sign_diagrams_polynomials}
\newcounter{lesson_factoring_quadratic_type}
\newcounter{lesson_factoring_summary}
\newcounter{lesson_polynomial_division}
\newcounter{lesson_synthetic_division}
\newcounter{lesson_end_behavior_polynomials}
\newcounter{lesson_local_behavior_polynomials}
\newcounter{lesson_rational_root_theorem}
\newcounter{lesson_polynomials_graphing_summary}
\newcounter{lesson_polynomial_inequalities}
\newcounter{lesson_rationals_introduction_and_terminology}
\newcounter{lesson_sign_diagrams_rationals}
\newcounter{lesson_horizontal_asymptotes}
\newcounter{lesson_slant_and_curvilinear_asymptotes}
\newcounter{lesson_vertical_asymptotes}
\newcounter{lesson_holes}
\newcounter{lesson_rationals_graphing_summary}

\setcounter{lesson_solving_linear_equations}{1}
\setcounter{lesson_equations_containing_absolute_values}{2}
\setcounter{lesson_graphing_lines}{3}
\setcounter{lesson_two_forms_of_a_linear_equation}{4}
\setcounter{lesson_parallel_and_perpendicular_lines}{5}
\setcounter{lesson_linear_inequalities}{6}
\setcounter{lesson_compound_inequalities}{7}
\setcounter{lesson_inequalities_containing_absolute_values}{8}
\setcounter{lesson_graphing_systems}{9}
\setcounter{lesson_substitution}{10}
\setcounter{lesson_elimination}{11}
\setcounter{lesson_quadratics_introduction}{16}
\setcounter{lesson_factoring_GCF}{17}
\setcounter{lesson_factoring_grouping}{18}
\setcounter{lesson_factoring_trinomials_a_is_1}{19}
\setcounter{lesson_factoring_trinomials_a_neq_1}{20}
\setcounter{lesson_solving_by_factoring}{21}
\setcounter{lesson_square_roots}{22}
\setcounter{lesson_i_and_complex_numbers}{23}
\setcounter{lesson_vertex_form_and_graphing}{24}
\setcounter{lesson_solve_by_square_roots}{25}
\setcounter{lesson_extracting_square_roots}{26}
\setcounter{lesson_the_discriminant}{27}
\setcounter{lesson_the_quadratic_formula}{28}
\setcounter{lesson_quadratic_inequalities}{29}
\setcounter{lesson_functions_and_relations}{12}
\setcounter{lesson_evaluating_functions}{13}
\setcounter{lesson_finding_domain_and_range_graphically}{14}
\setcounter{lesson_fundamental_functions}{15}
\setcounter{lesson_finding_domain_algebraically}{30}
\setcounter{lesson_solving_functions}{31}
\setcounter{lesson_function_arithmetic}{32}
\setcounter{lesson_composite_functions}{33}
\setcounter{lesson_inverse_functions_definition_and_HLT}{34}
\setcounter{lesson_finding_an_inverse_function}{35}
\setcounter{lesson_transformations_translations}{36}
\setcounter{lesson_transformations_reflections}{37}
\setcounter{lesson_transformations_scalings}{38}
\setcounter{lesson_transformations_summary}{39}
\setcounter{lesson_piecewise_functions}{40}
\setcounter{lesson_functions_containing_absolute_values}{41}
\setcounter{lesson_absolute_as_piecewise}{42}
\setcounter{lesson_polynomials_introduction}{43}
\setcounter{lesson_sign_diagrams_polynomials}{44}
\setcounter{lesson_factoring_quadratic_type}{46}
\setcounter{lesson_factoring_summary}{45}
\setcounter{lesson_polynomial_division}{47}
\setcounter{lesson_synthetic_division}{48}
\setcounter{lesson_end_behavior_polynomials}{49}
\setcounter{lesson_local_behavior_polynomials}{50}
\setcounter{lesson_rational_root_theorem}{51}
\setcounter{lesson_polynomials_graphing_summary}{52}
\setcounter{lesson_polynomial_inequalities}{53}
\setcounter{lesson_rationals_introduction_and_terminology}{54}
\setcounter{lesson_sign_diagrams_rationals}{55}
\setcounter{lesson_horizontal_asymptotes}{56}
\setcounter{lesson_slant_and_curvilinear_asymptotes}{57}
\setcounter{lesson_vertical_asymptotes}{58}
\setcounter{lesson_holes}{59}
\setcounter{lesson_rationals_graphing_summary}{60}

\begin{document}
{\bf \large Lesson \arabic{lesson_extracting_square_roots}: Solve by Extracting Square Roots}\phantomsection\label{les:extracting_square_roots}
%\\ CC attribute: \href{http://www.wallace.ccfaculty.org/book/book.html}{\it{Beginning and Intermediate Algebra}} by T. Wallace. 
\\ CC attribute: \href{http://www.stitz-zeager.com}{\it{College Algebra}} by C. Stitz and J. Zeager. 
\hfill \doclicenseImage[imagewidth=5em]\\
\par
{\bf Objective:} Solve quadratic equations using the method of extracting square roots.\\
\par
{\bf Students will be able to:}
\begin{itemize}
	\item Solve quadratic equations in vertex form as an alternative to factoring or when factoring fails.
	\item Approximate an irrational root to a quadratic equation for the purposes of graphing.
\end{itemize}
{\bf Prerequisite Knowledge:}
\begin{itemize}
	\item Simplifying radicals.
	\item Solving by isolating a quadratic term.
	\item Complex numbers.
\end{itemize}
\hrulefill

{\bf Lesson:}\\
\ \par
We now introduce a new technique for solving quadratic equations, known as {\it extracting square roots}.  This method will only be employed once we have identified the vertex form for a given quadratic, $y=a(x-h)^2+k$.  The general steps for extracting square roots are shown in our first example, and the requirement of the vertex form will be essential.\\
\ \par
{\bf I - Motivating Example(s):}\\
\ \par
{\bf Example:} Determine the zeros of the quadratic equation $y=ax^2+bx+c$, where $a\neq 0$.\\
First obtain the vertex form:  $h=-\dfrac{b}{2a}$, set $x=h$ to find $k$.
\begin{eqnarray*}
a(x-h)^2+\cancel{k}=0~~~ & &\text{Vertex~form}\\
\underline{-\cancel{k}}~~~\underline{-k}~ & & \text{Subtract~} k \text{~from~both~sides}\\
\cancel{a}(x-h)^2=-k & &\\
\overline{~~~~~\cancel{a}~~~~}~~~~~~\overline{~a~} & &  \text{Divide~both~sides~by~} a\\
(x-h)^2=-\frac{k}{a}& &\\
\sqrt{(x-h)^2}=\pm\sqrt{-\frac{k}{a}}& &\text{Take~square~root~of~both~sides}\\
& & ~~~\text{to~extract~radicand,~} x-h\\
%\end{eqnarray*}
%\begin{eqnarray*}
x-\cancel{h}=\pm\sqrt{-\frac{k}{a}} & & \\
\underline{+\cancel{h}}~~~~~~~~~~\underline{+h} & & \text{Add~} h \text{~to~both~sides}\\
x=h\pm\sqrt{-\frac{k}{a}} & & \text{Our~solution}
\end{eqnarray*}

{\bf Example:} Use the method of extracting square roots to find the zeros of the equation $y = (x+4)^2-9$.
\begin{multicols}{2}
\begin{center}
\begin{tikzpicture}[xscale=0.45,yscale=0.45]
	\draw [<->](-8.5,0) -- coordinate (x axis mid) (0.5,0) node[below right] {$x$};
	\draw [<->](0,-9.5) -- coordinate (x axis mid) (0,2) node[above right] {$y$};
	\draw [<->] plot [domain=-7.25:-0.75, samples=100] (\x,{(\x+4)^2-9});
	\foreach \x in {-7,-5,...,-1}
		\draw (\x,2pt) -- (\x,-2pt)	node[anchor=north] {\scriptsize \x};
	\foreach \y in {-1,-3,...,-9}
		\draw (2pt,\y) -- (-2pt,\y)	node[anchor=west] {\scriptsize \y}; 
	\foreach \y in {1}
		\draw (2pt,\y) -- (-2pt,\y)	node[anchor=west] {\scriptsize \y}; 
 \draw[fill] (-4,-9) ellipse (0.1 and 0.1);
 \draw[fill] (-1,0) ellipse (0.1 and 0.1);
 \draw[fill] (-7,0) ellipse (0.1 and 0.1);
\end{tikzpicture}
\end{center}

\columnbreak

\begin{eqnarray*}
%    \ y = (x+4)^2-9 &  & \tmop{A}  \tmop{quadratic} \tmop{in} \tmop{vertex} \tmop{form}\\
    \ 0 = (x+4)^2-9 &  &  \text{Set equal to zero}\\
    \ {\bf \underline{+9}~~~~~~~~~~~~~~~\underline{+9}}   & & \text{Isolate the square}\\
    \ 9=(x+4)^2~~~~~ & & \text{Introduce a}\\
		\pm\sqrt{9} = \sqrt{(x+4)^2}~~ & &  ~~\text{square root; include} \ \pm \\
		\ \pm3 = x + 4~~~~~~~~~  & &  \text{Solve for} \ x\\
		\ {\bf \underline{-4}~~~~~~\underline{-4}}~~~~~~~~~  & & \\
		\ x=\pm3 - 4~~~~~~~ & & \text{Two solutions} \\
		\ x=3 - 4~~ \Rightarrow x = -1   & & \text{First solution}\\
		\ x=-3 - 4 \Rightarrow x = -7  & & \text{Second solution}
\end{eqnarray*}
\end{multicols}
Our zeros are $x=-7$ and $-1$.  The corresponding $x-$intercepts are at $(-7,0)$ and $(-1,0)$.\\
\ \par
{\bf Example:} Use the method of extracting square roots to find the zeros of the equation $y = -2(x+3)^2+48$.\\
\ \par
We show our solution below, this time omitting each step in the overall simplification.
\begin{eqnarray*}
-2(x+3)^2+48 & = & 0 \\
-2(x+3)^2&=&-48 \\
(x+3)^2&=&24 \\
\sqrt{(x+3)^2}&=&\pm\sqrt{24}\\
x+3&=&\pm\sqrt{4}\sqrt{6}\\
x&=&-3\pm2\sqrt{6}
\end{eqnarray*}
Note that we can approximate our two roots, by realizing that 
$$2=\sqrt{4}<\sqrt{6}<\sqrt{9}=3.$$
Since $\sqrt{6}\approx 2.4,$ we can say that our two roots are $x\approx -3\pm 4.8.$  This reduces to $x\approx 1.8$ and $x\approx -7.8$.  We may conclude that our $x-$intercepts are approximately located at $(1.8,0)$ and $(-7.8,0)$.
\newpage
{\bf II - Demo/Discussion Problems:}\\
\ \par
Solve each of the following equations for all possible $x$.  Classify each solution as either real or imaginary.  If your answer includes a square root, find a decimal approximation for your answer(s).
\begin{enumerate}
	\item $y=-3(x-1)^2+12$
	\item $y=(x+5)^2-8$
	\item $y=\dfrac{1}{8}(x-6)^2-5$
	\item $y=-1(x+3)^2-4$
	\item $y=-\dfrac{1}{4}(x+3)^2+27$
	\end{enumerate}
\ \par
{\bf III - Practice Problems:}\\
\ \par
Solve each of the following equations for all possible $x$.  Classify each solution as either real or imaginary.  If your answer includes a square root, find a decimal approximation for your answer(s).
\begin{multicols}{3}
	\begin{enumerate}
	\item $y=2(x-4)^2-200$
  \item $y=-2(x-7)^2+50$
  \item $y=(x-4)^2-98$
  \item $y=(x-12)^2-5$
  \item $y=x^2+18$
  \item $y=(x-16)^2$
  \item $y=-3(x-3)^2+30$
  \item $y=-4(x-1)^2+20$
  \item $y=-4(x+6)^2+8$
  \item $y=\dfrac{1}{20}(x-1)^2-15$
  \item $y=(x+2)^2+12$
  \item $y=9(x-11)^2-81$
	\end{enumerate}
\end{multicols}
\newpage
\ \newpage
\end{document}