\documentclass[12pt]{article}
\usepackage[top=1in,left=1in,bottom=1in,right=1in,headsep=2pt]{geometry}	
\usepackage{amssymb,amsmath,amsthm,amsfonts}
\usepackage{chapterfolder,docmute,setspace}
\usepackage{cancel,multicol,tikz,verbatim,framed,polynom,enumitem}
\usepackage[colorlinks, hyperindex, plainpages=false, linkcolor=blue, urlcolor=blue, pdfpagelabels]{hyperref}
% Use the cc-by-nc-sa license for any content linked with Stitz and Zeager's text.  Otherwise, use the cc-by-sa license.
%\usepackage[type={CC},modifier={by-sa},version={4.0},]{doclicense}
\usepackage[type={CC},modifier={by-nc-sa},version={4.0},]{doclicense}

\theoremstyle{definition}
\newtheorem{example}{Example}
\newcommand{\Desmos}{\href{https://www.desmos.com/}{Desmos}}
\setlength{\parindent}{0em}
\setlist{itemsep=0em}
\setlength{\parskip}{0.1em}
% This document is used for ordering of lessons.  If an instructor wishes to change the ordering of assessments, the following steps must be taken:

% 1) Reassign the appropriate numbers for each lesson in the \setcounter commands included in this file.
% 2) Rearrange the \include commands in the master file (the file with 'Course Pack' in the name) to accurately reflect the changes.  
% 3) Rearrange the \items in the measureable_outcomes file to accurately reflect the changes.  Be mindful of page breaks when moving items.
% 4) Re-build all affected files (master file, measureable_outcomes file, and any lesson whose numbering has changed).

%Note: The placement of each \newcounter and \setcounter command reflects the original/default ordering of topics (linears, systems, quadratics, functions, polynomials, rationals).

\newcounter{lesson_solving_linear_equations}
\newcounter{lesson_equations_containing_absolute_values}
\newcounter{lesson_graphing_lines}
\newcounter{lesson_two_forms_of_a_linear_equation}
\newcounter{lesson_parallel_and_perpendicular_lines}
\newcounter{lesson_linear_inequalities}
\newcounter{lesson_compound_inequalities}
\newcounter{lesson_inequalities_containing_absolute_values}
\newcounter{lesson_graphing_systems}
\newcounter{lesson_substitution}
\newcounter{lesson_elimination}
\newcounter{lesson_quadratics_introduction}
\newcounter{lesson_factoring_GCF}
\newcounter{lesson_factoring_grouping}
\newcounter{lesson_factoring_trinomials_a_is_1}
\newcounter{lesson_factoring_trinomials_a_neq_1}
\newcounter{lesson_solving_by_factoring}
\newcounter{lesson_square_roots}
\newcounter{lesson_i_and_complex_numbers}
\newcounter{lesson_vertex_form_and_graphing}
\newcounter{lesson_solve_by_square_roots}
\newcounter{lesson_extracting_square_roots}
\newcounter{lesson_the_discriminant}
\newcounter{lesson_the_quadratic_formula}
\newcounter{lesson_quadratic_inequalities}
\newcounter{lesson_functions_and_relations}
\newcounter{lesson_evaluating_functions}
\newcounter{lesson_finding_domain_and_range_graphically}
\newcounter{lesson_fundamental_functions}
\newcounter{lesson_finding_domain_algebraically}
\newcounter{lesson_solving_functions}
\newcounter{lesson_function_arithmetic}
\newcounter{lesson_composite_functions}
\newcounter{lesson_inverse_functions_definition_and_HLT}
\newcounter{lesson_finding_an_inverse_function}
\newcounter{lesson_transformations_translations}
\newcounter{lesson_transformations_reflections}
\newcounter{lesson_transformations_scalings}
\newcounter{lesson_transformations_summary}
\newcounter{lesson_piecewise_functions}
\newcounter{lesson_functions_containing_absolute_values}
\newcounter{lesson_absolute_as_piecewise}
\newcounter{lesson_polynomials_introduction}
\newcounter{lesson_sign_diagrams_polynomials}
\newcounter{lesson_factoring_quadratic_type}
\newcounter{lesson_factoring_summary}
\newcounter{lesson_polynomial_division}
\newcounter{lesson_synthetic_division}
\newcounter{lesson_end_behavior_polynomials}
\newcounter{lesson_local_behavior_polynomials}
\newcounter{lesson_rational_root_theorem}
\newcounter{lesson_polynomials_graphing_summary}
\newcounter{lesson_polynomial_inequalities}
\newcounter{lesson_rationals_introduction_and_terminology}
\newcounter{lesson_sign_diagrams_rationals}
\newcounter{lesson_horizontal_asymptotes}
\newcounter{lesson_slant_and_curvilinear_asymptotes}
\newcounter{lesson_vertical_asymptotes}
\newcounter{lesson_holes}
\newcounter{lesson_rationals_graphing_summary}

\setcounter{lesson_solving_linear_equations}{1}
\setcounter{lesson_equations_containing_absolute_values}{2}
\setcounter{lesson_graphing_lines}{3}
\setcounter{lesson_two_forms_of_a_linear_equation}{4}
\setcounter{lesson_parallel_and_perpendicular_lines}{5}
\setcounter{lesson_linear_inequalities}{6}
\setcounter{lesson_compound_inequalities}{7}
\setcounter{lesson_inequalities_containing_absolute_values}{8}
\setcounter{lesson_graphing_systems}{9}
\setcounter{lesson_substitution}{10}
\setcounter{lesson_elimination}{11}
\setcounter{lesson_quadratics_introduction}{16}
\setcounter{lesson_factoring_GCF}{17}
\setcounter{lesson_factoring_grouping}{18}
\setcounter{lesson_factoring_trinomials_a_is_1}{19}
\setcounter{lesson_factoring_trinomials_a_neq_1}{20}
\setcounter{lesson_solving_by_factoring}{21}
\setcounter{lesson_square_roots}{22}
\setcounter{lesson_i_and_complex_numbers}{23}
\setcounter{lesson_vertex_form_and_graphing}{24}
\setcounter{lesson_solve_by_square_roots}{25}
\setcounter{lesson_extracting_square_roots}{26}
\setcounter{lesson_the_discriminant}{27}
\setcounter{lesson_the_quadratic_formula}{28}
\setcounter{lesson_quadratic_inequalities}{29}
\setcounter{lesson_functions_and_relations}{12}
\setcounter{lesson_evaluating_functions}{13}
\setcounter{lesson_finding_domain_and_range_graphically}{14}
\setcounter{lesson_fundamental_functions}{15}
\setcounter{lesson_finding_domain_algebraically}{30}
\setcounter{lesson_solving_functions}{31}
\setcounter{lesson_function_arithmetic}{32}
\setcounter{lesson_composite_functions}{33}
\setcounter{lesson_inverse_functions_definition_and_HLT}{34}
\setcounter{lesson_finding_an_inverse_function}{35}
\setcounter{lesson_transformations_translations}{36}
\setcounter{lesson_transformations_reflections}{37}
\setcounter{lesson_transformations_scalings}{38}
\setcounter{lesson_transformations_summary}{39}
\setcounter{lesson_piecewise_functions}{40}
\setcounter{lesson_functions_containing_absolute_values}{41}
\setcounter{lesson_absolute_as_piecewise}{42}
\setcounter{lesson_polynomials_introduction}{43}
\setcounter{lesson_sign_diagrams_polynomials}{44}
\setcounter{lesson_factoring_quadratic_type}{46}
\setcounter{lesson_factoring_summary}{45}
\setcounter{lesson_polynomial_division}{47}
\setcounter{lesson_synthetic_division}{48}
\setcounter{lesson_end_behavior_polynomials}{49}
\setcounter{lesson_local_behavior_polynomials}{50}
\setcounter{lesson_rational_root_theorem}{51}
\setcounter{lesson_polynomials_graphing_summary}{52}
\setcounter{lesson_polynomial_inequalities}{53}
\setcounter{lesson_rationals_introduction_and_terminology}{54}
\setcounter{lesson_sign_diagrams_rationals}{55}
\setcounter{lesson_horizontal_asymptotes}{56}
\setcounter{lesson_slant_and_curvilinear_asymptotes}{57}
\setcounter{lesson_vertical_asymptotes}{58}
\setcounter{lesson_holes}{59}
\setcounter{lesson_rationals_graphing_summary}{60}

\begin{document}
{\bf \large Lesson \arabic{lesson_finding_domain_and_range_graphically}: Finding Domain and Range from a Graph}\phantomsection\label{les:finding_domain_and_range_graphically}
%\\ CC attribute: \href{http://www.wallace.ccfaculty.org/book/book.html}{\it{Beginning and Intermediate Algebra}} by T. Wallace. 
\\ CC attribute: \href{http://www.stitz-zeager.com}{\it{College Algebra}} by C. Stitz and J. Zeager. 
\hfill \doclicenseImage[imagewidth=5em]\\
\par
{\bf Objective:} Find the domain and range of a function from its graph.\\
\par
{\bf Students will be able to:}
\begin{itemize}
	\item Find domain graphically.
	\item Find range graphically.
\end{itemize}
{\bf Prerequisite Knowledge:}
\begin{itemize}
	\item Interval notation
	\item The definition of a function.
	\item Graphing a function on the coordinate plane.
\end{itemize}
\hrulefill

{\bf Lesson:}\\
{\bf I - Motivating Example(s):}\\
\ \par
{\bf Example:} Find the domain and range of the function $f$ whose graph is given below.

\begin{center}
	\begin{tikzpicture}[xscale=0.75,yscale=0.75]
		\draw[step=1.0,gray,very thin,dotted] (-3,-2) grid (3,5);
		\draw [<->](-3,0) -- coordinate (x axis mid) (3,0) node[below right] {$x$};
		\draw [<->](0,-2) -- coordinate (y axis mid) (0,5) node[above right] {$y$};
		\foreach \x in {1,...,2} \draw (\x,2pt) -- (\x,-2pt) node[anchor=north] {\scriptsize \x};
		\foreach \x in {-1,...,-2} \draw (\x,2pt) -- (\x,-2pt) node[anchor=north] {\scriptsize \x};
		\foreach \y in {1,...,4} \draw (2pt,\y) -- (-2pt,\y) node[anchor=east] {\scriptsize \y}; 
		\foreach \y in {-1} \draw (2pt,\y) -- (-2pt,\y) node[anchor=east] {\scriptsize \y}; 
		\draw [<->] plot [domain=-2.35:1, samples=100] (\x,{-1*(\x)^2+4});
		\draw[fill, white] (1,3) circle (0.08) node[above] {};
		\draw[line width=0.15mm] (1,3) circle (0.08) node[above] {};
		\draw (0,-2.75) node {\scriptsize The graph of $f$};
	\end{tikzpicture}
\end{center}

To determine the domain and range of $f$, we need to determine which $x$ and $y$-values respectively occur as coordinates of points on the given graph.\\
\ \par
To find the domain, it will be helpful to imagine collapsing the curve onto the $x$-axis and determining the portion of the $x$-axis that gets covered.  This is often described as {\bf projecting} the curve onto the $x$-axis.\\
\ \par
Before we project, we need to pay attention to two subtle notations on the graph:  the arrowhead on the lower left corner of the graph indicates that the graph continues to curve downwards to the left forever; and the open circle at $(1,3)$ indicates that the point $(1,3)$ is {\it not} on the graph, but all the points on the curve leading up to $(1,3)$ are on the graph.

\begin{center}
\begin{multicols}{2}
	\begin{tikzpicture}[xscale=0.7,yscale=0.7]
		\draw[step=1.0,gray,very thin,dotted] (-4,-3) grid (4,5);
		\draw [<->](-4,0) -- coordinate (x axis mid) (4,0) node[below right] {$x$};
		\draw [<->](0,-3) -- coordinate (y axis mid) (0,5) node[above right] {$y$};
		\foreach \x in {1,...,3} \draw (\x,2pt) -- (\x,-2pt) node[anchor=north] {\scriptsize \x};
		\foreach \x in {-1,...,-3} \draw (\x,2pt) -- (\x,-2pt) node[anchor=north] {\scriptsize \x};
		\foreach \y in {1,...,4} \draw (2pt,\y) -- (-2pt,\y) node[anchor=east] {\scriptsize \y}; 
		\foreach \y in {-1,-2} \draw (2pt,\y) -- (-2pt,\y) node[anchor=east] {\scriptsize \y}; 
		\draw [<->] plot [domain=-2.35:1, samples=100] (\x,{-1*(\x)^2+4});
		\draw[fill, white] (1,3) circle (0.08) node[above] {};
		\draw[line width=0.15mm] (1,3) circle (0.08) node[above] {};
		\draw [<-] plot [domain=0.5:2.25, samples=100] (2,{\x}) node[above] {\scriptsize project down};
		\draw [<-] plot [domain=-0.75:-2.25, samples=100] (-3,{\x}) node[below] {\scriptsize project up};
	\end{tikzpicture}

\columnbreak

	\begin{tikzpicture}[xscale=0.7,yscale=0.7]
		\draw[step=1.0,gray,very thin,dotted] (-4,-3) grid (4,5);
		\draw [<->](-4,0) -- coordinate (x axis mid) (4,0) node[below right] {$x$};
		\draw [<->](0,-3) -- coordinate (y axis mid) (0,5) node[above right] {$y$};
		\foreach \x in {1,...,3} \draw (\x,2pt) -- (\x,-2pt) node[anchor=north] {\scriptsize \x};
		\foreach \x in {-1,...,-3} \draw (\x,2pt) -- (\x,-2pt) node[anchor=north] {\scriptsize \x};
		\foreach \y in {1,...,4} \draw (2pt,\y) -- (-2pt,\y) node[anchor=east] {\scriptsize \y}; 
		\foreach \y in {-1,-2} \draw (2pt,\y) -- (-2pt,\y) node[anchor=east] {\scriptsize \y}; 
		\draw [<->] plot [domain=-2.35:1, samples=100] (\x,{-1*(\x)^2+4});
		\draw[fill, white] (1,3) circle (0.08) node[above] {};
		\draw[line width=0.15mm] (1,3) circle (0.08) node[above] {};
		\draw [<-, line width=.75mm] plot [domain=-4:1, samples=100] (\x,{0});
		\draw (0.95,0) node {\huge )};
	\end{tikzpicture}
\end{multicols}
\end{center}

We see from the figure that if we project the graph of $f$ to the $x$-axis, we get all real numbers less than $1$.  Using interval notation, we write the domain of $f$ as $(-\infty, 1)$.\\
\ \par
To determine the range of $f$, we use a similar method, projecting the curve onto the $y$-axis as follows.

\begin{center}
\begin{multicols}{2}
	\begin{tikzpicture}[xscale=0.7,yscale=0.7]
		\draw[step=1.0,gray,very thin,dotted] (-4,-3) grid (4,5);
		\draw [<->](-4,0) -- coordinate (x axis mid) (4,0) node[below right] {$x$};
		\draw [<->](0,-3) -- coordinate (y axis mid) (0,5) node[above right] {$y$};
		\foreach \x in {1,...,3} \draw (\x,2pt) -- (\x,-2pt) node[anchor=north] {\scriptsize \x};
		\foreach \x in {-1,...,-3} \draw (\x,2pt) -- (\x,-2pt) node[anchor=north] {\scriptsize \x};
		\foreach \y in {1,...,4} \draw (2pt,\y) -- (-2pt,\y) node[anchor=east] {\scriptsize \y}; 
		\foreach \y in {-1,-2} \draw (2pt,\y) -- (-2pt,\y) node[anchor=east] {\scriptsize \y}; 
		\draw [<->] plot [domain=-2.35:1, samples=100] (\x,{-1*(\x)^2+4});
		\draw[fill, white] (1,3) circle (0.08) node[above] {};
		\draw[line width=0.15mm] (1,3) circle (0.08) node[above] {};
		\draw [->] plot [domain=2.5:1, samples=100] (\x,{2});
		\draw (1.75,1.5) node {\scriptsize project left};
		\draw [->] plot [domain=-3.5:-2, samples=100] (\x,{2.5});
		\draw (-2.75,2) node {\scriptsize project right};
	\end{tikzpicture}

\columnbreak

	\begin{tikzpicture}[xscale=0.7,yscale=0.7]
		\draw[step=1.0,gray,very thin,dotted] (-4,-3) grid (4,5);
		\draw [<->](-4,0) -- coordinate (x axis mid) (4,0) node[below right] {$x$};
		\draw [<->](0,-3) -- coordinate (y axis mid) (0,5) node[above right] {$y$};
		\foreach \x in {1,...,3} \draw (\x,2pt) -- (\x,-2pt) node[anchor=north] {\scriptsize \x};
		\foreach \x in {-1,...,-3} \draw (\x,2pt) -- (\x,-2pt) node[anchor=north] {\scriptsize \x};
		\foreach \y in {1,...,4} \draw (2pt,\y) -- (-2pt,\y) node[anchor=east] {\scriptsize \y}; 
		\foreach \y in {-1,-2} \draw (2pt,\y) -- (-2pt,\y) node[anchor=east] {\scriptsize \y}; 
		\draw [<->] plot [domain=-2.35:1, samples=100] (\x,{-1*(\x)^2+4});
		\draw[fill, white] (1,3) circle (0.08) node[above] {};
		\draw[line width=0.15mm] (1,3) circle (0.08) node[above] {};
		\draw [->, line width=.75mm] plot [domain=4:-3, samples=100] (0,{\x});
		\node[rotate=90] at (0,4) {\huge ]};
	\end{tikzpicture}
\end{multicols}
\end{center}

Note that even though there is an open circle at $(1,3)$, we still include the $y$ value of $3$ in our range, since the point $(-1,3)$ is on the graph of $f$.  We also include $y=4$ in our answer, since the point $(0,4)$ is also on our graph.  Consequently, the range of $f$ is all real numbers less than or equal to $4$, or $(-\infty, 4]$.
\newpage

{\bf II - Demo/Discussion Problems:}\\
\ \par
For each of the following graphs, identify the corresponding domain and range.  Express your answers using interval notation.
\begin{center}
\begin{multicols}{3}
\begin{enumerate}
\item[1.]
	\begin{tikzpicture}[xscale=0.4,yscale=0.4]
		\draw[step=1.0,gray,very thin,dotted] (-5,-5) grid (5,5);
		\draw [<->](-5,0) -- coordinate (x axis mid) (5,0) node[below right] {$x$};
		\draw [<->](0,-5) -- coordinate (y axis mid) (0,5) node[above right] {$y$};
		\foreach \x in {1,...,4} \draw (\x,2pt) -- (\x,-2pt) node[anchor=north] {\tiny \x};
		\foreach \x in {-1,...,-4} \draw (\x,2pt) -- (\x,-2pt) node[anchor=south] {\tiny \x};
		\foreach \y in {1,...,4} \draw (2pt,\y) -- (-2pt,\y) node[anchor=west] {\tiny \y}; 
		\foreach \y in {-1,...,-4} \draw (2pt,\y) -- (-2pt,\y) node[anchor=east] {\tiny \y}; 
		\draw [<->] plot [domain=-1.85:1.85, samples=100] (\x,{(\x)^2+1});
	\end{tikzpicture}
\columnbreak
\item[2.]
	\begin{tikzpicture}[xscale=0.4,yscale=0.4]
		\draw[step=1.0,gray,very thin,dotted] (-5,-5) grid (5,5);
		\draw [<->](-5,0) -- coordinate (x axis mid) (5,0) node[below right] {$x$};
		\draw [<->](0,-5) -- coordinate (y axis mid) (0,5) node[above right] {$y$};
		\foreach \x in {1,...,4} \draw (\x,2pt) -- (\x,-2pt) node[anchor=north] {\tiny \x};
		\foreach \x in {-1,...,-4} \draw (\x,2pt) -- (\x,-2pt) node[anchor=south] {\tiny \x};
		\foreach \y in {1,...,4} \draw (2pt,\y) -- (-2pt,\y) node[anchor=west] {\tiny \y}; 
		\foreach \y in {-1,...,-4} \draw (2pt,\y) -- (-2pt,\y) node[anchor=east] {\tiny \y}; 
		\draw [<->] plot [domain=-2.5:1.75, samples=100] (\x,{(\x+2)*(\x)*(\x-1)});
	\end{tikzpicture}
\columnbreak
\item[3.]
	\begin{tikzpicture}[xscale=0.4,yscale=0.4]
		\draw[step=1.0,gray,very thin,dotted] (-5,-5) grid (5,5);
		\draw [<->](-5,0) -- coordinate (x axis mid) (5,0) node[below right] {$x$};
		\draw [<->](0,-5) -- coordinate (y axis mid) (0,5) node[above right] {$y$};
		\foreach \x in {1,...,4} \draw (\x,2pt) -- (\x,-2pt) node[anchor=north] {\tiny \x};
		\foreach \x in {-1,...,-4} \draw (\x,2pt) -- (\x,-2pt) node[anchor=south] {\tiny \x};
		\foreach \y in {1,...,4} \draw (2pt,\y) -- (-2pt,\y) node[anchor=west] {\tiny \y}; 
		\foreach \y in {-1,...,-4} \draw (2pt,\y) -- (-2pt,\y) node[anchor=east] {\tiny \y}; 
		\draw [->] plot [domain=2:5, samples=100] (\x,{2*(\x-2)^0.5});
		\draw[fill] (2,0) circle (0.12) node[above] {};
	\end{tikzpicture}
\end{enumerate}
\end{multicols}
\end{center}

\begin{center}
\begin{multicols}{3}
\begin{enumerate}
\item[4.]
	\begin{tikzpicture}[xscale=0.4,yscale=0.4]
		\draw[step=1.0,gray,very thin,dotted] (-5,-5) grid (5,5);
		\draw [<->](-5,0) -- coordinate (x axis mid) (5,0) node[below right] {$x$};
		\draw [<->](0,-5) -- coordinate (y axis mid) (0,5) node[above right] {$y$};
		\foreach \x in {1,...,4} \draw (\x,2pt) -- (\x,-2pt) node[anchor=north] {\tiny \x};
		\foreach \x in {-1,...,-4} \draw (\x,2pt) -- (\x,-2pt) node[anchor=south] {\tiny \x};
		\foreach \y in {1,...,4} \draw (2pt,\y) -- (-2pt,\y) node[anchor=west] {\tiny \y}; 
		\foreach \y in {-1,...,-4} \draw (2pt,\y) -- (-2pt,\y) node[anchor=east] {\tiny \y}; 
		\draw [<->] plot [domain=-4.9:4.9, samples=100] (\x,{4/((\x)^2+1)});
	\end{tikzpicture}
\columnbreak
\item[5.]
	\begin{tikzpicture}[xscale=0.4,yscale=0.4]
		\draw[step=1.0,gray,very thin,dotted] (-5,-5) grid (5,5);
		\draw [<->](-5,0) -- coordinate (x axis mid) (5,0) node[below right] {$x$};
		\draw [<->](0,-5) -- coordinate (y axis mid) (0,5) node[above right] {$y$};
		\foreach \x in {1,...,4} \draw (\x,2pt) -- (\x,-2pt) node[anchor=north] {\tiny \x};
		\foreach \x in {-1,...,-4} \draw (\x,2pt) -- (\x,-2pt) node[anchor=south] {\tiny \x};
		\foreach \y in {1,...,4} \draw (2pt,\y) -- (-2pt,\y) node[anchor=west] {\tiny \y}; 
		\foreach \y in {-1,...,-4} \draw (2pt,\y) -- (-2pt,\y) node[anchor=east] {\tiny \y}; 
		\draw [-] plot [domain=-5:4, samples=100] (\x,{0.0502*(\x)^3 - 0.0344*(\x)^2 - 0.2010*\x + 2.138});
		\draw[fill] (-5,-4) circle (0.12) node[above] {};
		\draw[fill, white] (4,4) circle (0.12) node[above] {};
		\draw[line width=0.15mm] (4,4) circle (0.12) node[above] {};
	\end{tikzpicture}
\columnbreak
\item[6.]
	\begin{tikzpicture}[xscale=0.4,yscale=0.4]
		\draw[step=1.0,gray,very thin,dotted] (-5,-5) grid (5,5);
		\draw [<->](-5,0) -- coordinate (x axis mid) (5,0) node[below right] {$x$};
		\draw [<->](0,-5) -- coordinate (y axis mid) (0,5) node[above right] {$y$};
		\foreach \x in {1,...,4} \draw (\x,2pt) -- (\x,-2pt) node[anchor=north] {\tiny \x};
		\foreach \x in {-1,...,-4} \draw (\x,2pt) -- (\x,-2pt) node[anchor=south] {\tiny \x};
		\foreach \y in {1,...,4} \draw (2pt,\y) -- (-2pt,\y) node[anchor=west] {\tiny \y}; 
		\foreach \y in {-1,...,-4} \draw (2pt,\y) -- (-2pt,\y) node[anchor=east] {\tiny \y}; 
		\draw [<->] plot [domain=-5:5, samples=100] (\x,{2});
	\end{tikzpicture}
\end{enumerate}
\end{multicols}
\end{center}
\begin{center}
\begin{multicols}{3}
\begin{enumerate}
\item[7.]
	\begin{tikzpicture}[xscale=0.4,yscale=0.4]
		\draw[step=1.0,gray,very thin,dotted] (-5,-5) grid (5,5);
		\draw [<->](-5,0) -- coordinate (x axis mid) (5,0) node[below right] {$x$};
		\draw [<->](0,-5) -- coordinate (y axis mid) (0,5) node[above right] {$y$};
		\foreach \x in {1,...,4} \draw (\x,2pt) -- (\x,-2pt) node[anchor=north] {\tiny \x};
		\foreach \x in {-1,...,-4} \draw (\x,2pt) -- (\x,-2pt) node[anchor=south] {\tiny \x};
		\foreach \y in {1,...,4} \draw (2pt,\y) -- (-2pt,\y) node[anchor=west] {\tiny \y}; 
		\foreach \y in {-1,...,-4} \draw (2pt,\y) -- (-2pt,\y) node[anchor=east] {\tiny \y}; 
		\draw [<->] plot [domain=-1.7:1.7, samples=100] (\x,{(-4/((\x)^2-4))+1});
		\draw [<->] plot [domain=2.2:5, samples=100] (\x,{(-4/((\x)^2-4))+1});
		\draw [<->] plot [domain=-5:-2.2, samples=100] (\x,{(-4/((\x)^2-4))+1});
		\draw [<->, dashed, line width=0.2mm] plot [domain=-5:5, samples=100] (2,{\x});
		\draw [<->, dashed, line width=0.2mm] plot [domain=-5:5, samples=100] (-2,{\x});
		\draw [<->, dashed, line width=0.2mm] plot [domain=-5:5, samples=100] (\x,{1});
	\end{tikzpicture}
\columnbreak
\item[8.]
	\begin{tikzpicture}[xscale=0.4,yscale=0.4]
		\draw[step=1.0,gray,very thin,dotted] (-5,-5) grid (5,5);
		\draw [<->](-5,0) -- coordinate (x axis mid) (5,0) node[below right] {$x$};
		\draw [<->](0,-5) -- coordinate (y axis mid) (0,5) node[above right] {$y$};
		\foreach \x in {1,...,4} \draw (\x,2pt) -- (\x,-2pt) node[anchor=north] {\tiny \x};
		\foreach \x in {-1,...,-4} \draw (\x,2pt) -- (\x,-2pt) node[anchor=south] {\tiny \x};
		\foreach \y in {1,...,4} \draw (2pt,\y) -- (-2pt,\y) node[anchor=west] {\tiny \y}; 
		\foreach \y in {-1,...,-4} \draw (2pt,\y) -- (-2pt,\y) node[anchor=east] {\tiny \y}; 
		\draw [<->] plot [domain=-3.05:3.05, samples=100] (\x,{-0.25*((\x)^2-6.25)*((\x)^2-1)+1.25});
	\end{tikzpicture}
\columnbreak
\item[9.]
	\begin{tikzpicture}[xscale=0.4,yscale=0.4]
		\draw[step=1.0,gray,very thin,dotted] (-5,-5) grid (5,5);
		\draw [<->](-5,0) -- coordinate (x axis mid) (5,0) node[below right] {$x$};
		\draw [<->](0,-5) -- coordinate (y axis mid) (0,5) node[above right] {$y$};
		\foreach \x in {1,...,4} \draw (\x,2pt) -- (\x,-2pt) node[anchor=north] {\tiny \x};
		\foreach \x in {-1,...,-4} \draw (\x,2pt) -- (\x,-2pt) node[anchor=south] {\tiny \x};
		\foreach \y in {1,...,4} \draw (2pt,\y) -- (-2pt,\y) node[anchor=west] {\tiny \y}; 
		\foreach \y in {-1,...,-4} \draw (2pt,\y) -- (-2pt,\y) node[anchor=east] {\tiny \y}; 
		\draw [->] plot [domain=2:5, samples=100] (\x,{-2*(\x-2)^0.5});
		\draw[fill, white] (2,0) circle (0.12) node[above] {};
		\draw[line width=0.15mm] (2,0) circle (0.12) node[above] {};
		\draw [->] plot [domain=-2:-5, samples=100] (\x,{-2*(-\x-2)^0.5});
		\draw[fill, white] (-2,0) circle (0.12) node[above] {};
		\draw[line width=0.15mm] (-2,0) circle (0.12) node[above] {};
	\end{tikzpicture}
\end{enumerate}
\end{multicols}
\end{center}
\newpage
{\bf III - Practice Problems:}\\
\ \par
For each of the following graphs, identify the corresponding domain and range.  Express your answers using interval notation.
\begin{center}
\begin{multicols}{3}
\begin{enumerate}
\item[1.]
	\begin{tikzpicture}[xscale=0.4,yscale=0.4]
		\draw[step=1.0,gray,very thin,dotted] (-5,-5) grid (5,5);
		\draw [<->](-5,0) -- coordinate (x axis mid) (5,0) node[below right] {$x$};
		\draw [<->](0,-5) -- coordinate (y axis mid) (0,5) node[above right] {$y$};
		\foreach \x in {1,...,4} \draw (\x,2pt) -- (\x,-2pt) node[anchor=north] {\tiny \x};
		\foreach \x in {-1,...,-4} \draw (\x,2pt) -- (\x,-2pt) node[anchor=south] {\tiny \x};
		\foreach \y in {1,...,4} \draw (2pt,\y) -- (-2pt,\y) node[anchor=west] {\tiny \y}; 
		\foreach \y in {-1,...,-4} \draw (2pt,\y) -- (-2pt,\y) node[anchor=east] {\tiny \y}; 
		\draw [<->] plot [domain=-3:3, samples=100] (\x,{-1*(\x)^2+4});
	\end{tikzpicture}
\columnbreak
\item[2.]
	\begin{tikzpicture}[xscale=0.4,yscale=0.4]
		\draw[step=1.0,gray,very thin,dotted] (-5,-5) grid (5,5);
		\draw [<->](-5,0) -- coordinate (x axis mid) (5,0) node[below right] {$x$};
		\draw [<->](0,-5) -- coordinate (y axis mid) (0,5) node[above right] {$y$};
		\foreach \x in {1,...,4} \draw (\x,2pt) -- (\x,-2pt) node[anchor=north] {\tiny \x};
		\foreach \x in {-1,...,-4} \draw (\x,2pt) -- (\x,-2pt) node[anchor=south] {\tiny \x};
		\foreach \y in {1,...,4} \draw (2pt,\y) -- (-2pt,\y) node[anchor=west] {\tiny \y}; 
		\foreach \y in {-1,...,-4} \draw (2pt,\y) -- (-2pt,\y) node[anchor=east] {\tiny \y}; 
		\draw [<->] plot [domain=-5:-0.25, samples=100] (\x,{1/(\x)});
		\draw [<->] plot [domain=5:0.25, samples=100] (\x,{-1/(\x)});
	\end{tikzpicture}
\columnbreak
\item[3.]
	\begin{tikzpicture}[xscale=0.4,yscale=0.4]
		\draw[step=1.0,gray,very thin,dotted] (-5,-5) grid (5,5);
		\draw [<->](-5,0) -- coordinate (x axis mid) (5,0) node[below right] {$x$};
		\draw [<->](0,-5) -- coordinate (y axis mid) (0,5) node[above right] {$y$};
		\foreach \x in {1,...,4} \draw (\x,2pt) -- (\x,-2pt) node[anchor=north] {\tiny \x};
		\foreach \x in {-1,...,-4} \draw (\x,2pt) -- (\x,-2pt) node[anchor=south] {\tiny \x};
		\foreach \y in {1,...,4} \draw (2pt,\y) -- (-2pt,\y) node[anchor=west] {\tiny \y}; 
		\foreach \y in {-1,...,-4} \draw (2pt,\y) -- (-2pt,\y) node[anchor=east] {\tiny \y}; 
		\draw [->] plot [domain=-2:5, samples=100] (\x,{-3*(\x+2)^0.5+3});
		\draw[fill] (-2,3) circle (0.12) node[above] {};
		\draw (-2,3) circle (0.12) node[above] {};
	\end{tikzpicture}
\end{enumerate}
\end{multicols}
\end{center}

\begin{center}
\begin{multicols}{3}
\begin{enumerate}
\item[4.]
	\begin{tikzpicture}[xscale=0.4,yscale=0.4]
		\draw[step=1.0,gray,very thin,dotted] (-5,-5) grid (5,5);
		\draw [<->](-5,0) -- coordinate (x axis mid) (5,0) node[below right] {$x$};
		\draw [<->](0,-5) -- coordinate (y axis mid) (0,5) node[above right] {$y$};
		\foreach \x in {1,...,4} \draw (\x,2pt) -- (\x,-2pt) node[anchor=north] {\tiny \x};
		\foreach \x in {-1,...,-4} \draw (\x,2pt) -- (\x,-2pt) node[anchor=south] {\tiny \x};
		\foreach \y in {1,...,4} \draw (2pt,\y) -- (-2pt,\y) node[anchor=west] {\tiny \y}; 
		\foreach \y in {-1,...,-4} \draw (2pt,\y) -- (-2pt,\y) node[anchor=east] {\tiny \y}; 
		\draw [->] plot [domain=1:2.7, samples=100] (\x,{-5*(\x-1)+4});
		\draw [<-] plot [domain=-4:1, samples=100] (\x,{1.667*(\x-1)+4});
	\end{tikzpicture}
\columnbreak
\item[5.]
	\begin{tikzpicture}[xscale=0.4,yscale=0.4]
		\draw[step=1.0,gray,very thin,dotted] (-5,-5) grid (5,5);
		\draw [<->](-5,0) -- coordinate (x axis mid) (5,0) node[below right] {$x$};
		\draw [<->](0,-5) -- coordinate (y axis mid) (0,5) node[above right] {$y$};
		\foreach \x in {1,...,4} \draw (\x,2pt) -- (\x,-2pt) node[anchor=north] {\tiny \x};
		\foreach \x in {-1,...,-4} \draw (\x,2pt) -- (\x,-2pt) node[anchor=south] {\tiny \x};
		\foreach \y in {1,...,4} \draw (2pt,\y) -- (-2pt,\y) node[anchor=west] {\tiny \y}; 
		\foreach \y in {-1,...,-4} \draw (2pt,\y) -- (-2pt,\y) node[anchor=east] {\tiny \y}; 
		\draw [->] plot [domain=1:5, samples=100] (\x,{2});
		\draw [<-] plot [domain=-4:0, samples=100] (\x,{1.5*(\x)+1});
		\draw[fill,white] (1,2) circle (0.12) node[above] {};
		\draw[line width=0.15mm] (1,2) circle (0.12) node[above] {};
		\draw[fill] (0,1) circle (0.12) node[above] {};
	\end{tikzpicture}
\columnbreak
\item[6.]
	\begin{tikzpicture}[xscale=0.4,yscale=0.4]
		\draw[step=1.0,gray,very thin,dotted] (-5,-5) grid (5,5);
		\draw [<->](-5,0) -- coordinate (x axis mid) (5,0) node[below right] {$x$};
		\draw [<->](0,-5) -- coordinate (y axis mid) (0,5) node[above right] {$y$};
		\foreach \x in {1,...,4} \draw (\x,2pt) -- (\x,-2pt) node[anchor=north] {\tiny \x};
		\foreach \x in {-1,...,-4} \draw (\x,2pt) -- (\x,-2pt) node[anchor=south] {\tiny \x};
		\foreach \y in {1,...,4} \draw (2pt,\y) -- (-2pt,\y) node[anchor=west] {\tiny \y}; 
		\foreach \y in {-1,...,-4} \draw (2pt,\y) -- (-2pt,\y) node[anchor=east] {\tiny \y}; 
		\draw [-] plot [domain=-3:3, samples=100] (\x,{2*sin(60*\x)});
		\draw[fill] (3,0) circle (0.12) node[above] {};
		\draw[fill] (-3,0) circle (0.12) node[above] {};
	\end{tikzpicture}
\end{enumerate}
\end{multicols}
\end{center}
\begin{center}
\begin{multicols}{3}
\begin{enumerate}
\item[7.]
	\begin{tikzpicture}[xscale=0.4,yscale=0.4]
		\draw[step=1.0,gray,very thin,dotted] (-5,-5) grid (5,5);
		\draw [<->](-5,0) -- coordinate (x axis mid) (5,0) node[below right] {$x$};
		\draw [<->](0,-5) -- coordinate (y axis mid) (0,5) node[above right] {$y$};
		\foreach \x in {1,...,4} \draw (\x,2pt) -- (\x,-2pt) node[anchor=north] {\tiny \x};
		\foreach \x in {-1,...,-4} \draw (\x,2pt) -- (\x,-2pt) node[anchor=south] {\tiny \x};
		\foreach \y in {1,...,4} \draw (2pt,\y) -- (-2pt,\y) node[anchor=west] {\tiny \y}; 
		\foreach \y in {-1,...,-4} \draw (2pt,\y) -- (-2pt,\y) node[anchor=east] {\tiny \y}; 
		\draw [->] plot [domain=-2:3.33, samples=100] (\x,{0.8888*(\x-1)^2-3});
		\draw[fill] (-2,5) circle (0.12) node[above] {};
	\end{tikzpicture}
\columnbreak
\item[8.]
	\begin{tikzpicture}[xscale=0.4,yscale=0.4]
		\draw[step=1.0,gray,very thin,dotted] (-5,-5) grid (5,5);
		\draw [<->](-5,0) -- coordinate (x axis mid) (5,0) node[below right] {$x$};
		\draw [<->](0,-5) -- coordinate (y axis mid) (0,5) node[above right] {$y$};
		\foreach \x in {1,...,4} \draw (\x,2pt) -- (\x,-2pt) node[anchor=north] {\tiny \x};
		\foreach \x in {-1,...,-4} \draw (\x,2pt) -- (\x,-2pt) node[anchor=south] {\tiny \x};
		\foreach \y in {1,...,4} \draw (2pt,\y) -- (-2pt,\y) node[anchor=west] {\tiny \y}; 
		\foreach \y in {-1,...,-4} \draw (2pt,\y) -- (-2pt,\y) node[anchor=east] {\tiny \y}; 
		\draw [-] plot [domain=4:5, samples=100] (\x,{-2*(\x-6)});
		\draw [-] plot [domain=3:4, samples=100] (\x,{3*(\x-4)+4});
		\draw [-] plot [domain=0:3, samples=100] (\x,{-1*(\x)-1});
		\draw[fill,white] (3,1) circle (0.12) node[above] {};
		\draw[line width=0.15mm] (3,1) circle (0.12) node[above] {};
		\draw[fill,white] (3,-4) circle (0.12) node[above] {};
		\draw[line width=0.15mm] (3,-4) circle (0.12) node[above] {};
		\draw[fill] (0,-1) circle (0.12) node[above] {};
		\draw[fill] (4,4) circle (0.12) node[above] {};
		\draw[fill] (5,2) circle (0.12) node[above] {};
	\end{tikzpicture}
\columnbreak
\item[9.]
	\begin{tikzpicture}[xscale=0.4,yscale=0.4]
		\draw[step=1.0,gray,very thin,dotted] (-5,-5) grid (5,5);
		\draw [<->](-5,0) -- coordinate (x axis mid) (5,0) node[below right] {$x$};
		\draw [<->](0,-5) -- coordinate (y axis mid) (0,5) node[above right] {$y$};
		\foreach \x in {1,...,4} \draw (\x,2pt) -- (\x,-2pt) node[anchor=north] {\tiny \x};
		\foreach \x in {-1,...,-4} \draw (\x,2pt) -- (\x,-2pt) node[anchor=south] {\tiny \x};
		\foreach \y in {1,...,4} \draw (2pt,\y) -- (-2pt,\y) node[anchor=west] {\tiny \y}; 
		\foreach \y in {-1,...,-4} \draw (2pt,\y) -- (-2pt,\y) node[anchor=east] {\tiny \y}; 
		\draw [<->] plot [domain=1.4:5, samples=100] (\x,{(1/(\x-1))+2});
		\draw [-] plot [domain=-4:-2, samples=100] (\x,{(-2*(2+\x))/\x-2});
		\draw[fill] (-2,-2) circle (0.12) node[above] {};
		\draw [<->, dashed, line width=0.2mm] plot [domain=-5:5, samples=100] (\x,{2});
		\draw [<->, dashed, line width=0.2mm] plot [domain=-5:5, samples=100] ({1},\x);
		\draw[fill,white] (-4,-3) circle (0.12) node[above] {};
		\draw[] (-4,-3) circle (0.12) node[above] {};
	\end{tikzpicture}
\end{enumerate}
\end{multicols}
\end{center}
\newpage
\end{document}