\documentclass[12pt]{article}
\usepackage[top=1in,left=1in,bottom=1in,right=1in,headsep=2pt]{geometry}	
\usepackage{amssymb,amsmath,amsthm,amsfonts}
\usepackage{chapterfolder,docmute,setspace}
\usepackage{cancel,multicol,tikz,verbatim,framed,polynom,enumitem}
\usepackage[colorlinks, hyperindex, plainpages=false, linkcolor=blue, urlcolor=blue, pdfpagelabels]{hyperref}
% Use the cc-by-nc-sa license for any content linked with Stitz and Zeager's text.  Otherwise, use the cc-by-sa license.
\usepackage[type={CC},modifier={by-sa},version={4.0},]{doclicense}
%\usepackage[type={CC},modifier={by-nc-sa},version={4.0},]{doclicense}

\theoremstyle{definition}
\newtheorem{example}{Example}
\newcommand{\Desmos}{\href{https://www.desmos.com/}{Desmos}}
\setlength{\parindent}{0em}
\setlist{itemsep=0em}
\setlength{\parskip}{0.1em}
% This document is used for ordering of lessons.  If an instructor wishes to change the ordering of assessments, the following steps must be taken:

% 1) Reassign the appropriate numbers for each lesson in the \setcounter commands included in this file.
% 2) Rearrange the \include commands in the master file (the file with 'Course Pack' in the name) to accurately reflect the changes.  
% 3) Rearrange the \items in the measureable_outcomes file to accurately reflect the changes.  Be mindful of page breaks when moving items.
% 4) Re-build all affected files (master file, measureable_outcomes file, and any lesson whose numbering has changed).

%Note: The placement of each \newcounter and \setcounter command reflects the original/default ordering of topics (linears, systems, quadratics, functions, polynomials, rationals).

\newcounter{lesson_solving_linear_equations}
\newcounter{lesson_equations_containing_absolute_values}
\newcounter{lesson_graphing_lines}
\newcounter{lesson_two_forms_of_a_linear_equation}
\newcounter{lesson_parallel_and_perpendicular_lines}
\newcounter{lesson_linear_inequalities}
\newcounter{lesson_compound_inequalities}
\newcounter{lesson_inequalities_containing_absolute_values}
\newcounter{lesson_graphing_systems}
\newcounter{lesson_substitution}
\newcounter{lesson_elimination}
\newcounter{lesson_quadratics_introduction}
\newcounter{lesson_factoring_GCF}
\newcounter{lesson_factoring_grouping}
\newcounter{lesson_factoring_trinomials_a_is_1}
\newcounter{lesson_factoring_trinomials_a_neq_1}
\newcounter{lesson_solving_by_factoring}
\newcounter{lesson_square_roots}
\newcounter{lesson_i_and_complex_numbers}
\newcounter{lesson_vertex_form_and_graphing}
\newcounter{lesson_solve_by_square_roots}
\newcounter{lesson_extracting_square_roots}
\newcounter{lesson_the_discriminant}
\newcounter{lesson_the_quadratic_formula}
\newcounter{lesson_quadratic_inequalities}
\newcounter{lesson_functions_and_relations}
\newcounter{lesson_evaluating_functions}
\newcounter{lesson_finding_domain_and_range_graphically}
\newcounter{lesson_fundamental_functions}
\newcounter{lesson_finding_domain_algebraically}
\newcounter{lesson_solving_functions}
\newcounter{lesson_function_arithmetic}
\newcounter{lesson_composite_functions}
\newcounter{lesson_inverse_functions_definition_and_HLT}
\newcounter{lesson_finding_an_inverse_function}
\newcounter{lesson_transformations_translations}
\newcounter{lesson_transformations_reflections}
\newcounter{lesson_transformations_scalings}
\newcounter{lesson_transformations_summary}
\newcounter{lesson_piecewise_functions}
\newcounter{lesson_functions_containing_absolute_values}
\newcounter{lesson_absolute_as_piecewise}
\newcounter{lesson_polynomials_introduction}
\newcounter{lesson_sign_diagrams_polynomials}
\newcounter{lesson_factoring_quadratic_type}
\newcounter{lesson_factoring_summary}
\newcounter{lesson_polynomial_division}
\newcounter{lesson_synthetic_division}
\newcounter{lesson_end_behavior_polynomials}
\newcounter{lesson_local_behavior_polynomials}
\newcounter{lesson_rational_root_theorem}
\newcounter{lesson_polynomials_graphing_summary}
\newcounter{lesson_polynomial_inequalities}
\newcounter{lesson_rationals_introduction_and_terminology}
\newcounter{lesson_sign_diagrams_rationals}
\newcounter{lesson_horizontal_asymptotes}
\newcounter{lesson_slant_and_curvilinear_asymptotes}
\newcounter{lesson_vertical_asymptotes}
\newcounter{lesson_holes}
\newcounter{lesson_rationals_graphing_summary}

\setcounter{lesson_solving_linear_equations}{1}
\setcounter{lesson_equations_containing_absolute_values}{2}
\setcounter{lesson_graphing_lines}{3}
\setcounter{lesson_two_forms_of_a_linear_equation}{4}
\setcounter{lesson_parallel_and_perpendicular_lines}{5}
\setcounter{lesson_linear_inequalities}{6}
\setcounter{lesson_compound_inequalities}{7}
\setcounter{lesson_inequalities_containing_absolute_values}{8}
\setcounter{lesson_graphing_systems}{9}
\setcounter{lesson_substitution}{10}
\setcounter{lesson_elimination}{11}
\setcounter{lesson_quadratics_introduction}{16}
\setcounter{lesson_factoring_GCF}{17}
\setcounter{lesson_factoring_grouping}{18}
\setcounter{lesson_factoring_trinomials_a_is_1}{19}
\setcounter{lesson_factoring_trinomials_a_neq_1}{20}
\setcounter{lesson_solving_by_factoring}{21}
\setcounter{lesson_square_roots}{22}
\setcounter{lesson_i_and_complex_numbers}{23}
\setcounter{lesson_vertex_form_and_graphing}{24}
\setcounter{lesson_solve_by_square_roots}{25}
\setcounter{lesson_extracting_square_roots}{26}
\setcounter{lesson_the_discriminant}{27}
\setcounter{lesson_the_quadratic_formula}{28}
\setcounter{lesson_quadratic_inequalities}{29}
\setcounter{lesson_functions_and_relations}{12}
\setcounter{lesson_evaluating_functions}{13}
\setcounter{lesson_finding_domain_and_range_graphically}{14}
\setcounter{lesson_fundamental_functions}{15}
\setcounter{lesson_finding_domain_algebraically}{30}
\setcounter{lesson_solving_functions}{31}
\setcounter{lesson_function_arithmetic}{32}
\setcounter{lesson_composite_functions}{33}
\setcounter{lesson_inverse_functions_definition_and_HLT}{34}
\setcounter{lesson_finding_an_inverse_function}{35}
\setcounter{lesson_transformations_translations}{36}
\setcounter{lesson_transformations_reflections}{37}
\setcounter{lesson_transformations_scalings}{38}
\setcounter{lesson_transformations_summary}{39}
\setcounter{lesson_piecewise_functions}{40}
\setcounter{lesson_functions_containing_absolute_values}{41}
\setcounter{lesson_absolute_as_piecewise}{42}
\setcounter{lesson_polynomials_introduction}{43}
\setcounter{lesson_sign_diagrams_polynomials}{44}
\setcounter{lesson_factoring_quadratic_type}{46}
\setcounter{lesson_factoring_summary}{45}
\setcounter{lesson_polynomial_division}{47}
\setcounter{lesson_synthetic_division}{48}
\setcounter{lesson_end_behavior_polynomials}{49}
\setcounter{lesson_local_behavior_polynomials}{50}
\setcounter{lesson_rational_root_theorem}{51}
\setcounter{lesson_polynomials_graphing_summary}{52}
\setcounter{lesson_polynomial_inequalities}{53}
\setcounter{lesson_rationals_introduction_and_terminology}{54}
\setcounter{lesson_sign_diagrams_rationals}{55}
\setcounter{lesson_horizontal_asymptotes}{56}
\setcounter{lesson_slant_and_curvilinear_asymptotes}{57}
\setcounter{lesson_vertical_asymptotes}{58}
\setcounter{lesson_holes}{59}
\setcounter{lesson_rationals_graphing_summary}{60}

\begin{document}
{\bf \large Lesson \arabic{lesson_compound_inequalities}: Compound Inequalities}\phantomsection\label{les:compound_inequalities}
\\ CC attribute: \href{http://www.wallace.ccfaculty.org/book/book.html}{\it{Beginning and Intermediate Algebra}} by T. Wallace. 
%\\ CC attribute: \href{http://www.stitz-zeager.com}{\it{College Algebra}} by C. Stitz and J. Zeager. 
\hfill \doclicenseImage[imagewidth=5em]\\
\par
{\bf Objective:} Solve, graph, and give interval notation to the solution
of a compound inequality.\\
\par
{\bf Students will be able to:}
\begin{itemize}
	\item Understand the distinctions between ``OR'' and ``AND'' inequalities.
	\item Recognize a double inequality as an ``AND'' inequality.
	\item Represent the solution to a compound inequality using the three notations (algebraically, graphically, and using interval notation).
\end{itemize}
{\bf Prerequisite Knowledge:}
\begin{itemize}
	\item Solving linear inequalities.
	\item Graphing on a number line.
	\item Interval notation, including unions and intersections.
\end{itemize}
\hrulefill

{\bf Lesson:}\\
\ \par
Several inequalities can be combined together to form what are called compound inequalities. There are three types of compound inequalities which we will investigate in this lesson.
	\begin{itemize}
		\item The first type of a compound inequality is an ``OR'' inequality. A solution for this type of inequality will produce a true statement for either one inequality OR the other inequality OR both.  Solutions to OR inequalities will often (but not always) consist of a union of two intervals, denoted by a $\cup$.
		\item The second type of compound inequality is an ``AND'' inequality. These inequalities require {\it both} statements to be true for a given solution.  Solutions to AND inequalities equal the intersection (or overlap) of two intervals.  Intersections are denoted by a $\cap$, but can be always be simplified.
		\item The third type of compound inequality is a special type of AND inequality.  When a variable (or expression containing the variable) is between two numbers, we can write this as a single mathematical sentence with three parts, such as $5 < x \leq 8$, to show $x$ is greater than 5 AND $x$ is less than or equal to 8. Since this sentence contains two inequalities (both always pointing in the same direction),  we refer to it as a {\it double inequality}.
	\end{itemize}
\newpage
{\bf I - Motivating Example(s):}\\
\ \par
{\bf Example:} Solve the inequality below, graph the solution, and provide the interval notation of your solution.
$$2 x - 5 > 3 \ \ \text{OR} \ \ 4 - x \geq 6$$
Here, we isolate the variable for each inequality.  In the first case, adding 5 and dividing by 2 produces $x>4$.\\
\ \par
For the second inequality, we subtract 4 and multiply (or divide) by a $-1$, which will change the direction of our inequality.  Here, we get $x\leq -2$.\\
$$x>4 \ \ \text{OR} \ \ x\leq-2$$
To express our answer graphically, we will sketch three separate intervals: one for the first inequality, one for the second inequality, and one for the union of the two, which will be our answer.
\begin{center}
\begin{tikzpicture}[xscale=0.7,yscale=0.7]
	\draw [<->](-6.25,2) -- coordinate (x axis mid) (6.25,2) node[right] {\ \ $x>4$};
	\draw [->,line width=0.8mm](4,2) -- coordinate (x axis mid) (6.25,2);
	\draw (4,2) node {{\bf (}};
	\draw [<->](-6.25,0) -- coordinate (y axis mid) (6.25,0) node[right] {\ \ $x\leq-2$};
	\draw [<-,line width=0.8mm](-6.25,0) -- coordinate (x axis mid) (-2,0);
	\draw (-2,0) node {{\bf ]}};
	\draw [<->](-6.25,-3) -- coordinate (y axis mid) (6.25,-3) node[right] {\ \ $x>4$ OR $x\leq-2$};
	\draw [->,line width=0.8mm](4,-3) -- coordinate (x axis mid) (6.25,-3);
	\draw [<-,line width=0.8mm](-6.25,-3) -- coordinate (x axis mid) (-2,-3);
	\draw (4,-3) node {{\bf (}};
	\draw (-2,-3) node {{\bf ]}};
	\draw [-](4,0.1) -- coordinate (y axis mid) (4,-0.1) node[below] {$4$};
	\draw [-](-2,0.1) -- coordinate (y axis mid) (-2,-0.1) node[below] {$-2$};
	\draw [-](4,2.1) -- coordinate (y axis mid) (4,1.9) node[below] {$4$};
	\draw [-](-2,2.1) -- coordinate (y axis mid) (-2,1.9) node[below] {$-2$};
	\draw [-](4,-2.9) -- coordinate (y axis mid) (4,-3.1) node[below] {$4$};
	\draw [-](-2,-2.9) -- coordinate (y axis mid) (-2,-3.1) node[below] {$-2$};
	\draw (0,-1.5) node {$\Downarrow$};
\end{tikzpicture}
\end{center}
Since our graphical answer includes two pieces, we use a union in our interval notation.
$$(-\infty,-2]\cup(4,\infty)$$
{\bf Example:} Solve the inequality below, graph the solution, and provide the interval notation of your solution.
$$2x+8\geq 5x-7 \ \ \text{AND} \ \ 5x-3>3x+1$$
Solving each inequality separately gives us the following.
\begin{multicols}{2}

\begin{eqnarray*}
2x+8 & \geq & 5x-7\\
2x & \geq & 5x-15\\
-3x & \geq & -15\\
x & \leq & 5
\end{eqnarray*}

\columnbreak
\begin{eqnarray*}
5x-3 & > & 3x+1\\
5x & > & 3x+4\\
2x & > & 4\\
x & > & 2
\end{eqnarray*}
\end{multicols}
Next, we graph the two inequalities separately, and take their intersection (overlap) for our final answer.
\begin{center}
\begin{tikzpicture}[xscale=0.7,yscale=0.7]
	\draw [<->](-6.25,2) -- coordinate (x axis mid) (6.25,2) node[right] {\ \ $x\leq 5$};
	\draw [<-,line width=0.8mm](-6.25,2) -- coordinate (x axis mid) (5,2);
	\draw (5,2) node {{\bf ]}};
	\draw [<->](-6.25,0) -- coordinate (y axis mid) (6.25,0) node[right] {\ \ $x>2$};
	\draw [->,line width=0.8mm](2,0) -- coordinate (x axis mid) (6.25,0);
	\draw (2,0) node {{\bf (}};
	\draw [<->](-6.25,-3) -- coordinate (y axis mid) (6.25,-3) node[right] {\ \ $x\leq 5$ AND $x>2$};
	\draw [-,line width=0.8mm](2,-3) -- coordinate (x axis mid) (5,-3);
	\draw (2,-3) node {{\bf (}};
	\draw (5,-3) node {{\bf ]}};
	\draw [-](5,0.1) -- coordinate (y axis mid) (5,-0.1) node[below] {$5$};
	\draw [-](2,0.1) -- coordinate (y axis mid) (2,-0.1) node[below] {$2$};
	\draw [-](5,2.1) -- coordinate (y axis mid) (5,1.9) node[below] {$5$};
	\draw [-](2,2.1) -- coordinate (y axis mid) (2,1.9) node[below] {$2$};
	\draw [-](5,-2.9) -- coordinate (y axis mid) (5,-3.1) node[below] {$5$};
	\draw [-](2,-2.9) -- coordinate (y axis mid) (2,-3.1) node[below] {$2$};
	\draw (0,-1.5) node {$\Downarrow$};
\end{tikzpicture}
\end{center}
Our answer, in interval notation is $(2,5]$.\\
\ \par
{\bf II - Demo/Discussion Problems:}\\
\ \par
Graph each compound inequality on a real number line and provide the corresponding interval notation.
\begin{enumerate}
	\item $x\leq-3 \ \ \text{OR} \ \ x<-1$
	\item $x\geq-3 \ \ \text{OR} \ \ x<-1$
	\item $x<-1 \ \ \text{AND} \ \ x<-2$
	\item $x>2 \ \ \text{AND} \ \ x<-1$
\end{enumerate}
Solve each of the given compound inequalities. Graph each solution on a real number line and provide the corresponding interval notation.
\begin{enumerate}
  \item[5.] $9 + n < 2  \ \ \text{OR} \ \  5 n > 40$
  \item[6.] $\dfrac{v}{8} > - 1  \ \ \text{AND} \ \  v - 2 < 1$
	\item[7.] $- 6 \leq - 4 x + 2 < 2$
	\item[8.] $- 4 < 8 - 3 m \leq 11$
\end{enumerate}
%\newpage
{\bf III - Practice Problems:}\\
\ \par
Solve each of the given compound inequalities. Graph each solution on a real number line and provide the corresponding interval notation.
\begin{enumerate}
\begin{multicols}{2}
  \item $\dfrac{n}{3} \leq - 3  \ \ \text{OR} \ \  - 5 n \leq - 10$
  \item $6 m \geq - 24  \ \ \text{OR} \ \  m - 7 < - 12$
  \item $x + 7 \geq 12  \ \ \text{OR} \ \  9 x < - 45$
  \item $10 r > 0  \ \ \text{OR} \ \  r - 5 < - 12$
  \item $x - 6 < - 13  \ \ \text{OR} \ \  6 x \leq - 60$
  \item $- 9 x < 63  \ \ \text{AND} \ \  \dfrac{x}{4} < 1$
  \item $- 8 + b < - 3  \ \ \text{AND} \ \  4 b < 20$
  \item $- 6 n \leq 12  \ \ \text{AND} \ \  \dfrac{n}{3} \leq 2$
  \item $a + 10 \geq 3  \ \ \text{AND} \ \  8 a \leq 48$
  \item $- 6 + v \geq 0  \ \ \text{AND} \ \  2 v > 4$
  \item $3 + 7 r > 59  \ \ \text{OR} \ \  - 6 r - 3 > 33$
  \item $- 6 - 8 x \geq - 6  \ \ \text{OR} \ \  2 + 10 x > 82$
\end{multicols}
\begin{multicols}{3}
  \item $3 \leq 9 + x \leq 7$\\
  \item $0 \geq \dfrac{x}{9} \geq - 1$\\
  \item $11 < 8 + k \leq 12$\\
  \item $- 11 \leq n - 9 \leq - 5$\\
  \item $- 3 < x - 1 < 1$\\
  \item $- 2 < 1-3x \leq 10$\\
  \item $1 \leq \dfrac{p}{8} \leq 0$\\
  \item $- 22 \leq 2 n - 10 \leq - 16$\\
  \item $\dfrac{1}{2}< 5-\dfrac{x}{3} \leq 4$\\
	\end{multicols}
	\item $- 5 b + 10 \leq 30  \ \ \text{AND} \ \  7 b + 2 \leq - 40$\\
  \item $n + 10 \geq 15  \ \ \text{OR} \ \  4 n - 5 < - 1$\\
	\item $3 x - 9 < 2 x + 10  \ \ \text{AND} \ \  5 + 7 x \leq 10 x - 10$\\
  \item $4 n + 8 < 3 n - 6  \ \ \text{OR} \ \  10 n - 8 \geq 9 + 9 n$\\
  \item $- 8 - 6 v \leq 8 - 8 v  \ \ \text{AND} \ \  7 v + 9 \leq 6 + 10 v$\\
  \item $5 - 2 a \geq 2 a + 1  \ \ \text{OR} \ \  10 a - 10 \geq 9 a + 9$\\
  \item $1 + 5 k \leq 7 k - 3  \ \ \text{OR} \ \  k - 10 > 2 k + 10$\\
  \item $8 - 10 r \leq 8 + 4 r  \ \ \text{OR} \ \  - 6 + 8 r < 2 + 8 r$\\
  \item $2 x + 9 \geq 10 x + 1  \ \ \text{AND} \ \  3 x - 2 < 7 x + 2$\\
  \item $- 9 m + 2 < - 10 - 6 m  \ \ \text{OR} \ \  - m + 5 \geq 10 + 4 m$
\end{enumerate}
\newpage
\end{document}
