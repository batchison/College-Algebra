\documentclass[12pt]{article}
\usepackage[top=1in,left=1in,bottom=1in,right=1in,headsep=2pt]{geometry}	
\usepackage{amssymb,amsmath,amsthm,amsfonts}
\usepackage{chapterfolder,docmute,setspace}
\usepackage{cancel,multicol,tikz,verbatim,framed,polynom,enumitem}
\usepackage[colorlinks, hyperindex, plainpages=false, linkcolor=blue, urlcolor=blue, pdfpagelabels]{hyperref}
% Use the cc-by-nc-sa license for any content linked with Stitz and Zeager's text.  Otherwise, use the cc-by-sa license.
\usepackage[type={CC},modifier={by-sa},version={4.0},]{doclicense}
%\usepackage[type={CC},modifier={by-nc-sa},version={4.0},]{doclicense}

\theoremstyle{definition}
\newtheorem{example}{Example}
\newcommand{\Desmos}{\href{https://www.desmos.com/}{Desmos}}
\setlength{\parindent}{0em}
\setlist{itemsep=0em}
\setlength{\parskip}{0.1em}
% This document is used for ordering of lessons.  If an instructor wishes to change the ordering of assessments, the following steps must be taken:

% 1) Reassign the appropriate numbers for each lesson in the \setcounter commands included in this file.
% 2) Rearrange the \include commands in the master file (the file with 'Course Pack' in the name) to accurately reflect the changes.  
% 3) Rearrange the \items in the measureable_outcomes file to accurately reflect the changes.  Be mindful of page breaks when moving items.
% 4) Re-build all affected files (master file, measureable_outcomes file, and any lesson whose numbering has changed).

%Note: The placement of each \newcounter and \setcounter command reflects the original/default ordering of topics (linears, systems, quadratics, functions, polynomials, rationals).

\newcounter{lesson_solving_linear_equations}
\newcounter{lesson_equations_containing_absolute_values}
\newcounter{lesson_graphing_lines}
\newcounter{lesson_two_forms_of_a_linear_equation}
\newcounter{lesson_parallel_and_perpendicular_lines}
\newcounter{lesson_linear_inequalities}
\newcounter{lesson_compound_inequalities}
\newcounter{lesson_inequalities_containing_absolute_values}
\newcounter{lesson_graphing_systems}
\newcounter{lesson_substitution}
\newcounter{lesson_elimination}
\newcounter{lesson_quadratics_introduction}
\newcounter{lesson_factoring_GCF}
\newcounter{lesson_factoring_grouping}
\newcounter{lesson_factoring_trinomials_a_is_1}
\newcounter{lesson_factoring_trinomials_a_neq_1}
\newcounter{lesson_solving_by_factoring}
\newcounter{lesson_square_roots}
\newcounter{lesson_i_and_complex_numbers}
\newcounter{lesson_vertex_form_and_graphing}
\newcounter{lesson_solve_by_square_roots}
\newcounter{lesson_extracting_square_roots}
\newcounter{lesson_the_discriminant}
\newcounter{lesson_the_quadratic_formula}
\newcounter{lesson_quadratic_inequalities}
\newcounter{lesson_functions_and_relations}
\newcounter{lesson_evaluating_functions}
\newcounter{lesson_finding_domain_and_range_graphically}
\newcounter{lesson_fundamental_functions}
\newcounter{lesson_finding_domain_algebraically}
\newcounter{lesson_solving_functions}
\newcounter{lesson_function_arithmetic}
\newcounter{lesson_composite_functions}
\newcounter{lesson_inverse_functions_definition_and_HLT}
\newcounter{lesson_finding_an_inverse_function}
\newcounter{lesson_transformations_translations}
\newcounter{lesson_transformations_reflections}
\newcounter{lesson_transformations_scalings}
\newcounter{lesson_transformations_summary}
\newcounter{lesson_piecewise_functions}
\newcounter{lesson_functions_containing_absolute_values}
\newcounter{lesson_absolute_as_piecewise}
\newcounter{lesson_polynomials_introduction}
\newcounter{lesson_sign_diagrams_polynomials}
\newcounter{lesson_factoring_quadratic_type}
\newcounter{lesson_factoring_summary}
\newcounter{lesson_polynomial_division}
\newcounter{lesson_synthetic_division}
\newcounter{lesson_end_behavior_polynomials}
\newcounter{lesson_local_behavior_polynomials}
\newcounter{lesson_rational_root_theorem}
\newcounter{lesson_polynomials_graphing_summary}
\newcounter{lesson_polynomial_inequalities}
\newcounter{lesson_rationals_introduction_and_terminology}
\newcounter{lesson_sign_diagrams_rationals}
\newcounter{lesson_horizontal_asymptotes}
\newcounter{lesson_slant_and_curvilinear_asymptotes}
\newcounter{lesson_vertical_asymptotes}
\newcounter{lesson_holes}
\newcounter{lesson_rationals_graphing_summary}

\setcounter{lesson_solving_linear_equations}{1}
\setcounter{lesson_equations_containing_absolute_values}{2}
\setcounter{lesson_graphing_lines}{3}
\setcounter{lesson_two_forms_of_a_linear_equation}{4}
\setcounter{lesson_parallel_and_perpendicular_lines}{5}
\setcounter{lesson_linear_inequalities}{6}
\setcounter{lesson_compound_inequalities}{7}
\setcounter{lesson_inequalities_containing_absolute_values}{8}
\setcounter{lesson_graphing_systems}{9}
\setcounter{lesson_substitution}{10}
\setcounter{lesson_elimination}{11}
\setcounter{lesson_quadratics_introduction}{16}
\setcounter{lesson_factoring_GCF}{17}
\setcounter{lesson_factoring_grouping}{18}
\setcounter{lesson_factoring_trinomials_a_is_1}{19}
\setcounter{lesson_factoring_trinomials_a_neq_1}{20}
\setcounter{lesson_solving_by_factoring}{21}
\setcounter{lesson_square_roots}{22}
\setcounter{lesson_i_and_complex_numbers}{23}
\setcounter{lesson_vertex_form_and_graphing}{24}
\setcounter{lesson_solve_by_square_roots}{25}
\setcounter{lesson_extracting_square_roots}{26}
\setcounter{lesson_the_discriminant}{27}
\setcounter{lesson_the_quadratic_formula}{28}
\setcounter{lesson_quadratic_inequalities}{29}
\setcounter{lesson_functions_and_relations}{12}
\setcounter{lesson_evaluating_functions}{13}
\setcounter{lesson_finding_domain_and_range_graphically}{14}
\setcounter{lesson_fundamental_functions}{15}
\setcounter{lesson_finding_domain_algebraically}{30}
\setcounter{lesson_solving_functions}{31}
\setcounter{lesson_function_arithmetic}{32}
\setcounter{lesson_composite_functions}{33}
\setcounter{lesson_inverse_functions_definition_and_HLT}{34}
\setcounter{lesson_finding_an_inverse_function}{35}
\setcounter{lesson_transformations_translations}{36}
\setcounter{lesson_transformations_reflections}{37}
\setcounter{lesson_transformations_scalings}{38}
\setcounter{lesson_transformations_summary}{39}
\setcounter{lesson_piecewise_functions}{40}
\setcounter{lesson_functions_containing_absolute_values}{41}
\setcounter{lesson_absolute_as_piecewise}{42}
\setcounter{lesson_polynomials_introduction}{43}
\setcounter{lesson_sign_diagrams_polynomials}{44}
\setcounter{lesson_factoring_quadratic_type}{46}
\setcounter{lesson_factoring_summary}{45}
\setcounter{lesson_polynomial_division}{47}
\setcounter{lesson_synthetic_division}{48}
\setcounter{lesson_end_behavior_polynomials}{49}
\setcounter{lesson_local_behavior_polynomials}{50}
\setcounter{lesson_rational_root_theorem}{51}
\setcounter{lesson_polynomials_graphing_summary}{52}
\setcounter{lesson_polynomial_inequalities}{53}
\setcounter{lesson_rationals_introduction_and_terminology}{54}
\setcounter{lesson_sign_diagrams_rationals}{55}
\setcounter{lesson_horizontal_asymptotes}{56}
\setcounter{lesson_slant_and_curvilinear_asymptotes}{57}
\setcounter{lesson_vertical_asymptotes}{58}
\setcounter{lesson_holes}{59}
\setcounter{lesson_rationals_graphing_summary}{60}

\begin{document}
{\bf \large Lesson \arabic{lesson_equations_containing_absolute_values}: Equations Containing Absolute Values}\phantomsection\label{les:equations_containing_absolute_values}
\\ CC attribute: \href{http://www.wallace.ccfaculty.org/book/book.html}{\it{Beginning and Intermediate Algebra}} by T. Wallace. 
%\\ CC attribute: \href{http://www.stitz-zeager.com}{\it{College Algebra}} by C. Stitz and J. Zeager. 
\hfill \doclicenseImage[imagewidth=5em]\\
\par
{\bf Objective:} Solve an equation that contains one or more absolute value(s).\\
\par
{\bf Students will be able to:}
\begin{itemize}
	\item Solve and check the solutions to an equation that contains an absolute value.
\end{itemize}
{\bf Prerequisite Knowledge:}
\begin{itemize}
	\item Evaluating absolute value expressions.
	\item Applying the distributive property.
	\item Checking solutions to equations.
\end{itemize}
\hrulefill

{\bf Lesson:}
\par
{\bf I - Motivating Example(s):}
  \begin{eqnarray*}
    |x| = 7 &  & \text{Absolute value can be positive or negative}\\
    x = \pm 7 &  & \text{Our solution}
  \end{eqnarray*}
Notice that we have considered two possibilities, both the positive and negative. Either way, the absolute value of our number will be positive $7$.  When we have absolute values in our problem it is important to first isolate the absolute value, then remove the absolute value by considering both the positive and negative solutions.\par
\ \par
{\bf II - Demo/Discussion Problems:}\\
\par
Solve each of the following equations containing absolute values.
\begin{enumerate}
	\item $5|x|-4=26$\\
	\item $2-4|2x+3|=-18$\\
	\item $7+|2x-5|=4$\\
	\item $|2x-7|=|4x+6|$
\end{enumerate}
\newpage
{\bf III - Practice Problems:}\\
\par
Solve each equation.
\par
\begin{multicols}{3}
  1) $| x| = 8$\\ \ \\
  2) $| n | = 7$\\ \ \\
  3) $| b| = 1$\\ \ \\
  4) $| x | = 2$\\ \ \\
  5) $| 5 + 8 a| = 53$\\ \ \\
  6) $|9n + 8| = 46$\\ \ \\
  7) $|3k + 8| = 2$\\ \ \\
  8) $|3 - x| = 6$\\ \ \\
  9) $|9 + 7 x| = 30$\\ \ \\
  10) $|5n + 7| = 23$\\ \ \\
  11) $|8 + 6 m| = 50$\\ \ \\
  12) $|9p + 6| = 3$\\ \ \\ \ \\
  13) $|6 - 2 x| = 24$\\ \ \\
  14) $|3n - 2| = 7$\\ \ \\
  15) $- 7| - 3 - 3 r| = - 21$\\ \ \\
  16) $| 2 + 2 b| + 1 = 3$\\ \ \\
  17) $7 | - 7 x - 3| = 21$\\ \ \\
  18) $\dfrac{| - 4 - 3 n|}{4} = 2$\\ \ \\
  19) $\dfrac{| - 4 b - 10|}{8} = 3$\\ \ \\
  20) $8 |5p + 8| - 5 = 11$\\ \ \\
  21) $8 | x + 7 | - 3 = 5$\\ \ \\
  22) $3 - |6n + 7| = - 40$\\ \ \\
  23) $5 |3 + 7 m| + 1 = 51$\\ \ \\
  24) $4 |r + 7| + 3 = 59$\\ \ \\ \ \\
  25) $3 + 5 |8 - 2 x| = 63$\\ \ \\
  26) $5 + 8| - 10 n - 2| = 101$\\ \ \\
  27) $|6b - 2| + 10 = 44$\\ \ \\
  28) $7 |10v - 2| - 9 = 5$\\ \ \\
  29) $- 7 + 8| - 7 x - 3| = 73$\\ \ \\
  30) $8 |3 - 3 n| - 5 = 91$\\ \ \\
  31) $|5x + 3| = |2x - 1|$\\ \ \\
	32) $| 2 + 3 x| = |4 - 2 x|$\\ \ \\
	33) $| 3 x - 4| = |2x + 3|$\\ \ \\
	34) $\left| \dfrac{2 x - 5}{3} \right| = \left| \dfrac{3 x + 4}{2} \right|$\\ \ \\
	35) $\left| \dfrac{4 x - 2}{5} \right| = \left| \dfrac{6 x + 3}{2} \right|$\\ \ \\
	36) $\left| \dfrac{3 x + 2}{2} \right| = \left| \dfrac{2 x - 3}{3} \right|$\\ \ \\
\end{multicols}
\newpage
\end{document}