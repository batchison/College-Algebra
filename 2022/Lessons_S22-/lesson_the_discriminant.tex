\documentclass[12pt]{article}
\usepackage[top=1in,left=1in,bottom=1in,right=1in,headsep=2pt]{geometry}	
\usepackage{amssymb,amsmath,amsthm,amsfonts}
\usepackage{chapterfolder,docmute,setspace}
\usepackage{cancel,multicol,tikz,verbatim,framed,polynom,enumitem}
\usepackage[colorlinks, hyperindex, plainpages=false, linkcolor=blue, urlcolor=blue, pdfpagelabels]{hyperref}
% Use the cc-by-nc-sa license for any content linked with Stitz and Zeager's text.  Otherwise, use the cc-by-sa license.
%\usepackage[type={CC},modifier={by-sa},version={4.0},]{doclicense}
\usepackage[type={CC},modifier={by-nc-sa},version={4.0},]{doclicense}

\theoremstyle{definition}
\newtheorem{example}{Example}
\newcommand{\Desmos}{\href{https://www.desmos.com/}{Desmos}}
\setlength{\parindent}{0em}
\setlist{itemsep=0em}
\setlength{\parskip}{0.1em}
% This document is used for ordering of lessons.  If an instructor wishes to change the ordering of assessments, the following steps must be taken:

% 1) Reassign the appropriate numbers for each lesson in the \setcounter commands included in this file.
% 2) Rearrange the \include commands in the master file (the file with 'Course Pack' in the name) to accurately reflect the changes.  
% 3) Rearrange the \items in the measureable_outcomes file to accurately reflect the changes.  Be mindful of page breaks when moving items.
% 4) Re-build all affected files (master file, measureable_outcomes file, and any lesson whose numbering has changed).

%Note: The placement of each \newcounter and \setcounter command reflects the original/default ordering of topics (linears, systems, quadratics, functions, polynomials, rationals).

\newcounter{lesson_solving_linear_equations}
\newcounter{lesson_equations_containing_absolute_values}
\newcounter{lesson_graphing_lines}
\newcounter{lesson_two_forms_of_a_linear_equation}
\newcounter{lesson_parallel_and_perpendicular_lines}
\newcounter{lesson_linear_inequalities}
\newcounter{lesson_compound_inequalities}
\newcounter{lesson_inequalities_containing_absolute_values}
\newcounter{lesson_graphing_systems}
\newcounter{lesson_substitution}
\newcounter{lesson_elimination}
\newcounter{lesson_quadratics_introduction}
\newcounter{lesson_factoring_GCF}
\newcounter{lesson_factoring_grouping}
\newcounter{lesson_factoring_trinomials_a_is_1}
\newcounter{lesson_factoring_trinomials_a_neq_1}
\newcounter{lesson_solving_by_factoring}
\newcounter{lesson_square_roots}
\newcounter{lesson_i_and_complex_numbers}
\newcounter{lesson_vertex_form_and_graphing}
\newcounter{lesson_solve_by_square_roots}
\newcounter{lesson_extracting_square_roots}
\newcounter{lesson_the_discriminant}
\newcounter{lesson_the_quadratic_formula}
\newcounter{lesson_quadratic_inequalities}
\newcounter{lesson_functions_and_relations}
\newcounter{lesson_evaluating_functions}
\newcounter{lesson_finding_domain_and_range_graphically}
\newcounter{lesson_fundamental_functions}
\newcounter{lesson_finding_domain_algebraically}
\newcounter{lesson_solving_functions}
\newcounter{lesson_function_arithmetic}
\newcounter{lesson_composite_functions}
\newcounter{lesson_inverse_functions_definition_and_HLT}
\newcounter{lesson_finding_an_inverse_function}
\newcounter{lesson_transformations_translations}
\newcounter{lesson_transformations_reflections}
\newcounter{lesson_transformations_scalings}
\newcounter{lesson_transformations_summary}
\newcounter{lesson_piecewise_functions}
\newcounter{lesson_functions_containing_absolute_values}
\newcounter{lesson_absolute_as_piecewise}
\newcounter{lesson_polynomials_introduction}
\newcounter{lesson_sign_diagrams_polynomials}
\newcounter{lesson_factoring_quadratic_type}
\newcounter{lesson_factoring_summary}
\newcounter{lesson_polynomial_division}
\newcounter{lesson_synthetic_division}
\newcounter{lesson_end_behavior_polynomials}
\newcounter{lesson_local_behavior_polynomials}
\newcounter{lesson_rational_root_theorem}
\newcounter{lesson_polynomials_graphing_summary}
\newcounter{lesson_polynomial_inequalities}
\newcounter{lesson_rationals_introduction_and_terminology}
\newcounter{lesson_sign_diagrams_rationals}
\newcounter{lesson_horizontal_asymptotes}
\newcounter{lesson_slant_and_curvilinear_asymptotes}
\newcounter{lesson_vertical_asymptotes}
\newcounter{lesson_holes}
\newcounter{lesson_rationals_graphing_summary}

\setcounter{lesson_solving_linear_equations}{1}
\setcounter{lesson_equations_containing_absolute_values}{2}
\setcounter{lesson_graphing_lines}{3}
\setcounter{lesson_two_forms_of_a_linear_equation}{4}
\setcounter{lesson_parallel_and_perpendicular_lines}{5}
\setcounter{lesson_linear_inequalities}{6}
\setcounter{lesson_compound_inequalities}{7}
\setcounter{lesson_inequalities_containing_absolute_values}{8}
\setcounter{lesson_graphing_systems}{9}
\setcounter{lesson_substitution}{10}
\setcounter{lesson_elimination}{11}
\setcounter{lesson_quadratics_introduction}{16}
\setcounter{lesson_factoring_GCF}{17}
\setcounter{lesson_factoring_grouping}{18}
\setcounter{lesson_factoring_trinomials_a_is_1}{19}
\setcounter{lesson_factoring_trinomials_a_neq_1}{20}
\setcounter{lesson_solving_by_factoring}{21}
\setcounter{lesson_square_roots}{22}
\setcounter{lesson_i_and_complex_numbers}{23}
\setcounter{lesson_vertex_form_and_graphing}{24}
\setcounter{lesson_solve_by_square_roots}{25}
\setcounter{lesson_extracting_square_roots}{26}
\setcounter{lesson_the_discriminant}{27}
\setcounter{lesson_the_quadratic_formula}{28}
\setcounter{lesson_quadratic_inequalities}{29}
\setcounter{lesson_functions_and_relations}{12}
\setcounter{lesson_evaluating_functions}{13}
\setcounter{lesson_finding_domain_and_range_graphically}{14}
\setcounter{lesson_fundamental_functions}{15}
\setcounter{lesson_finding_domain_algebraically}{30}
\setcounter{lesson_solving_functions}{31}
\setcounter{lesson_function_arithmetic}{32}
\setcounter{lesson_composite_functions}{33}
\setcounter{lesson_inverse_functions_definition_and_HLT}{34}
\setcounter{lesson_finding_an_inverse_function}{35}
\setcounter{lesson_transformations_translations}{36}
\setcounter{lesson_transformations_reflections}{37}
\setcounter{lesson_transformations_scalings}{38}
\setcounter{lesson_transformations_summary}{39}
\setcounter{lesson_piecewise_functions}{40}
\setcounter{lesson_functions_containing_absolute_values}{41}
\setcounter{lesson_absolute_as_piecewise}{42}
\setcounter{lesson_polynomials_introduction}{43}
\setcounter{lesson_sign_diagrams_polynomials}{44}
\setcounter{lesson_factoring_quadratic_type}{46}
\setcounter{lesson_factoring_summary}{45}
\setcounter{lesson_polynomial_division}{47}
\setcounter{lesson_synthetic_division}{48}
\setcounter{lesson_end_behavior_polynomials}{49}
\setcounter{lesson_local_behavior_polynomials}{50}
\setcounter{lesson_rational_root_theorem}{51}
\setcounter{lesson_polynomials_graphing_summary}{52}
\setcounter{lesson_polynomial_inequalities}{53}
\setcounter{lesson_rationals_introduction_and_terminology}{54}
\setcounter{lesson_sign_diagrams_rationals}{55}
\setcounter{lesson_horizontal_asymptotes}{56}
\setcounter{lesson_slant_and_curvilinear_asymptotes}{57}
\setcounter{lesson_vertical_asymptotes}{58}
\setcounter{lesson_holes}{59}
\setcounter{lesson_rationals_graphing_summary}{60}

\begin{document}
{\bf \large Lesson \arabic{lesson_the_discriminant}: The Discriminant}\phantomsection\label{les:the_discriminant}
%\\ CC attribute: \href{http://www.wallace.ccfaculty.org/book/book.html}{\it{Beginning and Intermediate Algebra}} by T. Wallace. 
\\ CC attribute: \href{http://www.stitz-zeager.com}{\it{College Algebra}} by C. Stitz and J. Zeager. 
\hfill \doclicenseImage[imagewidth=5em]\\
\par
{\bf Objective:} Use the discriminant to determine the number of real solutions to a quadratic equation.\\
\par
{\bf Students will be able to:}
\begin{itemize}
	\item Find, simplify, and interpret the discriminant of a quadratic equation in standard form.
\end{itemize}
{\bf Prerequisite Knowledge:}
\begin{itemize}
	\item Identifying coefficients of a quadratic in standard form.
	\item Order of operations.
\end{itemize}
\hrulefill

{\bf Lesson:}\\
\ \par
The {\it discriminant} of a quadratic expression $ax^2+bx+c$ is defined as the real number $D=b^2-4ac$.  In the next lesson, we will see that the discriminant is one piece of the larger quadratic formula,
$$x=\dfrac{-b\pm\sqrt{b^2-4ac}}{2a},$$
which is used for identifying the roots of the equation $y=ax^2+bx+c$.
Since the discriminant appears underneath of a square root in the quadratic formula, whether it is positive, negative, or zero will determine the number of real roots of a quadratic, and consequently the number of $x-$intercepts on its corresponding parabola.

\begin{center}
\begin{tikzpicture}[xscale=0.75,yscale=0.75]
	\draw [<->](-9.5,0) -- coordinate (x axis mid) (-4.5,0) node[below right] {$x$};
	\draw [<->](-2.5,0) -- coordinate (x axis mid) (2.5,0) node[below right] {$x$};
	\draw [<->](4.5,0) -- coordinate (x axis mid) (9.5,0) node[below right] {$x$};
	\draw [<->](-7,-1.5) -- coordinate (x axis mid) (-7,2.5) node[above right] {$y$};
	\draw [<->](0,-1.5) -- coordinate (x axis mid) (0,2.5) node[above right] {$y$};
	\draw [<->](7,-1.5) -- coordinate (x axis mid) (7,2.5) node[above right] {$y$};
	\draw [<->] plot [domain=-0.4:1.9, samples=100] (\x,{1.7*(\x-0.75)^2});
	\draw [<->] plot [domain=6.1:8.4, samples=100] (\x,{1.7*(\x-7.25)^2-1});
	\draw [<->] plot [domain=-6.35:-8.65, samples=100] (\x,{1.7*(\x+7.5)^2+0.5});
	\draw[fill] (0.75,0) circle (0.05);
	\draw[fill] (8.017,0) circle (0.05);
	\draw[fill] (6.483,0) circle (0.05);
 \end{tikzpicture}
\end{center}

\begin{center}
\begin{multicols}{3}
Negative Discriminant\\
$b^2-4ac<0$\\
No Real Solutions\\
Zero Discriminant\\
$b^2-4ac=0$\\
One Real Solution\\
Positive Discriminant\\
$b^2-4ac>0$\\
Two Real Solutions\\
\end{multicols}
\end{center}

\begin{center}
\begin{tikzpicture}[xscale=0.75,yscale=0.75]
	\draw [<->](-9.5,0) -- coordinate (x axis mid) (-4.5,0) node[below right] {$x$};
	\draw [<->](-2.5,0) -- coordinate (x axis mid) (2.5,0) node[below right] {$x$};
	\draw [<->](4.5,0) -- coordinate (x axis mid) (9.5,0) node[below right] {$x$};
	\draw [<->](-7,-2.5) -- coordinate (x axis mid) (-7,1.5) node[above right] {$y$};
	\draw [<->](0,-2.5) -- coordinate (x axis mid) (0,1.5) node[above right] {$y$};
	\draw [<->](7,-2.5) -- coordinate (x axis mid) (7,1.5) node[above right] {$y$};
	\draw [<->] plot [domain=-0.4:1.9, samples=100] (\x,{-1.7*(\x-0.75)^2});
	\draw [<->] plot [domain=6.1:8.4, samples=100] (\x,{-1.7*(\x-7.25)^2+1});
	\draw [<->] plot [domain=-6.35:-8.65, samples=100] (\x,{-1.7*(\x+7.5)^2-0.5});
	\draw[fill] (0.75,0) circle (0.05);
	\draw[fill] (8.017,0) circle (0.05);
	\draw[fill] (6.483,0) circle (0.05);
 \end{tikzpicture}
\end{center}
  
{\bf I - Motivating Example(s):}\\
\ \par
{\bf Example:}  Determine the number of real roots for the quadratic equation below.
$$y=3x^2-5x+2$$
\begin{eqnarray*}
D & = & b^2-4ac\\
  & = & (-5)^2-4(3)(2)\\
	& = & 25-24\\
	& = & 1
\end{eqnarray*}
Since $D>0,$ the given equation has two real roots.\\
\ \par
{\bf II - Demo/Discussion Problems:}\\
\ \par
Determine the number of real roots for each of the quadratic equations below.
\begin{enumerate}
	\item $y=-4x^2-15$
	\item $y=3x^2-12x-15$
	\item $y=2x^2+4x+2$
	\item $y=10x^2+31x+24$
	\item $y=x^2-4x+13$
\end{enumerate}
\ \par
{\bf III - Practice Problems:}\\
\ \par
Determine the number of real roots for each of the quadratic equations below.
\begin{multicols}{3}
\begin{enumerate}
  \item $y=x^2+6$
  \item $y=x^2+2x-1$
	\item $y=-3x^2-12x-5$
  \item $y=3x^2+12x-1$
  \item $y=-5x^2-40x$
  \item $y=x^2-8x+15$
  \item $y=x^2+4x-2$
  \item $y=x^2+16x-2$
  \item $y=4x^2+10x$
	\item $y=5x^2-4x+1$
	\item $y=-x^2+3x-9$
	\item $y=x^2+6x+9$
\end{enumerate}
\end{multicols}
\newpage
\end{document}