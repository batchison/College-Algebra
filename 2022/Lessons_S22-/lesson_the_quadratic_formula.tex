\documentclass[12pt]{article}
\usepackage[top=1in,left=1in,bottom=1in,right=1in,headsep=2pt]{geometry}	
\usepackage{amssymb,amsmath,amsthm,amsfonts}
\usepackage{chapterfolder,docmute,setspace}
\usepackage{cancel,multicol,tikz,verbatim,framed,polynom,enumitem}
\usepackage[colorlinks, hyperindex, plainpages=false, linkcolor=blue, urlcolor=blue, pdfpagelabels]{hyperref}
% Use the cc-by-nc-sa license for any content linked with Stitz and Zeager's text.  Otherwise, use the cc-by-sa license.
%\usepackage[type={CC},modifier={by-sa},version={4.0},]{doclicense}
\usepackage[type={CC},modifier={by-nc-sa},version={4.0},]{doclicense}

\theoremstyle{definition}
\newtheorem{example}{Example}
\newcommand{\Desmos}{\href{https://www.desmos.com/}{Desmos}}
\setlength{\parindent}{0em}
\setlist{itemsep=0em}
\setlength{\parskip}{0.1em}
% This document is used for ordering of lessons.  If an instructor wishes to change the ordering of assessments, the following steps must be taken:

% 1) Reassign the appropriate numbers for each lesson in the \setcounter commands included in this file.
% 2) Rearrange the \include commands in the master file (the file with 'Course Pack' in the name) to accurately reflect the changes.  
% 3) Rearrange the \items in the measureable_outcomes file to accurately reflect the changes.  Be mindful of page breaks when moving items.
% 4) Re-build all affected files (master file, measureable_outcomes file, and any lesson whose numbering has changed).

%Note: The placement of each \newcounter and \setcounter command reflects the original/default ordering of topics (linears, systems, quadratics, functions, polynomials, rationals).

\newcounter{lesson_solving_linear_equations}
\newcounter{lesson_equations_containing_absolute_values}
\newcounter{lesson_graphing_lines}
\newcounter{lesson_two_forms_of_a_linear_equation}
\newcounter{lesson_parallel_and_perpendicular_lines}
\newcounter{lesson_linear_inequalities}
\newcounter{lesson_compound_inequalities}
\newcounter{lesson_inequalities_containing_absolute_values}
\newcounter{lesson_graphing_systems}
\newcounter{lesson_substitution}
\newcounter{lesson_elimination}
\newcounter{lesson_quadratics_introduction}
\newcounter{lesson_factoring_GCF}
\newcounter{lesson_factoring_grouping}
\newcounter{lesson_factoring_trinomials_a_is_1}
\newcounter{lesson_factoring_trinomials_a_neq_1}
\newcounter{lesson_solving_by_factoring}
\newcounter{lesson_square_roots}
\newcounter{lesson_i_and_complex_numbers}
\newcounter{lesson_vertex_form_and_graphing}
\newcounter{lesson_solve_by_square_roots}
\newcounter{lesson_extracting_square_roots}
\newcounter{lesson_the_discriminant}
\newcounter{lesson_the_quadratic_formula}
\newcounter{lesson_quadratic_inequalities}
\newcounter{lesson_functions_and_relations}
\newcounter{lesson_evaluating_functions}
\newcounter{lesson_finding_domain_and_range_graphically}
\newcounter{lesson_fundamental_functions}
\newcounter{lesson_finding_domain_algebraically}
\newcounter{lesson_solving_functions}
\newcounter{lesson_function_arithmetic}
\newcounter{lesson_composite_functions}
\newcounter{lesson_inverse_functions_definition_and_HLT}
\newcounter{lesson_finding_an_inverse_function}
\newcounter{lesson_transformations_translations}
\newcounter{lesson_transformations_reflections}
\newcounter{lesson_transformations_scalings}
\newcounter{lesson_transformations_summary}
\newcounter{lesson_piecewise_functions}
\newcounter{lesson_functions_containing_absolute_values}
\newcounter{lesson_absolute_as_piecewise}
\newcounter{lesson_polynomials_introduction}
\newcounter{lesson_sign_diagrams_polynomials}
\newcounter{lesson_factoring_quadratic_type}
\newcounter{lesson_factoring_summary}
\newcounter{lesson_polynomial_division}
\newcounter{lesson_synthetic_division}
\newcounter{lesson_end_behavior_polynomials}
\newcounter{lesson_local_behavior_polynomials}
\newcounter{lesson_rational_root_theorem}
\newcounter{lesson_polynomials_graphing_summary}
\newcounter{lesson_polynomial_inequalities}
\newcounter{lesson_rationals_introduction_and_terminology}
\newcounter{lesson_sign_diagrams_rationals}
\newcounter{lesson_horizontal_asymptotes}
\newcounter{lesson_slant_and_curvilinear_asymptotes}
\newcounter{lesson_vertical_asymptotes}
\newcounter{lesson_holes}
\newcounter{lesson_rationals_graphing_summary}

\setcounter{lesson_solving_linear_equations}{1}
\setcounter{lesson_equations_containing_absolute_values}{2}
\setcounter{lesson_graphing_lines}{3}
\setcounter{lesson_two_forms_of_a_linear_equation}{4}
\setcounter{lesson_parallel_and_perpendicular_lines}{5}
\setcounter{lesson_linear_inequalities}{6}
\setcounter{lesson_compound_inequalities}{7}
\setcounter{lesson_inequalities_containing_absolute_values}{8}
\setcounter{lesson_graphing_systems}{9}
\setcounter{lesson_substitution}{10}
\setcounter{lesson_elimination}{11}
\setcounter{lesson_quadratics_introduction}{16}
\setcounter{lesson_factoring_GCF}{17}
\setcounter{lesson_factoring_grouping}{18}
\setcounter{lesson_factoring_trinomials_a_is_1}{19}
\setcounter{lesson_factoring_trinomials_a_neq_1}{20}
\setcounter{lesson_solving_by_factoring}{21}
\setcounter{lesson_square_roots}{22}
\setcounter{lesson_i_and_complex_numbers}{23}
\setcounter{lesson_vertex_form_and_graphing}{24}
\setcounter{lesson_solve_by_square_roots}{25}
\setcounter{lesson_extracting_square_roots}{26}
\setcounter{lesson_the_discriminant}{27}
\setcounter{lesson_the_quadratic_formula}{28}
\setcounter{lesson_quadratic_inequalities}{29}
\setcounter{lesson_functions_and_relations}{12}
\setcounter{lesson_evaluating_functions}{13}
\setcounter{lesson_finding_domain_and_range_graphically}{14}
\setcounter{lesson_fundamental_functions}{15}
\setcounter{lesson_finding_domain_algebraically}{30}
\setcounter{lesson_solving_functions}{31}
\setcounter{lesson_function_arithmetic}{32}
\setcounter{lesson_composite_functions}{33}
\setcounter{lesson_inverse_functions_definition_and_HLT}{34}
\setcounter{lesson_finding_an_inverse_function}{35}
\setcounter{lesson_transformations_translations}{36}
\setcounter{lesson_transformations_reflections}{37}
\setcounter{lesson_transformations_scalings}{38}
\setcounter{lesson_transformations_summary}{39}
\setcounter{lesson_piecewise_functions}{40}
\setcounter{lesson_functions_containing_absolute_values}{41}
\setcounter{lesson_absolute_as_piecewise}{42}
\setcounter{lesson_polynomials_introduction}{43}
\setcounter{lesson_sign_diagrams_polynomials}{44}
\setcounter{lesson_factoring_quadratic_type}{46}
\setcounter{lesson_factoring_summary}{45}
\setcounter{lesson_polynomial_division}{47}
\setcounter{lesson_synthetic_division}{48}
\setcounter{lesson_end_behavior_polynomials}{49}
\setcounter{lesson_local_behavior_polynomials}{50}
\setcounter{lesson_rational_root_theorem}{51}
\setcounter{lesson_polynomials_graphing_summary}{52}
\setcounter{lesson_polynomial_inequalities}{53}
\setcounter{lesson_rationals_introduction_and_terminology}{54}
\setcounter{lesson_sign_diagrams_rationals}{55}
\setcounter{lesson_horizontal_asymptotes}{56}
\setcounter{lesson_slant_and_curvilinear_asymptotes}{57}
\setcounter{lesson_vertical_asymptotes}{58}
\setcounter{lesson_holes}{59}
\setcounter{lesson_rationals_graphing_summary}{60}

\begin{document}
{\bf \large Lesson \arabic{lesson_the_quadratic_formula}: The Quadratic Formula}\phantomsection\label{les:the_quadratic_formula}
%\\ CC attribute: \href{http://www.wallace.ccfaculty.org/book/book.html}{\it{Beginning and Intermediate Algebra}} by T. Wallace. 
\\ CC attribute: \href{http://www.stitz-zeager.com}{\it{College Algebra}} by C. Stitz and J. Zeager. 
\hfill \doclicenseImage[imagewidth=5em]\\
\par
{\bf Objective:} Solve quadratic equations using the quadratic formula.\\
\par
{\bf Students will be able to:}
\begin{itemize}
	\item Use the quadratic formula to solve a quadratic equation.
	\item Fully simplify solutions to quadratic equations obtained using the quadratic formula.
	\item Approximate a decimal solution to a quadratic equation for graphing purposes.
\end{itemize}
{\bf Prerequisite Knowledge:}
\begin{itemize}
	\item Identifying coefficients of a quadratic in standard form.
	\item Order of operations.
	\item Simplifying radicals.
	\item Evaluating expressions.
\end{itemize}
\hrulefill

{\bf Lesson:}\\
\ \par
The {\it Quadratic Formula} states that the solutions to the equation $ax^2+bx+c=0$ are given by the formula
$$x=\dfrac{-b\pm\sqrt{b^2-4ac}}{2a}.$$
  
{\bf I - Motivating Example(s):}\\
\ \par
{\bf Example:}  Solve the given equation for all values of $x$.
  \begin{eqnarray*}
    x^2 - 4 x - 1 = 0 &  & \\
		a = 1,~ b = -4,~ c = -1 & & \text{Identify~} a,b, \text{~and~}c\\ 
		& & \\
    x = \frac{-(- 4) \pm \sqrt[]{(-4)^2 - 4 (1) (-1)}}{2 (1)} &  & \text{Use quadratic formula}\\
    x = \frac{4 \pm \sqrt[]{16 + 4}}{2} &  & \text{Simplify}\\
    x = \frac{4 \pm \sqrt[]{20}}{2} &  & \text{Discriminant~is~}20 \text{~(positive)}\\
	  x = \frac{4}{2} \pm \frac{2\sqrt{5}}{2} &  & \text{Split up fraction}\\
		x = 2 \pm \sqrt{5} & & \text{Our solutions}\\
    x \approx 2\pm 2.2 &  & \sqrt{5}\approx 2.2\\
    x \approx 4.2 \text{~~or~~} x\approx -0.2 &  & \text{Decimal approximations}
  \end{eqnarray*}\\
\ \par
{\bf II - Demo/Discussion Problems:}\\
\ \par
Use the quadratic formula to find the roots of each of the following equations.  If your answer contains a square root, find a decimal approximation.
\begin{enumerate}
	\item $y=x^2+7x-8$
	\item $y=x^2-13x-30$
	\item $y=25x^2-30x-11$
	\item $y=4x^2-12x+9$
	\item $y=x^2-6x+25$
\end{enumerate}
{\bf III - Practice Problems:}\\
\ \par
Use the quadratic formula to find the roots of each of the following equations.  If your answer contains a square root, find a decimal approximation.
\begin{multicols}{3}
\begin{enumerate}
  \item $y=x^2+6$
  \item $y=x^2+2x-1$
	\item $y=-3x^2-12x-5$
  \item $y=3x^2+12x-1$
  \item $y=-5x^2-40x$
  \item $y=x^2-8x+15$
  \item $y=x^2+4x-2$
  \item $y=x^2+16x-2$
  \item $y=4x^2+10x$
	\item $y=5x^2-4x+1$
	\item $y=-x^2+3x-9$
	\item $y=x^2+6x+9$
\end{enumerate}
\end{multicols}
Solve each of the following equations using any means.
\begin{multicols}{3}
\begin{enumerate}
\setcounter{enumi}{12}
  \item $4 a^2 + 6 = 0$
  \item $3 k^2 + 2 = 0$
  \item $2 x^2 - 8 x - 2 = 0$
  \item $6 n^2 - 1 = 0$
  \item $2 m^2 - 3 = 0$
  \item $5 p^2 + 2 p + 6 = 0$
  \item $3 r^2 - 2 r - 1 = 0$
  \item $2 x^2 - 2 x - 15 = 0$
  \item $4 n^2 - 36 = 0$
  \item $3 b^2 + 6 = 0$
  \item $v^2 - 4 v - 5 = - 8$
  \item $2 x^2 + 4 x + 12 = 8$
  \item $2 a^2 + 3 a + 14 = 6$
  \item $6 n^2 - 3 n + 3 = - 4$
  \item $3 k^2 + 3 k - 4 = 7$
  \item $4 x^2 - 14 = - 2$
  \item $7 x^2 + 3 x - 16 = - 2$
  \item $4 n^2 + 5 n = 7$
  \item $2 p^2 + 6 p - 16 = 4$
  \item $m^2 + 4 m - 48 = - 3$
  \item $3 n^2 + 3 n = - 3$
  \item $3 b^2 - 3 = 8 b$
  \item $2 x^2 = - 7 x + 49$
  \item $3 r^2 + 4 = - 6 r$
  \item $5 x^2 = 7 x + 7$
  \item $6 a^2 = - 5 a + 13$
  \item $8 n^2 = - 3 n - 8$
  \item $6 v^2 = 4 + 6 v$
  \item $2 x^2 + 5 x = - 3$
  \item $x^2 = 8$
  \item $4 a^2 - 64 = 0$
  \item $2 k^2 + 6 k - 16 = 2 k$
  \item $4 p^2 + 5 p - 36 = 3 p^2$
  \item $12 x^2 + x + 7 = 5 x^2 + 5 x$
  \item $- 5 n^2 - 3 n - 52 = 2 - 7 n^2$
  \item $7 m^2 - 6 m + 6 = - m$
  \item $7 r^2 - 12 = - 3 r$
  \item $3 x^2 - 3 = x^2$
  \item $2 n^2 - 9 = 4$
  \item $6 b^2 = b^2 + 7 - b$
\end{enumerate}
\end{multicols}
\newpage
\end{document}