\documentclass[12pt]{article}
\usepackage[top=1in,left=1in,bottom=1in,right=1in,headsep=2pt]{geometry}	
\usepackage{amssymb,amsmath,amsthm,amsfonts}
\usepackage{chapterfolder,docmute,setspace}
\usepackage{cancel,multicol,tikz,verbatim,framed,polynom,enumitem}
\usepackage[colorlinks, hyperindex, plainpages=false, linkcolor=blue, urlcolor=blue, pdfpagelabels]{hyperref}
% Use the cc-by-nc-sa license for any content linked with Stitz and Zeager's text.  Otherwise, use the cc-by-sa license.
\usepackage[type={CC},modifier={by-sa},version={4.0},]{doclicense}
%\usepackage[type={CC},modifier={by-nc-sa},version={4.0},]{doclicense}

\theoremstyle{definition}
\newtheorem{example}{Example}
\newcommand{\Desmos}{\href{https://www.desmos.com/}{Desmos}}
\setlength{\parindent}{0em}
\setlist{itemsep=0em}
\setlength{\parskip}{0.1em}
% This document is used for ordering of lessons.  If an instructor wishes to change the ordering of assessments, the following steps must be taken:

% 1) Reassign the appropriate numbers for each lesson in the \setcounter commands included in this file.
% 2) Rearrange the \include commands in the master file (the file with 'Course Pack' in the name) to accurately reflect the changes.  
% 3) Rearrange the \items in the measureable_outcomes file to accurately reflect the changes.  Be mindful of page breaks when moving items.
% 4) Re-build all affected files (master file, measureable_outcomes file, and any lesson whose numbering has changed).

%Note: The placement of each \newcounter and \setcounter command reflects the original/default ordering of topics (linears, systems, quadratics, functions, polynomials, rationals).

\newcounter{lesson_solving_linear_equations}
\newcounter{lesson_equations_containing_absolute_values}
\newcounter{lesson_graphing_lines}
\newcounter{lesson_two_forms_of_a_linear_equation}
\newcounter{lesson_parallel_and_perpendicular_lines}
\newcounter{lesson_linear_inequalities}
\newcounter{lesson_compound_inequalities}
\newcounter{lesson_inequalities_containing_absolute_values}
\newcounter{lesson_graphing_systems}
\newcounter{lesson_substitution}
\newcounter{lesson_elimination}
\newcounter{lesson_quadratics_introduction}
\newcounter{lesson_factoring_GCF}
\newcounter{lesson_factoring_grouping}
\newcounter{lesson_factoring_trinomials_a_is_1}
\newcounter{lesson_factoring_trinomials_a_neq_1}
\newcounter{lesson_solving_by_factoring}
\newcounter{lesson_square_roots}
\newcounter{lesson_i_and_complex_numbers}
\newcounter{lesson_vertex_form_and_graphing}
\newcounter{lesson_solve_by_square_roots}
\newcounter{lesson_extracting_square_roots}
\newcounter{lesson_the_discriminant}
\newcounter{lesson_the_quadratic_formula}
\newcounter{lesson_quadratic_inequalities}
\newcounter{lesson_functions_and_relations}
\newcounter{lesson_evaluating_functions}
\newcounter{lesson_finding_domain_and_range_graphically}
\newcounter{lesson_fundamental_functions}
\newcounter{lesson_finding_domain_algebraically}
\newcounter{lesson_solving_functions}
\newcounter{lesson_function_arithmetic}
\newcounter{lesson_composite_functions}
\newcounter{lesson_inverse_functions_definition_and_HLT}
\newcounter{lesson_finding_an_inverse_function}
\newcounter{lesson_transformations_translations}
\newcounter{lesson_transformations_reflections}
\newcounter{lesson_transformations_scalings}
\newcounter{lesson_transformations_summary}
\newcounter{lesson_piecewise_functions}
\newcounter{lesson_functions_containing_absolute_values}
\newcounter{lesson_absolute_as_piecewise}
\newcounter{lesson_polynomials_introduction}
\newcounter{lesson_sign_diagrams_polynomials}
\newcounter{lesson_factoring_quadratic_type}
\newcounter{lesson_factoring_summary}
\newcounter{lesson_polynomial_division}
\newcounter{lesson_synthetic_division}
\newcounter{lesson_end_behavior_polynomials}
\newcounter{lesson_local_behavior_polynomials}
\newcounter{lesson_rational_root_theorem}
\newcounter{lesson_polynomials_graphing_summary}
\newcounter{lesson_polynomial_inequalities}
\newcounter{lesson_rationals_introduction_and_terminology}
\newcounter{lesson_sign_diagrams_rationals}
\newcounter{lesson_horizontal_asymptotes}
\newcounter{lesson_slant_and_curvilinear_asymptotes}
\newcounter{lesson_vertical_asymptotes}
\newcounter{lesson_holes}
\newcounter{lesson_rationals_graphing_summary}

\setcounter{lesson_solving_linear_equations}{1}
\setcounter{lesson_equations_containing_absolute_values}{2}
\setcounter{lesson_graphing_lines}{3}
\setcounter{lesson_two_forms_of_a_linear_equation}{4}
\setcounter{lesson_parallel_and_perpendicular_lines}{5}
\setcounter{lesson_linear_inequalities}{6}
\setcounter{lesson_compound_inequalities}{7}
\setcounter{lesson_inequalities_containing_absolute_values}{8}
\setcounter{lesson_graphing_systems}{9}
\setcounter{lesson_substitution}{10}
\setcounter{lesson_elimination}{11}
\setcounter{lesson_quadratics_introduction}{16}
\setcounter{lesson_factoring_GCF}{17}
\setcounter{lesson_factoring_grouping}{18}
\setcounter{lesson_factoring_trinomials_a_is_1}{19}
\setcounter{lesson_factoring_trinomials_a_neq_1}{20}
\setcounter{lesson_solving_by_factoring}{21}
\setcounter{lesson_square_roots}{22}
\setcounter{lesson_i_and_complex_numbers}{23}
\setcounter{lesson_vertex_form_and_graphing}{24}
\setcounter{lesson_solve_by_square_roots}{25}
\setcounter{lesson_extracting_square_roots}{26}
\setcounter{lesson_the_discriminant}{27}
\setcounter{lesson_the_quadratic_formula}{28}
\setcounter{lesson_quadratic_inequalities}{29}
\setcounter{lesson_functions_and_relations}{12}
\setcounter{lesson_evaluating_functions}{13}
\setcounter{lesson_finding_domain_and_range_graphically}{14}
\setcounter{lesson_fundamental_functions}{15}
\setcounter{lesson_finding_domain_algebraically}{30}
\setcounter{lesson_solving_functions}{31}
\setcounter{lesson_function_arithmetic}{32}
\setcounter{lesson_composite_functions}{33}
\setcounter{lesson_inverse_functions_definition_and_HLT}{34}
\setcounter{lesson_finding_an_inverse_function}{35}
\setcounter{lesson_transformations_translations}{36}
\setcounter{lesson_transformations_reflections}{37}
\setcounter{lesson_transformations_scalings}{38}
\setcounter{lesson_transformations_summary}{39}
\setcounter{lesson_piecewise_functions}{40}
\setcounter{lesson_functions_containing_absolute_values}{41}
\setcounter{lesson_absolute_as_piecewise}{42}
\setcounter{lesson_polynomials_introduction}{43}
\setcounter{lesson_sign_diagrams_polynomials}{44}
\setcounter{lesson_factoring_quadratic_type}{46}
\setcounter{lesson_factoring_summary}{45}
\setcounter{lesson_polynomial_division}{47}
\setcounter{lesson_synthetic_division}{48}
\setcounter{lesson_end_behavior_polynomials}{49}
\setcounter{lesson_local_behavior_polynomials}{50}
\setcounter{lesson_rational_root_theorem}{51}
\setcounter{lesson_polynomials_graphing_summary}{52}
\setcounter{lesson_polynomial_inequalities}{53}
\setcounter{lesson_rationals_introduction_and_terminology}{54}
\setcounter{lesson_sign_diagrams_rationals}{55}
\setcounter{lesson_horizontal_asymptotes}{56}
\setcounter{lesson_slant_and_curvilinear_asymptotes}{57}
\setcounter{lesson_vertical_asymptotes}{58}
\setcounter{lesson_holes}{59}
\setcounter{lesson_rationals_graphing_summary}{60}

\begin{document}
{\bf \large Lesson \arabic{lesson_i_and_complex_numbers}: Complex Numbers}\phantomsection\label{les:i_and_complex_numbers}
%\\ CC attribute: \href{http://www.wallace.ccfaculty.org/book/book.html}{\it{Beginning and Intermediate Algebra}} by T. Wallace. 
%\\ CC attribute: \href{http://www.stitz-zeager.com}{\it{College Algebra}} by C. Stitz and J. Zeager. 
\hfill \doclicenseImage[imagewidth=5em]\\
\par
{\bf Objective:} Simplify expressions involving complex numbers.\\
\par
{\bf Students will be able to:}
\begin{itemize}
	\item Define the form of a complex number.
	\item Simplify square roots with negative radicands.
	\item Add, subtract, multiply, rationalize, and simplify expressions using complex numbers.
\end{itemize}
{\bf Prerequisite Knowledge:}
\begin{itemize}
	\item Properties of exponents.
	\item Combining like terms.
	\item Polynomial arithmetic.
\end{itemize}
\hrulefill

{\bf Lesson:}\\
\ \par
To work with the square root of a negative number, mathematicians have defined what we now know as imaginary and complex numbers.
\begin{eqnarray*}
  \text{{\bf Imaginary\ Number \ }} i:~ i^2
  = - 1~ (\text{thus\ } i = \sqrt{- 1}) &  & 
\end{eqnarray*}
Examples of imaginary numbers include $3 i, - 6 i, \frac{3}{5} i,$ and $3 \sqrt{5}i$. A {\it complex number} is one that contains both a real and imaginary part, such as $2 + 5 i$.
\begin{eqnarray*}
  \text{{\bf Complex \ Number: \ }} a+bi, \text{\ where \ } a \text{\ and \ } b \text{\ are \ real \ numbers, \ } i = \sqrt{- 1} &  & 
\end{eqnarray*}
With this definition, the square root of a negative number will no longer be considered undefined. We now will be able to perform basic operations with the square root of a negative number.\\
\ \par
First, we consider powers of the imaginary number $i$.
As the exponents of $i^n$ increase, our simplified value for $i^n$ will cycle through the simplified values for $i,$ $i^2=- 1,$ $i^3= - i,$ $i^4=1$. As there are 4 different possible answers in this
cycle, if we divide the exponent $n$ by 4 and consider the remainder, we can easily simplify any power of $i$ by knowing the following four values:

\begin{center}
  {\bf Cyclic Property of Powers of $i$}
\end{center}
\[ \begin{array}{rcc}
     i^0 &=& 1\\
     i^1 &=& i\\
     i^2 &=& - 1\\
     i^3 &=& - i\\
		 i^4=i^0&=&1
   \end{array} \]
{\bf I - Motivating Example(s):}\\
\ \par
{\bf Example:} Write the given expression as $a+bi$, where $a$ and $b$ are real numbers.
  \begin{eqnarray*}
    (2 + 5 i) + (4 - 7 i) &  & \text{Combine \ like \ terms, \ } 2 + 4
    \text{ \ and \ } 5 i - 7 i\\
    6 - 2 i &  & \text{Our \ solution}
  \end{eqnarray*}

{\bf Example:} Write the given expression as $a+bi$, where $a$ and $b$ are real numbers.
  \begin{eqnarray*}
    (2 - 4 i) (3 + 5 i) &  & \text{Expand}\\
    6 + 10 i - 12 i - 20 i^2 &  & \text{Simplify, \ } i^2 = - 1\\
    6 + 10 i - 12 i - 20 (- 1) &  & \text{Multiply}\\
    6 + 10 i - 12 i + 20 &  & \text{Combine \ like \ terms \ } 6 + 20
    \text{ \ and \ } 10 i - 12 i\\
    26 - 2 i &  & \text{Our \ solution}
  \end{eqnarray*}

{\bf Example:} Write the given expression as $a+bi$, where $a$ and $b$ are real numbers.
  \begin{eqnarray*}
    \frac{2 - 6 i}{4 + 8 i} &  & \text{Binomial in denominator},\\
    & & ~~~\text{multiply by conjugate}, 4 - 8 i\\
    \frac{2 - 6 i}{4 + 8 i} \left( \frac{4 - 8 i}{4 - 8 i} \right) &  &
    \text{Expand the numerator},\\
		&&~~~\text{denominator is a difference of two squares}\\
    &  & \\
    \frac{8 - 16 i - 24 i + 48 i^2}{16 - 64 i^2} &  & \text{Simplify \ } i^2 = -
    1\\
    &  & \\
    \frac{8 - 16 i - 24 i + 48 (- 1)}{16 - 64 (- 1)} &  & \text{Multiply}\\
    &  & \\
    \frac{8 - 16 i - 24 i - 48}{16 + 64} &  & \text{Combine like terms}\\
		&  & \\
    \frac{- 40 - 40 i}{80} &  & \text{Reduce, factor out 40 and divide}\\
    &  & \\
    \frac{- 1 - i}{2}& & \text{Rewrite as \ } a+bi\\
		-\dfrac{1}{2} - \dfrac{1}{2}i &  & \text{Our solution}
  \end{eqnarray*}
\newpage
{\bf II - Demo/Discussion Problems:}\\
\ \par
Rewrite each of the following complex numbers in the form $a+bi$, where $a$ and $b$ are real numbers and $i=\sqrt{-1}$.
\begin{multicols}{3}
\begin{enumerate}
	\item $\sqrt{- 16}$
	\item $\sqrt[]{- 24}$
	\item $\sqrt[]{- 6} ~\sqrt[]{3}$
	\item $\dfrac{- 15 - \sqrt[]{- 200}}{20}$
	\item $i^{35}$
	\item $i^{124}$
	\item $(4 - 8 i) - (3 - 5 i)$
	\item $5 i - (3 + 8 i) + (- 4 + 7 i)$
	\item $(3 i) (7 i)$
	\item $5 i (3 i - 7)$
	\item $(3 i) (6 i) (2 - 3 i)$
	\item $(4 - 5 i)^2$
	\item $\dfrac{7 + 3 i}{- 5 i}$
	\end{enumerate}
\end{multicols}
\ \par
{\bf III - Practice Problems:}\\
\ \par
Rewrite each of the following complex numbers in the form $a+bi$, where $a$ and $b$ are real numbers and $i=\sqrt{-1}$.
\begin{enumerate}
\begin{multicols}{4}
  \item $\sqrt{- 81}$
  \item $\sqrt{- 45}$
  \item $\sqrt{- 10} \sqrt{- 2}$
  \item $\sqrt{- 12} \sqrt{- 2}$
\end{multicols}
\begin{multicols}{4}
  \item $\dfrac{3 + \sqrt{- 27}}{6}$
  \item $\dfrac{- 4 - \sqrt{- 8}}{- 4}$
  \item $\dfrac{8 - \sqrt{- 16}}{4}$
  \item $\dfrac{6 + \sqrt{- 32}}{4}$
\end{multicols}
\begin{multicols}{4}
  \item $i^{73}$
  \item $i^{251}$
  \item $i^{48}$
  \item $i^{68}$
  \item $i^{62}$
  \item $i^{181}$
  \item $i^{154}$
  \item $i^{51}$
\end{multicols}
\begin{multicols}{3}
  \item $3 - (- 8 + 4 i)$
  \item $3 i - (7 i)$
  \item $7 i - (3 - 2 i)$
  \item $5 + (- 6 - 6 i)$
  \item $- 6 i - (3 + 7 i)$
  \item $- 8 i - 7 i - (5 - 3 i)$
  \item $(3 - 3 i) + (- 7 - 8 i)$
  \item $(- 4 - i) + (1 - 5 i)$
  \item $-6+i - (2 + 3 i)$
  \item $(5 - 4 i) + (8 - 4 i)$
  \item $(6 i) (- 8 i)$
  \item $(3 i) (- 8 i)$
  \item $(- 5 i) (8 i)$
  \item $(8 i) (- 4 i)$
  \item $(- 7 i)^2$
  \item $(- i) (7 i) (4 - 3 i)$
  \item $(6 + 5 i)^2$
  \item $(8 i) (- 2 i) (- 2 - 8 i)$
  \item $(- 7 - 4 i) (- 8 + 6 i)$
  \item $(3 i) (- 3 i) (4 - 4 i)$
  \item $(- 4 + 5 i) (2 - 7 i)$
  \item $- 8 (4 - 8 i) - 2 (- 2 - 6 i)$
  \item $(- 8 - 6 i) (- 4 + 2 i)$
  \item $(- 6 i) (3 - 2 i) - (7 i) (4 i)$
  \item $(1 + 5 i) (2 + i)$
  \item $(- 2 + i) (3 - 5 i)$
\end{multicols}
\begin{multicols}{4}
  \item $\dfrac{- 9 + 5 i}{i}$
  \item $\dfrac{- 3 + 2 i}{- 3 i}$
  \item $\dfrac{- 10 - 9 i}{6 i}$
  \item $\dfrac{- 4 + 2 i}{3 i}$
  \item $\dfrac{- 3 - 6 i}{4 i}$
  \item $\dfrac{- 5 + 9 i}{9 i}$
  \item $\dfrac{10 - i}{- i}$
  \item $\dfrac{10}{5 i}$
  \item $\dfrac{4 i}{- 10 + i}$
  \item $\dfrac{9 i}{1 - 5 i}$
  \item $\dfrac{8}{7 - 6 i}$
  \item $\dfrac{4}{4 + 6 i}$
  \item $\dfrac{7}{10 - 7 i}$
  \item $\dfrac{9}{- 8 - 6 i}$
  \item $\dfrac{5 i}{- 6 - i}$
  \item $\dfrac{8 i}{6 - 7 i}$
\end{multicols}
\end{enumerate}
\newpage
\ \newpage
\end{document}